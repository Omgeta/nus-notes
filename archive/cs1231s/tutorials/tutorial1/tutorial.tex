\documentclass[12pt, a4paper]{article}

\usepackage[a4paper, margin=1in]{geometry}

\usepackage[utf8]{inputenc}
\usepackage[mathscr]{euscript}
\let\euscr\mathscr \let\mathscr\relax
\usepackage[scr]{rsfso}
\usepackage{amssymb,amsmath,amsthm,amsfonts}
\usepackage[shortlabels]{enumitem}
\usepackage{multicol,multirow}
\usepackage{lipsum}
\usepackage{balance}
\usepackage{calc}
\usepackage[colorlinks=true,citecolor=blue,linkcolor=blue]{hyperref}
\usepackage{import}
\usepackage{xifthen}
\usepackage{pdfpages}
\usepackage{transparent}
\usepackage{listings}

\newcommand{\incfig}[2][1.0]{
    \def\svgwidth{#1\columnwidth}
    \import{./figures/}{#2.pdf_tex}
}

\newlist{enumproof}{enumerate}{4}
\setlist[enumproof,1]{label=\arabic*., parsep=1em}
\setlist[enumproof,2]{label=\arabic{enumproofi}.\arabic*., parsep=1em}
\setlist[enumproof,3]{label=\arabic{enumproofi}.\arabic{enumproofii}.\arabic*., parsep=1em}
\setlist[enumproof,4]{label=\arabic{enumproofi}.\arabic{enumproofii}.\arabic{enumproofiii}.\arabic*., parsep=1em}

\renewcommand{\qedsymbol}{\ensuremath{\blacksquare}}

\lstdefinestyle{mystyle}{
  language=C, % Set the language to C
  commentstyle=\color{codegray}, % Color for comments
  keywordstyle=\color{orange}, % Color for basic keywords
  stringstyle=\color{mauve}, % Color for strings
  basicstyle={\ttfamily\footnotesize}, % Basic font style
  breakatwhitespace=false,         
  breaklines=true,                 
  captionpos=b,                    
  keepspaces=true,                 
  numbers=none,                    
  tabsize=2,
  morekeywords=[2]{\#include, \#define, \#ifdef, \#ifndef, \#endif, \#pragma, \#else, \#elif}, % Preprocessor directives
  keywordstyle=[2]\color{codegreen}, % Style for preprocessor directives
  morekeywords=[3]{int, char, float, double, void, struct, union, enum, const, volatile, static, extern, register, inline, restrict, _Bool, _Complex, _Imaginary, size_t, ssize_t, FILE}, % C standard types and common identifiers
  keywordstyle=[3]\color{identblue}, % Style for types and common identifiers
  morekeywords=[4]{printf, scanf, fopen, fclose, malloc, free, calloc, realloc, perror, strtok, strncpy, strcpy, strcmp, strlen}, % Standard library functions
  keywordstyle=[4]\color{cyan}, % Style for library functions
}

% Things Lie
\newcommand{\kb}{\mathfrak b}
\newcommand{\kg}{\mathfrak g}
\newcommand{\kh}{\mathfrak h}
\newcommand{\kn}{\mathfrak n}
\newcommand{\ku}{\mathfrak u}
\newcommand{\kz}{\mathfrak z}
\DeclareMathOperator{\Ext}{Ext} % Ext functor
\DeclareMathOperator{\Tor}{Tor} % Tor functor
\newcommand{\gl}{\opname{\mathfrak{gl}}} % frak gl group
\renewcommand{\sl}{\opname{\mathfrak{sl}}} % frak sl group chktex 6

% More script letters etc.
\newcommand{\SA}{\mathcal A}
\newcommand{\SB}{\mathcal B}
\newcommand{\SC}{\mathcal C}
\newcommand{\SF}{\mathcal F}
\newcommand{\SG}{\mathcal G}
\newcommand{\SH}{\mathcal H}
\newcommand{\OO}{\mathcal O}

\newcommand{\SCA}{\mathscr A}
\newcommand{\SCB}{\mathscr B}
\newcommand{\SCC}{\mathscr C}
\newcommand{\SCD}{\mathscr D}
\newcommand{\SCE}{\mathscr E}
\newcommand{\SCF}{\mathscr F}
\newcommand{\SCG}{\mathscr G}
\newcommand{\SCH}{\mathscr H}

% Mathfrak primes
\newcommand{\km}{\mathfrak m}
\newcommand{\kp}{\mathfrak p}
\newcommand{\kq}{\mathfrak q}

% number sets
\newcommand{\RR}[1][]{\ensuremath{\ifstrempty{#1}{\mathbb{R}}{\mathbb{R}^{#1}}}}
\newcommand{\NN}[1][]{\ensuremath{\ifstrempty{#1}{\mathbb{N}}{\mathbb{N}^{#1}}}}
\newcommand{\ZZ}[1][]{\ensuremath{\ifstrempty{#1}{\mathbb{Z}}{\mathbb{Z}^{#1}}}}
\newcommand{\QQ}[1][]{\ensuremath{\ifstrempty{#1}{\mathbb{Q}}{\mathbb{Q}^{#1}}}}
\newcommand{\CC}[1][]{\ensuremath{\ifstrempty{#1}{\mathbb{C}}{\mathbb{C}^{#1}}}}
\newcommand{\PP}[1][]{\ensuremath{\ifstrempty{#1}{\mathbb{P}}{\mathbb{P}^{#1}}}}
\newcommand{\HH}[1][]{\ensuremath{\ifstrempty{#1}{\mathbb{H}}{\mathbb{H}^{#1}}}}
\newcommand{\FF}[1][]{\ensuremath{\ifstrempty{#1}{\mathbb{F}}{\mathbb{F}^{#1}}}}
% expected value
\newcommand{\EE}{\ensuremath{\mathbb{E}}}
\newcommand{\charin}{\text{ char }}
\DeclareMathOperator{\sign}{sign}
\DeclareMathOperator{\Aut}{Aut}
\DeclareMathOperator{\Inn}{Inn}
\DeclareMathOperator{\Syl}{Syl}
\DeclareMathOperator{\Gal}{Gal}
\DeclareMathOperator{\GL}{GL} % General linear group
\DeclareMathOperator{\SL}{SL} % Special linear group

%---------------------------------------
% BlackBoard Math Fonts :-
%---------------------------------------

%Captital Letters
\newcommand{\bbA}{\mathbb{A}}	\newcommand{\bbB}{\mathbb{B}}
\newcommand{\bbC}{\mathbb{C}}	\newcommand{\bbD}{\mathbb{D}}
\newcommand{\bbE}{\mathbb{E}}	\newcommand{\bbF}{\mathbb{F}}
\newcommand{\bbG}{\mathbb{G}}	\newcommand{\bbH}{\mathbb{H}}
\newcommand{\bbI}{\mathbb{I}}	\newcommand{\bbJ}{\mathbb{J}}
\newcommand{\bbK}{\mathbb{K}}	\newcommand{\bbL}{\mathbb{L}}
\newcommand{\bbM}{\mathbb{M}}	\newcommand{\bbN}{\mathbb{N}}
\newcommand{\bbO}{\mathbb{O}}	\newcommand{\bbP}{\mathbb{P}}
\newcommand{\bbQ}{\mathbb{Q}}	\newcommand{\bbR}{\mathbb{R}}
\newcommand{\bbS}{\mathbb{S}}	\newcommand{\bbT}{\mathbb{T}}
\newcommand{\bbU}{\mathbb{U}}	\newcommand{\bbV}{\mathbb{V}}
\newcommand{\bbW}{\mathbb{W}}	\newcommand{\bbX}{\mathbb{X}}
\newcommand{\bbY}{\mathbb{Y}}	\newcommand{\bbZ}{\mathbb{Z}}

%---------------------------------------
% MathCal Fonts :-
%---------------------------------------

%Captital Letters
\newcommand{\mcA}{\mathcal{A}}	\newcommand{\mcB}{\mathcal{B}}
\newcommand{\mcC}{\mathcal{C}}	\newcommand{\mcD}{\mathcal{D}}
\newcommand{\mcE}{\mathcal{E}}	\newcommand{\mcF}{\mathcal{F}}
\newcommand{\mcG}{\mathcal{G}}	\newcommand{\mcH}{\mathcal{H}}
\newcommand{\mcI}{\mathcal{I}}	\newcommand{\mcJ}{\mathcal{J}}
\newcommand{\mcK}{\mathcal{K}}	\newcommand{\mcL}{\mathcal{L}}
\newcommand{\mcM}{\mathcal{M}}	\newcommand{\mcN}{\mathcal{N}}
\newcommand{\mcO}{\mathcal{O}}	\newcommand{\mcP}{\mathcal{P}}
\newcommand{\mcQ}{\mathcal{Q}}	\newcommand{\mcR}{\mathcal{R}}
\newcommand{\mcS}{\mathcal{S}}	\newcommand{\mcT}{\mathcal{T}}
\newcommand{\mcU}{\mathcal{U}}	\newcommand{\mcV}{\mathcal{V}}
\newcommand{\mcW}{\mathcal{W}}	\newcommand{\mcX}{\mathcal{X}}
\newcommand{\mcY}{\mathcal{Y}}	\newcommand{\mcZ}{\mathcal{Z}}

%---------------------------------------
% Bold Math Fonts :-
%---------------------------------------

%Captital Letters
\newcommand{\bmA}{\boldsymbol{A}}	\newcommand{\bmB}{\boldsymbol{B}}
\newcommand{\bmC}{\boldsymbol{C}}	\newcommand{\bmD}{\boldsymbol{D}}
\newcommand{\bmE}{\boldsymbol{E}}	\newcommand{\bmF}{\boldsymbol{F}}
\newcommand{\bmG}{\boldsymbol{G}}	\newcommand{\bmH}{\boldsymbol{H}}
\newcommand{\bmI}{\boldsymbol{I}}	\newcommand{\bmJ}{\boldsymbol{J}}
\newcommand{\bmK}{\boldsymbol{K}}	\newcommand{\bmL}{\boldsymbol{L}}
\newcommand{\bmM}{\boldsymbol{M}}	\newcommand{\bmN}{\boldsymbol{N}}
\newcommand{\bmO}{\boldsymbol{O}}	\newcommand{\bmP}{\boldsymbol{P}}
\newcommand{\bmQ}{\boldsymbol{Q}}	\newcommand{\bmR}{\boldsymbol{R}}
\newcommand{\bmS}{\boldsymbol{S}}	\newcommand{\bmT}{\boldsymbol{T}}
\newcommand{\bmU}{\boldsymbol{U}}	\newcommand{\bmV}{\boldsymbol{V}}
\newcommand{\bmW}{\boldsymbol{W}}	\newcommand{\bmX}{\boldsymbol{X}}
\newcommand{\bmY}{\boldsymbol{Y}}	\newcommand{\bmZ}{\boldsymbol{Z}}
%Small Letters
\newcommand{\bma}{\boldsymbol{a}}	\newcommand{\bmb}{\boldsymbol{b}}
\newcommand{\bmc}{\boldsymbol{c}}	\newcommand{\bmd}{\boldsymbol{d}}
\newcommand{\bme}{\boldsymbol{e}}	\newcommand{\bmf}{\boldsymbol{f}}
\newcommand{\bmg}{\boldsymbol{g}}	\newcommand{\bmh}{\boldsymbol{h}}
\newcommand{\bmi}{\boldsymbol{i}}	\newcommand{\bmj}{\boldsymbol{j}}
\newcommand{\bmk}{\boldsymbol{k}}	\newcommand{\bml}{\boldsymbol{l}}
\newcommand{\bmm}{\boldsymbol{m}}	\newcommand{\bmn}{\boldsymbol{n}}
\newcommand{\bmo}{\boldsymbol{o}}	\newcommand{\bmp}{\boldsymbol{p}}
\newcommand{\bmq}{\boldsymbol{q}}	\newcommand{\bmr}{\boldsymbol{r}}
\newcommand{\bms}{\boldsymbol{s}}	\newcommand{\bmt}{\boldsymbol{t}}
\newcommand{\bmu}{\boldsymbol{u}}	\newcommand{\bmv}{\boldsymbol{v}}
\newcommand{\bmw}{\boldsymbol{w}}	\newcommand{\bmx}{\boldsymbol{x}}
\newcommand{\bmy}{\boldsymbol{y}}	\newcommand{\bmz}{\boldsymbol{z}}

%---------------------------------------
% Scr Math Fonts :-
%---------------------------------------

\newcommand{\sA}{{\mathscr{A}}}   \newcommand{\sB}{{\mathscr{B}}}
\newcommand{\sC}{{\mathscr{C}}}   \newcommand{\sD}{{\mathscr{D}}}
\newcommand{\sE}{{\mathscr{E}}}   \newcommand{\sF}{{\mathscr{F}}}
\newcommand{\sG}{{\mathscr{G}}}   \newcommand{\sH}{{\mathscr{H}}}
\newcommand{\sI}{{\mathscr{I}}}   \newcommand{\sJ}{{\mathscr{J}}}
\newcommand{\sK}{{\mathscr{K}}}   \newcommand{\sL}{{\mathscr{L}}}
\newcommand{\sM}{{\mathscr{M}}}   \newcommand{\sN}{{\mathscr{N}}}
\newcommand{\sO}{{\mathscr{O}}}   \newcommand{\sP}{{\mathscr{P}}}
\newcommand{\sQ}{{\mathscr{Q}}}   \newcommand{\sR}{{\mathscr{R}}}
\newcommand{\sS}{{\mathscr{S}}}   \newcommand{\sT}{{\mathscr{T}}}
\newcommand{\sU}{{\mathscr{U}}}   \newcommand{\sV}{{\mathscr{V}}}
\newcommand{\sW}{{\mathscr{W}}}   \newcommand{\sX}{{\mathscr{X}}}
\newcommand{\sY}{{\mathscr{Y}}}   \newcommand{\sZ}{{\mathscr{Z}}}


%---------------------------------------
% Math Fraktur Font
%---------------------------------------

%Captital Letters
\newcommand{\mfA}{\mathfrak{A}}	\newcommand{\mfB}{\mathfrak{B}}
\newcommand{\mfC}{\mathfrak{C}}	\newcommand{\mfD}{\mathfrak{D}}
\newcommand{\mfE}{\mathfrak{E}}	\newcommand{\mfF}{\mathfrak{F}}
\newcommand{\mfG}{\mathfrak{G}}	\newcommand{\mfH}{\mathfrak{H}}
\newcommand{\mfI}{\mathfrak{I}}	\newcommand{\mfJ}{\mathfrak{J}}
\newcommand{\mfK}{\mathfrak{K}}	\newcommand{\mfL}{\mathfrak{L}}
\newcommand{\mfM}{\mathfrak{M}}	\newcommand{\mfN}{\mathfrak{N}}
\newcommand{\mfO}{\mathfrak{O}}	\newcommand{\mfP}{\mathfrak{P}}
\newcommand{\mfQ}{\mathfrak{Q}}	\newcommand{\mfR}{\mathfrak{R}}
\newcommand{\mfS}{\mathfrak{S}}	\newcommand{\mfT}{\mathfrak{T}}
\newcommand{\mfU}{\mathfrak{U}}	\newcommand{\mfV}{\mathfrak{V}}
\newcommand{\mfW}{\mathfrak{W}}	\newcommand{\mfX}{\mathfrak{X}}
\newcommand{\mfY}{\mathfrak{Y}}	\newcommand{\mfZ}{\mathfrak{Z}}
%Small Letters
\newcommand{\mfa}{\mathfrak{a}}	\newcommand{\mfb}{\mathfrak{b}}
\newcommand{\mfc}{\mathfrak{c}}	\newcommand{\mfd}{\mathfrak{d}}
\newcommand{\mfe}{\mathfrak{e}}	\newcommand{\mff}{\mathfrak{f}}
\newcommand{\mfg}{\mathfrak{g}}	\newcommand{\mfh}{\mathfrak{h}}
\newcommand{\mfi}{\mathfrak{i}}	\newcommand{\mfj}{\mathfrak{j}}
\newcommand{\mfk}{\mathfrak{k}}	\newcommand{\mfl}{\mathfrak{l}}
\newcommand{\mfm}{\mathfrak{m}}	\newcommand{\mfn}{\mathfrak{n}}
\newcommand{\mfo}{\mathfrak{o}}	\newcommand{\mfp}{\mathfrak{p}}
\newcommand{\mfq}{\mathfrak{q}}	\newcommand{\mfr}{\mathfrak{r}}
\newcommand{\mfs}{\mathfrak{s}}	\newcommand{\mft}{\mathfrak{t}}
\newcommand{\mfu}{\mathfrak{u}}	\newcommand{\mfv}{\mathfrak{v}}
\newcommand{\mfw}{\mathfrak{w}}	\newcommand{\mfx}{\mathfrak{x}}
\newcommand{\mfy}{\mathfrak{y}}	\newcommand{\mfz}{\mathfrak{z}}


\newcommand{\mytitle}{CS1231S Tutorial 1}
\newcommand{\myauthor}{github/omgeta}
\newcommand{\mydate}{AY 24/25 Sem 1}

\begin{document}
\raggedright
\footnotesize
\begin{center}
{\normalsize{\textbf{\mytitle}}} \\
{\footnotesize{\mydate\hspace{2pt}\textemdash\hspace{2pt}\myauthor}}
\end{center}
\setlist{topsep=-1em, itemsep=-1em, parsep=2em}
\renewcommand{\neg}{\mathord{\sim}}
%%%%%%%%%%%%%%%%%%%%%%%%%%%%%%%%%%%%%%%%%%%%%%%%%%%%%%
%                      Begin                         %
%%%%%%%%%%%%%%%%%%%%%%%%%%%%%%%%%%%%%%%%%%%%%%%%%%%%%%
\begin{enumerate}[Q\arabic*.]
  \item Let $p$ be "I use the umbrella", $q$ be "it rains"
  \begin{enumerate}[(\alph*)]
    \item "I use the umbrella if it rains" $\equiv q \rightarrow p$ \\
      "I use the umbrella only if it rains" $\equiv p \rightarrow q$
    \item In "I use the umbrella if it rains": \\
      "it rains" is sufficient for "I use the umbrella" \\
      "I use the umbrella" is necessary for "it rains"
    \item "I use the umbrella if and only if it rains"$\equiv (q \rightarrow p) \land (p \rightarrow q) \equiv p \leftrightarrow q$
    \item In "I use the umbrella if and only if it rains": \\
      "I use the umbrella" is a necessary and sufficient condition for "it rains"
  \end{enumerate}

  \item The students do not preserve the brackets after applying De Morgan's law which makes the logical statement ambiguous as $\land$ and $\lor$ have equal precedence
    \begin{enumerate}[(\alph*)]
      \item $a \land \neg(b\land c) \equiv a \land (\neg b \lor \neg c)$ 
      \item $\neg(x \lor y) \lor z \equiv (\neg x \land \neg y) \lor z$ 
    \end{enumerate}

  \item
    \begin{enumerate}[(\alph*)]
      \item $\neg a \land (\neg a \rightarrow (b \land a))$\\
      $\equiv \neg a \land (\neg (\neg a) \lor (b \land a))$\hfill(Implication law)\\
      $\equiv \neg a \land (a \lor (b \land a))$ \hfill (Double negation law)\\
      $\equiv \neg a \land (a \lor (a \land b))$ \hfill (Commutative law)\\
      $\equiv \neg a \land (a)$ \hfill (Absorption law)\\
      $\equiv$ false \hfill (Negation law)\\

    \item $(p \lor \neg q) \rightarrow q$ \\
      $\equiv \neg (p \lor \neg q) \lor q$\hfill(Implication law)\\
      $\equiv (\neg p \land q) \lor q$\hfill(DeMorgan's law)\\
      $\equiv q \lor (q \land \neg p)$\hfill(Commutative law)\\
      $\equiv q$\hfill(Absorption law)

    \item $\neg(p\lor \neg q)\lor(\neg p \land \neg q)$\\
      $\equiv (\neg p \land q)\lor(\neg p \land \neg q)$\hfill(DeMorgan's law)\\
      $\equiv \neg p \land (q \lor \neg q)$\hfill(Distributive law)\\
      $\equiv \neg p \land true$\hfill(Negation law)\\
      $\equiv \neg p$\hfill(Identity law)

    \item $(p \rightarrow q)\rightarrow r$\\
      $\equiv (\neg p \lor q) \rightarrow r$\hfill(Implication law)\\
      $\equiv (\neg(\neg p \lor q)) \lor r$\hfill(Implication law)\\
      $\equiv (p \land \neg q) \lor r$\hfill(DeMorgan's law)\\
    \end{enumerate}

  \item Since the truth tables do not match, $(p \rightarrow q) \rightarrow r \not\equiv p \rightarrow (q \rightarrow r)$
  \begin{displaymath}
    \begin{array}{|c c c|c|c|c|c|}
      p & q & r & p \rightarrow q & q \rightarrow r & (p \rightarrow q) \rightarrow r & p \rightarrow (q \rightarrow r)\\ % Use & to separate the columns
      \hline % Put a horizontal line between the table header and the rest.
      T & T & T & T & T & T & T\\
      T & T & F & T & F & F & F\\
      T & F & T & F & T & T & T\\
      F & T & T & T & T & T & T\\
      T & F & F & F & T & T & T\\
      F & T & F & T & F & \mathbf{F} & \mathbf{T}\\
      F & F & T & T & T & T & T\\
      F & F & F & T & T & \mathbf{F} & \mathbf{T}\\
    \end{array}
  \end{displaymath}

  \vfill
\item Let $p$ be $12x - 7 = 29$, and $q$ be $x = 3$ \\
  Original: $p \rightarrow q$ \\
  Negation: $p \land \neg q$ \\
  Contrapositive: $\neg q \rightarrow \neg p$ \\
  Converse: $q \rightarrow p$ \\
  Inverse: $\neg p \rightarrow \neg q$ \\
  \hfill\\
  Suppose $12x - 7 = 29$, then $12x = 36$ and $x = 3$, which matches the conclusion. Therefore, the conditional statement is true.

  Suppose $x = 3$, then indeed $12x - 7 = 29$, which matches the conclusion. Therefore, the converse statement is also true.

  No, the converse and the inverse are logically equivalent because they are contrapositive of each other.
  \begin{displaymath}
    \begin{array}{|c c|c|c|}
      p & q & (q \rightarrow p) & (\neg p \rightarrow \neg q)\\ % Use & to separate the columns
      \hline % Put a horizontal line between the table header and the rest.
      T & T & T & T\\
      T & F & T & T\\
      F & T & F & F\\
      F & F & T & T\\
    \end{array}
  \end{displaymath}

\item Alternative 1, it is evident that the transitive rule of inference does not hold
  \begin{displaymath}
    \begin{array}{|c c c|c|c|c|}
      p & q & r & p \rightarrow_a q & q \rightarrow_a r & p \rightarrow_a r\\ % Use & to separate the columns
      \hline % Put a horizontal line between the table header and the rest.
      T & T & T & T & T & T\\
      T & T & F & T & F & F\\
      T & F & T & F & F & T\\
      F & T & T & F & T & F\\
      T & F & F & F & F & F\\
      F & F & T & F & F & F\\
      F & T & F & F & F & F\\
      F & F & F & F & F & F\\
    \end{array}
  \end{displaymath}

  Alternative 2, it is evident that the transitive rule of inference does not hold
  \begin{displaymath}
    \begin{array}{|c c c|c|c|c|}
      p & q & r & p \rightarrow_b q & q \rightarrow_b r & p \rightarrow_b r\\ % Use & to separate the columns
      \hline % Put a horizontal line between the table header and the rest.
      T & T & T & T & T & T\\
      T & T & F & T & F & F\\
      T & F & T & F & T & T\\
      F & T & T & T & T & T\\
      T & F & F & F & F & F\\
      F & F & T & F & T & T\\
      F & T & F & T & F & F\\
      F & F & F & F & F & F\\
    \end{array}
  \end{displaymath}

  Alternative 3, it is evident that the transitive rule of inference does not hold
  \begin{displaymath}
    \begin{array}{|c c c|c|c|c|}
      p & q & r & p \rightarrow_c q & q \rightarrow_c r & p \rightarrow_c r\\ % Use & to separate the columns
      \hline % Put a horizontal line between the table header and the rest.
      T & T & T & T & T & T\\
      T & T & F & T & F & F\\
      T & F & T & F & F & T\\
      F & T & T & F & T & F\\
      T & F & F & F & T & F\\
      F & F & T & T & F & F\\
      F & T & F & F & F & T\\
      F & F & F & T & T & T\\
    \end{array}
  \end{displaymath}

  \item
    \begin{enumerate}[(\alph*)]
      \item Let $p$ be "Sandra knows Java", $q$ be "Sandra knows C++"\\
        $p \land q$\\
        $\therefore q$ (By specialization)
      \item Let $p$ be "at least one of these two numbers is divisible by 6", $q$ be "product of these two numbers is divisible by 6" \\
        $p \rightarrow q$\\
        $\neg p$ \\
        $\therefore \neg q$ (Inverse error)
      \item Let $p$ be "there are as many rational numbers as there are irrational numbers", $q$ be "the set of all irrational numbers is infinite" \\
        $p \rightarrow q$\\
        $q$\\
        $\therefore p$ (Converse error)
      \item Let $p$ be "I get a Christmas bonus", $q$ be "I sell my motorcycle", $r$ be "I’ll buy a stereo" \\
        $p \rightarrow r$ \\
        $q \rightarrow r$\\
        $\therefore (p \lor q) \rightarrow r$ (By construction)
    \end{enumerate}

  \item
    \begin{enumerate}[(\alph*)]
      \item $\neg p \implies p =$ false \\
        Since $p \lor (q \land q) =$ true and $p =$ false, $(q \land r)$ must be true $\implies q = r =$ true \\
        Therefore, the conclusion is also true, and the argument is valid.
      \item Let $p =$ true, $q =$ false, $r =$ false \\
        Premise 1: $p \lor (q \land r) $is true\\
        Premise 2: $\neg(p \land q)$ is true\\
        Conclusion: $r$ is false
        Which shows that the argument is not valid
      \item Let $p$ be "I go to the beach", $q$ be "I will take my shades", $r$ be "I will take my sunscreen"\\
        $p \rightarrow (q \lor r)$\\
        $q$\\
        $\neg r$\\
        $\therefore p$ (Converse error, therefore the argument is invalid)
      \item Let $p$ be "I will buy a new goat", $q$ be "I will buy a used Yugo", $r$ be "I will need a loan"\\
        $p \lor q$\\
        $(p \land q) \rightarrow r$\\
        $q \land \neg r$\\
        $\therefore \neg p$ \\
        The argument is valid
    \end{enumerate}
  \item \begin{enumerate}[(\alph*)]
      \item \textbf{Proof (by contradiction).}
      \begin{enumerate}[label=\arabic*., itemsep=-2em]
          \item If $A$ is a knight, then:
            \begin{enumerate}[label=1.\arabic*, itemsep=-2em]
              \item What $A$ says is true. \hfill (by definition of knight)
              \item $\therefore B$ is a knight too. \hfill (that's what $A$ says)
              \item $\therefore$ What $B$ says is true. \hfill (by definition of knight)
              \item $\therefore A$ is a knave. \hfill (that's what $B$ says)
              \item $\therefore A$ is not a knight. 
              \item $\therefore$ Contradiction to 1.
          \end{enumerate}
          \item $\therefore A$ is not a knight.
          \item $\therefore A$ is a knave. \hfill (since $A$ is either a knight or a knave, but not both)
          \item $\therefore$ What $B$ says is true.
          \item $\therefore B$ cannot be a knave. \hfill (as $B$ has said something true)
          \item $\therefore B$ is a knight. \hfill (one is a knight or a knave)
      \end{enumerate}
      \item \textbf{Proof (by exhaustion).}
      \begin{enumerate}[label=\arabic*., itemsep=-2em]
          \item If $C$ is a knight, then:
          \begin{enumerate}[label=1.\arabic*, itemsep=-2em]
            \item What $C$ says is true. \hfill (by definition of knight)
            \item $\therefore D$ is a knave. \hfill (that's what $C$ says)
            \item $\therefore$ What $D$ says is false. \hfill (by definition of knave)
            \item $\therefore C$ is not a knave. \hfill (that's what $D$ says)
            \item $\therefore C$ is a knight.
            \item $\therefore$ there is no contradiction.
            \item $\therefore$ there is 1 knight and 1 knave.
          \end{enumerate}
          \item If $C$ is a knave, then:
          \begin{enumerate}[label=1.\arabic*, itemsep=-2em]
            \item What $C$ says is false. \hfill (by definition of knave)
            \item $\therefore D$ is not a knave. \hfill (that's what $C$ says)
            \item $\therefore D$ is a knight.
            \item $\therefore$ What $D$ says is true. \hfill (by definition of knight)
            \item $\therefore C$ is a knave. \hfill (that's what $D$ says)
            \item $\therefore$ there is no contradiction.
            \item $\therefore$ there is 1 knight and 1 knave.
          \end{enumerate}
          \item $\therefore$ there is always 1 knight and 1 knave. \hfill (in both cases)
      \end{enumerate}
    \end{enumerate}
  \item Let $x = 2n+1, y=2m+1$ be two odd numbers,
    \begin{align*}
      x\cdot y &= (2n+1)\cdot(2m+1) \\
               &= 4mn + 2n + 2m + 1 \\
               &= 2(2mn + m + n) + 1 \\
               &= 2k + 1 \text{, where $k=2mn+m+n \in \ZZ$} \\
               &\text{ is odd, by definition of odd numbers}
    \end{align*}
\end{enumerate}
%%%%%%%%%%%%%%%%%%%%%%%%%%%%%%%%%%%%%%%%%%%%%%%%%%%%%%
%                       End                          %
%%%%%%%%%%%%%%%%%%%%%%%%%%%%%%%%%%%%%%%%%%%%%%%%%%%%%%

\end{document}

\documentclass[12pt, a4paper]{article}

\usepackage[a4paper, margin=1in]{geometry}

\usepackage[utf8]{inputenc}
\usepackage[mathscr]{euscript}
\let\euscr\mathscr \let\mathscr\relax
\usepackage[scr]{rsfso}
\usepackage{amssymb,amsmath,amsthm,amsfonts}
\usepackage[shortlabels]{enumitem}
\usepackage{multicol,multirow}
\usepackage{lipsum}
\usepackage{balance}
\usepackage{calc}
\usepackage[colorlinks=true,citecolor=blue,linkcolor=blue]{hyperref}
\usepackage{import}
\usepackage{xifthen}
\usepackage{pdfpages}
\usepackage{transparent}
\usepackage{tabularx}

\newcommand{\incfig}[2][1.0]{
    \def\svgwidth{#1\columnwidth}
    \import{./figures/}{#2.pdf_tex}
}
\newcommand{\incimg}[2][1.0]{
  \includegraphics[width=#1\columnwidth]{./figures/#2}
}


\input{letterfonts}

\newcommand{\mytitle}{CS1231S Tutorial 7}
\newcommand{\myauthor}{github/omgeta}
\newcommand{\mydate}{AY 24/25 Sem 1}

\begin{document}
\raggedright
\footnotesize
\begin{center}
{\normalsize{\textbf{\mytitle}}} \\
{\footnotesize{\mydate\hspace{2pt}\textemdash\hspace{2pt}\myauthor}}
\end{center}
\setlist{topsep=-1em, itemsep=-1em, parsep=2em}
%%%%%%%%%%%%%%%%%%%%%%%%%%%%%%%%%%%%%%%%%%%%%%%%%%%%%%
%                      Begin                         %
%%%%%%%%%%%%%%%%%%%%%%%%%%%%%%%%%%%%%%%%%%%%%%%%%%%%%%
\begin{enumerate}[Q\arabic*.]
  \item 
    \begin{enumerate}[(\alph*)]
      \item Predicate $P(n)$ cannot be used as a binary operand $\qed$
      \item We cannot assume equality to $P(k+1)$, we must show $P(k) \rightarrow P(k+1)$ $\qed$ 
      \item If we assume $P(k)$ is true for all $k\in\ZZ^+$, then there is nothing to prove. $\qed$
    \end{enumerate}
  
  \item \textbf{Proof by 1MI}
    \begin{enumproof}
    \item Let $P(n) \equiv (1^2+2^2+\cdots+n^2=\frac{1}{6}n(n+1)(2x+1)), \forall n\in\ZZ^+$
    \item Basis step: 
      \begin{enumproof}
      \item $1^2 = \frac{1}{6}(2)(3)$, therefore $P(1)$ is true
      \end{enumproof}
    \item Assume $P(k)$ is true for some $k\geq1 \implies 1^2+\cdots+k^2 = \frac{k(k+1)(2k+1)}{6}$
    \item Inductive step: 
      \begin{enumproof}
      \item $1^2+2^2+\cdots+k^2+(k+1)^2 = \frac{k(k+1)(2k+1)}{6}+(k+1)^2$ \\$= \frac{k(k+1)(2k+1)+6(k+1)^2}{6} = \frac{(k+1)(2k^2 + 7k + 6)}{6} = \frac{(k+1)(k+2)(2k+3)}{6}$
      \item Therefore, $P(k+1)$ is true
      \end{enumproof}
    \item Therefore, $P(n)$ is true for all $n \in \ZZ^+ \qed$
    \end{enumproof}

  \item \textbf{Proof by 1MI}
    \begin{enumproof}
    \item Let $P(n) \equiv (1 + nx \leq (1+x)^n), \forall n\in\ZZ^+, x \in \ZZ_{\geq -1}$
    \item Basis step: 
      \begin{enumproof}
      \item $1 + x \leq (1+x)^{1}$, therefore $P(1)$ is true
      \end{enumproof}
    \item Assume $P(k)$ is true for some $k\geq1 \implies 1+kx \leq (1+x)^k$
    \item Inductive step: 
      \begin{enumproof}
      \item $1+(k+1)x = 1+ kx + x \leq (1+x)^k + x \leq (1+x)^k +x(1+x)^k = (1+x)^{k+1}$
      \item Therefore, $P(k+1)$ is true
      \end{enumproof}
    \item Therefore, $P(n)$ is true for all $n \in \ZZ^+ \qed$
    \end{enumproof}

  \item \textbf{Proof by 1MI}
    \begin{enumproof}
    \item Let $P(n) \equiv (2^{n+2} \mid (a^{2^n} - 1)), \forall n\in\ZZ^+, a$ is any odd integer
    \item Basis step: 
      \begin{enumproof}
      \item $a^{2^1} - 1 = (a+1)(a-2)$\hfill(Basic algebra)
      \item $= (2m+2)(2m) = 4(m+1)(m)$\hfill(Definition of odd numbers)
      \item $= 4(2k)$\hfill(Prod. of consecutive integers is even)
      \item $= 8k = k\cdot 2^3$
      \item $\therefore 2^3\mid (a^{2^1} - 1)$\hfill(Definition of divides)
      \item Therefore, $P(1)$ is true
      \end{enumproof}
    \item Assume $P(k)$ is true for some $k\in\ZZ^+$:
      \begin{enumproof}
        \item $2^{k+2} \mid a^{2^n}-1$\hfill(Definition of $P(n)$)
        \item $\exists m\in\ZZ,$ $m\cdot2^{k+2} = a^{2^k}-1$\hfill(Definition of divides)
      \end{enumproof}
    \item Inductive step: 
      \begin{enumproof}
      \item $a^{2^{k+1}} - 1 = (a^{2^k})^2 - 1= (a^{2^k}-1)(a^{2^k}+1)$\hfill(Basic algebra)
      \item $= m \cdot 2^{k+2}\cdot (a^{2^k}+1)$\hfill(By inductive hypothesis)
      \item $= m \cdot 2^{k+2} \cdot (m\cdot 2^{k+2}+2)$\hfill(By indutive hypothesis)
      \item $m \cdot 2^{k+3}(m \cdot 2^{k+1} + 1)$\hfill(Basic algebra)
      \item Therefore, $P(k+1)$ is true
      \end{enumproof}
    \item Therefore, $P(n)$ is true for all $n \in \ZZ^+ \qed$
    \end{enumproof}
  \pagebreak
  \item \textbf{Proof by 2MI}
    \begin{enumproof}
    \item Let $P(n) \equiv (n = 3x+5y), \forall n \geq \ZZ_{\geq8},\exists x,y \in \NN$
    \item Basis step: 
      \begin{enumproof}
      \item $8 = 3(1) + 5(1)$, therefore $P(8)$ is true
      \item $9 = 3(3) + 5(0)$, therefore $P(9)$ is true
      \item $10 = 3(0) + 5(2)$, therefore $P(10)$ is true
      \end{enumproof}
    \item Assume $P(i)$ is true for $8 \leq i \leq k$ for some $k$
    \item Inductive step: 
      \begin{enumproof}
      \item $P(k-2)$ is true $\implies k-2 = 3a+5b$, for some $a, b \in \ZZ$
      \item $k+1 = (k-2) + 3 = 3a + 5b + 3 = 3(a+1) + b$
      \item Therefore, $P(k+1)$ is true
      \end{enumproof}
    \item Therefore, $P(n)$ is true for all $n \in \ZZ_{\geq 8} \qed$
    \end{enumproof}

  \item \textbf{Proof by 2MI}
    \begin{enumproof}
    \item Let $P(n) \equiv (i_1< i_2<\cdots<i_l \land n = 2^{i_1} + 2^{i_2} + \cdots + 2^{i_l}), \forall n \in \ZZ^+ \exists l \in \ZZ^+ \exists i_1,i_2,\cdots,i_l \in \NN$
    \item Basis step: $1 = 2^0 \implies P(1)$ is true
    \item Assume $P(i)$ is true for $1 \leq i \leq k$ for some $k$
    \item Inductive step: 
      \begin{enumproof}
      \item Case 1 ($k+1$ is odd):
        \begin{enumproof}
        \item $k+1 = 2m+1$, $m = \frac{k+1}{2} \in \ZZ$\hfill(Definition of odd numbers) 
        \item $m = 2^{i_1} + \cdots + 2^{i_l}$\hfill(By inductive hypothesis)
        \item $k = 2(2^{i_1} + \cdots + 2^{i_l}) = 2^{i_1+1} + \cdots + 2^{i_l + 1}$ where $i_1+1, i_2+1, \cdots i_l+1 \geq 1$
        \item $k+1 = 2^{i_1+1} + \cdots + 2^{i_l + 1} + 2^0$
        \item Therefore, $P(k+1)$ is true
        \end{enumproof}
      \item Case 2 ($k+1$ is even):
        \begin{enumproof}
        \item $k+1 = 2m$, $m = \frac{k+1}{2} \in \ZZ$\hfill(Definition of even numbers) 
        \item $m = 2^{i_1} + \cdots + 2^{i_l}$\hfill(By inductive hypothesis)
        \item $k+1 = 2(2^{i_1} + \cdots + 2^{i_l}) = 2^{i_1+1} + \cdots + 2^{i_l + 1}$
        \item Therefore, $P(k+1)$ is true
        \end{enumproof}
      \item In all cases, $P(k+1)$ is true
      \end{enumproof}
    \item Therefore, $P(n)$ is true for all $n \in \ZZ^+ \qed$
    \end{enumproof}
    
  \item \textbf{Proof by 2MI}
    \begin{enumproof}
    \item Let $P(n) \equiv (a_n < 3^n), \forall n \in \NN$
    \item Basis step: $a_0 = 0 < 1 = 3^0$, therefore $P(0)$ is true
    \item Assume $P(i)$ is true for $0 \leq i \leq k$ for some $k$
    \item Inductive step: 
      \begin{enumproof}
      \item $a_{k+1} = a_{k} + a_{k-1} + a_{k-2} < 3^k + 3^{k-1} + 3^{k-2} < 3^k + 3^k + 3^k = 3^{k+1}$
      \item Therefore, $P(k+1)$ is true
      \end{enumproof}
    \item Therefore, $P(n)$ is true for all $n \in \NN \qed$
    \end{enumproof}
    \pagebreak

  \item 
    \begin{enumerate}[(\alph*)]
      \item $F(0+b) = F(b) = (F(1) \times F(b) + F(0) \times F(b-1))$, therefore $P(0, b)$ is true $\qed$\\
        $F(1+b) = F(b) + F(b-1) = (F(2) \times F(b) + F(1) \times F(b-1))$, therefore $P(1, b)$ is true $\qed$
      \item 
        \begin{enumproof}
        \item Assume $P(k-1, b) \land P(k, b)$ for some $k \in \ZZ^+$:
          \begin{align*}
            F(k-1+b) = F(k)\times F(b) + F(k-1) \times F(b-1)\\
            F(k+b) = F(k+1)\times F(b) + F(k) \times F(b-1)
          \end{align*}
          \vspace{-2em}
        \item Inductive step:
          \begin{enumproof}
          \item $F(k+1+b) = F(k+b) + F(k+b-1)$\hfill(Definition of Fibonacci sequence)
          \item $= (F(k+1) \times F(b) + F(k) \times F(b-1)) + (F(k) \times F(b) + F(k-1) \times F(b-1))$\\
          \item $= F(b) \times (F(k+1) + F(k)) + F(b-1) \times (F(k) + F(k-1))$\hfill(Distributive law)
          \item $= F(b) \times F(k+2) + F(b-1) \times F(k+1)$\hfill(Definition of Fibonacci sequence)
          \item Therefore, $P(k+1, b)$ is true
          \end{enumproof}
        \item Therefore, $P(n+1,b)$ is true for all $n \in \ZZ^+ \qed$
        \end{enumproof}
    \end{enumerate}

  \item \textbf{Proof by 1MI}
    \begin{enumproof}
    \item Basis step: $1 = 2^05^05^0$, therefore $P(1)$ is true
    \item Assume $P(m)$ is true for some $m$, i.e. $\exists!i\exists!j\exists!k((i,j,k\geq0) \land m=2^i 3^j 5^k)$
    \item Inductive step:
      \begin{enumproof}
      \item $2m = 2 \cdot 2^i 3^j 5^k = 2^{i+1} 3^k 5^k \implies P(2m)$\hfill(By inductive hypothesis)
      \item $3m = 3 \cdot 2^i 3^j 5^k = 2^i 3^{k+1} 5^k \implies P(3m)$\hfill(By inductive hypothesis)
      \item $5m = 5 \cdot 2^i 3^j 5^k = 2^i 3^k 5^{k+1} \implies P(5m)$\hfill(By inductive hypothesis)
      \item Therefore $P(m) \rightarrow P(2m) \land P(3m) \land P(5m)$\hfill(Conjunction)
      \end{enumproof}
    \item Therefore, $\forall n \in H$, $P(n) \qed$\hfill(Given 1MI rule)
    \end{enumproof}

  \item $0, 15 \not\in S$ and $6,16,36 \in S \qed$
  
  \item 
    \begin{enumerate}[(\alph*)]
      \item Yes; $C = (A \setminus B) \cup (B \setminus A) \in S \qed$
      \item No $\qed$
    \end{enumerate}
\end{enumerate}
%%%%%%%%%%%%%%%%%%%%%%%%%%%%%%%%%%%%%%%%%%%%%%%%%%%%%%
%                       End                          %
%%%%%%%%%%%%%%%%%%%%%%%%%%%%%%%%%%%%%%%%%%%%%%%%%%%%%%

\end{document}

\documentclass[12pt, a4paper]{article}

\usepackage[a4paper, margin=1in]{geometry}

\usepackage[utf8]{inputenc}
\usepackage[mathscr]{euscript}
\let\euscr\mathscr \let\mathscr\relax
\usepackage[scr]{rsfso}
\usepackage{amssymb,amsmath,amsthm,amsfonts}
\usepackage[shortlabels]{enumitem}
\usepackage{multicol,multirow}
\usepackage{lipsum}
\usepackage{balance}
\usepackage{calc}
\usepackage[colorlinks=true,citecolor=blue,linkcolor=blue]{hyperref}
\usepackage{import}
\usepackage{xifthen}
\usepackage{pdfpages}
\usepackage{transparent}
\usepackage{tabularx}

\newcommand{\incfig}[2][1.0]{
    \def\svgwidth{#1\columnwidth}
    \import{./figures/}{#2.pdf_tex}
}
\newcommand{\incimg}[2][1.0]{
  \includegraphics[width=#1\columnwidth]{./figures/#2}
}


\input{letterfonts}

\newcommand{\mytitle}{CS1231S Tutorial 8}
\newcommand{\myauthor}{github/omgeta}
\newcommand{\mydate}{AY 24/25 Sem 1}

\begin{document}
\raggedright
\footnotesize
\begin{center}
{\normalsize{\textbf{\mytitle}}} \\
{\footnotesize{\mydate\hspace{2pt}\textemdash\hspace{2pt}\myauthor}}
\end{center}
\setlist{topsep=-1em, itemsep=-1em, parsep=2em}
%%%%%%%%%%%%%%%%%%%%%%%%%%%%%%%%%%%%%%%%%%%%%%%%%%%%%%
%                      Begin                         %
%%%%%%%%%%%%%%%%%%%%%%%%%%%%%%%%%%%%%%%%%%%%%%%%%%%%%%
\begin{enumerate}[Q\arabic*.]
  \item $\displaystyle g(n) = (-1)^n\ceil{\frac{n}{2}} \qed$ 

  \item 
    \begin{enumerate}[(\alph*)]
      \item \textbf{Direct Proof} 
        \begin{enumproof}
        \item $b_1, b_2, \cdots$ is a sequence in which every element of $B$ appears\hfill(Lemma 9.2)
        \item Let $|C| = n$, then $C = \{c_1, c_2, \cdots, c_n\}$\hfill(Definition of finite sets)
        \item Then $c_1, c_2, \cdots, c_n,b_1, b_2, \cdots$ is a sequence in which every element of $B \cup C$ appears
        \item $\therefore B\cup C$ is countable$\qed$\hfill(Lemma 9.2)
        \end{enumproof}

      \item \textbf{Direct Proof}
        \begin{enumproof}
        \item Since $B$ is countably infinite set, $\exists$ bijection $f: \ZZ^+ \rightarrow B$
        \item Let $C' = C \setminus B = \{c_1, c_2, \cdots, c_k\}$
        \item Define $g: \ZZ^+ \rightarrow B \cup C$:
          \begin{align*}
            g(i) = 
            \begin{cases}
              c_i & i <= k\\
              f(i-k) & \text{otherwise}
            \end{cases}
          \end{align*}
        \item Prove $g$ is injective:
          \begin{enumproof}
          \item Suppose $x_1, x_2 \in \ZZ^+$ s.t. $g(x_1) = g(x_2)$
          \item Case 1 ($x_1, x_2 \leq k$): $c_{x_1} = c_{x_2} \implies x_1 = x_2$\hfill(Distinct values of $C$)
          \item Case 2 ($x_1, x_2 > k$): $f(x_1 - k) = f(x_2 - k) \implies x_1 = x_2$\hfill(Injectivity of $f$)
          \item Case 3 (either $x_1$ or $x_2 \leq k$): WLOG, $x_1\leq k \implies c_{x_1} = f(x_1 - k)$ but this is a contradiction with $B \cap C' = \phi$
          \item In both cases, $x_1 = x_2$
          \end{enumproof}
        \item Prove $g$ is surjective:
          \begin{enumproof}
          \item Suppose $y \in B \cup C$
          \item Case 1 ($y \in B$): $\exists i, (g(i) = y)$\hfill(Surjectivity of $f$)
          \item Case 2 ($y \not\in B$): $\exists c_i = y \implies \exists i, g(i) = y$
          \item In both cases, $\exists i, (g(i) = y)$
          \end{enumproof}
        \item $\therefore g: \ZZ^+ \rightarrow B \cup C$ is bijective\hfill(Definition of bijection)
        \item $\therefore B \cup C$ is countable$\qed$\hfill(Definition of countably infinite) 
        \end{enumproof}
    \end{enumerate}

  \item
    \begin{enumerate}[(\alph*)]
      \item We cannot assume $A_{k+1} = \phi$, and must instead solve for the general case where $A_{k+1}$ is any finite set $\qed$
      \item Suppose $A_k = \{k\}$, then $\displaystyle \bigcup_{k=1}^{\infty}A_k = \ZZ^+$ which is infinite, disproving the statement $\qed$
    \end{enumerate}

  \item 
    \begin{enumerate}[(\alph*)]
      \item \textbf{Proof by 1MI}
      \begin{enumproof}
      \item Let $P(n) \equiv \bigcup_{i=1}^n A_i$ is countable for $n \in \ZZ^+$
      \item Basis step: $\bigcup_{i=1}^1 A_1 = A_1$ which is given to be countable, therefore $P(1)$ is true
      \item Assume $P(k)$ for some $k \in \ZZ^+$
      \item Inductive step:
        \begin{enumproof}
        \item $\bigcup_{i=1}^{k+1} = (\bigcup_{i=1}^kA_i) \cup A_{k+1}$
        \item By induction hypothesis $\bigcup_{i=1}^kA_i$ is countable, and $A_{k+1}$ is given to be countable, therefore their union is countable\hfill(Lemma 9.4)
        \item $P(k+1)$ is true
        \end{enumproof}
      \item Therefore, $\bigcup_{i=1}^nA_i$ is countable for any $n \in \ZZ^+ \qed$
      \end{enumproof}
      \item No, by Qn 3(b) $\qed$
    \end{enumerate}
  \pagebreak

  \item \textbf{Direct Proof}
    \begin{enumproof}
    \item Given $\ZZ^+ \times \ZZ^+$ is countable, we have bijection $f: \ZZ^+ \rightarrow \ZZ^+ \times \ZZ^+$
    \item There exists sequence $s_{i1}, s_{i2}, \cdots \in S_i$ in which every element of $S_i$ appears\hfill(Lemma 9.2) 
    \item Hence, $\forall s_{ij} \in \bigcup_{i\in\ZZ^+}$, we have $(i, j) \in \ZZ^+ \times \ZZ^+$
    \item Define sequence $c_1, c_2, \cdots$, s.t. $c_k = b_{ij}$ whenever $f(k) = (i, j)$
    \item It suffices to show any element of $\bigcup_{i\in\ZZ^+}S_i$ appears in the sequence defined in line 4:
      \begin{enumproof}
      \item Let $x \in \bigcup_{i\in\ZZ^+}S_i$ 
      \item $\exists i\in\ZZ^+(x\in S_i)$\hfill(Definition of union)
      \item $\exists j \in \ZZ^+ (x = b_{ij})$\hfill(Line 3)
      \item $\exists k \in \ZZ^+(x=c_k)$\hfill(Definition of sequence)
      \end{enumproof}
    \item Therefore, $\bigcup_{i\in\ZZ^+}$ is countable$\qed$\hfill(Lemma 9.2)
    \end{enumproof}

  \item \textbf{Direct Proof} 
    \begin{enumproof}
    \item Take $B' \subseteq B$ s.t. $B'$ is countably infinite\hfill(Proposition 9.3)
    \item Since $B$ is countably infinite set, $\exists$ bijection $f: \ZZ^+ \rightarrow B$
    \item Let $C' = C \setminus B = \{c_1, c_2, \cdots, c_k\}$
    \item $B' \cup C'$ is countable\hfill(Qn 2)
    \item $\exists$ bijection $f: B' \cup C' \rightarrow B'$\hfill(Definition of cardinality)
    \item Define $g: B \cup C \rightarrow B$:
      \begin{align*}
        g(x) =
        \begin{cases}
          f(x) & x \in B' \cup C'\\
          x & \text{otherwise}
        \end{cases} \qed
      \end{align*}
    \end{enumproof}
  
  \item \textbf{Proof by Contradiction}
    \begin{enumproof}
    \item Suppose not, that is, $\powerset(A)$ is countable
      \begin{enumproof}
      \item $\forall a \in A, \{a\} \in \powerset(A) \implies \powerset(A)$ is infinite 
      \item $\exists$ sequence $a_1, a_2, \cdots \in A$ in which every element of $A$ appears\hfill(Lemma 9.2)
      \item $\exists$ sequence $S_1, S_2, \cdots \in \powerset(A)$ in which every element of $\powerset(A)$ appears\hfill(Lemma 9.2)
      \item Construct $X = \{a_i : a_i \not\in S_i\}$
      \item Note that $\forall S_i \in \powerset(A), X \neq S_i$
      \item But $X \in \powerset(A)$ which contradicts 1.3.
      \end{enumproof}
    \item Hence, the supposition is false.
    \item $\therefore \powerset(A)$ is uncountable$\qed$\hfill(Contradiction rule)
    \end{enumproof}

  \item 
    \begin{enumerate}[(\alph*)]
      \item \textbf{Direct Proof}
        \begin{enumproof}
        \item Suppose $R$ is a reflexive relation on $A$ for $|A| = n \in \NN$
        \item $A = \{a_1, a_2, \cdots, a_n\}$\hfill(Definition of finite set)
        \item $\forall a \in A, (a, a) \in R$\hfill(Definition of reflexive relation)
        \item Define $f: A\rightarrow R$. where $f(a) = (a, a)$, $\forall a \in A$
        \item $\forall x, y \in A, f(x) = f(y) \implies (x, x) = (y, y) \implies x=y$. Therefore, $f$ is injective.
        \item $|A| \leq |R| \qed$\hfill(By pigeonhole principle)
        \end{enumproof}
      \item \textbf{Counterexample:} $A = \{a\}, R = \phi \implies |A| = 1 > 0 = |R| \qed$
      \item \textbf{Counterexample:} $A = \{a\}, R = \phi \implies |A| = 1 > 0 = |R| \qed$
    \end{enumerate}
  \pagebreak
  \item \textbf{Proof by 2MI}
    \begin{enumproof}
    \item Let $P(n) \equiv Even(F_n) \iff Even(F_{n+3}), \forall n \in \NN$
    \item Basis step: $F_0 = 0$ and $F_3 = 2$ are both even, therefore $P(0)$ is true
    \item Assume $P(i)$ is true for $0 \leq i \leq k$:
      \begin{align*}
        Even(F_{k}) \iff Even(F_{k+3})
      \end{align*}
    \item Inductive step:
      \begin{enumproof}
      \item $Even(F_{k+1}) \iff Even(F_{k} + F_{k-1})$\hfill(Definition of $F$)
      \item $\quad\quad\iff (Even(F_k) \iff Even(F_{k-1}))$\hfill(Fact 1)
      \item $\quad\quad\iff (Even(F_{k+3}) \iff Even(F_{k+2}))$\hfill(By inductive hypothesis)
      \item $\quad\quad\iff Even(F_{k+3} + F_{k+2})$\hfill(Fact 1)
      \item $\quad\quad\iff Even(F_{k+4})$\hfill(Definition of $F$)
      \item $P(k+1)$ is true
      \end{enumproof}
    \item Therefore, $P(n)$ is true for all $n \in \NN \qed$
    \end{enumproof}

  \item 
    \begin{enumerate}[(\alph*)]
      \item \textbf{Direct Proof}
      \begin{enumproof}
      \item Prove F1:
        \begin{enumproof}
        \item Suppose $[x] \in X /\mathord{\sim}$
        \item Since $g$ is a function defined on $\forall x \in X$, $f([x]) = g(x)$
        \item $\forall [x] \in X /\mathord{\sim}$ $\exists g(x) \in Y (g(x) = f([x]))$
        \end{enumproof}
      \item Prove F2:
        \begin{enumproof}
        \item Suppose $[x] = [y]$ s.t. $f([x]) = f([y])$ 
        \item $x \sim y$\hfill(Definition of equivalence class)
        \item $g(x) = g(y)$\hfill(Definition of $g$)
        \item $\forall [x] \in X /\mathord{\sim} z_1, z_2 \in Y(([x], z_1) \in f \land ([x], z_2) \in f \rightarrow z_1 = z_2)$
        \end{enumproof}
      \item Therefore, $f$ is well-defined $\qed$
      \end{enumproof}
      \item \textbf{Direct Proof}
        \begin{enumproof}
        \item Suppose $f([x]) = f([y])$
        \item $g(x) = g(y)$\hfill(Definition of $f$)
        \item $x \sim y$\hfill(Definition of $g$)
        \item $[x] = [y]$\hfill(Definition of equivalent classes)
        \item $f$ is injective$\qed$
        \end{enumproof}
      \item $f: 4 \qed$\\
        $g: 4 \qed$\\
        $f^{-1}\circ g: 3 \qed$
    \end{enumerate}
\end{enumerate}
%%%%%%%%%%%%%%%%%%%%%%%%%%%%%%%%%%%%%%%%%%%%%%%%%%%%%%
%                       End                          %
%%%%%%%%%%%%%%%%%%%%%%%%%%%%%%%%%%%%%%%%%%%%%%%%%%%%%%

\end{document}

\documentclass[12pt, a4paper]{article}

\usepackage[a4paper, margin=1in]{geometry}

\usepackage[utf8]{inputenc}
\usepackage[mathscr]{euscript}
\let\euscr\mathscr \let\mathscr\relax
\usepackage[scr]{rsfso}
\usepackage{amssymb,amsmath,amsthm,amsfonts}
\usepackage[shortlabels]{enumitem}
\usepackage{multicol,multirow}
\usepackage{lipsum}
\usepackage{balance}
\usepackage{calc}
\usepackage[colorlinks=true,citecolor=blue,linkcolor=blue]{hyperref}
\usepackage{import}
\usepackage{xifthen}
\usepackage{pdfpages}
\usepackage{transparent}
\usepackage{listings}

\newcommand{\incfig}[2][1.0]{
    \def\svgwidth{#1\columnwidth}
    \import{./figures/}{#2.pdf_tex}
}

\newlist{enumproof}{enumerate}{4}
\setlist[enumproof,1]{label=\arabic*., parsep=1em}
\setlist[enumproof,2]{label=\arabic{enumproofi}.\arabic*., parsep=1em}
\setlist[enumproof,3]{label=\arabic{enumproofi}.\arabic{enumproofii}.\arabic*., parsep=1em}
\setlist[enumproof,4]{label=\arabic{enumproofi}.\arabic{enumproofii}.\arabic{enumproofiii}.\arabic*., parsep=1em}

\renewcommand{\qedsymbol}{\ensuremath{\blacksquare}}

\lstdefinestyle{mystyle}{
  language=C, % Set the language to C
  commentstyle=\color{codegray}, % Color for comments
  keywordstyle=\color{orange}, % Color for basic keywords
  stringstyle=\color{mauve}, % Color for strings
  basicstyle={\ttfamily\footnotesize}, % Basic font style
  breakatwhitespace=false,         
  breaklines=true,                 
  captionpos=b,                    
  keepspaces=true,                 
  numbers=none,                    
  tabsize=2,
  morekeywords=[2]{\#include, \#define, \#ifdef, \#ifndef, \#endif, \#pragma, \#else, \#elif}, % Preprocessor directives
  keywordstyle=[2]\color{codegreen}, % Style for preprocessor directives
  morekeywords=[3]{int, char, float, double, void, struct, union, enum, const, volatile, static, extern, register, inline, restrict, _Bool, _Complex, _Imaginary, size_t, ssize_t, FILE}, % C standard types and common identifiers
  keywordstyle=[3]\color{identblue}, % Style for types and common identifiers
  morekeywords=[4]{printf, scanf, fopen, fclose, malloc, free, calloc, realloc, perror, strtok, strncpy, strcpy, strcmp, strlen}, % Standard library functions
  keywordstyle=[4]\color{cyan}, % Style for library functions
}

% Things Lie
\newcommand{\kb}{\mathfrak b}
\newcommand{\kg}{\mathfrak g}
\newcommand{\kh}{\mathfrak h}
\newcommand{\kn}{\mathfrak n}
\newcommand{\ku}{\mathfrak u}
\newcommand{\kz}{\mathfrak z}
\DeclareMathOperator{\Ext}{Ext} % Ext functor
\DeclareMathOperator{\Tor}{Tor} % Tor functor
\newcommand{\gl}{\opname{\mathfrak{gl}}} % frak gl group
\renewcommand{\sl}{\opname{\mathfrak{sl}}} % frak sl group chktex 6

% More script letters etc.
\newcommand{\SA}{\mathcal A}
\newcommand{\SB}{\mathcal B}
\newcommand{\SC}{\mathcal C}
\newcommand{\SF}{\mathcal F}
\newcommand{\SG}{\mathcal G}
\newcommand{\SH}{\mathcal H}
\newcommand{\OO}{\mathcal O}

\newcommand{\SCA}{\mathscr A}
\newcommand{\SCB}{\mathscr B}
\newcommand{\SCC}{\mathscr C}
\newcommand{\SCD}{\mathscr D}
\newcommand{\SCE}{\mathscr E}
\newcommand{\SCF}{\mathscr F}
\newcommand{\SCG}{\mathscr G}
\newcommand{\SCH}{\mathscr H}

% Mathfrak primes
\newcommand{\km}{\mathfrak m}
\newcommand{\kp}{\mathfrak p}
\newcommand{\kq}{\mathfrak q}

% number sets
\newcommand{\RR}[1][]{\ensuremath{\ifstrempty{#1}{\mathbb{R}}{\mathbb{R}^{#1}}}}
\newcommand{\NN}[1][]{\ensuremath{\ifstrempty{#1}{\mathbb{N}}{\mathbb{N}^{#1}}}}
\newcommand{\ZZ}[1][]{\ensuremath{\ifstrempty{#1}{\mathbb{Z}}{\mathbb{Z}^{#1}}}}
\newcommand{\QQ}[1][]{\ensuremath{\ifstrempty{#1}{\mathbb{Q}}{\mathbb{Q}^{#1}}}}
\newcommand{\CC}[1][]{\ensuremath{\ifstrempty{#1}{\mathbb{C}}{\mathbb{C}^{#1}}}}
\newcommand{\PP}[1][]{\ensuremath{\ifstrempty{#1}{\mathbb{P}}{\mathbb{P}^{#1}}}}
\newcommand{\HH}[1][]{\ensuremath{\ifstrempty{#1}{\mathbb{H}}{\mathbb{H}^{#1}}}}
\newcommand{\FF}[1][]{\ensuremath{\ifstrempty{#1}{\mathbb{F}}{\mathbb{F}^{#1}}}}
% expected value
\newcommand{\EE}{\ensuremath{\mathbb{E}}}
\newcommand{\charin}{\text{ char }}
\DeclareMathOperator{\sign}{sign}
\DeclareMathOperator{\Aut}{Aut}
\DeclareMathOperator{\Inn}{Inn}
\DeclareMathOperator{\Syl}{Syl}
\DeclareMathOperator{\Gal}{Gal}
\DeclareMathOperator{\GL}{GL} % General linear group
\DeclareMathOperator{\SL}{SL} % Special linear group

%---------------------------------------
% BlackBoard Math Fonts :-
%---------------------------------------

%Captital Letters
\newcommand{\bbA}{\mathbb{A}}	\newcommand{\bbB}{\mathbb{B}}
\newcommand{\bbC}{\mathbb{C}}	\newcommand{\bbD}{\mathbb{D}}
\newcommand{\bbE}{\mathbb{E}}	\newcommand{\bbF}{\mathbb{F}}
\newcommand{\bbG}{\mathbb{G}}	\newcommand{\bbH}{\mathbb{H}}
\newcommand{\bbI}{\mathbb{I}}	\newcommand{\bbJ}{\mathbb{J}}
\newcommand{\bbK}{\mathbb{K}}	\newcommand{\bbL}{\mathbb{L}}
\newcommand{\bbM}{\mathbb{M}}	\newcommand{\bbN}{\mathbb{N}}
\newcommand{\bbO}{\mathbb{O}}	\newcommand{\bbP}{\mathbb{P}}
\newcommand{\bbQ}{\mathbb{Q}}	\newcommand{\bbR}{\mathbb{R}}
\newcommand{\bbS}{\mathbb{S}}	\newcommand{\bbT}{\mathbb{T}}
\newcommand{\bbU}{\mathbb{U}}	\newcommand{\bbV}{\mathbb{V}}
\newcommand{\bbW}{\mathbb{W}}	\newcommand{\bbX}{\mathbb{X}}
\newcommand{\bbY}{\mathbb{Y}}	\newcommand{\bbZ}{\mathbb{Z}}

%---------------------------------------
% MathCal Fonts :-
%---------------------------------------

%Captital Letters
\newcommand{\mcA}{\mathcal{A}}	\newcommand{\mcB}{\mathcal{B}}
\newcommand{\mcC}{\mathcal{C}}	\newcommand{\mcD}{\mathcal{D}}
\newcommand{\mcE}{\mathcal{E}}	\newcommand{\mcF}{\mathcal{F}}
\newcommand{\mcG}{\mathcal{G}}	\newcommand{\mcH}{\mathcal{H}}
\newcommand{\mcI}{\mathcal{I}}	\newcommand{\mcJ}{\mathcal{J}}
\newcommand{\mcK}{\mathcal{K}}	\newcommand{\mcL}{\mathcal{L}}
\newcommand{\mcM}{\mathcal{M}}	\newcommand{\mcN}{\mathcal{N}}
\newcommand{\mcO}{\mathcal{O}}	\newcommand{\mcP}{\mathcal{P}}
\newcommand{\mcQ}{\mathcal{Q}}	\newcommand{\mcR}{\mathcal{R}}
\newcommand{\mcS}{\mathcal{S}}	\newcommand{\mcT}{\mathcal{T}}
\newcommand{\mcU}{\mathcal{U}}	\newcommand{\mcV}{\mathcal{V}}
\newcommand{\mcW}{\mathcal{W}}	\newcommand{\mcX}{\mathcal{X}}
\newcommand{\mcY}{\mathcal{Y}}	\newcommand{\mcZ}{\mathcal{Z}}

%---------------------------------------
% Bold Math Fonts :-
%---------------------------------------

%Captital Letters
\newcommand{\bmA}{\boldsymbol{A}}	\newcommand{\bmB}{\boldsymbol{B}}
\newcommand{\bmC}{\boldsymbol{C}}	\newcommand{\bmD}{\boldsymbol{D}}
\newcommand{\bmE}{\boldsymbol{E}}	\newcommand{\bmF}{\boldsymbol{F}}
\newcommand{\bmG}{\boldsymbol{G}}	\newcommand{\bmH}{\boldsymbol{H}}
\newcommand{\bmI}{\boldsymbol{I}}	\newcommand{\bmJ}{\boldsymbol{J}}
\newcommand{\bmK}{\boldsymbol{K}}	\newcommand{\bmL}{\boldsymbol{L}}
\newcommand{\bmM}{\boldsymbol{M}}	\newcommand{\bmN}{\boldsymbol{N}}
\newcommand{\bmO}{\boldsymbol{O}}	\newcommand{\bmP}{\boldsymbol{P}}
\newcommand{\bmQ}{\boldsymbol{Q}}	\newcommand{\bmR}{\boldsymbol{R}}
\newcommand{\bmS}{\boldsymbol{S}}	\newcommand{\bmT}{\boldsymbol{T}}
\newcommand{\bmU}{\boldsymbol{U}}	\newcommand{\bmV}{\boldsymbol{V}}
\newcommand{\bmW}{\boldsymbol{W}}	\newcommand{\bmX}{\boldsymbol{X}}
\newcommand{\bmY}{\boldsymbol{Y}}	\newcommand{\bmZ}{\boldsymbol{Z}}
%Small Letters
\newcommand{\bma}{\boldsymbol{a}}	\newcommand{\bmb}{\boldsymbol{b}}
\newcommand{\bmc}{\boldsymbol{c}}	\newcommand{\bmd}{\boldsymbol{d}}
\newcommand{\bme}{\boldsymbol{e}}	\newcommand{\bmf}{\boldsymbol{f}}
\newcommand{\bmg}{\boldsymbol{g}}	\newcommand{\bmh}{\boldsymbol{h}}
\newcommand{\bmi}{\boldsymbol{i}}	\newcommand{\bmj}{\boldsymbol{j}}
\newcommand{\bmk}{\boldsymbol{k}}	\newcommand{\bml}{\boldsymbol{l}}
\newcommand{\bmm}{\boldsymbol{m}}	\newcommand{\bmn}{\boldsymbol{n}}
\newcommand{\bmo}{\boldsymbol{o}}	\newcommand{\bmp}{\boldsymbol{p}}
\newcommand{\bmq}{\boldsymbol{q}}	\newcommand{\bmr}{\boldsymbol{r}}
\newcommand{\bms}{\boldsymbol{s}}	\newcommand{\bmt}{\boldsymbol{t}}
\newcommand{\bmu}{\boldsymbol{u}}	\newcommand{\bmv}{\boldsymbol{v}}
\newcommand{\bmw}{\boldsymbol{w}}	\newcommand{\bmx}{\boldsymbol{x}}
\newcommand{\bmy}{\boldsymbol{y}}	\newcommand{\bmz}{\boldsymbol{z}}

%---------------------------------------
% Scr Math Fonts :-
%---------------------------------------

\newcommand{\sA}{{\mathscr{A}}}   \newcommand{\sB}{{\mathscr{B}}}
\newcommand{\sC}{{\mathscr{C}}}   \newcommand{\sD}{{\mathscr{D}}}
\newcommand{\sE}{{\mathscr{E}}}   \newcommand{\sF}{{\mathscr{F}}}
\newcommand{\sG}{{\mathscr{G}}}   \newcommand{\sH}{{\mathscr{H}}}
\newcommand{\sI}{{\mathscr{I}}}   \newcommand{\sJ}{{\mathscr{J}}}
\newcommand{\sK}{{\mathscr{K}}}   \newcommand{\sL}{{\mathscr{L}}}
\newcommand{\sM}{{\mathscr{M}}}   \newcommand{\sN}{{\mathscr{N}}}
\newcommand{\sO}{{\mathscr{O}}}   \newcommand{\sP}{{\mathscr{P}}}
\newcommand{\sQ}{{\mathscr{Q}}}   \newcommand{\sR}{{\mathscr{R}}}
\newcommand{\sS}{{\mathscr{S}}}   \newcommand{\sT}{{\mathscr{T}}}
\newcommand{\sU}{{\mathscr{U}}}   \newcommand{\sV}{{\mathscr{V}}}
\newcommand{\sW}{{\mathscr{W}}}   \newcommand{\sX}{{\mathscr{X}}}
\newcommand{\sY}{{\mathscr{Y}}}   \newcommand{\sZ}{{\mathscr{Z}}}


%---------------------------------------
% Math Fraktur Font
%---------------------------------------

%Captital Letters
\newcommand{\mfA}{\mathfrak{A}}	\newcommand{\mfB}{\mathfrak{B}}
\newcommand{\mfC}{\mathfrak{C}}	\newcommand{\mfD}{\mathfrak{D}}
\newcommand{\mfE}{\mathfrak{E}}	\newcommand{\mfF}{\mathfrak{F}}
\newcommand{\mfG}{\mathfrak{G}}	\newcommand{\mfH}{\mathfrak{H}}
\newcommand{\mfI}{\mathfrak{I}}	\newcommand{\mfJ}{\mathfrak{J}}
\newcommand{\mfK}{\mathfrak{K}}	\newcommand{\mfL}{\mathfrak{L}}
\newcommand{\mfM}{\mathfrak{M}}	\newcommand{\mfN}{\mathfrak{N}}
\newcommand{\mfO}{\mathfrak{O}}	\newcommand{\mfP}{\mathfrak{P}}
\newcommand{\mfQ}{\mathfrak{Q}}	\newcommand{\mfR}{\mathfrak{R}}
\newcommand{\mfS}{\mathfrak{S}}	\newcommand{\mfT}{\mathfrak{T}}
\newcommand{\mfU}{\mathfrak{U}}	\newcommand{\mfV}{\mathfrak{V}}
\newcommand{\mfW}{\mathfrak{W}}	\newcommand{\mfX}{\mathfrak{X}}
\newcommand{\mfY}{\mathfrak{Y}}	\newcommand{\mfZ}{\mathfrak{Z}}
%Small Letters
\newcommand{\mfa}{\mathfrak{a}}	\newcommand{\mfb}{\mathfrak{b}}
\newcommand{\mfc}{\mathfrak{c}}	\newcommand{\mfd}{\mathfrak{d}}
\newcommand{\mfe}{\mathfrak{e}}	\newcommand{\mff}{\mathfrak{f}}
\newcommand{\mfg}{\mathfrak{g}}	\newcommand{\mfh}{\mathfrak{h}}
\newcommand{\mfi}{\mathfrak{i}}	\newcommand{\mfj}{\mathfrak{j}}
\newcommand{\mfk}{\mathfrak{k}}	\newcommand{\mfl}{\mathfrak{l}}
\newcommand{\mfm}{\mathfrak{m}}	\newcommand{\mfn}{\mathfrak{n}}
\newcommand{\mfo}{\mathfrak{o}}	\newcommand{\mfp}{\mathfrak{p}}
\newcommand{\mfq}{\mathfrak{q}}	\newcommand{\mfr}{\mathfrak{r}}
\newcommand{\mfs}{\mathfrak{s}}	\newcommand{\mft}{\mathfrak{t}}
\newcommand{\mfu}{\mathfrak{u}}	\newcommand{\mfv}{\mathfrak{v}}
\newcommand{\mfw}{\mathfrak{w}}	\newcommand{\mfx}{\mathfrak{x}}
\newcommand{\mfy}{\mathfrak{y}}	\newcommand{\mfz}{\mathfrak{z}}


\newcommand{\mytitle}{CS1231S Tutorial 8}
\newcommand{\myauthor}{github/omgeta}
\newcommand{\mydate}{AY 24/25 Sem 1}

\begin{document}
\raggedright
\footnotesize
\begin{center}
{\normalsize{\textbf{\mytitle}}} \\
{\footnotesize{\mydate\hspace{2pt}\textemdash\hspace{2pt}\myauthor}}
\end{center}
\setlist{topsep=-1em, itemsep=-1em, parsep=2em}
%%%%%%%%%%%%%%%%%%%%%%%%%%%%%%%%%%%%%%%%%%%%%%%%%%%%%%
%                      Begin                         %
%%%%%%%%%%%%%%%%%%%%%%%%%%%%%%%%%%%%%%%%%%%%%%%%%%%%%%
\begin{enumerate}[Q\arabic*.]
  \item $\displaystyle g(n) = (-1)^n\ceil{\frac{n}{2}} \qed$ 

  \item 
    \begin{enumerate}[(\alph*)]
      \item \textbf{Direct Proof} 
        \begin{enumproof}
        \item $b_1, b_2, \cdots$ is a sequence in which every element of $B$ appears\hfill(Lemma 9.2)
        \item Let $|C| = n$, then $C = \{c_1, c_2, \cdots, c_n\}$\hfill(Definition of finite sets)
        \item Then $c_1, c_2, \cdots, c_n,b_1, b_2, \cdots$ is a sequence in which every element of $B \cup C$ appears
        \item $\therefore B\cup C$ is countable$\qed$\hfill(Lemma 9.2)
        \end{enumproof}

      \item \textbf{Direct Proof}
        \begin{enumproof}
        \item Since $B$ is countably infinite set, $\exists$ bijection $f: \ZZ^+ \rightarrow B$
        \item Let $C' = C \setminus B = \{c_1, c_2, \cdots, c_k\}$
        \item Define $g: \ZZ^+ \rightarrow B \cup C$:
          \begin{align*}
            g(i) = 
            \begin{cases}
              c_i & i <= k\\
              f(i-k) & \text{otherwise}
            \end{cases}
          \end{align*}
        \item Prove $g$ is injective:
          \begin{enumproof}
          \item Suppose $x_1, x_2 \in \ZZ^+$ s.t. $g(x_1) = g(x_2)$
          \item Case 1 ($x_1, x_2 \leq k$): $c_{x_1} = c_{x_2} \implies x_1 = x_2$\hfill(Distinct values of $C$)
          \item Case 2 ($x_1, x_2 > k$): $f(x_1 - k) = f(x_2 - k) \implies x_1 = x_2$\hfill(Injectivity of $f$)
          \item Case 3 (either $x_1$ or $x_2 \leq k$): WLOG, $x_1\leq k \implies c_{x_1} = f(x_1 - k)$ but this is a contradiction with $B \cap C' = \phi$
          \item In both cases, $x_1 = x_2$
          \end{enumproof}
        \item Prove $g$ is surjective:
          \begin{enumproof}
          \item Suppose $y \in B \cup C$
          \item Case 1 ($y \in B$): $\exists i, (g(i) = y)$\hfill(Surjectivity of $f$)
          \item Case 2 ($y \not\in B$): $\exists c_i = y \implies \exists i, g(i) = y$
          \item In both cases, $\exists i, (g(i) = y)$
          \end{enumproof}
        \item $\therefore g: \ZZ^+ \rightarrow B \cup C$ is bijective\hfill(Definition of bijection)
        \item $\therefore B \cup C$ is countable$\qed$\hfill(Definition of countably infinite) 
        \end{enumproof}
    \end{enumerate}

  \item
    \begin{enumerate}[(\alph*)]
      \item We cannot assume $A_{k+1} = \phi$, and must instead solve for the general case where $A_{k+1}$ is any finite set $\qed$
      \item Suppose $A_k = \{k\}$, then $\displaystyle \bigcup_{k=1}^{\infty}A_k = \ZZ^+$ which is infinite, disproving the statement $\qed$
    \end{enumerate}

  \item 
    \begin{enumerate}[(\alph*)]
      \item \textbf{Proof by 1MI}
      \begin{enumproof}
      \item Let $P(n) \equiv \bigcup_{i=1}^n A_i$ is countable for $n \in \ZZ^+$
      \item Basis step: $\bigcup_{i=1}^1 A_1 = A_1$ which is given to be countable, therefore $P(1)$ is true
      \item Assume $P(k)$ for some $k \in \ZZ^+$
      \item Inductive step:
        \begin{enumproof}
        \item $\bigcup_{i=1}^{k+1} = (\bigcup_{i=1}^kA_i) \cup A_{k+1}$
        \item By induction hypothesis $\bigcup_{i=1}^kA_i$ is countable, and $A_{k+1}$ is given to be countable, therefore their union is countable\hfill(Lemma 9.4)
        \item $P(k+1)$ is true
        \end{enumproof}
      \item Therefore, $\bigcup_{i=1}^nA_i$ is countable for any $n \in \ZZ^+ \qed$
      \end{enumproof}
      \item No, by Qn 3(b) $\qed$
    \end{enumerate}
  \pagebreak

  \item \textbf{Direct Proof}
    \begin{enumproof}
    \item Given $\ZZ^+ \times \ZZ^+$ is countable, we have bijection $f: \ZZ^+ \rightarrow \ZZ^+ \times \ZZ^+$
    \item There exists sequence $s_{i1}, s_{i2}, \cdots \in S_i$ in which every element of $S_i$ appears\hfill(Lemma 9.2) 
    \item Hence, $\forall s_{ij} \in \bigcup_{i\in\ZZ^+}$, we have $(i, j) \in \ZZ^+ \times \ZZ^+$
    \item Define sequence $c_1, c_2, \cdots$, s.t. $c_k = b_{ij}$ whenever $f(k) = (i, j)$
    \item It suffices to show any element of $\bigcup_{i\in\ZZ^+}S_i$ appears in the sequence defined in line 4:
      \begin{enumproof}
      \item Let $x \in \bigcup_{i\in\ZZ^+}S_i$ 
      \item $\exists i\in\ZZ^+(x\in S_i)$\hfill(Definition of union)
      \item $\exists j \in \ZZ^+ (x = b_{ij})$\hfill(Line 3)
      \item $\exists k \in \ZZ^+(x=c_k)$\hfill(Definition of sequence)
      \end{enumproof}
    \item Therefore, $\bigcup_{i\in\ZZ^+}$ is countable$\qed$\hfill(Lemma 9.2)
    \end{enumproof}

  \item \textbf{Direct Proof} 
    \begin{enumproof}
    \item Take $B' \subseteq B$ s.t. $B'$ is countably infinite\hfill(Proposition 9.3)
    \item Since $B$ is countably infinite set, $\exists$ bijection $f: \ZZ^+ \rightarrow B$
    \item Let $C' = C \setminus B = \{c_1, c_2, \cdots, c_k\}$
    \item $B' \cup C'$ is countable\hfill(Qn 2)
    \item $\exists$ bijection $f: B' \cup C' \rightarrow B'$\hfill(Definition of cardinality)
    \item Define $g: B \cup C \rightarrow B$:
      \begin{align*}
        g(x) =
        \begin{cases}
          f(x) & x \in B' \cup C'\\
          x & \text{otherwise}
        \end{cases} \qed
      \end{align*}
    \end{enumproof}
  
  \item \textbf{Proof by Contradiction}
    \begin{enumproof}
    \item Suppose not, that is, $\powerset(A)$ is countable
      \begin{enumproof}
      \item $\forall a \in A, \{a\} \in \powerset(A) \implies \powerset(A)$ is infinite 
      \item $\exists$ sequence $a_1, a_2, \cdots \in A$ in which every element of $A$ appears\hfill(Lemma 9.2)
      \item $\exists$ sequence $S_1, S_2, \cdots \in \powerset(A)$ in which every element of $\powerset(A)$ appears\hfill(Lemma 9.2)
      \item Construct $X = \{a_i : a_i \not\in S_i\}$
      \item Note that $\forall S_i \in \powerset(A), X \neq S_i$
      \item But $X \in \powerset(A)$ which contradicts 1.3.
      \end{enumproof}
    \item Hence, the supposition is false.
    \item $\therefore \powerset(A)$ is uncountable$\qed$\hfill(Contradiction rule)
    \end{enumproof}

  \item 
    \begin{enumerate}[(\alph*)]
      \item \textbf{Direct Proof}
        \begin{enumproof}
        \item Suppose $R$ is a reflexive relation on $A$ for $|A| = n \in \NN$
        \item $A = \{a_1, a_2, \cdots, a_n\}$\hfill(Definition of finite set)
        \item $\forall a \in A, (a, a) \in R$\hfill(Definition of reflexive relation)
        \item Define $f: A\rightarrow R$. where $f(a) = (a, a)$, $\forall a \in A$
        \item $\forall x, y \in A, f(x) = f(y) \implies (x, x) = (y, y) \implies x=y$. Therefore, $f$ is injective.
        \item $|A| \leq |R| \qed$\hfill(By pigeonhole principle)
        \end{enumproof}
      \item \textbf{Counterexample:} $A = \{a\}, R = \phi \implies |A| = 1 > 0 = |R| \qed$
      \item \textbf{Counterexample:} $A = \{a\}, R = \phi \implies |A| = 1 > 0 = |R| \qed$
    \end{enumerate}
  \pagebreak
  \item \textbf{Proof by 2MI}
    \begin{enumproof}
    \item Let $P(n) \equiv Even(F_n) \iff Even(F_{n+3}), \forall n \in \NN$
    \item Basis step: $F_0 = 0$ and $F_3 = 2$ are both even, therefore $P(0)$ is true
    \item Assume $P(i)$ is true for $0 \leq i \leq k$:
      \begin{align*}
        Even(F_{k}) \iff Even(F_{k+3})
      \end{align*}
    \item Inductive step:
      \begin{enumproof}
      \item $Even(F_{k+1}) \iff Even(F_{k} + F_{k-1})$\hfill(Definition of $F$)
      \item $\quad\quad\iff (Even(F_k) \iff Even(F_{k-1}))$\hfill(Fact 1)
      \item $\quad\quad\iff (Even(F_{k+3}) \iff Even(F_{k+2}))$\hfill(By inductive hypothesis)
      \item $\quad\quad\iff Even(F_{k+3} + F_{k+2})$\hfill(Fact 1)
      \item $\quad\quad\iff Even(F_{k+4})$\hfill(Definition of $F$)
      \item $P(k+1)$ is true
      \end{enumproof}
    \item Therefore, $P(n)$ is true for all $n \in \NN \qed$
    \end{enumproof}

  \item 
    \begin{enumerate}[(\alph*)]
      \item \textbf{Direct Proof}
      \begin{enumproof}
      \item Prove F1:
        \begin{enumproof}
        \item Suppose $[x] \in X /\mathord{\sim}$
        \item Since $g$ is a function defined on $\forall x \in X$, $f([x]) = g(x)$
        \item $\forall [x] \in X /\mathord{\sim}$ $\exists g(x) \in Y (g(x) = f([x]))$
        \end{enumproof}
      \item Prove F2:
        \begin{enumproof}
        \item Suppose $[x] = [y]$ s.t. $f([x]) = f([y])$ 
        \item $x \sim y$\hfill(Definition of equivalence class)
        \item $g(x) = g(y)$\hfill(Definition of $g$)
        \item $\forall [x] \in X /\mathord{\sim} z_1, z_2 \in Y(([x], z_1) \in f \land ([x], z_2) \in f \rightarrow z_1 = z_2)$
        \end{enumproof}
      \item Therefore, $f$ is well-defined $\qed$
      \end{enumproof}
      \item \textbf{Direct Proof}
        \begin{enumproof}
        \item Suppose $f([x]) = f([y])$
        \item $g(x) = g(y)$\hfill(Definition of $f$)
        \item $x \sim y$\hfill(Definition of $g$)
        \item $[x] = [y]$\hfill(Definition of equivalent classes)
        \item $f$ is injective$\qed$
        \end{enumproof}
      \item $f: 4 \qed$\\
        $g: 4 \qed$\\
        $f^{-1}\circ g: 3 \qed$
    \end{enumerate}
\end{enumerate}
%%%%%%%%%%%%%%%%%%%%%%%%%%%%%%%%%%%%%%%%%%%%%%%%%%%%%%
%                       End                          %
%%%%%%%%%%%%%%%%%%%%%%%%%%%%%%%%%%%%%%%%%%%%%%%%%%%%%%

\end{document}

\documentclass[12pt, a4paper]{article}

\usepackage[utf8]{inputenc}
\usepackage[mathscr]{euscript}
\let\euscr\mathscr \let\mathscr\relax
\usepackage[scr]{rsfso}
\usepackage{amssymb,amsmath,amsthm,amsfonts}
\usepackage[shortlabels]{enumitem}
\usepackage{multicol,multirow}
\usepackage{lipsum}
\usepackage{balance}
\usepackage{calc}
\usepackage[colorlinks=true,citecolor=blue,linkcolor=blue]{hyperref}
\usepackage{import}
\usepackage{xifthen}
\usepackage{pdfpages}
\usepackage{transparent}
\usepackage{listings}

\newcommand{\incfig}[2][1.0]{
    \def\svgwidth{#1\columnwidth}
    \import{./figures/}{#2.pdf_tex}
}

\newlist{enumproof}{enumerate}{4}
\setlist[enumproof,1]{label=\arabic*., parsep=1em}
\setlist[enumproof,2]{label=\arabic{enumproofi}.\arabic*., parsep=1em}
\setlist[enumproof,3]{label=\arabic{enumproofi}.\arabic{enumproofii}.\arabic*., parsep=1em}
\setlist[enumproof,4]{label=\arabic{enumproofi}.\arabic{enumproofii}.\arabic{enumproofiii}.\arabic*., parsep=1em}

\renewcommand{\qedsymbol}{\ensuremath{\blacksquare}}

\lstdefinestyle{mystyle}{
  language=C, % Set the language to C
  commentstyle=\color{codegray}, % Color for comments
  keywordstyle=\color{orange}, % Color for basic keywords
  stringstyle=\color{mauve}, % Color for strings
  basicstyle={\ttfamily\footnotesize}, % Basic font style
  breakatwhitespace=false,         
  breaklines=true,                 
  captionpos=b,                    
  keepspaces=true,                 
  numbers=none,                    
  tabsize=2,
  morekeywords=[2]{\#include, \#define, \#ifdef, \#ifndef, \#endif, \#pragma, \#else, \#elif}, % Preprocessor directives
  keywordstyle=[2]\color{codegreen}, % Style for preprocessor directives
  morekeywords=[3]{int, char, float, double, void, struct, union, enum, const, volatile, static, extern, register, inline, restrict, _Bool, _Complex, _Imaginary, size_t, ssize_t, FILE}, % C standard types and common identifiers
  keywordstyle=[3]\color{identblue}, % Style for types and common identifiers
  morekeywords=[4]{printf, scanf, fopen, fclose, malloc, free, calloc, realloc, perror, strtok, strncpy, strcpy, strcmp, strlen}, % Standard library functions
  keywordstyle=[4]\color{cyan}, % Style for library functions
}

\usepackage{ifthen}
\usepackage[landscape]{geometry}
\usepackage[shortlabels]{enumitem}

\ifthenelse{\lengthtest { \paperwidth = 11in}}
    { \geometry{top=.5in,left=.5in,right=.5in,bottom=.5in} }
	{\ifthenelse{ \lengthtest{ \paperwidth = 297mm}}
		{\geometry{top=1cm,left=1cm,right=1cm,bottom=1cm} }
		{\geometry{top=1cm,left=1cm,right=1cm,bottom=1cm} }
	}

\pagestyle{empty}
\makeatletter
\renewcommand\thesection{\arabic{section}.}
\renewcommand{\section}{\@startsection{section}{1}{0mm}%
                                {-1ex plus -.5ex minus -.2ex}%
                                {0.05ex}%x
                                {\normalfont\normalsize\bfseries}}
\renewcommand{\subsection}{\@startsection{subsection}{2}{0mm}%
                                {-1ex plus -.5ex minus -.2ex}%
                                {0.05ex}%
                                {\normalfont\small\bfseries}}
\renewcommand{\subsubsection}{\@startsection{subsubsection}{3}{0mm}%
                                {-1ex plus -.5ex minus -.2ex}%
                                {0.05ex}%
                                {\normalfont\footnotesize\bfseries}}
\newcommand{\colbreak}{\vfill\null\columnbreak}
\makeatother
\setcounter{secnumdepth}{1}
\setlength{\parindent}{0pt}
\setlength{\parskip}{0.7em}

\setlist[itemize]{itemsep=0.6ex, topsep=-2pt, partopsep=0pt, parsep=0pt}
\setlist[enumerate]{itemsep=0.6ex, topsep=-2pt, partopsep=0pt, parsep=0pt}

% Things Lie
\newcommand{\kb}{\mathfrak b}
\newcommand{\kg}{\mathfrak g}
\newcommand{\kh}{\mathfrak h}
\newcommand{\kn}{\mathfrak n}
\newcommand{\ku}{\mathfrak u}
\newcommand{\kz}{\mathfrak z}
\DeclareMathOperator{\Ext}{Ext} % Ext functor
\DeclareMathOperator{\Tor}{Tor} % Tor functor
\newcommand{\gl}{\opname{\mathfrak{gl}}} % frak gl group
\renewcommand{\sl}{\opname{\mathfrak{sl}}} % frak sl group chktex 6

% More script letters etc.
\newcommand{\SA}{\mathcal A}
\newcommand{\SB}{\mathcal B}
\newcommand{\SC}{\mathcal C}
\newcommand{\SF}{\mathcal F}
\newcommand{\SG}{\mathcal G}
\newcommand{\SH}{\mathcal H}
\newcommand{\OO}{\mathcal O}

\newcommand{\SCA}{\mathscr A}
\newcommand{\SCB}{\mathscr B}
\newcommand{\SCC}{\mathscr C}
\newcommand{\SCD}{\mathscr D}
\newcommand{\SCE}{\mathscr E}
\newcommand{\SCF}{\mathscr F}
\newcommand{\SCG}{\mathscr G}
\newcommand{\SCH}{\mathscr H}

% Mathfrak primes
\newcommand{\km}{\mathfrak m}
\newcommand{\kp}{\mathfrak p}
\newcommand{\kq}{\mathfrak q}

% number sets
\newcommand{\RR}[1][]{\ensuremath{\ifstrempty{#1}{\mathbb{R}}{\mathbb{R}^{#1}}}}
\newcommand{\NN}[1][]{\ensuremath{\ifstrempty{#1}{\mathbb{N}}{\mathbb{N}^{#1}}}}
\newcommand{\ZZ}[1][]{\ensuremath{\ifstrempty{#1}{\mathbb{Z}}{\mathbb{Z}^{#1}}}}
\newcommand{\QQ}[1][]{\ensuremath{\ifstrempty{#1}{\mathbb{Q}}{\mathbb{Q}^{#1}}}}
\newcommand{\CC}[1][]{\ensuremath{\ifstrempty{#1}{\mathbb{C}}{\mathbb{C}^{#1}}}}
\newcommand{\PP}[1][]{\ensuremath{\ifstrempty{#1}{\mathbb{P}}{\mathbb{P}^{#1}}}}
\newcommand{\HH}[1][]{\ensuremath{\ifstrempty{#1}{\mathbb{H}}{\mathbb{H}^{#1}}}}
\newcommand{\FF}[1][]{\ensuremath{\ifstrempty{#1}{\mathbb{F}}{\mathbb{F}^{#1}}}}
% expected value
\newcommand{\EE}{\ensuremath{\mathbb{E}}}
\newcommand{\charin}{\text{ char }}
\DeclareMathOperator{\sign}{sign}
\DeclareMathOperator{\Aut}{Aut}
\DeclareMathOperator{\Inn}{Inn}
\DeclareMathOperator{\Syl}{Syl}
\DeclareMathOperator{\Gal}{Gal}
\DeclareMathOperator{\GL}{GL} % General linear group
\DeclareMathOperator{\SL}{SL} % Special linear group

%---------------------------------------
% BlackBoard Math Fonts :-
%---------------------------------------

%Captital Letters
\newcommand{\bbA}{\mathbb{A}}	\newcommand{\bbB}{\mathbb{B}}
\newcommand{\bbC}{\mathbb{C}}	\newcommand{\bbD}{\mathbb{D}}
\newcommand{\bbE}{\mathbb{E}}	\newcommand{\bbF}{\mathbb{F}}
\newcommand{\bbG}{\mathbb{G}}	\newcommand{\bbH}{\mathbb{H}}
\newcommand{\bbI}{\mathbb{I}}	\newcommand{\bbJ}{\mathbb{J}}
\newcommand{\bbK}{\mathbb{K}}	\newcommand{\bbL}{\mathbb{L}}
\newcommand{\bbM}{\mathbb{M}}	\newcommand{\bbN}{\mathbb{N}}
\newcommand{\bbO}{\mathbb{O}}	\newcommand{\bbP}{\mathbb{P}}
\newcommand{\bbQ}{\mathbb{Q}}	\newcommand{\bbR}{\mathbb{R}}
\newcommand{\bbS}{\mathbb{S}}	\newcommand{\bbT}{\mathbb{T}}
\newcommand{\bbU}{\mathbb{U}}	\newcommand{\bbV}{\mathbb{V}}
\newcommand{\bbW}{\mathbb{W}}	\newcommand{\bbX}{\mathbb{X}}
\newcommand{\bbY}{\mathbb{Y}}	\newcommand{\bbZ}{\mathbb{Z}}

%---------------------------------------
% MathCal Fonts :-
%---------------------------------------

%Captital Letters
\newcommand{\mcA}{\mathcal{A}}	\newcommand{\mcB}{\mathcal{B}}
\newcommand{\mcC}{\mathcal{C}}	\newcommand{\mcD}{\mathcal{D}}
\newcommand{\mcE}{\mathcal{E}}	\newcommand{\mcF}{\mathcal{F}}
\newcommand{\mcG}{\mathcal{G}}	\newcommand{\mcH}{\mathcal{H}}
\newcommand{\mcI}{\mathcal{I}}	\newcommand{\mcJ}{\mathcal{J}}
\newcommand{\mcK}{\mathcal{K}}	\newcommand{\mcL}{\mathcal{L}}
\newcommand{\mcM}{\mathcal{M}}	\newcommand{\mcN}{\mathcal{N}}
\newcommand{\mcO}{\mathcal{O}}	\newcommand{\mcP}{\mathcal{P}}
\newcommand{\mcQ}{\mathcal{Q}}	\newcommand{\mcR}{\mathcal{R}}
\newcommand{\mcS}{\mathcal{S}}	\newcommand{\mcT}{\mathcal{T}}
\newcommand{\mcU}{\mathcal{U}}	\newcommand{\mcV}{\mathcal{V}}
\newcommand{\mcW}{\mathcal{W}}	\newcommand{\mcX}{\mathcal{X}}
\newcommand{\mcY}{\mathcal{Y}}	\newcommand{\mcZ}{\mathcal{Z}}

%---------------------------------------
% Bold Math Fonts :-
%---------------------------------------

%Captital Letters
\newcommand{\bmA}{\boldsymbol{A}}	\newcommand{\bmB}{\boldsymbol{B}}
\newcommand{\bmC}{\boldsymbol{C}}	\newcommand{\bmD}{\boldsymbol{D}}
\newcommand{\bmE}{\boldsymbol{E}}	\newcommand{\bmF}{\boldsymbol{F}}
\newcommand{\bmG}{\boldsymbol{G}}	\newcommand{\bmH}{\boldsymbol{H}}
\newcommand{\bmI}{\boldsymbol{I}}	\newcommand{\bmJ}{\boldsymbol{J}}
\newcommand{\bmK}{\boldsymbol{K}}	\newcommand{\bmL}{\boldsymbol{L}}
\newcommand{\bmM}{\boldsymbol{M}}	\newcommand{\bmN}{\boldsymbol{N}}
\newcommand{\bmO}{\boldsymbol{O}}	\newcommand{\bmP}{\boldsymbol{P}}
\newcommand{\bmQ}{\boldsymbol{Q}}	\newcommand{\bmR}{\boldsymbol{R}}
\newcommand{\bmS}{\boldsymbol{S}}	\newcommand{\bmT}{\boldsymbol{T}}
\newcommand{\bmU}{\boldsymbol{U}}	\newcommand{\bmV}{\boldsymbol{V}}
\newcommand{\bmW}{\boldsymbol{W}}	\newcommand{\bmX}{\boldsymbol{X}}
\newcommand{\bmY}{\boldsymbol{Y}}	\newcommand{\bmZ}{\boldsymbol{Z}}
%Small Letters
\newcommand{\bma}{\boldsymbol{a}}	\newcommand{\bmb}{\boldsymbol{b}}
\newcommand{\bmc}{\boldsymbol{c}}	\newcommand{\bmd}{\boldsymbol{d}}
\newcommand{\bme}{\boldsymbol{e}}	\newcommand{\bmf}{\boldsymbol{f}}
\newcommand{\bmg}{\boldsymbol{g}}	\newcommand{\bmh}{\boldsymbol{h}}
\newcommand{\bmi}{\boldsymbol{i}}	\newcommand{\bmj}{\boldsymbol{j}}
\newcommand{\bmk}{\boldsymbol{k}}	\newcommand{\bml}{\boldsymbol{l}}
\newcommand{\bmm}{\boldsymbol{m}}	\newcommand{\bmn}{\boldsymbol{n}}
\newcommand{\bmo}{\boldsymbol{o}}	\newcommand{\bmp}{\boldsymbol{p}}
\newcommand{\bmq}{\boldsymbol{q}}	\newcommand{\bmr}{\boldsymbol{r}}
\newcommand{\bms}{\boldsymbol{s}}	\newcommand{\bmt}{\boldsymbol{t}}
\newcommand{\bmu}{\boldsymbol{u}}	\newcommand{\bmv}{\boldsymbol{v}}
\newcommand{\bmw}{\boldsymbol{w}}	\newcommand{\bmx}{\boldsymbol{x}}
\newcommand{\bmy}{\boldsymbol{y}}	\newcommand{\bmz}{\boldsymbol{z}}

%---------------------------------------
% Scr Math Fonts :-
%---------------------------------------

\newcommand{\sA}{{\mathscr{A}}}   \newcommand{\sB}{{\mathscr{B}}}
\newcommand{\sC}{{\mathscr{C}}}   \newcommand{\sD}{{\mathscr{D}}}
\newcommand{\sE}{{\mathscr{E}}}   \newcommand{\sF}{{\mathscr{F}}}
\newcommand{\sG}{{\mathscr{G}}}   \newcommand{\sH}{{\mathscr{H}}}
\newcommand{\sI}{{\mathscr{I}}}   \newcommand{\sJ}{{\mathscr{J}}}
\newcommand{\sK}{{\mathscr{K}}}   \newcommand{\sL}{{\mathscr{L}}}
\newcommand{\sM}{{\mathscr{M}}}   \newcommand{\sN}{{\mathscr{N}}}
\newcommand{\sO}{{\mathscr{O}}}   \newcommand{\sP}{{\mathscr{P}}}
\newcommand{\sQ}{{\mathscr{Q}}}   \newcommand{\sR}{{\mathscr{R}}}
\newcommand{\sS}{{\mathscr{S}}}   \newcommand{\sT}{{\mathscr{T}}}
\newcommand{\sU}{{\mathscr{U}}}   \newcommand{\sV}{{\mathscr{V}}}
\newcommand{\sW}{{\mathscr{W}}}   \newcommand{\sX}{{\mathscr{X}}}
\newcommand{\sY}{{\mathscr{Y}}}   \newcommand{\sZ}{{\mathscr{Z}}}


%---------------------------------------
% Math Fraktur Font
%---------------------------------------

%Captital Letters
\newcommand{\mfA}{\mathfrak{A}}	\newcommand{\mfB}{\mathfrak{B}}
\newcommand{\mfC}{\mathfrak{C}}	\newcommand{\mfD}{\mathfrak{D}}
\newcommand{\mfE}{\mathfrak{E}}	\newcommand{\mfF}{\mathfrak{F}}
\newcommand{\mfG}{\mathfrak{G}}	\newcommand{\mfH}{\mathfrak{H}}
\newcommand{\mfI}{\mathfrak{I}}	\newcommand{\mfJ}{\mathfrak{J}}
\newcommand{\mfK}{\mathfrak{K}}	\newcommand{\mfL}{\mathfrak{L}}
\newcommand{\mfM}{\mathfrak{M}}	\newcommand{\mfN}{\mathfrak{N}}
\newcommand{\mfO}{\mathfrak{O}}	\newcommand{\mfP}{\mathfrak{P}}
\newcommand{\mfQ}{\mathfrak{Q}}	\newcommand{\mfR}{\mathfrak{R}}
\newcommand{\mfS}{\mathfrak{S}}	\newcommand{\mfT}{\mathfrak{T}}
\newcommand{\mfU}{\mathfrak{U}}	\newcommand{\mfV}{\mathfrak{V}}
\newcommand{\mfW}{\mathfrak{W}}	\newcommand{\mfX}{\mathfrak{X}}
\newcommand{\mfY}{\mathfrak{Y}}	\newcommand{\mfZ}{\mathfrak{Z}}
%Small Letters
\newcommand{\mfa}{\mathfrak{a}}	\newcommand{\mfb}{\mathfrak{b}}
\newcommand{\mfc}{\mathfrak{c}}	\newcommand{\mfd}{\mathfrak{d}}
\newcommand{\mfe}{\mathfrak{e}}	\newcommand{\mff}{\mathfrak{f}}
\newcommand{\mfg}{\mathfrak{g}}	\newcommand{\mfh}{\mathfrak{h}}
\newcommand{\mfi}{\mathfrak{i}}	\newcommand{\mfj}{\mathfrak{j}}
\newcommand{\mfk}{\mathfrak{k}}	\newcommand{\mfl}{\mathfrak{l}}
\newcommand{\mfm}{\mathfrak{m}}	\newcommand{\mfn}{\mathfrak{n}}
\newcommand{\mfo}{\mathfrak{o}}	\newcommand{\mfp}{\mathfrak{p}}
\newcommand{\mfq}{\mathfrak{q}}	\newcommand{\mfr}{\mathfrak{r}}
\newcommand{\mfs}{\mathfrak{s}}	\newcommand{\mft}{\mathfrak{t}}
\newcommand{\mfu}{\mathfrak{u}}	\newcommand{\mfv}{\mathfrak{v}}
\newcommand{\mfw}{\mathfrak{w}}	\newcommand{\mfx}{\mathfrak{x}}
\newcommand{\mfy}{\mathfrak{y}}	\newcommand{\mfz}{\mathfrak{z}}


\newcommand{\mytitle}{CS2040S Data Structures and Algorithms}
\newcommand{\myauthor}{github/omgeta}
\newcommand{\mydate}{AY 24/25 Sem 2}

\begin{document}
\raggedright
\footnotesize
\begin{multicols*}{3}
\setlength{\premulticols}{1pt}
\setlength{\postmulticols}{1pt}
\setlength{\multicolsep}{1pt}
\setlength{\columnsep}{2pt}

{\normalsize{\textbf{\mytitle}}} \\
{\footnotesize{\mydate\hspace{2pt}\textemdash\hspace{2pt}\myauthor}}
\vspace{-0.5em}
%%%%%%%%%%%%%%%%%%%%%%%%%%%%%%%%%%%%%%%%%%%%%%%%%%%%%%
%                      Begin                         %
%%%%%%%%%%%%%%%%%%%%%%%%%%%%%%%%%%%%%%%%%%%%%%%%%%%%%%
\section{Asymptotic Analysis}
Preconditions are conditions which must be true for the algorithm to work correctly.
Postconditions are conditions guaranteed to be true after the algorithm ends.

Invariants are conditions that remain unchanged and consistently true throughout the algorithm. \\Loop invariants are invariants for each loop iteration.

Runtime $T(n)$ is given if $\exists c, n_0>0$ s.t. $\forall n > n_0$:
\begin{enumerate}[\roman*.]
  \item $T(n) \leq cf(n) \iff T(n)=O(f(n))$
  \item $T(n) \geq cf(n) \iff T(n)=\Omega(f(n))$
  \item $T(n) = O(f(n)), \Omega(f(n))\iff T(n)=\Theta(f(n))$
\end{enumerate}

Order of growth is:\\
$O(1) < O(\log\log n) < O(\log n) < O(\log^2 n) < O(\sqrt{n}) < O(n) < O(n\log n) < O(n^2) < O(2^n) < O(2^{2n}) < O(n!)$

Common recurrences:
\begin{enumerate}[\roman*.]
  \item $T(n) = T(n/2) + O(1) = O(\log n)$
  \item $T(n) = T(n/2) + O(n) = O(n)$
  \item $T(n) = 2T(n/2) + O(1) = O(n)$
  \item $T(n) = 2T(n/2) + O(n) = O(n\log n)$
  \item $T(n) = T(n - 1) + O(1) = O(n)$
  \item $T(n) = T(n - 1) + O(n) = O(n^2)$
  \item $T(n) = 2T(n - 1) + O(1) = O(2^n)$
\end{enumerate}

\subsection{Mathematical Properties}
Useful properties:
\begin{enumerate}[\roman*.]
  \item $\log(xy) = \log x + \log y$\hfill(Product rule) 
  \item $\log(x/y) = \log x - \log y$\hfill(Quotient rule)
  \item $\log(x^k) = k\log x$\hfill(Power rule)
  \item $\log_bx = \frac{\log_ax}{\log_ab}$\hfill(Change of base)
  \item $x^a \cdot x^b = x^{a+b}$\hfill(Product rule) 
  \item $\frac{x^a}{x^b} = x^{a-b}$\hfill(Quotient rule)
  \item $(x^a)^b = x^{ab}$\hfill(Power rule)
\end{enumerate}
\colbreak

\subsection{Master Theorem}

Dividing function $T(n) = aT(n/b) + f(n)$ where $a>1, b>1, f(n) = \Theta(n^k\log^pn)$ has cases:
\begin{enumerate}[\roman*.]
  \item $\log_ba > k \implies T(n)=\Theta(n^{\log_ba})$
  \item $\log_ba = k \land p>-1 \implies T(n) = \Theta(n^k\log^{p+1}n)$\\
    \hspace{4.5em}$\land\hspace{0.25em}p=-1 \implies T(n) = \Theta(n^k\log\log n)$\\
    \hspace{4.5em}$\land\hspace{0.25em}p<-1 \implies T(n) = \Theta(n^k)$
  \item $\log_ba < k \land p \geq 0 \implies T(n) = \Theta(n^k\log^pn)$\\
    \hspace{4.5em}$\land\hspace{0.25em}p<0 \implies T(n) = \Theta(n^k)$
\end{enumerate}

Decreasing function $T(n) = aT(n-b) + f(n)$ where $a>0, b>0$ has cases:
\begin{enumerate}[\roman*.]
  \item $a < 1 \implies T(n)=O(f(n))$
  \item $a = 1 \implies T(n) = O(n\cdot f(n))$
  \item $a > 1 \implies T(n) = O(a^{n/b}\cdot f(n))$
\end{enumerate}

\section{Searching}
Binary search, $O(\log n)$, in a sorted array cuts search space in half until the target is found or range exhausted.
\begin{lstlisting}
int binarySearch(T[] arr, T x) {
  int lo = 0, hi = arr.length - 1;
  while (lo < hi) {
    int mid = lo + (hi - lo) / 2;
    if (arr[mid] < x) lo = mid + 1;
    else hi = mid;
  }
  return lo;
}
\end{lstlisting}
1D Peak Finding, $O(\log n)$:
\begin{enumerate}[\roman*.]
  \item Binary search on side with rising neighbour if not peak
\end{enumerate}

2D Peak Finding on $n\times m$, $O(n\log m)$:
\begin{enumerate}[\roman*.]
  \item Take global max of column, binary search on columns with rising neighbour if not peak.
  \item Optimise: Quadrant DnC. $O(n+m)$. 
\end{enumerate}

$k$th Smallest, $O(n)$:
\begin{enumerate}[\roman*.]
  \item QuickSelect, recurse on half with $k$
\end{enumerate}

\colbreak
\section{Sorting}
BubbleSort repeatedly swaps inverted adjacent elements up to form a globally sorted suffix. 
\begin{enumerate}[\roman*.]
  \item After $i$ iterations, last $i$ elements are correct
  \item Time: $O(n)$ (Best), $O(n^2)$ (Worst) 
  \item In-place and stable
\end{enumerate}

SelectionSort selects the smallest unsorted element and places it in the globally sorted prefix.
\begin{enumerate}[\roman*.]
  \item After $i$ iterations, first $i$ elements are correct
  \item Time: $O(n^2)$ (Best \& Worst)
  \item In-place but unstable
\end{enumerate}

InsertionSort adds elements to a locally sorted prefix.
\begin{enumerate}[\roman*.]
  \item After $i$ iterations, first $i$ elements are sorted
  \item Time: $O(n)$ (Best), $O(n^2)$ (Worst) 
  \item In-place and stable
\end{enumerate}

MergeSort divides the array into halves, sorts each half, and merges the sorted halves.
\begin{enumerate}[\roman*.]
  \item Locally sorted prefix in power of 2
  \item Time: $O(n\log n)$ (Best \& Worst)
  \item Space: $O(n\log n)$
  \item Out-of-place and stable
\end{enumerate}

QuickSort partitions the array around $\leq$ and $>$ pivot, then recursing on both partitions.
\begin{enumerate}[\roman*.]
  \item Elements before pivot are $\leq$, after pivot are $>$
  \item Time: $O(n\log n)$ (Best), $O(n^2)$ (Worst)
  \item In-place but unstable
  \item Optimise: randomized pivot, 3-way partition for $O(n\log n)$ worst-case, insertion sort for small $n$
\end{enumerate}

CountingSort counts number of objects with distinct keys.
\begin{enumerate}[\roman*.]
  \item Time: $O(n + k)$ where $k$ is number of keys
  \item Space: $O(n + k)$
\end{enumerate}

RadixSort sorts integers by individual digits.
\begin{enumerate}[\roman*.]
  \item Time: $O(nd)$ where $d$ is number of digits
  \item Space: $O(n + k)$
\end{enumerate}

\section{Trees}
\subsection{Binary Search Trees (BSTs)}
Binary search tree maintains that for any node, all left descendants are $\leq$ and all right descendants are $>$. Operations are $O(h)$ where $h$ is the height of the tree.

BSTs are balanced if $h = O(\log n)$.\\If balanced, (non-traversal) operations are $O(\log n)$. In the worst case, they are $O(n)$.

\begin{enumerate}[\roman*.]
  \item Height, $h$ is given by $\log n -1 \leq h \leq n$
  \item \lstinline|search|: Traverse left/right based on comparisons.
  \item \lstinline|insert|: Find position via search and add node.
  \item \lstinline|delete|: If not leaf, replace with successor.
  \item Traversal yields sorted elements. Time: $O(n)$
\end{enumerate}


\subsection{AVL Trees}
Self-balancing BSTs which are height-balanced, i.e. height difference between left/right subtrees differ by at most 1. 

\begin{enumerate}[\roman*.]
  \item Balance factor $= |height_{right}-height_{left}| \leq 1$.\\Nodes are left-heavy with balance factor $\leq -1$, or right-heavy with balance factor $\geq 1$.
  \item $h < 2\log n \iff 2^{\frac{h}{2}} \leq n \leq 2^{h+1}-1$ 
  \item \lstinline|insert|: After insertion, walk up tree and fix lowest unbalanced node (max 2 rotations)
  \item \lstinline|delete|: After deletion, fix all unbalanced nodes until root (up to $\Theta(\log n)$ rotations)
\end{enumerate}

To balance left-heavy node \lstinline|v|:
\begin{enumerate}[\roman*.]
  \item \lstinline|v.left| balanced or left-heavy\\$\implies$ \lstinline|rightRotate(v)|
  \item \lstinline|v.left| right-heavy\\$\implies$ \lstinline|leftRotate(v.left); rightRotate(v)|
\end{enumerate}

To balance right-heavy node \lstinline|v|:
\begin{enumerate}[\roman*.]
  \item \lstinline|v.right| balanced or right-heavy\\$\implies$ \lstinline|leftRotate(v)|
  \item \lstinline|v.right| left-heavy\\$\implies$ \lstinline|rightRotate(v.right); leftRotate(v)|
\end{enumerate}

\colbreak
\subsubsection{Rotations}
\incimg[0.8]{right_rotation}
\begin{minipage}{0.45\columnwidth}
  \begin{lstlisting}
Node leftRotate(B)
  D = B.right
  D.par = B.par
  B.par = D
  B.right = D.left
  D.left = B
  return D
  \end{lstlisting}
\end{minipage}
\begin{minipage}{0.5\columnwidth}
  \begin{lstlisting}
Node rightRotate(D)
  B = D.left
  B.par = D.par
  D.par = B
  D.left = B.right
  B.right = D
  return B
  \end{lstlisting}
\end{minipage}
\vspace{-2em}
\subsubsection{Problems}
\vspace{-0.5em}
Order Statistics, $O(\log n)$ if balanced
\begin{enumerate}[\roman*.]
  \item Augment nodes with subtree size $= size_{left} + size_{right} + 1$
  \item \lstinline|select|: find kth smallest element.
  \item \lstinline|rank|: get node position in sorted order.
\end{enumerate}

Counting Inversions, $O(n\log n)$:
\begin{enumerate}[\roman*.]
  \item After inserting each element, add \lstinline|i - tree.rank(ele)| to running count.
\end{enumerate}

Interval Query, $O(\log n)$
\begin{enumerate}[\roman*.]
  \item Nodes store interval, key is lower interval bound
  \item Augment nodes with subtree max upper bound
  \item \lstinline|interval|: If left subtree exists with $max \geq x.low$, recurse on left subtree, else on right subtree
\end{enumerate}

1D Range Query, $O(\log n + k)$ for $k$ elements in range:
\begin{enumerate}[\roman*.]
  \item Leaves are elements. Internal nodes are max of left subtree.
  \item \lstinline|range|: Find split node $O(\log n)$ and traverse leafs
\end{enumerate}

2D Range Query, $O(\log^2 n + k)$ for $k$ elements in range:
\begin{enumerate}[\roman*.]
  \item Build 1D x-tree for each x, then for each internal node, build y-tree for all y
  \item \lstinline|range|: $O(\log n)$ y-tree searches each of $O(\log n)$ 
\end{enumerate}

\colbreak
\subsection{Tries}
Tree storing strings where each node is a character, and each path is a key. Operations are $O(L)$, where $L$ is length of the key. Faster than $O(Lh)$ of strings in bBST. Both use $O(\text{text size})$ space but Trie has node overhead.
\begin{enumerate}[\roman*.]
\item \lstinline|prefixSearch|: Find all keys with prefix.\\Time: $O(L + k)$, where $k$ matches
\end{enumerate}

\subsection{(a,b)-Trees}
\incimg{ab_tree}
Self-balancing search trees where non-root nodes have between $a, b$ children for $2 \leq a \leq \frac{b+1}{2}$
\begin{enumerate}[\roman*.]
  \item Root has $[2, b]$ children.\\Internal nodes have $[a, b]$ children.
  \item Nodes with $k$ children have $k-1$ keys.\\Leaves have $[a-1,b-1]$ keys.
  \item All leaves at same depth
  \item \lstinline|search|: Time: $O(b\log_a n) = O(\log n)$
  \item \lstinline|insert|: Proactively split full nodes ($b-1$ keys) during search, guaranteeing parent of split node won\'t have excess keys after insertion of split node.\\Time: $O(\log n)$
  \item \lstinline|split|: Split about median node $z$ with $\geq 2a$ keys, s.t. left has $\geq a-1$ keys and right has $\geq a$ keys. Then, $z$ offered to parent with $\leq b-2$ keys.\\
    Time: $O(b)$ for splitting node with $b$ children
  \item \lstinline|delete|: Passively, delete node then recursively check parents for violation, carrying out merge/share with smallest adjacent sibling. For internal nodes, replace with successor.\\Time: $O(\log n)$ 
  \item \lstinline|merge|: Demote split node and join when $< b-1$ keys. Time: $O(b)$ 
  \item \lstinline|share|: \lstinline|merge| then \lstinline|split| when $\geq b-1$ keys
\end{enumerate}

\section{Hash Tables}

Hash tables map $n$ key-value pairs to $m$ buckets.
\begin{enumerate}[\roman*.]
  \item Load = $\alpha = \frac{n}{m} =$ expected items per bucket
  \item \lstinline|hashcode| computes hash, by default memory address
\end{enumerate}

Linear Chaining, colliding pairs are added to linked list:
\begin{enumerate}[\roman*.]
  \item \lstinline|get|: Time: $O(1+\alpha)$ (Expected), $O(n)$ (Worst)
  \item \lstinline|insert|: Time: $O(1)$
  \item \lstinline|delete|: Time: $O(1+\alpha)$ (Expected), $O(n)$ (Worst)
  \item Expected max chain length: $O(\log n)$ (Expected)
  \item Space: $O(m+n)$
  \item Assumption: Follows SUHA
\end{enumerate}

Linear Probing, probe following sequence of buckets:
\begin{enumerate}[\roman*.]
  \item \lstinline|get|: Time: $O(1)$ (Expected $\alpha < 1$; worse as $\alpha \rightarrow 1$)
  \item \lstinline|insert|: Time: $O(1)$ (Expected), $O(n)$ (Worst)
  \item \lstinline|delete|: Use tombstone instead of set \lstinline|null|. Time: $O(1)$ (Expected), $O(n)$ (Worst)
  \item \lstinline|resize|: Resize to keep operations amortised $O(1)$. When $n == m$, grow $m=2m$.\\ When $n < \frac{m}{4}$, shrink $m=\frac{m}{2}$.\\Time: $O(m_1+m_2+n)=O(n)$
  \item Space: $O(m)$
  \item Assumption: Violates UHA with expected cluster size $= \Omega(\log n)$ when $a = \frac{1}{4}$ 
  \item Clusters take advantage of caching.
  \item More sensitive to hash function quality.
\end{enumerate}

\begin{center}
\begin{tabular}{|c|c|c|}
\hline
\textbf{Growth} & \textbf{Resize} & \textbf{Insert $n$ Items} \\
\hline
Increment by 1        & $O(n)$           & $O(n^2)$ \\
Double                & $O(n)$           & $O(n)$ (Amortised $O(1)$) \\
Square                & $O(n^2)$         & $O(n)$ \\
\hline
\end{tabular}
\end{center}

\colbreak
\section{Heaps}
\subsection{Binary Heap (MaxHeap)}
{\centering
\incimg[0.4]{heap}
\par}
Binary heaps are complete binary trees with heap order.
\begin{enumerate}[\roman*.]
  \item Shape: Complete binary tree (all levels filled left-to-right)
  \item Heap: $priority_{parent} \geq priority_{child}$
  \item \lstinline|insert|: Insert as leaf and bubble up.\\Time: $O(\log n)$
  \item \lstinline|extractMax|/\lstinline|delete|: Remove (root), replace with last node and bubble down with larger child.\\Time: $O(\log n)$
  \item \lstinline|increaseKey|/\lstinline|decreaseKey|: Keep HashMap mapping id to index to access in $O(1)$. Then, update and bubble up/down with larger child.\\Time: $O(\log n)$
  \item \lstinline|contains|: Use HashMap. Time: $O(1)$
\end{enumerate}

Array implementation, Space: $O(n)$:
\begin{enumerate}[\roman*.]
  \item Size at index $0$
  \item Parent of node $i$ at $\floor{\frac{i}{2}}$
  \item Left child of node $i$ at $2i$
  \item Right child of node $i$ at $2i+1$
\end{enumerate}

Heapify, Time: $O(n)$:
\begin{enumerate}[\roman*.]
  \item For each level bottom-up, bubble down from right-to-left.
\end{enumerate}

HeapSort builds a max-heap in $O(n)$ and repeatedly extract max to end of sorted array.
\begin{enumerate}[\roman*.]
  \item Time: $O(n\log n)$
  \item In-place but unstable
\end{enumerate}

\colbreak
\section{Union-Find Disjoint Sets}

UFDS manages disjoint sets.
\begin{enumerate}[\roman*.]
  \item \lstinline|find|: Check if $x, y$ belong to same set
  \item \lstinline|union|: Merges set of $x, y$
\end{enumerate}

Variations:
\begin{enumerate}[\roman*.]
  \item Quick-Find keeps track of node's set componentId.\\
    \lstinline|find|: $O(1)$, \lstinline|union|: $O(n)$
  \item Quick-Union keeps set in tree, with arbitrary union.\\
    \lstinline|find|: $O(n)$, \lstinline|union|: $O(n)$
  \item Weighted-Union keeps set in tree, with tree union connecting smaller tree to root of larger tree.\\
    \lstinline|find|: $O(\log n)$, \lstinline|union|: $O(\log n)$
  \item Path-Compression on find reparents all nodes on path to root, to the root.\\
    \lstinline|find|: $O(\log n)$, \lstinline|union|: $O(\log n)$
  \item WC+PC for $m$ operations.\\
    \lstinline|find|: $a(m, n)$, \lstinline|union|: $a(m, n)$
\end{enumerate}

\section{Amortised Analysis}
Amortised analysis gives the average time per operation over a worst-case sequence of operations, even if some operations are expensive.
\begin{enumerate}[\roman*.]
  \item Guarantees avg. cost over $k$ operations $\leq k T(n)$
\end{enumerate}

Methods:
\begin{enumerate}[\roman*.]
  \item Aggregate: Total cost $= T \Rightarrow$ am. cost $= T/k$
  \item Accounting: Overcharge cheap ops to "pay" for expensive ops later
\end{enumerate}

Hash Table Resizing:
\begin{enumerate}[\roman*.]
  \item Resize cost $O(n)$, spread over $n$ ops $\Rightarrow O(1)$ am.
\end{enumerate}

\colbreak
\section{Graphs}
Graphs $G$ consist of vertices $V$ connected by edges $E$.
\begin{enumerate}[\roman*.]
  \item Degree (node): Number of edges incident on a node
  \item Degree (graph): Max. degree of any node
  \item Diameter: Max. shortest path between nodes 
  \item Clique: Complete graph
  \item Dense: $|E| = \Theta(V^2)$
  \item Adjacency Matrix: Fast query neighbour ($u,v$)\\Space: $O(V^2)$
  \item Adjacency List: Fast enumerate neighbours\\Space: $O(V+E)$
\end{enumerate}

Traversal, $O(V+E)$ (Adj. list) $O(V^2)$ (Adj. matrix):
\begin{enumerate}[\roman*.]
  \item BFS: Explore by level, add nodes to frontier queue.
  \item DFS: Explore deeply before backtrack, uses stack.
\end{enumerate}

\subsection{SSSP}

Conserved across adding constant to edge.

BFS/DFS handle trees. BFS handles unweighted graphs:
\begin{enumerate}[\roman*.]
  \item Relax edges in BFS/DFS order
  \item Time: $O(V+E)$
\end{enumerate}

Djikstra handles positive-weighted graphs 
\begin{enumerate}[\roman*.]
  \item Using PQ for closest node, $|V|$ \lstinline|insert|/\lstinline|extractMin| and $|E|$ \lstinline|relax|/\lstinline|decreaseKey| 
  \item Time: $O((V+E)\log V) = O(E\log V)$, or $O(E+V\log V)$ (Fib. Heap)
\end{enumerate}

Bellman-Ford handles negative-weights and detects negative cycles:
\begin{enumerate}[\roman*.]
  \item $|V|-1$ iterations of relaxing all $|E|$ edges, terminating early when estimates aren't changed
  \item Time: $O(VE)$
\end{enumerate}

DAG\_SSSP handles DAGs:
\begin{enumerate}[\roman*.]
  \item TopoSort, then relax edges in topological order 
  \item Time: $O(V+E)$
  \item For longest path, negate edges/ modify relax
\end{enumerate}

\colbreak
\subsection{Topological Order}

Topological sorting on DAGs:
\begin{enumerate}[\roman*.]
  \item Post-order DFS: Time: $O(V+E)$
  \item Kahn's Algorithm: Add nodes with no incoming edges, then remove their outgoing edges and repeat.\\
    Time: $O(V+E)$
\end{enumerate}

\subsection{Minimum Spanning Trees}
Minimum spanning trees contain all vertices with minimum total edge weight. Conserved across adding/multiplying constant to edge.
\begin{enumerate}[\roman*.]
  \item Cutting an MST produces two MSTs
  \item Cycle: Max. weight edge of cycle not in MST
  \item Cut: Min. weight edge across cut is in MST 
\end{enumerate}

Generic MST Algorithm:
\begin{enumerate}[\roman*.]
  \item Color max. edge in each cycle red, and min. edge in each cut blue; greedily apply this rule until no more edges uncolored with blue edges forming MST 
\end{enumerate}

Prim's Algorithm:
\begin{enumerate}[\roman*.]
  \item Using PQ for lightest edge, grow MST by adding lighted edge to unvisited nodes 
  \item Time: $O((V+E)\log V) = O(E\log V)$
\end{enumerate}

Kruskal's Algorithm:
\begin{enumerate}[\roman*.]
  \item Using UFDS to keep track of cycles, sort edges by weight, adding if endpoints are not connected. 
  \item Time: $O(E\log E) = O(E\log V)$
\end{enumerate}

Boruvka's Algorithm:
\begin{enumerate}[\roman*.]
  \item Each component adds min. outgoing edge. Repeat $\log V$ times
  \item Time: $O(E \log V)$
\end{enumerate}

Directed Graph (with root):
\begin{enumerate}
  \item For each node except root, add min. incoming edge
  \item Time: $O(E)$
\end{enumerate}

\colbreak
\section{Dynamic Programming}

DP solves problems with optimal substructure and overlapping subproblems. 
\begin{enumerate}[\roman*.]
  \item Top-down (Memoization): Recursive with cache
  \item Bottom-up (Tabulation): Iteratively fill table
\end{enumerate}

\subsection*{Problems}

Fibonacci:
\begin{enumerate}[\roman*.]
  \item $F(n) = F(n-1) + F(n-2)$
  \item Time: $O(n)$ with memoization or bottom-up
\end{enumerate}

Longest Increasing Subsequence (LIS):
\begin{enumerate}[\roman*.]
  \item $dp[i] = \max(dp[j] + 1)$ for $j < i$ and $A[j] < A[i]$
  \item Time: $O(n^2)$ or $O(n \log n)$ with binary search
\end{enumerate}

Knapsack 0/1:
\begin{enumerate}[\roman*.]
  \item $dp[i][w] = \max(dp[i-1][w], dp[i-1][w-w_i] + v_i)$
  \item Time: $O(nW)$
\end{enumerate}

Matrix Chain Multiplication:
\begin{enumerate}[\roman*.]
  \item $dp[i][j] = \min_{i \leq k < j}(dp[i][k] + dp[k+1][j] + cost)$
  \item Time: $O(n^3)$
\end{enumerate}

Floyd-Warshall (All-Pairs Shortest Path):
\begin{enumerate}[\roman*.]
  \item $dist[i][j] = \min(dist[i][j], dist[i][k] + dist[k][j])$
  \item Time: $O(V^3)$
\end{enumerate}

Prize Collecting (on Graph):
\begin{enumerate}[\roman*.]
  \item $dp[v][k] = \max(dp[u][k-1] + prize(u,v))$
  \item Time: $O(kE)$
\end{enumerate}

Vertex Cover on Tree:
\begin{enumerate}[\roman*.]
  \item $dp[v][0] =$ min cover if $v$ not included\\
        $dp[v][1] =$ min cover if $v$ included
  \item Recurrence:
    \begin{itemize}
      \item $dp[v][0] = \sum dp[c][1]$ for all children $c$
      \item $dp[v][1] = 1 + \sum \min(dp[c][0], dp[c][1])$
    \end{itemize}
  \item Time: $O(V)$
\end{enumerate}

%%%%%%%%%%%%%%%%%%%%%%%%%%%%%%%%%%%%%%%%%%%%%%%%%%%%%%
%                       End                          %
%%%%%%%%%%%%%%%%%%%%%%%%%%%%%%%%%%%%%%%%%%%%%%%%%%%%%%
\end{multicols*}
\end{document}

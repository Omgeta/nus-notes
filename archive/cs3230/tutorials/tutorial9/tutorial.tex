\documentclass[12pt, a4paper]{article}

\usepackage[a4paper, margin=1in]{geometry}

\usepackage[utf8]{inputenc}
\usepackage[mathscr]{euscript}
\let\euscr\mathscr \let\mathscr\relax
\usepackage[scr]{rsfso}
\usepackage{amssymb,amsmath,amsthm,amsfonts}
\usepackage[shortlabels]{enumitem}
\usepackage{multicol,multirow}
\usepackage{lipsum}
\usepackage{balance}
\usepackage{calc}
\usepackage[colorlinks=true,citecolor=blue,linkcolor=blue]{hyperref}
\usepackage{import}
\usepackage{xifthen}
\usepackage{pdfpages}
\usepackage{transparent}
\usepackage{tabularx}

\newcommand{\incfig}[2][1.0]{
    \def\svgwidth{#1\columnwidth}
    \import{./figures/}{#2.pdf_tex}
}
\newcommand{\incimg}[2][1.0]{
  \includegraphics[width=#1\columnwidth]{./figures/#2}
}


\input{letterfonts}

\newcommand{\mytitle}{CS3230 Tutorial 9}
\newcommand{\myauthor}{github/omgeta}
\newcommand{\mydate}{AY 25/26 Sem 1}

\begin{document}
\raggedright
\footnotesize
\begin{center}
{\normalsize{\textbf{\mytitle}}} \\
{\footnotesize{\mydate\hspace{2pt}\textemdash\hspace{2pt}\myauthor}}
\end{center}
\setlist{topsep=-1em, itemsep=-1em, parsep=2em}
%%%%%%%%%%%%%%%%%%%%%%%%%%%%%%%%%%%%%%%%%%%%%%%%%%%%%%
%                      Begin                         %
%%%%%%%%%%%%%%%%%%%%%%%%%%%%%%%%%%%%%%%%%%%%%%%%%%%%%%
\begin{enumerate}[Q\arabic*).]
  \item 
    \begin{enumerate}[(\alph*.)]
      \item True; reduce to solving for optimum and checking if $\leq k$
      \item True; reduce to solving decision problem $O(n)$ times
      \item True; follows from contraposition in (a) of polynomial-reduction
      \item True; follows from contraposition in (b) of polynomial-reduction
    \end{enumerate}

  \item (c); this reduction ignores the preservation of set $S$ 

  \item (a); trivial\\
    (b); if a subset sums to $\frac{S}{2}$, just choosing those items for the knapsack satisfies $W \leq \frac{S}{2} \leq V$\\
    (c); since every item has same weight as value, and we solve for $W \leq \frac{S}{2} \leq V$, we force $W = V = \frac{S}{2}$ so the knapsack contains exactly half the sum $\rightarrow$ Partition YES

  \item Algorithm: transform unweighted graph $G$ into complete weighted $G'$ with $w(u, v) = \begin{cases}1,&\text{if $(u, v)\in G$}\\2, &\text{otherwise}\end{cases}$

    Transformation is $O(n^2)$ by building edges between each of the $n$ edges, and comparing with original graph in $O(1)$ each.

    Suppose HC-Yes, then the same tour in $G'$ costs $|V|$ due to each edge only costing $1$ $\rightarrow$ TSP-YES for $|V|$

    Suppose TSP-Yes, then the tour must have all edges cost 1 to keep total cost $\leq |V| \rightarrow$ HC-YES
\end{enumerate}
%%%%%%%%%%%%%%%%%%%%%%%%%%%%%%%%%%%%%%%%%%%%%%%%%%%%%%
%                       End                          %
%%%%%%%%%%%%%%%%%%%%%%%%%%%%%%%%%%%%%%%%%%%%%%%%%%%%%%

\end{document}

\documentclass[12pt, a4paper]{article}

\usepackage[a4paper, margin=1in]{geometry}

\usepackage[utf8]{inputenc}
\usepackage[mathscr]{euscript}
\let\euscr\mathscr \let\mathscr\relax
\usepackage[scr]{rsfso}
\usepackage{amssymb,amsmath,amsthm,amsfonts}
\usepackage[shortlabels]{enumitem}
\usepackage{multicol,multirow}
\usepackage{lipsum}
\usepackage{balance}
\usepackage{calc}
\usepackage[colorlinks=true,citecolor=blue,linkcolor=blue]{hyperref}
\usepackage{import}
\usepackage{xifthen}
\usepackage{pdfpages}
\usepackage{transparent}
\usepackage{tabularx}

\newcommand{\incfig}[2][1.0]{
    \def\svgwidth{#1\columnwidth}
    \import{./figures/}{#2.pdf_tex}
}
\newcommand{\incimg}[2][1.0]{
  \includegraphics[width=#1\columnwidth]{./figures/#2}
}


\input{letterfonts}

\newcommand{\mytitle}{GEA1000 Tutorial 4}
\newcommand{\myauthor}{github/omgeta}
\newcommand{\mydate}{AY 24/25 Sem 2}

\begin{document}
\raggedright
\footnotesize
\begin{center}
{\normalsize{\textbf{\mytitle}}} \\
{\footnotesize{\mydate\hspace{2pt}\textemdash\hspace{2pt}\myauthor}}
\end{center}
\setlist{topsep=-1em, itemsep=-1em, parsep=2em}
%%%%%%%%%%%%%%%%%%%%%%%%%%%%%%%%%%%%%%%%%%%%%%%%%%%%%%
%                      Begin                         %
%%%%%%%%%%%%%%%%%%%%%%%%%%%%%%%%%%%%%%%%%%%%%%%%%%%%%%
\begin{enumerate}[Q\arabic*.]
  \item 
    \begin{enumerate}[(\alph*.)]
      \item
        \begin{enumerate}[(\roman*.)]
          \item T
          \item H
          \item H --- 9. \{H, T, H, H, H\}\\
          \item H --- 26. \{T, T, H, H, T\}
        \end{enumerate}

      \item $P(T|HHHH) = 0.5$

      \item Disagree; the probability of the result of each toss is independent of previous events.

      \item 
        \begin{enumerate}[(\roman*.)]
          \item $\frac{5}{32}$

          \item No, because $P(X \geq 4) = \frac{6}{32} = 18.75\% > 5\%$
        \end{enumerate}
    \end{enumerate}

  \item 
    \begin{enumerate}[(\alph*.)]
      \item $P(\text{infected} | \text{positive}) = \frac{2400}{2470} \approx 0.9717$
        \begin{table}[h!]
        \centering
        \begin{tabular}{|c|c|c|}
        \hline
         & \textbf{Infected (3000)} & \textbf{Not Infected (7000)} \\
        \hline
        \textbf{Positive (2470)} & 2400 (True Positives) & 70 (False Positives) \\
        \hline
        \textbf{Negative (7530)} & 600 (False Negatives) & 6930 (True Negatives) \\
        \hline
        \end{tabular}
        \end{table}

      \item No; base rate fallacy assumes the same prevalence as in Country X, which may not hold in Country Y.
    \end{enumerate}

  \item 
    \begin{enumerate}[(\alph*.)]
      \item Sample proportion, $p^* = 0.525$

      \item 
        \begin{enumerate}[(\roman*.)]
          \item Margin of error, $\displaystyle e = 1.96 \cdot \sqrt{\frac{0.495(1-0.495)}{200}} \approx 0.069$

          \item Yes; Confidence interval $= 0.495 \pm 0.069$ which includes $0.506$
        \end{enumerate}

      \item 
        \begin{enumerate}[(\roman*.)]
          \item Larger; higher sample size reduces error margin

          \item $n \geq \frac{1.96^2 \cdot 0.525(1-0.525)}{0.03^2} = 1064.\cdots \implies n = 1065$
        \end{enumerate}

      \item 
        \begin{enumerate}[(\roman*.)]
          \item Sample size $n$ is directly proportional to $p(1-p)$

          \item $n \geq \frac{1.96^2\cdot 0.5\cdot 0.5}{0.03^2} \approx 1067.11 \implies n = 1068$
        \end{enumerate}

      \item Null hypothesis, $H_0$: $\mu = 28$\\
        Alternative hypothesis, $H_1$: $\mu < 28$\\
        Level of significance, $\alpha = 0.05$\\
        Test statistic $\displaystyle = \frac{27.038 - 28}{5.206 / \sqrt{1000}} = -5.843$\\
        Since $p < 0.001 < \alpha = 0.05$, therefore we reject the null hypothesis to conclude that mean age is less than 28.
    \end{enumerate}

\end{enumerate}
%%%%%%%%%%%%%%%%%%%%%%%%%%%%%%%%%%%%%%%%%%%%%%%%%%%%%%
%                       End                          %
%%%%%%%%%%%%%%%%%%%%%%%%%%%%%%%%%%%%%%%%%%%%%%%%%%%%%%

\end{document}

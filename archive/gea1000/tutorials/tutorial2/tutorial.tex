\documentclass[12pt, a4paper]{article}

\usepackage[a4paper, margin=1in]{geometry}

\usepackage[utf8]{inputenc}
\usepackage[mathscr]{euscript}
\let\euscr\mathscr \let\mathscr\relax
\usepackage[scr]{rsfso}
\usepackage{amssymb,amsmath,amsthm,amsfonts}
\usepackage[shortlabels]{enumitem}
\usepackage{multicol,multirow}
\usepackage{lipsum}
\usepackage{balance}
\usepackage{calc}
\usepackage[colorlinks=true,citecolor=blue,linkcolor=blue]{hyperref}
\usepackage{import}
\usepackage{xifthen}
\usepackage{pdfpages}
\usepackage{transparent}
\usepackage{tabularx}

\newcommand{\incfig}[2][1.0]{
    \def\svgwidth{#1\columnwidth}
    \import{./figures/}{#2.pdf_tex}
}
\newcommand{\incimg}[2][1.0]{
  \includegraphics[width=#1\columnwidth]{./figures/#2}
}


\input{letterfonts}

\newcommand{\mytitle}{GEA1000 Tutorial 2}
\newcommand{\myauthor}{github/omgeta}
\newcommand{\mydate}{AY 24/25 Sem 2}

\begin{document}
\raggedright
\footnotesize
\begin{center}
{\normalsize{\textbf{\mytitle}}} \\
{\footnotesize{\mydate\hspace{2pt}\textemdash\hspace{2pt}\myauthor}}
\end{center}
\setlist{topsep=-1em, itemsep=-1em, parsep=2em}
%%%%%%%%%%%%%%%%%%%%%%%%%%%%%%%%%%%%%%%%%%%%%%%%%%%%%%
%                      Begin                         %
%%%%%%%%%%%%%%%%%%%%%%%%%%%%%%%%%%%%%%%%%%%%%%%%%%%%%%
\begin{enumerate}[Q\arabic*.]
  \item 
    \begin{enumerate}[(\alph*.)]
      \item Based on the data, before 2018 is positively associated with recycling of Paper/Cardboard
        \begin{table}[h]
          \centering
          \begin{tabular}{|l|c|c|}
              \hline
              & \textbf{Before 2018} & \textbf{2018 onwards} \\
              \hline
              \textbf{Recycled (‘000 tonnes)} & 1176.1 & 1904 \\
              \hline
              \textbf{Disposed (‘000 tonnes)} & 1152 & 2439\\
              \hline
              \textbf{Total (‘000 tonnes)} & 2328.1 & 4343\\
              \hline
          \end{tabular}
      \end{table}

    \item No, the total amount is greater because more data points were counted from multiple years instead of from a single year before 2018. However, the rate of recycling is still lower.

    \item Yes; From (a) we know rate(recycled | before 2018) > rate(recycled | 2018 onwards). By symmetry rule, rate(before 2018 | recycled) > rate(before 2018 | disposed). Then by basic rule on rates, we have rate(before 2018 | recycled) > rate(before 2018) = 34.9\% which is Chuan's claim 

    \item Straits Time article says that domestic overall recycling rate is between 19-22\% but Tammy's data points for different sources of waste are all below 19\% making it impossible to reach an overall rate of 19-22\%. This is assuming there are no other additional sources of waste not shown in Tammy's data.

    \item Positive association between high average monthly usage of Paper/Cardboard and healthy recycling habits.

    \item Trend absorved for majority of years is high usage is negatively associated with healthy habits, which is reversed for the overall rate. Therefore, year of study is a confounder. 

    \item For both foreigners and non-foreigners, high usage has equal or higher healthy rates than low usage, so Simpson's Paradox is not observed and therefore, foreigner status is not a confounder.

    \item Selection bias exists from the sampling at only 1 location and during a limited time period. Hence, we cannot generalize our findings. Sampling frame does not cover all NUS students.
    \end{enumerate}
\end{enumerate}
%%%%%%%%%%%%%%%%%%%%%%%%%%%%%%%%%%%%%%%%%%%%%%%%%%%%%%
%                       End                          %
%%%%%%%%%%%%%%%%%%%%%%%%%%%%%%%%%%%%%%%%%%%%%%%%%%%%%%

\end{document}

\documentclass[12pt, a4paper]{article}

\usepackage[a4paper, margin=1in]{geometry}

\usepackage[utf8]{inputenc}
\usepackage[mathscr]{euscript}
\let\euscr\mathscr \let\mathscr\relax
\usepackage[scr]{rsfso}
\usepackage{amssymb,amsmath,amsthm,amsfonts}
\usepackage[shortlabels]{enumitem}
\usepackage{multicol,multirow}
\usepackage{lipsum}
\usepackage{balance}
\usepackage{calc}
\usepackage[colorlinks=true,citecolor=blue,linkcolor=blue]{hyperref}
\usepackage{import}
\usepackage{xifthen}
\usepackage{pdfpages}
\usepackage{transparent}
\usepackage{tabularx}

\newcommand{\incfig}[2][1.0]{
    \def\svgwidth{#1\columnwidth}
    \import{./figures/}{#2.pdf_tex}
}
\newcommand{\incimg}[2][1.0]{
  \includegraphics[width=#1\columnwidth]{./figures/#2}
}


\input{letterfonts}

\newcommand{\mytitle}{GEA1000 Tutorial 3}
\newcommand{\myauthor}{github/omgeta}
\newcommand{\mydate}{AY 24/25 Sem 2}

\begin{document}
\raggedright
\footnotesize
\begin{center}
{\normalsize{\textbf{\mytitle}}} \\
{\footnotesize{\mydate\hspace{2pt}\textemdash\hspace{2pt}\myauthor}}
\end{center}
\setlist{topsep=-1em, itemsep=-1em, parsep=2em}
%%%%%%%%%%%%%%%%%%%%%%%%%%%%%%%%%%%%%%%%%%%%%%%%%%%%%%
%                      Begin                         %
%%%%%%%%%%%%%%%%%%%%%%%%%%%%%%%%%%%%%%%%%%%%%%%%%%%%%%
\begin{enumerate}[Q\arabic*.]
  \item 
    \begin{enumerate}[(\alph*.)]
      \item There are $63$ outliers.
      \begin{center}
        \begin{table}[h]
          \centering
          \begin{tabular}{|l|c|}
              \hline
              \textbf{Statistic} & \textbf{Value} \\ 
              \hline
              Mean      & 9.240 \\ 
              \hline
              Median    & 6.005 \\ 
              \hline
              Minimum   & 0 \\ 
              \hline
              Maximum   & 32.748 \\ 
              \hline
              SD & 7.533 \\ 
              \hline
              Q1 & 4.372 \\ 
              \hline
              Q3 & 14.497 \\ 
              \hline
              IQR & 10.125 \\ 
              \hline
          \end{tabular}
          \label{tab:summary}
      \end{table}
        \incimg[0.7]{salinity}
      \end{center}

    \pagebreak
    \item Summer has the largest IQR for temperature, Winter has the smallest IQR for temperature.
      \begin{center}
      \incimg[0.65]{temperature}
      \end{center}

    \item Winter has the largest IQR for salinity, Summer has the smallest IQR for salinity.
      \begin{center}
      \incimg[0.65]{salinity_season}
      \end{center}

    \item Correlation coefficient, $r = 0.01$. There is very small positive correlation.
      \begin{center}
      \incimg[0.6]{salinity_temperature}
      \end{center}

    \item Solution:\\
      \begin{center}
      \incimg[0.5]{salinity_temperature_eco}
      \end{center}

    \item Ecological fallacy

    \item Salinity and temperature may have different relationships dependent on the season.
    \end{enumerate}

  \pagebreak
  \item 
    \begin{enumerate}[(\alph*.)]
      \item There are some outliers which have 0 or negative age. 
        \begin{center}
            \begin{tabular}{|l|c|}
                \hline
                \textbf{Statistic} & \textbf{Value} \\ 
                \hline
                Mean      & 46.705 \\ 
                \hline
                Median    & 47.615 \\ 
                \hline
                Minimum   & -12.620 \\ 
                \hline
                Maximum   & 76.230 \\ 
                \hline
                SD & 11.844 \\ 
                \hline
                Q1 & 40.450 \\ 
                \hline
                Q3 & 54.943 \\ 
                \hline
                IQR & 14.492 \\ 
                \hline
            \end{tabular}
        \end{center}
      \begin{minipage}[t]{.4\textwidth}
        \incimg{median_age_box}
      \end{minipage}%
      \begin{minipage}[t]{.4\textwidth}
        \incimg{median_age_histogram}
      \end{minipage}%

    \item There is a position correlation between Population and Median\_Age. Using the natural logarithm, we can normalize data which follows a natural exponential scaling to more accurate fit the correlation to Median\_Age.
      \begin{minipage}[t]{.4\textwidth}
        \incimg{pop_vs_age}
      \end{minipage}%
      \begin{minipage}[t]{.4\textwidth}
        \incimg{lnpop_vs_age}
      \end{minipage}%

    \item $\ln \text{Population} = 0.045 \times \text{Median\_Age } + 12.96$

    \item Slope $0.045$ gives the direction for greatest change between Median\_Age and $\ln$ Population. Intercept $12.96$ gives us the baseline for $\ln$ Population at Median\_Age 0 which is not meaningful.

    \item Predicted $\ln$ Population $= 14.524$. Therefore, predicted population is $e^{14.524} = 2030921$
    \end{enumerate}

\end{enumerate}
%%%%%%%%%%%%%%%%%%%%%%%%%%%%%%%%%%%%%%%%%%%%%%%%%%%%%%
%                       End                          %
%%%%%%%%%%%%%%%%%%%%%%%%%%%%%%%%%%%%%%%%%%%%%%%%%%%%%%

\end{document}

\documentclass[12pt, a4paper]{article}

\usepackage[a4paper, margin=1in]{geometry}

\usepackage[utf8]{inputenc}
\usepackage[mathscr]{euscript}
\let\euscr\mathscr \let\mathscr\relax
\usepackage[scr]{rsfso}
\usepackage{amssymb,amsmath,amsthm,amsfonts}
\usepackage[shortlabels]{enumitem}
\usepackage{multicol,multirow}
\usepackage{lipsum}
\usepackage{balance}
\usepackage{calc}
\usepackage[colorlinks=true,citecolor=blue,linkcolor=blue]{hyperref}
\usepackage{import}
\usepackage{xifthen}
\usepackage{pdfpages}
\usepackage{transparent}
\usepackage{tabularx}

\newcommand{\incfig}[2][1.0]{
    \def\svgwidth{#1\columnwidth}
    \import{./figures/}{#2.pdf_tex}
}
\newcommand{\incimg}[2][1.0]{
  \includegraphics[width=#1\columnwidth]{./figures/#2}
}


\input{letterfonts}

\newcommand{\mytitle}{MA1521 Homework 1}
\newcommand{\myauthor}{github/omgeta}
\newcommand{\mydate}{AY 24/25 Sem 1}

\begin{document}
\raggedright
\footnotesize
\begin{center}
{\normalsize{\textbf{\mytitle}}} \\
{\footnotesize{\mydate\hspace{2pt}\textemdash\hspace{2pt}\myauthor}}
\end{center}
\setlist{topsep=-1em, itemsep=-1em, parsep=2em}

%%%%%%%%%%%%%%%%%%%%%%%%%%%%%%%%%%%%%%%%%%%%%%%%%%%%%%
%                      Begin                         %
%%%%%%%%%%%%%%%%%%%%%%%%%%%%%%%%%%%%%%%%%%%%%%%%%%%%%%
\begin{enumerate}[Q\arabic*.]
  \item For each of the following functions, find all real values of $x$ for which it is defined, i.e.
the maximal domain of each function:
  \begin{enumerate}[(\alph*)]
    \item $\displaystyle f(x) = \frac{81 - x^2}{(4+x^2)(27-x^3)(16-x^4)}$

      For $f(x)$ to be defined, denominator $(4+x^2)(27-x^3)(16-x^4) \neq 0$.
    \begin{align*}
      4 + x^2 = 0 &\implies \text{(no real solutions)} \\
      27 - x^3 = 0 &\implies x = 3 \\
      16 - x^4 = 0 &\implies x = \pm 2
    \end{align*}
    Therefore, domain of \( f(x) \) is:
    \[
    \RR \setminus \{-2, 2, 3\} \qed
    \]
  \item $g(x) = \sqrt{2 - \ln(x+1)}$

    For $g(x)$ to be defined, argument $(2 - \ln(x+1))$ must be non-negative
    \begin{align*}
      2 - \ln(x+1) &\geq 0 \\
      2 &\geq \ln(x+1) \\
      e^2 &\geq x + 1 \\
      e^2 - 1 &\geq x
    \end{align*}
    For $\ln(x+1)$ to be defined, $x + 1 > 0 \implies x > -1 $. 
    
    Therefore, domain of $g(x)$ is:
    \[
      \{x \in \RR : -1 < x \leq e^2 - 1\} \qed
    \]

  \item $\displaystyle h(x) = \frac{\ln(\sqrt{16-4x}+1)}{\sqrt{\ln x}-1}$

      For $\ln(\sqrt{16-4x}+1)$ to be defined
        \begin{align*}
          \sqrt{16-4x} + 1 &> 0 \\
          16 - 4x &\geq 0 \\
          x &\leq 4
        \end{align*}
      For $\sqrt{\ln x}$ to be defined, $\ln x \geq 0 \implies x \geq 1$

      For $h(x)$ to be defined, denominator must be non-zero 
        \begin{align*}
          \sqrt{\ln x} - 1 &\neq 0 \\
          \ln x &\neq 1 \\
          x &\neq e
        \end{align*}
    Therefore, domain of \( h(x) \) is:
    \[
    1 \leq x \leq 4 \text{ and } x \neq e \qed
    \] 
  \end{enumerate}

\item Let $f(x)$ be defined on $(-\infty, \infty)$ such that $f(x) = \begin{cases} 
      4 & x \leq -2 \\
      x^2-1 & -2 < x\leq-1 \\
      0 & -1< x \leq 1 \\
      \displaystyle\frac{1}{x-1} & x > 1
   \end{cases}
$ \\ Find all $x$ such that $f$ is not continuous at $x$

\[
x = -2, 1 \qed
\]

\item Let $f(x)$ be defined on $[0, 8]$ such that $f(x) = \begin{cases} 
    p^\frac{1}{3}\sqrt{x} & 0 \leq x < 4 \\
      7 & x=4 \\
      q(x-2)^2 + 5 & 4 < x \leq 6 \\
      \displaystyle\frac{2r}{x-5} & 6 < x \leq 8
   \end{cases}$ \\
   It is given that $f$ is continuous at $x = 4$ and $\lim_{x\to6} f(x)$  exists. Find the values of $p, q, r$.
   
Since $f$ is continuous at $x = 4$,
\begin{align*}
  p^\frac{1}{3}\sqrt{4} &= 7 \\
  p^\frac{1}{3} &= \frac{7}{2} \\
  p &= \frac{343}{8} \qed \\
    &\\
  q(4 - 2)^2 + 5 &= 7 \\
  4q &= 2 \\ 
  q &= \frac{1}{2} \qed
\end{align*}
Since $\lim_{x\to6} f(x)$ exists, when $x = 6$,
\begin{align*}
  \frac{1}{2}(6-2)^2 + 5 &= \frac{2r}{6-5} \\
  8 + 5 &= 2r \\
  r &= \frac{13}{2} \qed
\end{align*}

\item Evaluate each of the following limits if it exists:
  \begin{enumerate}[(\alph*)]
    \item $\displaystyle \lim_{x\to2}\frac{4-x^2}{x^2-3x+2}$
      \begin{align*}
        \lim_{x\to2}\frac{4-x^2}{x^2-3x+2} &=  \lim_{x\to2}\frac{-(x-2)(x+2)}{(x-2)(x-1)} \\
                                           &=  \lim_{x\to2}\frac{-(x+2)}{x-1} \\
                                           &= \frac{-(2+2)}{2-1} \\
                                           &= -4 \qed
      \end{align*}
    \item $\displaystyle \lim_{x\to-2}\frac{4-x^2}{\sqrt{x^2-x-2}-\sqrt{2-x}}$
      \begin{align*}
        \lim_{x\to-2}\frac{4-x^2}{\sqrt{x^2-x-2}-\sqrt{2-x}} &= \lim_{x\to-2}\frac{(4-x^2)(\sqrt{x^2-x-2}+\sqrt{2-x})}{(\sqrt{x^2-x-2})^2-(\sqrt{2-x})^2} \\
                                                             &= \lim_{x\to-2}\frac{(4-x^2)(\sqrt{x^2-x-2}+\sqrt{2-x})}{(x^2-x-2)-(2-x)} \\
                                                             &= \lim_{x\to-2}\frac{-(x^2-4)(\sqrt{x^2-x-2}+\sqrt{2-x})}{x^2-4} \\
                                                             &= \lim_{x\to-2}-(\sqrt{x^2-x-2}+\sqrt{2-x}) \\
                                                             &= -4 \qed
      \end{align*}
    \item $\displaystyle \lim_{x\to2}\frac{x^3-8}{(x-2)^2}$
      \begin{align*}
        \lim_{x\to2}\frac{x^3-8}{(x-2)^2} &= \lim_{x\to2}\frac{(x-2)(x^2+2x+4)}{(x-2)^2} \\
                                          &= \lim_{x\to2}\frac{x^2+2x+4}{x-2} \\
                                          &= \frac{12}{0}
      \end{align*}
       Limit is $\pm\infty$ depending on LHS or RHS limit, therefore limit does not exist \qed
  \end{enumerate}

  \item Evaluate the following limits:
    \begin{enumerate}[(\alph*)]
      \item $\displaystyle \lim_{x\to\infty}\sqrt{\frac{9x^{10}+3x-1}{(x^2+3x-5)^3(2x+5)^4}}$
        \begin{align*}
        \lim_{x\to\infty}\sqrt{\frac{9x^{10}+3x-1}{(x^2+3x-5)^3(2x+5)^4}} &= \lim_{x\to\infty}\sqrt{\frac{9x^{10}+\ldots}{16x^{10}+\ldots}}\\
                                                                          &= \sqrt{\frac{9}{16}} \\
                                                                          &= \frac{3}{4} \qed
        \end{align*}
      \item $\displaystyle \lim_{x\to-\infty}\frac{1}{x}\sqrt{\frac{9x^{10}+3x-1}{(x^2+3x-5)^3(2x+5)^2}}$
        \begin{align*}
          \lim_{x\to-\infty}\frac{1}{x}\sqrt{\frac{9x^{10}+3x-1}{(x^2+3x-5)^3(2x+5)^2}} 
                      &= \lim_{x\to-\infty}-\frac{1}{\sqrt{x^2}}\sqrt{\frac{9x^{10}+\ldots}{4x^8+\ldots}} \\
                                                                                        &= \lim_{x\to-\infty}-\sqrt{\frac{9x^{10}+\ldots}{4x^{10}+\ldots}} \\
                                                                 &= -\sqrt{\frac{9}{4}} \\
                                                                 &= -\frac{3}{2} \qed
        \end{align*}
      \item $\displaystyle \lim_{x\to-\infty}\frac{\sqrt{9x^{10}+3x-1}}{(1+2x)^2(x^2+x-1)}$
        \begin{align*}
        \lim_{x\to-\infty}\frac{\sqrt{9x^{10}+3x-1}}{(1+2x)^2(x^2+x-1)} &= \lim_{x\to-\infty}\sqrt{\frac{9x^{10}+\ldots}{16x^8+\ldots}} \\
                                                                          &= \infty \qed
        \end{align*}
    \end{enumerate}
\end{enumerate}
%%%%%%%%%%%%%%%%%%%%%%%%%%%%%%%%%%%%%%%%%%%%%%%%%%%%%%
%                       End                          %
%%%%%%%%%%%%%%%%%%%%%%%%%%%%%%%%%%%%%%%%%%%%%%%%%%%%%%

\end{document}

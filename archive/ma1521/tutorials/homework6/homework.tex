\documentclass[12pt, a4paper]{article}

\usepackage[a4paper, margin=1in]{geometry}

\usepackage[utf8]{inputenc}
\usepackage[mathscr]{euscript}
\let\euscr\mathscr \let\mathscr\relax
\usepackage[scr]{rsfso}
\usepackage{amssymb,amsmath,amsthm,amsfonts}
\usepackage[shortlabels]{enumitem}
\usepackage{multicol,multirow}
\usepackage{lipsum}
\usepackage{balance}
\usepackage{calc}
\usepackage[colorlinks=true,citecolor=blue,linkcolor=blue]{hyperref}
\usepackage{import}
\usepackage{xifthen}
\usepackage{pdfpages}
\usepackage{transparent}
\usepackage{listings}

\newcommand{\incfig}[2][1.0]{
    \def\svgwidth{#1\columnwidth}
    \import{./figures/}{#2.pdf_tex}
}

\newlist{enumproof}{enumerate}{4}
\setlist[enumproof,1]{label=\arabic*., parsep=1em}
\setlist[enumproof,2]{label=\arabic{enumproofi}.\arabic*., parsep=1em}
\setlist[enumproof,3]{label=\arabic{enumproofi}.\arabic{enumproofii}.\arabic*., parsep=1em}
\setlist[enumproof,4]{label=\arabic{enumproofi}.\arabic{enumproofii}.\arabic{enumproofiii}.\arabic*., parsep=1em}

\renewcommand{\qedsymbol}{\ensuremath{\blacksquare}}

\lstdefinestyle{mystyle}{
  language=C, % Set the language to C
  commentstyle=\color{codegray}, % Color for comments
  keywordstyle=\color{orange}, % Color for basic keywords
  stringstyle=\color{mauve}, % Color for strings
  basicstyle={\ttfamily\footnotesize}, % Basic font style
  breakatwhitespace=false,         
  breaklines=true,                 
  captionpos=b,                    
  keepspaces=true,                 
  numbers=none,                    
  tabsize=2,
  morekeywords=[2]{\#include, \#define, \#ifdef, \#ifndef, \#endif, \#pragma, \#else, \#elif}, % Preprocessor directives
  keywordstyle=[2]\color{codegreen}, % Style for preprocessor directives
  morekeywords=[3]{int, char, float, double, void, struct, union, enum, const, volatile, static, extern, register, inline, restrict, _Bool, _Complex, _Imaginary, size_t, ssize_t, FILE}, % C standard types and common identifiers
  keywordstyle=[3]\color{identblue}, % Style for types and common identifiers
  morekeywords=[4]{printf, scanf, fopen, fclose, malloc, free, calloc, realloc, perror, strtok, strncpy, strcpy, strcmp, strlen}, % Standard library functions
  keywordstyle=[4]\color{cyan}, % Style for library functions
}

% Things Lie
\newcommand{\kb}{\mathfrak b}
\newcommand{\kg}{\mathfrak g}
\newcommand{\kh}{\mathfrak h}
\newcommand{\kn}{\mathfrak n}
\newcommand{\ku}{\mathfrak u}
\newcommand{\kz}{\mathfrak z}
\DeclareMathOperator{\Ext}{Ext} % Ext functor
\DeclareMathOperator{\Tor}{Tor} % Tor functor
\newcommand{\gl}{\opname{\mathfrak{gl}}} % frak gl group
\renewcommand{\sl}{\opname{\mathfrak{sl}}} % frak sl group chktex 6

% More script letters etc.
\newcommand{\SA}{\mathcal A}
\newcommand{\SB}{\mathcal B}
\newcommand{\SC}{\mathcal C}
\newcommand{\SF}{\mathcal F}
\newcommand{\SG}{\mathcal G}
\newcommand{\SH}{\mathcal H}
\newcommand{\OO}{\mathcal O}

\newcommand{\SCA}{\mathscr A}
\newcommand{\SCB}{\mathscr B}
\newcommand{\SCC}{\mathscr C}
\newcommand{\SCD}{\mathscr D}
\newcommand{\SCE}{\mathscr E}
\newcommand{\SCF}{\mathscr F}
\newcommand{\SCG}{\mathscr G}
\newcommand{\SCH}{\mathscr H}

% Mathfrak primes
\newcommand{\km}{\mathfrak m}
\newcommand{\kp}{\mathfrak p}
\newcommand{\kq}{\mathfrak q}

% number sets
\newcommand{\RR}[1][]{\ensuremath{\ifstrempty{#1}{\mathbb{R}}{\mathbb{R}^{#1}}}}
\newcommand{\NN}[1][]{\ensuremath{\ifstrempty{#1}{\mathbb{N}}{\mathbb{N}^{#1}}}}
\newcommand{\ZZ}[1][]{\ensuremath{\ifstrempty{#1}{\mathbb{Z}}{\mathbb{Z}^{#1}}}}
\newcommand{\QQ}[1][]{\ensuremath{\ifstrempty{#1}{\mathbb{Q}}{\mathbb{Q}^{#1}}}}
\newcommand{\CC}[1][]{\ensuremath{\ifstrempty{#1}{\mathbb{C}}{\mathbb{C}^{#1}}}}
\newcommand{\PP}[1][]{\ensuremath{\ifstrempty{#1}{\mathbb{P}}{\mathbb{P}^{#1}}}}
\newcommand{\HH}[1][]{\ensuremath{\ifstrempty{#1}{\mathbb{H}}{\mathbb{H}^{#1}}}}
\newcommand{\FF}[1][]{\ensuremath{\ifstrempty{#1}{\mathbb{F}}{\mathbb{F}^{#1}}}}
% expected value
\newcommand{\EE}{\ensuremath{\mathbb{E}}}
\newcommand{\charin}{\text{ char }}
\DeclareMathOperator{\sign}{sign}
\DeclareMathOperator{\Aut}{Aut}
\DeclareMathOperator{\Inn}{Inn}
\DeclareMathOperator{\Syl}{Syl}
\DeclareMathOperator{\Gal}{Gal}
\DeclareMathOperator{\GL}{GL} % General linear group
\DeclareMathOperator{\SL}{SL} % Special linear group

%---------------------------------------
% BlackBoard Math Fonts :-
%---------------------------------------

%Captital Letters
\newcommand{\bbA}{\mathbb{A}}	\newcommand{\bbB}{\mathbb{B}}
\newcommand{\bbC}{\mathbb{C}}	\newcommand{\bbD}{\mathbb{D}}
\newcommand{\bbE}{\mathbb{E}}	\newcommand{\bbF}{\mathbb{F}}
\newcommand{\bbG}{\mathbb{G}}	\newcommand{\bbH}{\mathbb{H}}
\newcommand{\bbI}{\mathbb{I}}	\newcommand{\bbJ}{\mathbb{J}}
\newcommand{\bbK}{\mathbb{K}}	\newcommand{\bbL}{\mathbb{L}}
\newcommand{\bbM}{\mathbb{M}}	\newcommand{\bbN}{\mathbb{N}}
\newcommand{\bbO}{\mathbb{O}}	\newcommand{\bbP}{\mathbb{P}}
\newcommand{\bbQ}{\mathbb{Q}}	\newcommand{\bbR}{\mathbb{R}}
\newcommand{\bbS}{\mathbb{S}}	\newcommand{\bbT}{\mathbb{T}}
\newcommand{\bbU}{\mathbb{U}}	\newcommand{\bbV}{\mathbb{V}}
\newcommand{\bbW}{\mathbb{W}}	\newcommand{\bbX}{\mathbb{X}}
\newcommand{\bbY}{\mathbb{Y}}	\newcommand{\bbZ}{\mathbb{Z}}

%---------------------------------------
% MathCal Fonts :-
%---------------------------------------

%Captital Letters
\newcommand{\mcA}{\mathcal{A}}	\newcommand{\mcB}{\mathcal{B}}
\newcommand{\mcC}{\mathcal{C}}	\newcommand{\mcD}{\mathcal{D}}
\newcommand{\mcE}{\mathcal{E}}	\newcommand{\mcF}{\mathcal{F}}
\newcommand{\mcG}{\mathcal{G}}	\newcommand{\mcH}{\mathcal{H}}
\newcommand{\mcI}{\mathcal{I}}	\newcommand{\mcJ}{\mathcal{J}}
\newcommand{\mcK}{\mathcal{K}}	\newcommand{\mcL}{\mathcal{L}}
\newcommand{\mcM}{\mathcal{M}}	\newcommand{\mcN}{\mathcal{N}}
\newcommand{\mcO}{\mathcal{O}}	\newcommand{\mcP}{\mathcal{P}}
\newcommand{\mcQ}{\mathcal{Q}}	\newcommand{\mcR}{\mathcal{R}}
\newcommand{\mcS}{\mathcal{S}}	\newcommand{\mcT}{\mathcal{T}}
\newcommand{\mcU}{\mathcal{U}}	\newcommand{\mcV}{\mathcal{V}}
\newcommand{\mcW}{\mathcal{W}}	\newcommand{\mcX}{\mathcal{X}}
\newcommand{\mcY}{\mathcal{Y}}	\newcommand{\mcZ}{\mathcal{Z}}

%---------------------------------------
% Bold Math Fonts :-
%---------------------------------------

%Captital Letters
\newcommand{\bmA}{\boldsymbol{A}}	\newcommand{\bmB}{\boldsymbol{B}}
\newcommand{\bmC}{\boldsymbol{C}}	\newcommand{\bmD}{\boldsymbol{D}}
\newcommand{\bmE}{\boldsymbol{E}}	\newcommand{\bmF}{\boldsymbol{F}}
\newcommand{\bmG}{\boldsymbol{G}}	\newcommand{\bmH}{\boldsymbol{H}}
\newcommand{\bmI}{\boldsymbol{I}}	\newcommand{\bmJ}{\boldsymbol{J}}
\newcommand{\bmK}{\boldsymbol{K}}	\newcommand{\bmL}{\boldsymbol{L}}
\newcommand{\bmM}{\boldsymbol{M}}	\newcommand{\bmN}{\boldsymbol{N}}
\newcommand{\bmO}{\boldsymbol{O}}	\newcommand{\bmP}{\boldsymbol{P}}
\newcommand{\bmQ}{\boldsymbol{Q}}	\newcommand{\bmR}{\boldsymbol{R}}
\newcommand{\bmS}{\boldsymbol{S}}	\newcommand{\bmT}{\boldsymbol{T}}
\newcommand{\bmU}{\boldsymbol{U}}	\newcommand{\bmV}{\boldsymbol{V}}
\newcommand{\bmW}{\boldsymbol{W}}	\newcommand{\bmX}{\boldsymbol{X}}
\newcommand{\bmY}{\boldsymbol{Y}}	\newcommand{\bmZ}{\boldsymbol{Z}}
%Small Letters
\newcommand{\bma}{\boldsymbol{a}}	\newcommand{\bmb}{\boldsymbol{b}}
\newcommand{\bmc}{\boldsymbol{c}}	\newcommand{\bmd}{\boldsymbol{d}}
\newcommand{\bme}{\boldsymbol{e}}	\newcommand{\bmf}{\boldsymbol{f}}
\newcommand{\bmg}{\boldsymbol{g}}	\newcommand{\bmh}{\boldsymbol{h}}
\newcommand{\bmi}{\boldsymbol{i}}	\newcommand{\bmj}{\boldsymbol{j}}
\newcommand{\bmk}{\boldsymbol{k}}	\newcommand{\bml}{\boldsymbol{l}}
\newcommand{\bmm}{\boldsymbol{m}}	\newcommand{\bmn}{\boldsymbol{n}}
\newcommand{\bmo}{\boldsymbol{o}}	\newcommand{\bmp}{\boldsymbol{p}}
\newcommand{\bmq}{\boldsymbol{q}}	\newcommand{\bmr}{\boldsymbol{r}}
\newcommand{\bms}{\boldsymbol{s}}	\newcommand{\bmt}{\boldsymbol{t}}
\newcommand{\bmu}{\boldsymbol{u}}	\newcommand{\bmv}{\boldsymbol{v}}
\newcommand{\bmw}{\boldsymbol{w}}	\newcommand{\bmx}{\boldsymbol{x}}
\newcommand{\bmy}{\boldsymbol{y}}	\newcommand{\bmz}{\boldsymbol{z}}

%---------------------------------------
% Scr Math Fonts :-
%---------------------------------------

\newcommand{\sA}{{\mathscr{A}}}   \newcommand{\sB}{{\mathscr{B}}}
\newcommand{\sC}{{\mathscr{C}}}   \newcommand{\sD}{{\mathscr{D}}}
\newcommand{\sE}{{\mathscr{E}}}   \newcommand{\sF}{{\mathscr{F}}}
\newcommand{\sG}{{\mathscr{G}}}   \newcommand{\sH}{{\mathscr{H}}}
\newcommand{\sI}{{\mathscr{I}}}   \newcommand{\sJ}{{\mathscr{J}}}
\newcommand{\sK}{{\mathscr{K}}}   \newcommand{\sL}{{\mathscr{L}}}
\newcommand{\sM}{{\mathscr{M}}}   \newcommand{\sN}{{\mathscr{N}}}
\newcommand{\sO}{{\mathscr{O}}}   \newcommand{\sP}{{\mathscr{P}}}
\newcommand{\sQ}{{\mathscr{Q}}}   \newcommand{\sR}{{\mathscr{R}}}
\newcommand{\sS}{{\mathscr{S}}}   \newcommand{\sT}{{\mathscr{T}}}
\newcommand{\sU}{{\mathscr{U}}}   \newcommand{\sV}{{\mathscr{V}}}
\newcommand{\sW}{{\mathscr{W}}}   \newcommand{\sX}{{\mathscr{X}}}
\newcommand{\sY}{{\mathscr{Y}}}   \newcommand{\sZ}{{\mathscr{Z}}}


%---------------------------------------
% Math Fraktur Font
%---------------------------------------

%Captital Letters
\newcommand{\mfA}{\mathfrak{A}}	\newcommand{\mfB}{\mathfrak{B}}
\newcommand{\mfC}{\mathfrak{C}}	\newcommand{\mfD}{\mathfrak{D}}
\newcommand{\mfE}{\mathfrak{E}}	\newcommand{\mfF}{\mathfrak{F}}
\newcommand{\mfG}{\mathfrak{G}}	\newcommand{\mfH}{\mathfrak{H}}
\newcommand{\mfI}{\mathfrak{I}}	\newcommand{\mfJ}{\mathfrak{J}}
\newcommand{\mfK}{\mathfrak{K}}	\newcommand{\mfL}{\mathfrak{L}}
\newcommand{\mfM}{\mathfrak{M}}	\newcommand{\mfN}{\mathfrak{N}}
\newcommand{\mfO}{\mathfrak{O}}	\newcommand{\mfP}{\mathfrak{P}}
\newcommand{\mfQ}{\mathfrak{Q}}	\newcommand{\mfR}{\mathfrak{R}}
\newcommand{\mfS}{\mathfrak{S}}	\newcommand{\mfT}{\mathfrak{T}}
\newcommand{\mfU}{\mathfrak{U}}	\newcommand{\mfV}{\mathfrak{V}}
\newcommand{\mfW}{\mathfrak{W}}	\newcommand{\mfX}{\mathfrak{X}}
\newcommand{\mfY}{\mathfrak{Y}}	\newcommand{\mfZ}{\mathfrak{Z}}
%Small Letters
\newcommand{\mfa}{\mathfrak{a}}	\newcommand{\mfb}{\mathfrak{b}}
\newcommand{\mfc}{\mathfrak{c}}	\newcommand{\mfd}{\mathfrak{d}}
\newcommand{\mfe}{\mathfrak{e}}	\newcommand{\mff}{\mathfrak{f}}
\newcommand{\mfg}{\mathfrak{g}}	\newcommand{\mfh}{\mathfrak{h}}
\newcommand{\mfi}{\mathfrak{i}}	\newcommand{\mfj}{\mathfrak{j}}
\newcommand{\mfk}{\mathfrak{k}}	\newcommand{\mfl}{\mathfrak{l}}
\newcommand{\mfm}{\mathfrak{m}}	\newcommand{\mfn}{\mathfrak{n}}
\newcommand{\mfo}{\mathfrak{o}}	\newcommand{\mfp}{\mathfrak{p}}
\newcommand{\mfq}{\mathfrak{q}}	\newcommand{\mfr}{\mathfrak{r}}
\newcommand{\mfs}{\mathfrak{s}}	\newcommand{\mft}{\mathfrak{t}}
\newcommand{\mfu}{\mathfrak{u}}	\newcommand{\mfv}{\mathfrak{v}}
\newcommand{\mfw}{\mathfrak{w}}	\newcommand{\mfx}{\mathfrak{x}}
\newcommand{\mfy}{\mathfrak{y}}	\newcommand{\mfz}{\mathfrak{z}}

\newcommand{\mytitle}{MA1521 Homework 6}
\newcommand{\myauthor}{github/omgeta}
\newcommand{\mydate}{AY 24/25 Sem 1}

\begin{document}
\raggedright
\footnotesize
\begin{center}
{\normalsize{\textbf{\mytitle}}} \\
{\footnotesize{\mydate\hspace{2pt}\textemdash\hspace{2pt}\myauthor}}
\end{center}
\setlist{topsep=-1em, itemsep=-1em, parsep=2em}

%%%%%%%%%%%%%%%%%%%%%%%%%%%%%%%%%%%%%%%%%%%%%%%%%%%%%%
%                      Begin                         %
%%%%%%%%%%%%%%%%%%%%%%%%%%%%%%%%%%%%%%%%%%%%%%%%%%%%%%
\begin{enumerate}[Q\arabic*.]
  \item 
    \begin{enumerate}[(\alph*)]
      \item $\displaystyle \sum^{\infty}_{n=1}\cos^2\frac{1}{n}$
      \begin{align*}
        \lim_{n\rightarrow\infty}\cos^2\frac{1}{n} = 1
      \end{align*}
      $\therefore$ By nth term test, $\sum^{\infty}_{n=1}\cos^2\frac{1}{n}$ diverges. $\qed$ 

      \item $\displaystyle \sum^{\infty}_{n=2}\frac{1}{n(\ln n)^{r+1}}$
        \begin{align*}
        \int^{\infty}_2 \frac{1}{x(\ln x)^{r+1}} dx &= \int^{\infty}_{\ln 2}\frac{1}{u^{r+1}} du\tag*{(Sub $u=\ln x \implies dx = x du$)}\\
                                                    &= -\frac{1}{r}[u^{-r}]^{\infty}_{\ln 2}\\
                                                    &= -\frac{1}{r}[0 - (\ln  2)^{-r}]\\
                                                    &= \frac{1}{r(\ln 2)^r}
      \end{align*}
      $\therefore$ By integral test, $\sum^{\infty}_{n=2}\frac{1}{n(\ln n)^{r+1}}$ is convergent. $\qed$

    \item $\displaystyle \sum^{\infty}_{n=1} \sin^{2n}(\frac{1}{\sqrt{n}})$
      \begin{align*}
        \sin^{2n}(\frac{1}{\sqrt{n}}) &\approx (\frac{1}{\sqrt{n}})^{2n}\tag*{(For large $n$)}\\
                                      &= \frac{1}{n^2}\\
      \end{align*}
      Since $\displaystyle 0 \leq \sin^{2n}\frac{1}{\sqrt{n}} \leq \frac{1}{n^2}$, and $\displaystyle \sum^{\infty}_{n=1}\frac{1}{n^2}$ is a convergent p-series, then by comparison test, $\displaystyle \sum^{\infty}_{n=1} \sin^{2n}\frac{1}{\sqrt{n}}$ is convergent. $\qed$

    \item $\displaystyle \sum^{\infty}_{n=1}(-1)^n\frac{c}{\sqrt{d+n^2}}$
      \begin{align*}
        \frac{d}{dn}(\frac{c}{\sqrt{d+n^2}}) &= -\frac{c \cdot n}{(d+n^2)^{3/2}}\\
                                             &< 0\text{, for all $n \geq 1$}\\
        \lim_{n\rightarrow\infty}\frac{c}{\sqrt{d+n^2}} &= \lim_{n\rightarrow\infty}\frac{c}{n}\\
                                                        &= 0
      \end{align*}
      $\therefore$ by alternating series test, $\displaystyle \sum^{\infty}_{n=1}(-1)^n\frac{c}{\sqrt{d+n^2}}$ is convergent. $\qed$

    \item $\displaystyle \sum^{\infty}_{n=1} \frac{3+\sin n}{n^3}$
      \begin{align*}
      -1 \leq \sin n \leq 1\\
      2 \leq 3 + \sin n \leq 4\\
      \frac{2}{n^3} \leq \frac{3+\sin n}{n^3} \leq \frac{4}{n^3}\\
      \end{align*}
      Since both $\displaystyle \sum^{\infty}_{n=1} \frac{2}{n^3}$ and $\displaystyle \sum^{\infty}_{n=1} \frac{4}{n^3}$ are convergent p-series, then by comparison test, $\displaystyle \sum^{\infty}_{n=1} \frac{3+\sin n}{n^3}$ is convergent. $\qed$ 
    \item $\displaystyle \sum^{\infty}_{n=1}\frac{2^{1+3n}(n+1)}{n^25^{1+n}}$
      \begin{align*}
        \lim_{n\rightarrow\infty}|\frac{\frac{2^{1+3n+3}(n+2)}{(n+1)^25^{2+n}}}{\frac{2^{1+3n}(n+1)}{n^25^{1+n}}}| &= \lim_{n\rightarrow\infty}|\frac{2^3\cdot n^2(n+2)}{5(n+1)^3}|\\
                                                                                                                   &= \lim_{n\rightarrow\infty}|\frac{2^3(n^3+\ldots)}{5(n^3+\ldots)}|\\
                                                                                                                   &= \frac{8}{5}\\
                                                                                                                   &> 1
      \end{align*}
      $\therefore$ by ratio test, $\displaystyle \sum^{\infty}_{n=1}\frac{2^{1+3n}(n+1)}{n^25^{1+n}}$ is divergent. $\qed$ 
    \end{enumerate}

  \pagebreak
  \item 
    \begin{enumerate}[(\alph*)]
      \item $\displaystyle \sum^{\infty}_{n=1}(-1)^n \frac{(2x+3)^n}{n}$\\
        For the power series to be convergent:
        \begin{align*}
          \lim_{n\rightarrow\infty}|\frac{(\frac{(-1)^{n+1}(2x+3)^{n+1}}{n+1})}{\frac{(-1)^n(2x+3)^n}{n}}| &< 1\\
          \lim_{n\rightarrow\infty}|-(2x+3)\cdot \frac{n}{n+1}| &< 1\\
          |-(2x+3)| &< 1\\
          |2x+3| &< 1\\
          -1 < 2x+3 &< 1\\
          -2 < x &< -1
        \end{align*}
        At $x=-1$:
        \begin{align*}
          \sum^{\infty}_{n=1}(-1)^n \frac{(2x+3)^n}{n} &= \sum^{\infty}_{n=1}\frac{1}{n}\\&\text{ which is the divergent harmonic series}
        \end{align*}
        At $x=-2$:
        \begin{align*}
          \sum^{\infty}_{n=1}(-1)^n \frac{(2x+3)^n}{n} &= \sum^{\infty}_{n=1}\frac{(-1)^n}{n}\\&\text{ which is the convergent alternating harmonic series}
        \end{align*}

        $\therefore$ radius of convergence is $\frac{1}{2}$ and interval of convergence is $(-2, -1] \qed$ 

      \item $\displaystyle \sum^{\infty}_{n=1} (nx)^{n/5}$\\
        By ratio test:
        \begin{align*}
          \lim_{n\rightarrow\infty}|\frac{((n+1)x)^{n+1/5}}{(nx)^{n/5}}| &= \lim_{n\rightarrow\infty}|\frac{(n+1)^{n+1/5}}{n^{n/5}} \cdot x^{1 /5}|\\
                                                                         &= \lim_{n\rightarrow\infty}|(\frac{n+1}{n})^{n/5}\cdot (n+1)^{1/5} \cdot x^{1/5}|\\
                                                                         &= \lim_{n\rightarrow\infty}|e^{1/5} \cdot (n+1)^{1/5} \cdot x^{1/5}|\\
                                                                         &= \infty
        \end{align*}
        $\therefore$ radius of convergence is $0 \qed$
    \end{enumerate}
    \pagebreak

  \item $\displaystyle \sum^{\infty}_{n=1} a_n(-1)^nx^{2n}$\\
    For the power series to be convergent:
    \begin{align*}
      \lim_{n\rightarrow\infty}|\frac{a_{n+1}(-1)^{n+1}x^{2n+2}}{a_n(-1)^nx^{2n}}| &< 1\\
      \frac{1}{5}|x|^2 &< 1\\
      |x|^2 &< 5\\
      |x| &< \sqrt{5}\\
    \end{align*}
    $\therefore$ radius of convergence is $\sqrt{5} \qed$

  \item 
    \begin{enumerate}[(\alph*)]
      \item $\displaystyle \frac{x}{1-x}$ at $x=0$
        \begin{align*}
          \frac{x}{1-x} &= x \cdot \frac{1}{1-x}\\
                        &= x \cdot \sum^{\infty}_{n=0}x^n\\
                        &= \sum^{\infty}_{n=0}x^{n+1} \qed
        \end{align*}

      \item $\displaystyle \frac{1}{x^2}$ at $x=1$
        \begin{align*}
          f(x) &= \frac{1}{x^2}\\
          f'(x) &= -\frac{2}{x^3}\\
          f''(x) &= \frac{6}{x^4}\\
          f'''(x) &= -\frac{24}{x^5}\\
          \frac{1}{x^2} &= 1 + (-\frac{2}{(1)^3})(\frac{1}{1!})(x-1) + (\frac{6}{1^4})(\frac{1}{2!})(x-1)^2 + \ldots\\
                        &= 1 -2(x-1) + 3(x-1)^2 -4(x-1)^3 +\ldots\\
                        &= \sum^{\infty}_{n=0} (-1)^n(n+1)(x-1)^n \qed
        \end{align*}

      \item $\displaystyle \frac{x}{1+x}$ at $x=-2$
        \begin{align*}
          f(x) &= \frac{x}{1+x}\\
          f'(x) &= \frac{1}{(1+x)^2}\\
          f''(x) &= \frac{-2}{(1+x)^3}\\
          \frac{x}{1+x} &= 2(\frac{1}{0!})(x+2)^0 + 1(\frac{1}{1!})(x+2)^1 + (2)(\frac{1}{2!})(x+2)^2 + \ldots\\
                        &= 2 + \sum^{\infty}_{n=1}(x+2)^n \qed
        \end{align*}
    \end{enumerate}
  \pagebreak
  \item 
    Given $\displaystyle e^x = \sum^{\infty}_{n=0} \frac{x^n}{n!}$:
    \begin{align*}
      xe^x &= x\sum^{\infty}_{n=0} \frac{x^n}{n!}\\
           &= \sum^{\infty}_{n=0} \frac{x^{n+1}}{n!}\\
      \int^1_0 xe^x dx &= [\sum^{\infty}_{n=0} \frac{x^{n+2}}{(n+2)n!}]^1_0\\
      [xe^x - e^x]^1_0 &= \sum^{\infty}_{n=0} \frac{1}{(n+2)n!}\\
      1 &= S\\
      \therefore S &= 1 \qed
    \end{align*}
\end{enumerate}
%%%%%%%%%%%%%%%%%%%%%%%%%%%%%%%%%%%%%%%%%%%%%%%%%%%%%%
%                       End                          %
%%%%%%%%%%%%%%%%%%%%%%%%%%%%%%%%%%%%%%%%%%%%%%%%%%%%%%

\end{document}

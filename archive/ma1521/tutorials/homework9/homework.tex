\documentclass[12pt, a4paper]{article}

\usepackage[a4paper, margin=1in]{geometry}

\usepackage[utf8]{inputenc}
\usepackage[mathscr]{euscript}
\let\euscr\mathscr \let\mathscr\relax
\usepackage[scr]{rsfso}
\usepackage{amssymb,amsmath,amsthm,amsfonts}
\usepackage[shortlabels]{enumitem}
\usepackage{multicol,multirow}
\usepackage{lipsum}
\usepackage{balance}
\usepackage{calc}
\usepackage[colorlinks=true,citecolor=blue,linkcolor=blue]{hyperref}
\usepackage{import}
\usepackage{xifthen}
\usepackage{pdfpages}
\usepackage{transparent}
\usepackage{listings}

\newcommand{\incfig}[2][1.0]{
    \def\svgwidth{#1\columnwidth}
    \import{./figures/}{#2.pdf_tex}
}

\newlist{enumproof}{enumerate}{4}
\setlist[enumproof,1]{label=\arabic*., parsep=1em}
\setlist[enumproof,2]{label=\arabic{enumproofi}.\arabic*., parsep=1em}
\setlist[enumproof,3]{label=\arabic{enumproofi}.\arabic{enumproofii}.\arabic*., parsep=1em}
\setlist[enumproof,4]{label=\arabic{enumproofi}.\arabic{enumproofii}.\arabic{enumproofiii}.\arabic*., parsep=1em}

\renewcommand{\qedsymbol}{\ensuremath{\blacksquare}}

\lstdefinestyle{mystyle}{
  language=C, % Set the language to C
  commentstyle=\color{codegray}, % Color for comments
  keywordstyle=\color{orange}, % Color for basic keywords
  stringstyle=\color{mauve}, % Color for strings
  basicstyle={\ttfamily\footnotesize}, % Basic font style
  breakatwhitespace=false,         
  breaklines=true,                 
  captionpos=b,                    
  keepspaces=true,                 
  numbers=none,                    
  tabsize=2,
  morekeywords=[2]{\#include, \#define, \#ifdef, \#ifndef, \#endif, \#pragma, \#else, \#elif}, % Preprocessor directives
  keywordstyle=[2]\color{codegreen}, % Style for preprocessor directives
  morekeywords=[3]{int, char, float, double, void, struct, union, enum, const, volatile, static, extern, register, inline, restrict, _Bool, _Complex, _Imaginary, size_t, ssize_t, FILE}, % C standard types and common identifiers
  keywordstyle=[3]\color{identblue}, % Style for types and common identifiers
  morekeywords=[4]{printf, scanf, fopen, fclose, malloc, free, calloc, realloc, perror, strtok, strncpy, strcpy, strcmp, strlen}, % Standard library functions
  keywordstyle=[4]\color{cyan}, % Style for library functions
}

% Things Lie
\newcommand{\kb}{\mathfrak b}
\newcommand{\kg}{\mathfrak g}
\newcommand{\kh}{\mathfrak h}
\newcommand{\kn}{\mathfrak n}
\newcommand{\ku}{\mathfrak u}
\newcommand{\kz}{\mathfrak z}
\DeclareMathOperator{\Ext}{Ext} % Ext functor
\DeclareMathOperator{\Tor}{Tor} % Tor functor
\newcommand{\gl}{\opname{\mathfrak{gl}}} % frak gl group
\renewcommand{\sl}{\opname{\mathfrak{sl}}} % frak sl group chktex 6

% More script letters etc.
\newcommand{\SA}{\mathcal A}
\newcommand{\SB}{\mathcal B}
\newcommand{\SC}{\mathcal C}
\newcommand{\SF}{\mathcal F}
\newcommand{\SG}{\mathcal G}
\newcommand{\SH}{\mathcal H}
\newcommand{\OO}{\mathcal O}

\newcommand{\SCA}{\mathscr A}
\newcommand{\SCB}{\mathscr B}
\newcommand{\SCC}{\mathscr C}
\newcommand{\SCD}{\mathscr D}
\newcommand{\SCE}{\mathscr E}
\newcommand{\SCF}{\mathscr F}
\newcommand{\SCG}{\mathscr G}
\newcommand{\SCH}{\mathscr H}

% Mathfrak primes
\newcommand{\km}{\mathfrak m}
\newcommand{\kp}{\mathfrak p}
\newcommand{\kq}{\mathfrak q}

% number sets
\newcommand{\RR}[1][]{\ensuremath{\ifstrempty{#1}{\mathbb{R}}{\mathbb{R}^{#1}}}}
\newcommand{\NN}[1][]{\ensuremath{\ifstrempty{#1}{\mathbb{N}}{\mathbb{N}^{#1}}}}
\newcommand{\ZZ}[1][]{\ensuremath{\ifstrempty{#1}{\mathbb{Z}}{\mathbb{Z}^{#1}}}}
\newcommand{\QQ}[1][]{\ensuremath{\ifstrempty{#1}{\mathbb{Q}}{\mathbb{Q}^{#1}}}}
\newcommand{\CC}[1][]{\ensuremath{\ifstrempty{#1}{\mathbb{C}}{\mathbb{C}^{#1}}}}
\newcommand{\PP}[1][]{\ensuremath{\ifstrempty{#1}{\mathbb{P}}{\mathbb{P}^{#1}}}}
\newcommand{\HH}[1][]{\ensuremath{\ifstrempty{#1}{\mathbb{H}}{\mathbb{H}^{#1}}}}
\newcommand{\FF}[1][]{\ensuremath{\ifstrempty{#1}{\mathbb{F}}{\mathbb{F}^{#1}}}}
% expected value
\newcommand{\EE}{\ensuremath{\mathbb{E}}}
\newcommand{\charin}{\text{ char }}
\DeclareMathOperator{\sign}{sign}
\DeclareMathOperator{\Aut}{Aut}
\DeclareMathOperator{\Inn}{Inn}
\DeclareMathOperator{\Syl}{Syl}
\DeclareMathOperator{\Gal}{Gal}
\DeclareMathOperator{\GL}{GL} % General linear group
\DeclareMathOperator{\SL}{SL} % Special linear group

%---------------------------------------
% BlackBoard Math Fonts :-
%---------------------------------------

%Captital Letters
\newcommand{\bbA}{\mathbb{A}}	\newcommand{\bbB}{\mathbb{B}}
\newcommand{\bbC}{\mathbb{C}}	\newcommand{\bbD}{\mathbb{D}}
\newcommand{\bbE}{\mathbb{E}}	\newcommand{\bbF}{\mathbb{F}}
\newcommand{\bbG}{\mathbb{G}}	\newcommand{\bbH}{\mathbb{H}}
\newcommand{\bbI}{\mathbb{I}}	\newcommand{\bbJ}{\mathbb{J}}
\newcommand{\bbK}{\mathbb{K}}	\newcommand{\bbL}{\mathbb{L}}
\newcommand{\bbM}{\mathbb{M}}	\newcommand{\bbN}{\mathbb{N}}
\newcommand{\bbO}{\mathbb{O}}	\newcommand{\bbP}{\mathbb{P}}
\newcommand{\bbQ}{\mathbb{Q}}	\newcommand{\bbR}{\mathbb{R}}
\newcommand{\bbS}{\mathbb{S}}	\newcommand{\bbT}{\mathbb{T}}
\newcommand{\bbU}{\mathbb{U}}	\newcommand{\bbV}{\mathbb{V}}
\newcommand{\bbW}{\mathbb{W}}	\newcommand{\bbX}{\mathbb{X}}
\newcommand{\bbY}{\mathbb{Y}}	\newcommand{\bbZ}{\mathbb{Z}}

%---------------------------------------
% MathCal Fonts :-
%---------------------------------------

%Captital Letters
\newcommand{\mcA}{\mathcal{A}}	\newcommand{\mcB}{\mathcal{B}}
\newcommand{\mcC}{\mathcal{C}}	\newcommand{\mcD}{\mathcal{D}}
\newcommand{\mcE}{\mathcal{E}}	\newcommand{\mcF}{\mathcal{F}}
\newcommand{\mcG}{\mathcal{G}}	\newcommand{\mcH}{\mathcal{H}}
\newcommand{\mcI}{\mathcal{I}}	\newcommand{\mcJ}{\mathcal{J}}
\newcommand{\mcK}{\mathcal{K}}	\newcommand{\mcL}{\mathcal{L}}
\newcommand{\mcM}{\mathcal{M}}	\newcommand{\mcN}{\mathcal{N}}
\newcommand{\mcO}{\mathcal{O}}	\newcommand{\mcP}{\mathcal{P}}
\newcommand{\mcQ}{\mathcal{Q}}	\newcommand{\mcR}{\mathcal{R}}
\newcommand{\mcS}{\mathcal{S}}	\newcommand{\mcT}{\mathcal{T}}
\newcommand{\mcU}{\mathcal{U}}	\newcommand{\mcV}{\mathcal{V}}
\newcommand{\mcW}{\mathcal{W}}	\newcommand{\mcX}{\mathcal{X}}
\newcommand{\mcY}{\mathcal{Y}}	\newcommand{\mcZ}{\mathcal{Z}}

%---------------------------------------
% Bold Math Fonts :-
%---------------------------------------

%Captital Letters
\newcommand{\bmA}{\boldsymbol{A}}	\newcommand{\bmB}{\boldsymbol{B}}
\newcommand{\bmC}{\boldsymbol{C}}	\newcommand{\bmD}{\boldsymbol{D}}
\newcommand{\bmE}{\boldsymbol{E}}	\newcommand{\bmF}{\boldsymbol{F}}
\newcommand{\bmG}{\boldsymbol{G}}	\newcommand{\bmH}{\boldsymbol{H}}
\newcommand{\bmI}{\boldsymbol{I}}	\newcommand{\bmJ}{\boldsymbol{J}}
\newcommand{\bmK}{\boldsymbol{K}}	\newcommand{\bmL}{\boldsymbol{L}}
\newcommand{\bmM}{\boldsymbol{M}}	\newcommand{\bmN}{\boldsymbol{N}}
\newcommand{\bmO}{\boldsymbol{O}}	\newcommand{\bmP}{\boldsymbol{P}}
\newcommand{\bmQ}{\boldsymbol{Q}}	\newcommand{\bmR}{\boldsymbol{R}}
\newcommand{\bmS}{\boldsymbol{S}}	\newcommand{\bmT}{\boldsymbol{T}}
\newcommand{\bmU}{\boldsymbol{U}}	\newcommand{\bmV}{\boldsymbol{V}}
\newcommand{\bmW}{\boldsymbol{W}}	\newcommand{\bmX}{\boldsymbol{X}}
\newcommand{\bmY}{\boldsymbol{Y}}	\newcommand{\bmZ}{\boldsymbol{Z}}
%Small Letters
\newcommand{\bma}{\boldsymbol{a}}	\newcommand{\bmb}{\boldsymbol{b}}
\newcommand{\bmc}{\boldsymbol{c}}	\newcommand{\bmd}{\boldsymbol{d}}
\newcommand{\bme}{\boldsymbol{e}}	\newcommand{\bmf}{\boldsymbol{f}}
\newcommand{\bmg}{\boldsymbol{g}}	\newcommand{\bmh}{\boldsymbol{h}}
\newcommand{\bmi}{\boldsymbol{i}}	\newcommand{\bmj}{\boldsymbol{j}}
\newcommand{\bmk}{\boldsymbol{k}}	\newcommand{\bml}{\boldsymbol{l}}
\newcommand{\bmm}{\boldsymbol{m}}	\newcommand{\bmn}{\boldsymbol{n}}
\newcommand{\bmo}{\boldsymbol{o}}	\newcommand{\bmp}{\boldsymbol{p}}
\newcommand{\bmq}{\boldsymbol{q}}	\newcommand{\bmr}{\boldsymbol{r}}
\newcommand{\bms}{\boldsymbol{s}}	\newcommand{\bmt}{\boldsymbol{t}}
\newcommand{\bmu}{\boldsymbol{u}}	\newcommand{\bmv}{\boldsymbol{v}}
\newcommand{\bmw}{\boldsymbol{w}}	\newcommand{\bmx}{\boldsymbol{x}}
\newcommand{\bmy}{\boldsymbol{y}}	\newcommand{\bmz}{\boldsymbol{z}}

%---------------------------------------
% Scr Math Fonts :-
%---------------------------------------

\newcommand{\sA}{{\mathscr{A}}}   \newcommand{\sB}{{\mathscr{B}}}
\newcommand{\sC}{{\mathscr{C}}}   \newcommand{\sD}{{\mathscr{D}}}
\newcommand{\sE}{{\mathscr{E}}}   \newcommand{\sF}{{\mathscr{F}}}
\newcommand{\sG}{{\mathscr{G}}}   \newcommand{\sH}{{\mathscr{H}}}
\newcommand{\sI}{{\mathscr{I}}}   \newcommand{\sJ}{{\mathscr{J}}}
\newcommand{\sK}{{\mathscr{K}}}   \newcommand{\sL}{{\mathscr{L}}}
\newcommand{\sM}{{\mathscr{M}}}   \newcommand{\sN}{{\mathscr{N}}}
\newcommand{\sO}{{\mathscr{O}}}   \newcommand{\sP}{{\mathscr{P}}}
\newcommand{\sQ}{{\mathscr{Q}}}   \newcommand{\sR}{{\mathscr{R}}}
\newcommand{\sS}{{\mathscr{S}}}   \newcommand{\sT}{{\mathscr{T}}}
\newcommand{\sU}{{\mathscr{U}}}   \newcommand{\sV}{{\mathscr{V}}}
\newcommand{\sW}{{\mathscr{W}}}   \newcommand{\sX}{{\mathscr{X}}}
\newcommand{\sY}{{\mathscr{Y}}}   \newcommand{\sZ}{{\mathscr{Z}}}


%---------------------------------------
% Math Fraktur Font
%---------------------------------------

%Captital Letters
\newcommand{\mfA}{\mathfrak{A}}	\newcommand{\mfB}{\mathfrak{B}}
\newcommand{\mfC}{\mathfrak{C}}	\newcommand{\mfD}{\mathfrak{D}}
\newcommand{\mfE}{\mathfrak{E}}	\newcommand{\mfF}{\mathfrak{F}}
\newcommand{\mfG}{\mathfrak{G}}	\newcommand{\mfH}{\mathfrak{H}}
\newcommand{\mfI}{\mathfrak{I}}	\newcommand{\mfJ}{\mathfrak{J}}
\newcommand{\mfK}{\mathfrak{K}}	\newcommand{\mfL}{\mathfrak{L}}
\newcommand{\mfM}{\mathfrak{M}}	\newcommand{\mfN}{\mathfrak{N}}
\newcommand{\mfO}{\mathfrak{O}}	\newcommand{\mfP}{\mathfrak{P}}
\newcommand{\mfQ}{\mathfrak{Q}}	\newcommand{\mfR}{\mathfrak{R}}
\newcommand{\mfS}{\mathfrak{S}}	\newcommand{\mfT}{\mathfrak{T}}
\newcommand{\mfU}{\mathfrak{U}}	\newcommand{\mfV}{\mathfrak{V}}
\newcommand{\mfW}{\mathfrak{W}}	\newcommand{\mfX}{\mathfrak{X}}
\newcommand{\mfY}{\mathfrak{Y}}	\newcommand{\mfZ}{\mathfrak{Z}}
%Small Letters
\newcommand{\mfa}{\mathfrak{a}}	\newcommand{\mfb}{\mathfrak{b}}
\newcommand{\mfc}{\mathfrak{c}}	\newcommand{\mfd}{\mathfrak{d}}
\newcommand{\mfe}{\mathfrak{e}}	\newcommand{\mff}{\mathfrak{f}}
\newcommand{\mfg}{\mathfrak{g}}	\newcommand{\mfh}{\mathfrak{h}}
\newcommand{\mfi}{\mathfrak{i}}	\newcommand{\mfj}{\mathfrak{j}}
\newcommand{\mfk}{\mathfrak{k}}	\newcommand{\mfl}{\mathfrak{l}}
\newcommand{\mfm}{\mathfrak{m}}	\newcommand{\mfn}{\mathfrak{n}}
\newcommand{\mfo}{\mathfrak{o}}	\newcommand{\mfp}{\mathfrak{p}}
\newcommand{\mfq}{\mathfrak{q}}	\newcommand{\mfr}{\mathfrak{r}}
\newcommand{\mfs}{\mathfrak{s}}	\newcommand{\mft}{\mathfrak{t}}
\newcommand{\mfu}{\mathfrak{u}}	\newcommand{\mfv}{\mathfrak{v}}
\newcommand{\mfw}{\mathfrak{w}}	\newcommand{\mfx}{\mathfrak{x}}
\newcommand{\mfy}{\mathfrak{y}}	\newcommand{\mfz}{\mathfrak{z}}

\newcommand{\mytitle}{MA1521 Homework 9}
\newcommand{\myauthor}{github/omgeta}
\newcommand{\mydate}{AY 24/25 Sem 1}

\begin{document}
\raggedright
\footnotesize
\begin{center}
{\normalsize{\textbf{\mytitle}}} \\
{\footnotesize{\mydate\hspace{2pt}\textemdash\hspace{2pt}\myauthor}}
\end{center}
\setlist{topsep=-1em, itemsep=-1em, parsep=2em}

%%%%%%%%%%%%%%%%%%%%%%%%%%%%%%%%%%%%%%%%%%%%%%%%%%%%%%
%                      Begin                         %
%%%%%%%%%%%%%%%%%%%%%%%%%%%%%%%%%%%%%%%%%%%%%%%%%%%%%%
\begin{enumerate}[Q\arabic*.]
  \item Given $f(x, y) = xe^{-y}$:
    \begin{align*}
      f_x = e^{-y},&\quad f_y = -xe^{-y}\\
      \implies \nabla f(x, y) &= \langle e^{-y}, -xe^{-y} \rangle
    \end{align*}
    
    \begin{enumerate}[(\alph*)]
      \item 
        \begin{enumerate}[(\roman*)]
          \item At $P(-2, 0)$ and $\hat{u} = \langle \frac{1}{\sqrt{2}}, \frac{1}{\sqrt{2}}\rangle$:
            \begin{align*}
              D_{\hat{u}}f(-2, 0) &= \nabla f(-2, 0) \cdot \langle \frac{1}{\sqrt{2}}, \frac{1}{\sqrt{2}}\rangle\\
                                  &= \langle 1, 2 \rangle \cdot \langle \frac{1}{\sqrt{2}}, \frac{1}{\sqrt{2}}\rangle\\
                                  &= \frac{3}{\sqrt{2}}\qed
            \end{align*}
          \item At $P(-2, 0)$ and $\vec{u} = \langle 3, 4\rangle \implies \hat{u} = \frac{1}{5}\langle 3, 4 \rangle$:
            \begin{align*}
              D_{\hat{u}}f(-2, 0) &= \nabla f(-2, 0) \cdot \frac{1}{5} \langle 3, 4\rangle\\
                                  &= \langle 1, 2 \rangle \cdot \frac{1}{5}\langle 3,4\rangle\\
                                  &= \frac{11}{5}\qed
            \end{align*}
        \end{enumerate}
      \item $f$ increases fastest at $P(-2, 0)$ with the unit gradient vector:
        \begin{align*}
          \frac{\nabla f(-2, 0)}{|\nabla f(-2, 0)|} &= \frac{\langle 1. 2\rangle}{\sqrt{5}}\\
                                                    &= \frac{1}{\sqrt{5}}\hat{i} + \frac{2}{\sqrt{5}}\hat{j} \qed
        \end{align*}
    \end{enumerate}

  \item Given $f(x,y,z) = xy + \sin (xyz)$:
    \begin{align*}
      f_x = y+yz\cos(xyz),\quad f_y = xz\cos(xyz),\quad f_z = xy\cos(xyz)\\
      \implies \nabla f(x,y,z) = \langle y + yz\cos(xyz), xz\cos(xyz), xy\cos(xyz)\rangle
    \end{align*}
    \begin{enumerate}[(\roman*)]
      \item At $P(\frac{1}{2}, \frac{1}{3}, \pi)$ and $\hat{u} = \langle \frac{1}{\sqrt{3}}, -\frac{1}{\sqrt{3}}, \frac{1}{\sqrt{3}}\rangle$:
        \begin{align*}
          D_{\hat{u}}f(P) &= \nabla f(P) \cdot \hat{u}\\
                          &= \langle \frac{1}{3} + \frac{\pi\sqrt{3}}{6}, \frac{1}{2} + \frac{\pi\sqrt{3}}{4}, \frac{\sqrt{3}}{12}\rangle \cdot \langle \frac{1}{\sqrt{3}}, -\frac{1}{\sqrt{3}}, \frac{1}{\sqrt{3}} \rangle\\
                          &= (\frac{1}{3\sqrt{3}} + \frac{\pi}{6}) + (-\frac{1}{2\sqrt{3}} - \frac{\pi}{4}) + \frac{1}{12}\\
                          &= (\frac{2}{6\sqrt{3}} - \frac{3}{6\sqrt{3}}) + (\frac{\pi}{6} - \frac{\pi}{4} + \frac{1}{12})\\
                          &= \frac{1}{12}(1-\pi) - \frac{1}{6\sqrt{3}} \qed
        \end{align*}

      \item By linear approximation,
        \begin{align*}
          \Delta f &= D_{\hat{u}}f(P) \cdot 0.1\\
                   &\approx -0.0275 \qed
    \end{align*}
    \end{enumerate}

  \item 
    \begin{enumerate}[(\roman*)]
      \item Given $f(x,y) = \ln(x^2 y) - xy -2x +2,$ where $x>0, y>0$:
        \begin{align*}
          f_x = \frac{2}{x} - y -2,&\quad f_y = \frac{1}{y} - x\\
          f_{xx} = -\frac{2}{x^2},\quad f_{xy}&=-1,\quad f_{yy}=-\frac{1}{y^2}
        \end{align*}
        At critical points:
        \begin{align*}
          f_x &= \frac{2}{x} - y -2 = 0\\
          f_y &= \frac{1}{y} - x = 0
        \end{align*}
        Solving simultaneously, $x=\frac{1}{2}, y=2$, which by second derivative test:
        \begin{align*}
          D &= f_{xx}(\frac{1}{2}, 2)f_{yy}(\frac{1}{2}, 2) - (f_{xy}(\frac{1}{2}, 2))^2\\
            &= (-8)(-\frac{1}{4}) - (-1)^2\\
            &= 1 > 0
        \end{align*}
        Therefore, $f(\frac{1}{2}, 2) = -\ln 2$ is a local maximum.$\qed$

      \item Given $g(x,y) = xy(1-x-y)$:
        \begin{align*}
          f_x = y-2xy-y^2,&\quad\quad f_y = x-x^2-2xy\\
          f_{xx} = -2y,\quad f_{xy}=&1-2x-2y,\quad f_{yy}=-2x
        \end{align*}
        At critical points:
        \begin{align*}
          f_x &= y-2xy-y^2 = 0\\
          f_y &= x-x^2-2xy = 0
        \end{align*}
        Solving simultaneously, we get points $(0,0), (1,0), (0,1), (\frac{1}{3},\frac{1}{3})$ which by second derivative test:
        \begin{align*}
          D(0,0) &= -1 < 0\\
          D(1,0) &= -1 < 0\\
          D(0,1) &= -1 < 0\\
          D(\frac{1}{3},\frac{1}{3}) &= \frac{1}{3} > 0, f_{xx} = -\frac{2}{3} < 0
        \end{align*}
        Therefore, $(0,0), (1,0), (0,1)$ are saddle points and $g(\frac{1}{3},\frac{1}{3})=\frac{1}{27}$ is a local maximum.$\qed$ 
    \end{enumerate}
    \pagebreak

  \item 
    \begin{enumerate}[(\alph*)]
      \item 
        \begin{align*}
          \int \int_R x^3 + y^3 dA &= \int^b_0\int^a_0 x^3 + y^3 dx dy\\
                                   &= \int^b_0 [\frac{1}{4}x^4 + xy^3]^a_0 dy\\
                                   &= \int^b_0 \frac{1}{4}a^4 + ay^3 dy\\
                                   &= [\frac{1}{4}a^4y + \frac{1}{4}ay^3]^b_0\\
                                   &= \frac{1}{4}a^4b + \frac{1}{4}ab^4\qed
        \end{align*}

      \item 
        \begin{align*}
          \int \int_R \frac{xy}{\sqrt{4-x^2}}dA &= \int^3_1\int^2_0 \frac{xy}{\sqrt{4-x^2}}dxdy\\
                                                &= \int^3_1 y \int^2_0 \frac{x}{\sqrt{4-x^2}} dxdy\\
                                                &= \int^3_1 y \int^0_4 -\frac{1}{2} \cdot \frac{1}{\sqrt{u}} du dy\tag*{($u=4-x^2$, $du=-2x dx$)}\\
                                                &= \int^3_1 y[\sqrt{u}]^4_0 dy\\
                                                &= \int^3_1 2y dy\\
                                                &= [y^2]^3_1\\
                                                &= 8\qed
        \end{align*}
        Alternatively:
        \begin{align*}
          \int \int_R \frac{xy}{\sqrt{4-x^2}}dA &= (\int^2_0 \frac{x}{\sqrt{4-x^2}}dx)(\int^3_1 y dy)\\
                                                &= (\int^4_0 \frac{1}{2\sqrt{u}} du)[\frac{1}{2}y^2]^3_1\tag*{($u=4-x^2$, $du=-2xdx$)}\\
                                                &= [\sqrt{u}]^4_0 \cdot 4\\
                                                &= 8 \qed
        \end{align*}
    \end{enumerate}

  \item 
    \begin{align*}
      \int^3_0 \int^2_0 2 + (x-1)^2 + 4y^2 dydx &= \int^3_0 [2y+(x-1)^2y + \frac{4}{3}y^3]^2_0 dx\\
                                                &= \int^3_0 4 + 2(x-1)^2 + \frac{32}{3} dx\\
                                                &= \int^3_0 2x^2 - 4x + \frac{50}{3} dx\\
                                                &= [\frac{2}{3}x^3 -2x^2 + \frac{50}{3}x]^3_0\\
                                                &= 50\qed
    \end{align*}
\end{enumerate}
%%%%%%%%%%%%%%%%%%%%%%%%%%%%%%%%%%%%%%%%%%%%%%%%%%%%%%
%                       End                          %
%%%%%%%%%%%%%%%%%%%%%%%%%%%%%%%%%%%%%%%%%%%%%%%%%%%%%%

\end{document}

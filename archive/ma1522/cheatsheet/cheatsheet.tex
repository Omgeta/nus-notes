\documentclass[12pt, a4paper]{article}

\usepackage[utf8]{inputenc}
\usepackage[mathscr]{euscript}
\let\euscr\mathscr \let\mathscr\relax
\usepackage[scr]{rsfso}
\usepackage{amssymb,amsmath,amsthm,amsfonts}
\usepackage[shortlabels]{enumitem}
\usepackage{multicol,multirow}
\usepackage{lipsum}
\usepackage{balance}
\usepackage{calc}
\usepackage[colorlinks=true,citecolor=blue,linkcolor=blue]{hyperref}
\usepackage{import}
\usepackage{xifthen}
\usepackage{pdfpages}
\usepackage{transparent}
\usepackage{tabularx}

\newcommand{\incfig}[2][1.0]{
    \def\svgwidth{#1\columnwidth}
    \import{./figures/}{#2.pdf_tex}
}
\newcommand{\incimg}[2][1.0]{
  \includegraphics[width=#1\columnwidth]{./figures/#2}
}


\usepackage{ifthen}
\usepackage[landscape]{geometry}
\usepackage[shortlabels]{enumitem}

\ifthenelse{\lengthtest { \paperwidth = 11in}}
    { \geometry{top=.5in,left=.5in,right=.5in,bottom=.5in} }
	{\ifthenelse{ \lengthtest{ \paperwidth = 297mm}}
		{\geometry{top=1cm,left=1cm,right=1cm,bottom=1cm} }
		{\geometry{top=1cm,left=1cm,right=1cm,bottom=1cm} }
	}

\pagestyle{empty}
\makeatletter
\renewcommand\thesection{\arabic{section}.}
\renewcommand{\section}{\@startsection{section}{1}{0mm}%
                                {-1ex plus -.5ex minus -.2ex}%
                                {0.05ex}%x
                                {\normalfont\normalsize\bfseries}}
\renewcommand{\subsection}{\@startsection{subsection}{2}{0mm}%
                                {-1ex plus -.5ex minus -.2ex}%
                                {0.05ex}%
                                {\normalfont\small\bfseries}}
\renewcommand{\subsubsection}{\@startsection{subsubsection}{3}{0mm}%
                                {-1ex plus -.5ex minus -.2ex}%
                                {0.05ex}%
                                {\normalfont\footnotesize\bfseries}}
\newcommand{\colbreak}{\vfill\null\columnbreak}
\makeatother
\setcounter{secnumdepth}{1}
\setlength{\parindent}{0pt}
\setlength{\parskip}{0.7em}

\setlist[itemize]{itemsep=0.6ex, topsep=-2pt, partopsep=0pt, parsep=0pt}
\setlist[enumerate]{itemsep=0.6ex, topsep=-2pt, partopsep=0pt, parsep=0pt}

\input{letterfonts}

\newcommand{\mytitle}{MA1522 Linear Algebra for Computing}
\newcommand{\myauthor}{github/omgeta}
\newcommand{\mydate}{AY 24/25 Sem 1}

\begin{document}
\raggedright
\footnotesize
\begin{multicols*}{3}
\setlength{\premulticols}{1pt}
\setlength{\postmulticols}{1pt}
\setlength{\multicolsep}{1pt}
\setlength{\columnsep}{1pt}

{\normalsize{\textbf{\mytitle}}} \\
{\footnotesize{\mydate\hspace{2pt}\textemdash\hspace{2pt}\myauthor}}
\vspace{-1em}
%%%%%%%%%%%%%%%%%%%%%%%%%%%%%%%%%%%%%%%%%%%%%%%%%%%%%%
%                      Begin                         %
%%%%%%%%%%%%%%%%%%%%%%%%%%%%%%%%%%%%%%%%%%%%%%%%%%%%%%
\section{Vector Spaces}
A vector space $V$ is a nonempty set of vectors with the following properties for all vectors $\vec{u}, \vec{v}, \vec{w} \in V$ and for all scalars $c$ and $d$:
\begin{enumerate}[\roman*.]
  \item $\vec{u} + \vec{v} \in V$
  \item $c\vec{u} \in V$
  \item $\vec{u} + \vec{v} = \vec{v} + \vec{u}$
  \item $(\vec{u} + \vec{v}) + \vec{w} = \vec{u} + (\vec{v} + \vec{w})$
  \item $\vec{0} \in V \text{ such that } \vec{u} + \vec{0} = \vec{u}$
  \item $c(\vec{u} + \vec{v}) = c\vec{u} + c\vec{v}$
  \item $(c + d)\vec{u} = c\vec{u} + d\vec{u}$
  \item $c(d\vec{u}) = (cd)\vec{u}$ 
  \item $1\vec{u} = \vec{u}$
\end{enumerate}

A subspace $W$ of a vector space $V$ is a subset with the following properties for all vectors $\vec{u}, \vec{v} \in W$ and for all scalars $c$:
\begin{enumerate}[\roman*.]
  \item $0 \in W$\hfill(Includes zero vector)
  \item $\vec{u} + \vec{v} \in W$\hfill(Closure over addition)
  \item $c\vec{u} \in W$\hfill(Closure over multiplication)
\end{enumerate}

A linear map from $V$ to $W$ is a function $T: V \rightarrow W$ with the following properties for linear maps $R, S, T$ and for all scalars $c, d$ for which the following are defined:
\begin{enumerate}[\roman*.]
  \item $S + T = T + S$\hfill(Commutative)
  \item $(R + S) + T = R + (S + T)$\hfill(Associative)
  \item $T + \mathbf{0} = T$\hfill(Additive Identity)
  \item $T + (-T) = \mathbf{0}$\hfill(Additive Inverse)
  \item $c(dT) = (cd)T$\hfill(Associative)
  \item $c(S + T) = cS + cT$\hfill(Distributive)
  \item $(c+d)T = cT + dT$\hfill(Scalar Addition)
  \item $R(ST) = (RS)T$\hfill(Associative)
  \item $R(S + T) = RS + RT$\hfill(Distributive)
  \item $(S+T)R = SR + TR$\hfill(Distributive)
  \item $c(ST) = (cS)T = S(cT)$\hfill(Associative)
  \item $TI = IT = T$\hfill(Identity)
\end{enumerate}

\colbreak

\section{Vectors}
For some vector $\vec{v} \in \RR^n$, where $v_1, \ldots, v_n \in \RR$:
\begin{align*}
  \vec{v} = \begin{bmatrix}
    v_1 \\
    \vdots \\
    v_n
  \end{bmatrix}
\end{align*}

\subsection{Linear Combinations}
For vectors $\vec{v}_1,\ldots,\vec{v}_p \in V$ and scalars $c_1,\ldots,c_p$, the vector $\vec{y}$ given by:
\begin{align*}
  \vec{y} = c_1\vec{v}_1 + \ldots + c_p\vec{v}_p
\end{align*}
is a linear combination of $\vec{v}_1,\ldots,\vec{v}_p$ with weights $c_1,\ldots,c_p$

\subsection{Linear Span}
For a set of vectors $S = \{\vec{v}_1,\ldots,\vec{v}_p\} \subseteq V$, Span$(S) \subseteq V$ denotes the set of all linear combinations of $\vec{v}_1,\ldots,\vec{v}_p$ and is given by:
\begin{align*}   
  \text{Span}(S) = \left\{ \sum^p_{i=1}c_i\vec{v}_i | \vec{v}_i \in S, c_i \in \RR\right\}
\end{align*}

\subsection{Linear Dependence}
For a set of non-zero vectors  $S = \{\vec{v}_1,\ldots,\vec{v}_p\} \subseteq V$, $S$ is linearly dependent if and only if some vector $\vec{v}_i$ is a linear combination of the others. Any set $\{\vec{v_1},\ldots,\vec{v_p}\} \in \RR^n$ is linearly dependent if $p>n$.

A linearly independent set of vectors forms a matrix with a pivot position in every column.

\subsection{Basis}
For a set of non-zero vectors  $S = \{\vec{v}_1,\ldots,\vec{v}_p\} \subseteq V$, $S$ is a basis for $W \subseteq V$ if:
\begin{enumerate}[\roman*.]
  \item S is a linearly independent set, and
  \item Span$(S) = W$
\end{enumerate}

\colbreak

\section{Matrices}
For some matrix $A \in \RR^{m{\times}n}$, where $a_{11},\ldots,a_{mn} \in \RR$:
\begin{align*}
  A = \begin{bmatrix}
    a_{11} & \cdots & a_{1n} \\
    \vdots & \ddots & \vdots \\
    a_{m1} & \cdots & a_{mn} \\
  \end{bmatrix}
\end{align*}

For some matrices $A \in \RR^{m{\times}n}$ and $B \in \RR^{n{\times}p}$:
\begin{align*}
  AB &= A\begin{bmatrix}\vec{b_1} & \cdots & \vec{b_p}\end{bmatrix} \\
                                 &= \begin{bmatrix}A\vec{b_1} & \cdots & A\vec{b_p}\end{bmatrix} \\ 
  (AB)_{ij} &= \sum^n_{k=1}A_{ik}B_{kj} \\
  \text{row}_i(AB) &= \text{row}_i(A){\cdot}B
\end{align*}

Additional properties:
\begin{enumerate}[\roman*.]
  \item $AB \neq BA$ in general\hfill(Not commutative)
  \item $\exists A,B\neq 0,$ $AB = 0$\hfill(Zero divisor)
  \item $A0 = 0$\hfill(Zero matrix)
  \item $A$ has zero row$\implies AB$ has zero row
  \item $B$ has zero column$\implies AB$ has zero column
\end{enumerate}

\subsection{Transpose}
For some matrix $A \in \RR^{m{\times}n}$, the transpose $A^T \in \RR^{n{\times}m}$ is given by:
\begin{align*}
  A^T_{ij} &= A_{ji} \\
  \text{row}_i(A^T) &= \text{column}_i(A)
\end{align*}

Properties:
\begin{enumerate}[\roman*.]
  \item $(A^T)^T = A$
  \item $(cA)^T = cA^T$
  \item $(A + B)^T = A^T + B^T$
  \item $(AB)^T = B^TA^T$
  \item $A^TA = 0 \iff A = 0$
\end{enumerate}

\colbreak
\subsection{Systems of Linear Equations}
Systems of linear equations of the form:
\begin{gather*}
  a_{11}x_1 + \cdots + a_{1n}x_n = b_1 \\
  \vdots \\
  a_{m1}x_1 + \cdots + a_{mn}x_n = b_n
\end{gather*}
can be expressed in the matrix-vector form, $A\vec{x} = \vec{b}$, where $A \in \RR^{m{\times}n}$ is the coefficient matrix and $\vec{x} \in \RR^n$ is the solution vector:
\begin{gather*}
  \begin{bmatrix}a_{11} & \cdots & a_{1n} \\ \vdots & \ddots & \vdots \\ a_{m1} & \cdots & a_{mn}\end{bmatrix}\begin{bmatrix}x_1 \\ \vdots \\ x_n \end{bmatrix} = \begin{bmatrix}b_1 \\ \vdots \\ b_n \end{bmatrix}
\end{gather*}
By finding the RREF of the augmented matrix, we can find the solution set for the original system.

\subsection{Row Equivalence}
Two matrices $A, B \in \RR^{m{\times}n}$ are row equivalent if one can be changed to the other by a sequence of elementary row operations, that is:
\begin{align*}
  B = E_kE_{k-1}{\ldots}E_1A \iff A \sim B
\end{align*}

Elementary row operations:
\begin{enumerate}[\roman*.]
  \item Add a multiple of a row to another row\hfill($R_n + cR_m$)
  \item Scale a row by a nonzero constant\hfill($cR_n$)
  \item Interchange two rows\hfill($R_n \leftrightarrow R_m$)
\end{enumerate}

\subsection{Row Echelon Forms}
A matrix is in row echelon form (REF) if:
\begin{enumerate}[\roman*.]
  \item All nonzero rows are above all zero rows
  \item Each pivot is to the right of the pivot of the row above it
  \item All entries below a pivot are zeros
\end{enumerate}

A matrix is in reduced row echelon form (RREF) if:
\begin{enumerate}[\roman*.]
  \item All pivots are 1
  \item All other entries in the pivot column are 0
\end{enumerate}

\colbreak

\subsection{Inverse}
For some square matrix $A \in \RR^{n\times n}$, $A$ is invertible/nonsingular if there exists the inverse $A^{-1} \in \RR^{n\times n}$ such that:
\begin{align*}
  AA^{-1} = A^{-1}A = I
\end{align*}

For matrix $A = \begin{bmatrix} a & b \\ c & d\end{bmatrix}, A^{-1} = \frac{1}{ad-bc}\begin{bmatrix} d & -b \\ -c & a\end{bmatrix}$

Using row reduction, we can solve for $A^{-1}$ by:
\begin{align*}
  \begin{bmatrix}A & | & I\end{bmatrix} \xrightarrow{RREF} \begin{bmatrix}I & | & A^{-1}\end{bmatrix}
\end{align*}

Properties:
\begin{enumerate}[\roman*.]
  \item $(A^{-1})^{-1} = A$
  \item $(cA)^{-1} = c^{-1}A^{-1}$
  \item $(A^T)^{-1} = (A^{-1})^T$
  \item $(AB)^{-1} = B^{-1}A^{-1}$
  \item $AB = AC \implies B = C$
  \item $BA = CA \implies B = C$
\end{enumerate}

\subsubsection{Invertible Matrix Theorem}
Let $A$ be a $n \times n$ matrix. The following statements are equivalent:
\begin{enumerate}[\roman*.]
  \item $A$ is invertible
  \item $A^T$ is invertible
  \item $A$ has a left inverse, $C$, such that $CA = I$
  \item $A$ has a right inverse, $D$, such that $AD = I$
  \item RREF of $A$ is $I$
  \item Columns of $A$ form a basis for $\RR^n$
  \item $A\vec{x} = \vec{0}$ has only the trivial solution
  \item $A\vec{x} = \vec{b}$ has a unique solution, $\forall \vec{b} \in \RR^n$
  \item $\nul(A) = \{\mathbf{0}\} \Leftrightarrow (\nul(A))^{\perp} = \RR^n \Leftrightarrow \nullity(A) = 0$
  \item $\col(A) = \RR^n \Leftrightarrow (\col(A))^{\perp} = \{\mathbf{0}\}\Leftrightarrow \rank(A) = n$
  \item $\det(A) \neq 0$
  \item $0$ is not an eigenvalue
\end{enumerate}

\colbreak

\subsection{Determinant}
For some square matrix $A \in \RR^{n{\times}n}$, the determinant $\det(A)$ or $|A|$ is given by:
\begin{align*}
  \det(A) = \sum^n_{j=1 \text{ or } i=1}a_{ij}A_{ij}
\end{align*}
where $A_{ij}$ is the $(i, j)$ cofactor of $A$ given by:
\begin{align*}
  A_{ij} = (-1)^{i+j}\det(M_{ij})
\end{align*}
where $M_{ij} \in \RR^{(n-1){\times}(n-1)}$ is the $(i, j)$ matrix minor of A obtained by deleting the $i$th row and $j$th column of A.

For matrix $A = \begin{bmatrix} a & b \\ c & d\end{bmatrix}, \det(A) = ad-bc $

Properties:
\begin{enumerate}[\roman*.]
  \item $\det(A^T) = \det(A)$
  \item $\det(AB) = \det(A)\det(B)$
  \item $\det(A^{-1}) = \frac{1}{\det(A)}$
  \item $\det(cA) = c^n\det(A)$ 
  \item If $A$ is triangular, $\det(A)$ is the product of the entries on the main diagonal of A
\end{enumerate}

For the elementary matrix $E \in \RR^{m{\times}n}$, $\det(A)$ is given by the type of elementary row operation:
\begin{enumerate}[\roman*.]
  \item $ R_n + cR_m\implies \det(E) = 1$
  \item $cR_n \implies \det(E) = c$
  \item $R_n \leftrightarrow R_m \implies \det(E) = -1$
\end{enumerate}

\subsubsection{Cramer's Rule}
Let an invertible matrix $A$ and any $b \in \RR^n$, the unique solution of $A\vec{x} = \vec{b}$ has it's entries given by:
\begin{align*}
  x_i = \frac{\det(A_i(\vec{b}))}{\det(A)}
\end{align*}
where $A_i(\vec{b})$ is the matrix obtained by replacing column$_i(A)$ with $\vec{b}$.

\colbreak

\subsection{Adjoint}
For some square matrix $A \in \RR^{n{\times}n}$, the adjoint $\adj(A)$ is given by:
\begin{gather*}
  \adj(A) = (A_{ij})^T = 
  \begin{bmatrix}
    A_{11} & A_{21} & \cdots & A_{n_1} \\
    A_{12} & A_{22} & \cdots & A_{n_2} \\
    \vdots & \vdots & \ddots & \vdots \\
    A_{1n} & A_{2n} & \cdots & A_{nn}
  \end{bmatrix}  
\end{gather*}
Properties:
\begin{enumerate}[\roman*.]
  \item $A\cdot\adj(A) = \det(A)\cdot I$
  \item $A$ is singular $\iff \adj(A)$ is singular
  \item $\adj(AB) = \adj(B)\adj(A)$
  \item $\det(\adj(A)) = \det(A)^{-1}$
  \item $\adj(cA) = c^{n-1}\adj(A)$
  \item $\adj(A^{-1}) = \adj(A)^{-1}$
  \item $\adj(\adj(A)) =  (\det(A))^{n-2}A$
\end{enumerate}

\subsection{Change of Basis}
For bases $\mcB = \{\vec{b_1},\ldots,\vec{b_n}\}$ and $\mcC = \{\vec{c_1},\ldots,\vec{c_n}\}$ of a vector space $\RR^n$, there is unique change of basis matrix $P_{\mcC{\leftarrow}\mcB} \in \RR^{n{\times}n} = \begin{bmatrix}[\vec{b_1}]_{\mcC} \cdots [\vec{b_n}]_{\mcC} \end{bmatrix}$ such that:
\begin{align*}
  \begin{bmatrix}\vec{x}\end{bmatrix}_\mcC &= P_{\mcC{\leftarrow}\mcB}\begin{bmatrix}\vec{x}\end{bmatrix}_\mcB \\
  (P_{\mcC{\leftarrow}\mcB})^{-1}\begin{bmatrix}\vec{x}\end{bmatrix}_\mcC &= \begin{bmatrix}\vec{x}\end{bmatrix}_\mcB \\ 
  \implies (P_{\mcC{\leftarrow}\mcB})^{-1} &= P_{\mcB{\leftarrow}\mcC}
\end{align*}
where $\begin{bmatrix}\vec{x}\end{bmatrix}_\mcB$ and $\begin{bmatrix}\vec{x}\end{bmatrix}_\mcC$ are the vector $\vec{x}$ represented in the coordinate system used by the bases $\mcB$ and $\mcC$ respectively.

Using row reduction, we can solve for ${P_{C{\leftarrow}B}}$ by:
\begin{align*}
  \begin{bmatrix}\vec{c_1} & \cdots & \vec{c_n} & | & \vec{b_1} & \cdots & \vec{b_n}\end{bmatrix} \xrightarrow{RREF} \begin{bmatrix}I & | & P_{C{\leftarrow}B}\end{bmatrix} 
\end{align*}

When converting from a basis $\mcB$ to the standard basis $\mcE = \{\vec{e_1},\ldots,\vec{e_n}\}$, change of basis matrix $P_\mcB$ is given by $P_\mcB =  \begin{bmatrix}\vec{b_1} \cdots \vec{b_n}\end{bmatrix}$ such that:
\begin{align*}
  \vec{x} &= P_\mcB\begin{bmatrix}\vec{x}\end{bmatrix}_\mcB \\
  (P_\mcB)^{-1}\vec{x} &= \begin{bmatrix}\vec{x}\end{bmatrix}_\mcB
\end{align*}
\colbreak

\section{Subspaces}

\subsection{Null Space}
For any matrix $A \in \RR^{m{\times}n}$, the null space $\nul(A) \in \RR^n$ is the solution space to the homogenous equation $A\vec{x} = \vec{0}$ given by:
\begin{align*}
  \nul(A) = \{\vec{x} \in \RR^n | A\vec{x} = \vec{0}\}
\end{align*}

\subsection{Column Space}
For any matrix $A = \begin{bmatrix}\vec{a_1} & \cdots & \vec{a_n}\end{bmatrix} \in \RR^{m{\times}n}$, the column space $\col(A) \in \RR^m$ is the set of all linear combinations of the columns of A given by:
\begin{gather*}
  \col(A) = \Span(\{\vec{a_1},\ldots,\vec{a_n}\}) \\
          = \{\vec{b} \in \RR^m : A\vec{x} = \vec{b} \text{ for some }\vec{x} \in \RR^n\}
\end{gather*}

\subsection{Row Space}
For any matrix $A \in \RR^{m{\times}n}$, the row space $\row(A) \in \RR^n$ is the set of all linear combinations of the rows of A given by:
\begin{align*}
  \row(A) = \col(A^T)
\end{align*}

\subsection{Dimension}
If a vector space $V$ is spanned by a finite set, V is finite-dimensional and the dimension $\dim(V)$ is the number of vectors in any basis for V.
\begin{align*}
  \rank(A) &= \dim(\col(A)) = \dim(\row(A)) \\
           &= \text{number of pivot columns} \\ 
           &= \text{number of pivot rows} \\
  \text{(Full rank)} \rank(A) &= min(m, n)\\
  \rank(AB) &\leq min(\rank(A), \rank(B))\\
  \nullity(A) &= \dim(\nul(A)) \\
              &= \text{number of free variables}
\end{align*}

\subsubsection{Rank-Nullity Theorem}
Rank and nullity of any matrix $A \in \RR^{m{\times}n}$ satisfy:
\begin{align*}
  \rank(A) + \nullity(A) = \text{number of columns in }A
\end{align*}

\colbreak

\section{Eigenvectors}

For some square matrix $A \in \RR^{n{\times}n}$, all nonzero eigenvectors $\vec{x}$ and corresponding eigenvalues $\lambda$ satisfy:
\begin{align*}
  A \vec{x} = \lambda \vec{x}
\end{align*}
For each eigenvalue $\lambda$, the corresponding eigenvectors are found as the nontrivial solutions to $(A - \lambda I) = 0$ as elements of the corresponding eigenspace of A.

Characteristic equation for eigenvalues $\lambda$:
\begin{align*}
  \det(A - \lambda I) = 0
\end{align*}
where algebraic mulitiplicity of $\lambda$ is the number of roots, and geometric multiplicity is the dimension of the eigenspace of $\lambda$.

\section{Dot Product}

Dot product of two vectors, $\vec{u}, \vec{v} \in \RR^n$ is given by:
\begin{align*}
  \vec{u} \cdot \vec{v} = \vec{u}^T \vec{v}
                    &= \begin{bmatrix}u_1 & \cdots & u_n\end{bmatrix} \begin{bmatrix}v_1 \\ \vdots \\ v_n\end{bmatrix}\\
                    &= u_1v_1 + \cdots + u_nv_n \\
                    &= \norm{\vec{u}}\norm{\vec{v}}\cos\theta
\end{align*}
where $\theta$ is the angle between $\vec{u}$ and $\vec{v}$.

Length of $\vec{v}$  is given by:
\begin{align*}
  \norm{v} &= \sqrt{\vec{v}\cdot\vec{v}} = \sqrt{v^2_1 + \cdots + v^2_n}\\
  \norm{v}^2 &= \vec{v}\cdot\vec{v}\\
  \dist(\vec{u}, \vec{v}) &= \norm{\vec{u} - \vec{v}}
\end{align*}

Properties:
\begin{enumerate}[\roman*.]
  \item $\vec{u} \cdot \vec{v} = \vec{v} \cdot \vec{u}$
  \item $(\vec{u} + \vec{v}) \cdot \vec{w} = \vec{u} \cdot \vec{w} + \vec{v} \cdot \vec{w}$
  \item $(c \vec{u}) \cdot \vec{v} = c (\vec{u} \cdot \vec{v}) = \vec{u} \cdot (c \vec{v})$
  \item $\vec{u} \cdot \vec{u} \geq 0,$ and $\vec{u} \cdot \vec{u} = 0 \iff \vec{u} = \vec{0}$
\end{enumerate}
\colbreak

\section{Orthogonality}

Let $\vec{u}, \vec{v} \in \RR^n$ be orthogonal vectors. The following statements are equivalent:
\begin{enumerate}[\roman*.]
  \item $\vec{u} \cdot \vec{v} = 0$
  \item $\norm{\vec{u} + \vec{v}}^2 = \norm{\vec{u}}^2 + \norm{\vec{v}}^2$
  \item $\dist(\vec{u}, \vec{v}) = \dist(\vec{u}, -\vec{v})$
\end{enumerate}

A set of vectors is orthogonal if all vectors are mutually orthogonal, and orthonormal if all vectors are unit vectors.

\subsection{Orthogonal Basis}
An orthogonal set of nonzero vectors is linearly independent and a basis for the subspace it spans.\\
Any orthonormal set is automatically an orthonormal basis for the subspace it spans.

For any orthogonal basis $S = \{\vec{u_1},\ldots,\vec{u_n}\}$ of subspace $V$ with matrix $Q$, $\forall \vec{v} \in V$:
\begin{align*}
  \vec{v} = \frac{\vec{v}\cdot\vec{u_1}}{\norm{u_1}^2}\vec{u_1} + \cdots +  \frac{\vec{v}\cdot\vec{u_n}}{\norm{u_n}^2}\vec{u_n}\\
  [\vec{v}]_S =  Q^T \vec{v} = 
  \begin{bmatrix}
    \vec{v} \cdot \vec{u_1} / \norm{u_1}^2\\
    \vdots\\
    \vec{v} \cdot \vec{u_n} / \norm{u_n}^2\\
  \end{bmatrix}
\end{align*}

Orthogonal complement $V^T$ for a subspace $V$ with orthogonal basis vectors in matrix $A$ is given by:
\begin{align*}
  V^{\perp} = \col(A)^{\perp} = \nul(A^T),\quad \row(A)^{\perp} = \nul(A)
\end{align*}

\subsection{Orthogonal Matrix}
For some square matrix $A \in \RR^{n{\times}n}$, $A$ is orthogonal if it has orthonormal columns (and equivalently rows) or:
\begin{align*}
  A^TA = AA^T = I \iff A^T = A^{-1}
\end{align*}
Properties:
\begin{enumerate}[\roman*.]
  \item $\norm{A\vec{x}} = \norm{\vec{x}}$
  \item $(A\vec{x})\cdot(A\vec{y}) = \vec{x}\cdot\vec{y}$
  \item $(A\vec{x})\cdot(A\vec{y}) = 0 \iff \vec{x}\cdot\vec{y} = 0$
  \item $\det(A) = \pm 1 $
\end{enumerate}

\colbreak

\subsection{Projection}
For any two vectors $\vec{x}, \vec{y} \in \RR^n$, the projection of $\vec{y}$ onto $\vec{x}$, $\proj_{\vec{x}}\vec{y}$, is given by:
\begin{align*}
  \proj_{\vec{x}}\vec{y} &= (\vec{y}\cdot\hat{x})\hat{x} = \frac{\vec{y}\cdot\vec{x}}{\norm{\vec{x}^2}}\vec{x}
\end{align*}

For any vector $\vec{y} \in \RR^n$ and subspace $W \subseteq \RR^n$ with orthogonal basis $\{\vec{v_1},\ldots,\vec{v_p}\}$, the projection of $\vec{y}$ onto $W$, $\proj_w\vec{y}$ or $\hat{y}$, is given by:
\begin{align*}
  \proj_W\vec{y} &= \hat{y} = \frac{\vec{y}\cdot\vec{v_1}}{\norm{\vec{v_1}}^2}\vec{v_1} + \cdots +  \frac{\vec{y}\cdot\vec{v_p}}{\norm{\vec{v_p}}^2}\vec{v_p}\\ 
  \vec{y} &= \hat{y} + e
\end{align*}

\subsubsection{Gram-Schmidt Process}
Let $X = \{\vec{x_1},\ldots,\vec{x_n}\}$ be a basis for a nonzero subspace $W \subseteq \RR^n$.
\begin{align*}
  \vec{v_1} &= \vec{x_1}. \\
  \vec{v_2} &= \vec{x_2} - \frac{\vec{x_2}\cdot\vec{v_1}}{\norm{v_1}^2}\vec{v_1} \\
  \cdots \\
  \vec{v_n} &= \vec{x_n} - \frac{\vec{x_n}\cdot\vec{v_1}}{\norm{v_1}^2}\vec{v_1} - \frac{\vec{x_n}\cdot\vec{v_2}}{\norm{v_2}^2}\vec{v_2} - \cdots -  \frac{\vec{x_n}\cdot\vec{v_{n-1}}}{\norm{v_{n-1}}^2}\vec{v_{n-1}}    
\end{align*}
Then $X' = \{\vec{v_1},\ldots,\vec{v_n}\}$ is an orthogonal basis for W

\subsection{Least Squares Solutions}
For any matrix equation $A\vec{x} = \vec{b}, A \in \RR^{m{\times}n}, \vec{b} \in \RR^m,$ the vector $\hat{x} \in \RR^n$ is the least-squares solution such that:
\begin{align*}
  \norm{\vec{b} - A\hat{x}} \leq \norm{\vec{b} - A\vec{x}}, \forall \vec{x} \in \RR^n
\end{align*}
with the least squares solution $\hat{x}$ given by the solution set:
\begin{align*}
  A^TA\hat{x} = A^T\vec{b}\\
  \hat{x} = (A^TA)^{-1}A^T \vec{b}\tag*{(If unique $\hat{x}$)}\\
  \hat{x} = R^{-1}Q^T \vec{b}\tag*{(If $A=QR$)}
\end{align*}

In particular, if $S = \{\vec{u_1},\cdots,\vec{u_n}\}$ is a basis for $V$, then:
\begin{align*}
  \proj_V \vec{w} &= A(A^TA)^{-1}A^T \vec{w}\\
  \proj_V \vec{w} &= AA^T \vec{w}\tag*{($A$ orthogonal)}
\end{align*}

\subsubsection{Full Column Rank Theorem}
Let $A$ be a ${m\times n}$ matrix. The following statements are equivalent:
\begin{enumerate}[\roman*.]
  \item $\rank(A) = \rank(A^TA) = n = \text{ number of columns}$
  \item $\nul(A) = \nul(A^TA) = \{\mathbf{0}\} \iff$ $\nullity(A) = 0$
  \item Rows of $A$ span $\RR^n \iff \row(A) = \RR^n$ 
  \item Columns of $A$ are linearly independent
  \item $A\vec{x} = \vec{0}$ has only the trivial solution 
  \item $A^TA$ is invertible
  \item $A$ has a left inverse
  \item $A$ has a QR factorization
  \item Least square solution of $A \vec{x} = \vec{b}$ is unique, $\forall \vec{b}\in \RR^m$
  \item $A$ is one-to-one
\end{enumerate}

\section{Stochaistic Matrices}
For some square matrix $A \in \RR^{n{\times}n}$, $A$ is stochaistic if and only if:
\begin{enumerate}[\roman*.]
  \item Entries are nonnegative
  \item Sum of each column is 1 
  \item 1 is an eigenvalue
\end{enumerate}

A stochaistic matrix $P$ is regular if for $k > 0$, $P^k$ has positive entries

A Markov chain is a sequence of probability vectors $\vec{x_0}, \cdots \vec{x_k}, \cdots$ with a stochaistic matrix $P$ such that:
\begin{align*}
  \vec{x_1} = P \vec{x_0},\quad \vec{x_2}=P \vec{x_1},\quad \vec{x_k}=P \vec{x_{k-1}}=P^k \vec{x_0}
\end{align*}

Steady-state/equilibrium vector for a stochaistic matrix $P$ is a probability vector which is an eigenvector for $\lambda = 1$. 

If there is a diagonalization of $P = QDQ^{-1}$, we can find steady-state vector $\vec{x_{\infty}} = QD^{k}Q$ as $k\rightarrow\infty$, where any $-1 < \lambda < 1$ approaches 0, or by solving $(I-P)x_{\infty} =0$.
\colbreak
\section{Factorisations}

\subsection{LU Factorisation}
Suppose matrix $A \in \RR^{m{\times}n}$ can be reduced to row echelon form $U$ without row interchanges, then $A$ can be factorised as:
\begin{align*}
  A &= LU \\
    &= \begin{bmatrix}1 & 0 & \cdots & 0 \\ * & 1 & \cdots & 0 \\ \vdots & \vdots & \ddots & 0 \\ * & * & \cdots & 1\end{bmatrix}
       \begin{bmatrix}* & * & \cdots & * \\ 0 & * & \cdots & * \\ 0 & 0 & \ddots & \vdots \\ 0 & 0 & 0 & *\end{bmatrix}
\end{align*}
where upper triangular matrix $U \in \RR^{m{\times}n}$ is given by:
\begin{align*}
  U &= REF(A), \text{ without row interchanges} \\
    &= E_k{\ldots}E_1A
\end{align*}
and unit lower triangular matrix $L \in \RR^{m{\times}m}$ is constructed from unit pivot columns of A (and additional columns of I if there are insufficient pivot columns) such that:
\begin{align*}
  E_k{\ldots}E_1L = I
\end{align*}

\subsection{QR Factorisation}
Suppose matrix $A \in \RR^{m{\times}n}$ has linearly independent columns (full column rank), then $A$ can be factorised as:
\begin{align*}
  A = QR
\end{align*}
where matrix $Q \in \RR^{m{\times}n}$ is constructed from orthonormal basis vectors for $\col(A)$ given by:
\begin{gather*}
  \text{Orthonormal Basis }A' = \text{Gram-Schmidt on } A \\
  Q = \begin{bmatrix}\vec{a_1}' \cdots \vec{a_n}'\end{bmatrix}
\end{gather*}
and invertible upper triangular matrix $R = P_{A\rightarrow Q} \in \RR^{n{\times}n}$ with positive diagonals given by:
\begin{align*}
  R &= Q^TA
\end{align*}
ensuring $R$ has positive diagonals by multiplying columns of $Q$ by -1 as needed.
\colbreak

\subsection{Diagonalisation}
Suppose matrix $A \in \RR^{n{\times}n}$ has $n$ linearly independent eigenvectors $\{\vec{v_1},\ldots,\vec{v_n}\}$ (algebraic multiplicity of each $\lambda =$ geometric multiplicity), then $A$ is diagonalisable as:
\begin{align*}
  A = PDP^{-1}
\end{align*}
where invertible change of basis matrix $P \in \RR^{n{\times}n}$ is constructed from the linearly independent eigenvectors of A such that:
\begin{align*}
  P = \begin{bmatrix}\vec{v_1} \cdots \vec{v_n}\end{bmatrix}
\end{align*}
and diagonal matrix $D \in \RR^{n{\times}n}$ is constructed from the corresponding eigenvalues of the eigenvectors chosen for the columns of $P$ in the same order:
\begin{align*}
  D = \begin{bmatrix}\lambda_1 & 0 & \cdots & 0\\ 0 & \lambda_2 & \cdots & 0\\ \vdots & \vdots & \ddots & \vdots \\ 0 & 0 & \cdots & \lambda_n \end{bmatrix}
\end{align*}

\subsection{Orthogonal Diagonalisation}
Suppose matrix $A \in \RR^{n{\times}n}$ is symmetric, then $A$ is orthogonally diagonalisable as:
\begin{align*}
  A = PDP^T
\end{align*}
where invertible change of basis matrix $P \in \RR^{n{\times}n}$ is also orthogonal such that:
\begin{align*}
  P = \begin{bmatrix}\vec{u_1} \cdots \vec{u_n}\end{bmatrix}
\end{align*}
and $A$ also has spectral decomposition given by:
\begin{align*}
  A = \lambda_1 \vec{u_1} \vec{u_1}^T + \cdots + \lambda_n \vec{u_n} \vec{u_n}^T
\end{align*}

\subsubsection{Spectral Theorem}
A symmetric matrix $A \in \RR^{n{\times}n}$ has the following properties:
\begin{enumerate}[\roman*.]
  \item $A$ has $n$ real eigenvalues, counting multiplicities
  \item Dimension of each eigenspace for eigenvalue $\lambda$ equals the algebraic multiplicity of $\lambda$
  \item Eigenspaces are mutually orthogonal, such that eigenvectors of different eigenvalues are orthogonal
  \item A is orthogonally diagonalisable
\end{enumerate}

\colbreak

\subsection{Singular Value Decomposition (SVD)}
Any matrix $A \in \RR^{m{\times}n}$ with $\rank r$ can be decomposed as:
\begin{align*}
A = U{\Sigma}V^T
\end{align*}
where "diagonal" matrix $\Sigma \in \RR^{m{\times}n}$ is constructed with the decreasing $r$ singular values $\sigma = \sqrt{\lambda}$ of $A$, for eigenvalues $\lambda$ of the symmetric matrix $A^TA$, such that:
\begin{align*}
  \Sigma &= \begin{bmatrix}D & 0 \\ 0 & 0\end{bmatrix} \\
  D &= \begin{bmatrix}\sigma_1 & 0 & \cdots & 0 \\ 0 & \sigma_2 & \cdots & 0 \\ \vdots & \vdots & \ddots & \vdots \\ 0 & 0 & \cdots & \sigma_r\end{bmatrix} \\
    & \sigma_1 \geq \sigma_2 \geq \ldots \geq \sigma_r 
\end{align*}
and orthogonal matrix $V \in \RR^{n{\times}n}$ is constructed with the corresponding right singular vectors of $A$, given by the corresponding unit eigenvectors $\{\vec{v_1},\ldots,\vec{v_n}\}$ of $A^TA$, such that:
\begin{align*}
  V = \begin{bmatrix}\vec{v_1} \cdots \vec{v_n}\end{bmatrix}
\end{align*}
and orthogonal matrix $U \in \RR^{m{\times}m}$ is constructed with the corresponding left singular vectors of $A$, such that:
\begin{align*}
  \vec{u_i} &= \frac{1}{\sigma_i}A\vec{v_i} \\
  U &= [\vec{u_1} \cdots \vec{u_m}]
\end{align*}

During construction of $U$ and $V$, if there are insufficient singular vectors to form an orthogonal matrix, additional orthonormal vectors can be formed from the Gram-Schmidt process.

\colbreak
\section*{MATLAB Commands}

\subsection{Miscellaneous}
\begin{tabular}{ll}
  \verb|format rat| & format fraction output\\
  \verb|format short| & format short output\\
  \verb|clc| & clear screen\\
  \verb|clear| & clear all variables\\
  \verb|clear x| & clear variable $x$\\
  \verb|syms x| & create symbol $x$
\end{tabular}

\subsection{Variable Generation}
\begin{tabular}{ll}
  \verb|A=[1 2; 3 4]| & $2\times 2$ matrix\\
  \verb|j:k| & row vector \verb|[j,j+1,\cdots,k]|\\
  \verb|j:i:k| & row vector \verb|[j,j+i,\cdots,k]|\\
  \verb|ones(m,n)| & $m\times n$ matrix of $1$s\\
  \verb|zeros(m,n)| & $m\times n$ matrix of $0$s\\
  \verb|eye(n)| & $n\times n$ identity matrix
\end{tabular}

\subsection{Matrix Manipulation}
\begin{tabular}{ll}
  \verb|A(:)| & all elements\\
  \verb|A(i,j)| & \verb|i,j| entry\\
  \verb|A(j,:)| & \verb|j|th row\\
  \verb|A(:,j)| & \verb|j|th column\\
  \verb|diag(A)| & main diagonal\\
  \verb|[A B]| & concatenate horizontally \\
  \verb|[A;B]| & concatenate vertically \\
  \verb|A*B| & matrix multiplication \\
  \verb|A*n| & scalar multiplication \\
  \verb|A.*B| & element-wise multiplication \\
  \verb|A/n| & scalar division \\
  \verb|A/B| & element-wise division \\
  \verb|A+B| & element-wise addition \\
  \verb|A-B| & element-wise subtraction \\
  \verb|A^n| & matrix power \\
  \verb|A.^n| & element-wise power \\
\end{tabular}

\subsection{Elementary Row Operations}
\begin{tabular}{ll}
  \verb|A(i, :) = A(i, :) + c * A(j, :)| & Add \verb|c| $\times$ row \verb|j| to row \verb|i|\\
  \verb|A(i, :) = c * A(i, :)| & Multiply row \verb|i| by \verb|c|\\
  \verb|A([i, j], :) = A([j, i], :)| & Swap rows \verb|i, j|\\
\end{tabular}

\colbreak
\subsection{Matrix Operations}
\begin{tabular}{ll}
  \verb|ref(A)| & REF Form\\
  \verb|sref(A)| & Symbolic REF Form\\
  \verb|rref(A)| & RREF Form\\
  \verb|srref(A)| & Symbolic REF Form\\
  \verb|null(A, "rational")| & Nullspace basis vectors\\
  \verb|col(sym(A))| & $\col(A)$ basis vectors\\
  \verb|det(A)| & Determinant\\
  \verb|charpoly(A,x)| & Characteristic polynomial\\
  \verb|solve(charpoly(A,x)==0,x)| & Eigenvalues\\
  \verb|transpose(A)| & Transpose \\
  \verb|inv(A)| & Inverse \\
  \verb|orth(sym(A))| & Orthonormal basis from $A$ \\
  \verb|orth(sym(A), "skipnormalization")| & Orthogonal basis from $A$ \\
  \verb|gram(sym(A))| & Gram-Schmidt on $A$ \\
  \verb|norm(b)| & Norm of $\vec{b}$\\
  \verb|[L U] = lu(sym(A))| & LU Decomposition \\
  \verb|[L U] = luc(sym(A))| & LU Decomposition (Symbolic) \\
  \verb|[P D] = eig(sym(A))| & Diagonalization \\
  \verb|[Q D] = ortheig(sym(A))| & Orthogonal Diagonalization \\
  \verb|[Q R] = qr(sym(A), 0)| & QR Factorisation \\
  \verb|linsolve(A, b)| & Least-squares solution \\
  \verb|[U S V] = svd(sym(A))| & SVD \\
\end{tabular}

%%%%%%%%%%%%%%%%%%%%%%%%%%%%%%%%%%%%%%%%%%%%%%%%%%%%%%
%                       End                          %
%%%%%%%%%%%%%%%%%%%%%%%%%%%%%%%%%%%%%%%%%%%%%%%%%%%%%%

\end{multicols*}
\end{document}


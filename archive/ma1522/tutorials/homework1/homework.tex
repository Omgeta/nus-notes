\documentclass[12pt, a4paper]{article}

\usepackage[a4paper, margin=1in]{geometry}

\usepackage{fancyhdr}
\pagestyle{fancy}
\fancyhf{}
\fancyhead[R]{\thepage}
\renewcommand{\headrulewidth}{0pt}

\usepackage[utf8]{inputenc}
\usepackage[mathscr]{euscript}
\let\euscr\mathscr \let\mathscr\relax
\usepackage[scr]{rsfso}
\usepackage{amssymb,amsmath,amsthm,amsfonts}
\usepackage[shortlabels]{enumitem}
\usepackage{multicol,multirow}
\usepackage{lipsum}
\usepackage{balance}
\usepackage{calc}
\usepackage[colorlinks=true,citecolor=blue,linkcolor=blue]{hyperref}
\usepackage{import}
\usepackage{xifthen}
\usepackage{pdfpages}
\usepackage{transparent}
\usepackage{tabularx}

\newcommand{\incfig}[2][1.0]{
    \def\svgwidth{#1\columnwidth}
    \import{./figures/}{#2.pdf_tex}
}
\newcommand{\incimg}[2][1.0]{
  \includegraphics[width=#1\columnwidth]{./figures/#2}
}


\input{letterfonts}

\newcommand{\mytitle}{MA1522 Homework 1}
\newcommand{\myauthor}{github/omgeta}
\newcommand{\mydate}{AY 24/25 Sem 1}

\begin{document}
\raggedright
\footnotesize
\begin{center}
{\normalsize{\textbf{\mytitle}}} \\
{\footnotesize{\mydate\hspace{2pt}\textemdash\hspace{2pt}\myauthor}}
\end{center}
\setlist{topsep=-1em, itemsep=-1em, parsep=2em}

%%%%%%%%%%%%%%%%%%%%%%%%%%%%%%%%%%%%%%%%%%%%%%%%%%%%%%
%                      Begin                         %
%%%%%%%%%%%%%%%%%%%%%%%%%%%%%%%%%%%%%%%%%%%%%%%%%%%%%%
\begin{enumerate}[Q\arabic*.]
  \item 
    \begin{enumerate}[(\alph*)]
      \item \begin{align*}
          2x + 3y + 4z &= 400\qed\tag*{(i)}\\
          1x + 2y + 1z &= 200\qed\tag*{(ii)}\\
          2y + 4z &= 160\qed\tag*{(iii)}\\
        \end{align*}

      \item Form and reduce the corresponding augmented matrix for the system of equations:
        \begin{align*}
          \begin{bmatrix}
            2&3&4&400\\
            1&2&1&200\\
            0&2&4&160
          \end{bmatrix}\xrightarrow{R_1\leftrightarrow R_2}
          &\begin{bmatrix}
            1&2&1&200\\
            2&3&4&400\\
            0&2&4&160
          \end{bmatrix}\\\xrightarrow{R_2-2R_1}
          &\begin{bmatrix}
            1&2&1&200\\
            0&-1&2&0\\
            0&2&4&160
          \end{bmatrix}\\\xrightarrow{R_1+2R_2}
          &\begin{bmatrix}
            1&0&5&200\\
            0&-1&2&0\\
            0&2&4&160
          \end{bmatrix}\\\xrightarrow{R_3+2R_2}
          &\begin{bmatrix}
            1&0&5&200\\
            0&-1&2&0\\
            0&0&8&160
          \end{bmatrix}\\\xrightarrow{\frac{1}{8}R_3}
          &\begin{bmatrix}
            1&0&5&200\\
            0&-1&2&0\\
            0&0&1&20
          \end{bmatrix}\\\xrightarrow{R_2-2R_3}
          &\begin{bmatrix}
            1&0&5&200\\
            0&-1&0&-40\\
            0&0&1&20
          \end{bmatrix}\\\xrightarrow{-R_2}
          &\begin{bmatrix}
            1&0&5&200\\
            0&1&0&40\\
            0&0&1&20
          \end{bmatrix}\\\xrightarrow{R_1-5R_3}
          &\begin{bmatrix}
            1&0&0&100\\
            0&1&0&40\\
            0&0&1&20
          \end{bmatrix}
        \end{align*}
      Hence, we find $x = 100, y = 40, z = 20$. Therefore, pumps of type X pump $100$ litres/sec, pumps of type Y pump $40$ litres/sec, and pumps of type Z pump $20$ litres/sec. $\qed$
    \end{enumerate}
    \pagebreak

  \item Reduce the corresponding augmented matrix:
    \begin{align*}
      \begin{bmatrix}
        a & 2 & a & (a+b) & (a-b)\\
        a & 2 & a & a & (a-b)\\
        3 & 3 & -b & 3 & -b\\
        (a+1) & 3 & (a+1) & (a+1) & (a-b+1)\\
      \end{bmatrix}
      &\xrightarrow{R_1-R_2}
      \begin{bmatrix}
        0 & 0 & 0 & b & 0\\
        a & 2 & a & a & (a-b)\\
        3 & 3 & -b & 3 & -b\\
        (a+1) & 3 & (a+1) & (a+1) & (a-b+1)\\
      \end{bmatrix}\\
      \xrightarrow{R_4-R_2}
      \begin{bmatrix}
        0 & 0 & 0 & b & 0\\
        a & 2 & a & a & (a-b)\\
        3 & 3 & -b & 3 & -b\\
        1 & 1 & 1 & 1 & 1\\
      \end{bmatrix}
      &\xrightarrow{R_3-3R_4}
      \begin{bmatrix}
        0 & 0 & 0 & b & 0\\
        a & 2 & a & a & (a-b)\\
        0 & 0 & (-b-3) & 0 & (-b-3)\\
        1 & 1 & 1 & 1 & 1\\
      \end{bmatrix}\\
      \xrightarrow{R_2-aR_4}
      \begin{bmatrix}
        0 & 0 & 0 & b & 0\\
        0 & (2-a) & 0 & 0 & -b\\
        0 & 0 & (-b-3) & 0 & (-b-3)\\
        1 & 1 & 1 & 1 & 1\\
      \end{bmatrix}
      &\xrightarrow{R_1\leftrightarrow R_4}
      \begin{bmatrix}
        1 & 1 & 1 & 1 & 1\\
        0 & (2-a) & 0 & 0 & -b\\
        0 & 0 & (-b-3) & 0 & (-b-3)\\
        0 & 0 & 0 & b & 0\\
      \end{bmatrix}\\
    \end{align*}

    \begin{enumerate}[(\alph*)]
      \item No solution: $a = 2 \land b \neq 0$. (Row 2 will have inconsistent equation $0 \neq 0$)$\qed$
      \item Unique solution: $a \neq 2 \land b \neq -3 \land b \neq 0$. (RREF has pivot in every row)$\qed$\\
        \begin{align*}
          \xrightarrow{\frac{1}{b}R_4}
          \begin{bmatrix}
            1 & 1 & 1 & 1 & 1\\
            0 & (2-a) & 0 & 0 & -b\\
            0 & 0 & (-b-3) & 0 & (-b-3)\\
            0 & 0 & 0 & 1 & 0\\
          \end{bmatrix}
          &\xrightarrow{R_1-R_4}
          \begin{bmatrix}
            1 & 1 & 1 & 0 & 1\\
            0 & (2-a) & 0 & 0 & -b\\
            0 & 0 & (-b-3) & 0 & (-b-3)\\
            0 & 0 & 0 & 1 & 0\\
          \end{bmatrix}\\
          \xrightarrow{\frac{1}{-b-3}R_3}
          \begin{bmatrix}
            1 & 1 & 1 & 0 & 1\\
            0 & (2-a) & 0 & 0 & -b\\
            0 & 0 & 1 & 0 & 1\\
            0 & 0 & 0 & 1 & 0\\
          \end{bmatrix}
          &\xrightarrow{R_1-3R_3}
          \begin{bmatrix}
            1 & 1 & 0 & 0 & 0\\
            0 & (2-a) & 0 & 0 & -b\\
            0 & 0 & 1 & 0 & 1\\
            0 & 0 & 0 & 1 & 0\\
          \end{bmatrix}\\
          \xrightarrow{\frac{1}{2-a}R_2}
          \begin{bmatrix}
            1 & 1 & 0 & 0 & 0\\
            0 & 1 & 0 & 0 & -\frac{b}{2-a}\\
            0 & 0 & 1 & 0 & 1\\
            0 & 0 & 0 & 1 & 0\\
          \end{bmatrix}
          &\xrightarrow{R_1-R_2}
          \begin{bmatrix}
            1 & 0 & 0 & 0 & \frac{b}{2-a}\\
            0 & 1 & 0 & 0 & -\frac{b}{2-a}\\
            0 & 0 & 1 & 0 & 1\\
            0 & 0 & 0 & 1 & 0\\
          \end{bmatrix}
        \end{align*}
        Hence, the unique solution will be $x_1 = -\frac{b}{a-2}, x_2 = \frac{b}{a-2}, x_3 = 1, x_4 = 0 \qed$
      \item Infinite solutions w/ 1 parameter: $a \neq 2 \land (b=-3 \lor b = 0)$. (There will be exactly one zero row $\implies$ there will be exactly 1 free variable)$\qed$  

        Suppose, there are infinitely many solutions and $x_3 = 1, x_4 = 0$:\\
        \textbf{Case 1:} $a\neq_2, b=-3$:
        \begin{align*}
          \xrightarrow{\frac{1}{2-a}R_2}
          \begin{bmatrix}
            1&1&1&1&1\\
            0&1&0&0&\frac{3}{2-a}\\
            0&0&0&0&0\\
            0&0&0&-3&0
          \end{bmatrix}
          &\xrightarrow{R_1-R_2}
          \begin{bmatrix}
            1&0&1&1&1-\frac{3}{2-a}\\
            0&1&0&0&\frac{3}{2-a}\\
            0&0&0&0&0\\
            0&0&0&-3&0
          \end{bmatrix}\\
          \xrightarrow{-\frac{1}{b}R_4}
          \begin{bmatrix}
            1&0&1&1&1-\frac{3}{2-a}\\
            0&1&0&0&\frac{3}{2-a}\\
            0&0&0&0&0\\
            0&0&0&1&0
          \end{bmatrix}
          &\xrightarrow{R_1-R_4}
          \begin{bmatrix}
            1&0&1&0&1-\frac{3}{2-a}\\
            0&1&0&0&\frac{3}{2-a}\\
            0&0&0&0&0\\
            0&0&0&1&0
          \end{bmatrix}\\
        \end{align*}
        Then, $x_1 = 1 - \frac{3}{2-a} - x_3 = \frac{3}{a-2} \qed$  

        \textbf{Case 2:} $a\neq_2, b=0$:
        \begin{align*}
          \xrightarrow{\frac{1}{2-a}R_2}
          \begin{bmatrix}
            1&1&1&1&1\\
            0&1&0&0&0\\
            0&0&-3&0&-3\\
            0&0&0&0&0
          \end{bmatrix}
          &\xrightarrow{R_1-R_2}
          \begin{bmatrix}
            1&0&1&1&1\\
            0&1&0&0&0\\
            0&0&-3&0&-3\\
            0&0&0&0&0
          \end{bmatrix}\\
          \xrightarrow{-\frac{1}{3}R_3}
          \begin{bmatrix}
            1&0&1&1&1\\
            0&1&0&0&0\\
            0&0&1&0&1\\
            0&0&0&0&0
          \end{bmatrix}
          &\xrightarrow{R_1-R_4}
          \begin{bmatrix}
            1&0&0&1&0\\
            0&1&0&0&0\\
            0&0&1&0&1\\
            0&0&0&0&0
          \end{bmatrix}\\
        \end{align*}
        Then, $x_1 = 0 - x_4 = 0 \qed$  
      \item It is not possible to have infinite solutions w/ 3 parameters. \\Firstly, we need to ensure the system is not inconsistent (when $a=2, b\neq 0$). \\Note that $x_1$ can never be free since it is not dependent on any variables $a, b$. \\Then, we can set $x_2$ free (when $a = 2 \land b=0$), $x_3$ free (when $b=-3$), or $x_4$ free (when $b=0$). \\
        However, since $a, b$ can only take one value at a time, we can at most satisfy 2 of the conditions simultaneously (when $a=2 \land b=0$), allowing at most 2 parameters $x_2, x_4$. $\qed$ 
    \end{enumerate}
    \pagebreak

  \item \begin{enumerate}[(\alph*)]
      \item Elementary row operations $A\xrightarrow{R_3+R_1}\xrightarrow{R_4-R_2}\xrightarrow{R_2+5R_1}U$ can also be expressed in terms with matrix multiplication of $A$ by elementary row matrices such as:
      \begin{align*}
        E_3E_2E_1A &= U \\
        E_3 = \begin{bmatrix}1&0&0&0\\5&1&0&0\\0&0&1&0\\0&0&0&1\end{bmatrix}
        ,\quad E_2 = &\begin{bmatrix}1&0&0&0\\0&1&0&0\\0&0&1&0\\0&-1&0&1\end{bmatrix}
        ,\quad E_1 = \begin{bmatrix}1&0&0&0\\0&1&0&0\\1&0&1&0\\0&0&0&1\end{bmatrix}
      \end{align*}
      Suppose there exists the LU factorisation, $A = LU$:
      \begin{align*}
        E_3E_2E_1LU &= U\tag*{(Substitute $A=LU$)}\\
        LU &= E_1^{-1}E_2^{-1}E_3^{-1}U\tag*{(Definition of inverse $EE^{-1} = E^{-1}E = I$)}\\ 
        L &= E_1^{-1}E_2^{-1}E_3^{-1} \\
          &= \begin{bmatrix}1&0&0&0\\0&1&0&0\\-1&0&1&0\\0&0&0&1\end{bmatrix} \begin{bmatrix}1&0&0&0\\0&1&0&0\\0&0&1&0\\0&1&0&1\end{bmatrix}  \begin{bmatrix}1&0&0&0\\-5&1&0&0\\0&0&1&0\\0&0&0&1\end{bmatrix}\\
          &= \begin{bmatrix}1&0&0&0\\-5&1&0&0\\-1&0&1&0\\-5&1&0&1\end{bmatrix} \qed
      \end{align*}

    \item To solve $A \vec{x} = LU \vec{x} = \begin{bmatrix}2\\8\\-2\\5\end{bmatrix}$, let $U \vec{x} = \vec{y}$ and solve $L \vec{y} = \begin{bmatrix}2\\8\\-2\\5\end{bmatrix}$:
      \begin{align*}
        \begin{bmatrix}
          1&0&0&0&2\\
          -5&1&0&0&8\\
          -1&0&1&0&-2\\
          -5&1&0&1&5
        \end{bmatrix} &\xrightarrow{RREF}
        \begin{bmatrix}
          1&0&0&0&2\\
          0&1&0&0&18\\
          0&0&1&0&0\\
          0&0&0&1&-2
        \end{bmatrix}\\
        \therefore \vec{y} &= \begin{bmatrix}2\\18\\0\\-2\end{bmatrix}
      \end{align*}
      Then solve $U \vec{x} =  \begin{bmatrix}2\\18\\0\\-2\end{bmatrix}$:
      \begin{align*}
        \begin{bmatrix}
          -1&2&-2&4&2\\
          0&15&-7&24&18\\
          0&0&1&-3&0\\
          0&0&0&-3&-2
        \end{bmatrix} &\xrightarrow{RREF}
        \begin{bmatrix}
          1&0&0&0&-2\\
          0&1&0&0&1\\
          0&0&1&0&3\\
          0&0&0&1&1
        \end{bmatrix}\\
        \therefore \vec{x} &= \begin{bmatrix}-2\\1\\3\\1\end{bmatrix}\qed
      \end{align*}

      \item By using properties of the determinant:
        \begin{align*}
          \det(A) &= \det(L) \cdot \det(U)\tag*{(Lay T3.6 Multiplicative property)}\\
                  &= (1 \cdot 1 \cdot 1 \cdot 1) \cdot (-1 \cdot 15 \cdot 1 \cdot -3)\tag*{(Lay T3.2 Determinant of triangular matrix)}\\
                  &= 1 \cdot 45\\
                  &= 45 \qed
        \end{align*}
    \end{enumerate}
    \pagebreak

  \item \begin{enumerate}[(\alph*)]
      \item \begin{enumerate}[(\roman*)]
          \item By the Invertible Matrix Theorem, $A$ is invertible $\iff \det(A) \neq 0$, so we find the values of $a$ for which $\det(A) \neq 0$:
            \begin{align*}
              \det(A) &\neq 0\\
              \begin{vmatrix}
                a&a&a\\
                1&1&0\\
                0&1&1
              \end{vmatrix} &\neq 0\\
              a \begin{vmatrix}1&0\\1&1\end{vmatrix} -  a \begin{vmatrix}1&0\\0&1\end{vmatrix} +  a \begin{vmatrix}1&1\\0&1\end{vmatrix} &\neq 0\tag*{\text{(Lay T3.1 Cofactor expansion)}}\\
              a(1-0) - a(1-0) + a(1-0) &\neq 0\tag*{\text{(Determinant of 2$\times$2 matrices)}}\\
              a &\neq 0
            \end{align*}
            Therefore, for $A$ to be invertible, $a \neq 0. \qed$

          \item Suppose $C_{ij}$ is the $(i, j)$ cofactor of $A$, and $M_{ij}$ is the $(i, j)$ matrix minor of $A$ obtained by deletion of the $i$th row and $j$th column. $C_{ij}$ is given by:
            \begin{align*}
              C_{ij} &= (-1)^{i+j}\det(M_{ij})\\
            \end{align*}
          First, find the cofactors of $A$, finding the determinant of each $M_{ij}$ by definition of determinant for $2\times2$ matrices:
            \begin{align*}
              C_{11} = +\begin{vmatrix}1&0\\1&1\end{vmatrix} = 1               ,\quad C_{21} &= -\begin{vmatrix}a&a\\1&1\end{vmatrix} = 0
              ,\quad C_{31} = +\begin{vmatrix}a&a\\1&0\end{vmatrix} = -a\\
              C_{12} = -\begin{vmatrix}1&0\\0&1\end{vmatrix} = -1
              ,\quad C_{22} &= +\begin{vmatrix}a&a\\0&1\end{vmatrix} = a
              ,\quad C_{32} = -\begin{vmatrix}a&a\\1&0\end{vmatrix} = a\\
              C_{13} = +\begin{vmatrix}1&1\\0&1\end{vmatrix} = 1
              ,\quad C_{23} &= -\begin{vmatrix}a&a\\0&1\end{vmatrix} = -a
              ,\quad C_{33} = +\begin{vmatrix}a&a\\1&1\end{vmatrix} = 0\\
            \end{align*}
            Then, $\adj(A)$ is then given by:
            \begin{align*}
              \adj(A) &= (C_{ij})^T\\
                      &= \begin{bmatrix}1&0&-a\\-1&a&a\\1&-a&0\end{bmatrix}\qed
            \end{align*}
      
      \item Suppose $A$ is invertible:
        \begin{align*}
          A^{-1} &= \frac{1}{\det(A)}\adj(A)\tag*{(Lay T3.8 Adjoint Formula for Inverse)}\\
                 &= \frac{1}{a}\begin{bmatrix}1&0&-a\\-1&a&a\\1&-a&0\end{bmatrix}\tag*{(From (i) and (ii))}\qed
        \end{align*}
        \end{enumerate}

      \item Suppose there is some matrix $cA$, where $c \in \RR$:
        \begin{align*}
          (cA)^{-1} &= \frac{1}{\det(cA)}\adj(cA)\tag*{(Lay T3.8 Adjoint Formula for Inverse)}\\
          \frac{1}{c}A^{-1} &= \frac{1}{\det(cA)}\adj(cA) \tag*{\text{(Chapter 2 Slide 87, $(aA)^{-1} = \frac{1}{a}A^{-1}$)}}\\
          \frac{1}{c} \cdot \frac{1}{\det(A)}\adj(A) &= \frac{1}{\det(cA)}\adj(cA) \tag*{\text{(Substituting in $A^{-1}$)}}\\
                                                     &= \frac{1}{c^n\det(A)}\adj(cA)\tag*{(Chapter 2 Slide 158, $\det(cA) = c^n\det(A)$)}\\
                                                     &= \frac{1}{c^n} \cdot \frac{1}{\det(A)}\adj(cA)\\
          \adj(A) &= \frac{1}{c^{n-1}} \adj(cA) \tag*{\text{(Cancelling common terms)}}\\
          \therefore \adj(cA) &= c^{n-1}\adj(A)
        \end{align*}
        Hence, for $\adj(A) = \begin{bmatrix}1&1&1&0\\1&1&0&1\\1&0&1&1\\0&1&1&1\end{bmatrix}$, $\adj(3A)$ is given by:
        \begin{align*}
          \adj(3A) &= 3^3 \cdot \adj(A)\\
                   &= 27\cdot \adj(A)\\
                   &= \begin{bmatrix}27&27&27&0\\27&27&0&27\\27&0&27&27\\0&27&27&27\end{bmatrix} \qed
        \end{align*}
    \end{enumerate}
\end{enumerate}
%%%%%%%%%%%%%%%%%%%%%%%%%%%%%%%%%%%%%%%%%%%%%%%%%%%%%%
%                       End                          %
%%%%%%%%%%%%%%%%%%%%%%%%%%%%%%%%%%%%%%%%%%%%%%%%%%%%%%

\end{document}

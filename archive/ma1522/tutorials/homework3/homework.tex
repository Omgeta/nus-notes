\documentclass[12pt, a4paper]{article}

\usepackage[a4paper, margin=1in]{geometry}

\usepackage{fancyhdr}
\pagestyle{fancy}
\fancyhf{}
\fancyhead[R]{\thepage}
\renewcommand{\headrulewidth}{0pt}

\usepackage[utf8]{inputenc}
\usepackage[mathscr]{euscript}
\let\euscr\mathscr \let\mathscr\relax
\usepackage[scr]{rsfso}
\usepackage{amssymb,amsmath,amsthm,amsfonts}
\usepackage[shortlabels]{enumitem}
\usepackage{multicol,multirow}
\usepackage{lipsum}
\usepackage{balance}
\usepackage{calc}
\usepackage[colorlinks=true,citecolor=blue,linkcolor=blue]{hyperref}
\usepackage{import}
\usepackage{xifthen}
\usepackage{pdfpages}
\usepackage{transparent}
\usepackage{tabularx}

\newcommand{\incfig}[2][1.0]{
    \def\svgwidth{#1\columnwidth}
    \import{./figures/}{#2.pdf_tex}
}
\newcommand{\incimg}[2][1.0]{
  \includegraphics[width=#1\columnwidth]{./figures/#2}
}


\input{letterfonts}

\newcommand{\mytitle}{MA1522 Homework 3}
\newcommand{\myauthor}{github/omgeta}
\newcommand{\mydate}{AY 24/25 Sem 1}

\begin{document}
\raggedright
\footnotesize
\begin{center}
{\normalsize{\textbf{\mytitle}}} \\
{\footnotesize{\mydate\hspace{2pt}\textemdash\hspace{2pt}\myauthor}}
\end{center}
\setlist{topsep=-1em, itemsep=-1em, parsep=2em}

%%%%%%%%%%%%%%%%%%%%%%%%%%%%%%%%%%%%%%%%%%%%%%%%%%%%%%
%                      Begin                         %
%%%%%%%%%%%%%%%%%%%%%%%%%%%%%%%%%%%%%%%%%%%%%%%%%%%%%%
\begin{enumerate}[Q\arabic*.]
  \item 
    \begin{enumerate}[(\alph*)]
      \item Using the information provided:
        \begin{align*}
          P &= \left(\begin{array}{ccc} 0 & 0.4 & 0.4\\ 0.3 & 0 & 0.6\\ 0.7 & 0.6 & 0 \end{array}\right) \qed
        \end{align*}

      \item Diagonalize stochastic matrix $P$:
        \begin{align*}
          P &= \left(\begin{array}{ccc} 0 & \frac{8}{11} & 2\\ -1 & \frac{9}{11} & -3\\ 1 & 1 & 1 \end{array}\right)\left(\begin{array}{ccc} -\frac{3}{5} & 0 & 0\\ 0 & 1 & 0\\ 0 & 0 & -\frac{2}{5} \end{array}\right)\left(\begin{array}{ccc} 0 & \frac{8}{11} & 2\\ -1 & \frac{9}{11} & -3\\ 1 & 1 & 1 \end{array}\right)^{-1}
        \end{align*}
        Then $P^n$ is given by:
        \begin{align*}
          P^n &= \left(\begin{array}{ccc} 0 & \frac{8}{11} & 2\\ -1 & \frac{9}{11} & -3\\ 1 & 1 & 1 \end{array}\right)\left(\begin{array}{ccc} -\frac{3}{5}^n & 0 & 0\\ 0 & 1^n & 0\\ 0 & 0 & -\frac{2}{5}^n \end{array}\right)\left(\begin{array}{ccc} 0 & \frac{8}{11} & 2\\ -1 & \frac{9}{11} & -3\\ 1 & 1 & 1 \end{array}\right)^{-1}
        \end{align*}
        Hence, we have steady state vector:
        \begin{align*}
          P^nx_0 = P^n\left(\begin{array}{c} a\\ b\\ c \end{array}\right) \xrightarrow[n\rightarrow\infty]{} x_{\infty} &= \left(\begin{array}{ccc} 0 & \frac{8}{11} & 2\\ -1 & \frac{9}{11} & -3\\ 1 & 1 & 1 \end{array}\right)\left(\begin{array}{ccc} 0 & 0 & 0\\ 0 & 1 & 0\\ 0 & 0 & 0 \end{array}\right)\left(\begin{array}{ccc} 0 & \frac{8}{11} & 2\\ -1 & \frac{9}{11} & -3\\ 1 & 1 & 1 \end{array}\right)^{-1}\left(\begin{array}{c} a\\ b\\ c \end{array}\right)\\
                                                                                                 &= \frac{1}{28}\left(\begin{array}{ccc} 8 & 8 & 8\\ 9 & 9 & 9\\ 11 & 11 & 11 \end{array}\right)\left(\begin{array}{c} a\\ b\\ c \end{array}\right)\\
                                                                                                 &= \frac{1}{28}\left(\begin{array}{c} 8\,a+8\,b+8\,c\\ 9\,a+9\,b+9\,c\\ 11\,a+11\,b+11\,c \end{array}\right)\\
                                                                                                 &= \frac{1}{28}\left(\begin{array}{c} 8\\ 9\\ 11 \end{array}\right)
        \end{align*}
        Therefore, we see from the steady state vector that in the long run: Ah Meng will visit gym C the most with probability $\displaystyle \frac{11}{28}$ and gym A the least with probability $\displaystyle \frac{8}{28}$. $\qed$
    \end{enumerate}

  \pagebreak

\item 
  \begin{enumerate}[(\alph*)]
    \item Suppose $M = \left(\begin{array}{cc} m_1   & m_2\\ m_3 & m_4 \end{array}\right)$, then by column-vector multiplication for the given equation:
      \begin{align*}
        a_n = m_1\cdot a_{n-1} + m_2\cdot a_{n} &\implies m_1 = 0 \text{ and } m_2 = 1\\
        a_{n+1} = m_3\cdot a_{n-1} + m_4\cdot a_{n} &\implies m_3 = 1 \text{ and } m_4 = 1\tag*{(Given recurrence relation)}
      \end{align*}
      Therefore $M = \left(\begin{array}{cc} 0 & 1\\ 1 & 1 \end{array}\right) \qed$

    \item Using repeated applications of the given relation with matrix $M$:
      \begin{align*}
        \left(\begin{array}{c} a_{n}\\ a_{n+1} \end{array}\right) &= M\left(\begin{array}{c} a_{n-1}\\ a_{n} \end{array}\right)\\ 
        \implies \left(\begin{array}{c} a_{21}\\ a_{22} \end{array}\right) &= M\left(\begin{array}{c} a_{20}\\ a_{21} \end{array}\right)\\ 
         &= M\left( M\left(\begin{array}{c} a_{19}\\ a_{20} \end{array}\right)\right) 
         &\vdots\\
         &= M^{21}\left(\begin{array}{c} a_{0}\\ a_{1} \end{array}\right)\\
         &= \left(\begin{array}{c} 10946\\ 17711 \end{array}\right)\\
      \end{align*}
      Therefore, $a_{22} = 17711 \qed$

    \item Use the characteristic equation to find the eigenvalues of $M$:
      \begin{align*}
        \det(M - \lambda I) &= 0\\
        \det\left(\begin{array}{cc} -\lambda & 1\\ 1 & 1-\lambda\end{array}\right) &= 0\\
        -\lambda(1-\lambda) - 1 &= 0\\
        \lambda^2 -\lambda - 1 &= 0\\
        \lambda &= \frac{1 \pm \sqrt{5}}{2}
      \end{align*}
      When $\lambda = \frac{1+\sqrt{5}}{2}$ for eigenvector $\vec{v_1}$, $\left(\begin{array}{cc} -\frac{1+\sqrt{5}}{2} & 1\\ 1 & \frac{1-\sqrt{5}}{2}\end{array}\right) \vec{v_1}= \vec{0}$:
      \begin{align*}
        \left(\begin{array}{ccc} -\frac{1+\sqrt{5}}{2} & 1 & 0\\ 1 & \frac{1-\sqrt{5}}{2} & 0 \end{array}\right) &\xrightarrow{RREF} \left(\begin{array}{ccc} 1 & \frac{1-\sqrt{5}}{2} & 0\\ 0 & 0 & 0 \end{array}\right)\\
        \therefore \vec{v_1} &= s\left(\begin{array}{c} -1+\sqrt{5}\\ 2 \end{array}\right), s\in\RR
      \end{align*}
      When $\lambda = \frac{1-\sqrt{5}}{2}$ for eigenvector $\vec{v_2}$, $\left(\begin{array}{cc} -\frac{1-\sqrt{5}}{2} & 1\\ 1 & \frac{1+\sqrt{5}}{2}\end{array}\right) \vec{v_2}= \vec{0}$:
      \begin{align*}
        \left(\begin{array}{ccc} -\frac{1-\sqrt{5}}{2} & 1 & 0\\ 1 & \frac{1+\sqrt{5}}{2} & 0 \end{array}\right) &\xrightarrow{RREF} \left(\begin{array}{ccc} 1 & \frac{1+\sqrt{5}}{2} & 0\\ 0 & 0 & 0 \end{array}\right)\\
        \therefore \vec{v_2} &= t\left(\begin{array}{c} -1-\sqrt{5}\\ 2 \end{array}\right), t\in\RR
      \end{align*}
      Then we can find the diagonalization:
      \begin{align*}
        M &= \left(\begin{array}{cc} -1+\sqrt{5} & -1-\sqrt{5}\\ 2 & 2\end{array}\right) \left(\begin{array}{cc} \frac{1+\sqrt{5}}{2} & 0\\ 0 & \frac{1-\sqrt{5}}{2}\end{array}\right)\left(\begin{array}{cc} -1+\sqrt{5} & -1-\sqrt{5}\\ 2 & 2\end{array}\right)^{-1} \qed 
      \end{align*}

    \item Using powers of the diagonalization of $M$:
      \begin{align*}
        \left(\begin{array}{c} a_{n-1}\\ a_n \end{array}\right) &= \left(\begin{array}{cc} -1+\sqrt{5} & -1-\sqrt{5}\\ 2 & 2\end{array}\right) \left(\begin{array}{cc} (\frac{1+\sqrt{5}}{2})^{n-1} & 0\\ 0 & (\frac{1-\sqrt{5}}{2})^{n-1}\end{array}\right)\left(\begin{array}{cc} -1+\sqrt{5} & -1-\sqrt{5}\\ 2 & 2\end{array}\right)^{-1}\left(\begin{array}{c} 0\\ 1 \end{array}\right)\\ 
                                                                &= \left(\begin{array}{cc} -1+\sqrt{5} & -1-\sqrt{5}\\ 2 & 2\end{array}\right) \left(\begin{array}{cc} (\frac{1+\sqrt{5}}{2})^{n-1} & 0\\ 0 & (\frac{1-\sqrt{5}}{2})^{n-1}\end{array}\right)\frac{1}{4\sqrt{5}} \left(\begin{array}{cc} 2 & 1+\sqrt{5}\\ -2 & -1+\sqrt{5}\end{array}\right)\left(\begin{array}{c} 0\\ 1 \end{array}\right)\\
                                                                &= \frac{1}{4\sqrt{5}}\left(\begin{array}{cc} -1+\sqrt{5} & -1-\sqrt{5}\\ 2 & 2\end{array}\right) \left(\begin{array}{cc} (\frac{1+\sqrt{5}}{2})^{n-1} & 0\\ 0 & (\frac{1-\sqrt{5}}{2})^{n-1}\end{array}\right)\left(\begin{array}{c} 1+\sqrt{5}\\ -1+\sqrt{5} \end{array}\right)\\
                                                                &= \frac{1}{4\sqrt{5}}\left(\begin{array}{cc} -1+\sqrt{5} & -1-\sqrt{5}\\ 2 & 2\end{array}\right) \left(\begin{array}{c} \frac{(1+\sqrt{5})^{n}}{2^{n-1}}\\ -\frac{(1-\sqrt{5})^n}{2^{n-1}} \end{array}\right)\\
        \therefore a_n &= \frac{1}{4\sqrt{5}}\left(2\frac{(1+\sqrt{5})^n}{2^{n-1}} - 2 \frac{(1-\sqrt{5})}{2^{n-1}}\right)\\
                       &= \frac{1}{\sqrt{5}}\left(\frac{(1+\sqrt{5})^n}{2^n} - \frac{(1-\sqrt{5})}{2^n}\right)\\
                       &= \frac{1}{\sqrt{5}}\left((\frac{1+\sqrt{5}}{2})^n - (\frac{1-\sqrt{5}}{2})^n\right)\qed\\
      \end{align*}

    \item
  \end{enumerate}
  \pagebreak

  \item
    \begin{enumerate}[(\alph*)]
      \item By Invertible Matrix Theorem, if $A$ has a zero row then $0$ must be an eigenvalue $\implies 0$ is the last singular value $\sigma_3$ $\qed$

      \item Compare eigenvalues of $A^TA$ with known eigenvalues:
        \begin{align*}
          \det(A^TA - \lambda I) &= 0\\
          \det\left(\begin{array}{ccc} 16+b^2-\lambda & 4a & 24\\ 4a & a^2-\lambda & 6a\\ 24 & 6a & 36-\lambda \end{array}\right) &= 0\\
          \lambda^3 - \lambda^2(a^2+b^2+52) + \lambda(a^2b^2 + 36b^2) &= 0\\
          (\lambda - 72)(\lambda - 20)\lambda &= 0\tag*{(Known)}\\
          \lambda^3 - 92\lambda^2 + 1440\lambda &= 0
        \end{align*}
        Which gives us two equations:
        \begin{align*}
          a^2+b^2+52 &= 92\tag*{(1)}\\
          a^2b^2 + 36b^2 &= 1440\tag*{(2)}
        \end{align*}
        Solving simultaneously and only including $a,b>0$:
        \begin{align*}
          a=2, b=6\qed
        \end{align*}

      \item Find the corresponding eigenvector for each singular value:
        \begin{align*}
          \lambda_1 = (6\sqrt{2})^2 &\implies \vec{v_1} = \left(\begin{array}{c} 4/3\\ 1/3\\ 1 \end{array}\right)\\
          \lambda_1 = (2\sqrt{5})^2 &\implies \vec{v_2} = \left(\begin{array}{c} -5/6\\ 1/3\\ 1 \end{array}\right)\\
          \lambda_1 = 0^2 &\implies \vec{v_3} = \left(\begin{array}{c} 0\\ -3\\ 1 \end{array}\right)
        \end{align*}
        Then form $V$ from the normalized unit eigenvectors:
        \begin{align*}
          V = \left(\begin{array}{ccc} 4/\sqrt{26} & -5 /\sqrt{65} & 0\\ 1 /\sqrt{26} & 2 /\sqrt{65} & -3 /\sqrt{10}\\ 3 /\sqrt{26} & 6 /\sqrt{65} & 1 /\sqrt{10} \end{array}\right) \qed
        \end{align*}

      \item For each $\vec{v_i}$ column unit eigenvector of $V$, find the corresponding left singular vector:
        \begin{align*}
          \sigma_1 = 6\sqrt{2} \implies u_1 &= \frac{1}{6\sqrt{2}}A \frac{1}{\sqrt{26}}\left(\begin{array}{c} 4\\ 1\\ 3 \end{array}\right) = \frac{1}{\sqrt{13}}\left(\begin{array}{c} 3\\ 2\\ 0 \end{array}\right)\\
          \sigma_2 = 2\sqrt{5} \implies u_2 &= \frac{1}{2\sqrt{5}}A \frac{1}{\sqrt{65}}\left(\begin{array}{c} -5\\ 2\\ 6 \end{array}\right) = \frac{1}{\sqrt{14}}\left(\begin{array}{c} 2\\ -3\\ 0 \end{array}\right)\\
          \sigma_3 = 0 \implies u_3 &= \left(\begin{array}{c} 0\\ 0\\ 1 \end{array}\right)\tag*{(By choosing orthogonal vector)}
        \end{align*}
        Therefore $U = \left(\begin{array}{ccc} 3/\sqrt{13} & 2/\sqrt{13} & 0\\ 2 /\sqrt{13} & -3 /\sqrt{13}  & 0\\ 0 & 0 & 1 \end{array}\right) \qed$
    \end{enumerate}

  \pagebreak

\item Check if the input vectors are linearly independent by reducing matrix $U$ which has columns as the input vectors:
  \begin{gather*}
    U = \left(\begin{array}{cccc} 1 & 0 & 0 & 0\\ 0 & 1 & 0 & 1\\ 1 & 1 & 1 & 0\\ 0 & 0 & 1 & 1 \end{array}\right)\xrightarrow{RREF} \left(\begin{array}{cccc} 1 & 0 & 0 & 0\\ 0 & 1 & 0 & 0\\ 0 & 0 & 1 & 0\\ 0 & 0 & 0 & 1 \end{array}\right)\\
    \implies U \text{ is linearly independent}\\
    \implies U \text{ is invertible}
  \end{gather*}
  Suppose $A$ is the standard matrix for $T$, and $V$ is the output vectors, then by the relation of $A, U, V$:
  \begin{align*}
    V &= AU\\
    A &= VU^{-1}\\
      &= \left(\begin{array}{cccc} 0 & -2 & 0 & -4\\ 0 & 1 & 0 & 1\\ 3 & 3 & 3 & 2\\ 3 & 6 & 3 & 11 \end{array}\right)\left(\begin{array}{cccc} 1 & 0 & 0 & 0\\ 0 & 1 & 0 & 1\\ 1 & 1 & 1 & 0\\ 0 & 0 & 1 & 1 \end{array}\right)^{-1}\\
      &= \left(\begin{array}{cccc} -1 & -3 & 1 & -1\\ 0 & 1 & 0 & 0\\ 1 & 1 & 2 & 1\\ 4 & 7 & -1 & 4 \end{array}\right)
  \end{align*}
  Then by using MATLAB, the corresponding eigenvector for eigenvalue $\lambda = 2$ is:
  \begin{align*}
    \vec{u} &= \left(\begin{array}{c} -1\\ 0\\ -2\\ 1 \end{array}\right) \qed
  \end{align*}
\end{enumerate}
%%%%%%%%%%%%%%%%%%%%%%%%%%%%%%%%%%%%%%%%%%%%%%%%%%%%%%
%                       End                          %
%%%%%%%%%%%%%%%%%%%%%%%%%%%%%%%%%%%%%%%%%%%%%%%%%%%%%%

\end{document}

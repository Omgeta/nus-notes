\documentclass[12pt, a4paper]{article}

\usepackage[a4paper, margin=1in]{geometry}

\usepackage{fancyhdr}
\pagestyle{fancy}
\fancyhf{}
\fancyhead[R]{\thepage}
\renewcommand{\headrulewidth}{0pt}

\usepackage[utf8]{inputenc}
\usepackage[mathscr]{euscript}
\let\euscr\mathscr \let\mathscr\relax
\usepackage[scr]{rsfso}
\usepackage{amssymb,amsmath,amsthm,amsfonts}
\usepackage[shortlabels]{enumitem}
\usepackage{multicol,multirow}
\usepackage{lipsum}
\usepackage{balance}
\usepackage{calc}
\usepackage[colorlinks=true,citecolor=blue,linkcolor=blue]{hyperref}
\usepackage{import}
\usepackage{xifthen}
\usepackage{pdfpages}
\usepackage{transparent}
\usepackage{listings}

\newcommand{\incfig}[2][1.0]{
    \def\svgwidth{#1\columnwidth}
    \import{./figures/}{#2.pdf_tex}
}

\newlist{enumproof}{enumerate}{4}
\setlist[enumproof,1]{label=\arabic*., parsep=1em}
\setlist[enumproof,2]{label=\arabic{enumproofi}.\arabic*., parsep=1em}
\setlist[enumproof,3]{label=\arabic{enumproofi}.\arabic{enumproofii}.\arabic*., parsep=1em}
\setlist[enumproof,4]{label=\arabic{enumproofi}.\arabic{enumproofii}.\arabic{enumproofiii}.\arabic*., parsep=1em}

\renewcommand{\qedsymbol}{\ensuremath{\blacksquare}}

\lstdefinestyle{mystyle}{
  language=C, % Set the language to C
  commentstyle=\color{codegray}, % Color for comments
  keywordstyle=\color{orange}, % Color for basic keywords
  stringstyle=\color{mauve}, % Color for strings
  basicstyle={\ttfamily\footnotesize}, % Basic font style
  breakatwhitespace=false,         
  breaklines=true,                 
  captionpos=b,                    
  keepspaces=true,                 
  numbers=none,                    
  tabsize=2,
  morekeywords=[2]{\#include, \#define, \#ifdef, \#ifndef, \#endif, \#pragma, \#else, \#elif}, % Preprocessor directives
  keywordstyle=[2]\color{codegreen}, % Style for preprocessor directives
  morekeywords=[3]{int, char, float, double, void, struct, union, enum, const, volatile, static, extern, register, inline, restrict, _Bool, _Complex, _Imaginary, size_t, ssize_t, FILE}, % C standard types and common identifiers
  keywordstyle=[3]\color{identblue}, % Style for types and common identifiers
  morekeywords=[4]{printf, scanf, fopen, fclose, malloc, free, calloc, realloc, perror, strtok, strncpy, strcpy, strcmp, strlen}, % Standard library functions
  keywordstyle=[4]\color{cyan}, % Style for library functions
}

% Things Lie
\newcommand{\kb}{\mathfrak b}
\newcommand{\kg}{\mathfrak g}
\newcommand{\kh}{\mathfrak h}
\newcommand{\kn}{\mathfrak n}
\newcommand{\ku}{\mathfrak u}
\newcommand{\kz}{\mathfrak z}
\DeclareMathOperator{\Ext}{Ext} % Ext functor
\DeclareMathOperator{\Tor}{Tor} % Tor functor
\newcommand{\gl}{\opname{\mathfrak{gl}}} % frak gl group
\renewcommand{\sl}{\opname{\mathfrak{sl}}} % frak sl group chktex 6

% More script letters etc.
\newcommand{\SA}{\mathcal A}
\newcommand{\SB}{\mathcal B}
\newcommand{\SC}{\mathcal C}
\newcommand{\SF}{\mathcal F}
\newcommand{\SG}{\mathcal G}
\newcommand{\SH}{\mathcal H}
\newcommand{\OO}{\mathcal O}

\newcommand{\SCA}{\mathscr A}
\newcommand{\SCB}{\mathscr B}
\newcommand{\SCC}{\mathscr C}
\newcommand{\SCD}{\mathscr D}
\newcommand{\SCE}{\mathscr E}
\newcommand{\SCF}{\mathscr F}
\newcommand{\SCG}{\mathscr G}
\newcommand{\SCH}{\mathscr H}

% Mathfrak primes
\newcommand{\km}{\mathfrak m}
\newcommand{\kp}{\mathfrak p}
\newcommand{\kq}{\mathfrak q}

% number sets
\newcommand{\RR}[1][]{\ensuremath{\ifstrempty{#1}{\mathbb{R}}{\mathbb{R}^{#1}}}}
\newcommand{\NN}[1][]{\ensuremath{\ifstrempty{#1}{\mathbb{N}}{\mathbb{N}^{#1}}}}
\newcommand{\ZZ}[1][]{\ensuremath{\ifstrempty{#1}{\mathbb{Z}}{\mathbb{Z}^{#1}}}}
\newcommand{\QQ}[1][]{\ensuremath{\ifstrempty{#1}{\mathbb{Q}}{\mathbb{Q}^{#1}}}}
\newcommand{\CC}[1][]{\ensuremath{\ifstrempty{#1}{\mathbb{C}}{\mathbb{C}^{#1}}}}
\newcommand{\PP}[1][]{\ensuremath{\ifstrempty{#1}{\mathbb{P}}{\mathbb{P}^{#1}}}}
\newcommand{\HH}[1][]{\ensuremath{\ifstrempty{#1}{\mathbb{H}}{\mathbb{H}^{#1}}}}
\newcommand{\FF}[1][]{\ensuremath{\ifstrempty{#1}{\mathbb{F}}{\mathbb{F}^{#1}}}}
% expected value
\newcommand{\EE}{\ensuremath{\mathbb{E}}}
\newcommand{\charin}{\text{ char }}
\DeclareMathOperator{\sign}{sign}
\DeclareMathOperator{\Aut}{Aut}
\DeclareMathOperator{\Inn}{Inn}
\DeclareMathOperator{\Syl}{Syl}
\DeclareMathOperator{\Gal}{Gal}
\DeclareMathOperator{\GL}{GL} % General linear group
\DeclareMathOperator{\SL}{SL} % Special linear group

%---------------------------------------
% BlackBoard Math Fonts :-
%---------------------------------------

%Captital Letters
\newcommand{\bbA}{\mathbb{A}}	\newcommand{\bbB}{\mathbb{B}}
\newcommand{\bbC}{\mathbb{C}}	\newcommand{\bbD}{\mathbb{D}}
\newcommand{\bbE}{\mathbb{E}}	\newcommand{\bbF}{\mathbb{F}}
\newcommand{\bbG}{\mathbb{G}}	\newcommand{\bbH}{\mathbb{H}}
\newcommand{\bbI}{\mathbb{I}}	\newcommand{\bbJ}{\mathbb{J}}
\newcommand{\bbK}{\mathbb{K}}	\newcommand{\bbL}{\mathbb{L}}
\newcommand{\bbM}{\mathbb{M}}	\newcommand{\bbN}{\mathbb{N}}
\newcommand{\bbO}{\mathbb{O}}	\newcommand{\bbP}{\mathbb{P}}
\newcommand{\bbQ}{\mathbb{Q}}	\newcommand{\bbR}{\mathbb{R}}
\newcommand{\bbS}{\mathbb{S}}	\newcommand{\bbT}{\mathbb{T}}
\newcommand{\bbU}{\mathbb{U}}	\newcommand{\bbV}{\mathbb{V}}
\newcommand{\bbW}{\mathbb{W}}	\newcommand{\bbX}{\mathbb{X}}
\newcommand{\bbY}{\mathbb{Y}}	\newcommand{\bbZ}{\mathbb{Z}}

%---------------------------------------
% MathCal Fonts :-
%---------------------------------------

%Captital Letters
\newcommand{\mcA}{\mathcal{A}}	\newcommand{\mcB}{\mathcal{B}}
\newcommand{\mcC}{\mathcal{C}}	\newcommand{\mcD}{\mathcal{D}}
\newcommand{\mcE}{\mathcal{E}}	\newcommand{\mcF}{\mathcal{F}}
\newcommand{\mcG}{\mathcal{G}}	\newcommand{\mcH}{\mathcal{H}}
\newcommand{\mcI}{\mathcal{I}}	\newcommand{\mcJ}{\mathcal{J}}
\newcommand{\mcK}{\mathcal{K}}	\newcommand{\mcL}{\mathcal{L}}
\newcommand{\mcM}{\mathcal{M}}	\newcommand{\mcN}{\mathcal{N}}
\newcommand{\mcO}{\mathcal{O}}	\newcommand{\mcP}{\mathcal{P}}
\newcommand{\mcQ}{\mathcal{Q}}	\newcommand{\mcR}{\mathcal{R}}
\newcommand{\mcS}{\mathcal{S}}	\newcommand{\mcT}{\mathcal{T}}
\newcommand{\mcU}{\mathcal{U}}	\newcommand{\mcV}{\mathcal{V}}
\newcommand{\mcW}{\mathcal{W}}	\newcommand{\mcX}{\mathcal{X}}
\newcommand{\mcY}{\mathcal{Y}}	\newcommand{\mcZ}{\mathcal{Z}}

%---------------------------------------
% Bold Math Fonts :-
%---------------------------------------

%Captital Letters
\newcommand{\bmA}{\boldsymbol{A}}	\newcommand{\bmB}{\boldsymbol{B}}
\newcommand{\bmC}{\boldsymbol{C}}	\newcommand{\bmD}{\boldsymbol{D}}
\newcommand{\bmE}{\boldsymbol{E}}	\newcommand{\bmF}{\boldsymbol{F}}
\newcommand{\bmG}{\boldsymbol{G}}	\newcommand{\bmH}{\boldsymbol{H}}
\newcommand{\bmI}{\boldsymbol{I}}	\newcommand{\bmJ}{\boldsymbol{J}}
\newcommand{\bmK}{\boldsymbol{K}}	\newcommand{\bmL}{\boldsymbol{L}}
\newcommand{\bmM}{\boldsymbol{M}}	\newcommand{\bmN}{\boldsymbol{N}}
\newcommand{\bmO}{\boldsymbol{O}}	\newcommand{\bmP}{\boldsymbol{P}}
\newcommand{\bmQ}{\boldsymbol{Q}}	\newcommand{\bmR}{\boldsymbol{R}}
\newcommand{\bmS}{\boldsymbol{S}}	\newcommand{\bmT}{\boldsymbol{T}}
\newcommand{\bmU}{\boldsymbol{U}}	\newcommand{\bmV}{\boldsymbol{V}}
\newcommand{\bmW}{\boldsymbol{W}}	\newcommand{\bmX}{\boldsymbol{X}}
\newcommand{\bmY}{\boldsymbol{Y}}	\newcommand{\bmZ}{\boldsymbol{Z}}
%Small Letters
\newcommand{\bma}{\boldsymbol{a}}	\newcommand{\bmb}{\boldsymbol{b}}
\newcommand{\bmc}{\boldsymbol{c}}	\newcommand{\bmd}{\boldsymbol{d}}
\newcommand{\bme}{\boldsymbol{e}}	\newcommand{\bmf}{\boldsymbol{f}}
\newcommand{\bmg}{\boldsymbol{g}}	\newcommand{\bmh}{\boldsymbol{h}}
\newcommand{\bmi}{\boldsymbol{i}}	\newcommand{\bmj}{\boldsymbol{j}}
\newcommand{\bmk}{\boldsymbol{k}}	\newcommand{\bml}{\boldsymbol{l}}
\newcommand{\bmm}{\boldsymbol{m}}	\newcommand{\bmn}{\boldsymbol{n}}
\newcommand{\bmo}{\boldsymbol{o}}	\newcommand{\bmp}{\boldsymbol{p}}
\newcommand{\bmq}{\boldsymbol{q}}	\newcommand{\bmr}{\boldsymbol{r}}
\newcommand{\bms}{\boldsymbol{s}}	\newcommand{\bmt}{\boldsymbol{t}}
\newcommand{\bmu}{\boldsymbol{u}}	\newcommand{\bmv}{\boldsymbol{v}}
\newcommand{\bmw}{\boldsymbol{w}}	\newcommand{\bmx}{\boldsymbol{x}}
\newcommand{\bmy}{\boldsymbol{y}}	\newcommand{\bmz}{\boldsymbol{z}}

%---------------------------------------
% Scr Math Fonts :-
%---------------------------------------

\newcommand{\sA}{{\mathscr{A}}}   \newcommand{\sB}{{\mathscr{B}}}
\newcommand{\sC}{{\mathscr{C}}}   \newcommand{\sD}{{\mathscr{D}}}
\newcommand{\sE}{{\mathscr{E}}}   \newcommand{\sF}{{\mathscr{F}}}
\newcommand{\sG}{{\mathscr{G}}}   \newcommand{\sH}{{\mathscr{H}}}
\newcommand{\sI}{{\mathscr{I}}}   \newcommand{\sJ}{{\mathscr{J}}}
\newcommand{\sK}{{\mathscr{K}}}   \newcommand{\sL}{{\mathscr{L}}}
\newcommand{\sM}{{\mathscr{M}}}   \newcommand{\sN}{{\mathscr{N}}}
\newcommand{\sO}{{\mathscr{O}}}   \newcommand{\sP}{{\mathscr{P}}}
\newcommand{\sQ}{{\mathscr{Q}}}   \newcommand{\sR}{{\mathscr{R}}}
\newcommand{\sS}{{\mathscr{S}}}   \newcommand{\sT}{{\mathscr{T}}}
\newcommand{\sU}{{\mathscr{U}}}   \newcommand{\sV}{{\mathscr{V}}}
\newcommand{\sW}{{\mathscr{W}}}   \newcommand{\sX}{{\mathscr{X}}}
\newcommand{\sY}{{\mathscr{Y}}}   \newcommand{\sZ}{{\mathscr{Z}}}


%---------------------------------------
% Math Fraktur Font
%---------------------------------------

%Captital Letters
\newcommand{\mfA}{\mathfrak{A}}	\newcommand{\mfB}{\mathfrak{B}}
\newcommand{\mfC}{\mathfrak{C}}	\newcommand{\mfD}{\mathfrak{D}}
\newcommand{\mfE}{\mathfrak{E}}	\newcommand{\mfF}{\mathfrak{F}}
\newcommand{\mfG}{\mathfrak{G}}	\newcommand{\mfH}{\mathfrak{H}}
\newcommand{\mfI}{\mathfrak{I}}	\newcommand{\mfJ}{\mathfrak{J}}
\newcommand{\mfK}{\mathfrak{K}}	\newcommand{\mfL}{\mathfrak{L}}
\newcommand{\mfM}{\mathfrak{M}}	\newcommand{\mfN}{\mathfrak{N}}
\newcommand{\mfO}{\mathfrak{O}}	\newcommand{\mfP}{\mathfrak{P}}
\newcommand{\mfQ}{\mathfrak{Q}}	\newcommand{\mfR}{\mathfrak{R}}
\newcommand{\mfS}{\mathfrak{S}}	\newcommand{\mfT}{\mathfrak{T}}
\newcommand{\mfU}{\mathfrak{U}}	\newcommand{\mfV}{\mathfrak{V}}
\newcommand{\mfW}{\mathfrak{W}}	\newcommand{\mfX}{\mathfrak{X}}
\newcommand{\mfY}{\mathfrak{Y}}	\newcommand{\mfZ}{\mathfrak{Z}}
%Small Letters
\newcommand{\mfa}{\mathfrak{a}}	\newcommand{\mfb}{\mathfrak{b}}
\newcommand{\mfc}{\mathfrak{c}}	\newcommand{\mfd}{\mathfrak{d}}
\newcommand{\mfe}{\mathfrak{e}}	\newcommand{\mff}{\mathfrak{f}}
\newcommand{\mfg}{\mathfrak{g}}	\newcommand{\mfh}{\mathfrak{h}}
\newcommand{\mfi}{\mathfrak{i}}	\newcommand{\mfj}{\mathfrak{j}}
\newcommand{\mfk}{\mathfrak{k}}	\newcommand{\mfl}{\mathfrak{l}}
\newcommand{\mfm}{\mathfrak{m}}	\newcommand{\mfn}{\mathfrak{n}}
\newcommand{\mfo}{\mathfrak{o}}	\newcommand{\mfp}{\mathfrak{p}}
\newcommand{\mfq}{\mathfrak{q}}	\newcommand{\mfr}{\mathfrak{r}}
\newcommand{\mfs}{\mathfrak{s}}	\newcommand{\mft}{\mathfrak{t}}
\newcommand{\mfu}{\mathfrak{u}}	\newcommand{\mfv}{\mathfrak{v}}
\newcommand{\mfw}{\mathfrak{w}}	\newcommand{\mfx}{\mathfrak{x}}
\newcommand{\mfy}{\mathfrak{y}}	\newcommand{\mfz}{\mathfrak{z}}


\newcommand{\mytitle}{MA1522 Homework 3}
\newcommand{\myauthor}{github/omgeta}
\newcommand{\mydate}{AY 24/25 Sem 1}

\begin{document}
\raggedright
\footnotesize
\begin{center}
{\normalsize{\textbf{\mytitle}}} \\
{\footnotesize{\mydate\hspace{2pt}\textemdash\hspace{2pt}\myauthor}}
\end{center}
\setlist{topsep=-1em, itemsep=-1em, parsep=2em}

%%%%%%%%%%%%%%%%%%%%%%%%%%%%%%%%%%%%%%%%%%%%%%%%%%%%%%
%                      Begin                         %
%%%%%%%%%%%%%%%%%%%%%%%%%%%%%%%%%%%%%%%%%%%%%%%%%%%%%%
\begin{enumerate}[Q\arabic*.]
  \item 
    \begin{enumerate}[(\alph*)]
      \item Using the information provided:
        \begin{align*}
          P &= \left(\begin{array}{ccc} 0 & 0.4 & 0.4\\ 0.3 & 0 & 0.6\\ 0.7 & 0.6 & 0 \end{array}\right) \qed
        \end{align*}

      \item Diagonalize stochastic matrix $P$:
        \begin{align*}
          P &= \left(\begin{array}{ccc} 0 & \frac{8}{11} & 2\\ -1 & \frac{9}{11} & -3\\ 1 & 1 & 1 \end{array}\right)\left(\begin{array}{ccc} -\frac{3}{5} & 0 & 0\\ 0 & 1 & 0\\ 0 & 0 & -\frac{2}{5} \end{array}\right)\left(\begin{array}{ccc} 0 & \frac{8}{11} & 2\\ -1 & \frac{9}{11} & -3\\ 1 & 1 & 1 \end{array}\right)^{-1}
        \end{align*}
        Then $P^n$ is given by:
        \begin{align*}
          P^n &= \left(\begin{array}{ccc} 0 & \frac{8}{11} & 2\\ -1 & \frac{9}{11} & -3\\ 1 & 1 & 1 \end{array}\right)\left(\begin{array}{ccc} -\frac{3}{5}^n & 0 & 0\\ 0 & 1^n & 0\\ 0 & 0 & -\frac{2}{5}^n \end{array}\right)\left(\begin{array}{ccc} 0 & \frac{8}{11} & 2\\ -1 & \frac{9}{11} & -3\\ 1 & 1 & 1 \end{array}\right)^{-1}
        \end{align*}
        Hence, we have steady state vector:
        \begin{align*}
          P^nx_0 = P^n\left(\begin{array}{c} a\\ b\\ c \end{array}\right) \xrightarrow[n\rightarrow\infty]{} x_{\infty} &= \left(\begin{array}{ccc} 0 & \frac{8}{11} & 2\\ -1 & \frac{9}{11} & -3\\ 1 & 1 & 1 \end{array}\right)\left(\begin{array}{ccc} 0 & 0 & 0\\ 0 & 1 & 0\\ 0 & 0 & 0 \end{array}\right)\left(\begin{array}{ccc} 0 & \frac{8}{11} & 2\\ -1 & \frac{9}{11} & -3\\ 1 & 1 & 1 \end{array}\right)^{-1}\left(\begin{array}{c} a\\ b\\ c \end{array}\right)\\
                                                                                                 &= \frac{1}{28}\left(\begin{array}{ccc} 8 & 8 & 8\\ 9 & 9 & 9\\ 11 & 11 & 11 \end{array}\right)\left(\begin{array}{c} a\\ b\\ c \end{array}\right)\\
                                                                                                 &= \frac{1}{28}\left(\begin{array}{c} 8\,a+8\,b+8\,c\\ 9\,a+9\,b+9\,c\\ 11\,a+11\,b+11\,c \end{array}\right)\\
                                                                                                 &= \frac{1}{28}\left(\begin{array}{c} 8\\ 9\\ 11 \end{array}\right)
        \end{align*}
        Therefore, we see from the steady state vector that in the long run: Ah Meng will visit gym C the most with probability $\displaystyle \frac{11}{28}$ and gym A the least with probability $\displaystyle \frac{8}{28}$. $\qed$
    \end{enumerate}

  \pagebreak

\item 
  \begin{enumerate}[(\alph*)]
    \item Suppose $M = \left(\begin{array}{cc} m_1   & m_2\\ m_3 & m_4 \end{array}\right)$, then by column-vector multiplication for the given equation:
      \begin{align*}
        a_n = m_1\cdot a_{n-1} + m_2\cdot a_{n} &\implies m_1 = 0 \text{ and } m_2 = 1\\
        a_{n+1} = m_3\cdot a_{n-1} + m_4\cdot a_{n} &\implies m_3 = 1 \text{ and } m_4 = 1\tag*{(Given recurrence relation)}
      \end{align*}
      Therefore $M = \left(\begin{array}{cc} 0 & 1\\ 1 & 1 \end{array}\right) \qed$

    \item Using repeated applications of the given relation with matrix $M$:
      \begin{align*}
        \left(\begin{array}{c} a_{n}\\ a_{n+1} \end{array}\right) &= M\left(\begin{array}{c} a_{n-1}\\ a_{n} \end{array}\right)\\ 
        \implies \left(\begin{array}{c} a_{21}\\ a_{22} \end{array}\right) &= M\left(\begin{array}{c} a_{20}\\ a_{21} \end{array}\right)\\ 
         &= M\left( M\left(\begin{array}{c} a_{19}\\ a_{20} \end{array}\right)\right) 
         &\vdots\\
         &= M^{21}\left(\begin{array}{c} a_{0}\\ a_{1} \end{array}\right)\\
         &= \left(\begin{array}{c} 10946\\ 17711 \end{array}\right)\\
      \end{align*}
      Therefore, $a_{22} = 17711 \qed$

    \item Use the characteristic equation to find the eigenvalues of $M$:
      \begin{align*}
        \det(M - \lambda I) &= 0\\
        \det\left(\begin{array}{cc} -\lambda & 1\\ 1 & 1-\lambda\end{array}\right) &= 0\\
        -\lambda(1-\lambda) - 1 &= 0\\
        \lambda^2 -\lambda - 1 &= 0\\
        \lambda &= \frac{1 \pm \sqrt{5}}{2}
      \end{align*}
      When $\lambda = \frac{1+\sqrt{5}}{2}$ for eigenvector $\vec{v_1}$, $\left(\begin{array}{cc} -\frac{1+\sqrt{5}}{2} & 1\\ 1 & \frac{1-\sqrt{5}}{2}\end{array}\right) \vec{v_1}= \vec{0}$:
      \begin{align*}
        \left(\begin{array}{ccc} -\frac{1+\sqrt{5}}{2} & 1 & 0\\ 1 & \frac{1-\sqrt{5}}{2} & 0 \end{array}\right) &\xrightarrow{RREF} \left(\begin{array}{ccc} 1 & \frac{1-\sqrt{5}}{2} & 0\\ 0 & 0 & 0 \end{array}\right)\\
        \therefore \vec{v_1} &= s\left(\begin{array}{c} -1+\sqrt{5}\\ 2 \end{array}\right), s\in\RR
      \end{align*}
      When $\lambda = \frac{1-\sqrt{5}}{2}$ for eigenvector $\vec{v_2}$, $\left(\begin{array}{cc} -\frac{1-\sqrt{5}}{2} & 1\\ 1 & \frac{1+\sqrt{5}}{2}\end{array}\right) \vec{v_2}= \vec{0}$:
      \begin{align*}
        \left(\begin{array}{ccc} -\frac{1-\sqrt{5}}{2} & 1 & 0\\ 1 & \frac{1+\sqrt{5}}{2} & 0 \end{array}\right) &\xrightarrow{RREF} \left(\begin{array}{ccc} 1 & \frac{1+\sqrt{5}}{2} & 0\\ 0 & 0 & 0 \end{array}\right)\\
        \therefore \vec{v_2} &= t\left(\begin{array}{c} -1-\sqrt{5}\\ 2 \end{array}\right), t\in\RR
      \end{align*}
      Then we can find the diagonalization:
      \begin{align*}
        M &= \left(\begin{array}{cc} -1+\sqrt{5} & -1-\sqrt{5}\\ 2 & 2\end{array}\right) \left(\begin{array}{cc} \frac{1+\sqrt{5}}{2} & 0\\ 0 & \frac{1-\sqrt{5}}{2}\end{array}\right)\left(\begin{array}{cc} -1+\sqrt{5} & -1-\sqrt{5}\\ 2 & 2\end{array}\right)^{-1} \qed 
      \end{align*}

    \item Using powers of the diagonalization of $M$:
      \begin{align*}
        \left(\begin{array}{c} a_{n-1}\\ a_n \end{array}\right) &= \left(\begin{array}{cc} -1+\sqrt{5} & -1-\sqrt{5}\\ 2 & 2\end{array}\right) \left(\begin{array}{cc} (\frac{1+\sqrt{5}}{2})^{n-1} & 0\\ 0 & (\frac{1-\sqrt{5}}{2})^{n-1}\end{array}\right)\left(\begin{array}{cc} -1+\sqrt{5} & -1-\sqrt{5}\\ 2 & 2\end{array}\right)^{-1}\left(\begin{array}{c} 0\\ 1 \end{array}\right)\\ 
                                                                &= \left(\begin{array}{cc} -1+\sqrt{5} & -1-\sqrt{5}\\ 2 & 2\end{array}\right) \left(\begin{array}{cc} (\frac{1+\sqrt{5}}{2})^{n-1} & 0\\ 0 & (\frac{1-\sqrt{5}}{2})^{n-1}\end{array}\right)\frac{1}{4\sqrt{5}} \left(\begin{array}{cc} 2 & 1+\sqrt{5}\\ -2 & -1+\sqrt{5}\end{array}\right)\left(\begin{array}{c} 0\\ 1 \end{array}\right)\\
                                                                &= \frac{1}{4\sqrt{5}}\left(\begin{array}{cc} -1+\sqrt{5} & -1-\sqrt{5}\\ 2 & 2\end{array}\right) \left(\begin{array}{cc} (\frac{1+\sqrt{5}}{2})^{n-1} & 0\\ 0 & (\frac{1-\sqrt{5}}{2})^{n-1}\end{array}\right)\left(\begin{array}{c} 1+\sqrt{5}\\ -1+\sqrt{5} \end{array}\right)\\
                                                                &= \frac{1}{4\sqrt{5}}\left(\begin{array}{cc} -1+\sqrt{5} & -1-\sqrt{5}\\ 2 & 2\end{array}\right) \left(\begin{array}{c} \frac{(1+\sqrt{5})^{n}}{2^{n-1}}\\ -\frac{(1-\sqrt{5})^n}{2^{n-1}} \end{array}\right)\\
        \therefore a_n &= \frac{1}{4\sqrt{5}}\left(2\frac{(1+\sqrt{5})^n}{2^{n-1}} - 2 \frac{(1-\sqrt{5})}{2^{n-1}}\right)\\
                       &= \frac{1}{\sqrt{5}}\left(\frac{(1+\sqrt{5})^n}{2^n} - \frac{(1-\sqrt{5})}{2^n}\right)\\
                       &= \frac{1}{\sqrt{5}}\left((\frac{1+\sqrt{5}}{2})^n - (\frac{1-\sqrt{5}}{2})^n\right)\qed\\
      \end{align*}

    \item
  \end{enumerate}
  \pagebreak

  \item
    \begin{enumerate}[(\alph*)]
      \item By Invertible Matrix Theorem, if $A$ has a zero row then $0$ must be an eigenvalue $\implies 0$ is the last singular value $\sigma_3$ $\qed$

      \item Compare eigenvalues of $A^TA$ with known eigenvalues:
        \begin{align*}
          \det(A^TA - \lambda I) &= 0\\
          \det\left(\begin{array}{ccc} 16+b^2-\lambda & 4a & 24\\ 4a & a^2-\lambda & 6a\\ 24 & 6a & 36-\lambda \end{array}\right) &= 0\\
          \lambda^3 - \lambda^2(a^2+b^2+52) + \lambda(a^2b^2 + 36b^2) &= 0\\
          (\lambda - 72)(\lambda - 20)\lambda &= 0\tag*{(Known)}\\
          \lambda^3 - 92\lambda^2 + 1440\lambda &= 0
        \end{align*}
        Which gives us two equations:
        \begin{align*}
          a^2+b^2+52 &= 92\tag*{(1)}\\
          a^2b^2 + 36b^2 &= 1440\tag*{(2)}
        \end{align*}
        Solving simultaneously and only including $a,b>0$:
        \begin{align*}
          a=2, b=6\qed
        \end{align*}

      \item Find the corresponding eigenvector for each singular value:
        \begin{align*}
          \lambda_1 = (6\sqrt{2})^2 &\implies \vec{v_1} = \left(\begin{array}{c} 4/3\\ 1/3\\ 1 \end{array}\right)\\
          \lambda_1 = (2\sqrt{5})^2 &\implies \vec{v_2} = \left(\begin{array}{c} -5/6\\ 1/3\\ 1 \end{array}\right)\\
          \lambda_1 = 0^2 &\implies \vec{v_3} = \left(\begin{array}{c} 0\\ -3\\ 1 \end{array}\right)
        \end{align*}
        Then form $V$ from the normalized unit eigenvectors:
        \begin{align*}
          V = \left(\begin{array}{ccc} 4/\sqrt{26} & -5 /\sqrt{65} & 0\\ 1 /\sqrt{26} & 2 /\sqrt{65} & -3 /\sqrt{10}\\ 3 /\sqrt{26} & 6 /\sqrt{65} & 1 /\sqrt{10} \end{array}\right) \qed
        \end{align*}

      \item For each $\vec{v_i}$ column unit eigenvector of $V$, find the corresponding left singular vector:
        \begin{align*}
          \sigma_1 = 6\sqrt{2} \implies u_1 &= \frac{1}{6\sqrt{2}}A \frac{1}{\sqrt{26}}\left(\begin{array}{c} 4\\ 1\\ 3 \end{array}\right) = \frac{1}{\sqrt{13}}\left(\begin{array}{c} 3\\ 2\\ 0 \end{array}\right)\\
          \sigma_2 = 2\sqrt{5} \implies u_2 &= \frac{1}{2\sqrt{5}}A \frac{1}{\sqrt{65}}\left(\begin{array}{c} -5\\ 2\\ 6 \end{array}\right) = \frac{1}{\sqrt{14}}\left(\begin{array}{c} 2\\ -3\\ 0 \end{array}\right)\\
          \sigma_3 = 0 \implies u_3 &= \left(\begin{array}{c} 0\\ 0\\ 1 \end{array}\right)\tag*{(By choosing orthogonal vector)}
        \end{align*}
        Therefore $U = \left(\begin{array}{ccc} 3/\sqrt{13} & 2/\sqrt{13} & 0\\ 2 /\sqrt{13} & -3 /\sqrt{13}  & 0\\ 0 & 0 & 1 \end{array}\right) \qed$
    \end{enumerate}

  \pagebreak

\item Check if the input vectors are linearly independent by reducing matrix $U$ which has columns as the input vectors:
  \begin{gather*}
    U = \left(\begin{array}{cccc} 1 & 0 & 0 & 0\\ 0 & 1 & 0 & 1\\ 1 & 1 & 1 & 0\\ 0 & 0 & 1 & 1 \end{array}\right)\xrightarrow{RREF} \left(\begin{array}{cccc} 1 & 0 & 0 & 0\\ 0 & 1 & 0 & 0\\ 0 & 0 & 1 & 0\\ 0 & 0 & 0 & 1 \end{array}\right)\\
    \implies U \text{ is linearly independent}\\
    \implies U \text{ is invertible}
  \end{gather*}
  Suppose $A$ is the standard matrix for $T$, and $V$ is the output vectors, then by the relation of $A, U, V$:
  \begin{align*}
    V &= AU\\
    A &= VU^{-1}\\
      &= \left(\begin{array}{cccc} 0 & -2 & 0 & -4\\ 0 & 1 & 0 & 1\\ 3 & 3 & 3 & 2\\ 3 & 6 & 3 & 11 \end{array}\right)\left(\begin{array}{cccc} 1 & 0 & 0 & 0\\ 0 & 1 & 0 & 1\\ 1 & 1 & 1 & 0\\ 0 & 0 & 1 & 1 \end{array}\right)^{-1}\\
      &= \left(\begin{array}{cccc} -1 & -3 & 1 & -1\\ 0 & 1 & 0 & 0\\ 1 & 1 & 2 & 1\\ 4 & 7 & -1 & 4 \end{array}\right)
  \end{align*}
  Then by using MATLAB, the corresponding eigenvector for eigenvalue $\lambda = 2$ is:
  \begin{align*}
    \vec{u} &= \left(\begin{array}{c} -1\\ 0\\ -2\\ 1 \end{array}\right) \qed
  \end{align*}
\end{enumerate}
%%%%%%%%%%%%%%%%%%%%%%%%%%%%%%%%%%%%%%%%%%%%%%%%%%%%%%
%                       End                          %
%%%%%%%%%%%%%%%%%%%%%%%%%%%%%%%%%%%%%%%%%%%%%%%%%%%%%%

\end{document}

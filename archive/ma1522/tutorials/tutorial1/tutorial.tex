\documentclass[12pt, a4paper]{article}

\usepackage[a4paper, margin=1in]{geometry}

\usepackage[utf8]{inputenc}
\usepackage[mathscr]{euscript}
\let\euscr\mathscr \let\mathscr\relax
\usepackage[scr]{rsfso}
\usepackage{amssymb,amsmath,amsthm,amsfonts}
\usepackage[shortlabels]{enumitem}
\usepackage{multicol,multirow}
\usepackage{lipsum}
\usepackage{balance}
\usepackage{calc}
\usepackage[colorlinks=true,citecolor=blue,linkcolor=blue]{hyperref}
\usepackage{import}
\usepackage{xifthen}
\usepackage{pdfpages}
\usepackage{transparent}
\usepackage{tabularx}

\newcommand{\incfig}[2][1.0]{
    \def\svgwidth{#1\columnwidth}
    \import{./figures/}{#2.pdf_tex}
}
\newcommand{\incimg}[2][1.0]{
  \includegraphics[width=#1\columnwidth]{./figures/#2}
}


\input{letterfonts}

\newcommand{\mytitle}{MA1522 Tutorial 1}
\newcommand{\myauthor}{github/omgeta}
\newcommand{\mydate}{AY 24/25 Sem 1}

\begin{document}
\raggedright
\footnotesize
\begin{center}
{\normalsize{\textbf{\mytitle}}} \\
{\footnotesize{\mydate\hspace{2pt}\textemdash\hspace{2pt}\myauthor}}
\end{center}
\setlist{topsep=-1em, itemsep=-1em, parsep=2em}

%%%%%%%%%%%%%%%%%%%%%%%%%%%%%%%%%%%%%%%%%%%%%%%%%%%%%%
%                      Begin                         %
%%%%%%%%%%%%%%%%%%%%%%%%%%%%%%%%%%%%%%%%%%%%%%%%%%%%%%
\begin{enumerate}[Q\arabic*.]
  \item
  \begin{enumerate}[(\alph*)]
    \item Substitute $y = t$ into $x = 1 + 2t$:
      \begin{align*}
        x &= 1 + 2y \\
        x - 2y &= 1 \qed
      \end{align*}
    \item Suppose $x = t$:
      \begin{align*}
        t - 2y &= 1 \\
        -2y &= 1 - t \\
        y &= \frac{1}{2}t - \frac{1}{2} \qed
      \end{align*}
  \end{enumerate}
  \item Substitute $y = s$ and $z = t$ into $x = 3 -4s +t$:
    \begin{align*}
      x &= 3 - 4y + z \\
      x + 4y - z &= 3 \qed  
    \end{align*}

  \item \begin{enumerate}[(\alph*)]
    \item Reduce the corresponding augmented matrix:
    \begin{align*}
      \begin{bmatrix}3 & 2 &-4 &3 \\ 2 &3 &3 &15 \\ 5 &-3 &1 &14\end{bmatrix} &\sim \begin{bmatrix}1 &0 &0 &3 \\ 0 &1 &0 &1 \\ 0 &0 &1 &2\end{bmatrix} \\
    \end{align*}
    Therefore, the unique solution is:
    \begin{align*}
      x = 3, y = 1, z = 2 \qed
    \end{align*}
    \item Reduce the corresponding augmented matrix:
    \begin{align*}
      \begin{bmatrix}1 & 1 & -1 & -2 & 0 \\ 2 & 1 & -1 & 1 & -2 \\ -1 & 1 & -3 & 1 & 4\end{bmatrix}
      \sim
      \begin{bmatrix}1 & 0 & 0 & 3 & -2 \\ 0 & 1 & 0 & -\frac{19}{2} & 2 \\ 0 & 0 &1 & -\frac{9}{2} & 0\end{bmatrix}
    \end{align*}
    Therefore, the general solution is:
    \begin{align*}
      \begin{cases}
        a &= -2 - 3d \\
        b &= 2 + \frac{19}{2}d \\
        c &= \frac{9}{2}d \\
        d & \text{ is free}
      \end{cases} \qed
    \end{align*}
    \item Reduce the corresponding augmented matrix:
    \begin{align*}
      \begin{bmatrix}1 & -4 &  2 & -2 \\ 1 &2 &-2 &-3 \\ 1 &-1 &0 &4\end{bmatrix} \sim \begin{bmatrix}1 &0 &-\frac{2}{3} &0 \\ 0 &1 &-\frac{2}{3} &0 \\ 0 &0 &0 &1\end{bmatrix} \\
    \end{align*}
    System of equations is inconsistent, therefore, there is no solution. $\qed$
  \end{enumerate}
  \item Reduce the augmented matrix:
    \begin{align*}
      \begin{bmatrix}
        a & 0 & b & 2 \\
        a & a & 4 & 4 \\
        0 & a & 2 & b
      \end{bmatrix}
      \sim
      \begin{bmatrix}
        a & 0 & b & 2 \\
        0 & a & 4-b & 2 \\
        0 & 0 & b-2 & 0
      \end{bmatrix}
    \end{align*}
    \begin{enumerate}[(\alph*)]
      \item There are no solutions when:
        \begin{align*}
        a = 0, b \neq 2 \qed
        \end{align*}
      \item There is a unique solution when:
        \begin{align*}
          a \neq 0, b \neq 2 \qed
        \end{align*}
      \item There are infinite solutions with one free parameter when:
        \begin{align*}
          a \neq 0, b = 2 \qed
        \end{align*}
      \item There are infinite solutions with two free parameters when:
        \begin{align*}
          a = 0, b = 2 \qed
        \end{align*}
    \end{enumerate}
  \item 
    \begin{enumerate}[(\alph*)]
      \item Yes $\qed$
        \begin{align*}
          x + y + z &= 0 \\
          x + y + z &= 1
        \end{align*}
      \item Yes $\qed$
        \begin{align*}
          x &= 2 \\
          y &= 3 \\
          x + y &= 5
        \end{align*}
      \item No $\qed$ \\
        For a unique solution, there must be no free variables.\\
        However, there must be more unknowns than equations, and each row can have at most 1 pivot.\\
        $\therefore$ there must be some column with no pivot. \\
        $\therefore$ there must be a non-pivot column. \\
        $\therefore$ cannot have a unique solution.
      \item Yes $\qed$
        \begin{align*}
          x - y &= 0 \\
          2x-2y &= 0\\
          3x-3y &= 0
        \end{align*}
    \end{enumerate}

  \item Reduce the corresponding augmented matrix:
    \begin{align*}
      \begin{bmatrix}
        1 & -1 & 2 & 6 \\
        2 & 2 & -5 & 3 \\
        2 & 5 & 1 & 9
      \end{bmatrix} \sim
      \begin{bmatrix}
        1 & 0 & 0 & 4 \\
        0 & 1 & 0 & 0 \\
        0 & 0 & 1 & 1
      \end{bmatrix}
    \end{align*}
    Therefore the solution is:
    \begin{align*}
      x^2 = 4, y^2 &= 0, z^2 = 1 \\
      x = \pm2, y &= 0, z = \pm1 \qed
    \end{align*}
  \item We have the linear equations
    \begin{align*}
      x_1 + x_3 = 800 \\
      x_1 - x_2 + x_4 = 200 \\
      x_2 - x_5 = 500 \\
      x_3 + x_6 = 750 \\
      x_4 + x_6 - x_7 = 600 \\
      x_5 - x_7 = -50
    \end{align*}
    Reduce the corresponding augmented matrix:
    \begin{align*}
      \begin{bmatrix}
        1 & 0 & 1 & 0 & 0 & 0 & 0 & 800 \\
        1 & -1 & 0 & 1 & 0 & 0 & 0 & 200 \\
        0 & 1 & 0 & 0 & -1 & 0 & 0 & 500 \\
        0 & 0 & 1 & 0 & 0 & 1 & 0 & 750 \\
        0 & 0 & 0 & 1 & 0 & 1 & -1 & 600 \\
        0 & 0 & 0 & 0 & 1 & 0 & -1 & -50
      \end{bmatrix} \sim
      \begin{bmatrix}
        1 & 0 & 0 & 0 & 0 & -1 & 0 & 50 \\
        0 & 1 & 0 & 0 & 0 & 0 & -1 & 450 \\
        0 & 0 & 1 & 0 & 0 & 1 & 0 & 750 \\
        0 & 0 & 0 & 1 & 0 & 1 & -1 & 600 \\
        0 & 0 & 0 & 0 & 1 & 0 & -1 & -50 \\
        0 & 0 & 0 & 0 & 0 & 0 & 0 & 0
      \end{bmatrix}
    \end{align*}
    Therefore, the general solution is:
    \begin{align*}
      \begin{cases}
        x_1 &= 50 + x_6 \\
        x_2 &= 450 + x_7 \\
        x_3 &= 750 - x_6 \\
        x_4 &= 600 - x_6 + x_7 \\
        x_5 &= -50 + x_7 \\
        x_6 & \text{ is free} \\
        x_7 & \text{ is free} \\
      \end{cases}
    \end{align*}
    \begin{enumerate}[(\alph*)]
      \item No, there are an infinite number of solutions $\qed$
      \item $x_1 = 100, x_2 = 550, x_3 = 700, x_4 = 650, x_5 = 50 \qed$
      \item Suppose $x_1 = 0$
      \begin{align*}
        \begin{bmatrix}
          0 & 0 & 1 & 0 & 0 & 0 & 0 & 800 \\
          0 & -1 & 0 & 1 & 0 & 0 & 0 & 200 \\
          0 & 1 & 0 & 0 & -1 & 0 & 0 & 500 \\
          0 & 0 & 1 & 0 & 0 & 1 & 0 & 750 \\
          0 & 0 & 0 & 1 & 0 & 1 & -1 & 600 \\
          0 & 0 & 0 & 0 & 1 & 0 & -1 & -50
        \end{bmatrix} \sim
        \begin{bmatrix}
          0 & 1 & 0 & 0 & 0 & 0 & -1 & 450 \\
          0 & 0 & 1 & 0 & 0 & 0 & 0 & 800 \\
          0 & 0 & 0 & 1 & 0 & 0 & -1 & 650 \\
          0 & 0 & 0 & 0 & 1 & 0 & -1 & -50 \\
          0 & 0 & 0 & 0 & 0 & 1 & 0 & -50 \\
          0 & 0 & 0 & 0 & 0 & 0 & 0 & 0
        \end{bmatrix}
      \end{align*}
      This solution is not possible as it would result in the impossible with negative traffic $\qed$
    \end{enumerate}
\end{enumerate}
%%%%%%%%%%%%%%%%%%%%%%%%%%%%%%%%%%%%%%%%%%%%%%%%%%%%%%
%                       End                          %
%%%%%%%%%%%%%%%%%%%%%%%%%%%%%%%%%%%%%%%%%%%%%%%%%%%%%%

\end{document}

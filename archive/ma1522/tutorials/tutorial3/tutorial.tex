\documentclass[12pt, a4paper]{article}

\usepackage[a4paper, margin=1in]{geometry}

\usepackage[utf8]{inputenc}
\usepackage[mathscr]{euscript}
\let\euscr\mathscr \let\mathscr\relax
\usepackage[scr]{rsfso}
\usepackage{amssymb,amsmath,amsthm,amsfonts}
\usepackage[shortlabels]{enumitem}
\usepackage{multicol,multirow}
\usepackage{lipsum}
\usepackage{balance}
\usepackage{calc}
\usepackage[colorlinks=true,citecolor=blue,linkcolor=blue]{hyperref}
\usepackage{import}
\usepackage{xifthen}
\usepackage{pdfpages}
\usepackage{transparent}
\usepackage{tabularx}

\newcommand{\incfig}[2][1.0]{
    \def\svgwidth{#1\columnwidth}
    \import{./figures/}{#2.pdf_tex}
}
\newcommand{\incimg}[2][1.0]{
  \includegraphics[width=#1\columnwidth]{./figures/#2}
}


\input{letterfonts}

\newcommand{\mytitle}{MA1522 Tutorial 3}
\newcommand{\myauthor}{github/omgeta}
\newcommand{\mydate}{AY 24/25 Sem 1}

\begin{document}
\raggedright
\footnotesize
\begin{center}
{\normalsize{\textbf{\mytitle}}} \\
{\footnotesize{\mydate\hspace{2pt}\textemdash\hspace{2pt}\myauthor}}
\end{center}
\setlist{topsep=-1em, itemsep=-1em, parsep=2em}

%%%%%%%%%%%%%%%%%%%%%%%%%%%%%%%%%%%%%%%%%%%%%%%%%%%%%%
%                      Begin                         %
%%%%%%%%%%%%%%%%%%%%%%%%%%%%%%%%%%%%%%%%%%%%%%%%%%%%%%
\begin{enumerate}[Q\arabic*.]
  \item If $A \in \RR^{4\times4}$ is obtained from $I$ by the following sequence of elementary row operations:
    \begin{align*}
      I \xrightarrow{\frac{1}{2}R_2}\xrightarrow{R_1-R_2}\xrightarrow{R_2\leftrightarrow R_4}\xrightarrow{R_3+3R_1}A
    \end{align*}
    Then $A$ is also obtained from $I$ by the following matrix multiplications:
    \begin{align*}
      A = E_4E_3E_2E_1I
    \end{align*}
    Where the elementary matrices $E_i$ are given by:
    \begin{align*}
      E_1 &= \begin{bmatrix}1&0&0&0\\0&1/2&0&0\\0&0&1&0\\0&0&0&1\end{bmatrix} \\
      E_2 &= \begin{bmatrix}1&-1&0&0\\0&1&0&0\\0&0&1&0\\0&0&0&1\end{bmatrix} \\
      E_3 &= \begin{bmatrix}1&0&0&0\\0&0&0&1\\0&0&1&0\\0&1&0&0\end{bmatrix} \\
      E_4 &= \begin{bmatrix}1&0&0&0\\0&1&0&0\\3&0&1&0\\0&0&0&1\end{bmatrix} \\
    \end{align*}
    And their inverse are given by:
    \begin{align*}
      E_1^{-1} &= \begin{bmatrix}1&0&0&0\\0&2&0&0\\0&0&1&0\\0&0&0&1\end{bmatrix} \\
      E_2^{-1} &= \begin{bmatrix}1&1&0&0\\0&1&0&0\\0&0&1&0\\0&0&0&1\end{bmatrix} \\
      E_3^{-1} &= \begin{bmatrix}1&0&0&0\\0&0&0&1\\0&0&1&0\\0&1&0&0\end{bmatrix} \\
      E_4^{-1} &= \begin{bmatrix}1&0&0&0\\0&1&0&0\\-3&0&1&0\\0&0&0&1\end{bmatrix} \\
    \end{align*}
    Then the inverse $A^{-1}$ is obtained from $I$ by the following matrix mulitiplications:
    \begin{align*}
      A^{-1} &= (E_4E_3E_2E_1I)^{-1} \\
             &= E_1^{-1}E_2^{-1}E_3^{-1}E_4^{-1} \qed
    \end{align*}

  \item 
    \begin{enumerate}[(\alph*)]
      \item Find the LU factorisation for $A$:
        \begin{align*}
          A = \begin{bmatrix}2&-1&2\\-6&0&-2\\8&-1&5\end{bmatrix} &\xrightarrow{R_2+3R_1}
          \begin{bmatrix}2&-1&2\\0&-3&4\\8&-1&5\end{bmatrix} \\&\xrightarrow{R_3-4R_1}
          \begin{bmatrix}2&-1&2\\0&-3&4\\8&3&-3\end{bmatrix} \\&\xrightarrow{R_2+3R_1}
          \begin{bmatrix}2&-1&2\\0&-3&4\\0&0&1\end{bmatrix} = U \\
          L = \begin{bmatrix}1&0&0\\-3&1&0\\4&-1&1\end{bmatrix} \\
          \therefore A =  \begin{bmatrix}1&0&0\\-3&1&0\\4&-1&1\end{bmatrix}\begin{bmatrix}2&-1&2\\0&-3&4\\0&0&1\end{bmatrix} \qed 
        \end{align*}
        Let $\vec{y} = U \vec{x}$ and solve $L \vec{y} = \vec{b}$
        \begin{align*}
          \begin{bmatrix}1&0&0&1\\-3&1&0&0\\4&-1&1&4\end{bmatrix} \xrightarrow{\text{RREF}}
          \begin{bmatrix}1&0&0&1\\0&1&0&1\\0&0&1&3\end{bmatrix}
        \end{align*}
        Then solve $U \vec{x} = \begin{bmatrix}1\\1\\3\end{bmatrix}$
        \begin{align*}
          \begin{bmatrix}2&-1&2&1\\0&-3&4&1\\0&0&1&3\end{bmatrix} \xrightarrow{\text{RREF}}
          \begin{bmatrix}1&0&0&-2/3\\0&1&0&11/3\\0&0&1&3\end{bmatrix}
        \end{align*}
        Therefore, $\vec{x} = \begin{bmatrix}-2/3\\11/3\\3\end{bmatrix} \qed$


      \item Find the LU factorisation for $A$:
        \begin{align*}
          A = \begin{bmatrix}2&-4&4&-2\\6&-9&7&-3\\-1&-4&8&0\end{bmatrix} &\xrightarrow[R_2+\frac{1}{2}R_1]{R_2-3R_1}
          \begin{bmatrix}2&-4&4&-2\\0&3&-5&3\\0&-6&10&-1\end{bmatrix} \\&\xrightarrow{R_3+2R_2}
          \begin{bmatrix}2&-4&4&-2\\0&3&-5&3\\0&0&0&5\end{bmatrix} = U \\
          L = \begin{bmatrix}1&0&0\\3&1&0\\-1/2&-2&1\end{bmatrix}\\
          \therefore A =  \begin{bmatrix}1&0&0\\3&1&0\\-1/2&-2&1\end{bmatrix} \begin{bmatrix}2&-4&4&-2\\0&3&-5&3\\0&0&0&5\end{bmatrix} \qed 
        \end{align*}
        Let $\vec{y} = U \vec{x}$ and solve $L \vec{y} = \vec{b}$
        \begin{align*}
          \begin{bmatrix}1&0&0&0\\3&1&0&0\\-1/2&-2&1&17\end{bmatrix} \xrightarrow{\text{RREF}}
          \begin{bmatrix}1&0&0&0\\0&1&0&0\\0&0&1&17\end{bmatrix}
        \end{align*}
        Then solve $U \vec{x} = \begin{bmatrix}0\\0\\17\end{bmatrix}$
        \begin{align*}
          \begin{bmatrix}2&-4&4&-2&0\\0&3&-5&3&0\\0&0&0&5&17\end{bmatrix} \xrightarrow{\text{RREF}}
          \begin{bmatrix}1&0&-4/3&0&-17/5\\0&1&-5/3&0&-17/5\\0&0&0&1&17/5\end{bmatrix}
        \end{align*}
        Therefore, $\vec{x} = \displaystyle \frac{17}{5}\begin{bmatrix}-1\\-1\\0\\1\end{bmatrix} + \frac{x_3}{3}\begin{bmatrix}-4\\-5\\3\\0\end{bmatrix} \qed$
    \end{enumerate}

  \item \begin{enumerate}[(\alph*)]
      \item Find an LU factorisation for A 
        \begin{align*}
          A = \begin{bmatrix}
            2 & -6 & 6\\
            -4 & 5 & -7\\
            3 & 5 & -1\\
            -6 & 4 & -8\\
            8 & -3 & 9
          \end{bmatrix} &\sim
          \begin{bmatrix}
            2 & -6 & 6\\
            0 & -7 & 5\\
            0 & 14 & -19\\
            0 & -14 & 10\\
            0 & 21 & 15
          \end{bmatrix}\\&\sim
          \begin{bmatrix}
            2 & -6 & 6\\
            0 & -7 & 5\\
            0 & 0 & -9\\
            0 & 0 & 0 \\
            0 & 0 & 30
          \end{bmatrix}\\&\sim
          \begin{bmatrix}
            2 & -6 & 6\\
            0 & -7 & 5\\
            0 & 0 & 0\\
            0 & 0 & 0 \\
            0 & 0 & 0
          \end{bmatrix} = U \qed \\
          L = \begin{bmatrix}
            1 & 0 & 0 & 0 & 0\\
            -2 & 1 & 0 & 0 & 0\\
            3/2 & -2 & 1 & 0 & 0\\
            -3 & 2 & 0 & 1 & 0\\
            4 & -3 & 0 & 0 & 1
          \end{bmatrix} \qed
        \end{align*}
      \item The MATLAB LU is the same. $\qed$
    \end{enumerate}

  \item First calculate the determinant by cofactor expansion:
    \begin{align*}
      \det(A) &= -x\begin{vmatrix}-x & 1 \\ -5 & 4-x\end{vmatrix} - \begin{vmatrix}0 & 1 \\ 2 & 4-x\end{vmatrix} \\
              &= -x[-x(4-x) - 1(-5)] - [0(4-x) - 1(2)]\\
              &= -x(-4x + x^2 + 5) - (-2) \\
              &= -x^3 + 4x^2 -5x + 2 \qed
    \end{align*}
    For $A$ to be singular, $\det(A) = 0$:
    \begin{align*}
      -x^3 + 4x^2 - 5x + 2 &= 0 \\
      (x-1)^2(x-2) &= 0\\
      x &= 1, 2 \qed 
    \end{align*}

  \item First, reduce the relevant matrices:
    \begin{align*}
      \begin{bmatrix}
        a+px & b+qx & c+rx\\
        p+ux & q+vx & r+wx\\
        u+ax & v+bx & w+cx
      \end{bmatrix}&\xrightarrow{R_2-xR_3}
      \begin{bmatrix}
        a+px & b+qx & c+rx\\
        p-ax^2 & q-bx^2 & r-cx^2\\
        u+ax & v+bx & w+cx
      \end{bmatrix}\\&\xrightarrow{R_1-xR_2}
      \begin{bmatrix}
        a(1+x^3) & b(1+x^3) & c(1+x^3)\\
        p-ax^2 & q-bx^2 & r-cx^2\\
        u+ax & v+bx & w+cx
      \end{bmatrix}\\
      \begin{bmatrix}
        a & b & c\\
        p-ax^2 & q-bx^2 & r-cx^2\\
        u+ax & v+bx & w+cx
      \end{bmatrix}&\xrightarrow{R_2+x^2R_1}
      \begin{bmatrix}
        a & b & c\\
        p & q & r\\
        u+ax & v+bx & w+cx
      \end{bmatrix}\\&\xrightarrow{R_3-xR_1}
      \begin{bmatrix}
        a & b & c\\
        p & q & r\\
        u & v & w
      \end{bmatrix}
    \end{align*}
    Then, we can conclude:
    \begin{align*}
      \begin{vmatrix}
        a+px & b+qx & c+rx\\
        p+ux & q+vx & r+wx\\
        u+ax & v+bx & w+cx
      \end{vmatrix}&=
      (1+x^3)
      \begin{vmatrix}
        a & b & c\\
        p-ax^2 & q-bx^2 & r-cx^2\\
        u+ax & v+bx & w+cx
      \end{vmatrix}\\ &=
      (1+x^3)
      \begin{vmatrix}
        a & b & c\\
        p & q & r\\
        u & v & w
      \end{vmatrix}
    \end{align*}

  \item First, find $\det(A)$ and $\det(B)$
    \begin{align*}
      \det(A) &= 1\begin{vmatrix}2&6&3\\0&1&2\\0&1&1\end{vmatrix}\\
              &= 2\begin{vmatrix}1&2\\1&1\end{vmatrix}\\
              &= 2[1(1)-2(1)]\\
              &= -2 \\
      \det(B) &= 1\begin{vmatrix}1&1&-1\\0&1&2\\0&0&3\end{vmatrix}\\
              &= 1\begin{vmatrix}1&2\\0&3\end{vmatrix}\\
              &= 3
    \end{align*}
    \begin{enumerate}[(\alph*)]
      \item $\det(3A^T) = 3^4\det(A) = -162 \qed$
      \item $\det(3AB^{-1}) = 3^4\frac{\det(A)}{\det(B)} = -54 \qed$
      \item $\det(3A^T) = \frac{1}{\det(3B)} = \frac{1}{3^4\det(B)} =  \frac{1}{243}\qed$
    \end{enumerate}

  \item Let $A = \begin{bmatrix}1&5&3\\0&2&-2\\0&1&3\end{bmatrix}, \vec{b} = \begin{bmatrix}1\\2\\0\end{bmatrix}$. Then, $\det(A) = 1\begin{vmatrix}2&-2\\1&3\end{vmatrix} = 8$. By Cramer's rule:
    \begin{align*}
      x_1 &= \frac{\begin{vmatrix}1&5&3\\2&2&-2\\0&1&3\end{vmatrix}}{8} \\
                &= \frac{(6+2)-2(15-3)}{8} \\
                &= -2 \\
      x_2 &= \frac{\begin{vmatrix}1&1&3\\0&2&-2\\0&0&3\end{vmatrix}}{8} \\
                &= \frac{6}{8} \\
                &= \frac{3}{4} \\
      x_3 &= \frac{\begin{vmatrix}1&5&1\\0&2&2\\0&1&0\end{vmatrix}}{8} \\
                &= \frac{-2}{8} \\
                &= -\frac{1}{4} \\
    \end{align*}
    Therefore, $\vec{x} = \begin{bmatrix}-2\\3/4\\-1/4\end{bmatrix} \qed$

  \item First find the adjoint of $A$:
    \begin{align*}
      \adj(A) &= \begin{bmatrix}A_{11}&A_{21}&A_{31}\\A_{12}&A_{22}&A_{32}\\A_{13}&A_{23}&A_{33}\end{bmatrix}\\
              &= \begin{bmatrix}12 & 6 & -5\\ 3 & 0 & -1\\ -6 & -3 & 2\end{bmatrix} \qed
    \end{align*}
    Then find the determinant:
    \begin{align*}
      \det(A) &= 1\begin{vmatrix}2&1\\0&6\end{vmatrix} + 3\begin{vmatrix}-1&2\\2&1\end{vmatrix}\\
              &= 12 + 3(-1 - 4)\\
              &= -3          
    \end{align*}
    Then the inverse $A^{-1}$ is given by:
    \begin{align*}
      A^{-1} &= \frac{1}{\det(A)}\adj(A)\\
             &= -\frac{1}{3}\begin{bmatrix}12&6&-5\\3&0&-1\\-6&-3&2\end{bmatrix} \qed
    \end{align*}
\end{enumerate}
%%%%%%%%%%%%%%%%%%%%%%%%%%%%%%%%%%%%%%%%%%%%%%%%%%%%%%
%                       End                          %
%%%%%%%%%%%%%%%%%%%%%%%%%%%%%%%%%%%%%%%%%%%%%%%%%%%%%%

\end{document}

\documentclass[12pt, a4paper]{article}

\usepackage[a4paper, margin=1in]{geometry}

\usepackage[utf8]{inputenc}
\usepackage[mathscr]{euscript}
\let\euscr\mathscr \let\mathscr\relax
\usepackage[scr]{rsfso}
\usepackage{amssymb,amsmath,amsthm,amsfonts}
\usepackage[shortlabels]{enumitem}
\usepackage{multicol,multirow}
\usepackage{lipsum}
\usepackage{balance}
\usepackage{calc}
\usepackage[colorlinks=true,citecolor=blue,linkcolor=blue]{hyperref}
\usepackage{import}
\usepackage{xifthen}
\usepackage{pdfpages}
\usepackage{transparent}
\usepackage{listings}

\newcommand{\incfig}[2][1.0]{
    \def\svgwidth{#1\columnwidth}
    \import{./figures/}{#2.pdf_tex}
}

\newlist{enumproof}{enumerate}{4}
\setlist[enumproof,1]{label=\arabic*., parsep=1em}
\setlist[enumproof,2]{label=\arabic{enumproofi}.\arabic*., parsep=1em}
\setlist[enumproof,3]{label=\arabic{enumproofi}.\arabic{enumproofii}.\arabic*., parsep=1em}
\setlist[enumproof,4]{label=\arabic{enumproofi}.\arabic{enumproofii}.\arabic{enumproofiii}.\arabic*., parsep=1em}

\renewcommand{\qedsymbol}{\ensuremath{\blacksquare}}

\lstdefinestyle{mystyle}{
  language=C, % Set the language to C
  commentstyle=\color{codegray}, % Color for comments
  keywordstyle=\color{orange}, % Color for basic keywords
  stringstyle=\color{mauve}, % Color for strings
  basicstyle={\ttfamily\footnotesize}, % Basic font style
  breakatwhitespace=false,         
  breaklines=true,                 
  captionpos=b,                    
  keepspaces=true,                 
  numbers=none,                    
  tabsize=2,
  morekeywords=[2]{\#include, \#define, \#ifdef, \#ifndef, \#endif, \#pragma, \#else, \#elif}, % Preprocessor directives
  keywordstyle=[2]\color{codegreen}, % Style for preprocessor directives
  morekeywords=[3]{int, char, float, double, void, struct, union, enum, const, volatile, static, extern, register, inline, restrict, _Bool, _Complex, _Imaginary, size_t, ssize_t, FILE}, % C standard types and common identifiers
  keywordstyle=[3]\color{identblue}, % Style for types and common identifiers
  morekeywords=[4]{printf, scanf, fopen, fclose, malloc, free, calloc, realloc, perror, strtok, strncpy, strcpy, strcmp, strlen}, % Standard library functions
  keywordstyle=[4]\color{cyan}, % Style for library functions
}

% Things Lie
\newcommand{\kb}{\mathfrak b}
\newcommand{\kg}{\mathfrak g}
\newcommand{\kh}{\mathfrak h}
\newcommand{\kn}{\mathfrak n}
\newcommand{\ku}{\mathfrak u}
\newcommand{\kz}{\mathfrak z}
\DeclareMathOperator{\Ext}{Ext} % Ext functor
\DeclareMathOperator{\Tor}{Tor} % Tor functor
\newcommand{\gl}{\opname{\mathfrak{gl}}} % frak gl group
\renewcommand{\sl}{\opname{\mathfrak{sl}}} % frak sl group chktex 6

% More script letters etc.
\newcommand{\SA}{\mathcal A}
\newcommand{\SB}{\mathcal B}
\newcommand{\SC}{\mathcal C}
\newcommand{\SF}{\mathcal F}
\newcommand{\SG}{\mathcal G}
\newcommand{\SH}{\mathcal H}
\newcommand{\OO}{\mathcal O}

\newcommand{\SCA}{\mathscr A}
\newcommand{\SCB}{\mathscr B}
\newcommand{\SCC}{\mathscr C}
\newcommand{\SCD}{\mathscr D}
\newcommand{\SCE}{\mathscr E}
\newcommand{\SCF}{\mathscr F}
\newcommand{\SCG}{\mathscr G}
\newcommand{\SCH}{\mathscr H}

% Mathfrak primes
\newcommand{\km}{\mathfrak m}
\newcommand{\kp}{\mathfrak p}
\newcommand{\kq}{\mathfrak q}

% number sets
\newcommand{\RR}[1][]{\ensuremath{\ifstrempty{#1}{\mathbb{R}}{\mathbb{R}^{#1}}}}
\newcommand{\NN}[1][]{\ensuremath{\ifstrempty{#1}{\mathbb{N}}{\mathbb{N}^{#1}}}}
\newcommand{\ZZ}[1][]{\ensuremath{\ifstrempty{#1}{\mathbb{Z}}{\mathbb{Z}^{#1}}}}
\newcommand{\QQ}[1][]{\ensuremath{\ifstrempty{#1}{\mathbb{Q}}{\mathbb{Q}^{#1}}}}
\newcommand{\CC}[1][]{\ensuremath{\ifstrempty{#1}{\mathbb{C}}{\mathbb{C}^{#1}}}}
\newcommand{\PP}[1][]{\ensuremath{\ifstrempty{#1}{\mathbb{P}}{\mathbb{P}^{#1}}}}
\newcommand{\HH}[1][]{\ensuremath{\ifstrempty{#1}{\mathbb{H}}{\mathbb{H}^{#1}}}}
\newcommand{\FF}[1][]{\ensuremath{\ifstrempty{#1}{\mathbb{F}}{\mathbb{F}^{#1}}}}
% expected value
\newcommand{\EE}{\ensuremath{\mathbb{E}}}
\newcommand{\charin}{\text{ char }}
\DeclareMathOperator{\sign}{sign}
\DeclareMathOperator{\Aut}{Aut}
\DeclareMathOperator{\Inn}{Inn}
\DeclareMathOperator{\Syl}{Syl}
\DeclareMathOperator{\Gal}{Gal}
\DeclareMathOperator{\GL}{GL} % General linear group
\DeclareMathOperator{\SL}{SL} % Special linear group

%---------------------------------------
% BlackBoard Math Fonts :-
%---------------------------------------

%Captital Letters
\newcommand{\bbA}{\mathbb{A}}	\newcommand{\bbB}{\mathbb{B}}
\newcommand{\bbC}{\mathbb{C}}	\newcommand{\bbD}{\mathbb{D}}
\newcommand{\bbE}{\mathbb{E}}	\newcommand{\bbF}{\mathbb{F}}
\newcommand{\bbG}{\mathbb{G}}	\newcommand{\bbH}{\mathbb{H}}
\newcommand{\bbI}{\mathbb{I}}	\newcommand{\bbJ}{\mathbb{J}}
\newcommand{\bbK}{\mathbb{K}}	\newcommand{\bbL}{\mathbb{L}}
\newcommand{\bbM}{\mathbb{M}}	\newcommand{\bbN}{\mathbb{N}}
\newcommand{\bbO}{\mathbb{O}}	\newcommand{\bbP}{\mathbb{P}}
\newcommand{\bbQ}{\mathbb{Q}}	\newcommand{\bbR}{\mathbb{R}}
\newcommand{\bbS}{\mathbb{S}}	\newcommand{\bbT}{\mathbb{T}}
\newcommand{\bbU}{\mathbb{U}}	\newcommand{\bbV}{\mathbb{V}}
\newcommand{\bbW}{\mathbb{W}}	\newcommand{\bbX}{\mathbb{X}}
\newcommand{\bbY}{\mathbb{Y}}	\newcommand{\bbZ}{\mathbb{Z}}

%---------------------------------------
% MathCal Fonts :-
%---------------------------------------

%Captital Letters
\newcommand{\mcA}{\mathcal{A}}	\newcommand{\mcB}{\mathcal{B}}
\newcommand{\mcC}{\mathcal{C}}	\newcommand{\mcD}{\mathcal{D}}
\newcommand{\mcE}{\mathcal{E}}	\newcommand{\mcF}{\mathcal{F}}
\newcommand{\mcG}{\mathcal{G}}	\newcommand{\mcH}{\mathcal{H}}
\newcommand{\mcI}{\mathcal{I}}	\newcommand{\mcJ}{\mathcal{J}}
\newcommand{\mcK}{\mathcal{K}}	\newcommand{\mcL}{\mathcal{L}}
\newcommand{\mcM}{\mathcal{M}}	\newcommand{\mcN}{\mathcal{N}}
\newcommand{\mcO}{\mathcal{O}}	\newcommand{\mcP}{\mathcal{P}}
\newcommand{\mcQ}{\mathcal{Q}}	\newcommand{\mcR}{\mathcal{R}}
\newcommand{\mcS}{\mathcal{S}}	\newcommand{\mcT}{\mathcal{T}}
\newcommand{\mcU}{\mathcal{U}}	\newcommand{\mcV}{\mathcal{V}}
\newcommand{\mcW}{\mathcal{W}}	\newcommand{\mcX}{\mathcal{X}}
\newcommand{\mcY}{\mathcal{Y}}	\newcommand{\mcZ}{\mathcal{Z}}

%---------------------------------------
% Bold Math Fonts :-
%---------------------------------------

%Captital Letters
\newcommand{\bmA}{\boldsymbol{A}}	\newcommand{\bmB}{\boldsymbol{B}}
\newcommand{\bmC}{\boldsymbol{C}}	\newcommand{\bmD}{\boldsymbol{D}}
\newcommand{\bmE}{\boldsymbol{E}}	\newcommand{\bmF}{\boldsymbol{F}}
\newcommand{\bmG}{\boldsymbol{G}}	\newcommand{\bmH}{\boldsymbol{H}}
\newcommand{\bmI}{\boldsymbol{I}}	\newcommand{\bmJ}{\boldsymbol{J}}
\newcommand{\bmK}{\boldsymbol{K}}	\newcommand{\bmL}{\boldsymbol{L}}
\newcommand{\bmM}{\boldsymbol{M}}	\newcommand{\bmN}{\boldsymbol{N}}
\newcommand{\bmO}{\boldsymbol{O}}	\newcommand{\bmP}{\boldsymbol{P}}
\newcommand{\bmQ}{\boldsymbol{Q}}	\newcommand{\bmR}{\boldsymbol{R}}
\newcommand{\bmS}{\boldsymbol{S}}	\newcommand{\bmT}{\boldsymbol{T}}
\newcommand{\bmU}{\boldsymbol{U}}	\newcommand{\bmV}{\boldsymbol{V}}
\newcommand{\bmW}{\boldsymbol{W}}	\newcommand{\bmX}{\boldsymbol{X}}
\newcommand{\bmY}{\boldsymbol{Y}}	\newcommand{\bmZ}{\boldsymbol{Z}}
%Small Letters
\newcommand{\bma}{\boldsymbol{a}}	\newcommand{\bmb}{\boldsymbol{b}}
\newcommand{\bmc}{\boldsymbol{c}}	\newcommand{\bmd}{\boldsymbol{d}}
\newcommand{\bme}{\boldsymbol{e}}	\newcommand{\bmf}{\boldsymbol{f}}
\newcommand{\bmg}{\boldsymbol{g}}	\newcommand{\bmh}{\boldsymbol{h}}
\newcommand{\bmi}{\boldsymbol{i}}	\newcommand{\bmj}{\boldsymbol{j}}
\newcommand{\bmk}{\boldsymbol{k}}	\newcommand{\bml}{\boldsymbol{l}}
\newcommand{\bmm}{\boldsymbol{m}}	\newcommand{\bmn}{\boldsymbol{n}}
\newcommand{\bmo}{\boldsymbol{o}}	\newcommand{\bmp}{\boldsymbol{p}}
\newcommand{\bmq}{\boldsymbol{q}}	\newcommand{\bmr}{\boldsymbol{r}}
\newcommand{\bms}{\boldsymbol{s}}	\newcommand{\bmt}{\boldsymbol{t}}
\newcommand{\bmu}{\boldsymbol{u}}	\newcommand{\bmv}{\boldsymbol{v}}
\newcommand{\bmw}{\boldsymbol{w}}	\newcommand{\bmx}{\boldsymbol{x}}
\newcommand{\bmy}{\boldsymbol{y}}	\newcommand{\bmz}{\boldsymbol{z}}

%---------------------------------------
% Scr Math Fonts :-
%---------------------------------------

\newcommand{\sA}{{\mathscr{A}}}   \newcommand{\sB}{{\mathscr{B}}}
\newcommand{\sC}{{\mathscr{C}}}   \newcommand{\sD}{{\mathscr{D}}}
\newcommand{\sE}{{\mathscr{E}}}   \newcommand{\sF}{{\mathscr{F}}}
\newcommand{\sG}{{\mathscr{G}}}   \newcommand{\sH}{{\mathscr{H}}}
\newcommand{\sI}{{\mathscr{I}}}   \newcommand{\sJ}{{\mathscr{J}}}
\newcommand{\sK}{{\mathscr{K}}}   \newcommand{\sL}{{\mathscr{L}}}
\newcommand{\sM}{{\mathscr{M}}}   \newcommand{\sN}{{\mathscr{N}}}
\newcommand{\sO}{{\mathscr{O}}}   \newcommand{\sP}{{\mathscr{P}}}
\newcommand{\sQ}{{\mathscr{Q}}}   \newcommand{\sR}{{\mathscr{R}}}
\newcommand{\sS}{{\mathscr{S}}}   \newcommand{\sT}{{\mathscr{T}}}
\newcommand{\sU}{{\mathscr{U}}}   \newcommand{\sV}{{\mathscr{V}}}
\newcommand{\sW}{{\mathscr{W}}}   \newcommand{\sX}{{\mathscr{X}}}
\newcommand{\sY}{{\mathscr{Y}}}   \newcommand{\sZ}{{\mathscr{Z}}}


%---------------------------------------
% Math Fraktur Font
%---------------------------------------

%Captital Letters
\newcommand{\mfA}{\mathfrak{A}}	\newcommand{\mfB}{\mathfrak{B}}
\newcommand{\mfC}{\mathfrak{C}}	\newcommand{\mfD}{\mathfrak{D}}
\newcommand{\mfE}{\mathfrak{E}}	\newcommand{\mfF}{\mathfrak{F}}
\newcommand{\mfG}{\mathfrak{G}}	\newcommand{\mfH}{\mathfrak{H}}
\newcommand{\mfI}{\mathfrak{I}}	\newcommand{\mfJ}{\mathfrak{J}}
\newcommand{\mfK}{\mathfrak{K}}	\newcommand{\mfL}{\mathfrak{L}}
\newcommand{\mfM}{\mathfrak{M}}	\newcommand{\mfN}{\mathfrak{N}}
\newcommand{\mfO}{\mathfrak{O}}	\newcommand{\mfP}{\mathfrak{P}}
\newcommand{\mfQ}{\mathfrak{Q}}	\newcommand{\mfR}{\mathfrak{R}}
\newcommand{\mfS}{\mathfrak{S}}	\newcommand{\mfT}{\mathfrak{T}}
\newcommand{\mfU}{\mathfrak{U}}	\newcommand{\mfV}{\mathfrak{V}}
\newcommand{\mfW}{\mathfrak{W}}	\newcommand{\mfX}{\mathfrak{X}}
\newcommand{\mfY}{\mathfrak{Y}}	\newcommand{\mfZ}{\mathfrak{Z}}
%Small Letters
\newcommand{\mfa}{\mathfrak{a}}	\newcommand{\mfb}{\mathfrak{b}}
\newcommand{\mfc}{\mathfrak{c}}	\newcommand{\mfd}{\mathfrak{d}}
\newcommand{\mfe}{\mathfrak{e}}	\newcommand{\mff}{\mathfrak{f}}
\newcommand{\mfg}{\mathfrak{g}}	\newcommand{\mfh}{\mathfrak{h}}
\newcommand{\mfi}{\mathfrak{i}}	\newcommand{\mfj}{\mathfrak{j}}
\newcommand{\mfk}{\mathfrak{k}}	\newcommand{\mfl}{\mathfrak{l}}
\newcommand{\mfm}{\mathfrak{m}}	\newcommand{\mfn}{\mathfrak{n}}
\newcommand{\mfo}{\mathfrak{o}}	\newcommand{\mfp}{\mathfrak{p}}
\newcommand{\mfq}{\mathfrak{q}}	\newcommand{\mfr}{\mathfrak{r}}
\newcommand{\mfs}{\mathfrak{s}}	\newcommand{\mft}{\mathfrak{t}}
\newcommand{\mfu}{\mathfrak{u}}	\newcommand{\mfv}{\mathfrak{v}}
\newcommand{\mfw}{\mathfrak{w}}	\newcommand{\mfx}{\mathfrak{x}}
\newcommand{\mfy}{\mathfrak{y}}	\newcommand{\mfz}{\mathfrak{z}}


\newcommand{\mytitle}{MA1522 Tutorial 4}
\newcommand{\myauthor}{github/omgeta}
\newcommand{\mydate}{AY 24/25 Sem 1}

\begin{document}
\raggedright
\footnotesize
\begin{center}
{\normalsize{\textbf{\mytitle}}} \\
{\footnotesize{\mydate\hspace{2pt}\textemdash\hspace{2pt}\myauthor}}
\end{center}
\setlist{topsep=-1em, itemsep=-1em, parsep=2em}

%%%%%%%%%%%%%%%%%%%%%%%%%%%%%%%%%%%%%%%%%%%%%%%%%%%%%%
%                      Begin                         %
%%%%%%%%%%%%%%%%%%%%%%%%%%%%%%%%%%%%%%%%%%%%%%%%%%%%%%
\begin{enumerate}[Q\arabic*.]
  \item
    \begin{enumerate}[(\alph*)]
      \item $A$ is a line passing through $\begin{bmatrix}1\\1\\1\end{bmatrix}$ and parallel to $\begin{bmatrix}1\\2\\3\end{bmatrix}$
      \item Let $x=1+t, y=1+2t, z=1+3t$
        \begin{align*}
          x+y-z &= (1+t) + (1+2t) - (1+3t)\\
                &= 1\\
          x-2y+z &= (1+t) - 2(1+2t) + (1+3t)\\
                 &= 0
        \end{align*}
        $\therefore A = \{(x, y, z) | x + y - z = 1 \land x - 2y + z = 0\} \qed$
      \item $M = \begin{bmatrix}1&1&-1\\1&-2&1\\0&0&0\end{bmatrix}, \vec{b} = \begin{bmatrix}1\\0\\0\end{bmatrix}$
    \end{enumerate}
  \pagebreak
  \item 
    \begin{enumerate}[(\alph*)]
      \item 
        \begin{enumerate}[(\roman*)]
          \item 
          \begin{align*}
            \begin{bmatrix}
              2 & 3 & -1 & 2\\
              1 & -1 & 0 & 3\\
              0 & 5 & 2 & -7\\
              3 & 2 & 1 & 3
            \end{bmatrix}\xrightarrow{RREF}
            \begin{bmatrix}
              1 & 0 & 0 & 2\\
              0 & 1 & 0 & -1\\
              0 & 0 & 1 & -1\\
              0 & 0 & 0 & 0
            \end{bmatrix}
          \end{align*}
          $\therefore \begin{bmatrix}2\\3\\-7\\3\end{bmatrix} = 2\vec{u_1} - \vec{u_2} - \vec{u_3} \qed$

          \item 
          $\begin{bmatrix}0\\0\\0\\0\end{bmatrix} = 0\vec{u_1} + 0\vec{u_2} + 0\vec{u_3} \qed$

        \item \begin{align*}
            \begin{bmatrix}
              2 & 3 & -1 & 1\\
              1 & -1 & 0 & 1\\
              0 & 5 & 2 & 1\\
              3 & 2 & 1 & 1
            \end{bmatrix}\xrightarrow{RREF}
            \begin{bmatrix}
              1 & 0 & 0 & 0\\
              0 & 1 & 0 & 0\\
              0 & 0 & 1 & 0\\
              0 & 0 & 0 & 1
            \end{bmatrix}
          \end{align*}
          $\therefore \begin{bmatrix}1\\1\\1\\1\end{bmatrix}$ cannot be expressed as a linear combination of $\vec{u_1}, \vec{u_2}, \vec{u_3} \qed$
          
        \item 
          \begin{align*}
            \begin{bmatrix}
              2 & 3 & -1 & -4\\
              1 & -1 & 0 & 6\\
              0 & 5 & 2 & -13\\
              3 & 2 & 1 & 4
            \end{bmatrix}\xrightarrow{RREF}
            \begin{bmatrix}
              1 & 0 & 0 & 3\\
              0 & 1 & 0 & -3\\
              0 & 0 & 1 & 1\\
              0 & 0 & 0 & 0
            \end{bmatrix}
          \end{align*}
          $\therefore \begin{bmatrix}-4\\6\\-13\\4\end{bmatrix} = 3\vec{u_1} - 3\vec{u_2} + \vec{u_3} \qed$
        \end{enumerate}
      \item Yes $\qed$
    \end{enumerate}
  \pagebreak 
  \item 
    \begin{enumerate}[(\alph*)]
    \item 
        $\begin{bmatrix}1\\1\\0\end{bmatrix}, \begin{bmatrix}5\\2\\3\end{bmatrix}$ satisfy $x-y-z=0$\\
        $\implies \begin{bmatrix}1\\1\\0\end{bmatrix}, \begin{bmatrix}5\\2\\3\end{bmatrix} \in V$\\
        $\implies$ by closure over addition and multiplication in subspaces, $\Span(S) \subseteq V$\\
        \hfill\\
        $x-y-z\implies x=y+z$\\
        $\implies \begin{bmatrix}x\\y\\z\end{bmatrix} = \begin{bmatrix}y+z\\y\\z\end{bmatrix} = y\begin{bmatrix}1\\1\\0\end{bmatrix} + z\begin{bmatrix}1\\0\\1\end{bmatrix}$\\
        $\implies V = \Span \Bigl\{\begin{bmatrix}1\\1\\0\end{bmatrix}, \begin{bmatrix}1\\0\\1\end{bmatrix}\Bigr\}$
        \begin{align*}
          \begin{bmatrix}
            1 & 5 & 1 & 1\\
            1 & 2 & 1 & 0\\
            0 & 3 & 0 & 1
          \end{bmatrix}\xrightarrow{RREF}
          \begin{bmatrix}
            1 & 0 & 1 & -2/3\\
            0 & 1 & 0 & 1/3\\
            0 & 0 & 0 & 0
          \end{bmatrix}
        \end{align*}
        $\implies V \subseteq \Span(S)$\\
        $\therefore \Span(S) = V \qed$

     \item 
       \begin{align*}
         \begin{bmatrix}
           1 & 5 & 0\\
           1 & 2 & 0\\
           0 & 3 & 1
         \end{bmatrix}\xrightarrow{RREF}
         \begin{bmatrix}
           1 & 0 & 0\\
           0 & 1 & 0\\
           0 & 0 & 1
         \end{bmatrix}
       \end{align*}
       $\therefore \rank(T) = 3 \implies \Span(T) = \RR^3 \qed$
    \end{enumerate}

  \item
    \begin{enumerate}[(\alph*)]
      \item 
        \begin{align*}
          \begin{bmatrix}
            1 & 0 & 1 & 1\\
            0&1&1&1\\
            0&0&1&1\\
            1&0&1&0
          \end{bmatrix}\xrightarrow{RREF}
          \begin{bmatrix}
            1 & 0 & 0 & 0\\
            0 & 1 & 0 & 0\\
            0 & 0 & 1 & 0\\
            0 & 0 & 0 & 1
          \end{bmatrix}
        \end{align*}
        $\therefore S$ spans $\RR^4 \qed$
      \item Since $S$ has only 3 vectors, $S$ does not span $\RR^4 \qed$

      \item 
        \begin{align*}
          \begin{bmatrix}
            6&2&3&5&0\\
            4&0&2&6&4\\
            -2&0&-1&-3&-2\\
            4&1&2&2&-1
          \end{bmatrix}\xrightarrow{RREF}
          \begin{bmatrix}
            1&0&1/2&0&-1/2\\
            0&1&0&0&-1\\
            0&0&0&1&1\\
            0&0&0&0&0
          \end{bmatrix}
        \end{align*}
        $\therefore S$ does not span $\RR^4 \qed$
      
      \item 
        \begin{align*}
          \begin{bmatrix}
            1&1&0&2&1\\
            1&2&0&1&2\\
            0&-1&1&2&3\\
            0&1&1&1&4
          \end{bmatrix}\xrightarrow{RREF}
          \begin{bmatrix}
            1&0&0&0&3\\
            0&1&0&0&0\\
            0&0&1&0&5\\
            0&0&0&1&-1
          \end{bmatrix}
        \end{align*}
        $\therefore S$ spans $\RR^4 \qed$
    \end{enumerate}

  \item
    \begin{enumerate}[(\alph*)]
      \item 
        \begin{align*}
          \begin{bmatrix}
            2&-1&0&|&1&0\\
            -2&1&0&|&-1&1\\
            0&-1&9&|&-5&1
          \end{bmatrix}\xrightarrow{RREF}
          \begin{bmatrix}
            1&0&-\frac{9}{2}&|&3&0\\
            0&1&-9&|&5&0\\
            0&0&0&|&0&1
          \end{bmatrix}
        \end{align*}
        $\therefore \Span\{\vec{v_1}, \vec{v_2}\} \not\subseteq \Span\{\vec{u_1}, \vec{u_2}, \vec{u_3}\} \qed$
        \begin{align*}
          \begin{bmatrix}
            1&0&|&2&-1&0\\
            -1&1&|&-2&1&0\\
            -5&1&|&0&-1&9
          \end{bmatrix}\xrightarrow{RREF}
          \begin{bmatrix}
            1&0&|&\frac{1}{5}&-\frac{9}{5}\\
            0&1&|&0&0&0\\
            0&0&|&1&-\frac{3}{5}&\frac{9}{10}
          \end{bmatrix}
        \end{align*}
        $\therefore \Span\{\vec{u_1}, \vec{u_2}, \vec{u_3}\} \not\subseteq \Span\{\vec{v_1}, \vec{v_2}\} \qed$
      
      \item 
        \begin{align*}
          \begin{bmatrix}
            1&2&-1&|&1&0\\
            6&4&2&|&-2&8\\
            4&-1&5&|&-5&9
          \end{bmatrix}\xrightarrow{RREF}
          \begin{bmatrix}
            1&0&1&|&-1&2\\
            0&1&-1&|&1&-1\\
            0&0&0&|&0&0
          \end{bmatrix}
        \end{align*}
        $\therefore \Span\{\vec{v_1}, \vec{v_2}\} \subseteq \Span\{\vec{u_1}, \vec{u_2}, \vec{u_3}\} \qed$
        \begin{align*}
          \begin{bmatrix}
            1&0&|&1&2&-1\\
            -2&8&|&6&4&2\\
            -5&9&|&4&-1&5
          \end{bmatrix}\xrightarrow{RREF}
          \begin{bmatrix}
            1&0&|&1&2&-1\\
            0&1&|&1&1&0\\
            0&0&|&0&0&0
          \end{bmatrix}
        \end{align*}
        $\therefore \Span\{\vec{u_1}, \vec{u_2}, \vec{u_3}\} \subseteq \Span\{\vec{v_1}, \vec{v_2}\} \qed$
    \end{enumerate}
  \pagebreak
  \item
    \begin{enumerate}[(\alph*)]
      \item $S = \Span\Bigl\{\begin{bmatrix}1\\0\\1\\0\end{bmatrix}, \begin{bmatrix}0\\1\\0\\1\end{bmatrix} \Bigr\} \qed$
      \item $S$ is not a linear span/subspace since $\begin{bmatrix}3\\2\\1\end{bmatrix} \in S \land -\begin{bmatrix}3\\2\\1\end{bmatrix} \not\in S \qed$
      \item $S = \Span\Bigl\{\begin{bmatrix}1\\4/3\\0\\-2/3\end{bmatrix}, \begin{bmatrix}0\\0\\1\\0\end{bmatrix} \Bigr\} \qed$
      \item 
        \begin{align*}
          \begin{vmatrix}
            1&0&1&0\\
            0&1&0&0\\
            1&0&0&1\\
            a&b&c&d
          \end{vmatrix} &= a-c-d\\
          \implies S &= \Bigl\{\begin{bmatrix}a\\b\\c\\d\end{bmatrix}: a - c- d = 0\Bigr\}\\
           \implies S &= \Bigl\{\begin{bmatrix}s+t\\u\\s\\t\end{bmatrix}: s,t,u \in \RR\Bigr\}\\
        \end{align*}
        $\therefore S = \Span\Bigl\{\begin{bmatrix}1\\0\\1\\0\end{bmatrix}, \begin{bmatrix}1\\0\\0\\1\end{bmatrix}, \begin{bmatrix}0\\1\\0\\0\end{bmatrix} \Bigr\} \qed$
      \item $S = \Span\Bigl\{\begin{bmatrix}-1\\1\\0\\0\end{bmatrix}, \begin{bmatrix}1\\0\\1\\0\end{bmatrix}, \begin{bmatrix}1\\0\\0\\1\end{bmatrix} \Bigr\} \qed$
      \item $S$ is not a linear span/subspace since $\begin{bmatrix}2\\0\\1\\0\end{bmatrix}, \begin{bmatrix}0\\2\\0\\1\end{bmatrix} \in S \land \begin{bmatrix}2\\0\\1\\0\end{bmatrix} + \begin{bmatrix}0\\2\\0\\1\end{bmatrix}=\begin{bmatrix}2\\2\\1\\1\end{bmatrix} \not\in S \qed$
      \item 
        \begin{align*}
          \begin{bmatrix}
            2&2&-1&0&1&|&0\\
            -1&-1&2&-3&1&|&0\\
            0&0&1&1&1&|&0\\
            1&1&-2&0&-1&|&0
          \end{bmatrix}&\xrightarrow{RREF}
          \begin{bmatrix}
            1&1&0&0&1&|&0\\
            0&0&1&0&1&|&0\\
            0&0&0&1&0&|&0\\
            0&0&0&0&0&|&0
          \end{bmatrix}\\
          \vec{x} &= \Bigl\{\begin{bmatrix}-s-t\\s\\-t\\0\\t\end{bmatrix}: s, t \in \RR \Bigr\}
        \end{align*}
        $\therefore S = \Span\Bigl\{\begin{bmatrix}-1\\1\\0\\0\\0\end{bmatrix}, \begin{bmatrix}-1\\0\\-1\\0\\1\end{bmatrix} \Bigr\} \qed$
    \end{enumerate}
\end{enumerate}
%%%%%%%%%%%%%%%%%%%%%%%%%%%%%%%%%%%%%%%%%%%%%%%%%%%%%%
%                       End                          %
%%%%%%%%%%%%%%%%%%%%%%%%%%%%%%%%%%%%%%%%%%%%%%%%%%%%%%

\end{document}

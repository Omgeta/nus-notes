\documentclass[12pt, a4paper]{article}

\usepackage[a4paper, margin=1in]{geometry}

\usepackage[utf8]{inputenc}
\usepackage[mathscr]{euscript}
\let\euscr\mathscr \let\mathscr\relax
\usepackage[scr]{rsfso}
\usepackage{amssymb,amsmath,amsthm,amsfonts}
\usepackage[shortlabels]{enumitem}
\usepackage{multicol,multirow}
\usepackage{lipsum}
\usepackage{balance}
\usepackage{calc}
\usepackage[colorlinks=true,citecolor=blue,linkcolor=blue]{hyperref}
\usepackage{import}
\usepackage{xifthen}
\usepackage{pdfpages}
\usepackage{transparent}
\usepackage{tabularx}

\newcommand{\incfig}[2][1.0]{
    \def\svgwidth{#1\columnwidth}
    \import{./figures/}{#2.pdf_tex}
}
\newcommand{\incimg}[2][1.0]{
  \includegraphics[width=#1\columnwidth]{./figures/#2}
}


\input{letterfonts}

\newcommand{\mytitle}{MA1522 Tutorial 7}
\newcommand{\myauthor}{github/omgeta}
\newcommand{\mydate}{AY 24/25 Sem 1}

\begin{document}
\raggedright
\footnotesize
\begin{center}
{\normalsize{\textbf{\mytitle}}} \\
{\footnotesize{\mydate\hspace{2pt}\textemdash\hspace{2pt}\myauthor}}
\end{center}
\setlist{topsep=-1em, itemsep=-1em, parsep=2em}

%%%%%%%%%%%%%%%%%%%%%%%%%%%%%%%%%%%%%%%%%%%%%%%%%%%%%%
%                      Begin                         %
%%%%%%%%%%%%%%%%%%%%%%%%%%%%%%%%%%%%%%%%%%%%%%%%%%%%%%
\begin{enumerate}[Q\arabic*.]
  \item 
    \begin{enumerate}[(\alph*)]
      \item $a = \begin{bmatrix}a_1\\a_2\\\vdots\\a_n\end{bmatrix}, \vec{x}=\begin{bmatrix}x_1\\x_2\\\vdots\\x_n\end{bmatrix} \qed$
      \item Reduce the corresponding matrix:
        \begin{align*}
          \left(\begin{array}{ccccc} 1 & 3 & -2 & 0 & 0\\ 2 & 6 & -5 & -2 & 0\\ 0 & 0 & 5 & 10 & 0 \end{array}\right)
          \xrightarrow{RREF}
          \left(\begin{array}{ccccc} 1 & 3 & 0 & 4 & 0\\ 0 & 0 & 1 & 2 & 0\\ 0 & 0 & 0 & 0 & 0 \end{array}\right)
        \end{align*}
        Then, the general solution is $s\left(\begin{array}{c} -3\\ 1\\ 0\\ 0 \end{array}\right) + t\left(\begin{array}{c} -4\\ -2\\ 0\\ 1 \end{array}\right)$ where $s,t \in \RR \qed$

      \item This is equivalent to solving the system:
        \begin{align*}
          v_1 + 3v_2 -2v_3 &=0\\
          2v_1 + 6v_2 -5v_3 - 2v_4 & = 0\\
          5v_3 + 10v_4 &=0
        \end{align*}
        From (b), choose $s=1, t=0$ then $\vec{v} = \left(\begin{array}{c} -3\\ 1\\0\\ 0 \end{array}\right) \qed$
    \end{enumerate}

  \item 
    \begin{enumerate}[(\alph*)]
      \item
        \begin{align*}
          \vec{x} \cdot \vec{y} &= (\vec{v_1} -2\vec{v_2}-2 \vec{v_3}) \cdot (2 \vec{v_1} -3 \vec{v_2} + \vec{v})\\
                                &= 2(\vec{v_1} \cdot \vec{v_1}) + 6(\vec{v_2} \cdot \vec{v_2})-2(\vec{v_3}\cdot \vec{v_3})\\
                                &= 2+6-2\\
                                &= 6\qed
        \end{align*}

      \item 
        \begin{align*}
          ||\vec{x}|| &= \sqrt{\vec{x} \cdot \vec{x}}\\
                      &= \sqrt{(\vec{v_1} \cdot \vec{v_1}) + 4(\vec{v_2} + \vec{v_2}) + 4(\vec{v_3} + \vec{v_3})}\\
                      &= \sqrt{1+4+4}\\
                      &= 3 \qed\\
          ||\vec{y}|| &= \sqrt{\vec{y} \cdot \vec{y}}\\
                      &= \sqrt{4(\vec{v_1} \cdot \vec{v_1}) + 9(\vec{v_2} + \vec{v_2}) + (\vec{v_3} + \vec{v_3})}\\
                      &= \sqrt{4+9+1}\\
                      &= \sqrt{14} \qed
        \end{align*}

      \item 
        \begin{align*}
          \theta &= \cos^{-1} \frac{6}{3\sqrt{14}}\\
                 &= 57.69^{\circ}\qed
        \end{align*}
    \end{enumerate}

  \item 
    \begin{enumerate}[(\alph*)]
      \item 
        \begin{align*}
          \vec{v_1} \cdot \vec{v_1} &= 6 \qed\\
          \vec{v_1} \cdot \vec{v_2} &= 0 \qed\\
          \vec{v_2} \cdot \vec{v_1} &= 0 \qed\\
          \vec{v_2} \cdot \vec{v_2} &= 2 \qed\\
        \end{align*}

      \item $V^TV = \left(\begin{array}{cc} 6 & 0\\ 0 & 2 \end{array}\right)$ represents $\left(\begin{array}{cc} \vec{v_1}\cdot \vec{v_1} & \vec{v_1} \cdot \vec{v_2}\\ \vec{v_2}\cdot \vec{v_1} & \vec{v_2}\cdot \vec{v_2} \end{array}\right) \qed$
    \end{enumerate}


  \item 
    \begin{enumerate}[(\alph*)]
      \item Let $A = \left(\begin{array}{ccc} 1 & 1 & 1\\ 1 & 2 & -1\\ 1 & -1 & 1\\ 1 & -2 & -1\\ 1 & 0 & 0 \end{array}\right)$, then $A^TA = \left(\begin{array}{ccc} 5 & 0 & 0\\ 0 & 10 & 0\\ 0 & 0 & 4 \end{array}\right)$ which shows $S$ is a set of orthongonal non-zero vectors which is automatically linearly independent$\qed$

      \item Shown in (a)

      \item By Q1, $W^T = \nul(A^T)$ which is a subspace. To find $W^T$ reduce $A^T$:
        \begin{align*}
          \left(\begin{array}{ccccc} 1 & 1 & 1 & 1 & 1\\ 1 & 2 & -1 & -2 & 0\\ 1 & -1 & 1 & -1 & 0 \end{array}\right)
          \xrightarrow{RREF}
          \left(\begin{array}{ccccc} 1 & 0 & 0 & -2 & -\frac{1}{4}\\ 0 & 1 & 0 & 1 & \frac{1}{2}\\ 0 & 0 & 1 & 2 & \frac{3}{4} \end{array}\right)
        \end{align*}
        Therefore, $\dim(W^T) = 2 \qed$

      \item From (a), $T = \left\{\frac{1}{\sqrt{5}}\left(\begin{array}{c} 1\\ 1\\ 1\\ 1\\ 1 \end{array}\right), \frac{1}{\sqrt{10}}\left(\begin{array}{c} 1\\ 2\\ -1\\ -1\\ 0 \end{array}\right), \frac{1}{2}\left(\begin{array}{c} 1\\ -1\\ 1\\ -1\\ 0 \end{array}\right)\right\} \qed$

      \item $\displaystyle \frac{\vec{v}\cdot \vec{w_1}}{||\vec{w_1}||^2}\vec{w_1} + \frac{\vec{v}\cdot \vec{w_2}}{||\vec{w_2}||^2}\vec{w_2} + \frac{\vec{v}\cdot \vec{w_3}}{||\vec{w_3}||^2}\vec{w_3} = \frac{1}{10}\left(\begin{array}{c} 10\\ -1\\ 12\\ 3\\ 6 \end{array}\right) \qed$ 

      \item 
        \begin{align*}
          A^T(\vec{v} - \vec{v_W}) &= \left(\begin{array}{c} 0\\ 0\\ 0 \end{array}\right)
        \end{align*}
        Therefore, $(\vec{v} - \vec{v_W}) \in W^{\perp}$
    \end{enumerate}

  \pagebreak
  \item 
    \begin{enumerate}[(\alph*)]
      \item Let $U = \left(\begin{array}{cccc} 1 & 1 & -1 & -2\\ 2 & 1 & 1 & 1\\ 2 & -1 & -1 & 1\\ -1 & 1 & -1 & 2 \end{array}\right)$
        \begin{align*}
          U^TU &= \left(\begin{array}{cccc} 10 & 0 & 0 & 0\\ 0 & 4 & 0 & 0\\ 0 & 0 & 4 & 0\\ 0 & 0 & 0 & 10 \end{array}\right)
        \end{align*}
        Since $U^TU$ is a scalar matrix, $S$ is an orthogonal set with linearly independent vectors. Since $|S| = 4 = \dim(\RR^4)$, $\Span(S) = \RR^4$. Therefore, $S$ is a basis for $\RR^4 \qed$

      \item No; because it is not possible to have a set of $5$ linearly independent vectors in $\RR^4 \qed$

      \item From (a), $T = \left\{\frac{1}{\sqrt{10}}\left(\begin{array}{c} 1\\ 2\\ 2\\ -1 \end{array}\right), \frac{1}{2}\left(\begin{array}{c} 1\\ 1\\ -1\\ 1 \end{array}\right), \frac{1}{2}\left(\begin{array}{c} -1\\ 1\\ -1\\ -1 \end{array}\right), \frac{1}{\sqrt{10}}\left(\begin{array}{c} -2\\ 1\\ 1\\ 2 \end{array}\right)\right\} \qed$

      \item 
        \begin{align*}
          [\vec{v}]_S &= (U^TU)^{-1}U^T \vec{V}
                      = \left(\begin{array}{c} \frac{3}{10}\\ \frac{1}{2}\\ -1\\ \frac{9}{10} \end{array}\right) \qed\\
          [\vec{v}]_T &= \left(\begin{array}{c} \vec{v}\cdot u_1'\\ \vec{v}\cdot u_2'\\ \vec{v}\cdot u_3'\\ \vec{v}\cdot u_4'\end{array}\right) = \left(\begin{array}{c} \frac{3}{\sqrt{10}}\\ 1\\ -2\\ \frac{9}{\sqrt{10}} \end{array}\right) \qed
        \end{align*}
    \end{enumerate}
\end{enumerate}
%%%%%%%%%%%%%%%%%%%%%%%%%%%%%%%%%%%%%%%%%%%%%%%%%%%%%%
%                       End                          %
%%%%%%%%%%%%%%%%%%%%%%%%%%%%%%%%%%%%%%%%%%%%%%%%%%%%%%

\end{document}

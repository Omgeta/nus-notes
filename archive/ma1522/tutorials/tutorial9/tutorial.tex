\documentclass[12pt, a4paper]{article}

\usepackage[a4paper, margin=1in]{geometry}

\usepackage[utf8]{inputenc}
\usepackage[mathscr]{euscript}
\let\euscr\mathscr \let\mathscr\relax
\usepackage[scr]{rsfso}
\usepackage{amssymb,amsmath,amsthm,amsfonts}
\usepackage[shortlabels]{enumitem}
\usepackage{multicol,multirow}
\usepackage{lipsum}
\usepackage{balance}
\usepackage{calc}
\usepackage[colorlinks=true,citecolor=blue,linkcolor=blue]{hyperref}
\usepackage{import}
\usepackage{xifthen}
\usepackage{pdfpages}
\usepackage{transparent}
\usepackage{tabularx}

\newcommand{\incfig}[2][1.0]{
    \def\svgwidth{#1\columnwidth}
    \import{./figures/}{#2.pdf_tex}
}
\newcommand{\incimg}[2][1.0]{
  \includegraphics[width=#1\columnwidth]{./figures/#2}
}


\input{letterfonts}

\newcommand{\mytitle}{MA1522 Tutorial 9}
\newcommand{\myauthor}{github/omgeta}
\newcommand{\mydate}{AY 24/25 Sem 1}

\begin{document}
\raggedright
\footnotesize
\begin{center}
{\normalsize{\textbf{\mytitle}}} \\
{\footnotesize{\mydate\hspace{2pt}\textemdash\hspace{2pt}\myauthor}}
\end{center}
\setlist{topsep=-1em, itemsep=-1em, parsep=2em}

%%%%%%%%%%%%%%%%%%%%%%%%%%%%%%%%%%%%%%%%%%%%%%%%%%%%%%
%                      Begin                         %
%%%%%%%%%%%%%%%%%%%%%%%%%%%%%%%%%%%%%%%%%%%%%%%%%%%%%%
\begin{enumerate}[Q\arabic*.]
  \item Let $x_0, x_1, x_2$ be the amount Jack, Jim, and John received respectively
    \begin{enumerate}[(\alph*)]
      \item Conditions create the system of equations:
        \begin{align*}
          x_0 + 2x_1 &= 300\\
          x_1 + x_2 &= 300\\
          x_0 - 2x_2 &= 300
        \end{align*}
        which is inconsistent $\qed$

      \item Let $A = \left(\begin{array}{ccc} 1 & 2 & 0\\ 0 & 1 & 1\\ 1 & 0 & -2 \end{array}\right), \vec{b} = \left(\begin{array}{c} 300\\ 300\\ 300 \end{array}\right)$, solve $A^TA \vec{x} = A^T \vec{b}$:
        \begin{align*}
          [A^TA \mid A^T \vec{b}] = \left(\begin{array}{ccccc} 2 & 2 & -2 &|& 600\\ 2 & 5 & 1 &|& 900\\ -2 & 1 & 5 &|& -300 \end{array}\right) \xrightarrow{RREF} \left(\begin{array}{ccccc} 1 & 0 & -2 &|& 200\\ 0 & 1 & 1 &|& 100\\ 0 & 0 & 0 &|& 0 \end{array}\right)
        \end{align*}
        so a least square solution $\vec{x} = \left(\begin{array}{c} 200\\ 100\\0 \end{array}\right) \qed$
    \end{enumerate}

  \item 
    \begin{enumerate}[(\alph*)]
    \item Suppose $Q = [q_1 \cdots q_n]$, where the columns of $Q$ form an orthonormal set in $\RR^m$, extend it to an orthonormal basis of $\RR^m$ to form the orthogonal matrix $Q' = [q_1 \cdots q_m]$. Then let $R' = \left(\begin{array}{c} R\\ 0_{(m-n)\times m} \end{array}\right) \qed$

    \item $Q = \left(\begin{array}{cccc} \frac{\sqrt{3}}{3} & 0 & -\frac{\sqrt{6}}{6} & \frac{\sqrt{2}}{2}\\ \frac{\sqrt{3}}{3} & 0 & -\frac{\sqrt{6}}{6} & -\frac{\sqrt{2}}{2}\\ \frac{\sqrt{3}}{3} & 0 & \frac{\sqrt{6}}{3} & 0\\ 0 & 1 & 0 & 0 \end{array}\right), R = \left(\begin{array}{ccc} \sqrt{3} & \sqrt{3} & \frac{\sqrt{3}}{3}\\ 0 & 1 & 1\\ 0 & 0 & \frac{\sqrt{6}}{3}\\ 0 & 0 & 0 \end{array}\right) \qed$

    \item Use $[Q R] = qr(A, x) \qed$
    \end{enumerate}

  \item 
    \begin{enumerate}[(\alph*)]
      \item $p(X) = 0^{3\times 3} \qed$

      \item $\det(X) = x^3 - 4x^2 - x + 4 = p(x) \qed$

      \item $X^3 -4X^2 -X + 4I = 0 \implies X(X^2-4X-I) = -4I \implies X^{-1} = -\frac{1}{4}(X^2-4X-I) \qed$
    \end{enumerate}

  \item 
    \begin{enumerate}[(\alph*)]
      \item $P = \left(\begin{array}{ccc} \frac{1}{2} & 1 & -1\\ \frac{1}{2} & 1 & 0\\ 1 & 0 & 1 \end{array}\right), P^{-1}AP = \left(\begin{array}{ccc} 4 & 0 & 0\\ 0 & -2 & 0\\ 0 & 0 & -2 \end{array}\right) \qed$

      \item $P = \left(\begin{array}{cccc} -2 & \frac{5}{6} & 1 & -\frac{4}{5}\\ 1 & -1 & 0 & 1\\ 1 & 0 & 0 & 0\\ 0 & 1 & 0 & 0 \end{array}\right), P^{-1}AP = \left(\begin{array}{cccc} 2 & 0 & 0 & 0\\ 0 & 3 & 0 & 0\\ 0 & 0 & 9 & 0\\ 0 & 0 & 0 & -1 \end{array}\right) \qed$

      \item Algebraic multiplicity of eigenvalue $1 = 3$ but geometric multiplicity $= \dim(\nul(A-I)) = 1$ so there is no diagonalization $\qed$

      \item $P = \left(\begin{array}{cccc} 1 & 0 & -1 & 0\\ 0 & 1 & 0 & -1\\ 1 & 0 & 1 & 0\\ 0 & 1 & 0 & 1 \end{array}\right), P^{-1}AP = \left(\begin{array}{cccc} 1 & 0 & 0 & 0\\ 0 & 1 & 0 & 0\\ 0 & 0 & -1 & 0\\ 0 & 0 & 0 & -1 \end{array}\right) \qed$

      \item Characteristic equation does not factor into linear factors so there is no diagonalization $\qed$
    \end{enumerate}

  \item 
    \begin{enumerate}[(\alph*)]
      \item $\det(A - \lambda I) = \det((A-\lambda I)^{T}) = \det(A^T - \lambda I)$ so their characteristic equations are the same $\qed$

      \item If $A = PDP^{-1}$, then $A^T = (PDP^{-1})^T = (P^{-1})^TD^TP^T = QDQ^{-1}$ since $D$ is diagonal and we let $Q = P^{-1}$, so there is a diagonalization for $A^T \qed$

      \item $A^k\vec{v} = A^{k-1}A\vec{v} = A^{k-1}\lambda \vec{v} = \cdots = \lambda^k \vec{v}$ so by definition $\lambda^k$ is an eigenvalue for $\vec{v}$ in $A^k \qed$

      \item $A \vec{v} = \lambda \vec{v} \implies \lambda^{-1} \vec{v} = A^{-1} \vec{v}$ so by definition and from (c), we have proven $\lambda^k$ is an eigenvalue for $\vec{v}$ in $A^k$ for negative $k \qed$ 

      \item $A^k \vec{v} = \vec{0} \implies \lambda^k \vec{v} = \vec{0}$, and since $\vec{v} \neq 0$, $\lambda = 0 \qed$

      \item If $\lambda$ is the only eigenvalue, then for diagoanlisation $A=PDP^{-1}$, $D=\lambda I$ so then $A = P\lambda I P^{-1} = \lambda PIP^{-1} = \lambda I \qed$

      \item $A$ is nilpotent $\implies 0$ is the only eigenvalue, therefore $D$ is the zero matrix and $A = P 0 P^{-1} = 0 \qed$
    \end{enumerate}
\end{enumerate}
%%%%%%%%%%%%%%%%%%%%%%%%%%%%%%%%%%%%%%%%%%%%%%%%%%%%%%
%                       End                          %
%%%%%%%%%%%%%%%%%%%%%%%%%%%%%%%%%%%%%%%%%%%%%%%%%%%%%%

\end{document}

\documentclass[12pt, a4paper]{article}

\usepackage[a4paper, margin=1in]{geometry}

\usepackage[utf8]{inputenc}
\usepackage[mathscr]{euscript}
\let\euscr\mathscr \let\mathscr\relax
\usepackage[scr]{rsfso}
\usepackage{amssymb,amsmath,amsthm,amsfonts}
\usepackage[shortlabels]{enumitem}
\usepackage{multicol,multirow}
\usepackage{lipsum}
\usepackage{balance}
\usepackage{calc}
\usepackage[colorlinks=true,citecolor=blue,linkcolor=blue]{hyperref}
\usepackage{import}
\usepackage{xifthen}
\usepackage{pdfpages}
\usepackage{transparent}
\usepackage{tabularx}

\newcommand{\incfig}[2][1.0]{
    \def\svgwidth{#1\columnwidth}
    \import{./figures/}{#2.pdf_tex}
}
\newcommand{\incimg}[2][1.0]{
  \includegraphics[width=#1\columnwidth]{./figures/#2}
}


\input{letterfonts}

\newcommand{\mytitle}{CS1231S Tutorial 1}
\newcommand{\myauthor}{github/omgeta}
\newcommand{\mydate}{AY 24/25 Sem 1}

\begin{document}
\raggedright
\footnotesize
\begin{center}
{\normalsize{\textbf{\mytitle}}} \\
{\footnotesize{\mydate\hspace{2pt}\textemdash\hspace{2pt}\myauthor}}
\end{center}
\setlist{topsep=-1em, itemsep=-1em, parsep=2em}
\renewcommand{\neg}{\mathord{\sim}}
%%%%%%%%%%%%%%%%%%%%%%%%%%%%%%%%%%%%%%%%%%%%%%%%%%%%%%
%                      Begin                         %
%%%%%%%%%%%%%%%%%%%%%%%%%%%%%%%%%%%%%%%%%%%%%%%%%%%%%%
\begin{enumerate}[Q\arabic*.]
  \item Let $p$ be "I use the umbrella", $q$ be "it rains"
  \begin{enumerate}[(\alph*)]
    \item "I use the umbrella if it rains" $\equiv q \rightarrow p$ \\
      "I use the umbrella only if it rains" $\equiv p \rightarrow q$
    \item In "I use the umbrella if it rains": \\
      "it rains" is sufficient for "I use the umbrella" \\
      "I use the umbrella" is necessary for "it rains"
    \item "I use the umbrella if and only if it rains"$\equiv (q \rightarrow p) \land (p \rightarrow q) \equiv p \leftrightarrow q$
    \item In "I use the umbrella if and only if it rains": \\
      "I use the umbrella" is a necessary and sufficient condition for "it rains"
  \end{enumerate}

  \item The students do not preserve the brackets after applying De Morgan's law which makes the logical statement ambiguous as $\land$ and $\lor$ have equal precedence
    \begin{enumerate}[(\alph*)]
      \item $a \land \neg(b\land c) \equiv a \land (\neg b \lor \neg c)$ 
      \item $\neg(x \lor y) \lor z \equiv (\neg x \land \neg y) \lor z$ 
    \end{enumerate}

  \item
    \begin{enumerate}[(\alph*)]
      \item $\neg a \land (\neg a \rightarrow (b \land a))$\\
      $\equiv \neg a \land (\neg (\neg a) \lor (b \land a))$\hfill(Implication law)\\
      $\equiv \neg a \land (a \lor (b \land a))$ \hfill (Double negation law)\\
      $\equiv \neg a \land (a \lor (a \land b))$ \hfill (Commutative law)\\
      $\equiv \neg a \land (a)$ \hfill (Absorption law)\\
      $\equiv$ false \hfill (Negation law)\\

    \item $(p \lor \neg q) \rightarrow q$ \\
      $\equiv \neg (p \lor \neg q) \lor q$\hfill(Implication law)\\
      $\equiv (\neg p \land q) \lor q$\hfill(DeMorgan's law)\\
      $\equiv q \lor (q \land \neg p)$\hfill(Commutative law)\\
      $\equiv q$\hfill(Absorption law)

    \item $\neg(p\lor \neg q)\lor(\neg p \land \neg q)$\\
      $\equiv (\neg p \land q)\lor(\neg p \land \neg q)$\hfill(DeMorgan's law)\\
      $\equiv \neg p \land (q \lor \neg q)$\hfill(Distributive law)\\
      $\equiv \neg p \land true$\hfill(Negation law)\\
      $\equiv \neg p$\hfill(Identity law)

    \item $(p \rightarrow q)\rightarrow r$\\
      $\equiv (\neg p \lor q) \rightarrow r$\hfill(Implication law)\\
      $\equiv (\neg(\neg p \lor q)) \lor r$\hfill(Implication law)\\
      $\equiv (p \land \neg q) \lor r$\hfill(DeMorgan's law)\\
    \end{enumerate}

  \item Since the truth tables do not match, $(p \rightarrow q) \rightarrow r \not\equiv p \rightarrow (q \rightarrow r)$
  \begin{displaymath}
    \begin{array}{|c c c|c|c|c|c|}
      p & q & r & p \rightarrow q & q \rightarrow r & (p \rightarrow q) \rightarrow r & p \rightarrow (q \rightarrow r)\\ % Use & to separate the columns
      \hline % Put a horizontal line between the table header and the rest.
      T & T & T & T & T & T & T\\
      T & T & F & T & F & F & F\\
      T & F & T & F & T & T & T\\
      F & T & T & T & T & T & T\\
      T & F & F & F & T & T & T\\
      F & T & F & T & F & \mathbf{F} & \mathbf{T}\\
      F & F & T & T & T & T & T\\
      F & F & F & T & T & \mathbf{F} & \mathbf{T}\\
    \end{array}
  \end{displaymath}

  \vfill
\item Let $p$ be $12x - 7 = 29$, and $q$ be $x = 3$ \\
  Original: $p \rightarrow q$ \\
  Negation: $p \land \neg q$ \\
  Contrapositive: $\neg q \rightarrow \neg p$ \\
  Converse: $q \rightarrow p$ \\
  Inverse: $\neg p \rightarrow \neg q$ \\
  \hfill\\
  Suppose $12x - 7 = 29$, then $12x = 36$ and $x = 3$, which matches the conclusion. Therefore, the conditional statement is true.

  Suppose $x = 3$, then indeed $12x - 7 = 29$, which matches the conclusion. Therefore, the converse statement is also true.

  No, the converse and the inverse are logically equivalent because they are contrapositive of each other.
  \begin{displaymath}
    \begin{array}{|c c|c|c|}
      p & q & (q \rightarrow p) & (\neg p \rightarrow \neg q)\\ % Use & to separate the columns
      \hline % Put a horizontal line between the table header and the rest.
      T & T & T & T\\
      T & F & T & T\\
      F & T & F & F\\
      F & F & T & T\\
    \end{array}
  \end{displaymath}

\item Alternative 1, it is evident that the transitive rule of inference does not hold
  \begin{displaymath}
    \begin{array}{|c c c|c|c|c|}
      p & q & r & p \rightarrow_a q & q \rightarrow_a r & p \rightarrow_a r\\ % Use & to separate the columns
      \hline % Put a horizontal line between the table header and the rest.
      T & T & T & T & T & T\\
      T & T & F & T & F & F\\
      T & F & T & F & F & T\\
      F & T & T & F & T & F\\
      T & F & F & F & F & F\\
      F & F & T & F & F & F\\
      F & T & F & F & F & F\\
      F & F & F & F & F & F\\
    \end{array}
  \end{displaymath}

  Alternative 2, it is evident that the transitive rule of inference does not hold
  \begin{displaymath}
    \begin{array}{|c c c|c|c|c|}
      p & q & r & p \rightarrow_b q & q \rightarrow_b r & p \rightarrow_b r\\ % Use & to separate the columns
      \hline % Put a horizontal line between the table header and the rest.
      T & T & T & T & T & T\\
      T & T & F & T & F & F\\
      T & F & T & F & T & T\\
      F & T & T & T & T & T\\
      T & F & F & F & F & F\\
      F & F & T & F & T & T\\
      F & T & F & T & F & F\\
      F & F & F & F & F & F\\
    \end{array}
  \end{displaymath}

  Alternative 3, it is evident that the transitive rule of inference does not hold
  \begin{displaymath}
    \begin{array}{|c c c|c|c|c|}
      p & q & r & p \rightarrow_c q & q \rightarrow_c r & p \rightarrow_c r\\ % Use & to separate the columns
      \hline % Put a horizontal line between the table header and the rest.
      T & T & T & T & T & T\\
      T & T & F & T & F & F\\
      T & F & T & F & F & T\\
      F & T & T & F & T & F\\
      T & F & F & F & T & F\\
      F & F & T & T & F & F\\
      F & T & F & F & F & T\\
      F & F & F & T & T & T\\
    \end{array}
  \end{displaymath}

  \item
    \begin{enumerate}[(\alph*)]
      \item Let $p$ be "Sandra knows Java", $q$ be "Sandra knows C++"\\
        $p \land q$\\
        $\therefore q$ (By specialization)
      \item Let $p$ be "at least one of these two numbers is divisible by 6", $q$ be "product of these two numbers is divisible by 6" \\
        $p \rightarrow q$\\
        $\neg p$ \\
        $\therefore \neg q$ (Inverse error)
      \item Let $p$ be "there are as many rational numbers as there are irrational numbers", $q$ be "the set of all irrational numbers is infinite" \\
        $p \rightarrow q$\\
        $q$\\
        $\therefore p$ (Converse error)
      \item Let $p$ be "I get a Christmas bonus", $q$ be "I sell my motorcycle", $r$ be "I’ll buy a stereo" \\
        $p \rightarrow r$ \\
        $q \rightarrow r$\\
        $\therefore (p \lor q) \rightarrow r$ (By construction)
    \end{enumerate}

  \item
    \begin{enumerate}[(\alph*)]
      \item $\neg p \implies p =$ false \\
        Since $p \lor (q \land q) =$ true and $p =$ false, $(q \land r)$ must be true $\implies q = r =$ true \\
        Therefore, the conclusion is also true, and the argument is valid.
      \item Let $p =$ true, $q =$ false, $r =$ false \\
        Premise 1: $p \lor (q \land r) $is true\\
        Premise 2: $\neg(p \land q)$ is true\\
        Conclusion: $r$ is false
        Which shows that the argument is not valid
      \item Let $p$ be "I go to the beach", $q$ be "I will take my shades", $r$ be "I will take my sunscreen"\\
        $p \rightarrow (q \lor r)$\\
        $q$\\
        $\neg r$\\
        $\therefore p$ (Converse error, therefore the argument is invalid)
      \item Let $p$ be "I will buy a new goat", $q$ be "I will buy a used Yugo", $r$ be "I will need a loan"\\
        $p \lor q$\\
        $(p \land q) \rightarrow r$\\
        $q \land \neg r$\\
        $\therefore \neg p$ \\
        The argument is valid
    \end{enumerate}
  \item \begin{enumerate}[(\alph*)]
      \item \textbf{Proof (by contradiction).}
      \begin{enumerate}[label=\arabic*., itemsep=-2em]
          \item If $A$ is a knight, then:
            \begin{enumerate}[label=1.\arabic*, itemsep=-2em]
              \item What $A$ says is true. \hfill (by definition of knight)
              \item $\therefore B$ is a knight too. \hfill (that's what $A$ says)
              \item $\therefore$ What $B$ says is true. \hfill (by definition of knight)
              \item $\therefore A$ is a knave. \hfill (that's what $B$ says)
              \item $\therefore A$ is not a knight. 
              \item $\therefore$ Contradiction to 1.
          \end{enumerate}
          \item $\therefore A$ is not a knight.
          \item $\therefore A$ is a knave. \hfill (since $A$ is either a knight or a knave, but not both)
          \item $\therefore$ What $B$ says is true.
          \item $\therefore B$ cannot be a knave. \hfill (as $B$ has said something true)
          \item $\therefore B$ is a knight. \hfill (one is a knight or a knave)
      \end{enumerate}
      \item \textbf{Proof (by exhaustion).}
      \begin{enumerate}[label=\arabic*., itemsep=-2em]
          \item If $C$ is a knight, then:
          \begin{enumerate}[label=1.\arabic*, itemsep=-2em]
            \item What $C$ says is true. \hfill (by definition of knight)
            \item $\therefore D$ is a knave. \hfill (that's what $C$ says)
            \item $\therefore$ What $D$ says is false. \hfill (by definition of knave)
            \item $\therefore C$ is not a knave. \hfill (that's what $D$ says)
            \item $\therefore C$ is a knight.
            \item $\therefore$ there is no contradiction.
            \item $\therefore$ there is 1 knight and 1 knave.
          \end{enumerate}
          \item If $C$ is a knave, then:
          \begin{enumerate}[label=1.\arabic*, itemsep=-2em]
            \item What $C$ says is false. \hfill (by definition of knave)
            \item $\therefore D$ is not a knave. \hfill (that's what $C$ says)
            \item $\therefore D$ is a knight.
            \item $\therefore$ What $D$ says is true. \hfill (by definition of knight)
            \item $\therefore C$ is a knave. \hfill (that's what $D$ says)
            \item $\therefore$ there is no contradiction.
            \item $\therefore$ there is 1 knight and 1 knave.
          \end{enumerate}
          \item $\therefore$ there is always 1 knight and 1 knave. \hfill (in both cases)
      \end{enumerate}
    \end{enumerate}
  \item Let $x = 2n+1, y=2m+1$ be two odd numbers,
    \begin{align*}
      x\cdot y &= (2n+1)\cdot(2m+1) \\
               &= 4mn + 2n + 2m + 1 \\
               &= 2(2mn + m + n) + 1 \\
               &= 2k + 1 \text{, where $k=2mn+m+n \in \ZZ$} \\
               &\text{ is odd, by definition of odd numbers}
    \end{align*}
\end{enumerate}
%%%%%%%%%%%%%%%%%%%%%%%%%%%%%%%%%%%%%%%%%%%%%%%%%%%%%%
%                       End                          %
%%%%%%%%%%%%%%%%%%%%%%%%%%%%%%%%%%%%%%%%%%%%%%%%%%%%%%

\end{document}

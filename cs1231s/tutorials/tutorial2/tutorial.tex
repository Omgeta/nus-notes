\documentclass[12pt, a4paper]{article}

\usepackage[a4paper, margin=1in]{geometry}

\usepackage[utf8]{inputenc}
\usepackage[mathscr]{euscript}
\let\euscr\mathscr \let\mathscr\relax
\usepackage[scr]{rsfso}
\usepackage{amssymb,amsmath,amsthm,amsfonts}
\usepackage[shortlabels]{enumitem}
\usepackage{multicol,multirow}
\usepackage{lipsum}
\usepackage{balance}
\usepackage{calc}
\usepackage[colorlinks=true,citecolor=blue,linkcolor=blue]{hyperref}
\usepackage{import}
\usepackage{xifthen}
\usepackage{pdfpages}
\usepackage{transparent}
\usepackage{listings}

\newcommand{\incfig}[2][1.0]{
    \def\svgwidth{#1\columnwidth}
    \import{./figures/}{#2.pdf_tex}
}

\newlist{enumproof}{enumerate}{4}
\setlist[enumproof,1]{label=\arabic*., parsep=1em}
\setlist[enumproof,2]{label=\arabic{enumproofi}.\arabic*., parsep=1em}
\setlist[enumproof,3]{label=\arabic{enumproofi}.\arabic{enumproofii}.\arabic*., parsep=1em}
\setlist[enumproof,4]{label=\arabic{enumproofi}.\arabic{enumproofii}.\arabic{enumproofiii}.\arabic*., parsep=1em}

\renewcommand{\qedsymbol}{\ensuremath{\blacksquare}}

\lstdefinestyle{mystyle}{
  language=C, % Set the language to C
  commentstyle=\color{codegray}, % Color for comments
  keywordstyle=\color{orange}, % Color for basic keywords
  stringstyle=\color{mauve}, % Color for strings
  basicstyle={\ttfamily\footnotesize}, % Basic font style
  breakatwhitespace=false,         
  breaklines=true,                 
  captionpos=b,                    
  keepspaces=true,                 
  numbers=none,                    
  tabsize=2,
  morekeywords=[2]{\#include, \#define, \#ifdef, \#ifndef, \#endif, \#pragma, \#else, \#elif}, % Preprocessor directives
  keywordstyle=[2]\color{codegreen}, % Style for preprocessor directives
  morekeywords=[3]{int, char, float, double, void, struct, union, enum, const, volatile, static, extern, register, inline, restrict, _Bool, _Complex, _Imaginary, size_t, ssize_t, FILE}, % C standard types and common identifiers
  keywordstyle=[3]\color{identblue}, % Style for types and common identifiers
  morekeywords=[4]{printf, scanf, fopen, fclose, malloc, free, calloc, realloc, perror, strtok, strncpy, strcpy, strcmp, strlen}, % Standard library functions
  keywordstyle=[4]\color{cyan}, % Style for library functions
}

% Things Lie
\newcommand{\kb}{\mathfrak b}
\newcommand{\kg}{\mathfrak g}
\newcommand{\kh}{\mathfrak h}
\newcommand{\kn}{\mathfrak n}
\newcommand{\ku}{\mathfrak u}
\newcommand{\kz}{\mathfrak z}
\DeclareMathOperator{\Ext}{Ext} % Ext functor
\DeclareMathOperator{\Tor}{Tor} % Tor functor
\newcommand{\gl}{\opname{\mathfrak{gl}}} % frak gl group
\renewcommand{\sl}{\opname{\mathfrak{sl}}} % frak sl group chktex 6

% More script letters etc.
\newcommand{\SA}{\mathcal A}
\newcommand{\SB}{\mathcal B}
\newcommand{\SC}{\mathcal C}
\newcommand{\SF}{\mathcal F}
\newcommand{\SG}{\mathcal G}
\newcommand{\SH}{\mathcal H}
\newcommand{\OO}{\mathcal O}

\newcommand{\SCA}{\mathscr A}
\newcommand{\SCB}{\mathscr B}
\newcommand{\SCC}{\mathscr C}
\newcommand{\SCD}{\mathscr D}
\newcommand{\SCE}{\mathscr E}
\newcommand{\SCF}{\mathscr F}
\newcommand{\SCG}{\mathscr G}
\newcommand{\SCH}{\mathscr H}

% Mathfrak primes
\newcommand{\km}{\mathfrak m}
\newcommand{\kp}{\mathfrak p}
\newcommand{\kq}{\mathfrak q}

% number sets
\newcommand{\RR}[1][]{\ensuremath{\ifstrempty{#1}{\mathbb{R}}{\mathbb{R}^{#1}}}}
\newcommand{\NN}[1][]{\ensuremath{\ifstrempty{#1}{\mathbb{N}}{\mathbb{N}^{#1}}}}
\newcommand{\ZZ}[1][]{\ensuremath{\ifstrempty{#1}{\mathbb{Z}}{\mathbb{Z}^{#1}}}}
\newcommand{\QQ}[1][]{\ensuremath{\ifstrempty{#1}{\mathbb{Q}}{\mathbb{Q}^{#1}}}}
\newcommand{\CC}[1][]{\ensuremath{\ifstrempty{#1}{\mathbb{C}}{\mathbb{C}^{#1}}}}
\newcommand{\PP}[1][]{\ensuremath{\ifstrempty{#1}{\mathbb{P}}{\mathbb{P}^{#1}}}}
\newcommand{\HH}[1][]{\ensuremath{\ifstrempty{#1}{\mathbb{H}}{\mathbb{H}^{#1}}}}
\newcommand{\FF}[1][]{\ensuremath{\ifstrempty{#1}{\mathbb{F}}{\mathbb{F}^{#1}}}}
% expected value
\newcommand{\EE}{\ensuremath{\mathbb{E}}}
\newcommand{\charin}{\text{ char }}
\DeclareMathOperator{\sign}{sign}
\DeclareMathOperator{\Aut}{Aut}
\DeclareMathOperator{\Inn}{Inn}
\DeclareMathOperator{\Syl}{Syl}
\DeclareMathOperator{\Gal}{Gal}
\DeclareMathOperator{\GL}{GL} % General linear group
\DeclareMathOperator{\SL}{SL} % Special linear group

%---------------------------------------
% BlackBoard Math Fonts :-
%---------------------------------------

%Captital Letters
\newcommand{\bbA}{\mathbb{A}}	\newcommand{\bbB}{\mathbb{B}}
\newcommand{\bbC}{\mathbb{C}}	\newcommand{\bbD}{\mathbb{D}}
\newcommand{\bbE}{\mathbb{E}}	\newcommand{\bbF}{\mathbb{F}}
\newcommand{\bbG}{\mathbb{G}}	\newcommand{\bbH}{\mathbb{H}}
\newcommand{\bbI}{\mathbb{I}}	\newcommand{\bbJ}{\mathbb{J}}
\newcommand{\bbK}{\mathbb{K}}	\newcommand{\bbL}{\mathbb{L}}
\newcommand{\bbM}{\mathbb{M}}	\newcommand{\bbN}{\mathbb{N}}
\newcommand{\bbO}{\mathbb{O}}	\newcommand{\bbP}{\mathbb{P}}
\newcommand{\bbQ}{\mathbb{Q}}	\newcommand{\bbR}{\mathbb{R}}
\newcommand{\bbS}{\mathbb{S}}	\newcommand{\bbT}{\mathbb{T}}
\newcommand{\bbU}{\mathbb{U}}	\newcommand{\bbV}{\mathbb{V}}
\newcommand{\bbW}{\mathbb{W}}	\newcommand{\bbX}{\mathbb{X}}
\newcommand{\bbY}{\mathbb{Y}}	\newcommand{\bbZ}{\mathbb{Z}}

%---------------------------------------
% MathCal Fonts :-
%---------------------------------------

%Captital Letters
\newcommand{\mcA}{\mathcal{A}}	\newcommand{\mcB}{\mathcal{B}}
\newcommand{\mcC}{\mathcal{C}}	\newcommand{\mcD}{\mathcal{D}}
\newcommand{\mcE}{\mathcal{E}}	\newcommand{\mcF}{\mathcal{F}}
\newcommand{\mcG}{\mathcal{G}}	\newcommand{\mcH}{\mathcal{H}}
\newcommand{\mcI}{\mathcal{I}}	\newcommand{\mcJ}{\mathcal{J}}
\newcommand{\mcK}{\mathcal{K}}	\newcommand{\mcL}{\mathcal{L}}
\newcommand{\mcM}{\mathcal{M}}	\newcommand{\mcN}{\mathcal{N}}
\newcommand{\mcO}{\mathcal{O}}	\newcommand{\mcP}{\mathcal{P}}
\newcommand{\mcQ}{\mathcal{Q}}	\newcommand{\mcR}{\mathcal{R}}
\newcommand{\mcS}{\mathcal{S}}	\newcommand{\mcT}{\mathcal{T}}
\newcommand{\mcU}{\mathcal{U}}	\newcommand{\mcV}{\mathcal{V}}
\newcommand{\mcW}{\mathcal{W}}	\newcommand{\mcX}{\mathcal{X}}
\newcommand{\mcY}{\mathcal{Y}}	\newcommand{\mcZ}{\mathcal{Z}}

%---------------------------------------
% Bold Math Fonts :-
%---------------------------------------

%Captital Letters
\newcommand{\bmA}{\boldsymbol{A}}	\newcommand{\bmB}{\boldsymbol{B}}
\newcommand{\bmC}{\boldsymbol{C}}	\newcommand{\bmD}{\boldsymbol{D}}
\newcommand{\bmE}{\boldsymbol{E}}	\newcommand{\bmF}{\boldsymbol{F}}
\newcommand{\bmG}{\boldsymbol{G}}	\newcommand{\bmH}{\boldsymbol{H}}
\newcommand{\bmI}{\boldsymbol{I}}	\newcommand{\bmJ}{\boldsymbol{J}}
\newcommand{\bmK}{\boldsymbol{K}}	\newcommand{\bmL}{\boldsymbol{L}}
\newcommand{\bmM}{\boldsymbol{M}}	\newcommand{\bmN}{\boldsymbol{N}}
\newcommand{\bmO}{\boldsymbol{O}}	\newcommand{\bmP}{\boldsymbol{P}}
\newcommand{\bmQ}{\boldsymbol{Q}}	\newcommand{\bmR}{\boldsymbol{R}}
\newcommand{\bmS}{\boldsymbol{S}}	\newcommand{\bmT}{\boldsymbol{T}}
\newcommand{\bmU}{\boldsymbol{U}}	\newcommand{\bmV}{\boldsymbol{V}}
\newcommand{\bmW}{\boldsymbol{W}}	\newcommand{\bmX}{\boldsymbol{X}}
\newcommand{\bmY}{\boldsymbol{Y}}	\newcommand{\bmZ}{\boldsymbol{Z}}
%Small Letters
\newcommand{\bma}{\boldsymbol{a}}	\newcommand{\bmb}{\boldsymbol{b}}
\newcommand{\bmc}{\boldsymbol{c}}	\newcommand{\bmd}{\boldsymbol{d}}
\newcommand{\bme}{\boldsymbol{e}}	\newcommand{\bmf}{\boldsymbol{f}}
\newcommand{\bmg}{\boldsymbol{g}}	\newcommand{\bmh}{\boldsymbol{h}}
\newcommand{\bmi}{\boldsymbol{i}}	\newcommand{\bmj}{\boldsymbol{j}}
\newcommand{\bmk}{\boldsymbol{k}}	\newcommand{\bml}{\boldsymbol{l}}
\newcommand{\bmm}{\boldsymbol{m}}	\newcommand{\bmn}{\boldsymbol{n}}
\newcommand{\bmo}{\boldsymbol{o}}	\newcommand{\bmp}{\boldsymbol{p}}
\newcommand{\bmq}{\boldsymbol{q}}	\newcommand{\bmr}{\boldsymbol{r}}
\newcommand{\bms}{\boldsymbol{s}}	\newcommand{\bmt}{\boldsymbol{t}}
\newcommand{\bmu}{\boldsymbol{u}}	\newcommand{\bmv}{\boldsymbol{v}}
\newcommand{\bmw}{\boldsymbol{w}}	\newcommand{\bmx}{\boldsymbol{x}}
\newcommand{\bmy}{\boldsymbol{y}}	\newcommand{\bmz}{\boldsymbol{z}}

%---------------------------------------
% Scr Math Fonts :-
%---------------------------------------

\newcommand{\sA}{{\mathscr{A}}}   \newcommand{\sB}{{\mathscr{B}}}
\newcommand{\sC}{{\mathscr{C}}}   \newcommand{\sD}{{\mathscr{D}}}
\newcommand{\sE}{{\mathscr{E}}}   \newcommand{\sF}{{\mathscr{F}}}
\newcommand{\sG}{{\mathscr{G}}}   \newcommand{\sH}{{\mathscr{H}}}
\newcommand{\sI}{{\mathscr{I}}}   \newcommand{\sJ}{{\mathscr{J}}}
\newcommand{\sK}{{\mathscr{K}}}   \newcommand{\sL}{{\mathscr{L}}}
\newcommand{\sM}{{\mathscr{M}}}   \newcommand{\sN}{{\mathscr{N}}}
\newcommand{\sO}{{\mathscr{O}}}   \newcommand{\sP}{{\mathscr{P}}}
\newcommand{\sQ}{{\mathscr{Q}}}   \newcommand{\sR}{{\mathscr{R}}}
\newcommand{\sS}{{\mathscr{S}}}   \newcommand{\sT}{{\mathscr{T}}}
\newcommand{\sU}{{\mathscr{U}}}   \newcommand{\sV}{{\mathscr{V}}}
\newcommand{\sW}{{\mathscr{W}}}   \newcommand{\sX}{{\mathscr{X}}}
\newcommand{\sY}{{\mathscr{Y}}}   \newcommand{\sZ}{{\mathscr{Z}}}


%---------------------------------------
% Math Fraktur Font
%---------------------------------------

%Captital Letters
\newcommand{\mfA}{\mathfrak{A}}	\newcommand{\mfB}{\mathfrak{B}}
\newcommand{\mfC}{\mathfrak{C}}	\newcommand{\mfD}{\mathfrak{D}}
\newcommand{\mfE}{\mathfrak{E}}	\newcommand{\mfF}{\mathfrak{F}}
\newcommand{\mfG}{\mathfrak{G}}	\newcommand{\mfH}{\mathfrak{H}}
\newcommand{\mfI}{\mathfrak{I}}	\newcommand{\mfJ}{\mathfrak{J}}
\newcommand{\mfK}{\mathfrak{K}}	\newcommand{\mfL}{\mathfrak{L}}
\newcommand{\mfM}{\mathfrak{M}}	\newcommand{\mfN}{\mathfrak{N}}
\newcommand{\mfO}{\mathfrak{O}}	\newcommand{\mfP}{\mathfrak{P}}
\newcommand{\mfQ}{\mathfrak{Q}}	\newcommand{\mfR}{\mathfrak{R}}
\newcommand{\mfS}{\mathfrak{S}}	\newcommand{\mfT}{\mathfrak{T}}
\newcommand{\mfU}{\mathfrak{U}}	\newcommand{\mfV}{\mathfrak{V}}
\newcommand{\mfW}{\mathfrak{W}}	\newcommand{\mfX}{\mathfrak{X}}
\newcommand{\mfY}{\mathfrak{Y}}	\newcommand{\mfZ}{\mathfrak{Z}}
%Small Letters
\newcommand{\mfa}{\mathfrak{a}}	\newcommand{\mfb}{\mathfrak{b}}
\newcommand{\mfc}{\mathfrak{c}}	\newcommand{\mfd}{\mathfrak{d}}
\newcommand{\mfe}{\mathfrak{e}}	\newcommand{\mff}{\mathfrak{f}}
\newcommand{\mfg}{\mathfrak{g}}	\newcommand{\mfh}{\mathfrak{h}}
\newcommand{\mfi}{\mathfrak{i}}	\newcommand{\mfj}{\mathfrak{j}}
\newcommand{\mfk}{\mathfrak{k}}	\newcommand{\mfl}{\mathfrak{l}}
\newcommand{\mfm}{\mathfrak{m}}	\newcommand{\mfn}{\mathfrak{n}}
\newcommand{\mfo}{\mathfrak{o}}	\newcommand{\mfp}{\mathfrak{p}}
\newcommand{\mfq}{\mathfrak{q}}	\newcommand{\mfr}{\mathfrak{r}}
\newcommand{\mfs}{\mathfrak{s}}	\newcommand{\mft}{\mathfrak{t}}
\newcommand{\mfu}{\mathfrak{u}}	\newcommand{\mfv}{\mathfrak{v}}
\newcommand{\mfw}{\mathfrak{w}}	\newcommand{\mfx}{\mathfrak{x}}
\newcommand{\mfy}{\mathfrak{y}}	\newcommand{\mfz}{\mathfrak{z}}


\newcommand{\mytitle}{CS1231S Tutorial 2}
\newcommand{\myauthor}{github/omgeta}
\newcommand{\mydate}{AY 24/25 Sem 1}

\begin{document}
\raggedright
\footnotesize
\begin{center}
{\normalsize{\textbf{\mytitle}}} \\
{\footnotesize{\mydate\hspace{2pt}\textemdash\hspace{2pt}\myauthor}}
\end{center}
\setlist{topsep=-1em, itemsep=-1em, parsep=2em}
%%%%%%%%%%%%%%%%%%%%%%%%%%%%%%%%%%%%%%%%%%%%%%%%%%%%%%
%                      Begin                         %
%%%%%%%%%%%%%%%%%%%%%%%%%%%%%%%%%%%%%%%%%%%%%%%%%%%%%%
\begin{enumerate}[Q\arabic*.]
  \item 
    \begin{enumerate}[(\alph*)]
      \item  \textbf{Original: }$\forall n \in \ZZ (6\mid  n \rightarrow 2\mid n \land 3\mid  n)$\textbf{ true}$\qed$\\
        \textbf{Converse:} $\forall n \in \ZZ (2\mid n \land 3\mid n \rightarrow 6\mid n)$\textbf{ true}$\qed$\\
        \textbf{Inverse:} $\forall n \in \ZZ (6\nmid n \rightarrow 2\nmid n \lor 3 \nmid n)$\textbf{ true}$\qed$\\
        \textbf{Contrapositive:} $\forall n \in \ZZ (2\nmid n \lor 3\nmid  n \rightarrow 6 \nmid n)$\textbf{ true}$\qed$\\

      \item  \textbf{Original: }$\forall x (x \in \QQ \rightarrow x \in \ZZ)$\textbf{ false} ($3/2 \in \QQ \land 3/2 \not\in \ZZ$)$\qed$\\
        \textbf{Converse:} $\forall x (x \in \ZZ \rightarrow x \in \QQ)$\textbf{ true}$\qed$\\
        \textbf{Inverse:} $\forall x (x \not\in \QQ \rightarrow x \not\in \ZZ)$\textbf{ true}$\qed$\\
        \textbf{Contrapositive:} $\forall x (x \not\in \ZZ \rightarrow x \not\in \QQ)$\textbf{ false} ($3/2 \not\in \ZZ \land 3/2 \in \QQ$)$\qed$\\
      
      \item  \textbf{Original: }$\forall p,q\in\ZZ (Even(p)\land Even(q) \rightarrow Even(p+q))$\textbf{ true}$\qed$\\
        \textbf{Converse:} $\forall p,q\in\ZZ (Even(p+q)\rightarrow Even(p) \land Even(q))$\textbf{ false} ($p=3,q=5$)$\qed$\\
        \textbf{Inverse:} $\forall p,q\in\ZZ (\neg Even(p) \lor \neg Even(q) \rightarrow \neg Even(p+q))$\textbf{ false} ($p=3,q=5$)$\qed$\\
        \textbf{Contrapositive:} $\forall p,q\in\ZZ (\neg Even(p+q) \rightarrow \neg Even(p) \lor \neg Even(q))$\textbf{ true}$\qed$\\
    \end{enumerate}
  
  \item \begin{enumerate}[(\alph*)]
      \item $\forall x \in \RR,\exists y \in \RR (y > x)$  $\qed$
      \item $\forall x,z\in \RR, \exists y\in\RR(x<z \rightarrow (x < z) \land (z < y))$ $\qed$
    \end{enumerate}

  \item \begin{enumerate}[(\alph*)]
      \item $R$ is reflexive $\iff \forall x \in A (xRx)$ $\qed$
      \item $R$ is symmetric $\iff \forall x,y \in A (xRy \rightarrow yRx)$$\qed$ 
      \item $R$ is transitive $\iff \forall x,y,z \in A (xRy \land yRz \rightarrow xRz)$ $\qed$
    \end{enumerate}

  \item \begin{enumerate}[(\alph*)]
    \item \textbf{Disproof by Counterexample}\\
      $3 \in \ZZ \land 2 \in \ZZ$ but $\frac{3}{2} \not\in \ZZ$$\qed$
    \item \textbf{Direct Proof}\\
      Let $x, y \in \RR$, then by definition of rational numbers, $\exists a,b,c,d \in \ZZ$ where $b,d\neq 0$ such that:
      \begin{align*}
        x &= \frac{a}{b}\\
        y &= \frac{c}{d}
      \end{align*}
      Consider addition of $x$ and $y$:
      \begin{align*}
        x+y &= \frac{a}{b} + \frac{c}{d} \\
            &= \frac{ad + bc}{bd}
      \end{align*}
      Since integers are closed over addition and multiplication:
      \begin{align*}
        A &= ad+bc \in \ZZ\\
        B &= bd \in \ZZ
      \end{align*}
      Since $b \neq 0, d \neq 0, B = bd\neq 0.$\\
      Therefore, $\forall x,y\in \RR, x+y=\frac{A}{B}$ is rational (by definition of rational numbers). Hence, rational numbers are closed under addition.$\qed$

    \item \textbf{Disproof by Counterexample}\\
      $\forall x \in \RR, 0\in \RR, \frac{x}{0} \not\in \RR$.$\qed$
  \end{enumerate}

\item 
  \begin{enumerate}[(\alph*)]
    \item False. If $x<y$ then $x-y\not\in B \qed$\\
    \item True.$\qed$
    \item False. Predicate is only true if $(x,y) \in \{(1,0), (3,2), (5, 4), (7, 6)\}$ and not $\forall x\in A, \forall y \in B$$\qed$
    \item True.$\qed$
    \item True.$\qed$
    \item False. $\exists x \in A, \exists y \in B (x = y + 1)$ $\qed$
    \item True.$\qed$
    \item True.$\qed$
    \item False. If $x \in {7, 11, 13}$ predicate is false because the largest element $y \in B$ is $6$.$\qed$
    \item True. Take $y = 0.$$\qed$
  \end{enumerate}

\item 
  \begin{enumerate}[(\alph*)]
    \item False. There is no title read by all the female readers. Ms Emily has only read "Dream of the Red Chamber", "Da Vinci Code", "She: A History of Adventure" and "Black Beauty", all of which are not read by both of the other two females.$\qed$

    \item False. Ms Dueet has not read any books in the Fantasy genre.$\qed$

    \item True. Ms Dueet has read all books of the Mystery genre.$\qed$

    \item True. Fantasy has none of its books read by Ms Dueet.$\qed$
  \end{enumerate}

\item
  \begin{enumerate}[(\alph*)]
    \item Universal statements in each of the cases cannot be proven by a single example.$\qed$
    \item There are 3 cases to consider: $x < 0, x = 0$ and $x > 0$. If $x < 0$, for example, $x = -1$, then $x^3 = -1 = x$; if $x = 0$, then $x^3 = 0 = x$; if $x > 0$, say $x = 1$, then $x^3 = -1 = x$. Therefore, in all cases, $x^3 = x$.$\qed$
    \item The proof is false because it assumes that a single counterexample is enough to prove the falsity of a statement for all real values of $x$. $\qed$
    \item Suppose $x^3 \neq x$ for all real numbers $x$. Let $x = 1$, then $x^3 = 1 = x$ which is a contradiction. Therefore, $\forall x \in \RR (x^3 = x) \qed$
  \end{enumerate}

\item 
  \begin{enumerate}[(\alph*)]
    \item No. In the case $r^2 \leq r$, our statement is vacuously true. $\qed$

    \item 2.3. \textbf{Case 1:} $r > 0 \land r-1>0$\\
      This implies $r>0$ and $r > 1$, which satisfies $r < 0 \lor r > 1$\\
    2.4. \textbf{Case 2:} $r < 0 \land r-1<0$\\
      This implies $r<0$ and $r < 1$, which satisfies $r < 0 \lor r > 1$\\
    2.5. In both cases, the conclusion $r < 0 \lor r > 1$ is satisfied.

  \item By proving that a statement holds for an arbitrary element, we can conclude a universal statement. $\qed$
  \end{enumerate}

  \item \begin{enumerate}[(\alph*)]
      \item $\forall v \in V (W(v)) \qed$
      \item $\forall v \in V (G(v) \rightarrow T(v)) \qed$
      \item $\exists v \in V (T(v) \land G(v)) \qed$
      \item $\forall v \in V (E(v) \rightarrow \neg W(v)) \qed$
      \item $\exists v \in V (T(v) \land E(v)) \land \exists v \in V (T(v) \land \neg E(v)) \qed$
  \end{enumerate}

\item \begin{enumerate}[(\alph*)]
    \item 3. Every black object is a square\\
      2. If an object is square, then it is above all the grey objects\\
      4. Every object that is above all the gray objects is above all the triangles.\\
      1. If an object is above all the triangles, then it is above all the blue objects.\\
      $\therefore$ If an object is black, then it is above all the blue object.
    \item 3. $\forall x(Black(x) \rightarrow Square(x))$\\
      2. $\forall x, y(Square(x) \rightarrow (Gray(y) \rightarrow Above(x, y)))$\\
      4. $\forall x, y, z((Gray(y) \rightarrow Above(x, y)) \rightarrow (Triangle(z) \rightarrow Above(x, z)))$\\
      1. $\forall x, y, z((Triangle(y) \rightarrow (Above(x, y))) \rightarrow (Blue(z) \rightarrow Above(x, z)))$\\
      $\therefore \forall x, y(Black(x) \rightarrow (Blue(y) \rightarrow Above(x, y)))$
  \end{enumerate}

\item \textbf{Proof by Contraposition.}\\
  \begin{enumerate}[\arabic*.]
    \item Let $a, b$ be two positive integers
    \item Assume the opposite of the conclusion, i.e., $a > n^{1/2}
  \land b > n^{1/2}$.
  \begin{enumerate}[label=2.\arabic*]
    \item $a\cdot b &> n^{1/2} \cdot n^{1/2}$\\
    \item $a \cdot b &> n$ \\
    \item $a \cdot b &\neq n$
  \end{enumerate}
\item Therefore, by contraposition, since $\forall a,b \in \ZZ^+, ((a > \sqrt{n}) \land (b > \sqrt{n})) \rightarrow n \neq a\cdot b$, then $\forall a, b \in \ZZ^+, n = ab \rightarrow ((a \leq \sqrt{n}) \lor (b \leq \sqrt{n}))$.
\end{enumerate}

   \textbf{Proof by Contradiction.}\\
  \begin{enumerate}[\arabic*.]
    \item Let $a, b$ be two positive integers
    \item Assume $n = ab$ and $a > n^{1/2}
  \land b > n^{1/2}$.
  \begin{enumerate}[label=2.\arabic*]
    \item $a\cdot b &> n^{1/2} \cdot n^{1/2}$\\
    \item $a \cdot b &> n$ \\
    \item This is a contradiction to $n = ab$
  \end{enumerate}
\item Therefore, by contradiction, since $n = ab \rightarrow (a \leq \sqrt{n} \lor b \leq \sqrt{n})$

\end{enumerate}
%%%%%%%%%%%%%%%%%%%%%%%%%%%%%%%%%%%%%%%%%%%%%%%%%%%%%%
%                       End                          %
%%%%%%%%%%%%%%%%%%%%%%%%%%%%%%%%%%%%%%%%%%%%%%%%%%%%%%

\end{document}

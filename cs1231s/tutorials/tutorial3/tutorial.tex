\documentclass[12pt, a4paper]{article}

\usepackage[a4paper, margin=1in]{geometry}

\usepackage[utf8]{inputenc}
\usepackage[mathscr]{euscript}
\let\euscr\mathscr \let\mathscr\relax
\usepackage[scr]{rsfso}
\usepackage{amssymb,amsmath,amsthm,amsfonts}
\usepackage[shortlabels]{enumitem}
\usepackage{multicol,multirow}
\usepackage{lipsum}
\usepackage{balance}
\usepackage{calc}
\usepackage[colorlinks=true,citecolor=blue,linkcolor=blue]{hyperref}
\usepackage{import}
\usepackage{xifthen}
\usepackage{pdfpages}
\usepackage{transparent}
\usepackage{listings}

\newcommand{\incfig}[2][1.0]{
    \def\svgwidth{#1\columnwidth}
    \import{./figures/}{#2.pdf_tex}
}

\newlist{enumproof}{enumerate}{4}
\setlist[enumproof,1]{label=\arabic*., parsep=1em}
\setlist[enumproof,2]{label=\arabic{enumproofi}.\arabic*., parsep=1em}
\setlist[enumproof,3]{label=\arabic{enumproofi}.\arabic{enumproofii}.\arabic*., parsep=1em}
\setlist[enumproof,4]{label=\arabic{enumproofi}.\arabic{enumproofii}.\arabic{enumproofiii}.\arabic*., parsep=1em}

\renewcommand{\qedsymbol}{\ensuremath{\blacksquare}}

\lstdefinestyle{mystyle}{
  language=C, % Set the language to C
  commentstyle=\color{codegray}, % Color for comments
  keywordstyle=\color{orange}, % Color for basic keywords
  stringstyle=\color{mauve}, % Color for strings
  basicstyle={\ttfamily\footnotesize}, % Basic font style
  breakatwhitespace=false,         
  breaklines=true,                 
  captionpos=b,                    
  keepspaces=true,                 
  numbers=none,                    
  tabsize=2,
  morekeywords=[2]{\#include, \#define, \#ifdef, \#ifndef, \#endif, \#pragma, \#else, \#elif}, % Preprocessor directives
  keywordstyle=[2]\color{codegreen}, % Style for preprocessor directives
  morekeywords=[3]{int, char, float, double, void, struct, union, enum, const, volatile, static, extern, register, inline, restrict, _Bool, _Complex, _Imaginary, size_t, ssize_t, FILE}, % C standard types and common identifiers
  keywordstyle=[3]\color{identblue}, % Style for types and common identifiers
  morekeywords=[4]{printf, scanf, fopen, fclose, malloc, free, calloc, realloc, perror, strtok, strncpy, strcpy, strcmp, strlen}, % Standard library functions
  keywordstyle=[4]\color{cyan}, % Style for library functions
}

% Things Lie
\newcommand{\kb}{\mathfrak b}
\newcommand{\kg}{\mathfrak g}
\newcommand{\kh}{\mathfrak h}
\newcommand{\kn}{\mathfrak n}
\newcommand{\ku}{\mathfrak u}
\newcommand{\kz}{\mathfrak z}
\DeclareMathOperator{\Ext}{Ext} % Ext functor
\DeclareMathOperator{\Tor}{Tor} % Tor functor
\newcommand{\gl}{\opname{\mathfrak{gl}}} % frak gl group
\renewcommand{\sl}{\opname{\mathfrak{sl}}} % frak sl group chktex 6

% More script letters etc.
\newcommand{\SA}{\mathcal A}
\newcommand{\SB}{\mathcal B}
\newcommand{\SC}{\mathcal C}
\newcommand{\SF}{\mathcal F}
\newcommand{\SG}{\mathcal G}
\newcommand{\SH}{\mathcal H}
\newcommand{\OO}{\mathcal O}

\newcommand{\SCA}{\mathscr A}
\newcommand{\SCB}{\mathscr B}
\newcommand{\SCC}{\mathscr C}
\newcommand{\SCD}{\mathscr D}
\newcommand{\SCE}{\mathscr E}
\newcommand{\SCF}{\mathscr F}
\newcommand{\SCG}{\mathscr G}
\newcommand{\SCH}{\mathscr H}

% Mathfrak primes
\newcommand{\km}{\mathfrak m}
\newcommand{\kp}{\mathfrak p}
\newcommand{\kq}{\mathfrak q}

% number sets
\newcommand{\RR}[1][]{\ensuremath{\ifstrempty{#1}{\mathbb{R}}{\mathbb{R}^{#1}}}}
\newcommand{\NN}[1][]{\ensuremath{\ifstrempty{#1}{\mathbb{N}}{\mathbb{N}^{#1}}}}
\newcommand{\ZZ}[1][]{\ensuremath{\ifstrempty{#1}{\mathbb{Z}}{\mathbb{Z}^{#1}}}}
\newcommand{\QQ}[1][]{\ensuremath{\ifstrempty{#1}{\mathbb{Q}}{\mathbb{Q}^{#1}}}}
\newcommand{\CC}[1][]{\ensuremath{\ifstrempty{#1}{\mathbb{C}}{\mathbb{C}^{#1}}}}
\newcommand{\PP}[1][]{\ensuremath{\ifstrempty{#1}{\mathbb{P}}{\mathbb{P}^{#1}}}}
\newcommand{\HH}[1][]{\ensuremath{\ifstrempty{#1}{\mathbb{H}}{\mathbb{H}^{#1}}}}
\newcommand{\FF}[1][]{\ensuremath{\ifstrempty{#1}{\mathbb{F}}{\mathbb{F}^{#1}}}}
% expected value
\newcommand{\EE}{\ensuremath{\mathbb{E}}}
\newcommand{\charin}{\text{ char }}
\DeclareMathOperator{\sign}{sign}
\DeclareMathOperator{\Aut}{Aut}
\DeclareMathOperator{\Inn}{Inn}
\DeclareMathOperator{\Syl}{Syl}
\DeclareMathOperator{\Gal}{Gal}
\DeclareMathOperator{\GL}{GL} % General linear group
\DeclareMathOperator{\SL}{SL} % Special linear group

%---------------------------------------
% BlackBoard Math Fonts :-
%---------------------------------------

%Captital Letters
\newcommand{\bbA}{\mathbb{A}}	\newcommand{\bbB}{\mathbb{B}}
\newcommand{\bbC}{\mathbb{C}}	\newcommand{\bbD}{\mathbb{D}}
\newcommand{\bbE}{\mathbb{E}}	\newcommand{\bbF}{\mathbb{F}}
\newcommand{\bbG}{\mathbb{G}}	\newcommand{\bbH}{\mathbb{H}}
\newcommand{\bbI}{\mathbb{I}}	\newcommand{\bbJ}{\mathbb{J}}
\newcommand{\bbK}{\mathbb{K}}	\newcommand{\bbL}{\mathbb{L}}
\newcommand{\bbM}{\mathbb{M}}	\newcommand{\bbN}{\mathbb{N}}
\newcommand{\bbO}{\mathbb{O}}	\newcommand{\bbP}{\mathbb{P}}
\newcommand{\bbQ}{\mathbb{Q}}	\newcommand{\bbR}{\mathbb{R}}
\newcommand{\bbS}{\mathbb{S}}	\newcommand{\bbT}{\mathbb{T}}
\newcommand{\bbU}{\mathbb{U}}	\newcommand{\bbV}{\mathbb{V}}
\newcommand{\bbW}{\mathbb{W}}	\newcommand{\bbX}{\mathbb{X}}
\newcommand{\bbY}{\mathbb{Y}}	\newcommand{\bbZ}{\mathbb{Z}}

%---------------------------------------
% MathCal Fonts :-
%---------------------------------------

%Captital Letters
\newcommand{\mcA}{\mathcal{A}}	\newcommand{\mcB}{\mathcal{B}}
\newcommand{\mcC}{\mathcal{C}}	\newcommand{\mcD}{\mathcal{D}}
\newcommand{\mcE}{\mathcal{E}}	\newcommand{\mcF}{\mathcal{F}}
\newcommand{\mcG}{\mathcal{G}}	\newcommand{\mcH}{\mathcal{H}}
\newcommand{\mcI}{\mathcal{I}}	\newcommand{\mcJ}{\mathcal{J}}
\newcommand{\mcK}{\mathcal{K}}	\newcommand{\mcL}{\mathcal{L}}
\newcommand{\mcM}{\mathcal{M}}	\newcommand{\mcN}{\mathcal{N}}
\newcommand{\mcO}{\mathcal{O}}	\newcommand{\mcP}{\mathcal{P}}
\newcommand{\mcQ}{\mathcal{Q}}	\newcommand{\mcR}{\mathcal{R}}
\newcommand{\mcS}{\mathcal{S}}	\newcommand{\mcT}{\mathcal{T}}
\newcommand{\mcU}{\mathcal{U}}	\newcommand{\mcV}{\mathcal{V}}
\newcommand{\mcW}{\mathcal{W}}	\newcommand{\mcX}{\mathcal{X}}
\newcommand{\mcY}{\mathcal{Y}}	\newcommand{\mcZ}{\mathcal{Z}}

%---------------------------------------
% Bold Math Fonts :-
%---------------------------------------

%Captital Letters
\newcommand{\bmA}{\boldsymbol{A}}	\newcommand{\bmB}{\boldsymbol{B}}
\newcommand{\bmC}{\boldsymbol{C}}	\newcommand{\bmD}{\boldsymbol{D}}
\newcommand{\bmE}{\boldsymbol{E}}	\newcommand{\bmF}{\boldsymbol{F}}
\newcommand{\bmG}{\boldsymbol{G}}	\newcommand{\bmH}{\boldsymbol{H}}
\newcommand{\bmI}{\boldsymbol{I}}	\newcommand{\bmJ}{\boldsymbol{J}}
\newcommand{\bmK}{\boldsymbol{K}}	\newcommand{\bmL}{\boldsymbol{L}}
\newcommand{\bmM}{\boldsymbol{M}}	\newcommand{\bmN}{\boldsymbol{N}}
\newcommand{\bmO}{\boldsymbol{O}}	\newcommand{\bmP}{\boldsymbol{P}}
\newcommand{\bmQ}{\boldsymbol{Q}}	\newcommand{\bmR}{\boldsymbol{R}}
\newcommand{\bmS}{\boldsymbol{S}}	\newcommand{\bmT}{\boldsymbol{T}}
\newcommand{\bmU}{\boldsymbol{U}}	\newcommand{\bmV}{\boldsymbol{V}}
\newcommand{\bmW}{\boldsymbol{W}}	\newcommand{\bmX}{\boldsymbol{X}}
\newcommand{\bmY}{\boldsymbol{Y}}	\newcommand{\bmZ}{\boldsymbol{Z}}
%Small Letters
\newcommand{\bma}{\boldsymbol{a}}	\newcommand{\bmb}{\boldsymbol{b}}
\newcommand{\bmc}{\boldsymbol{c}}	\newcommand{\bmd}{\boldsymbol{d}}
\newcommand{\bme}{\boldsymbol{e}}	\newcommand{\bmf}{\boldsymbol{f}}
\newcommand{\bmg}{\boldsymbol{g}}	\newcommand{\bmh}{\boldsymbol{h}}
\newcommand{\bmi}{\boldsymbol{i}}	\newcommand{\bmj}{\boldsymbol{j}}
\newcommand{\bmk}{\boldsymbol{k}}	\newcommand{\bml}{\boldsymbol{l}}
\newcommand{\bmm}{\boldsymbol{m}}	\newcommand{\bmn}{\boldsymbol{n}}
\newcommand{\bmo}{\boldsymbol{o}}	\newcommand{\bmp}{\boldsymbol{p}}
\newcommand{\bmq}{\boldsymbol{q}}	\newcommand{\bmr}{\boldsymbol{r}}
\newcommand{\bms}{\boldsymbol{s}}	\newcommand{\bmt}{\boldsymbol{t}}
\newcommand{\bmu}{\boldsymbol{u}}	\newcommand{\bmv}{\boldsymbol{v}}
\newcommand{\bmw}{\boldsymbol{w}}	\newcommand{\bmx}{\boldsymbol{x}}
\newcommand{\bmy}{\boldsymbol{y}}	\newcommand{\bmz}{\boldsymbol{z}}

%---------------------------------------
% Scr Math Fonts :-
%---------------------------------------

\newcommand{\sA}{{\mathscr{A}}}   \newcommand{\sB}{{\mathscr{B}}}
\newcommand{\sC}{{\mathscr{C}}}   \newcommand{\sD}{{\mathscr{D}}}
\newcommand{\sE}{{\mathscr{E}}}   \newcommand{\sF}{{\mathscr{F}}}
\newcommand{\sG}{{\mathscr{G}}}   \newcommand{\sH}{{\mathscr{H}}}
\newcommand{\sI}{{\mathscr{I}}}   \newcommand{\sJ}{{\mathscr{J}}}
\newcommand{\sK}{{\mathscr{K}}}   \newcommand{\sL}{{\mathscr{L}}}
\newcommand{\sM}{{\mathscr{M}}}   \newcommand{\sN}{{\mathscr{N}}}
\newcommand{\sO}{{\mathscr{O}}}   \newcommand{\sP}{{\mathscr{P}}}
\newcommand{\sQ}{{\mathscr{Q}}}   \newcommand{\sR}{{\mathscr{R}}}
\newcommand{\sS}{{\mathscr{S}}}   \newcommand{\sT}{{\mathscr{T}}}
\newcommand{\sU}{{\mathscr{U}}}   \newcommand{\sV}{{\mathscr{V}}}
\newcommand{\sW}{{\mathscr{W}}}   \newcommand{\sX}{{\mathscr{X}}}
\newcommand{\sY}{{\mathscr{Y}}}   \newcommand{\sZ}{{\mathscr{Z}}}


%---------------------------------------
% Math Fraktur Font
%---------------------------------------

%Captital Letters
\newcommand{\mfA}{\mathfrak{A}}	\newcommand{\mfB}{\mathfrak{B}}
\newcommand{\mfC}{\mathfrak{C}}	\newcommand{\mfD}{\mathfrak{D}}
\newcommand{\mfE}{\mathfrak{E}}	\newcommand{\mfF}{\mathfrak{F}}
\newcommand{\mfG}{\mathfrak{G}}	\newcommand{\mfH}{\mathfrak{H}}
\newcommand{\mfI}{\mathfrak{I}}	\newcommand{\mfJ}{\mathfrak{J}}
\newcommand{\mfK}{\mathfrak{K}}	\newcommand{\mfL}{\mathfrak{L}}
\newcommand{\mfM}{\mathfrak{M}}	\newcommand{\mfN}{\mathfrak{N}}
\newcommand{\mfO}{\mathfrak{O}}	\newcommand{\mfP}{\mathfrak{P}}
\newcommand{\mfQ}{\mathfrak{Q}}	\newcommand{\mfR}{\mathfrak{R}}
\newcommand{\mfS}{\mathfrak{S}}	\newcommand{\mfT}{\mathfrak{T}}
\newcommand{\mfU}{\mathfrak{U}}	\newcommand{\mfV}{\mathfrak{V}}
\newcommand{\mfW}{\mathfrak{W}}	\newcommand{\mfX}{\mathfrak{X}}
\newcommand{\mfY}{\mathfrak{Y}}	\newcommand{\mfZ}{\mathfrak{Z}}
%Small Letters
\newcommand{\mfa}{\mathfrak{a}}	\newcommand{\mfb}{\mathfrak{b}}
\newcommand{\mfc}{\mathfrak{c}}	\newcommand{\mfd}{\mathfrak{d}}
\newcommand{\mfe}{\mathfrak{e}}	\newcommand{\mff}{\mathfrak{f}}
\newcommand{\mfg}{\mathfrak{g}}	\newcommand{\mfh}{\mathfrak{h}}
\newcommand{\mfi}{\mathfrak{i}}	\newcommand{\mfj}{\mathfrak{j}}
\newcommand{\mfk}{\mathfrak{k}}	\newcommand{\mfl}{\mathfrak{l}}
\newcommand{\mfm}{\mathfrak{m}}	\newcommand{\mfn}{\mathfrak{n}}
\newcommand{\mfo}{\mathfrak{o}}	\newcommand{\mfp}{\mathfrak{p}}
\newcommand{\mfq}{\mathfrak{q}}	\newcommand{\mfr}{\mathfrak{r}}
\newcommand{\mfs}{\mathfrak{s}}	\newcommand{\mft}{\mathfrak{t}}
\newcommand{\mfu}{\mathfrak{u}}	\newcommand{\mfv}{\mathfrak{v}}
\newcommand{\mfw}{\mathfrak{w}}	\newcommand{\mfx}{\mathfrak{x}}
\newcommand{\mfy}{\mathfrak{y}}	\newcommand{\mfz}{\mathfrak{z}}


\newcommand{\mytitle}{CS1231S Tutorial 3}
\newcommand{\myauthor}{github/omgeta}
\newcommand{\mydate}{AY 24/25 Sem 1}

\begin{document}
\raggedright
\footnotesize
\begin{center}
{\normalsize{\textbf{\mytitle}}} \\
{\footnotesize{\mydate\hspace{2pt}\textemdash\hspace{2pt}\myauthor}}
\end{center}
\setlist{topsep=-1em, itemsep=-1em, parsep=2em}
%%%%%%%%%%%%%%%%%%%%%%%%%%%%%%%%%%%%%%%%%%%%%%%%%%%%%%
%                      Begin                         %
%%%%%%%%%%%%%%%%%%%%%%%%%%%%%%%%%%%%%%%%%%%%%%%%%%%%%%
\begin{enumerate}[Q\arabic*.]
  \item 
    \begin{enumerate}[(\alph*)]
      \item $\mcP(\{a, b, c\}) = \{\varnothing, \{a\}, \{b\}, \{c\}, \{a, b\}, \{b, c\}, \{a, c\}, \{a, b, c\}\} \qed$

      \item $\mcP(\mcP(\mcP(\varnothing))) = \{\varnothing, \{\varnothing\}, \{\{\varnothing\}\}, \{\varnothing, \{\varnothing\}\}\} \qed$
    \end{enumerate}

  \item 
    \begin{enumerate}[(\alph*)]
      \item $A \cup B = \{x \in \RR : -2 \leq x < 3\} = [-2, 3) \qed$
      \item $A \cap B = \{x \in \RR : -1 < x \leq 1\} = (-1, 1] \qed$
      \item $\overline{A} = \{x \in \RR : x < -2 \lor x > 1\} = (-\infty, -2) \cup (1, \infty) \qed$
      \item $\overline{A} \cap \overline{B} = \{x \in \RR : x < -2 \lor x > 3\} = (-\infty, -2) \cup (3, \infty) \qed$
      \item $A \setminus B = \{x \in \RR : -2 \leq x \leq -1\} = [-2, -1] \qed$
    \end{enumerate}

  \item  
    \begin{enumerate}[(\alph*)]
      \item True. If $A \cap B = \varnothing$, e.g. let $A = \{1\}, B = \{2\} \qed$ 
      \item True. If $A \cap B \neq \varnothing$, e.g. let $A = \{2\}, B = \{2\}\qed$
      \item False. $\forall A, B$, $\varnothing \in \mcP(A \times B)$ but $\varnothing \not\in A \times \mcP(B) \qed$
    \end{enumerate}

  \item 
    \begin{enumerate}[\arabic*.]
      \item Prove $A \subseteq B$:
        \begin{enumerate}[label=1.\arabic*]
          \item Suppose $a \in A$, then $a = 2n+1, n \in \ZZ$.
          \item Let $m = n+3$,  $m \in \ZZ$ by closure of integers over addition.
          \item $a = 2n+1 = 2(n+3)-5 = 2m-5 \in B$.
          \item $\therefore \forall a \in A, a \in B$\hfill(Universal generalization)
          \item $\therefore A \subseteq B$\hfill(Definition of subsets)
        \end{enumerate}
      \item Prove $B \subseteq A$:
        \begin{enumerate}[label=2.\arabic*]
          \item Suppose $b \in B$, then $a = 2n-5, n \in \ZZ$.
          \item Let $m = n-3$,  $m \in \ZZ$ by closure of integers over addition.
          \item $b = 2n-5 = 2(n-3)+1 = 2m+1 \in A$.
          \item $\therefore \forall b \in B, b \in A$\hfill(Universal generalization)
          \item $\therefore B \subseteq A$\hfill(Definition of subsets)
        \end{enumerate}
      \item $\therefore A \subseteq B \land B \subseteq A$\hfill(Conjunction)
      \item $\therefore A = B \qed$\hfill(Definition of set equality)
    \end{enumerate}
  \pagebreak

  \item Let $A, B, C$ be sets. To prove $A \cap (B \setminus C) = (A \cap B) \setminus C$: 
    \begin{enumerate}[\arabic*.]
      \item Prove $A \cap (B \setminus C) \subseteq (A \cap B) \setminus C$:
        \begin{enumerate}[label=1.\arabic*]
          \item $A \cap (B \setminus C) = \{x: x \in A \cap (B \setminus C)\}$
          \item $= \{x: (x \in A) \land (x \in B \setminus C)\}\hfill\text{(Definition of set difference)}$
          \item $= \{x: (x \in A) \land ((x \in B) \land (x \not\in C))\}\hfill\text{(Definition of set difference)}$
          \item $= \{x: ((x \in A) \land (x \in B)) \land (x \not\in C)\}\hfill\text{(Associative law)}$
          \item $= \{x: (x \in A \cap B) \land (x \not\in C)\}\hfill\text{(Definition of intersection)}$
          \item $= \{x: x \in (A \cap B) \setminus C\}\hfill\text{(Definition of set difference)}$
          \item $= (A \cap B) \setminus C$
        \end{enumerate}
      \item We must show $\forall x, x \in (A \cap B) \setminus C \rightarrow x \in A \cap (B \setminus C)$
        \begin{enumerate}[label=1.\arabic*]
          \item Let $x \in (A \cap B) \setminus C$
          \item $x \in (A \cap B) \land x \not\in C$\hfill(Definition of set difference)
          \item $x \not\in C$\hfill(Specialisation)
          \item $x \in (A \cap B)$\hfill(Specialisation)
          \item $x \in A \land x \in B$\hfill(Definition of set intersection)
          \item $x \in A$\hfill(Specialisation)
          \item $x \in B$\hfill(Specialisation)
          \item $x \in B \land x \not\in C$\hfill(Conjunction)
          \item $x \in (B \setminus C)$\hfill(Definition of set difference)
          \item $x \in A \land x \in (B \setminus C)$\hfill(Conjunction)
          \item $x \in A \cap (B \setminus C)$\hfill(Definition of set intersection)
        \end{enumerate}
      \item $\therefore (A \cap (B \setminus C) \subseteq (A \cap B) \setminus C) \land ((A \cap B) \setminus C \subseteq A \cap (B \setminus C))$\hfill(Conjunction)
      \item $\therefore A \cap (B \setminus C) = (A \cap B) \setminus C$\hfill(Definition of set equality)
    \end{enumerate}
    Therefore, $\forall A, B, C, A \cap (B \setminus C) = (A \cap B) \setminus C \qed$

  \pagebreak
  \item Let $A, B, C$ be sets.
    \begin{enumerate}[\arabic*.]
      \item $A \setminus (B \setminus C)$
      \item $= A \setminus (B \cap \overline{C})$\hfill(Set difference law)
      \item $= A \cap \overline{(B \cap \overline{C})}$\hfill(Set difference law) 
      \item $= A \cap (\overline{B} \cup C)$\hfill(DeMorgan's law)
      \item $= (A \cap \overline{B}) \cup (A \cap C)$\hfill(Distributive law)
      \item $= (A \setminus B) \cup (A \cap C)$\hfill(Set difference law)
    \end{enumerate}
    Therefore, $\forall A, B, C, A \setminus (B \setminus C) = (A \setminus B) \cup (A \cap C) \qed$

  \item 
    \begin{enumerate}[(\alph*)]
      \item $A \xor B = \{1, 9\} \qed$
      \item Let $A, B$ be sets, and $U$ is the universal set.
        \begin{enumerate}[\arabic*.]
          \item $A \xor B$
          \item $= (A \setminus B) \cup (B \setminus A)$\hfill(Definition of XOR)
          \item $= (A \cap \overline{B}) \cup (B \setminus A)$\hfill(Set difference law)
          \item $= (A \cap \overline{B}) \cup (B \cap \overline{A})$\hfill(Set difference law)
          \item $= ((A \cap \overline{B}) \cup B) \cap ((A \cap \overline{B}) \cup \overline{A})$\hfill(Distributive law)
          \item $= ((A \cup B) \cap (\overline{B} \cup B)) \cap ((A \cup \overline{A}) \cap (\overline{B} \cup \overline{A}))$\hfill(Distributive law)
          \item $= ((A \cup B) \cap U) \cap (U \cap (\overline{B} \cup \overline{A}))$\hfill(Complement law)
          \item $= (A \cup B) \cap (\overline{B} \cup \overline{A})$\hfill(Idempotent law)
          \item $= (A \cup B) \cap (\overline{A} \cup \overline{B})$\hfill(Commutative law)
          \item $= (A \cup B) \cap \overline{(A \cap B)}$\hfill(DeMorgan's law)
          \item $= (A \cup B) \setminus (A \cap B)$\hfill(Set difference law)
        \end{enumerate}
        Therefore, $\forall A, B, A \xor B = (A \cup B) \setminus (A \cap B) \qed$
    \end{enumerate}

  \pagebreak
  \item Let $A, B$ be sets. To prove $A \subseteq B \iff A \cup B = B$
    \begin{enumerate}[\arabic*.]
      \item Prove $A \subseteq B \rightarrow A \cup B = B$
        \begin{enumerate}[label=1.\arabic*]
          \item Suppose $A \subseteq B$
          \item Prove $A \cup B \subseteq B$
            \begin{enumerate}[label=1.2.\arabic*]
              \item Let $x \in A \cup B$
              \item $x \in A \lor x \in B$\hfill(Definition of union)
              \item Case 1: $x \in A \implies x \in B$\hfill(By 1.1)
              \item Case 2: $x \in B$
              \item In both cases, $x \in B$
              \item $\therefore A \cup B \subseteq B$\hfill(Definition of subset)
            \end{enumerate}
          \item Prove $B \subseteq A \cup B$
            \begin{enumerate}[label=1.3.\arabic*]
              \item Let $x \in B$
              \item $x \in A \lor x \in B$\hfill(Generalisation)
              \item $x \in (A \cup B)$\hfill(Definition of union)
              \item $\therefore B \subseteq A \cup B$\hfill(Definition of subset)
            \end{enumerate}
          \item $(A \cup B \subseteq B) \land (B \subseteq A \cup B)$\hfill(Conjunction)
          \item $A \cup B = B$\hfill(Definition of set equality)
        \end{enumerate}
      \item Prove $A \cup B = B \rightarrow A \subseteq B$
        \begin{enumerate}[label=2.\arabic*]
          \item Suppose $A \cup B = B$ 
          \item Let $x \in A$
          \item $x\in A \lor x \in B$\hfill(Generalisation)
          \item $x\in A \cup B$\hfill(Definition of union)
          \item $x \in B$\hfill(By 2.1)
          \item $\therefore A \subseteq B$\hfill(Definition of subset)
        \end{enumerate}
      \item $(A \subseteq B \rightarrow A \cup B = B) \land (A \cup B = B \rightarrow A \subseteq B)$\hfill(Conjunction)
      \item $A \subseteq B \iff A \cup B = B$\hfill(Definition of iff)
    \end{enumerate}

  \pagebreak
  \item 
    \begin{enumerate}[(\alph*)]
      \item Step 4 is an incorrect application of distribution over disjunction. $\qed$
      \item 
        \begin{enumerate}[\arabic*.]
          \item Prove $(A \setminus B) \cup (B \setminus A) \subseteq (A \cup B) \setminus (A \cap B)$
          \begin{enumerate}[label=1.\arabic*]
            \item Suppose $x \in (A \setminus B) \cup (B \setminus A)$
            \item $x \in (A \setminus B) \lor x\in(B \setminus A)$\hfill(Definition of union)
            \item $x \in (A \cap \overline{B}) \lor x\in(B \cap \overline{A})$\hfill(Set difference law)
            \item Case 1: $x \in (A \cap \overline{B}) \implies x \in A \land x \not\in B$\hfill(Definition of intersection)
            \item Case 2: $x \in (B \cap \overline{A}) \implies x \in B \land x \not\in A$\hfill(Definition of intersection)
            \item In either case, $x \in A \cup B$\hfill(Definition of union)
            \item In either case, $x \not\in A \cap B$\hfill(Definition of intersection)
            \item $x \in (A \cup B) \land x \in \overline{(A \cap B)}$\hfill(Conjunction)
            \item $x \in (A \cup B) \cap \overline{(A \cap B)}$\hfill(Definition of intersection)
            \item $x \in (A \cup B) \setminus (A \cap B)$\hfill(Set difference law)
            \item $\therefore (A \setminus B) \cup (B \setminus A) \subseteq (A \cup B) \setminus (A \cap B)$
          \end{enumerate}
          \item Prove $(A \cup B) \setminus (A \cap B) \subseteq (A \setminus B) \cup (B \setminus A)$
          \begin{enumerate}[label=2.\arabic*]
            \item Suppose $x \in (A \cup B) \setminus (A \cap B)$
            \item $x \in (A \cup B) \cap \overline{(A \cap B)}$\hfill(Set difference law)
            \item $x \in (A \cup B) \land x \in \overline{(A \cap B)}$\hfill(Definition of intersection)
            \item $(x \in A \lor x \in B) \land x \in \overline{(A \cap B)}$\hfill(Definition of union)
            \item Case 1: $x \in A \land x \not\in B \Rightarrow x \in A \setminus B$\hfill(Definition of set difference)
            \item Case 2: $x \in B \land x \not\in A \Rightarrow x \in B \setminus A$\hfill(Definition of set difference)
            \item In either case, $x \in (A \setminus B) \cup (B \setminus A)$\hfill(Definition of union)
            \item $\therefore (A \cup B) \setminus (A \cap B) \subseteq (A \setminus B) \cup (B \setminus A)$
          \end{enumerate}
          \item Therefore, $(A \setminus B) \cup (B \setminus A) = (A \cup B) \setminus (A \cap B)\qed$\hfill(Definition of set equality)
      \end{enumerate}
    \end{enumerate}

  \item For $\{G, H, R, S\}$ to be a partition of $HSWW$, the sets $G, H, R, S$ must be mutually disjoint and non-empty. Symbolically, if $HOUSE_i \in (G, H, R, S)$, when $n\neq k$, $HOUSE_n \cap HOUSE_k = \varnothing \land HOUSE_n \neq \varnothing$. $\qed$ 

    \pagebreak
  \item 
    \begin{enumerate}[(\alph*)]
      \item $A_{-2}$
        \begin{align*}
          A_{-2} &= \{\} \qed\\
                 &= \{x \in \ZZ : -2 \leq x \leq -4\}\qed\\
                 &= [-2, -4]\qed\\
        \end{align*}
      \item $\displaystyle \bigcup^5_{i=3}A_i$
        \begin{align*}
          \bigcup^5_{i=3}A_i &= \{3,4,\ldots,10\}\qed\\
                             &= \{x \in \ZZ : 3 \leq x \leq 10\}\qed\\
                             &= [3, 10] \qed\\
        \end{align*}
      \item $\displaystyle \bigcap^5_{i=3}$
        \begin{align*}
          \bigcap^5_{i=3} &= \{5, 6\} \qed \\
                          &= {x \in \ZZ : 5 \leq x \leq 6} \qed\\
                          &= [5, 6] \qed
        \end{align*}
    \end{enumerate}

  \pagebreak
  \item 
    \begin{enumerate}[(\alph*)]
      \item $\displaystyle \bigcup^4_{i=1}V_i$
        \begin{align*}
          \bigcup^4_{i=1}V_i &= V_1 \cup V_2 \cup V_3 \cup V_4\\
                             &= [-\frac{1}{1}, \frac{1}{1}] \cup [-\frac{1}{2}, \frac{1}{2}] \cup [-\frac{1}{3}, \frac{1}{3}] \cup [-\frac{1}{4}, \frac{1}{4}]\\
                             &= [-\frac{1}{1}, \frac{1}{1}]\\
                             &= [-1, 1] \qed
        \end{align*}
      \item $\displaystyle \bigcap^4_{i=1}V_i$
        \begin{align*}
          \bigcap^4_{i=1}V_i &= V_1 \cap V_2 \cap V_3 \cap V_4\\
                             &= [-\frac{1}{1}, \frac{1}{1}] \cap [-\frac{1}{2}, \frac{1}{2}] \cap [-\frac{1}{3}, \frac{1}{3}] \cap [-\frac{1}{4}, \frac{1}{4}]\\
                             &= [-\frac{1}{4}, \frac{1}{4}]\\
                             &= [-\frac{1}{4}, \frac{1}{4}] \qed
        \end{align*}
      \item $\displaystyle \bigcup^n_{i=1}V_i$
        \begin{align*}
          \bigcup^n_{i=1}V_i &= V_1 \cup \ldots \cup V_n\\
                             &= [-\frac{1}{1}, \frac{1}{1}] \cup \ldots \cup [-\frac{1}{n}, \frac{1}{n}]\\
                             &= [-\frac{1}{1}, \frac{1}{1}]\\
                             &= [-1, 1]\qed
        \end{align*}
      
      \item $\displaystyle \bigcap^n_{i=1}V_i$
        \begin{align*}
          \bigcap^n_{i=1}V_i &= V_1 \cap \ldots \cap V_n\\
                             &= [-\frac{1}{1}, \frac{1}{1}] \cap \ldots \cap [-\frac{1}{n}, \frac{1}{n}]\\
                             &= [-\frac{1}{n}, \frac{1}{n}]\\
                             &= [-\frac{1}{n}, \frac{1}{n}]\qed
        \end{align*}

      \item No, because $\forall V_n, 0 \in V_n.\qed$
    \end{enumerate}


\end{enumerate}
%%%%%%%%%%%%%%%%%%%%%%%%%%%%%%%%%%%%%%%%%%%%%%%%%%%%%%
%                       End                          %
%%%%%%%%%%%%%%%%%%%%%%%%%%%%%%%%%%%%%%%%%%%%%%%%%%%%%%

\end{document}

\documentclass[12pt, a4paper]{article}

\usepackage[a4paper, margin=1in]{geometry}

\usepackage[utf8]{inputenc}
\usepackage[mathscr]{euscript}
\let\euscr\mathscr \let\mathscr\relax
\usepackage[scr]{rsfso}
\usepackage{amssymb,amsmath,amsthm,amsfonts}
\usepackage[shortlabels]{enumitem}
\usepackage{multicol,multirow}
\usepackage{lipsum}
\usepackage{balance}
\usepackage{calc}
\usepackage[colorlinks=true,citecolor=blue,linkcolor=blue]{hyperref}
\usepackage{import}
\usepackage{xifthen}
\usepackage{pdfpages}
\usepackage{transparent}
\usepackage{listings}

\newcommand{\incfig}[2][1.0]{
    \def\svgwidth{#1\columnwidth}
    \import{./figures/}{#2.pdf_tex}
}

\newlist{enumproof}{enumerate}{4}
\setlist[enumproof,1]{label=\arabic*., parsep=1em}
\setlist[enumproof,2]{label=\arabic{enumproofi}.\arabic*., parsep=1em}
\setlist[enumproof,3]{label=\arabic{enumproofi}.\arabic{enumproofii}.\arabic*., parsep=1em}
\setlist[enumproof,4]{label=\arabic{enumproofi}.\arabic{enumproofii}.\arabic{enumproofiii}.\arabic*., parsep=1em}

\renewcommand{\qedsymbol}{\ensuremath{\blacksquare}}

\lstdefinestyle{mystyle}{
  language=C, % Set the language to C
  commentstyle=\color{codegray}, % Color for comments
  keywordstyle=\color{orange}, % Color for basic keywords
  stringstyle=\color{mauve}, % Color for strings
  basicstyle={\ttfamily\footnotesize}, % Basic font style
  breakatwhitespace=false,         
  breaklines=true,                 
  captionpos=b,                    
  keepspaces=true,                 
  numbers=none,                    
  tabsize=2,
  morekeywords=[2]{\#include, \#define, \#ifdef, \#ifndef, \#endif, \#pragma, \#else, \#elif}, % Preprocessor directives
  keywordstyle=[2]\color{codegreen}, % Style for preprocessor directives
  morekeywords=[3]{int, char, float, double, void, struct, union, enum, const, volatile, static, extern, register, inline, restrict, _Bool, _Complex, _Imaginary, size_t, ssize_t, FILE}, % C standard types and common identifiers
  keywordstyle=[3]\color{identblue}, % Style for types and common identifiers
  morekeywords=[4]{printf, scanf, fopen, fclose, malloc, free, calloc, realloc, perror, strtok, strncpy, strcpy, strcmp, strlen}, % Standard library functions
  keywordstyle=[4]\color{cyan}, % Style for library functions
}

% Things Lie
\newcommand{\kb}{\mathfrak b}
\newcommand{\kg}{\mathfrak g}
\newcommand{\kh}{\mathfrak h}
\newcommand{\kn}{\mathfrak n}
\newcommand{\ku}{\mathfrak u}
\newcommand{\kz}{\mathfrak z}
\DeclareMathOperator{\Ext}{Ext} % Ext functor
\DeclareMathOperator{\Tor}{Tor} % Tor functor
\newcommand{\gl}{\opname{\mathfrak{gl}}} % frak gl group
\renewcommand{\sl}{\opname{\mathfrak{sl}}} % frak sl group chktex 6

% More script letters etc.
\newcommand{\SA}{\mathcal A}
\newcommand{\SB}{\mathcal B}
\newcommand{\SC}{\mathcal C}
\newcommand{\SF}{\mathcal F}
\newcommand{\SG}{\mathcal G}
\newcommand{\SH}{\mathcal H}
\newcommand{\OO}{\mathcal O}

\newcommand{\SCA}{\mathscr A}
\newcommand{\SCB}{\mathscr B}
\newcommand{\SCC}{\mathscr C}
\newcommand{\SCD}{\mathscr D}
\newcommand{\SCE}{\mathscr E}
\newcommand{\SCF}{\mathscr F}
\newcommand{\SCG}{\mathscr G}
\newcommand{\SCH}{\mathscr H}

% Mathfrak primes
\newcommand{\km}{\mathfrak m}
\newcommand{\kp}{\mathfrak p}
\newcommand{\kq}{\mathfrak q}

% number sets
\newcommand{\RR}[1][]{\ensuremath{\ifstrempty{#1}{\mathbb{R}}{\mathbb{R}^{#1}}}}
\newcommand{\NN}[1][]{\ensuremath{\ifstrempty{#1}{\mathbb{N}}{\mathbb{N}^{#1}}}}
\newcommand{\ZZ}[1][]{\ensuremath{\ifstrempty{#1}{\mathbb{Z}}{\mathbb{Z}^{#1}}}}
\newcommand{\QQ}[1][]{\ensuremath{\ifstrempty{#1}{\mathbb{Q}}{\mathbb{Q}^{#1}}}}
\newcommand{\CC}[1][]{\ensuremath{\ifstrempty{#1}{\mathbb{C}}{\mathbb{C}^{#1}}}}
\newcommand{\PP}[1][]{\ensuremath{\ifstrempty{#1}{\mathbb{P}}{\mathbb{P}^{#1}}}}
\newcommand{\HH}[1][]{\ensuremath{\ifstrempty{#1}{\mathbb{H}}{\mathbb{H}^{#1}}}}
\newcommand{\FF}[1][]{\ensuremath{\ifstrempty{#1}{\mathbb{F}}{\mathbb{F}^{#1}}}}
% expected value
\newcommand{\EE}{\ensuremath{\mathbb{E}}}
\newcommand{\charin}{\text{ char }}
\DeclareMathOperator{\sign}{sign}
\DeclareMathOperator{\Aut}{Aut}
\DeclareMathOperator{\Inn}{Inn}
\DeclareMathOperator{\Syl}{Syl}
\DeclareMathOperator{\Gal}{Gal}
\DeclareMathOperator{\GL}{GL} % General linear group
\DeclareMathOperator{\SL}{SL} % Special linear group

%---------------------------------------
% BlackBoard Math Fonts :-
%---------------------------------------

%Captital Letters
\newcommand{\bbA}{\mathbb{A}}	\newcommand{\bbB}{\mathbb{B}}
\newcommand{\bbC}{\mathbb{C}}	\newcommand{\bbD}{\mathbb{D}}
\newcommand{\bbE}{\mathbb{E}}	\newcommand{\bbF}{\mathbb{F}}
\newcommand{\bbG}{\mathbb{G}}	\newcommand{\bbH}{\mathbb{H}}
\newcommand{\bbI}{\mathbb{I}}	\newcommand{\bbJ}{\mathbb{J}}
\newcommand{\bbK}{\mathbb{K}}	\newcommand{\bbL}{\mathbb{L}}
\newcommand{\bbM}{\mathbb{M}}	\newcommand{\bbN}{\mathbb{N}}
\newcommand{\bbO}{\mathbb{O}}	\newcommand{\bbP}{\mathbb{P}}
\newcommand{\bbQ}{\mathbb{Q}}	\newcommand{\bbR}{\mathbb{R}}
\newcommand{\bbS}{\mathbb{S}}	\newcommand{\bbT}{\mathbb{T}}
\newcommand{\bbU}{\mathbb{U}}	\newcommand{\bbV}{\mathbb{V}}
\newcommand{\bbW}{\mathbb{W}}	\newcommand{\bbX}{\mathbb{X}}
\newcommand{\bbY}{\mathbb{Y}}	\newcommand{\bbZ}{\mathbb{Z}}

%---------------------------------------
% MathCal Fonts :-
%---------------------------------------

%Captital Letters
\newcommand{\mcA}{\mathcal{A}}	\newcommand{\mcB}{\mathcal{B}}
\newcommand{\mcC}{\mathcal{C}}	\newcommand{\mcD}{\mathcal{D}}
\newcommand{\mcE}{\mathcal{E}}	\newcommand{\mcF}{\mathcal{F}}
\newcommand{\mcG}{\mathcal{G}}	\newcommand{\mcH}{\mathcal{H}}
\newcommand{\mcI}{\mathcal{I}}	\newcommand{\mcJ}{\mathcal{J}}
\newcommand{\mcK}{\mathcal{K}}	\newcommand{\mcL}{\mathcal{L}}
\newcommand{\mcM}{\mathcal{M}}	\newcommand{\mcN}{\mathcal{N}}
\newcommand{\mcO}{\mathcal{O}}	\newcommand{\mcP}{\mathcal{P}}
\newcommand{\mcQ}{\mathcal{Q}}	\newcommand{\mcR}{\mathcal{R}}
\newcommand{\mcS}{\mathcal{S}}	\newcommand{\mcT}{\mathcal{T}}
\newcommand{\mcU}{\mathcal{U}}	\newcommand{\mcV}{\mathcal{V}}
\newcommand{\mcW}{\mathcal{W}}	\newcommand{\mcX}{\mathcal{X}}
\newcommand{\mcY}{\mathcal{Y}}	\newcommand{\mcZ}{\mathcal{Z}}

%---------------------------------------
% Bold Math Fonts :-
%---------------------------------------

%Captital Letters
\newcommand{\bmA}{\boldsymbol{A}}	\newcommand{\bmB}{\boldsymbol{B}}
\newcommand{\bmC}{\boldsymbol{C}}	\newcommand{\bmD}{\boldsymbol{D}}
\newcommand{\bmE}{\boldsymbol{E}}	\newcommand{\bmF}{\boldsymbol{F}}
\newcommand{\bmG}{\boldsymbol{G}}	\newcommand{\bmH}{\boldsymbol{H}}
\newcommand{\bmI}{\boldsymbol{I}}	\newcommand{\bmJ}{\boldsymbol{J}}
\newcommand{\bmK}{\boldsymbol{K}}	\newcommand{\bmL}{\boldsymbol{L}}
\newcommand{\bmM}{\boldsymbol{M}}	\newcommand{\bmN}{\boldsymbol{N}}
\newcommand{\bmO}{\boldsymbol{O}}	\newcommand{\bmP}{\boldsymbol{P}}
\newcommand{\bmQ}{\boldsymbol{Q}}	\newcommand{\bmR}{\boldsymbol{R}}
\newcommand{\bmS}{\boldsymbol{S}}	\newcommand{\bmT}{\boldsymbol{T}}
\newcommand{\bmU}{\boldsymbol{U}}	\newcommand{\bmV}{\boldsymbol{V}}
\newcommand{\bmW}{\boldsymbol{W}}	\newcommand{\bmX}{\boldsymbol{X}}
\newcommand{\bmY}{\boldsymbol{Y}}	\newcommand{\bmZ}{\boldsymbol{Z}}
%Small Letters
\newcommand{\bma}{\boldsymbol{a}}	\newcommand{\bmb}{\boldsymbol{b}}
\newcommand{\bmc}{\boldsymbol{c}}	\newcommand{\bmd}{\boldsymbol{d}}
\newcommand{\bme}{\boldsymbol{e}}	\newcommand{\bmf}{\boldsymbol{f}}
\newcommand{\bmg}{\boldsymbol{g}}	\newcommand{\bmh}{\boldsymbol{h}}
\newcommand{\bmi}{\boldsymbol{i}}	\newcommand{\bmj}{\boldsymbol{j}}
\newcommand{\bmk}{\boldsymbol{k}}	\newcommand{\bml}{\boldsymbol{l}}
\newcommand{\bmm}{\boldsymbol{m}}	\newcommand{\bmn}{\boldsymbol{n}}
\newcommand{\bmo}{\boldsymbol{o}}	\newcommand{\bmp}{\boldsymbol{p}}
\newcommand{\bmq}{\boldsymbol{q}}	\newcommand{\bmr}{\boldsymbol{r}}
\newcommand{\bms}{\boldsymbol{s}}	\newcommand{\bmt}{\boldsymbol{t}}
\newcommand{\bmu}{\boldsymbol{u}}	\newcommand{\bmv}{\boldsymbol{v}}
\newcommand{\bmw}{\boldsymbol{w}}	\newcommand{\bmx}{\boldsymbol{x}}
\newcommand{\bmy}{\boldsymbol{y}}	\newcommand{\bmz}{\boldsymbol{z}}

%---------------------------------------
% Scr Math Fonts :-
%---------------------------------------

\newcommand{\sA}{{\mathscr{A}}}   \newcommand{\sB}{{\mathscr{B}}}
\newcommand{\sC}{{\mathscr{C}}}   \newcommand{\sD}{{\mathscr{D}}}
\newcommand{\sE}{{\mathscr{E}}}   \newcommand{\sF}{{\mathscr{F}}}
\newcommand{\sG}{{\mathscr{G}}}   \newcommand{\sH}{{\mathscr{H}}}
\newcommand{\sI}{{\mathscr{I}}}   \newcommand{\sJ}{{\mathscr{J}}}
\newcommand{\sK}{{\mathscr{K}}}   \newcommand{\sL}{{\mathscr{L}}}
\newcommand{\sM}{{\mathscr{M}}}   \newcommand{\sN}{{\mathscr{N}}}
\newcommand{\sO}{{\mathscr{O}}}   \newcommand{\sP}{{\mathscr{P}}}
\newcommand{\sQ}{{\mathscr{Q}}}   \newcommand{\sR}{{\mathscr{R}}}
\newcommand{\sS}{{\mathscr{S}}}   \newcommand{\sT}{{\mathscr{T}}}
\newcommand{\sU}{{\mathscr{U}}}   \newcommand{\sV}{{\mathscr{V}}}
\newcommand{\sW}{{\mathscr{W}}}   \newcommand{\sX}{{\mathscr{X}}}
\newcommand{\sY}{{\mathscr{Y}}}   \newcommand{\sZ}{{\mathscr{Z}}}


%---------------------------------------
% Math Fraktur Font
%---------------------------------------

%Captital Letters
\newcommand{\mfA}{\mathfrak{A}}	\newcommand{\mfB}{\mathfrak{B}}
\newcommand{\mfC}{\mathfrak{C}}	\newcommand{\mfD}{\mathfrak{D}}
\newcommand{\mfE}{\mathfrak{E}}	\newcommand{\mfF}{\mathfrak{F}}
\newcommand{\mfG}{\mathfrak{G}}	\newcommand{\mfH}{\mathfrak{H}}
\newcommand{\mfI}{\mathfrak{I}}	\newcommand{\mfJ}{\mathfrak{J}}
\newcommand{\mfK}{\mathfrak{K}}	\newcommand{\mfL}{\mathfrak{L}}
\newcommand{\mfM}{\mathfrak{M}}	\newcommand{\mfN}{\mathfrak{N}}
\newcommand{\mfO}{\mathfrak{O}}	\newcommand{\mfP}{\mathfrak{P}}
\newcommand{\mfQ}{\mathfrak{Q}}	\newcommand{\mfR}{\mathfrak{R}}
\newcommand{\mfS}{\mathfrak{S}}	\newcommand{\mfT}{\mathfrak{T}}
\newcommand{\mfU}{\mathfrak{U}}	\newcommand{\mfV}{\mathfrak{V}}
\newcommand{\mfW}{\mathfrak{W}}	\newcommand{\mfX}{\mathfrak{X}}
\newcommand{\mfY}{\mathfrak{Y}}	\newcommand{\mfZ}{\mathfrak{Z}}
%Small Letters
\newcommand{\mfa}{\mathfrak{a}}	\newcommand{\mfb}{\mathfrak{b}}
\newcommand{\mfc}{\mathfrak{c}}	\newcommand{\mfd}{\mathfrak{d}}
\newcommand{\mfe}{\mathfrak{e}}	\newcommand{\mff}{\mathfrak{f}}
\newcommand{\mfg}{\mathfrak{g}}	\newcommand{\mfh}{\mathfrak{h}}
\newcommand{\mfi}{\mathfrak{i}}	\newcommand{\mfj}{\mathfrak{j}}
\newcommand{\mfk}{\mathfrak{k}}	\newcommand{\mfl}{\mathfrak{l}}
\newcommand{\mfm}{\mathfrak{m}}	\newcommand{\mfn}{\mathfrak{n}}
\newcommand{\mfo}{\mathfrak{o}}	\newcommand{\mfp}{\mathfrak{p}}
\newcommand{\mfq}{\mathfrak{q}}	\newcommand{\mfr}{\mathfrak{r}}
\newcommand{\mfs}{\mathfrak{s}}	\newcommand{\mft}{\mathfrak{t}}
\newcommand{\mfu}{\mathfrak{u}}	\newcommand{\mfv}{\mathfrak{v}}
\newcommand{\mfw}{\mathfrak{w}}	\newcommand{\mfx}{\mathfrak{x}}
\newcommand{\mfy}{\mathfrak{y}}	\newcommand{\mfz}{\mathfrak{z}}


\newcommand{\mytitle}{CS1231S Tutorial 4}
\newcommand{\myauthor}{github/omgeta}
\newcommand{\mydate}{AY 24/25 Sem 1}

\begin{document}
\raggedright
\footnotesize
\begin{center}
{\normalsize{\textbf{\mytitle}}} \\
{\footnotesize{\mydate\hspace{2pt}\textemdash\hspace{2pt}\myauthor}}
\end{center}
\setlist{topsep=-1em, itemsep=-1em, parsep=2em}
%%%%%%%%%%%%%%%%%%%%%%%%%%%%%%%%%%%%%%%%%%%%%%%%%%%%%%
%                      Begin                         %
%%%%%%%%%%%%%%%%%%%%%%%%%%%%%%%%%%%%%%%%%%%%%%%%%%%%%%
\begin{enumerate}[Q\arabic*.]
  \item
    \begin{align*}
      R = \{
        (2,2), (2,4), (2,6), (2,8), (2, 10), (2, 12), (2, 14),\\
        (3,6), (3, 12),\\
        (5, 10),\\
        (7, 14),\\
      \}\qed\\
      R^{-1} = \{
        (14,2), (12,2), (10,2), (8,2), (6,2), (4,2), (2, 2),\\
        (12,3), (6,3),\\
        (10,5),\\
      (14,7)\} \qed\\
    \end{align*}


  \item 
    \begin{enumerate}[label=\arabic*., parsep=1em]
      \item Prove $R$ is symmetric $\rightarrow \forall x, y \in A(x R y \iff y R x)$:
        \begin{enumerate}[label=1.\arabic*., parsep=1em]
          \item Suppose $R$ is symmetric, i.e. $\forall x,y\in A(xRy \rightarrow yRx)$
          \item Then, $\forall y,x\in A(yRx\rightarrow xRy)$\hfill(By supposition 1.1)
          \item $\therefore \forall x,y\in A(xRy \iff yRx)$\hfill(Definition of iff)
        \end{enumerate}
      \item Prove $\forall x, y \in A(x R y \iff y R x) \rightarrow R = R^{-1}$:
        \begin{enumerate}[label=2.\arabic*., parsep=1em]
          \item Suppose $\forall x, y \in A(x R y \iff y R x)$
          \item Suppose $(x, y) \in R$:
          \begin{enumerate}[label=2.2.\arabic*., parsep=1em]
            \item $\iff xRy$\hfill(Definition of $R$)
            \item $\iff yRx$\hfill(By supposition 2.1)
            \item $\iff xR^{-1}y$\hfill(Definition of $R^{-1}$)
            \item $\iff (x, y) \in R^{-1}$
          \end{enumerate}
          \item $R = R^{-1}$
        \end{enumerate}
      \item Prove $R = R^{-1} \rightarrow R$ is symmetric:
        \begin{enumerate}[label=3.\arabic*., parsep=1em]
          \item Suppose $R = R^{-1}$ and $(x, y) \in R$
          \item $(y, x) \in R^{-1}$\hfill(Definition of $R^{-1}$)
          \item $(y, x)\in R$\hfill(By supposition 3.1)
          \item $\therefore \forall x, y \in A((x, y) \in R \rightarrow (y, x) \in R)$\hfill(Universal generalization)
          \item $\therefore \forall x,y\in A(xRy \rightarrow yRx)$\hfill(Definition of relation)
        \end{enumerate}
      \item Hence, $R$ is symmetric $\iff \forall x, y \in A(x R y \iff y R x) \iff R = R^{-1} \qed$\hfill(Definition of iff)
    \end{enumerate}
  \pagebreak
  \item 
    \begin{enumerate}[(\alph*)]
      \item $Q$ is reflexive.$\qed$\\
        $Q$ is not symmetric. Counterexample: $(1, 2) \in Q \land (2, 1) \not\in Q \qed$\\
        $Q$ is transitive.$\qed$\\
        $\therefore Q$ is not an equivalence relation. $\qed$
      \item $E$ is reflexive.$\qed$\\
        $E$ is symmetric.$\qed$\\
        $E$ is transitive.$\qed$\\
        $\therefore E$ is an equivalence relation. $\qed$
      \item $R$ is reflexive.$\qed$\\
        $R$ is symmetric.$\qed$\\
        $R$ is not transitive. Counterexample: $(1, 0) \in R \land (0, -1) \in R \land (1, -1) \not\in R\qed$\\
        $\therefore R$ is not an equivalence relation. $\qed$
      \item $S$ is not reflexive. Counterexample: $(0, 0) \not\in S\qed$\\
        $S$ is symmetric.$\qed$\\
        $S$ is transitive.$\qed$\\
        $\therefore R$ is not an equivalence relation. $\qed$
      \item $T$ is reflexive.$\qed$\\
        $T$ is symmetric.$\qed$\\
        $T$ is not transitive. Counterexample: $(2, 1) \in T \land (1, -1) \in T \land (2, -1) \not\in T\qed$\\
        $\therefore T$ is not an equivalence relation. $\qed$
    \end{enumerate}
  \pagebreak
  \item 
    \begin{enumerate}[(\alph*)]
      \item $R \circ R$ is not transitive. Counterexample: $(a, c) \in R\circ R \land (c, b) \in R\circ R \land (a, c) \not\in R\circ R \qed$
        \begin{center}
          \incfig[0.3]{4a}
        \end{center}

      \item $R \circ R \circ R$ is transitive. $\qed$
        \begin{center}
          \incfig[0.3]{4b}
        \end{center}

      \item $(R \circ R)\cup R$ is transitive. $\qed$
        \begin{center}
          \incfig[0.3]{4b}
        \end{center}
    \end{enumerate}

  \pagebreak
  \item
    \begin{enumerate}[(\alph*)]
      \item True. $\qed$
      \item True. $\qed$
        \begin{enumerate}[label=\arabic*.,parsep=1em]
          \item Suppose $(x, y) \in R$:
            \begin{enumerate}[label=1.\arabic*., parsep=1em]
              \item $(y, y) \in R$\hfill(Reflexivity of $R$)
              \item $(x, y) \in R\circ R$\hfill(Definition of composition)
          \end{enumerate}
          \item $\forall (x,y)\in R((x,y) \in R\circ R)$\hfill(Universal generalization)
          \item $R \subseteq R\circ R$
        \end{enumerate}
      \item True. $\qed$
        \begin{enumerate}[label=\arabic*.,parsep=1em]
          \item Suppose $(x, y) \in R\circ R$:
            \begin{enumerate}[label=1.\arabic*., parsep=1em]
              \item $\exists z(xRz \land zRy)$\hfill(Definition of $R\circ R$)
              \item $(x, z) \in R \land (z, y) \in R$\hfill(Definition of $R$)
              \item $(x, y) \in R$\hfill(Transitivity of $R$)
          \end{enumerate}
          \item $\forall (x,y)\in R\circ R((x,y) \in R)$\hfill(Universal generalization)
          \item $R\circ R\subseteq R$
        \end{enumerate}
      \item True. $\qed$
    \end{enumerate}

  \item From Q5, $R \subseteq R\circ R \land R\circ R \subseteq R$, therefore by definition of set equality, $R = R\circ R$. This means:
    \begin{align*}
      &R \circ R \circ R \circ R \circ R \circ R \circ R \\
      &= (R) \circ (R) \circ (R) \circ R \\
      &= (R) \circ (R)\\
      &= R \qed
    \end{align*}

  \item $T \circ (S \circ R)$\\
    $= \{(a,d) \in A\times D : \exists c \in C ((a, c) \in S\circ R \land (c, d) \in T)\}$\hfill(Definition of composition)
    $= \{(a,d) \in A\times D : \exists c \in C ((\exists b \in B ((a, b) \in R \land (b, c) \in S)) \land (c, d) \in T)\}$\hfill(Definition of $S\circ R$)
    $= \{(a,d) \in A\times D : \exists b \in B\exists c \in C ((a, b) \in R \land (b, c) \in S \land (c, d) \in T)\}$\hfill(Distributive law)
    $= \{(a,d) \in A\times D : \exists b \in B((a, b) \in R \land (\exists c \in C((b, c) \in S \land (c, d) \in T)))\}$\hfill(Distributive law)
    $= \{(a,d) \in A\times D : \exists b \in B((a, b) \in R \land (b, d) \in T \circ S)\}$\hfill(Definition of $T \circ S$)\\
   $= (T \circ S) \circ R \qed$\hfill(Definition of composition) 

  \item $[(1, 1)] = \{(1, 1)\} \qed$\\
    $[(4, 3)] = \{(4, 3), (3, 4), (6, 2), (2, 6), (12, 1), (1, 12)\} \qed$

  \item 
    \begin{enumerate}[(\alph*)]
      \item $S^{-1} = \{(n, m) \in \ZZ^2: (m, n) \in S\}$\hfill(Definition of inverse relation)\\
        $= \{(n, m) \in \ZZ^2: m^3 + n^3\text{ is even}\}$\hfill(Definition of S)\\
        $= \{(m, n) \in \ZZ^2: m^3 + n^3\text{ is even}\}$\hfill(F1. Commutativity of addition)\\
        $= S\qed$

      \item \begin{enumerate}[label=\arabic*., parsep=1em]
        \item Prove $S\circ S \subseteq S$
          \begin{enumerate}[label=1.\arabic*., parsep=1em]
            \item Suppose $(x, z) \in S \circ S$:
            \item $\exists y (x^3 + y^3$ is even $\land y^3 + z^3$is even$)$\hfill(Definition of composition)
        \item $x^3 + 2y^3 + z^3$ is even
            \item Since $2y^3$ is even, $x^3 + z^3$ is even
            \item $(x, z) \in S$\hfill(Definition of $S$)
          \end{enumerate}
        \item Prove $S \subseteq S\circ S$
          \begin{enumerate}[label=2.\arabic*., parsep=1em]
            \item Suppose $(x, z) \in S$:
            \item $(x, x) \in S$\hfill($x^3 + x^3$ is even)
            \item $(x, z) \in S\circ S$\hfill(Definition of composition)
          \end{enumerate}
        \item $\therefore S\circ S = S$
      \end{enumerate}

      \item $S\circ S^{-1} = S\circ S$\hfill(By 9a)\\
        $= S\qed$\hfill(By 9b)
    \end{enumerate}

  \pagebreak
  \item 
    \begin{enumerate}[(\alph*)]
      \item \begin{enumerate}[label=\arabic*., parsep=1em]
        \item Prove $\sim$ is reflexive:
          \begin{enumerate}[label=1.\arabic*., itemsep=-2em]
            \item Suppose $a \in \ZZ \setminus \{0\}$
            \item $a \cdot a = a^2 > 0$\hfill(T21. $a\neq0 \rightarrow a^2 > 0$)
            \item $\therefore \forall a \in \ZZ \setminus \{0\}, a \sim a$\hfill(Universal generalization)
            \item $\therefore$  $\sim$ is reflexive\hfill(Definition of reflexivity)
          \end{enumerate}
        \item Prove $\sim$ is symmetric:
          \begin{enumerate}[label=2.\arabic*., itemsep=-2em]
            \item Suppose $a \sim b$, then $ab > 0$\hfill(Definition of $\sim$)
            \item $ba = ab > 0$\hfill(F1. $\forall a, b \in \RR, ab = ba$)
            \item $b \sim a$\hfill(Definition of $\sim$)
            \item $\therefore \forall a, b \in \ZZ(a \sim b \rightarrow b \sim a)$\hfill(Universal generalization)
            \item $\therefore$ $\sim$ is symmetric\hfill(Definition of symmetry)
          \end{enumerate}
        \item Prove $\sim$ is transitive:
          \begin{enumerate}[label=3.\arabic*., itemsep=-2em]
            \item Suppose $a \sim b \land b \sim c$, then $ab > 0 \land bc > 0$\hfill(Definition of $\sim$)
            \item $(ab)(bc) =ab^2c > 0$\hfill(T25. $ab > 0 \iff (a>0 \land b > 0) \lor (a < 0 \land b < 0)$)
            \item $b^2 > 0$\hfill(T21. $a\neq 0 \rightarrow a^2 > 0$)
            \item $\therefore ac > 0$\hfill(T25. $ab > 0 \iff (a>0 \land b > 0) \lor (a < 0 \land b < 0)$)
            \item $a \sim c$\hfill(Definition of $\sim$)
            \item $\forall a,b,c\in\ZZ(((a\sim b) \land (b \sim c)) \rightarrow a \sim c)$\hfill(Universal generalization)
            \item $\therefore$ $\sim$ is transitive\hfill(Definition of transitivity)
          \end{enumerate}
        \item Hence, $\sim$ is reflexive, symmetric and transitive.
        \item $\therefore$ $\sim$ is an equivalence relation. $\qed$
      \end{enumerate}
      \item $(\ZZ\setminus\{0\})/\sim$\\ $=\{\ZZ^+, \ZZ^-\} \qed$
    \end{enumerate}
\end{enumerate}
%%%%%%%%%%%%%%%%%%%%%%%%%%%%%%%%%%%%%%%%%%%%%%%%%%%%%%
%                       End                          %
%%%%%%%%%%%%%%%%%%%%%%%%%%%%%%%%%%%%%%%%%%%%%%%%%%%%%%

\end{document}

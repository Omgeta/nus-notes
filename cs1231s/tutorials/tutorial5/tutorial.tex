\documentclass[12pt, a4paper]{article}

\usepackage[a4paper, margin=1in]{geometry}

\usepackage[utf8]{inputenc}
\usepackage[mathscr]{euscript}
\let\euscr\mathscr \let\mathscr\relax
\usepackage[scr]{rsfso}
\usepackage{amssymb,amsmath,amsthm,amsfonts}
\usepackage[shortlabels]{enumitem}
\usepackage{multicol,multirow}
\usepackage{lipsum}
\usepackage{balance}
\usepackage{calc}
\usepackage[colorlinks=true,citecolor=blue,linkcolor=blue]{hyperref}
\usepackage{import}
\usepackage{xifthen}
\usepackage{pdfpages}
\usepackage{transparent}
\usepackage{listings}

\newcommand{\incfig}[2][1.0]{
    \def\svgwidth{#1\columnwidth}
    \import{./figures/}{#2.pdf_tex}
}

\newlist{enumproof}{enumerate}{4}
\setlist[enumproof,1]{label=\arabic*., parsep=1em}
\setlist[enumproof,2]{label=\arabic{enumproofi}.\arabic*., parsep=1em}
\setlist[enumproof,3]{label=\arabic{enumproofi}.\arabic{enumproofii}.\arabic*., parsep=1em}
\setlist[enumproof,4]{label=\arabic{enumproofi}.\arabic{enumproofii}.\arabic{enumproofiii}.\arabic*., parsep=1em}

\renewcommand{\qedsymbol}{\ensuremath{\blacksquare}}

\lstdefinestyle{mystyle}{
  language=C, % Set the language to C
  commentstyle=\color{codegray}, % Color for comments
  keywordstyle=\color{orange}, % Color for basic keywords
  stringstyle=\color{mauve}, % Color for strings
  basicstyle={\ttfamily\footnotesize}, % Basic font style
  breakatwhitespace=false,         
  breaklines=true,                 
  captionpos=b,                    
  keepspaces=true,                 
  numbers=none,                    
  tabsize=2,
  morekeywords=[2]{\#include, \#define, \#ifdef, \#ifndef, \#endif, \#pragma, \#else, \#elif}, % Preprocessor directives
  keywordstyle=[2]\color{codegreen}, % Style for preprocessor directives
  morekeywords=[3]{int, char, float, double, void, struct, union, enum, const, volatile, static, extern, register, inline, restrict, _Bool, _Complex, _Imaginary, size_t, ssize_t, FILE}, % C standard types and common identifiers
  keywordstyle=[3]\color{identblue}, % Style for types and common identifiers
  morekeywords=[4]{printf, scanf, fopen, fclose, malloc, free, calloc, realloc, perror, strtok, strncpy, strcpy, strcmp, strlen}, % Standard library functions
  keywordstyle=[4]\color{cyan}, % Style for library functions
}

% Things Lie
\newcommand{\kb}{\mathfrak b}
\newcommand{\kg}{\mathfrak g}
\newcommand{\kh}{\mathfrak h}
\newcommand{\kn}{\mathfrak n}
\newcommand{\ku}{\mathfrak u}
\newcommand{\kz}{\mathfrak z}
\DeclareMathOperator{\Ext}{Ext} % Ext functor
\DeclareMathOperator{\Tor}{Tor} % Tor functor
\newcommand{\gl}{\opname{\mathfrak{gl}}} % frak gl group
\renewcommand{\sl}{\opname{\mathfrak{sl}}} % frak sl group chktex 6

% More script letters etc.
\newcommand{\SA}{\mathcal A}
\newcommand{\SB}{\mathcal B}
\newcommand{\SC}{\mathcal C}
\newcommand{\SF}{\mathcal F}
\newcommand{\SG}{\mathcal G}
\newcommand{\SH}{\mathcal H}
\newcommand{\OO}{\mathcal O}

\newcommand{\SCA}{\mathscr A}
\newcommand{\SCB}{\mathscr B}
\newcommand{\SCC}{\mathscr C}
\newcommand{\SCD}{\mathscr D}
\newcommand{\SCE}{\mathscr E}
\newcommand{\SCF}{\mathscr F}
\newcommand{\SCG}{\mathscr G}
\newcommand{\SCH}{\mathscr H}

% Mathfrak primes
\newcommand{\km}{\mathfrak m}
\newcommand{\kp}{\mathfrak p}
\newcommand{\kq}{\mathfrak q}

% number sets
\newcommand{\RR}[1][]{\ensuremath{\ifstrempty{#1}{\mathbb{R}}{\mathbb{R}^{#1}}}}
\newcommand{\NN}[1][]{\ensuremath{\ifstrempty{#1}{\mathbb{N}}{\mathbb{N}^{#1}}}}
\newcommand{\ZZ}[1][]{\ensuremath{\ifstrempty{#1}{\mathbb{Z}}{\mathbb{Z}^{#1}}}}
\newcommand{\QQ}[1][]{\ensuremath{\ifstrempty{#1}{\mathbb{Q}}{\mathbb{Q}^{#1}}}}
\newcommand{\CC}[1][]{\ensuremath{\ifstrempty{#1}{\mathbb{C}}{\mathbb{C}^{#1}}}}
\newcommand{\PP}[1][]{\ensuremath{\ifstrempty{#1}{\mathbb{P}}{\mathbb{P}^{#1}}}}
\newcommand{\HH}[1][]{\ensuremath{\ifstrempty{#1}{\mathbb{H}}{\mathbb{H}^{#1}}}}
\newcommand{\FF}[1][]{\ensuremath{\ifstrempty{#1}{\mathbb{F}}{\mathbb{F}^{#1}}}}
% expected value
\newcommand{\EE}{\ensuremath{\mathbb{E}}}
\newcommand{\charin}{\text{ char }}
\DeclareMathOperator{\sign}{sign}
\DeclareMathOperator{\Aut}{Aut}
\DeclareMathOperator{\Inn}{Inn}
\DeclareMathOperator{\Syl}{Syl}
\DeclareMathOperator{\Gal}{Gal}
\DeclareMathOperator{\GL}{GL} % General linear group
\DeclareMathOperator{\SL}{SL} % Special linear group

%---------------------------------------
% BlackBoard Math Fonts :-
%---------------------------------------

%Captital Letters
\newcommand{\bbA}{\mathbb{A}}	\newcommand{\bbB}{\mathbb{B}}
\newcommand{\bbC}{\mathbb{C}}	\newcommand{\bbD}{\mathbb{D}}
\newcommand{\bbE}{\mathbb{E}}	\newcommand{\bbF}{\mathbb{F}}
\newcommand{\bbG}{\mathbb{G}}	\newcommand{\bbH}{\mathbb{H}}
\newcommand{\bbI}{\mathbb{I}}	\newcommand{\bbJ}{\mathbb{J}}
\newcommand{\bbK}{\mathbb{K}}	\newcommand{\bbL}{\mathbb{L}}
\newcommand{\bbM}{\mathbb{M}}	\newcommand{\bbN}{\mathbb{N}}
\newcommand{\bbO}{\mathbb{O}}	\newcommand{\bbP}{\mathbb{P}}
\newcommand{\bbQ}{\mathbb{Q}}	\newcommand{\bbR}{\mathbb{R}}
\newcommand{\bbS}{\mathbb{S}}	\newcommand{\bbT}{\mathbb{T}}
\newcommand{\bbU}{\mathbb{U}}	\newcommand{\bbV}{\mathbb{V}}
\newcommand{\bbW}{\mathbb{W}}	\newcommand{\bbX}{\mathbb{X}}
\newcommand{\bbY}{\mathbb{Y}}	\newcommand{\bbZ}{\mathbb{Z}}

%---------------------------------------
% MathCal Fonts :-
%---------------------------------------

%Captital Letters
\newcommand{\mcA}{\mathcal{A}}	\newcommand{\mcB}{\mathcal{B}}
\newcommand{\mcC}{\mathcal{C}}	\newcommand{\mcD}{\mathcal{D}}
\newcommand{\mcE}{\mathcal{E}}	\newcommand{\mcF}{\mathcal{F}}
\newcommand{\mcG}{\mathcal{G}}	\newcommand{\mcH}{\mathcal{H}}
\newcommand{\mcI}{\mathcal{I}}	\newcommand{\mcJ}{\mathcal{J}}
\newcommand{\mcK}{\mathcal{K}}	\newcommand{\mcL}{\mathcal{L}}
\newcommand{\mcM}{\mathcal{M}}	\newcommand{\mcN}{\mathcal{N}}
\newcommand{\mcO}{\mathcal{O}}	\newcommand{\mcP}{\mathcal{P}}
\newcommand{\mcQ}{\mathcal{Q}}	\newcommand{\mcR}{\mathcal{R}}
\newcommand{\mcS}{\mathcal{S}}	\newcommand{\mcT}{\mathcal{T}}
\newcommand{\mcU}{\mathcal{U}}	\newcommand{\mcV}{\mathcal{V}}
\newcommand{\mcW}{\mathcal{W}}	\newcommand{\mcX}{\mathcal{X}}
\newcommand{\mcY}{\mathcal{Y}}	\newcommand{\mcZ}{\mathcal{Z}}

%---------------------------------------
% Bold Math Fonts :-
%---------------------------------------

%Captital Letters
\newcommand{\bmA}{\boldsymbol{A}}	\newcommand{\bmB}{\boldsymbol{B}}
\newcommand{\bmC}{\boldsymbol{C}}	\newcommand{\bmD}{\boldsymbol{D}}
\newcommand{\bmE}{\boldsymbol{E}}	\newcommand{\bmF}{\boldsymbol{F}}
\newcommand{\bmG}{\boldsymbol{G}}	\newcommand{\bmH}{\boldsymbol{H}}
\newcommand{\bmI}{\boldsymbol{I}}	\newcommand{\bmJ}{\boldsymbol{J}}
\newcommand{\bmK}{\boldsymbol{K}}	\newcommand{\bmL}{\boldsymbol{L}}
\newcommand{\bmM}{\boldsymbol{M}}	\newcommand{\bmN}{\boldsymbol{N}}
\newcommand{\bmO}{\boldsymbol{O}}	\newcommand{\bmP}{\boldsymbol{P}}
\newcommand{\bmQ}{\boldsymbol{Q}}	\newcommand{\bmR}{\boldsymbol{R}}
\newcommand{\bmS}{\boldsymbol{S}}	\newcommand{\bmT}{\boldsymbol{T}}
\newcommand{\bmU}{\boldsymbol{U}}	\newcommand{\bmV}{\boldsymbol{V}}
\newcommand{\bmW}{\boldsymbol{W}}	\newcommand{\bmX}{\boldsymbol{X}}
\newcommand{\bmY}{\boldsymbol{Y}}	\newcommand{\bmZ}{\boldsymbol{Z}}
%Small Letters
\newcommand{\bma}{\boldsymbol{a}}	\newcommand{\bmb}{\boldsymbol{b}}
\newcommand{\bmc}{\boldsymbol{c}}	\newcommand{\bmd}{\boldsymbol{d}}
\newcommand{\bme}{\boldsymbol{e}}	\newcommand{\bmf}{\boldsymbol{f}}
\newcommand{\bmg}{\boldsymbol{g}}	\newcommand{\bmh}{\boldsymbol{h}}
\newcommand{\bmi}{\boldsymbol{i}}	\newcommand{\bmj}{\boldsymbol{j}}
\newcommand{\bmk}{\boldsymbol{k}}	\newcommand{\bml}{\boldsymbol{l}}
\newcommand{\bmm}{\boldsymbol{m}}	\newcommand{\bmn}{\boldsymbol{n}}
\newcommand{\bmo}{\boldsymbol{o}}	\newcommand{\bmp}{\boldsymbol{p}}
\newcommand{\bmq}{\boldsymbol{q}}	\newcommand{\bmr}{\boldsymbol{r}}
\newcommand{\bms}{\boldsymbol{s}}	\newcommand{\bmt}{\boldsymbol{t}}
\newcommand{\bmu}{\boldsymbol{u}}	\newcommand{\bmv}{\boldsymbol{v}}
\newcommand{\bmw}{\boldsymbol{w}}	\newcommand{\bmx}{\boldsymbol{x}}
\newcommand{\bmy}{\boldsymbol{y}}	\newcommand{\bmz}{\boldsymbol{z}}

%---------------------------------------
% Scr Math Fonts :-
%---------------------------------------

\newcommand{\sA}{{\mathscr{A}}}   \newcommand{\sB}{{\mathscr{B}}}
\newcommand{\sC}{{\mathscr{C}}}   \newcommand{\sD}{{\mathscr{D}}}
\newcommand{\sE}{{\mathscr{E}}}   \newcommand{\sF}{{\mathscr{F}}}
\newcommand{\sG}{{\mathscr{G}}}   \newcommand{\sH}{{\mathscr{H}}}
\newcommand{\sI}{{\mathscr{I}}}   \newcommand{\sJ}{{\mathscr{J}}}
\newcommand{\sK}{{\mathscr{K}}}   \newcommand{\sL}{{\mathscr{L}}}
\newcommand{\sM}{{\mathscr{M}}}   \newcommand{\sN}{{\mathscr{N}}}
\newcommand{\sO}{{\mathscr{O}}}   \newcommand{\sP}{{\mathscr{P}}}
\newcommand{\sQ}{{\mathscr{Q}}}   \newcommand{\sR}{{\mathscr{R}}}
\newcommand{\sS}{{\mathscr{S}}}   \newcommand{\sT}{{\mathscr{T}}}
\newcommand{\sU}{{\mathscr{U}}}   \newcommand{\sV}{{\mathscr{V}}}
\newcommand{\sW}{{\mathscr{W}}}   \newcommand{\sX}{{\mathscr{X}}}
\newcommand{\sY}{{\mathscr{Y}}}   \newcommand{\sZ}{{\mathscr{Z}}}


%---------------------------------------
% Math Fraktur Font
%---------------------------------------

%Captital Letters
\newcommand{\mfA}{\mathfrak{A}}	\newcommand{\mfB}{\mathfrak{B}}
\newcommand{\mfC}{\mathfrak{C}}	\newcommand{\mfD}{\mathfrak{D}}
\newcommand{\mfE}{\mathfrak{E}}	\newcommand{\mfF}{\mathfrak{F}}
\newcommand{\mfG}{\mathfrak{G}}	\newcommand{\mfH}{\mathfrak{H}}
\newcommand{\mfI}{\mathfrak{I}}	\newcommand{\mfJ}{\mathfrak{J}}
\newcommand{\mfK}{\mathfrak{K}}	\newcommand{\mfL}{\mathfrak{L}}
\newcommand{\mfM}{\mathfrak{M}}	\newcommand{\mfN}{\mathfrak{N}}
\newcommand{\mfO}{\mathfrak{O}}	\newcommand{\mfP}{\mathfrak{P}}
\newcommand{\mfQ}{\mathfrak{Q}}	\newcommand{\mfR}{\mathfrak{R}}
\newcommand{\mfS}{\mathfrak{S}}	\newcommand{\mfT}{\mathfrak{T}}
\newcommand{\mfU}{\mathfrak{U}}	\newcommand{\mfV}{\mathfrak{V}}
\newcommand{\mfW}{\mathfrak{W}}	\newcommand{\mfX}{\mathfrak{X}}
\newcommand{\mfY}{\mathfrak{Y}}	\newcommand{\mfZ}{\mathfrak{Z}}
%Small Letters
\newcommand{\mfa}{\mathfrak{a}}	\newcommand{\mfb}{\mathfrak{b}}
\newcommand{\mfc}{\mathfrak{c}}	\newcommand{\mfd}{\mathfrak{d}}
\newcommand{\mfe}{\mathfrak{e}}	\newcommand{\mff}{\mathfrak{f}}
\newcommand{\mfg}{\mathfrak{g}}	\newcommand{\mfh}{\mathfrak{h}}
\newcommand{\mfi}{\mathfrak{i}}	\newcommand{\mfj}{\mathfrak{j}}
\newcommand{\mfk}{\mathfrak{k}}	\newcommand{\mfl}{\mathfrak{l}}
\newcommand{\mfm}{\mathfrak{m}}	\newcommand{\mfn}{\mathfrak{n}}
\newcommand{\mfo}{\mathfrak{o}}	\newcommand{\mfp}{\mathfrak{p}}
\newcommand{\mfq}{\mathfrak{q}}	\newcommand{\mfr}{\mathfrak{r}}
\newcommand{\mfs}{\mathfrak{s}}	\newcommand{\mft}{\mathfrak{t}}
\newcommand{\mfu}{\mathfrak{u}}	\newcommand{\mfv}{\mathfrak{v}}
\newcommand{\mfw}{\mathfrak{w}}	\newcommand{\mfx}{\mathfrak{x}}
\newcommand{\mfy}{\mathfrak{y}}	\newcommand{\mfz}{\mathfrak{z}}


\newcommand{\mytitle}{CS1231S Tutorial 5}
\newcommand{\myauthor}{github/omgeta}
\newcommand{\mydate}{AY 24/25 Sem 1}

\begin{document}
\raggedright
\footnotesize
\begin{center}
{\normalsize{\textbf{\mytitle}}} \\
{\footnotesize{\mydate\hspace{2pt}\textemdash\hspace{2pt}\myauthor}}
\end{center}
\setlist{topsep=-1em, itemsep=-1em, parsep=2em}
%%%%%%%%%%%%%%%%%%%%%%%%%%%%%%%%%%%%%%%%%%%%%%%%%%%%%%
%                      Begin                         %
%%%%%%%%%%%%%%%%%%%%%%%%%%%%%%%%%%%%%%%%%%%%%%%%%%%%%%
\begin{enumerate}[Q\arabic*.]
  \item \textbf{Disproof by Counterexample}
    \begin{enumproof}
      \item Suppose $a, b \in S$, $a = "s", b = "u"$
      \item $len(a) = 1 = len(b) \land a \neq b$
      \item $\exists a, b \in S((aRb \land bRa) \land (a \neq b))$
      \item $\therefore R$ is not antisymmetric\hfill(Definition of antisymmetry)
      \item $\therefore R$ is not a partial order$\qed$\hfill(Definition of partial order)
    \end{enumproof}

  \item 
    \begin{enumerate}[(\alph*)]
      \item False. $7\mid 21 \implies 7 \curlyleq 21 \implies 21 \not\curlyleq* 7$ (by antisymmetry) $\qed$
      \item True. $2, 3$ are minimal elements. E.g. $\{2, 3, 5, 7, 21, 30, 84, 99\} \qed$
      \item True. $21 \curlyleq 84 \land 5$ is noncomparable to $21, 84$. E.g. $\{2, 3, 7, 21, 5, 30, 84, 99\} \qed$
      \item True. $30, 84, 99$ are maximal elements. E.g. $\{2, 3, 5, 7, 21, 99, 84, 30\} \qed$
    \end{enumerate}

  \item For $A = \{11,12,13,14,15,16\}, F_x = \{k\in\ZZ^+: k\mid x\}$:
    \begin{align*}
      F_{11} = \{1, 11\} &\implies |F_{11}| = 2\\
      F_{12} = \{1, 2, 3, 4, 6, 12\} &\implies |F_{12}| = 6\\
      F_{13} = \{1, 13\} &\implies |F_{13}| = 2\\
      F_{14} = \{1, 2, 7, 14\} &\implies |F_{14}| = 4\\
      F_{15} = \{1, 3, 5, 15\} &\implies |F_{15}| = 4\\
      F_{16} = \{1, 2, 4, 8, 16\} &\implies |F_{16}| = 5\\
    \end{align*}
    Minimal elements are $11, 13$, largest and maximal element is $12 \qed$

  \item All linearizations are:
    \begin{align*}
      11 \curlyleq^* 13 \curlyleq^* 14 \curlyleq^*15 \curlyleq^* 16 \curlyleq^* 12\tag*{(Given)}\\
      11 \curlyleq^* 13 \curlyleq^* 15 \curlyleq^*14 \curlyleq^* 16 \curlyleq^* 12\\
      13 \curlyleq^* 11 \curlyleq^* 14 \curlyleq^*15 \curlyleq^* 16 \curlyleq^* 12\\
      13 \curlyleq^* 11 \curlyleq^* 15 \curlyleq^*14 \curlyleq^* 16 \curlyleq^* 12 \tag*{\qed}\\
    \end{align*}
  \pagebreak
  \item \textbf{Direct Proof}
    \begin{enumproof}
    \item Prove $\subseteq$ is reflexive:
      \begin{enumproof}
      \item Let $S \in \powerset(A)$
      \item $S \subseteq S$\hfill(Definition of subsets)
      \item $\therefore \forall S \in \powerset(A) (S \subseteq S)$\hfill(Universal generalization)
      \item $\therefore \subseteq$ is reflexive\hfill(Definition of reflexivity)
      \end{enumproof}
    \item Prove $\subseteq$ is antisymmetric:
      \begin{enumproof}[label=2.\arabic*.]
      \item Let $S, T \in \powerset(A)$
      \item Suppose $S \subseteq T \land T \subseteq S$
      \item $S = T$\hfill(Definition of set equality)
      \item $\therefore \forall S, T \in \powerset(A) (S\subseteq T \land T \subseteq S \rightarrow S = T)$\hfill(Universal generalization)
      \item $\therefore \subseteq$ is antisymmetric\hfill(Definition of antisymmetry)
      \end{enumproof}
    \item Prove $\subseteq$ is transitive:
      \begin{enumproof}[label=3.\arabic*.]
      \item Let $S, T, U \in \powerset(A)$
      \item Suppose $S \subseteq T \land T \subseteq U$
      \item $\forall x(x \in S \rightarrow x \in T \land x \in T \rightarrow x \in U)$\hfill(Definition of subset)
      \item $\forall x(x \in S \rightarrow x \in U)$\hfill(Transitivity of implication)
      \item $S \subseteq U$\hfill(Definition of subset)
      \item $\therefore \forall S, T, U \in \powerset(A) (S \subseteq T \land T \subseteq U \rightarrow S \subseteq U)$\hfill(Universal generalization)
      \item $\therefore \subseteq$ is transitive\hfill(Definition of transitivity)
      \end{enumproof}
    \item $\subseteq$ is reflexive, antisymmetric and transitive\hfill(Conjunction)
    \item $\therefore \subseteq$ is a partial order$\qed$\hfill(Definition of partial order)
    \end{enumproof}

  \pagebreak
  \item 
    \begin{enumerate}[(\alph*)]
      \item \textbf{Direct Proof}
        \begin{enumproof}
          \item Prove $R$ is reflexive:
            \begin{enumproof}[label=1.\arabic*.]
            \item Let $(a, b) \in B \times B$
            \item $a \leq a \land b \leq b$
            \item $(a, b) R (a, b)$\hfill(Definition of R)
            \item $\forall (a, b) \in B \times B ((a, b) R (a, b))$\hfill(Universal generalization)
            \item $\therefore R$ is reflexive\hfill(Definition of reflexive)
            \end{enumproof}
          \item Prove $R$ is antisymmetric:
            \begin{enumproof}[label=2.\arabic*.]
            \item Let $(a, b), (c, d) \in B \times B$
            \item Suppose $(a, b) R (c, d) \land (c, d) R (a, b)$
            \item $(a \leq c \land c \leq a) \land (b \leq d \land d \leq b)$\hfill(Definition of R)
            \item $a = c \land b = d$\hfill(Definition of $\leq$)
            \item $\forall (a, b), (c, d) \in B \times B ((a, b) R (c, d) \land (c, d) R (a, b) \rightarrow a = c \land b = d)$\hfill(Universal generalization)
            \item $R$ is antisymmetric\hfill(Definition of antisymmetry)
            \end{enumproof}
          \item Prove $R$ is transitive:
            \begin{enumproof}[label=3.\arabic*.]
            \item Let $(a, b), (c, d), (e, f) \in B \times B$
            \item Suppose $(a, b) R (c, d) \land (c, d) R (e, f)$
            \item $a \leq c \leq e \land b \leq d \leq f$\hfill(Definition of R)
            \item $a\leq e \land b \leq f$\hfill(T18. Transitivity of $\leq$)
            \item $(a, b) R (e, f)$\hfill(Definition of R)
            \item $\forall (a, b), (c, d), (e, f) \in B \times B ((a, b) R (c, d) \land (c, d) R (e, f) \rightarrow (a, b) R (e, f))$\hfill(Universal generalization)
            \item $\therefore R$ is transitive\hfill(Definition of transitivity)
            \end{enumproof}
          \item $\therefore R$ is reflexive, antisymmetric and transitive\hfill(Conjunction)
          \item $\therefore R$ is a partial order$\qed$\hfill(Definition of partial order)
        \end{enumproof}
      \item \quad\\
          \incfig[0.2]{6b}
      \item Maximal and largest element is $(1, 1)$. Minimal and smallest element is $(0, 0) \qed$
      \item No. Counterexample: $(0, 1) \not R (1, 0) \land (1, 0) \not R (0, 1) \qed$
    \end{enumerate}
\pagebreak
  \item $S$ is the reflexive closure of $R$
    \begin{enumerate}[(\alph*)]
      \item \textbf{Direct Proof}
        \begin{enumproof}
        \item Prove $S$ is reflexive:
          \begin{enumproof}
          \item Let $x \in A$
          \item $x = x$
          \item $xSx$\hfill(Definition of $S$)
          \item $\therefore \forall x \in A(xSx)$\hfill(Universal generalization)
          \item $\therefore S$ is reflexive$\qed$\hfill(Definition of reflexivity)
          \end{enumproof}
        \end{enumproof}

      \item \textbf{Direct Proof}
        \begin{enumproof}
        \item Prove $R \subseteq S$:
          \begin{enumproof}[label=1.\arabic*.]
          \item Let $(x, y) \in R$
          \item $xRy$\hfill(Definition of $R$)
          \item $xSy$\hfill(Definition of $S$)
          \item $(x, y) \in S$
          \item $\therefore\forall (x, y) \in A \times A ((x, y) \in R \rightarrow (x, y) \in S)$\hfill(Universal generalization)
          \item $\therefore R \subseteq S \qed$\hfill(Definition of subset)
          \end{enumproof}
        \end{enumproof}

      \item \textbf{Direct Proof} 
        \begin{enumproof}
        \item Prove $S \subseteq S'$:
          \begin{enumproof}[label=1.\arabic*.]
          \item Let $(x, y) \in S$
          \item $x = y \lor xRy$\hfill(Definition of S)
          \item Case 1 ($x = y$): $xS'y$\hfill(Reflexivity of S')
          \item Case 2 ($xRy$): $xS'y$\hfill(Definition of S')
          \item In all cases, $xS'y$
          \item $(x, y) \in S'$
          \item $\therefore \forall (x, y) \in A \times A((x, y) \in S \rightarrow (x, y) \in S')$\hfill(Universal generalization)
          \item $\therefore S \subseteq S'\qed$\hfill(Definition of subset)
          \end{enumproof}
        \end{enumproof}
    \end{enumerate}

  \item 
    \begin{enumerate}[(\alph*)]
      \item $xRy \iff x < y \qed$
      \item $xRy \iff x \leq y \qed$
      \item DNE. $\qed$
      \item $xRy \iff xy \geq 0 \qed$
    \end{enumerate}

  \item 
    \begin{enumerate}[(\alph*)]
      \item Comparable: $\{1, 1\}, \{1, 2\}, \{1, 4\}, \{1, 5\}, \{1, 10\}, \{1, 15\}, \{1, 20\}$\\
        $\{2, 2\}, \{2, 4\}, \{2, 10\}, \{2, 20\}$\\ 
        $\{4, 4\}, \{4, 20\}$\\ 
        $\{5, 5\}, \{5, 10\}, \{5, 15\}, \{5, 20\}$\\
        $\{10, 10\}, \{10, 20\}$\\
        $\{15, 15\}$\\
        $\{20, 20\} \qed$
      \item Compatible: $\{1, 1\}, \{1, 2\}, \{1, 4\}, \{1, 5\}, \{1, 10\}, \{1, 15\}, \{1, 20\}$\\
        $\{2, 2\}, \{2, 4\}, \{2, 5\}, \{2, 10\}, \{2, 20\}$\\
        $\{4, 4\}, \{4, 5\}, \{4, 10\}, \{4, 20\}$\\
        $\{5, 5\}, \{5, 10\}, \{5, 15\}, \{5, 20\}$\\
        $\{10, 10\}, \{10, 20\}$\\
        $\{15, 15\}$\\
        $\{20, 20\} \qed$
    \end{enumerate}
  \pagebreak

\item 
  \begin{enumerate}[(\alph*)]
    \item Maximal chains: $\{\phi, \{a\}, \{a, b\}, \{a, b, c\}, \{a, b, c, d\}\}$ and $\{\phi, \{a\}, \{a, c\}, \{a, b, c\}, \{a, b, c, d\}\} \qed$
    \item \quad\\
      \incfig[0.2]{10b}\\
Maximal chains: $\{11, 385\}$ and $\{2, 6, 12\} \qed$
  \end{enumerate}

  \item
    \begin{enumerate}[(\alph*)]
      \item True.\\
        \textbf{Direct Proof}
        \begin{enumproof}
        \item Prove $\forall a, b \in A (a, b$ are comparable$\rightarrow a,b$ are compatible$)$
        \item Suppose $a, b$ are comparable
        \begin{enumproof}
          \item $a \curlyleq b \lor b \curlyleq a$\hfill(Definition of comparable)
          \item Case 1 ($a \curlyleq b$):
            \begin{enumproof}
            \item $b \curlyleq b$\hfill(Reflexivity of partial order)
            \item $\therefore \exists c = b \in A(a \curlyleq c \land b \curlyleq c)$\hfill(Universal generalization)
            \item $\therefore a, b$ are compatible\hfill(Definition of compatible)
            \end{enumproof}
          \item Case 2 ($b \curlyleq a$):
            \begin{enumproof}
            \item $a \curlyleq a$\hfill(Reflexivity of partial order)
            \item $\therefore \exists c = a \in A(b \curlyleq c \land a \curlyleq c)$\hfill(Universal generalization)
            \item $\therefore a, b$ are compatible\hfill(Definition of compatible)
            \end{enumproof}
          \item In all cases, $a, b$ are compatible$\qed$
          \end{enumproof}
        \end{enumproof}
    \item False. Counterexample from Q9: $\{4, 10\}$ is compatible ($4 \mid 20 \land 10 \mid 20$) but not comparable ($4 \nmid 10 \land 10 \nmid 4$)$\qed$
    \end{enumerate}
\end{enumerate}
%%%%%%%%%%%%%%%%%%%%%%%%%%%%%%%%%%%%%%%%%%%%%%%%%%%%%%
%                       End                          %
%%%%%%%%%%%%%%%%%%%%%%%%%%%%%%%%%%%%%%%%%%%%%%%%%%%%%%

\end{document}

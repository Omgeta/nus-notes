\documentclass[12pt, a4paper]{article}

\usepackage[a4paper, margin=1in]{geometry}

\usepackage[utf8]{inputenc}
\usepackage[mathscr]{euscript}
\let\euscr\mathscr \let\mathscr\relax
\usepackage[scr]{rsfso}
\usepackage{amssymb,amsmath,amsthm,amsfonts}
\usepackage[shortlabels]{enumitem}
\usepackage{multicol,multirow}
\usepackage{lipsum}
\usepackage{balance}
\usepackage{calc}
\usepackage[colorlinks=true,citecolor=blue,linkcolor=blue]{hyperref}
\usepackage{import}
\usepackage{xifthen}
\usepackage{pdfpages}
\usepackage{transparent}
\usepackage{listings}

\newcommand{\incfig}[2][1.0]{
    \def\svgwidth{#1\columnwidth}
    \import{./figures/}{#2.pdf_tex}
}

\newlist{enumproof}{enumerate}{4}
\setlist[enumproof,1]{label=\arabic*., parsep=1em}
\setlist[enumproof,2]{label=\arabic{enumproofi}.\arabic*., parsep=1em}
\setlist[enumproof,3]{label=\arabic{enumproofi}.\arabic{enumproofii}.\arabic*., parsep=1em}
\setlist[enumproof,4]{label=\arabic{enumproofi}.\arabic{enumproofii}.\arabic{enumproofiii}.\arabic*., parsep=1em}

\renewcommand{\qedsymbol}{\ensuremath{\blacksquare}}

\lstdefinestyle{mystyle}{
  language=C, % Set the language to C
  commentstyle=\color{codegray}, % Color for comments
  keywordstyle=\color{orange}, % Color for basic keywords
  stringstyle=\color{mauve}, % Color for strings
  basicstyle={\ttfamily\footnotesize}, % Basic font style
  breakatwhitespace=false,         
  breaklines=true,                 
  captionpos=b,                    
  keepspaces=true,                 
  numbers=none,                    
  tabsize=2,
  morekeywords=[2]{\#include, \#define, \#ifdef, \#ifndef, \#endif, \#pragma, \#else, \#elif}, % Preprocessor directives
  keywordstyle=[2]\color{codegreen}, % Style for preprocessor directives
  morekeywords=[3]{int, char, float, double, void, struct, union, enum, const, volatile, static, extern, register, inline, restrict, _Bool, _Complex, _Imaginary, size_t, ssize_t, FILE}, % C standard types and common identifiers
  keywordstyle=[3]\color{identblue}, % Style for types and common identifiers
  morekeywords=[4]{printf, scanf, fopen, fclose, malloc, free, calloc, realloc, perror, strtok, strncpy, strcpy, strcmp, strlen}, % Standard library functions
  keywordstyle=[4]\color{cyan}, % Style for library functions
}

% Things Lie
\newcommand{\kb}{\mathfrak b}
\newcommand{\kg}{\mathfrak g}
\newcommand{\kh}{\mathfrak h}
\newcommand{\kn}{\mathfrak n}
\newcommand{\ku}{\mathfrak u}
\newcommand{\kz}{\mathfrak z}
\DeclareMathOperator{\Ext}{Ext} % Ext functor
\DeclareMathOperator{\Tor}{Tor} % Tor functor
\newcommand{\gl}{\opname{\mathfrak{gl}}} % frak gl group
\renewcommand{\sl}{\opname{\mathfrak{sl}}} % frak sl group chktex 6

% More script letters etc.
\newcommand{\SA}{\mathcal A}
\newcommand{\SB}{\mathcal B}
\newcommand{\SC}{\mathcal C}
\newcommand{\SF}{\mathcal F}
\newcommand{\SG}{\mathcal G}
\newcommand{\SH}{\mathcal H}
\newcommand{\OO}{\mathcal O}

\newcommand{\SCA}{\mathscr A}
\newcommand{\SCB}{\mathscr B}
\newcommand{\SCC}{\mathscr C}
\newcommand{\SCD}{\mathscr D}
\newcommand{\SCE}{\mathscr E}
\newcommand{\SCF}{\mathscr F}
\newcommand{\SCG}{\mathscr G}
\newcommand{\SCH}{\mathscr H}

% Mathfrak primes
\newcommand{\km}{\mathfrak m}
\newcommand{\kp}{\mathfrak p}
\newcommand{\kq}{\mathfrak q}

% number sets
\newcommand{\RR}[1][]{\ensuremath{\ifstrempty{#1}{\mathbb{R}}{\mathbb{R}^{#1}}}}
\newcommand{\NN}[1][]{\ensuremath{\ifstrempty{#1}{\mathbb{N}}{\mathbb{N}^{#1}}}}
\newcommand{\ZZ}[1][]{\ensuremath{\ifstrempty{#1}{\mathbb{Z}}{\mathbb{Z}^{#1}}}}
\newcommand{\QQ}[1][]{\ensuremath{\ifstrempty{#1}{\mathbb{Q}}{\mathbb{Q}^{#1}}}}
\newcommand{\CC}[1][]{\ensuremath{\ifstrempty{#1}{\mathbb{C}}{\mathbb{C}^{#1}}}}
\newcommand{\PP}[1][]{\ensuremath{\ifstrempty{#1}{\mathbb{P}}{\mathbb{P}^{#1}}}}
\newcommand{\HH}[1][]{\ensuremath{\ifstrempty{#1}{\mathbb{H}}{\mathbb{H}^{#1}}}}
\newcommand{\FF}[1][]{\ensuremath{\ifstrempty{#1}{\mathbb{F}}{\mathbb{F}^{#1}}}}
% expected value
\newcommand{\EE}{\ensuremath{\mathbb{E}}}
\newcommand{\charin}{\text{ char }}
\DeclareMathOperator{\sign}{sign}
\DeclareMathOperator{\Aut}{Aut}
\DeclareMathOperator{\Inn}{Inn}
\DeclareMathOperator{\Syl}{Syl}
\DeclareMathOperator{\Gal}{Gal}
\DeclareMathOperator{\GL}{GL} % General linear group
\DeclareMathOperator{\SL}{SL} % Special linear group

%---------------------------------------
% BlackBoard Math Fonts :-
%---------------------------------------

%Captital Letters
\newcommand{\bbA}{\mathbb{A}}	\newcommand{\bbB}{\mathbb{B}}
\newcommand{\bbC}{\mathbb{C}}	\newcommand{\bbD}{\mathbb{D}}
\newcommand{\bbE}{\mathbb{E}}	\newcommand{\bbF}{\mathbb{F}}
\newcommand{\bbG}{\mathbb{G}}	\newcommand{\bbH}{\mathbb{H}}
\newcommand{\bbI}{\mathbb{I}}	\newcommand{\bbJ}{\mathbb{J}}
\newcommand{\bbK}{\mathbb{K}}	\newcommand{\bbL}{\mathbb{L}}
\newcommand{\bbM}{\mathbb{M}}	\newcommand{\bbN}{\mathbb{N}}
\newcommand{\bbO}{\mathbb{O}}	\newcommand{\bbP}{\mathbb{P}}
\newcommand{\bbQ}{\mathbb{Q}}	\newcommand{\bbR}{\mathbb{R}}
\newcommand{\bbS}{\mathbb{S}}	\newcommand{\bbT}{\mathbb{T}}
\newcommand{\bbU}{\mathbb{U}}	\newcommand{\bbV}{\mathbb{V}}
\newcommand{\bbW}{\mathbb{W}}	\newcommand{\bbX}{\mathbb{X}}
\newcommand{\bbY}{\mathbb{Y}}	\newcommand{\bbZ}{\mathbb{Z}}

%---------------------------------------
% MathCal Fonts :-
%---------------------------------------

%Captital Letters
\newcommand{\mcA}{\mathcal{A}}	\newcommand{\mcB}{\mathcal{B}}
\newcommand{\mcC}{\mathcal{C}}	\newcommand{\mcD}{\mathcal{D}}
\newcommand{\mcE}{\mathcal{E}}	\newcommand{\mcF}{\mathcal{F}}
\newcommand{\mcG}{\mathcal{G}}	\newcommand{\mcH}{\mathcal{H}}
\newcommand{\mcI}{\mathcal{I}}	\newcommand{\mcJ}{\mathcal{J}}
\newcommand{\mcK}{\mathcal{K}}	\newcommand{\mcL}{\mathcal{L}}
\newcommand{\mcM}{\mathcal{M}}	\newcommand{\mcN}{\mathcal{N}}
\newcommand{\mcO}{\mathcal{O}}	\newcommand{\mcP}{\mathcal{P}}
\newcommand{\mcQ}{\mathcal{Q}}	\newcommand{\mcR}{\mathcal{R}}
\newcommand{\mcS}{\mathcal{S}}	\newcommand{\mcT}{\mathcal{T}}
\newcommand{\mcU}{\mathcal{U}}	\newcommand{\mcV}{\mathcal{V}}
\newcommand{\mcW}{\mathcal{W}}	\newcommand{\mcX}{\mathcal{X}}
\newcommand{\mcY}{\mathcal{Y}}	\newcommand{\mcZ}{\mathcal{Z}}

%---------------------------------------
% Bold Math Fonts :-
%---------------------------------------

%Captital Letters
\newcommand{\bmA}{\boldsymbol{A}}	\newcommand{\bmB}{\boldsymbol{B}}
\newcommand{\bmC}{\boldsymbol{C}}	\newcommand{\bmD}{\boldsymbol{D}}
\newcommand{\bmE}{\boldsymbol{E}}	\newcommand{\bmF}{\boldsymbol{F}}
\newcommand{\bmG}{\boldsymbol{G}}	\newcommand{\bmH}{\boldsymbol{H}}
\newcommand{\bmI}{\boldsymbol{I}}	\newcommand{\bmJ}{\boldsymbol{J}}
\newcommand{\bmK}{\boldsymbol{K}}	\newcommand{\bmL}{\boldsymbol{L}}
\newcommand{\bmM}{\boldsymbol{M}}	\newcommand{\bmN}{\boldsymbol{N}}
\newcommand{\bmO}{\boldsymbol{O}}	\newcommand{\bmP}{\boldsymbol{P}}
\newcommand{\bmQ}{\boldsymbol{Q}}	\newcommand{\bmR}{\boldsymbol{R}}
\newcommand{\bmS}{\boldsymbol{S}}	\newcommand{\bmT}{\boldsymbol{T}}
\newcommand{\bmU}{\boldsymbol{U}}	\newcommand{\bmV}{\boldsymbol{V}}
\newcommand{\bmW}{\boldsymbol{W}}	\newcommand{\bmX}{\boldsymbol{X}}
\newcommand{\bmY}{\boldsymbol{Y}}	\newcommand{\bmZ}{\boldsymbol{Z}}
%Small Letters
\newcommand{\bma}{\boldsymbol{a}}	\newcommand{\bmb}{\boldsymbol{b}}
\newcommand{\bmc}{\boldsymbol{c}}	\newcommand{\bmd}{\boldsymbol{d}}
\newcommand{\bme}{\boldsymbol{e}}	\newcommand{\bmf}{\boldsymbol{f}}
\newcommand{\bmg}{\boldsymbol{g}}	\newcommand{\bmh}{\boldsymbol{h}}
\newcommand{\bmi}{\boldsymbol{i}}	\newcommand{\bmj}{\boldsymbol{j}}
\newcommand{\bmk}{\boldsymbol{k}}	\newcommand{\bml}{\boldsymbol{l}}
\newcommand{\bmm}{\boldsymbol{m}}	\newcommand{\bmn}{\boldsymbol{n}}
\newcommand{\bmo}{\boldsymbol{o}}	\newcommand{\bmp}{\boldsymbol{p}}
\newcommand{\bmq}{\boldsymbol{q}}	\newcommand{\bmr}{\boldsymbol{r}}
\newcommand{\bms}{\boldsymbol{s}}	\newcommand{\bmt}{\boldsymbol{t}}
\newcommand{\bmu}{\boldsymbol{u}}	\newcommand{\bmv}{\boldsymbol{v}}
\newcommand{\bmw}{\boldsymbol{w}}	\newcommand{\bmx}{\boldsymbol{x}}
\newcommand{\bmy}{\boldsymbol{y}}	\newcommand{\bmz}{\boldsymbol{z}}

%---------------------------------------
% Scr Math Fonts :-
%---------------------------------------

\newcommand{\sA}{{\mathscr{A}}}   \newcommand{\sB}{{\mathscr{B}}}
\newcommand{\sC}{{\mathscr{C}}}   \newcommand{\sD}{{\mathscr{D}}}
\newcommand{\sE}{{\mathscr{E}}}   \newcommand{\sF}{{\mathscr{F}}}
\newcommand{\sG}{{\mathscr{G}}}   \newcommand{\sH}{{\mathscr{H}}}
\newcommand{\sI}{{\mathscr{I}}}   \newcommand{\sJ}{{\mathscr{J}}}
\newcommand{\sK}{{\mathscr{K}}}   \newcommand{\sL}{{\mathscr{L}}}
\newcommand{\sM}{{\mathscr{M}}}   \newcommand{\sN}{{\mathscr{N}}}
\newcommand{\sO}{{\mathscr{O}}}   \newcommand{\sP}{{\mathscr{P}}}
\newcommand{\sQ}{{\mathscr{Q}}}   \newcommand{\sR}{{\mathscr{R}}}
\newcommand{\sS}{{\mathscr{S}}}   \newcommand{\sT}{{\mathscr{T}}}
\newcommand{\sU}{{\mathscr{U}}}   \newcommand{\sV}{{\mathscr{V}}}
\newcommand{\sW}{{\mathscr{W}}}   \newcommand{\sX}{{\mathscr{X}}}
\newcommand{\sY}{{\mathscr{Y}}}   \newcommand{\sZ}{{\mathscr{Z}}}


%---------------------------------------
% Math Fraktur Font
%---------------------------------------

%Captital Letters
\newcommand{\mfA}{\mathfrak{A}}	\newcommand{\mfB}{\mathfrak{B}}
\newcommand{\mfC}{\mathfrak{C}}	\newcommand{\mfD}{\mathfrak{D}}
\newcommand{\mfE}{\mathfrak{E}}	\newcommand{\mfF}{\mathfrak{F}}
\newcommand{\mfG}{\mathfrak{G}}	\newcommand{\mfH}{\mathfrak{H}}
\newcommand{\mfI}{\mathfrak{I}}	\newcommand{\mfJ}{\mathfrak{J}}
\newcommand{\mfK}{\mathfrak{K}}	\newcommand{\mfL}{\mathfrak{L}}
\newcommand{\mfM}{\mathfrak{M}}	\newcommand{\mfN}{\mathfrak{N}}
\newcommand{\mfO}{\mathfrak{O}}	\newcommand{\mfP}{\mathfrak{P}}
\newcommand{\mfQ}{\mathfrak{Q}}	\newcommand{\mfR}{\mathfrak{R}}
\newcommand{\mfS}{\mathfrak{S}}	\newcommand{\mfT}{\mathfrak{T}}
\newcommand{\mfU}{\mathfrak{U}}	\newcommand{\mfV}{\mathfrak{V}}
\newcommand{\mfW}{\mathfrak{W}}	\newcommand{\mfX}{\mathfrak{X}}
\newcommand{\mfY}{\mathfrak{Y}}	\newcommand{\mfZ}{\mathfrak{Z}}
%Small Letters
\newcommand{\mfa}{\mathfrak{a}}	\newcommand{\mfb}{\mathfrak{b}}
\newcommand{\mfc}{\mathfrak{c}}	\newcommand{\mfd}{\mathfrak{d}}
\newcommand{\mfe}{\mathfrak{e}}	\newcommand{\mff}{\mathfrak{f}}
\newcommand{\mfg}{\mathfrak{g}}	\newcommand{\mfh}{\mathfrak{h}}
\newcommand{\mfi}{\mathfrak{i}}	\newcommand{\mfj}{\mathfrak{j}}
\newcommand{\mfk}{\mathfrak{k}}	\newcommand{\mfl}{\mathfrak{l}}
\newcommand{\mfm}{\mathfrak{m}}	\newcommand{\mfn}{\mathfrak{n}}
\newcommand{\mfo}{\mathfrak{o}}	\newcommand{\mfp}{\mathfrak{p}}
\newcommand{\mfq}{\mathfrak{q}}	\newcommand{\mfr}{\mathfrak{r}}
\newcommand{\mfs}{\mathfrak{s}}	\newcommand{\mft}{\mathfrak{t}}
\newcommand{\mfu}{\mathfrak{u}}	\newcommand{\mfv}{\mathfrak{v}}
\newcommand{\mfw}{\mathfrak{w}}	\newcommand{\mfx}{\mathfrak{x}}
\newcommand{\mfy}{\mathfrak{y}}	\newcommand{\mfz}{\mathfrak{z}}


\newcommand{\mytitle}{CS1231S Tutorial 7}
\newcommand{\myauthor}{github/omgeta}
\newcommand{\mydate}{AY 24/25 Sem 1}

\begin{document}
\raggedright
\footnotesize
\begin{center}
{\normalsize{\textbf{\mytitle}}} \\
{\footnotesize{\mydate\hspace{2pt}\textemdash\hspace{2pt}\myauthor}}
\end{center}
\setlist{topsep=-1em, itemsep=-1em, parsep=2em}
%%%%%%%%%%%%%%%%%%%%%%%%%%%%%%%%%%%%%%%%%%%%%%%%%%%%%%
%                      Begin                         %
%%%%%%%%%%%%%%%%%%%%%%%%%%%%%%%%%%%%%%%%%%%%%%%%%%%%%%
\begin{enumerate}[Q\arabic*.]
  \item 
    \begin{enumerate}[(\alph*)]
      \item Predicate $P(n)$ cannot be used as a binary operand $\qed$
      \item We cannot assume equality to $P(k+1)$, we must show $P(k) \rightarrow P(k+1)$ $\qed$ 
      \item If we assume $P(k)$ is true for all $k\in\ZZ^+$, then there is nothing to prove. $\qed$
    \end{enumerate}
  
  \item \textbf{Proof by 1MI}
    \begin{enumproof}
    \item Let $P(n) \equiv (1^2+2^2+\cdots+n^2=\frac{1}{6}n(n+1)(2x+1)), \forall n\in\ZZ^+$
    \item Basis step: 
      \begin{enumproof}
      \item $1^2 = \frac{1}{6}(2)(3)$, therefore $P(1)$ is true
      \end{enumproof}
    \item Assume $P(k)$ is true for some $k\geq1 \implies 1^2+\cdots+k^2 = \frac{k(k+1)(2k+1)}{6}$
    \item Inductive step: 
      \begin{enumproof}
      \item $1^2+2^2+\cdots+k^2+(k+1)^2 = \frac{k(k+1)(2k+1)}{6}+(k+1)^2$ \\$= \frac{k(k+1)(2k+1)+6(k+1)^2}{6} = \frac{(k+1)(2k^2 + 7k + 6)}{6} = \frac{(k+1)(k+2)(2k+3)}{6}$
      \item Therefore, $P(k+1)$ is true
      \end{enumproof}
    \item Therefore, $P(n)$ is true for all $n \in \ZZ^+ \qed$
    \end{enumproof}

  \item \textbf{Proof by 1MI}
    \begin{enumproof}
    \item Let $P(n) \equiv (1 + nx \leq (1+x)^n), \forall n\in\ZZ^+, x \in \ZZ_{\geq -1}$
    \item Basis step: 
      \begin{enumproof}
      \item $1 + x \leq (1+x)^{1}$, therefore $P(1)$ is true
      \end{enumproof}
    \item Assume $P(k)$ is true for some $k\geq1 \implies 1+kx \leq (1+x)^k$
    \item Inductive step: 
      \begin{enumproof}
      \item $1+(k+1)x = 1+ kx + x \leq (1+x)^k + x \leq (1+x)^k +x(1+x)^k = (1+x)^{k+1}$
      \item Therefore, $P(k+1)$ is true
      \end{enumproof}
    \item Therefore, $P(n)$ is true for all $n \in \ZZ^+ \qed$
    \end{enumproof}

  \item \textbf{Proof by 1MI}
    \begin{enumproof}
    \item Let $P(n) \equiv (2^{n+2} \mid (a^{2^n} - 1)), \forall n\in\ZZ^+, a$ is any odd integer
    \item Basis step: 
      \begin{enumproof}
      \item $a^{2^1} - 1 = (a+1)(a-2)$\hfill(Basic algebra)
      \item $= (2m+2)(2m) = 4(m+1)(m)$\hfill(Definition of odd numbers)
      \item $= 4(2k)$\hfill(Prod. of consecutive integers is even)
      \item $= 8k = k\cdot 2^3$
      \item $\therefore 2^3\mid (a^{2^1} - 1)$\hfill(Definition of divides)
      \item Therefore, $P(1)$ is true
      \end{enumproof}
    \item Assume $P(k)$ is true for some $k\in\ZZ^+$:
      \begin{enumproof}
        \item $2^{k+2} \mid a^{2^n}-1$\hfill(Definition of $P(n)$)
        \item $\exists m\in\ZZ,$ $m\cdot2^{k+2} = a^{2^k}-1$\hfill(Definition of divides)
      \end{enumproof}
    \item Inductive step: 
      \begin{enumproof}
      \item $a^{2^{k+1}} - 1 = (a^{2^k})^2 - 1= (a^{2^k}-1)(a^{2^k}+1)$\hfill(Basic algebra)
      \item $= m \cdot 2^{k+2}\cdot (a^{2^k}+1)$\hfill(By inductive hypothesis)
      \item $= m \cdot 2^{k+2} \cdot (m\cdot 2^{k+2}+2)$\hfill(By indutive hypothesis)
      \item $m \cdot 2^{k+3}(m \cdot 2^{k+1} + 1)$\hfill(Basic algebra)
      \item Therefore, $P(k+1)$ is true
      \end{enumproof}
    \item Therefore, $P(n)$ is true for all $n \in \ZZ^+ \qed$
    \end{enumproof}
  \pagebreak
  \item \textbf{Proof by 2MI}
    \begin{enumproof}
    \item Let $P(n) \equiv (n = 3x+5y), \forall n \geq \ZZ_{\geq8},\exists x,y \in \NN$
    \item Basis step: 
      \begin{enumproof}
      \item $8 = 3(1) + 5(1)$, therefore $P(8)$ is true
      \item $9 = 3(3) + 5(0)$, therefore $P(9)$ is true
      \item $10 = 3(0) + 5(2)$, therefore $P(10)$ is true
      \end{enumproof}
    \item Assume $P(i)$ is true for $8 \leq i \leq k$ for some $k$
    \item Inductive step: 
      \begin{enumproof}
      \item $P(k-2)$ is true $\implies k-2 = 3a+5b$, for some $a, b \in \ZZ$
      \item $k+1 = (k-2) + 3 = 3a + 5b + 3 = 3(a+1) + b$
      \item Therefore, $P(k+1)$ is true
      \end{enumproof}
    \item Therefore, $P(n)$ is true for all $n \in \ZZ_{\geq 8} \qed$
    \end{enumproof}

  \item \textbf{Proof by 2MI}
    \begin{enumproof}
    \item Let $P(n) \equiv (i_1< i_2<\cdots<i_l \land n = 2^{i_1} + 2^{i_2} + \cdots + 2^{i_l}), \forall n \in \ZZ^+ \exists l \in \ZZ^+ \exists i_1,i_2,\cdots,i_l \in \NN$
    \item Basis step: $1 = 2^0 \implies P(1)$ is true
    \item Assume $P(i)$ is true for $1 \leq i \leq k$ for some $k$
    \item Inductive step: 
      \begin{enumproof}
      \item Case 1 ($k+1$ is odd):
        \begin{enumproof}
        \item $k+1 = 2m+1$, $m = \frac{k+1}{2} \in \ZZ$\hfill(Definition of odd numbers) 
        \item $m = 2^{i_1} + \cdots + 2^{i_l}$\hfill(By inductive hypothesis)
        \item $k = 2(2^{i_1} + \cdots + 2^{i_l}) = 2^{i_1+1} + \cdots + 2^{i_l + 1}$ where $i_1+1, i_2+1, \cdots i_l+1 \geq 1$
        \item $k+1 = 2^{i_1+1} + \cdots + 2^{i_l + 1} + 2^0$
        \item Therefore, $P(k+1)$ is true
        \end{enumproof}
      \item Case 2 ($k+1$ is even):
        \begin{enumproof}
        \item $k+1 = 2m$, $m = \frac{k+1}{2} \in \ZZ$\hfill(Definition of even numbers) 
        \item $m = 2^{i_1} + \cdots + 2^{i_l}$\hfill(By inductive hypothesis)
        \item $k+1 = 2(2^{i_1} + \cdots + 2^{i_l}) = 2^{i_1+1} + \cdots + 2^{i_l + 1}$
        \item Therefore, $P(k+1)$ is true
        \end{enumproof}
      \item In all cases, $P(k+1)$ is true
      \end{enumproof}
    \item Therefore, $P(n)$ is true for all $n \in \ZZ^+ \qed$
    \end{enumproof}
    
  \item \textbf{Proof by 2MI}
    \begin{enumproof}
    \item Let $P(n) \equiv (a_n < 3^n), \forall n \in \NN$
    \item Basis step: $a_0 = 0 < 1 = 3^0$, therefore $P(0)$ is true
    \item Assume $P(i)$ is true for $0 \leq i \leq k$ for some $k$
    \item Inductive step: 
      \begin{enumproof}
      \item $a_{k+1} = a_{k} + a_{k-1} + a_{k-2} < 3^k + 3^{k-1} + 3^{k-2} < 3^k + 3^k + 3^k = 3^{k+1}$
      \item Therefore, $P(k+1)$ is true
      \end{enumproof}
    \item Therefore, $P(n)$ is true for all $n \in \NN \qed$
    \end{enumproof}
    \pagebreak

  \item 
    \begin{enumerate}[(\alph*)]
      \item $F(0+b) = F(b) = (F(1) \times F(b) + F(0) \times F(b-1))$, therefore $P(0, b)$ is true $\qed$\\
        $F(1+b) = F(b) + F(b-1) = (F(2) \times F(b) + F(1) \times F(b-1))$, therefore $P(1, b)$ is true $\qed$
      \item 
        \begin{enumproof}
        \item Assume $P(k-1, b) \land P(k, b)$ for some $k \in \ZZ^+$:
          \begin{align*}
            F(k-1+b) = F(k)\times F(b) + F(k-1) \times F(b-1)\\
            F(k+b) = F(k+1)\times F(b) + F(k) \times F(b-1)
          \end{align*}
          \vspace{-2em}
        \item Inductive step:
          \begin{enumproof}
          \item $F(k+1+b) = F(k+b) + F(k+b-1)$\hfill(Definition of Fibonacci sequence)
          \item $= (F(k+1) \times F(b) + F(k) \times F(b-1)) + (F(k) \times F(b) + F(k-1) \times F(b-1))$\\
          \item $= F(b) \times (F(k+1) + F(k)) + F(b-1) \times (F(k) + F(k-1))$\hfill(Distributive law)
          \item $= F(b) \times F(k+2) + F(b-1) \times F(k+1)$\hfill(Definition of Fibonacci sequence)
          \item Therefore, $P(k+1, b)$ is true
          \end{enumproof}
        \item Therefore, $P(n+1,b)$ is true for all $n \in \ZZ^+ \qed$
        \end{enumproof}
    \end{enumerate}

  \item \textbf{Proof by 1MI}
    \begin{enumproof}
    \item Basis step: $1 = 2^05^05^0$, therefore $P(1)$ is true
    \item Assume $P(m)$ is true for some $m$, i.e. $\exists!i\exists!j\exists!k((i,j,k\geq0) \land m=2^i 3^j 5^k)$
    \item Inductive step:
      \begin{enumproof}
      \item $2m = 2 \cdot 2^i 3^j 5^k = 2^{i+1} 3^k 5^k \implies P(2m)$\hfill(By inductive hypothesis)
      \item $3m = 3 \cdot 2^i 3^j 5^k = 2^i 3^{k+1} 5^k \implies P(3m)$\hfill(By inductive hypothesis)
      \item $5m = 5 \cdot 2^i 3^j 5^k = 2^i 3^k 5^{k+1} \implies P(5m)$\hfill(By inductive hypothesis)
      \item Therefore $P(m) \rightarrow P(2m) \land P(3m) \land P(5m)$\hfill(Conjunction)
      \end{enumproof}
    \item Therefore, $\forall n \in H$, $P(n) \qed$\hfill(Given 1MI rule)
    \end{enumproof}

  \item $0, 15 \not\in S$ and $6,16,36 \in S \qed$
  
  \item 
    \begin{enumerate}[(\alph*)]
      \item Yes; $C = (A \setminus B) \cup (B \setminus A) \in S \qed$
      \item No $\qed$
    \end{enumerate}
\end{enumerate}
%%%%%%%%%%%%%%%%%%%%%%%%%%%%%%%%%%%%%%%%%%%%%%%%%%%%%%
%                       End                          %
%%%%%%%%%%%%%%%%%%%%%%%%%%%%%%%%%%%%%%%%%%%%%%%%%%%%%%

\end{document}

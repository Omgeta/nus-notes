\documentclass[12pt, a4paper]{article}

\usepackage[utf8]{inputenc}
\usepackage[mathscr]{euscript}
\let\euscr\mathscr \let\mathscr\relax
\usepackage[scr]{rsfso}
\usepackage{amssymb,amsmath,amsthm,amsfonts}
\usepackage[shortlabels]{enumitem}
\usepackage{multicol,multirow}
\usepackage{lipsum}
\usepackage{balance}
\usepackage{calc}
\usepackage[colorlinks=true,citecolor=blue,linkcolor=blue]{hyperref}
\usepackage{import}
\usepackage{xifthen}
\usepackage{pdfpages}
\usepackage{transparent}
\usepackage{tabularx}

\newcommand{\incfig}[2][1.0]{
    \def\svgwidth{#1\columnwidth}
    \import{./figures/}{#2.pdf_tex}
}
\newcommand{\incimg}[2][1.0]{
  \includegraphics[width=#1\columnwidth]{./figures/#2}
}


\usepackage{ifthen}
\usepackage[landscape]{geometry}
\usepackage[shortlabels]{enumitem}

\ifthenelse{\lengthtest { \paperwidth = 11in}}
    { \geometry{top=.5in,left=.5in,right=.5in,bottom=.5in} }
	{\ifthenelse{ \lengthtest{ \paperwidth = 297mm}}
		{\geometry{top=1cm,left=1cm,right=1cm,bottom=1cm} }
		{\geometry{top=1cm,left=1cm,right=1cm,bottom=1cm} }
	}

\pagestyle{empty}
\makeatletter
\renewcommand\thesection{\arabic{section}.}
\renewcommand{\section}{\@startsection{section}{1}{0mm}%
                                {-1ex plus -.5ex minus -.2ex}%
                                {0.05ex}%x
                                {\normalfont\normalsize\bfseries}}
\renewcommand{\subsection}{\@startsection{subsection}{2}{0mm}%
                                {-1ex plus -.5ex minus -.2ex}%
                                {0.05ex}%
                                {\normalfont\small\bfseries}}
\renewcommand{\subsubsection}{\@startsection{subsubsection}{3}{0mm}%
                                {-1ex plus -.5ex minus -.2ex}%
                                {0.05ex}%
                                {\normalfont\footnotesize\bfseries}}
\newcommand{\colbreak}{\vfill\null\columnbreak}
\makeatother
\setcounter{secnumdepth}{1}
\setlength{\parindent}{0pt}
\setlength{\parskip}{0.7em}

\setlist[itemize]{itemsep=0.6ex, topsep=-2pt, partopsep=0pt, parsep=0pt}
\setlist[enumerate]{itemsep=0.6ex, topsep=-2pt, partopsep=0pt, parsep=0pt}

\input{letterfonts}

\newcommand{\mytitle}{CS2030S Programming Methodology II}
\newcommand{\myauthor}{github/omgeta}
\newcommand{\mydate}{AY 24/25 Sem 1}

\begin{document}
\raggedright
\footnotesize
\begin{multicols*}{3}
\setlength{\premulticols}{1pt}
\setlength{\postmulticols}{1pt}
\setlength{\multicolsep}{1pt}
\setlength{\columnsep}{2pt}

{\normalsize{\textbf{\mytitle}}} \\
{\footnotesize{\mydate\hspace{2pt}\textemdash\hspace{2pt}\myauthor}}

%%%%%%%%%%%%%%%%%%%%%%%%%%%%%%%%%%%%%%%%%%%%%%%%%%%%%%
%                      Begin                         %
%%%%%%%%%%%%%%%%%%%%%%%%%%%%%%%%%%%%%%%%%%%%%%%%%%%%%%
\section{Java Language}
Java is a high-level programming language which compiles into JVM bytecode.
\begin{center}
  \incimg[0.65]{tombstone}
\end{center}

Types in Java have the two general properties:
\begin{enumerate}[\roman*.]
  \item CTT of variables cannot be changed\hfill(Static)
  \item Strict compiler type safety checks\hfill(Strong)
\end{enumerate}

\subsection{Stack and Heap Diagram}
\begin{center}
  \incimg[0.7]{stack1}
  \incimg[0.7]{stack2}
\end{center}


\section{Types}
Types determine the meaning and operations defined on variables. 

For types \lstinline|S|, \lstinline|T| where \lstinline|S| is a subtype of \lstinline|T|, \lstinline|S <: T|:
\begin{enumerate}[\roman*.]
  \item \lstinline|S <: S|\hfill(Reflexive)
  \item \lstinline|S <: T| $\land$ \lstinline|T <: S| $\rightarrow$ \lstinline|S = T|\hfill(Anti-symmetric)
  \item \lstinline|S <: T| $\land$ \lstinline|T <: U| $\rightarrow$ \lstinline|S <: U|\hfill(Transitive)
\end{enumerate}

For complex type \lstinline|C(X)| with component types \lstinline|S|, \lstinline|T|, there exists a mutually exclusive subtype relationship:
\begin{enumerate}[\roman*.]
  \item \lstinline|S <: T| $\rightarrow$ \lstinline|C(S) <: C(T)|\hfill(Covariant)
  \item \lstinline|S <: T| $\rightarrow$ \lstinline|C(T) <: C(S)|\hfill(Contravariant)
  \item Neither\hfill(Invariant)
\end{enumerate}

\subsection{Type Conversion}
For types \lstinline|S|, \lstinline|T| where \lstinline|S <: T| and the CTT are related, there exist possible conversions:
\begin{enumerate}[\roman*.]
  \item \lstinline|S|$\rightarrow$\lstinline|T|\quad(E.g. \lstinline|T t = s|)\hfill(Widening - implicit)
  \item \lstinline|T|$\rightarrow$\lstinline|S|\quad(E.g. \lstinline|S s = (S) t|)\hfill(Narrowing - casting)
\end{enumerate}

Narrowing throws \lstinline|ClassCastException| if RTT $\not<:$ CTT.

\subsection{Primitives}
Primitives are predefined types for numbers and booleans.\\They store copies of their values.

\raisebox{0pt}{
\incimg{primitivesubtype}} 
\subsection{Reference Types}
Reference types are types for holding references to objects.\\ They can refer to the same underlying object.\\Classes are used to define new reference types.\\All reference types are a subtype of \lstinline|Object|.

\colbreak
\section{Object Oriented Programming}
Encapsulation is the bundling of data with methods that act on it within a class.

Abstraction is the separation of concerns by hiding implementation details from clients.

Inheritance allows a class to inherit fields and methods from a superclass, enabling code reuse and extensibility.

Polymorphism allows clients to use a single interface to represent multiple underlying implementations, enhancing flexibility.

\subsection{Abstract Classes}
Abstract classes cannot be instantiated, but can have fields and methods (abstract or concrete) to be inherited.
\begin{lstlisting}
abstract class Shape {
  private Point c;
  public abstract int distTo(Point p);
}
\end{lstlisting}

\subsection{Interfaces}
Interfaces cannot be instantiated, but can have abstract methods to be implemented.
\begin{lstlisting}
interface GetAreable {
  int getArea();
}
\end{lstlisting}

\subsection{Concrete Classes}
Concrete classes can be instantiated if and only if there are no abstract methods after overriding.
\begin{lstlisting}
class Circle extends Shape implements GetAreable {
  private int r;
  @Override
  public int distTo(Point p) { ... };
  @Override
  public int getArea(){ ... };
}
\end{lstlisting}

\subsection{Modifiers}
Access modifiers:
\begin{enumerate}[\roman*.]
  \item \lstinline|public|: accessible from anywhere
  \item \lstinline|protected|: accessible within the same package and subclasses
  \item \lstinline|default|: accessible within the same package
  \item \lstinline|private|: accessible within the defining class
\end{enumerate}

Non-Access Modifiers:
\begin{enumerate}[\roman*.]
  \item \lstinline|static|: belongs to the class, shared by instances 
  \item \lstinline|final|: cannot be overriden or modified
\end{enumerate}

\subsection{Method Overloading}
Method overloading allows classes to have methods with the same name but a different method signature, i.e., a different set of parameters.

\subsection{Method Overriding}
Method overriding allows subclasses to override inherited methods with the same method descriptor using the \lstinline|@Override| decorator.

\subsection{Dynamic Binding}
For objects \lstinline|obj|, \lstinline|arg| and method call \lstinline|obj.foo(arg)|, the process of dynamic binding is given by:
\begin{enumerate}[\arabic*.]
  \item Find all matching method descriptors in supertypes of CTT(\lstinline|obj|) that accept parameters of CTT(\lstinline|arg|).
  \item Determine the most specific method descriptor, else throw a compilation error.
  \item Starting from RTT(\lstinline|obj|) and going up to superclasses, search for the method descriptor and run the method.
\end{enumerate}

Static methods do not support dynamic binding, and the method to invoke is resolved entirely using CTT.

\subsubsection{Method Specificity}
For methods \lstinline|M|,\lstinline|N|, \lstinline|M| is more specific than \lstinline|N| if its arguments can be passed to \lstinline|N| without compilation error.

\section{Generic Types}
Generic types are complex types which operate on types specified at runtime. All generic types are invariant.

For type parameters \lstinline|S|, \lstinline|T| we can restrict the type parameter to:
\begin{enumerate}[\roman*.]
  \item \lstinline|<S extends T>|\\(Upper bounded)
\end{enumerate}

\subsection{Wildcards}
Wildcards allow for generic types to have variance in subtyping relations.

For type parameters \lstinline|S|,\lstinline|T| where \lstinline|S <: T|, we can restrict the type parameter to:
\begin{enumerate}[\roman*.]
  \item \lstinline|<? extends S> <: <? extends T>|\\(Upper bounded - covariant)
  \item \lstinline|<? super T> <: <? super S>|\\(Lower bounded - contravariant)
  \item \lstinline|<?>|\\(Unbounded - invariant)
\end{enumerate}

\subsubsection{Producer Extends, Consumer Super (PECS)}
PECS is a guideline for using wildcards:
\begin{enumerate}[\roman*.]
  \item Use \lstinline|<? extends T>| when methods read values of \lstinline|T|\hfill(Producer Extends)
  \item Use \lstinline|<? super T>| when methods set values of \lstinline|T|\hfill(Consumer Super)
\end{enumerate}

\subsection{Type Erasure}
During compilation, type parameters are arguments are replaced by their bounds (or \lstinline|Object| if unbounded), limiting runtime access to generic type information.

\colbreak
\section{Exceptions}
Checked exceptions are exceptions which the programmer has no control over, and must be declared or caught.\\E.g.: \lstinline|IOException|, \lstinline|FileNotFoundException|.

Unchecked exceptions are exceptions caused by programmer errors. All unchecked exceptions are subclasses of \lstinline|RuntimeException|.\\E.g.: \lstinline|IllegalArgumentException|, \lstinline|NullPointerException|, \lstinline|ClassCastException|.

\subsection{Unchecked Warnings}
Unchecked warnings are thrown by potential type safety runtime errors due to type erasures.

If the code is provably type-safe, we can suppress warnings with \lstinline|@SuppressWarning("unchecked")| annotated with a descriptive comment.

\section{Design Principles}
\subsection{Liskov Substitution Principle (LSP)}
LSP is the principle that a supertype should be able to be replaced by a subtype without breaking the supertype's properties.
\subsection{Information Hiding}
Information hiding is the principle that access to an object's internal implementation must be restricted, i.e., there must be an abstraction barrier separating the concerns of the client and implementer.

\subsection{Tell Don't Ask}
Tell Don't Ask is the principle that the client should tell an object what to do instead of asking to get the value of a field to perform the computation on the object's behalf.
%%%%%%%%%%%%%%%%%%%%%%%%%%%%%%%%%%%%%%%%%%%%%%%%%%%%%%
%                       End                          %
%%%%%%%%%%%%%%%%%%%%%%%%%%%%%%%%%%%%%%%%%%%%%%%%%%%%%%

\end{multicols*}
\end{document}

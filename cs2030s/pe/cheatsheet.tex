\documentclass[12pt, a4paper]{article}

\usepackage[utf8]{inputenc}
\usepackage[mathscr]{euscript}
\let\euscr\mathscr \let\mathscr\relax
\usepackage[scr]{rsfso}
\usepackage{amssymb,amsmath,amsthm,amsfonts}
\usepackage[shortlabels]{enumitem}
\usepackage{multicol,multirow}
\usepackage{lipsum}
\usepackage{balance}
\usepackage{calc}
\usepackage[colorlinks=true,citecolor=blue,linkcolor=blue]{hyperref}
\usepackage{import}
\usepackage{xifthen}
\usepackage{pdfpages}
\usepackage{transparent}
\usepackage{tabularx}

\newcommand{\incfig}[2][1.0]{
    \def\svgwidth{#1\columnwidth}
    \import{./figures/}{#2.pdf_tex}
}
\newcommand{\incimg}[2][1.0]{
  \includegraphics[width=#1\columnwidth]{./figures/#2}
}


\usepackage{ifthen}
\usepackage[landscape]{geometry}
\usepackage[shortlabels]{enumitem}

\ifthenelse{\lengthtest { \paperwidth = 11in}}
    { \geometry{top=.5in,left=.5in,right=.5in,bottom=.5in} }
	{\ifthenelse{ \lengthtest{ \paperwidth = 297mm}}
		{\geometry{top=1cm,left=1cm,right=1cm,bottom=1cm} }
		{\geometry{top=1cm,left=1cm,right=1cm,bottom=1cm} }
	}

\pagestyle{empty}
\makeatletter
\renewcommand\thesection{\arabic{section}.}
\renewcommand{\section}{\@startsection{section}{1}{0mm}%
                                {-1ex plus -.5ex minus -.2ex}%
                                {0.05ex}%x
                                {\normalfont\normalsize\bfseries}}
\renewcommand{\subsection}{\@startsection{subsection}{2}{0mm}%
                                {-1ex plus -.5ex minus -.2ex}%
                                {0.05ex}%
                                {\normalfont\small\bfseries}}
\renewcommand{\subsubsection}{\@startsection{subsubsection}{3}{0mm}%
                                {-1ex plus -.5ex minus -.2ex}%
                                {0.05ex}%
                                {\normalfont\footnotesize\bfseries}}
\newcommand{\colbreak}{\vfill\null\columnbreak}
\makeatother
\setcounter{secnumdepth}{1}
\setlength{\parindent}{0pt}
\setlength{\parskip}{0.7em}

\setlist[itemize]{itemsep=0.6ex, topsep=-2pt, partopsep=0pt, parsep=0pt}
\setlist[enumerate]{itemsep=0.6ex, topsep=-2pt, partopsep=0pt, parsep=0pt}

\input{letterfonts}

\newcommand{\mytitle}{CS2030S Programming Methodology II (PE)}
\newcommand{\myauthor}{github/omgeta}
\newcommand{\mydate}{AY 24/25 Sem 1}

\begin{document}
\raggedright
\footnotesize
\begin{multicols*}{2}
\setlength{\premulticols}{1pt}
\setlength{\postmulticols}{1pt}
\setlength{\multicolsep}{1pt}
\setlength{\columnsep}{2pt}

{\normalsize{\textbf{\mytitle}}} \\
{\footnotesize{\mydate\hspace{2pt}\textemdash\hspace{2pt}\myauthor}}

%%%%%%%%%%%%%%%%%%%%%%%%%%%%%%%%%%%%%%%%%%%%%%%%%%%%%%
%                      Begin                         %
%%%%%%%%%%%%%%%%%%%%%%%%%%%%%%%%%%%%%%%%%%%%%%%%%%%%%%
\section{PE1}
\subsection{Array$<$T$>$}
\begin{lstlisting}
class Array<T> {
  private T[] array;

  Array(int size) {
    // The only way we can put an object into the array is through
    // the method set() and we can only put an object of type T inside. 
    // So it is type safe to cast `Object[]` to `T[]`
    @SuppressWarnings("unchecked")
    T[] a = (T[]) new Object[size];
    this.array = a;
  }

  public void set(int index, T item) {
    this.array[index] = item;
  }

  public T get(int index) {
    return this.array[index];
  }

  public void copyFrom(Array<? extends T> src) {
    int len = Math.min(this.array.length, src.array.length);
    for (int i = 0; i < len; i++) {
      this.set(i, src.get(i));
    }
  }

  public void copyTo(Array<? super T> dest) {
    int len = Math.min(this.array.length, dest.array.length);
    for (int i = 0; i < len; i++) {
      dest.set(i, this.get(i));
    }
  }
}
\end{lstlisting}
\colbreak

\subsection{Implementing Comparable$<$T$>$}
\begin{lstlisting}
class Packet implements Comparable<Packet> {
  private String message;

  public Packet(String message) {
    this.message = message;
  }

  @Override
  public String toString() {
    return this.message;
  }

  public int compareTo(Packet other) {
    if (this.message.length() == other.message.length()) {
      return 0;
    } else if (this.message.length() < other.message.length()) {
      return 1;
    } else {
      return -1;
    }
  }
}
\end{lstlisting}

\subsection{Composing Comparable$<$T$>$}
\begin{lstlisting}
public class Buffer<T extends Comparable<T>> {
  private T[] messages;
  private int endIndex;

  public Buffer(int size) {
    // The only way to put an object into array is through
    // Buffer::send and we only put Object of type T inside.
    // Thus it is safe to cast `Object[]` to `T[]`.
    @SuppressWarnings("unchecked")
    T[] temp = (T[]) new Comparable<?>[size];
    this.messages = temp;
    this.endIndex = 0;
  }
}
\end{lstlisting}

\section{PE2}
\subsection{Immutability}
Checklist:
\begin{enumerate}[\roman*.]
  \item All fields are \lstinline|final| (not necessary)
  \item All types in fields are immutable
  \item Arrays are copied before assignment
  \item No mutator (or return a new instance)
  \item Class is \lstinline|final|
\end{enumerate}

\subsection{Functional Interfaces}
\begin{lstlisting}
@FunctionalInterface
public interface BiFunction<T, U, R> {
   R apply(T t, U u);
}
\end{lstlisting}

Equivalent Interfaces:
\begin{enumerate}[\roman*.]
  \item \lstinline|BooleanCondition<T>::test| $\iff$ \lstinline|Predicate<T>::test|
  \item \lstinline|Producer<T>::produce| $\iff$ \lstinline|Supplier<T>::get|
  \item \lstinline|Consumer<T>::consume| $\iff$ \lstinline|Consumer<T>::accept|
  \item \lstinline|Transformer<T, R>::transform| $\iff$ \lstinline|Function<T, R>::apply| /\\
    \quad\quad\quad\quad\quad\quad\quad\quad\quad\quad\quad\quad\quad\quad\quad\quad\quad\quad\quad\quad\lstinline|UnaryOp<T>::apply|
\end{enumerate}

\subsection{Monads and Functors}
Monad Laws:
\begin{enumerate}[\roman*.]
  \item \lstinline|Monad.of(x).flatMap(x -> f(x))| $\equiv$ \lstinline|f(x)|\hfill(Left Identity)
  \item \lstinline|monad.flatMap(x -> Monad.of(x))| $\equiv$ \lstinline|monad|\hfill(Right Identity)
  \item \lstinline|monad.flatMap(x -> f(x)).flatMap(x -> g(x))| $\equiv$ \lstinline|monad.flatMap(x -> f(x).flatMap(y -> g(y)))|\hfill(Associative)
\end{enumerate}

Functor Laws:
\begin{enumerate}[\roman*.]
  \item \lstinline|functor.map(x -> x)| $\equiv$ \lstinline|functor|\hfill(Identity)
  \item \lstinline|functor.map(x -> f(x)).map(x -> g(x))| $\equiv$\\ \lstinline|functor.map(x -> g(f(x))|\hfill(Composition)
\end{enumerate}
\colbreak

\subsection{Stream$<$T$>$}
Creation:
\begin{enumerate}[\roman*.]
  \item \lstinline|Stream::of(T...)| : \lstinline|Stream<T>|
  \item \lstinline|Stream::generate(Supplier<T>)| : \lstinline|Stream<T>|
  \item \lstinline|Stream::iterate(T, UnaryOp<T>)| : \lstinline|Stream<T>|
  \item \lstinline|Stream::iterate(T, Predicate<? super T>, UnaryOp<T>)| : \lstinline|Stream<T>|
  \item \lstinline|List.stream()| : \lstinline|Stream<T>|
\end{enumerate}

Intermediate:
\begin{enumerate}[\roman*.]
  \item \lstinline|filter(Predicate<? super T>)|
  \item \lstinline|map(Function<? super T, ? extends R>)|
  \item \lstinline|flatMap(Function<? super T, ? extends Stream<? extends R>>)|
  \item \lstinline|takeWhile(Predicate<? super T>)|
  \item \lstinline|dropWhile(Predicate<? super T>)|
  \item \lstinline|distinct()|
  \item \lstinline|sorted()|
  \item \lstinline|sorted(Comparator<? super T>)|
  \item \lstinline|peek(Consumer<? super T>)|
  \item \lstinline|limit(long)|
  \item \lstinline|skip(long)|
\end{enumerate}

Terminal:
\begin{enumerate}[\roman*.]
  \item \lstinline|anyMatch(Predicate<? super T>)| : \lstinline|boolean|
  \item \lstinline|allMatch(Predicate<? super T>)| : \lstinline|boolean|
  \item \lstinline|noneMatch(Predicate<? super T>)| : \lstinline|boolean|
  \item \lstinline|count()| : \lstinline|long|
  \item \lstinline|findAny()| : \lstinline|T|
  \item \lstinline|findFirst()| : \lstinline|Optional<T>|
  \item \lstinline|forEach(Consumer<? super T>)| : \lstinline|void|
  \item \lstinline|forEachOrdered(Consumer<? super T>)| : \lstinline|void|
  \item \lstinline|min((x,y) -> x.compareTo(y))| : \lstinline|Optional<T>|
  \item \lstinline|max((x,y) -> x.compareTo(y))| : \lstinline|Optional<T>|
  \item \lstinline|reduce(T, BinaryOperator<T>)| : \lstinline|T|
  \item \lstinline|reduce(U, BiFunction<U, ? super T, U>, BinaryOperator<U>)| : \lstinline|U|
  \item \lstinline|toArray()| : \lstinline|Object[]|
  \item \lstinline|toList()| : \lstinline|List<T>|
\end{enumerate}

\subsection{Maybe$<$T$>$}
Creation:
\begin{enumerate}[\roman*.]
  \item \lstinline|Maybe::of(T)| : \lstinline|Maybe<T>| (\lstinline|Some<T>| if not \lstinline|null| else \lstinline|None<T>|)
  \item \lstinline|Maybe::some()| : \lstinline|Some<T>|
  \item \lstinline|Maybe::none()| : \lstinline|None<T>|
\end{enumerate}

Intermediate:
\begin{enumerate}[\roman*.]
  \item \lstinline|filter(BooleanCondition<? super T>)|
  \item \lstinline|map(Transformer<? super T, ? extends R>)|
  \item \lstinline|flatMap(Transformer<? super T, ? extends Maybe<? extends R>>)|
\end{enumerate}

Terminal:
\begin{enumerate}[\roman*.]
  \item \lstinline|orElse(Producer<? extends T>)| : \lstinline|T|
  \item \lstinline|ifPresent(Consumer<? super T>)| : \lstinline|void|
  \item \lstinline|toString()| : \lstinline|String|
  \item \lstinline|equals(Object)| : \lstinline|boolean|
\end{enumerate}

\subsection{Lazy$<$T$>$}
Creation:
\begin{enumerate}[\roman*.]
  \item \lstinline|Lazy::of(T)| : \lstinline|Lazy<T>| (pre-evaluated)
  \item \lstinline|Lazy::of(Producer<? extends T>)| : \lstinline|Lazy<T>|
\end{enumerate}

Intermediate:
\begin{enumerate}[\roman*.]
  \item \lstinline|filter(BooleanCondition<? super T>)|
  \item \lstinline|map(Transformer<? super T, ? extends R>)|
  \item \lstinline|flatMap(Transformer<? super T, ? extends Lazy<? extends R>>)|
\end{enumerate}

Terminal:
\begin{enumerate}[\roman*.]
  \item \lstinline|get()| : \lstinline|T|
  \item \lstinline|equals(Object)| : \lstinline|boolean|
\end{enumerate}
\colbreak

\subsection{Parallelization}
Parallelizing Streams:
\begin{enumerate}[\roman*.]
  \item \lstinline|Collection::parallelStream()|
  \item \lstinline|Stream::parallel()|
\end{enumerate}

Conditions for Parallelization:
\begin{enumerate}[\roman*.]
  \item Non-interference with data source
  \item Avoid side-effects
  \item Stateless lambdas
  \item Prefer unordered (or use \lstinline|.unordered()|)
\end{enumerate}
%%%%%%%%%%%%%%%%%%%%%%%%%%%%%%%%%%%%%%%%%%%%%%%%%%%%%%
%                       End                          %
%%%%%%%%%%%%%%%%%%%%%%%%%%%%%%%%%%%%%%%%%%%%%%%%%%%%%%

\end{multicols*}
\end{document}

\documentclass[12pt, a4paper]{article}

\usepackage[utf8]{inputenc}
\usepackage[mathscr]{euscript}
\let\euscr\mathscr \let\mathscr\relax
\usepackage[scr]{rsfso}
\usepackage{amssymb,amsmath,amsthm,amsfonts}
\usepackage[shortlabels]{enumitem}
\usepackage{multicol,multirow}
\usepackage{lipsum}
\usepackage{balance}
\usepackage{calc}
\usepackage[colorlinks=true,citecolor=blue,linkcolor=blue]{hyperref}
\usepackage{import}
\usepackage{xifthen}
\usepackage{pdfpages}
\usepackage{transparent}
\usepackage{listings}

\newcommand{\incfig}[2][1.0]{
    \def\svgwidth{#1\columnwidth}
    \import{./figures/}{#2.pdf_tex}
}

\newlist{enumproof}{enumerate}{4}
\setlist[enumproof,1]{label=\arabic*., parsep=1em}
\setlist[enumproof,2]{label=\arabic{enumproofi}.\arabic*., parsep=1em}
\setlist[enumproof,3]{label=\arabic{enumproofi}.\arabic{enumproofii}.\arabic*., parsep=1em}
\setlist[enumproof,4]{label=\arabic{enumproofi}.\arabic{enumproofii}.\arabic{enumproofiii}.\arabic*., parsep=1em}

\renewcommand{\qedsymbol}{\ensuremath{\blacksquare}}

\lstdefinestyle{mystyle}{
  language=C, % Set the language to C
  commentstyle=\color{codegray}, % Color for comments
  keywordstyle=\color{orange}, % Color for basic keywords
  stringstyle=\color{mauve}, % Color for strings
  basicstyle={\ttfamily\footnotesize}, % Basic font style
  breakatwhitespace=false,         
  breaklines=true,                 
  captionpos=b,                    
  keepspaces=true,                 
  numbers=none,                    
  tabsize=2,
  morekeywords=[2]{\#include, \#define, \#ifdef, \#ifndef, \#endif, \#pragma, \#else, \#elif}, % Preprocessor directives
  keywordstyle=[2]\color{codegreen}, % Style for preprocessor directives
  morekeywords=[3]{int, char, float, double, void, struct, union, enum, const, volatile, static, extern, register, inline, restrict, _Bool, _Complex, _Imaginary, size_t, ssize_t, FILE}, % C standard types and common identifiers
  keywordstyle=[3]\color{identblue}, % Style for types and common identifiers
  morekeywords=[4]{printf, scanf, fopen, fclose, malloc, free, calloc, realloc, perror, strtok, strncpy, strcpy, strcmp, strlen}, % Standard library functions
  keywordstyle=[4]\color{cyan}, % Style for library functions
}

\usepackage{ifthen}
\usepackage[landscape]{geometry}
\usepackage[shortlabels]{enumitem}

\ifthenelse{\lengthtest { \paperwidth = 11in}}
    { \geometry{top=.5in,left=.5in,right=.5in,bottom=.5in} }
	{\ifthenelse{ \lengthtest{ \paperwidth = 297mm}}
		{\geometry{top=1cm,left=1cm,right=1cm,bottom=1cm} }
		{\geometry{top=1cm,left=1cm,right=1cm,bottom=1cm} }
	}

\pagestyle{empty}
\makeatletter
\renewcommand\thesection{\arabic{section}.}
\renewcommand{\section}{\@startsection{section}{1}{0mm}%
                                {-1ex plus -.5ex minus -.2ex}%
                                {0.05ex}%x
                                {\normalfont\normalsize\bfseries}}
\renewcommand{\subsection}{\@startsection{subsection}{2}{0mm}%
                                {-1ex plus -.5ex minus -.2ex}%
                                {0.05ex}%
                                {\normalfont\small\bfseries}}
\renewcommand{\subsubsection}{\@startsection{subsubsection}{3}{0mm}%
                                {-1ex plus -.5ex minus -.2ex}%
                                {0.05ex}%
                                {\normalfont\footnotesize\bfseries}}
\newcommand{\colbreak}{\vfill\null\columnbreak}
\makeatother
\setcounter{secnumdepth}{1}
\setlength{\parindent}{0pt}
\setlength{\parskip}{0.7em}

\setlist[itemize]{itemsep=0.6ex, topsep=-2pt, partopsep=0pt, parsep=0pt}
\setlist[enumerate]{itemsep=0.6ex, topsep=-2pt, partopsep=0pt, parsep=0pt}

% Things Lie
\newcommand{\kb}{\mathfrak b}
\newcommand{\kg}{\mathfrak g}
\newcommand{\kh}{\mathfrak h}
\newcommand{\kn}{\mathfrak n}
\newcommand{\ku}{\mathfrak u}
\newcommand{\kz}{\mathfrak z}
\DeclareMathOperator{\Ext}{Ext} % Ext functor
\DeclareMathOperator{\Tor}{Tor} % Tor functor
\newcommand{\gl}{\opname{\mathfrak{gl}}} % frak gl group
\renewcommand{\sl}{\opname{\mathfrak{sl}}} % frak sl group chktex 6

% More script letters etc.
\newcommand{\SA}{\mathcal A}
\newcommand{\SB}{\mathcal B}
\newcommand{\SC}{\mathcal C}
\newcommand{\SF}{\mathcal F}
\newcommand{\SG}{\mathcal G}
\newcommand{\SH}{\mathcal H}
\newcommand{\OO}{\mathcal O}

\newcommand{\SCA}{\mathscr A}
\newcommand{\SCB}{\mathscr B}
\newcommand{\SCC}{\mathscr C}
\newcommand{\SCD}{\mathscr D}
\newcommand{\SCE}{\mathscr E}
\newcommand{\SCF}{\mathscr F}
\newcommand{\SCG}{\mathscr G}
\newcommand{\SCH}{\mathscr H}

% Mathfrak primes
\newcommand{\km}{\mathfrak m}
\newcommand{\kp}{\mathfrak p}
\newcommand{\kq}{\mathfrak q}

% number sets
\newcommand{\RR}[1][]{\ensuremath{\ifstrempty{#1}{\mathbb{R}}{\mathbb{R}^{#1}}}}
\newcommand{\NN}[1][]{\ensuremath{\ifstrempty{#1}{\mathbb{N}}{\mathbb{N}^{#1}}}}
\newcommand{\ZZ}[1][]{\ensuremath{\ifstrempty{#1}{\mathbb{Z}}{\mathbb{Z}^{#1}}}}
\newcommand{\QQ}[1][]{\ensuremath{\ifstrempty{#1}{\mathbb{Q}}{\mathbb{Q}^{#1}}}}
\newcommand{\CC}[1][]{\ensuremath{\ifstrempty{#1}{\mathbb{C}}{\mathbb{C}^{#1}}}}
\newcommand{\PP}[1][]{\ensuremath{\ifstrempty{#1}{\mathbb{P}}{\mathbb{P}^{#1}}}}
\newcommand{\HH}[1][]{\ensuremath{\ifstrempty{#1}{\mathbb{H}}{\mathbb{H}^{#1}}}}
\newcommand{\FF}[1][]{\ensuremath{\ifstrempty{#1}{\mathbb{F}}{\mathbb{F}^{#1}}}}
% expected value
\newcommand{\EE}{\ensuremath{\mathbb{E}}}
\newcommand{\charin}{\text{ char }}
\DeclareMathOperator{\sign}{sign}
\DeclareMathOperator{\Aut}{Aut}
\DeclareMathOperator{\Inn}{Inn}
\DeclareMathOperator{\Syl}{Syl}
\DeclareMathOperator{\Gal}{Gal}
\DeclareMathOperator{\GL}{GL} % General linear group
\DeclareMathOperator{\SL}{SL} % Special linear group

%---------------------------------------
% BlackBoard Math Fonts :-
%---------------------------------------

%Captital Letters
\newcommand{\bbA}{\mathbb{A}}	\newcommand{\bbB}{\mathbb{B}}
\newcommand{\bbC}{\mathbb{C}}	\newcommand{\bbD}{\mathbb{D}}
\newcommand{\bbE}{\mathbb{E}}	\newcommand{\bbF}{\mathbb{F}}
\newcommand{\bbG}{\mathbb{G}}	\newcommand{\bbH}{\mathbb{H}}
\newcommand{\bbI}{\mathbb{I}}	\newcommand{\bbJ}{\mathbb{J}}
\newcommand{\bbK}{\mathbb{K}}	\newcommand{\bbL}{\mathbb{L}}
\newcommand{\bbM}{\mathbb{M}}	\newcommand{\bbN}{\mathbb{N}}
\newcommand{\bbO}{\mathbb{O}}	\newcommand{\bbP}{\mathbb{P}}
\newcommand{\bbQ}{\mathbb{Q}}	\newcommand{\bbR}{\mathbb{R}}
\newcommand{\bbS}{\mathbb{S}}	\newcommand{\bbT}{\mathbb{T}}
\newcommand{\bbU}{\mathbb{U}}	\newcommand{\bbV}{\mathbb{V}}
\newcommand{\bbW}{\mathbb{W}}	\newcommand{\bbX}{\mathbb{X}}
\newcommand{\bbY}{\mathbb{Y}}	\newcommand{\bbZ}{\mathbb{Z}}

%---------------------------------------
% MathCal Fonts :-
%---------------------------------------

%Captital Letters
\newcommand{\mcA}{\mathcal{A}}	\newcommand{\mcB}{\mathcal{B}}
\newcommand{\mcC}{\mathcal{C}}	\newcommand{\mcD}{\mathcal{D}}
\newcommand{\mcE}{\mathcal{E}}	\newcommand{\mcF}{\mathcal{F}}
\newcommand{\mcG}{\mathcal{G}}	\newcommand{\mcH}{\mathcal{H}}
\newcommand{\mcI}{\mathcal{I}}	\newcommand{\mcJ}{\mathcal{J}}
\newcommand{\mcK}{\mathcal{K}}	\newcommand{\mcL}{\mathcal{L}}
\newcommand{\mcM}{\mathcal{M}}	\newcommand{\mcN}{\mathcal{N}}
\newcommand{\mcO}{\mathcal{O}}	\newcommand{\mcP}{\mathcal{P}}
\newcommand{\mcQ}{\mathcal{Q}}	\newcommand{\mcR}{\mathcal{R}}
\newcommand{\mcS}{\mathcal{S}}	\newcommand{\mcT}{\mathcal{T}}
\newcommand{\mcU}{\mathcal{U}}	\newcommand{\mcV}{\mathcal{V}}
\newcommand{\mcW}{\mathcal{W}}	\newcommand{\mcX}{\mathcal{X}}
\newcommand{\mcY}{\mathcal{Y}}	\newcommand{\mcZ}{\mathcal{Z}}

%---------------------------------------
% Bold Math Fonts :-
%---------------------------------------

%Captital Letters
\newcommand{\bmA}{\boldsymbol{A}}	\newcommand{\bmB}{\boldsymbol{B}}
\newcommand{\bmC}{\boldsymbol{C}}	\newcommand{\bmD}{\boldsymbol{D}}
\newcommand{\bmE}{\boldsymbol{E}}	\newcommand{\bmF}{\boldsymbol{F}}
\newcommand{\bmG}{\boldsymbol{G}}	\newcommand{\bmH}{\boldsymbol{H}}
\newcommand{\bmI}{\boldsymbol{I}}	\newcommand{\bmJ}{\boldsymbol{J}}
\newcommand{\bmK}{\boldsymbol{K}}	\newcommand{\bmL}{\boldsymbol{L}}
\newcommand{\bmM}{\boldsymbol{M}}	\newcommand{\bmN}{\boldsymbol{N}}
\newcommand{\bmO}{\boldsymbol{O}}	\newcommand{\bmP}{\boldsymbol{P}}
\newcommand{\bmQ}{\boldsymbol{Q}}	\newcommand{\bmR}{\boldsymbol{R}}
\newcommand{\bmS}{\boldsymbol{S}}	\newcommand{\bmT}{\boldsymbol{T}}
\newcommand{\bmU}{\boldsymbol{U}}	\newcommand{\bmV}{\boldsymbol{V}}
\newcommand{\bmW}{\boldsymbol{W}}	\newcommand{\bmX}{\boldsymbol{X}}
\newcommand{\bmY}{\boldsymbol{Y}}	\newcommand{\bmZ}{\boldsymbol{Z}}
%Small Letters
\newcommand{\bma}{\boldsymbol{a}}	\newcommand{\bmb}{\boldsymbol{b}}
\newcommand{\bmc}{\boldsymbol{c}}	\newcommand{\bmd}{\boldsymbol{d}}
\newcommand{\bme}{\boldsymbol{e}}	\newcommand{\bmf}{\boldsymbol{f}}
\newcommand{\bmg}{\boldsymbol{g}}	\newcommand{\bmh}{\boldsymbol{h}}
\newcommand{\bmi}{\boldsymbol{i}}	\newcommand{\bmj}{\boldsymbol{j}}
\newcommand{\bmk}{\boldsymbol{k}}	\newcommand{\bml}{\boldsymbol{l}}
\newcommand{\bmm}{\boldsymbol{m}}	\newcommand{\bmn}{\boldsymbol{n}}
\newcommand{\bmo}{\boldsymbol{o}}	\newcommand{\bmp}{\boldsymbol{p}}
\newcommand{\bmq}{\boldsymbol{q}}	\newcommand{\bmr}{\boldsymbol{r}}
\newcommand{\bms}{\boldsymbol{s}}	\newcommand{\bmt}{\boldsymbol{t}}
\newcommand{\bmu}{\boldsymbol{u}}	\newcommand{\bmv}{\boldsymbol{v}}
\newcommand{\bmw}{\boldsymbol{w}}	\newcommand{\bmx}{\boldsymbol{x}}
\newcommand{\bmy}{\boldsymbol{y}}	\newcommand{\bmz}{\boldsymbol{z}}

%---------------------------------------
% Scr Math Fonts :-
%---------------------------------------

\newcommand{\sA}{{\mathscr{A}}}   \newcommand{\sB}{{\mathscr{B}}}
\newcommand{\sC}{{\mathscr{C}}}   \newcommand{\sD}{{\mathscr{D}}}
\newcommand{\sE}{{\mathscr{E}}}   \newcommand{\sF}{{\mathscr{F}}}
\newcommand{\sG}{{\mathscr{G}}}   \newcommand{\sH}{{\mathscr{H}}}
\newcommand{\sI}{{\mathscr{I}}}   \newcommand{\sJ}{{\mathscr{J}}}
\newcommand{\sK}{{\mathscr{K}}}   \newcommand{\sL}{{\mathscr{L}}}
\newcommand{\sM}{{\mathscr{M}}}   \newcommand{\sN}{{\mathscr{N}}}
\newcommand{\sO}{{\mathscr{O}}}   \newcommand{\sP}{{\mathscr{P}}}
\newcommand{\sQ}{{\mathscr{Q}}}   \newcommand{\sR}{{\mathscr{R}}}
\newcommand{\sS}{{\mathscr{S}}}   \newcommand{\sT}{{\mathscr{T}}}
\newcommand{\sU}{{\mathscr{U}}}   \newcommand{\sV}{{\mathscr{V}}}
\newcommand{\sW}{{\mathscr{W}}}   \newcommand{\sX}{{\mathscr{X}}}
\newcommand{\sY}{{\mathscr{Y}}}   \newcommand{\sZ}{{\mathscr{Z}}}


%---------------------------------------
% Math Fraktur Font
%---------------------------------------

%Captital Letters
\newcommand{\mfA}{\mathfrak{A}}	\newcommand{\mfB}{\mathfrak{B}}
\newcommand{\mfC}{\mathfrak{C}}	\newcommand{\mfD}{\mathfrak{D}}
\newcommand{\mfE}{\mathfrak{E}}	\newcommand{\mfF}{\mathfrak{F}}
\newcommand{\mfG}{\mathfrak{G}}	\newcommand{\mfH}{\mathfrak{H}}
\newcommand{\mfI}{\mathfrak{I}}	\newcommand{\mfJ}{\mathfrak{J}}
\newcommand{\mfK}{\mathfrak{K}}	\newcommand{\mfL}{\mathfrak{L}}
\newcommand{\mfM}{\mathfrak{M}}	\newcommand{\mfN}{\mathfrak{N}}
\newcommand{\mfO}{\mathfrak{O}}	\newcommand{\mfP}{\mathfrak{P}}
\newcommand{\mfQ}{\mathfrak{Q}}	\newcommand{\mfR}{\mathfrak{R}}
\newcommand{\mfS}{\mathfrak{S}}	\newcommand{\mfT}{\mathfrak{T}}
\newcommand{\mfU}{\mathfrak{U}}	\newcommand{\mfV}{\mathfrak{V}}
\newcommand{\mfW}{\mathfrak{W}}	\newcommand{\mfX}{\mathfrak{X}}
\newcommand{\mfY}{\mathfrak{Y}}	\newcommand{\mfZ}{\mathfrak{Z}}
%Small Letters
\newcommand{\mfa}{\mathfrak{a}}	\newcommand{\mfb}{\mathfrak{b}}
\newcommand{\mfc}{\mathfrak{c}}	\newcommand{\mfd}{\mathfrak{d}}
\newcommand{\mfe}{\mathfrak{e}}	\newcommand{\mff}{\mathfrak{f}}
\newcommand{\mfg}{\mathfrak{g}}	\newcommand{\mfh}{\mathfrak{h}}
\newcommand{\mfi}{\mathfrak{i}}	\newcommand{\mfj}{\mathfrak{j}}
\newcommand{\mfk}{\mathfrak{k}}	\newcommand{\mfl}{\mathfrak{l}}
\newcommand{\mfm}{\mathfrak{m}}	\newcommand{\mfn}{\mathfrak{n}}
\newcommand{\mfo}{\mathfrak{o}}	\newcommand{\mfp}{\mathfrak{p}}
\newcommand{\mfq}{\mathfrak{q}}	\newcommand{\mfr}{\mathfrak{r}}
\newcommand{\mfs}{\mathfrak{s}}	\newcommand{\mft}{\mathfrak{t}}
\newcommand{\mfu}{\mathfrak{u}}	\newcommand{\mfv}{\mathfrak{v}}
\newcommand{\mfw}{\mathfrak{w}}	\newcommand{\mfx}{\mathfrak{x}}
\newcommand{\mfy}{\mathfrak{y}}	\newcommand{\mfz}{\mathfrak{z}}


\newcommand{\mytitle}{CS2030S Programming Methodology II (PE)}
\newcommand{\myauthor}{github/omgeta}
\newcommand{\mydate}{AY 24/25 Sem 1}

\begin{document}
\raggedright
\footnotesize
\begin{multicols*}{2}
\setlength{\premulticols}{1pt}
\setlength{\postmulticols}{1pt}
\setlength{\multicolsep}{1pt}
\setlength{\columnsep}{2pt}

{\normalsize{\textbf{\mytitle}}} \\
{\footnotesize{\mydate\hspace{2pt}\textemdash\hspace{2pt}\myauthor}}

%%%%%%%%%%%%%%%%%%%%%%%%%%%%%%%%%%%%%%%%%%%%%%%%%%%%%%
%                      Begin                         %
%%%%%%%%%%%%%%%%%%%%%%%%%%%%%%%%%%%%%%%%%%%%%%%%%%%%%%
\section{PE1}
\subsection{Array$<$T$>$}
\begin{lstlisting}
class Array<T> {
  private T[] array;

  Array(int size) {
    // The only way we can put an object into the array is through
    // the method set() and we can only put an object of type T inside. 
    // So it is type safe to cast `Object[]` to `T[]`
    @SuppressWarnings("unchecked")
    T[] a = (T[]) new Object[size];
    this.array = a;
  }

  public void set(int index, T item) {
    this.array[index] = item;
  }

  public T get(int index) {
    return this.array[index];
  }

  public void copyFrom(Array<? extends T> src) {
    int len = Math.min(this.array.length, src.array.length);
    for (int i = 0; i < len; i++) {
      this.set(i, src.get(i));
    }
  }

  public void copyTo(Array<? super T> dest) {
    int len = Math.min(this.array.length, dest.array.length);
    for (int i = 0; i < len; i++) {
      dest.set(i, this.get(i));
    }
  }
}
\end{lstlisting}
\colbreak

\subsection{Implementing Comparable$<$T$>$}
\begin{lstlisting}
class Packet implements Comparable<Packet> {
  private String message;

  public Packet(String message) {
    this.message = message;
  }

  @Override
  public String toString() {
    return this.message;
  }

  public int compareTo(Packet other) {
    if (this.message.length() == other.message.length()) {
      return 0;
    } else if (this.message.length() < other.message.length()) {
      return 1;
    } else {
      return -1;
    }
  }
}
\end{lstlisting}

\subsection{Composing Comparable$<$T$>$}
\begin{lstlisting}
public class Buffer<T extends Comparable<T>> {
  private T[] messages;
  private int endIndex;

  public Buffer(int size) {
    // The only way to put an object into array is through
    // Buffer::send and we only put Object of type T inside.
    // Thus it is safe to cast `Object[]` to `T[]`.
    @SuppressWarnings("unchecked")
    T[] temp = (T[]) new Comparable<?>[size];
    this.messages = temp;
    this.endIndex = 0;
  }
}
\end{lstlisting}

\section{PE2}
\subsection{Immutability}
Checklist:
\begin{enumerate}[\roman*.]
  \item All fields are \lstinline|final| (not necessary)
  \item All types in fields are immutable
  \item Arrays are copied before assignment
  \item No mutator (or return a new instance)
  \item Class is \lstinline|final|
\end{enumerate}

\subsection{Functional Interfaces}
\begin{lstlisting}
@FunctionalInterface
public interface BiFunction<T, U, R> {
   R apply(T t, U u);
}
\end{lstlisting}

Equivalent Interfaces:
\begin{enumerate}[\roman*.]
  \item \lstinline|BooleanCondition<T>::test| $\iff$ \lstinline|Predicate<T>::test|
  \item \lstinline|Producer<T>::produce| $\iff$ \lstinline|Supplier<T>::get|
  \item \lstinline|Consumer<T>::consume| $\iff$ \lstinline|Consumer<T>::accept|
  \item \lstinline|Transformer<T, R>::transform| $\iff$ \lstinline|Function<T, R>::apply| /\\
    \quad\quad\quad\quad\quad\quad\quad\quad\quad\quad\quad\quad\quad\quad\quad\quad\quad\quad\quad\quad\lstinline|UnaryOp<T>::apply|
\end{enumerate}

\subsection{Monads and Functors}
Monad Laws:
\begin{enumerate}[\roman*.]
  \item \lstinline|Monad.of(x).flatMap(x -> f(x))| $\equiv$ \lstinline|f(x)|\hfill(Left Identity)
  \item \lstinline|monad.flatMap(x -> Monad.of(x))| $\equiv$ \lstinline|monad|\hfill(Right Identity)
  \item \lstinline|monad.flatMap(x -> f(x)).flatMap(x -> g(x))| $\equiv$ \lstinline|monad.flatMap(x -> f(x).flatMap(y -> g(y)))|\hfill(Associative)
\end{enumerate}

Functor Laws:
\begin{enumerate}[\roman*.]
  \item \lstinline|functor.map(x -> x)| $\equiv$ \lstinline|functor|\hfill(Identity)
  \item \lstinline|functor.map(x -> f(x)).map(x -> g(x))| $\equiv$\\ \lstinline|functor.map(x -> g(f(x))|\hfill(Composition)
\end{enumerate}
\colbreak

\subsection{Stream$<$T$>$}
Creation:
\begin{enumerate}[\roman*.]
  \item \lstinline|Stream::of(T...)| : \lstinline|Stream<T>|
  \item \lstinline|Stream::generate(Supplier<T>)| : \lstinline|Stream<T>|
  \item \lstinline|Stream::iterate(T, UnaryOp<T>)| : \lstinline|Stream<T>|
  \item \lstinline|Stream::iterate(T, Predicate<? super T>, UnaryOp<T>)| : \lstinline|Stream<T>|
  \item \lstinline|List.stream()| : \lstinline|Stream<T>|
\end{enumerate}

Intermediate:
\begin{enumerate}[\roman*.]
  \item \lstinline|filter(Predicate<? super T>)|
  \item \lstinline|map(Function<? super T, ? extends R>)|
  \item \lstinline|flatMap(Function<? super T, ? extends Stream<? extends R>>)|
  \item \lstinline|takeWhile(Predicate<? super T>)|
  \item \lstinline|dropWhile(Predicate<? super T>)|
  \item \lstinline|distinct()|
  \item \lstinline|sorted()|
  \item \lstinline|sorted(Comparator<? super T>)|
  \item \lstinline|peek(Consumer<? super T>)|
  \item \lstinline|limit(long)|
  \item \lstinline|skip(long)|
\end{enumerate}

Terminal:
\begin{enumerate}[\roman*.]
  \item \lstinline|anyMatch(Predicate<? super T>)| : \lstinline|boolean|
  \item \lstinline|allMatch(Predicate<? super T>)| : \lstinline|boolean|
  \item \lstinline|noneMatch(Predicate<? super T>)| : \lstinline|boolean|
  \item \lstinline|count()| : \lstinline|long|
  \item \lstinline|findAny()| : \lstinline|T|
  \item \lstinline|findFirst()| : \lstinline|Optional<T>|
  \item \lstinline|forEach(Consumer<? super T>)| : \lstinline|void|
  \item \lstinline|forEachOrdered(Consumer<? super T>)| : \lstinline|void|
  \item \lstinline|min((x,y) -> x.compareTo(y))| : \lstinline|Optional<T>|
  \item \lstinline|max((x,y) -> x.compareTo(y))| : \lstinline|Optional<T>|
  \item \lstinline|reduce(T, BinaryOperator<T>)| : \lstinline|T|
  \item \lstinline|reduce(U, BiFunction<U, ? super T, U>, BinaryOperator<U>)| : \lstinline|U|
  \item \lstinline|toArray()| : \lstinline|Object[]|
  \item \lstinline|toList()| : \lstinline|List<T>|
\end{enumerate}

\subsection{Maybe$<$T$>$}
Creation:
\begin{enumerate}[\roman*.]
  \item \lstinline|Maybe::of(T)| : \lstinline|Maybe<T>| (\lstinline|Some<T>| if not \lstinline|null| else \lstinline|None<T>|)
  \item \lstinline|Maybe::some()| : \lstinline|Some<T>|
  \item \lstinline|Maybe::none()| : \lstinline|None<T>|
\end{enumerate}

Intermediate:
\begin{enumerate}[\roman*.]
  \item \lstinline|filter(BooleanCondition<? super T>)|
  \item \lstinline|map(Transformer<? super T, ? extends R>)|
  \item \lstinline|flatMap(Transformer<? super T, ? extends Maybe<? extends R>>)|
\end{enumerate}

Terminal:
\begin{enumerate}[\roman*.]
  \item \lstinline|orElse(Producer<? extends T>)| : \lstinline|T|
  \item \lstinline|ifPresent(Consumer<? super T>)| : \lstinline|void|
  \item \lstinline|toString()| : \lstinline|String|
  \item \lstinline|equals(Object)| : \lstinline|boolean|
\end{enumerate}

\subsection{Lazy$<$T$>$}
Creation:
\begin{enumerate}[\roman*.]
  \item \lstinline|Lazy::of(T)| : \lstinline|Lazy<T>| (pre-evaluated)
  \item \lstinline|Lazy::of(Producer<? extends T>)| : \lstinline|Lazy<T>|
\end{enumerate}

Intermediate:
\begin{enumerate}[\roman*.]
  \item \lstinline|filter(BooleanCondition<? super T>)|
  \item \lstinline|map(Transformer<? super T, ? extends R>)|
  \item \lstinline|flatMap(Transformer<? super T, ? extends Lazy<? extends R>>)|
\end{enumerate}

Terminal:
\begin{enumerate}[\roman*.]
  \item \lstinline|get()| : \lstinline|T|
  \item \lstinline|equals(Object)| : \lstinline|boolean|
\end{enumerate}
\colbreak

\subsection{Parallelization}
Parallelizing Streams:
\begin{enumerate}[\roman*.]
  \item \lstinline|Collection::parallelStream()|
  \item \lstinline|Stream::parallel()|
\end{enumerate}

Conditions for Parallelization:
\begin{enumerate}[\roman*.]
  \item Non-interference with data source
  \item Avoid side-effects
  \item Stateless lambdas
  \item Prefer unordered (or use \lstinline|.unordered()|)
\end{enumerate}
%%%%%%%%%%%%%%%%%%%%%%%%%%%%%%%%%%%%%%%%%%%%%%%%%%%%%%
%                       End                          %
%%%%%%%%%%%%%%%%%%%%%%%%%%%%%%%%%%%%%%%%%%%%%%%%%%%%%%

\end{multicols*}
\end{document}

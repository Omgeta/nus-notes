\documentclass[12pt, a4paper]{article}

\usepackage[utf8]{inputenc}
\usepackage[mathscr]{euscript}
\let\euscr\mathscr \let\mathscr\relax
\usepackage[scr]{rsfso}
\usepackage{amssymb,amsmath,amsthm,amsfonts}
\usepackage[shortlabels]{enumitem}
\usepackage{multicol,multirow}
\usepackage{lipsum}
\usepackage{balance}
\usepackage{calc}
\usepackage[colorlinks=true,citecolor=blue,linkcolor=blue]{hyperref}
\usepackage{import}
\usepackage{xifthen}
\usepackage{pdfpages}
\usepackage{transparent}
\usepackage{tabularx}

\newcommand{\incfig}[2][1.0]{
    \def\svgwidth{#1\columnwidth}
    \import{./figures/}{#2.pdf_tex}
}
\newcommand{\incimg}[2][1.0]{
  \includegraphics[width=#1\columnwidth]{./figures/#2}
}


\usepackage{ifthen}
\usepackage[landscape]{geometry}
\usepackage[shortlabels]{enumitem}

\ifthenelse{\lengthtest { \paperwidth = 11in}}
    { \geometry{top=.5in,left=.5in,right=.5in,bottom=.5in} }
	{\ifthenelse{ \lengthtest{ \paperwidth = 297mm}}
		{\geometry{top=1cm,left=1cm,right=1cm,bottom=1cm} }
		{\geometry{top=1cm,left=1cm,right=1cm,bottom=1cm} }
	}

\pagestyle{empty}
\makeatletter
\renewcommand\thesection{\arabic{section}.}
\renewcommand{\section}{\@startsection{section}{1}{0mm}%
                                {-1ex plus -.5ex minus -.2ex}%
                                {0.05ex}%x
                                {\normalfont\normalsize\bfseries}}
\renewcommand{\subsection}{\@startsection{subsection}{2}{0mm}%
                                {-1ex plus -.5ex minus -.2ex}%
                                {0.05ex}%
                                {\normalfont\small\bfseries}}
\renewcommand{\subsubsection}{\@startsection{subsubsection}{3}{0mm}%
                                {-1ex plus -.5ex minus -.2ex}%
                                {0.05ex}%
                                {\normalfont\footnotesize\bfseries}}
\newcommand{\colbreak}{\vfill\null\columnbreak}
\makeatother
\setcounter{secnumdepth}{1}
\setlength{\parindent}{0pt}
\setlength{\parskip}{0.7em}

\setlist[itemize]{itemsep=0.6ex, topsep=-2pt, partopsep=0pt, parsep=0pt}
\setlist[enumerate]{itemsep=0.6ex, topsep=-2pt, partopsep=0pt, parsep=0pt}

\input{letterfonts}

\newcommand{\mytitle}{CS2040S Data Structures and Algorithms}
\newcommand{\myauthor}{github/omgeta}
\newcommand{\mydate}{AY 24/25 Sem 2}

\begin{document}
\raggedright
\footnotesize
\begin{multicols*}{3}
\setlength{\premulticols}{1pt}
\setlength{\postmulticols}{1pt}
\setlength{\multicolsep}{1pt}
\setlength{\columnsep}{2pt}

{\normalsize{\textbf{\mytitle}}} \\
{\footnotesize{\mydate\hspace{2pt}\textemdash\hspace{2pt}\myauthor}}
\vspace{-0.5em}
%%%%%%%%%%%%%%%%%%%%%%%%%%%%%%%%%%%%%%%%%%%%%%%%%%%%%%
%                      Begin                         %
%%%%%%%%%%%%%%%%%%%%%%%%%%%%%%%%%%%%%%%%%%%%%%%%%%%%%%
\section{Asymptotic Analysis}
Preconditions are conditions which must be true for the algorithm to work correctly.
Postconditions are conditions guaranteed to be true after the algorithm ends.

Invariants are conditions that remain unchanged and consistently true throughout the algorithm. \\Loop invariants are invariants for each loop iteration.

Runtime $T(n)$ is given if $\exists c, n_0>0$ s.t. $\forall n > n_0$:
\begin{enumerate}[\roman*.]
  \item $T(n) \leq cf(n) \iff T(n)=O(f(n))$
  \item $T(n) \geq cf(n) \iff T(n)=\Omega(f(n))$
  \item $T(n) = O(f(n)), \Omega(f(n))\iff T(n)=\Theta(f(n))$
\end{enumerate}

Order of growth is:\\
$O(1) < O(\log\log n) < O(\log n) < O(\log^2 n) < O(\sqrt{n}) < O(n) < O(n\log n) < O(n^2) < O(2^n) < O(2^{2n}) < O(n!)$

Common recurrences:
\begin{enumerate}[\roman*.]
  \item $T(n) = T(n/2) + O(1) = O(\log n)$
  \item $T(n) = T(n/2) + O(n) = O(n)$
  \item $T(n) = 2T(n/2) + O(1) = O(n)$
  \item $T(n) = 2T(n/2) + O(n) = O(n\log n)$
  \item $T(n) = T(n - 1) + O(1) = O(n)$
  \item $T(n) = T(n - 1) + O(n) = O(n^2)$
  \item $T(n) = 2T(n - 1) + O(1) = O(2^n)$
\end{enumerate}

\subsection{Mathematical Properties}
Useful properties:
\begin{enumerate}[\roman*.]
  \item $\log(xy) = \log x + \log y$\hfill(Product rule) 
  \item $\log(x/y) = \log x - \log y$\hfill(Quotient rule)
  \item $\log(x^k) = k\log x$\hfill(Power rule)
  \item $\log_bx = \frac{\log_ax}{\log_ab}$\hfill(Change of base)
  \item $x^a \cdot x^b = x^{a+b}$\hfill(Product rule) 
  \item $\frac{x^a}{x^b} = x^{a-b}$\hfill(Quotient rule)
  \item $(x^a)^b = x^{ab}$\hfill(Power rule)
\end{enumerate}
\colbreak

\subsection{Master Theorem}

Dividing function $T(n) = aT(n/b) + f(n)$ where $a>1, b>1, f(n) = \Theta(n^k\log^pn)$ has cases:
\begin{enumerate}[\roman*.]
  \item $\log_ba > k \implies T(n)=\Theta(n^{\log_ba})$
  \item $\log_ba = k \land p>-1 \implies T(n) = \Theta(n^k\log^{p+1}n)$\\
    \hspace{4.5em}$\land\hspace{0.25em}p=-1 \implies T(n) = \Theta(n^k\log\log n)$\\
    \hspace{4.5em}$\land\hspace{0.25em}p<-1 \implies T(n) = \Theta(n^k)$
  \item $\log_ba < k \land p \geq 0 \implies T(n) = \Theta(n^k\log^pn)$\\
    \hspace{4.5em}$\land\hspace{0.25em}p<0 \implies T(n) = \Theta(n^k)$
\end{enumerate}

Decreasing function $T(n) = aT(n-b) + f(n)$ where $a>0, b>0$ has cases:
\begin{enumerate}[\roman*.]
  \item $a < 1 \implies T(n)=O(f(n))$
  \item $a = 1 \implies T(n) = O(n\cdot f(n))$
  \item $a > 1 \implies T(n) = O(a^{n/b}\cdot f(n))$
\end{enumerate}

\section{Searching}
Binary search, $O(\log n)$, in a sorted array cuts search space in half until the target is found or range exhausted.
\begin{lstlisting}
int binarySearch(T[] arr, T x) {
  int lo = 0, hi = arr.length - 1;
  while (lo < hi) {
    int mid = lo + (hi - lo) / 2;
    if (arr[mid] < x) lo = mid + 1;
    else hi = mid;
  }
  return lo;
}
\end{lstlisting}
1D Peak Finding, $O(\log n)$:
\begin{enumerate}[\roman*.]
  \item Binary search on side with rising neighbour if not peak
\end{enumerate}

2D Peak Finding on $n\times m$, $O(n\log m)$:
\begin{enumerate}[\roman*.]
  \item Take global max of column, binary search on columns with rising neighbour if not peak.
  \item Optimise: Quadrant DnC. $O(n+m)$. 
\end{enumerate}

$k$th Smallest, $O(n)$:
\begin{enumerate}[\roman*.]
  \item QuickSelect, recurse on half with $k$
\end{enumerate}

\colbreak
\section{Sorting}
BubbleSort repeatedly swaps inverted adjacent elements up to form a globally sorted suffix. 
\begin{enumerate}[\roman*.]
  \item After $i$ iterations, last $i$ elements are correct
  \item Time: $O(n)$ (Best), $O(n^2)$ (Worst) 
  \item In-place and stable
\end{enumerate}

SelectionSort selects the smallest unsorted element and places it in the globally sorted prefix.
\begin{enumerate}[\roman*.]
  \item After $i$ iterations, first $i$ elements are correct
  \item Time: $O(n^2)$ (Best \& Worst)
  \item In-place but unstable
\end{enumerate}

InsertionSort adds elements to a locally sorted prefix.
\begin{enumerate}[\roman*.]
  \item After $i$ iterations, first $i$ elements are sorted
  \item Time: $O(n)$ (Best), $O(n^2)$ (Worst) 
  \item In-place and stable
\end{enumerate}

MergeSort divides the array into halves, sorts each half, and merges the sorted halves.
\begin{enumerate}[\roman*.]
  \item Locally sorted prefix in power of 2
  \item Time: $O(n\log n)$ (Best \& Worst)
  \item Space: $O(n\log n)$
  \item Out-of-place and stable
\end{enumerate}

QuickSort partitions the array around $\leq$ and $>$ pivot, then recursing on both partitions.
\begin{enumerate}[\roman*.]
  \item Elements before pivot are $\leq$, after pivot are $>$
  \item Time: $O(n\log n)$ (Best), $O(n^2)$ (Worst)
  \item In-place but unstable
  \item Optimise: randomized pivot, 3-way partition for $O(n\log n)$ worst-case, insertion sort for small $n$
\end{enumerate}

CountingSort counts number of objects with distinct keys.
\begin{enumerate}[\roman*.]
  \item Time: $O(n + k)$ where $k$ is number of keys
  \item Space: $O(n + k)$
\end{enumerate}

RadixSort sorts integers by individual digits.
\begin{enumerate}[\roman*.]
  \item Time: $O(nd)$ where $d$ is number of digits
  \item Space: $O(n + k)$
\end{enumerate}

\section{Trees}
\subsection{Binary Search Trees (BSTs)}
Binary search tree maintains that for any node, all left descendants are $\leq$ and all right descendants are $>$. Operations are $O(h)$ where $h$ is the height of the tree.

BSTs are balanced if $h = O(\log n)$.\\If balanced, (non-traversal) operations are $O(\log n)$. In the worst case, they are $O(n)$.

\begin{enumerate}[\roman*.]
  \item Height, $h$ is given by $\log n -1 \leq h \leq n$
  \item \lstinline|search|: Traverse left/right based on comparisons.
  \item \lstinline|insert|: Find position via search and add node.
  \item \lstinline|delete|: If not leaf, replace with successor.
  \item Traversal yields sorted elements. Time: $O(n)$
\end{enumerate}


\subsection{AVL Trees}
Self-balancing BSTs which are height-balanced, i.e. height difference between left/right subtrees differ by at most 1. 

\begin{enumerate}[\roman*.]
  \item Balance factor $= |height_{right}-height_{left}| \leq 1$.\\Nodes are left-heavy with balance factor $\leq -1$, or right-heavy with balance factor $\geq 1$.
  \item $h < 2\log n \iff 2^{\frac{h}{2}} \leq n \leq 2^{h+1}-1$ 
  \item \lstinline|insert|: After insertion, walk up tree and fix lowest unbalanced node (max 2 rotations)
  \item \lstinline|delete|: After deletion, fix all unbalanced nodes until root (up to $\Theta(\log n)$ rotations)
\end{enumerate}

To balance left-heavy node \lstinline|v|:
\begin{enumerate}[\roman*.]
  \item \lstinline|v.left| balanced or left-heavy\\$\implies$ \lstinline|rightRotate(v)|
  \item \lstinline|v.left| right-heavy\\$\implies$ \lstinline|leftRotate(v.left); rightRotate(v)|
\end{enumerate}

To balance right-heavy node \lstinline|v|:
\begin{enumerate}[\roman*.]
  \item \lstinline|v.right| balanced or right-heavy\\$\implies$ \lstinline|leftRotate(v)|
  \item \lstinline|v.right| left-heavy\\$\implies$ \lstinline|rightRotate(v.right); leftRotate(v)|
\end{enumerate}

\colbreak
\subsubsection{Rotations}
\incimg[0.8]{right_rotation}
\begin{minipage}{0.45\columnwidth}
  \begin{lstlisting}
Node leftRotate(B)
  D = B.right
  D.par = B.par
  B.par = D
  B.right = D.left
  D.left = B
  return D
  \end{lstlisting}
\end{minipage}
\begin{minipage}{0.5\columnwidth}
  \begin{lstlisting}
Node rightRotate(D)
  B = D.left
  B.par = D.par
  D.par = B
  D.left = B.right
  B.right = D
  return B
  \end{lstlisting}
\end{minipage}
\vspace{-2em}
\subsubsection{Problems}
\vspace{-0.5em}
Order Statistics, $O(\log n)$ if balanced
\begin{enumerate}[\roman*.]
  \item Augment nodes with subtree size $= size_{left} + size_{right} + 1$
  \item \lstinline|select|: find kth smallest element.
  \item \lstinline|rank|: get node position in sorted order.
\end{enumerate}

Counting Inversions, $O(n\log n)$:
\begin{enumerate}[\roman*.]
  \item After inserting each element, add \lstinline|i - tree.rank(ele)| to running count.
\end{enumerate}

Interval Query, $O(\log n)$
\begin{enumerate}[\roman*.]
  \item Nodes store interval, key is lower interval bound
  \item Augment nodes with subtree max upper bound
  \item \lstinline|interval|: If left subtree exists with $max \geq x.low$, recurse on left subtree, else on right subtree
\end{enumerate}

1D Range Query, $O(\log n + k)$ for $k$ elements in range:
\begin{enumerate}[\roman*.]
  \item Leaves are elements. Internal nodes are max of left subtree.
  \item \lstinline|range|: Find split node $O(\log n)$ and traverse leafs
\end{enumerate}

2D Range Query, $O(\log^2 n + k)$ for $k$ elements in range:
\begin{enumerate}[\roman*.]
  \item Build 1D x-tree for each x, then for each internal node, build y-tree for all y
  \item \lstinline|range|: $O(\log n)$ y-tree searches each of $O(\log n)$ 
\end{enumerate}

\colbreak
\subsection{Tries}
Tree storing strings where each node is a character, and each path is a key. Operations are $O(L)$, where $L$ is length of the key. Faster than $O(Lh)$ of strings in bBST. Both use $O(\text{text size})$ space but Trie has node overhead.
\begin{enumerate}[\roman*.]
\item \lstinline|prefixSearch|: Find all keys with prefix.\\Time: $O(L + k)$, where $k$ matches
\end{enumerate}

\subsection{(a,b)-Trees}
\incimg{ab_tree}
Self-balancing search trees where non-root nodes have between $a, b$ children for $2 \leq a \leq \frac{b+1}{2}$
\begin{enumerate}[\roman*.]
  \item Root has $[2, b]$ children.\\Internal nodes have $[a, b]$ children.
  \item Nodes with $k$ children have $k-1$ keys.\\Leaves have $[a-1,b-1]$ keys.
  \item All leaves at same depth
  \item \lstinline|search|: Time: $O(b\log_a n) = O(\log n)$
  \item \lstinline|insert|: Proactively split full nodes ($b-1$ keys) during search, guaranteeing parent of split node won\'t have excess keys after insertion of split node.\\Time: $O(\log n)$
  \item \lstinline|split|: Split about median node $z$ with $\geq 2a$ keys, s.t. left has $\geq a-1$ keys and right has $\geq a$ keys. Then, $z$ offered to parent with $\leq b-2$ keys.\\
    Time: $O(b)$ for splitting node with $b$ children
  \item \lstinline|delete|: Passively, delete node then recursively check parents for violation, carrying out merge/share with smallest adjacent sibling. For internal nodes, replace with successor.\\Time: $O(\log n)$ 
  \item \lstinline|merge|: Demote split node and join when $< b-1$ keys. Time: $O(b)$ 
  \item \lstinline|share|: \lstinline|merge| then \lstinline|split| when $\geq b-1$ keys
\end{enumerate}

%%%%%%%%%%%%%%%%%%%%%%%%%%%%%%%%%%%%%%%%%%%%%%%%%%%%%%
%                       End                          %
%%%%%%%%%%%%%%%%%%%%%%%%%%%%%%%%%%%%%%%%%%%%%%%%%%%%%%
\end{multicols*}
\end{document}

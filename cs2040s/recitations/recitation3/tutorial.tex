\documentclass[12pt, a4paper]{article}

\usepackage[a4paper, margin=1in]{geometry}

\usepackage[utf8]{inputenc}
\usepackage[mathscr]{euscript}
\let\euscr\mathscr \let\mathscr\relax
\usepackage[scr]{rsfso}
\usepackage{amssymb,amsmath,amsthm,amsfonts}
\usepackage[shortlabels]{enumitem}
\usepackage{multicol,multirow}
\usepackage{lipsum}
\usepackage{balance}
\usepackage{calc}
\usepackage[colorlinks=true,citecolor=blue,linkcolor=blue]{hyperref}
\usepackage{import}
\usepackage{xifthen}
\usepackage{pdfpages}
\usepackage{transparent}
\usepackage{tabularx}

\newcommand{\incfig}[2][1.0]{
    \def\svgwidth{#1\columnwidth}
    \import{./figures/}{#2.pdf_tex}
}
\newcommand{\incimg}[2][1.0]{
  \includegraphics[width=#1\columnwidth]{./figures/#2}
}


\input{letterfonts}

\newcommand{\mytitle}{CS2040S Recitation 3}
\newcommand{\myauthor}{github/omgeta}
\newcommand{\mydate}{AY 24/25 Sem 2}

\begin{document}
\raggedright
\footnotesize
\begin{center}
{\normalsize{\textbf{\mytitle}}} \\
{\footnotesize{\mydate\hspace{2pt}\textemdash\hspace{2pt}\myauthor}}
\end{center}
\setlist{topsep=-1em, itemsep=-1em, parsep=2em}
%%%%%%%%%%%%%%%%%%%%%%%%%%%%%%%%%%%%%%%%%%%%%%%%%%%%%%
%                      Begin                         %
%%%%%%%%%%%%%%%%%%%%%%%%%%%%%%%%%%%%%%%%%%%%%%%%%%%%%%
\begin{enumerate}[Q\arabic*.]
  \item 
    \begin{enumerate}[(\alph*.)]
      \item Use a MergeSort algorithm. Our merge step is only $O(n)$ because when we assume smaller subarrays are sorted, we only need at most $n$ cost for reversal. Therefore, our final time complexity is $O(n\log n)$

      \item Use the binary algorithm as a partition algorithm by considering each element as $<$ or $\geq$ the pivot. We have $T(n) = 2T(\frac{n}{2}) + O(n\log n) = O(n\log^2 n)$

      \item Implement a 3-way partition using the binary partition twice. Once to split between $<$ and $\geq$. Second time to split between $=$ and $>$. 
    \end{enumerate}

  \item 
    \begin{enumerate}[(\alph*.)]
      \item We want permutations which are random. That is, each of the $n!$ permutations must have probability exactly $\frac{1}{n!}$

      \item Create an new array and for each index, choose a random element in the original. Time complexity $O(n)$. Space complexity $O(n)$. 

      \item No, it does not have a uniform distribution.

      \item It maintains a prefix of $i$ randomly sorted elements. This produces good permutations.

      \item Probability of an element remaining in its place $= \frac{(n-1)!}{n!} = \frac{1}{n}$. Expected number of elements in its same position $= n \cdot \frac{1}{n} = 1$

      \item Not without modifications. If we fail to get elements out of their own position, try again.

      \item Better algorithm for this situation does not use truly random permutations to ensure expected number of students with their own assignment is $0$.
    \end{enumerate}
\end{enumerate}
%%%%%%%%%%%%%%%%%%%%%%%%%%%%%%%%%%%%%%%%%%%%%%%%%%%%%%
%                       End                          %
%%%%%%%%%%%%%%%%%%%%%%%%%%%%%%%%%%%%%%%%%%%%%%%%%%%%%%

\end{document}

\documentclass[12pt, a4paper]{article}

\usepackage[a4paper, margin=1in]{geometry}

\usepackage[utf8]{inputenc}
\usepackage[mathscr]{euscript}
\let\euscr\mathscr \let\mathscr\relax
\usepackage[scr]{rsfso}
\usepackage{amssymb,amsmath,amsthm,amsfonts}
\usepackage[shortlabels]{enumitem}
\usepackage{multicol,multirow}
\usepackage{lipsum}
\usepackage{balance}
\usepackage{calc}
\usepackage[colorlinks=true,citecolor=blue,linkcolor=blue]{hyperref}
\usepackage{import}
\usepackage{xifthen}
\usepackage{pdfpages}
\usepackage{transparent}
\usepackage{tabularx}

\newcommand{\incfig}[2][1.0]{
    \def\svgwidth{#1\columnwidth}
    \import{./figures/}{#2.pdf_tex}
}
\newcommand{\incimg}[2][1.0]{
  \includegraphics[width=#1\columnwidth]{./figures/#2}
}


\input{letterfonts}

\newcommand{\mytitle}{CS2040S Tutorial 1}
\newcommand{\myauthor}{github/omgeta}
\newcommand{\mydate}{AY 24/25 Sem 2}

\begin{document}
\raggedright
\footnotesize
\begin{center}
{\normalsize{\textbf{\mytitle}}} \\
{\footnotesize{\mydate\hspace{2pt}\textemdash\hspace{2pt}\myauthor}}
\end{center}
\setlist{topsep=-1em, itemsep=-1em, parsep=2em}
%%%%%%%%%%%%%%%%%%%%%%%%%%%%%%%%%%%%%%%%%%%%%%%%%%%%%%
%                      Begin                         %
%%%%%%%%%%%%%%%%%%%%%%%%%%%%%%%%%%%%%%%%%%%%%%%%%%%%%%
\begin{enumerate}[Q\arabic*.]
  \item 
    \begin{enumerate}[(\alph*.)]
      \item A class is a blueprint for specific instances/object which share similar interfaces and fields. $\qed$

      \item \lstinline|main| can be called without needing to instantiate an object of the class. $\qed$

      \item Example: 
        \begin{lstlisting}
public class Box {
    private static x = 3;
}

public class Main {
    public static void main(String[] args) {
      System.out.println(Box.x) // Accessing a private variable outside of the class
    }
} 
        \end{lstlisting}

      \item Interfaces are used to define an abstract public interface for similar classes. Example:
        \begin{lstlisting}
public interface Runnable {
    void run();
}

public class Car implements Runnable {
    public void run() {
      // implementation here
    }
}

public class WashingMachine implements Runnable {
    public void run() {
      // implementation here
    }
}
        \end{lstlisting}
        Yes; we can return objects with an interface type.

      \item The final value of $j$ will be $8$ but the final value of $i, k$ will still be $7$. In \lstinline|addOne|, the $int$ value is passed by value and any changes do not actually affect the original value. In \lstinline|myOtherIntAddOne|, $k$ is only a variable holding a reference to the original and reassignment only reassigns where the variable points to and does the change the original $k$. $\qed$

      \item Yes, but the parameter name will shadow the unqualified member name. To still access the member/static variable, use a qualified name like \lstinline|this.x| (for member) or \lstinline|Main.x| (for static) $\qed$
    \end{enumerate}

  \item 
    \begin{enumerate}[(\alph*.)]
      \item $O(n^3) \qed$
      \item $O(n^2 \log n) \qed$
      \item $O(n^5) \qed$
      \item $O(2^{n^2}) \qed$
    \end{enumerate}

  \item 
    \begin{enumerate}[(\alph*.)]
      \item $O(n) \qed$
      \item $O(n\log n) \qed$
      \item $O(\log n) \qed$
      \item $O(n^{\log n}) \qed$
    \end{enumerate}

  \item Naive solution: iterate through each value in the array checking if the next increments by $1$\\
    Fast solution: use binary search comparing value at $mid$ with its expected index choosing left or right appropriately $\qed$

  \item Use binary search with the initial values of $low$ and $high$ being $1$ and the pile taking the most time. At each step, find $mid$ and check if its possible to finish within $h$ hours using a helper function \lstinline|isFeasible(piles, k, h)|. If it is possible, set $low = mid$ else $high = mid$ and continue iterating until $low$ and $high$ converge on a single value which is the minimum time spent. $\qed$

  \item $O(n)$ solution: iterate through each point recording the maximum and minimum $x,y$ values. At the end, the bounding box is simply $(minX, minY), (minX, maxY), (maxX, maxY), (maxX, maxY) \qed$
\end{enumerate}
%%%%%%%%%%%%%%%%%%%%%%%%%%%%%%%%%%%%%%%%%%%%%%%%%%%%%%
%                       End                          %
%%%%%%%%%%%%%%%%%%%%%%%%%%%%%%%%%%%%%%%%%%%%%%%%%%%%%%

\end{document}

\documentclass[12pt, a4paper]{article}

\usepackage[utf8]{inputenc}
\usepackage[mathscr]{euscript}
\let\euscr\mathscr \let\mathscr\relax
\usepackage[scr]{rsfso}
\usepackage{amssymb,amsmath,amsthm,amsfonts}
\usepackage[shortlabels]{enumitem}
\usepackage{multicol,multirow}
\usepackage{lipsum}
\usepackage{balance}
\usepackage{calc}
\usepackage[colorlinks=true,citecolor=blue,linkcolor=blue]{hyperref}
\usepackage{import}
\usepackage{xifthen}
\usepackage{pdfpages}
\usepackage{transparent}
\usepackage{tabularx}

\newcommand{\incfig}[2][1.0]{
    \def\svgwidth{#1\columnwidth}
    \import{./figures/}{#2.pdf_tex}
}
\newcommand{\incimg}[2][1.0]{
  \includegraphics[width=#1\columnwidth]{./figures/#2}
}


\usepackage{ifthen}
\usepackage[landscape]{geometry}
\usepackage[shortlabels]{enumitem}

\ifthenelse{\lengthtest { \paperwidth = 11in}}
    { \geometry{top=.5in,left=.5in,right=.5in,bottom=.5in} }
	{\ifthenelse{ \lengthtest{ \paperwidth = 297mm}}
		{\geometry{top=1cm,left=1cm,right=1cm,bottom=1cm} }
		{\geometry{top=1cm,left=1cm,right=1cm,bottom=1cm} }
	}

\pagestyle{empty}
\makeatletter
\renewcommand\thesection{\arabic{section}.}
\renewcommand{\section}{\@startsection{section}{1}{0mm}%
                                {-1ex plus -.5ex minus -.2ex}%
                                {0.05ex}%x
                                {\normalfont\normalsize\bfseries}}
\renewcommand{\subsection}{\@startsection{subsection}{2}{0mm}%
                                {-1ex plus -.5ex minus -.2ex}%
                                {0.05ex}%
                                {\normalfont\small\bfseries}}
\renewcommand{\subsubsection}{\@startsection{subsubsection}{3}{0mm}%
                                {-1ex plus -.5ex minus -.2ex}%
                                {0.05ex}%
                                {\normalfont\footnotesize\bfseries}}
\newcommand{\colbreak}{\vfill\null\columnbreak}
\makeatother
\setcounter{secnumdepth}{1}
\setlength{\parindent}{0pt}
\setlength{\parskip}{0.7em}

\setlist[itemize]{itemsep=0.6ex, topsep=-2pt, partopsep=0pt, parsep=0pt}
\setlist[enumerate]{itemsep=0.6ex, topsep=-2pt, partopsep=0pt, parsep=0pt}

\input{letterfonts}

\newcommand{\mytitle}{CS2100 Computer Organisation}
\newcommand{\myauthor}{github/omgeta}
\newcommand{\mydate}{AY 24/25 Sem 2}

\begin{document}
\raggedright
\footnotesize
\begin{multicols*}{3}
\setlength{\premulticols}{1pt}
\setlength{\postmulticols}{1pt}
\setlength{\multicolsep}{1pt}
\setlength{\columnsep}{2pt}

{\normalsize{\textbf{\mytitle}}} \\
{\footnotesize{\mydate\hspace{2pt}\textemdash\hspace{2pt}\myauthor}}
%%%%%%%%%%%%%%%%%%%%%%%%%%%%%%%%%%%%%%%%%%%%%%%%%%%%%%
%                      Begin                         %
%%%%%%%%%%%%%%%%%%%%%%%%%%%%%%%%%%%%%%%%%%%%%%%%%%%%%%

\section{Number Systems}

Weighted-positional number systems with base-$R$, has each position weighted in powers of $R$.
\begin{enumerate}[\roman*.]
  \item $N$ positions give $R^N$ values
  \item $M$ values need $\ceil{\log_RM}$ positions
\end{enumerate}

\subsection{Conversion}

Base-$R$$\rightarrow$Decimal: repeated multiply by $R^{n-1}$

Decimal$\rightarrow$Binary (Generalised for Base-$R$):
\begin{enumerate}[\roman*.]
  \item Whole numbers: repeated division by 2, $LSB \rightarrow MSB$
  \item Fractional: repeated multiplication by 2, $MSB \rightarrow LSB$
\end{enumerate}

Binary$\rightarrow$Octal: partition in groups of 3

Octal$\rightarrow$Binary: convert each digit to 3 bits

Binary$\rightarrow$Hexadecimal: partition in groups of 4

Hexadecimal$\rightarrow$Binary: convert each digit to 4 bits

\subsection{C Programming}

ASCII Chars are represented with $7$ bits and $1$ parity bit

\lstinline|char| is 1 byte, \lstinline|int + float| are 4 bytes, \lstinline|long + double| are 8 bytes.

Functions only modify \lstinline|struct| if passed as pointer or in array with: \lstinline|(*object_p).a = 1;|, \lstinline|object_p->a = 1;|

Arrays in \lstinline|struct|s are deep-copied (but not pointers to arrays).

\colbreak
\subsection{Negative Numbers}

Sign-and-Magnitude:
\begin{enumerate}[\roman*.]
  \item MSB is the sign bit ($0$ is positive)
  \item Range: $-(2^{N-1}-1)$ to $2^{N-1}-1$
  \item Negation: Flip MSB
\end{enumerate}

1s Complement (useful for arithmetic):
\begin{enumerate}[\roman*.]
  \item Negated $x$, $-x = 2^N- x -1$
  \item Range: $-(2^{N-1}-1)$ to $2^{N-1}-1$
  \item Negation: Invert bits
  \item Overflow: add carry to result, wrap around
  \item $(b-1)s$: $-x = b^N- x -1$
\end{enumerate}

2s Complement (useful for arithmetic):
\begin{enumerate}[\roman*.]
  \item Negated $x$, $-x = 2^N- x$
  \item Range: $-2^{N-1}$ to $2^{N-1}-1$
  \item Negation: Invert bits, then add $1$
  \item Overflow: truncate 
  \item $bs$: $-x = b^N- x$
\end{enumerate}

Excess-$M$ (useful for comparisons):
\begin{enumerate}[\roman*.]
  \item Start at $-M = -2^{N-1}$, for $N$ bits.
  \item Range: $-2^{N-1}$ to $2^{N-1}-1$
  \item Express $x$ in Excess-$N$: $x + N$
\end{enumerate}

\subsection{Real Numbers}

Fixed-Point (limited range):
\begin{enumerate}[\roman*.]
  \item Reserve bits for whole numbers and for fraction, converting with 1s or 2s complement
\end{enumerate}

IEEE 754 Floating-Point (more complex):
\begin{enumerate}[\roman*.]
  \item Sign 1-bit (0 is positive)
  \item Exponent 8-bits (excess-127)
  \item Mantissa 23-bits (normalised to $1.m \times 2^{exp}$)
\end{enumerate}
\incimg[0.75]{ieee754}
\colbreak

\section{ISA}

Instruction Set Architecture (ISA) defines instructions for how software controls the hardware. 

von Neumann Architecture:
\begin{enumerate}[\roman*.]
  \item Processor: Perform computations.
  \item Memory: Stores code and data (stored memory).
  \item Bus: Bridge between processor and memory.
\end{enumerate}

Storage Architectures:
\begin{enumerate}[\roman*.]
  \item Stack: Operands are implicitly on top of the stack.
  \item Accumulator: One operand is implicitly in the accumulator.
  \item Memory-Memory: All operands in memory.
  \item Register-Register: All operands in registers.
\end{enumerate}

Endianness (ordering of bytes in a word):
\begin{enumerate}[\roman*.]
  \item Big Endian: MSB in lowest address
  \item Little Endian: LSB in lowest address
\end{enumerate}

Instruction Length:
\begin{enumerate}[\roman*.]
  \item Fixed-Length: easy fetch and decode, simplified pipelining, instruction bits are scarce
  \item Variable-Length: more flexible
\end{enumerate}

Instruction Encoding, under fixed-length instructions involves extending opcode for instruction types with unused bits:
\begin{enumerate}
  \item Minimise: maximise opcode range of instruction type with least opcode bits
  \item Maximise: minimise opcode range of instruction type with least opcode bits
\end{enumerate}

Shortcut for Maximise given $A > B > C$ instruction types, where $|X|$ is the minimum number of instruction $X$ = $2^A - |B|2^{A-B} - |C|2^{A-C} + |B| + |C|$

\colbreak

\section{MIPS}

MIPS Assembly Language:
\begin{enumerate}[\roman*.]
  \item Mainly Register-Register, Fixed-Length
  \item $32$ registers, each $32$-bits ($4$-byte) long
  \item $2^{30}$ words contains $32$-bits ($4$-byte) long 
  \item Memory addresses are $32$-bits long
\end{enumerate}

Arithmetic Instructions:
\begin{enumerate}[\roman*.]
  \item $C16, C5$ are $16, 5$ bit patterns
  \item $C16_{2S}$ is a sign-extended 2's complement
  \item NOT: \lstinline|nor| with \lstinline|$zero|, or \lstinline|xor| with all 1s
  \item Large constants: \lstinline|lui| (31:16 bits) + \lstinline|ori| (15:0 bits)
\end{enumerate}
\incimg{mips_basic}
Memory Instructions:
\begin{enumerate}[\roman*.]
  \item \lstinline|lw $rt, offset($rs)|: load contents to \lstinline|$rt|
  \item \lstinline|sw $rt, offset($rs)|: store contents to \lstinline|$rs|
  \item Use \lstinline|lb, sb| for loading bytes (like chars)
\end{enumerate}

Control Instructions:
\begin{enumerate}[\roman*.]
  \item \lstinline|beq/bne $rs, $rt, label/imm|: jump if $=$/$\neq$.
  \item Branch: adds \lstinline|imm x 4| to \lstinline|$PC = $PC + 4| on branch. Range: $\pm 2^{15}$ instructions
  \item \lstinline|j label/imm|: jump unconditionally.
  \item Pseudo-direct: takes 4 MSBs from \lstinline|$PC + 4|, and 2 LSB omitted since instructions are word-aligned. Range: $256$-byte boundary of \lstinline|$PC + 4| 4 MSBs
  \item \lstinline|slt $rd, $rs, $rt|: set \lstinline|$rd| to 1 if \lstinline|$rs < $rt|
\end{enumerate}
\colbreak

\subsection{Datapath}
\incimg[0.9]{datapath_cycle}

Clock:
\begin{enumerate}[\roman*.]
  \item Single-Cycle: Read + Compute are done within clock period. PC Write on rising clock edge. Time taken depends on longest instruction.
\incimg[0.8]{singlecycle}
  \item Multicycle: Break up execution into multiple steps (e.g. fetch, decode, execute, memory, writeback), with each step taking one cycle. Time taken depends on slowest step.\\
\incimg[0.9]{multicycle}
\end{enumerate}

Critical Paths:
\begin{enumerate}[\roman*.]
  \item R-type/Immediate:\\
    Inst Mem$\rightarrow$RegFile$\rightarrow$MUX (ALUSrc)\\$\rightarrow$ALU$\rightarrow$MUX (MemToReg)$\rightarrow$RegFile
  \item \lstinline|lw/sw|:\\
    Inst Mem$\rightarrow$RegFile$\rightarrow$ALU$\rightarrow$DataMem$\rightarrow$MUX (MemToReg)$\rightarrow$RegFile
  \item \lstinline|beq|:\\
    Inst Mem$\rightarrow$RegFile$\rightarrow$MUX (ALUSrc)$\rightarrow$ALU$\rightarrow$AND$\rightarrow$MUX (PCSrc)
\end{enumerate}

\colbreak
\subsection{Control}
Control Signals:
\begin{enumerate}[\roman*.]
  \item \lstinline|RegDst| @Decode: Selects destination register.\\
    (0) $\rightarrow$ \lstinline|$rt|
    (1) $\rightarrow$ \lstinline|$rd|
  \item \lstinline|RegWrite| @Decode: Enable register write.\\
    (0) $\rightarrow$ no write
    (1) $\rightarrow$ \lstinline|WD| written to \lstinline|WR|
  \item \lstinline|ALUSrc| @ALU: Second 2nd ALU operand.\\ 
    (0) $\rightarrow$ \lstinline|RD2|
    (1) $\rightarrow$ \lstinline|SignExt(Imm)|
  \item \lstinline|ALUcontrol| @ALU: Second ALU operation.\\
    2-bit \lstinline|ALUop| from opcode + funct for 4-bits
  \item \lstinline|MemRead| @Memory: Enable memory read.\\
    (0) $\rightarrow$ no read
    (1) $\rightarrow$ read \lstinline|Addr.| into \lstinline|Read Data|
  \item \lstinline|MemWrite| @Memory: Enable memory write.\\
    (0) $\rightarrow$ no write
    (1) $\rightarrow$ write \lstinline|Write Data| into \lstinline|Addr.|
  \item \lstinline|MemToReg| @Writeback: Select writeback result.\\
    (0) $\rightarrow$ \lstinline|ALU result|
    (1) $\rightarrow$ Memory \lstinline|Read data|
  \item \lstinline|Branch| @Memory: Select next \lstinline|$PC|.\\
    (0) $\rightarrow$ \lstinline|$PC + 4|
    (1) $\rightarrow$ \lstinline|($PC + 4) + (imm x 4)|
\end{enumerate}

\incimg[0.9]{alucontrolunit}
\subsubsection{1-bit ALU}
\incimg{1bitalu}

\subsection{Pipelining}

Pipelining improves throughput of workload by running instructions in parallel, with each pipeline stage taking one cycle with no overlapping of same stage.

\incimg{pipeline}
Pipelining Stages:
\begin{enumerate}[\roman*.]
  \item \textbf{IF} (Fetch): Fetch instruction, \lstinline|PC + 4|
  \item \textbf{ID} (Decode): Opcode, register file read 
  \item \textbf{EX} (Execute): ALU execution or branch calc
  \item \textbf{MEM} (Memory Access): \lstinline|lw/sw| access memory
  \item \textbf{WB} (Writeback): Write result to register file
\end{enumerate}

Pipeline Registers:
\begin{enumerate}[\roman*.]
  \item \textbf{IF/ID:} Instruction, PC+4
  \item \textbf{ID/EX:} RD1/2, Imm, PC+4, WriteReg
  \item \textbf{EX/MEM:} ALU result, (PC+4)+(imm$\times 4$), RD2, WriteReg
  \item \textbf{MEM/WB:} ALU result, Mem Read Data, WriteReg
\end{enumerate}

\subsection{Performance}

Time for $I$ instructions, for $T_k$ stage time, under $N$ stages:
\begin{enumerate}[\roman*.]
  \item Single-Cycle:
    \begin{itemize}[leftmargin=*]
    \item $CT_{seq} = max(\sum^N_k=1 T_k)$
    \item $Time_{seq} = I \times CT_{seq}$
  \end{itemize}
  \item Multi-Cycle:
  \begin{itemize}[leftmargin=*]
    \item $CT_{multi} = max(T_k)$
    \item $Time_{multi} = I \times Average CPI \times CT_{multi}$
  \end{itemize}
  \item Pipeline:
  \begin{itemize}[leftmargin=*]
    \item $CT_{pipeline} = max(T_k) + t_d$, with overhead $t_d$
    \item $Time_{pipeline} = Cycles \times CT_{pipeline}$
    \item Ideal Cycles = $I + N - 1$
  \end{itemize}
\end{enumerate}

Ideally, speedup over single-cycle is $\frac{Time_{seq}}{Time_{pipeline}} \approx N$


\colbreak

\subsection{Hazards}
\textbf{Structural:} Conflict in shared resource (e.g. MEM)

\textbf{Data:} Data dependencies (e.g. read-after-write; RAW)
\begin{enumerate}[\roman*.]
  \item Forwarding: EX/MEM or MEM/WB $\rightarrow$ EX
  \item Stalling: For \lstinline|lw| delay
\end{enumerate}

\textbf{Control:} Branch taken not known till EX stage, worst case 3-cycle stall
\begin{enumerate}[\roman*.]
  \item Early branch: Branch check at ID for 1-cycle stall
  \item Prediction: Assume all branches not taken, flush pipeline if wrong
  \item Delayed branch: Move instructions executed regardless of branch outcome into branch delay slots (1 per stall) 
\end{enumerate}

\subsubsection{Forwarding Paths}
RAW:
\begin{enumerate}[\roman*.]
  \item EX/MEM$\rightarrow$EX
  \item EX/MEM$\rightarrow$MEM (\lstinline|xx|$\rightarrow$\lstinline|sw|)
\end{enumerate}

Memory:
\begin{enumerate}[\roman*.]
  \item MEM$\rightarrow$MEM (\lstinline|lw|$\rightarrow$\lstinline|sw|)
\end{enumerate}

Early Branch:
\begin{enumerate}[\roman*.]
  \item EX/MEM$\rightarrow$ID 
\end{enumerate}

\subsection{Common Delays}

Without forwarding:
\begin{tabular}{|l|c|}
\hline
\textbf{Instruction} & \textbf{Delay} \\
\hline
RAW & $+2$ \\
Load Word & $+2$ \\
Branch at MEM & $+3$ \\
Branch at ID & $+1$ \\
\hline
\end{tabular}

With forwarding:
\begin{tabular}{|l|c|}
\hline
\textbf{Instruction} & \textbf{Delay} \\
\hline
RAW & $+0$ \\
Load Word & $+1$ \\
Branch at MEM & $+3$ \\
Branch at ID & $+1$ \\
RAW + Branch at ID & $+1$ \\
Load + Branch at ID & $+2$ \\
\hline
\end{tabular}
\vspace{-1em}
\colbreak
\section{Boolean Algebra}

Standard forms:
\begin{enumerate}[\roman*.]
  \item Literal: boolean variable or it's complement
  \item Product term: logical product (AND) of literals
  \item Sum term: logical sum (OR) of literals
  \item Sum-of-Products: logical sum of product terms
  \item Product-of-Sums: logical product of sum terms
\end{enumerate}

Canonical forms:
\begin{enumerate}[\roman*.]
  \item Minterm (of $n$ variables): prod. term of $n$ literals. Denoted $m0$ to $m[2^n-1]$; e.g. $m3  = x'\cdot y\cdot z$
  \item Maxterm (of $n$ variables): sum term of $n$ literals; Denoted $M0$ to $M[2^n-1]$; e.g. $M3 = X+ Y'+ Z'$
  \item Sum-of-Minterms: canonical SOP expression\\
    e.g. $\sum m(0,1,2,3) = m0 + m1 + m2 + m3$
  \item Product-of-Maxterms: canonical POS expression
    e.g. $\Pi M(4,5,6,7) = M4 \cdot M5 \cdot M6 \cdot M7$
\end{enumerate}

\subsection{Logic Gates}
{\centering
\incimg[0.9]{gates}
\par}
Complete sets of logic are sufficient for any boolean function: \{AND, OR, NOT\}, \{NAND\}, \{NOR\}

SOP Circuit: 2-level AND-OR, or 2-level NAND.\\
POS Circuit: 2-level OR-AND, or 2-level NOR. 

\subsection{Karnaugh Maps}

Implicants are product terms covering minterms:
\begin{enumerate}
  \item Prime implicant (PI): implicant of maximum size
  \item Essential prime implicant (EPI): PI including minterm not covered by any other PI
\end{enumerate}



\colbreak
\section{Combinational Circuits}
Combinational circuit outputs depend only on inputs.

\subsubsection{Half Adder}
{\centering\incimg[0.7]{halfadder}\par}
\vspace{-3em}
\begin{gather*}
  C = X\cdot Y\\
  S = X'\cdot Y + X\cdot Y' = X \oplus Y
\end{gather*}
\vspace{-2em}
\subsubsection{Full Adder}
{\centering\incimg[0.5]{fulladder}\par}
\vspace{-3em}
\begin{gather*}
  C = X\cdot Y + X\cdot Z + Y\cdot Z = X\cdot Y + (X \oplus Y)\cdot Z\\
  S = X\oplus Y\oplus Z\\
  = X'\cdot Y\cdot Z + X\cdot Y'\cdot Z + X\cdot Y\cdot Z' + X\cdot Y\cdot Z
\end{gather*}
\vspace{-2em}
\subsubsection{Magnitude Comparator}
Let $A = A_3A_2A_1A_0,  B = B_3B_2B_1B_0; x_i = A_i\cdot B_i + A_i'\cdot B_i'$
{\centering\incimg{magnitudecomparator}\par}
\vspace{-3em}
\colbreak
\subsubsection{Circuit Delays}
Inputs with delays $t_1, \cdots, t_n$ into gate with delay $t$ will have delay $max(t_1, \cdots, t_n)+t$.

Delay of $n$-bit parallel adder, $t$ is delay of full adders:
\begin{align*}
  S_n = ((n-1)2 + 2)t,\quad C_{n+1} = ((n-1)2 + 3)t
\end{align*}
\subsection{MSI Components}
\subsubsection{Decoders ($n$ to $2^n$)}
Output line are minterms, and circuits can be represented with sum of decoder outputs.\\
{\centering\incimg[0.8]{decoder}\par}
\vspace{-1em}
\subsubsection{Encoders ($2^n$ to $n$)}
Priority encoders allow multiple inputs on, with highest priority taking precedence.\\
{\centering\incimg[0.8]{encoder}\par}
\vspace{-1em}
\subsubsection{Demultiplexer (data decoders)}
{\centering\incimg[0.8]{demux}\par}
\vspace{-1em}
\subsubsection{Multiplexer (data encoders)}
{\centering\incimg[0.8]{mux}\par}
\vspace{-1em}
\colbreak
\section{Sequential Circuits}
Sequential circuit outputs depend on inputs and state:
\begin{enumerate}[\roman*.]
  \item Latches pulse-triggered; flip-flops edge-triggered
  \item Gated circuits are enable-input
  \item Self-correcting if unused states $\rightarrow$ used states
\end{enumerate}

\subsubsection{SR Flip-Flop}
\incimg[0.9]{sr}
\vspace{-1em}
\subsubsection{D Flip-Flop}
\incimg[0.9]{d}
\vspace{-1em}
\subsubsection{JK Flip-Flop}
\incimg[0.9]{jk}
\vspace{-1em}
\subsubsection{T Flip-Flop}
\incimg[0.9]{t}

\subsubsection{Excitation}
\incimg[0.9]{excitation}
\vspace{-1em}
\colbreak
\section{Cache}
Caches reduce frequency of slow direct memory access in DRAM (slow \& cheap) by using SRAM (fast \& expensive).
\begin{enumerate}[\roman*.]
  \item Hit: data in cache 
    \begin{itemize}[leftmargin=*]
      \item Rate: fraction of accesses that are hits
      \item Time: time to access cache data
    \end{itemize}
  \item Miss: data not in cache
    \begin{itemize}[leftmargin=*]
      \item Penalty: time to replace cache block + hit time
    \end{itemize}
\end{enumerate}
{\centering\incimg[0.75]{avgaccesstime}\par}
Principle of Locality:
\begin{enumerate}[\roman*.]
  \item Temporal: Same data likely re-accessed soon.
  \item Spatial: Nearby data likely be assessed soon.
\end{enumerate}

Types of Misses:
\begin{enumerate}[\roman*.]
  \item Cold: Due to never being previously allocated.
  \item Conflict: In direct mapped/set associative cache; due to previous eviction by different block. 
  \item Capacity: In fully associative cache; due to previous eviction as a result of small cache size. 
\end{enumerate}
\vspace{-0.75em}
\subsection{Policies}
Read Miss Policy:
\begin{enumerate}[\roman*.]
  \item Load data into cache, then from cache to register
\end{enumerate}

Write Policy:
\begin{enumerate}[\roman*.]
  \item Write-through: Write to both cache and memory.
  \item Write-back: Write to cache and set dirty bit; write to memory on block eviction and reset dirty bit
\end{enumerate}

Write Miss Policy (data to write not in cache):
\begin{enumerate}[\roman*.]
  \item Write Allocate: Cache data and follow write policy. 
  \item Write Around: Write directly to memory only.
\end{enumerate}

Block Replacement Policy:
\begin{enumerate}[\roman*.]
  \item Least Recently Used (LRU; most common)
  \item First In First Out (FIFO)
  \item Random Replacement (RR)
  \item Least Frequently Used (LFU)
\end{enumerate}

Addresses can be grouped by block number:
\begin{enumerate}[\roman*.]
  \item $2^N$ bytes in block $\rightarrow N$ offsets
\end{enumerate}
{\centering\incimg[0.7]{block}\par}

\subsection{Direct Mapped Cache}
Blocks mapped to location index in cache:
\begin{enumerate}[\roman*.]
  \item $2^M$ cache blocks $\rightarrow M$ indexes
  \item Block stored with identification tag and valid bit
\end{enumerate}
{\centering\incimg[0.7]{directmap}\par}

\subsection{k-way Associative Cache}
Blocks mapped to $k$ locations at set index in cache: 
\begin{enumerate}[\roman*.]
  \item $2^M$ cache sets (blocks / $k$) $\rightarrow M$ set indexes
  \item Block stored with identification tag and valid bit 
\end{enumerate}
{\centering\incimg[0.7]{nwayassoc}\par}

\subsection{Fully Associative Cache}
Blocks placed anywhere but need to search all blocks 

{\centering\incimg[0.7]{fullyassoc}\par}

\subsection{Performance}

Direct mapped cache of size $N$ has almost same miss rate as $2$-way associative cache of size $N/2$.

Miss Efficiencies:
\begin{enumerate}[\roman*.]
  \item Cold miss does not depend on size/associativity
  \item Conflict miss decreases with increasing associativity
  \item Capacity miss does not depend on associativity but decreases with increasing size
\end{enumerate}

\end{multicols*}
\section*{Reference Sheet}
\incimg{datapath}

\begin{minipage}{0.40\columnwidth}
  \incimg{controlsig}
\end{minipage}
\begin{minipage}{0.58\columnwidth}
  \incimg{instcontrol}

  \incimg{alucontrol}
\end{minipage}

\incimg[0.98]{ascii}

\Large
\begin{center}
\begin{tabular}{|l|l|l|}
\hline
\multicolumn{3}{|c|}{\textbf{Boolean Algebra Laws}} \\
\hline
Identity & $A + 0 = 0 + A = A$ & $A \cdot 1 = 1 \cdot A = A$ \\
\hline
Complement & $A + A' = 1$ & $A \cdot A' = 0$ \\
\hline
Commutative & $A + B = B + A$ & $A \cdot B = B \cdot A$ \\
\hline
Associative & $A + (B + C) = (A + B) + C$ & $A \cdot (B \cdot C) = (A \cdot B) \cdot C$ \\
\hline
Distributive & $A \cdot (B + C) = (A \cdot B) + (A \cdot C)$ & $A + (B \cdot C) = (A + B) \cdot (A + C)$ \\
\hline
Idempotency & $X + X = X$ & $X \cdot X = X$ \\
\hline
One/ Zero Element & $X + 1 = 1$ & $X \cdot 0 = 0$ \\
\hline
Involution & \multicolumn{2}{c|}{$(X')' = X$} \\
\hline
Absorption 1 & $X + X \cdot Y = X$ & $X \cdot (X + Y) = X$ \\
\hline
Absorption 2 & $X + X' \cdot Y = X + Y$ & $X \cdot (X' + Y) = X \cdot Y$ \\
\hline
De Morgan's & $(X + Y)' = X' \cdot Y'$ & $(X \cdot Y)' = X' + Y'$ \\
\hline
Consensus & $X + Y \cdot Z + Y' \cdot Z = X + Y' \cdot Z$ & $(X + Y) \cdot (X' + Z) \cdot (Y + Z) = (X + Y) \cdot (X' + Z)$ \\
\hline
\end{tabular}
\end{center}

\begin{minipage}{0.74\textwidth}
\begin{center}
\begin{tabular}{|l|l|l|l|}
\hline
\multicolumn{3}{|c|}{\textbf{C Precedence}} \\
\hline
Postfix Unary & \texttt{++, --} & Left to right\\
\hline
Prefix Unary & \texttt{+, -, ++, --, \&, *, (type), !} & Right to left\\
\hline
Multiplicative Binary & \texttt{*, /, \%} & Left to right\\
\hline
Additive Binary & \texttt{+, -} & Left to right\\ 
\hline
Relational & \texttt{<=, >=, <, >} & Left to right\\
\hline
Equality & \texttt{==, !=} & Left to right \\
\hline
Conjunction & \texttt{\&\&} & Left to right \\
\hline
Disjunction & \texttt{||} & Left to right \\
\hline
Assignment & \texttt{=, +=, -=, *=, /=, \%=} & Right to left\\
\hline
\end{tabular}
\end{center}
\end{minipage}
\begin{minipage}{0.1\textwidth}
\begin{center}
\begin{tabular}{|l|l|l|l|}
\hline
\multicolumn{4}{|c|}{\textbf{Prefix Multipliers}} \\
\hline
Nano & n & $10^{-9}$ & $2^{-30}$ \\
Micro & \textmu & $10^{-6}$ & $2^{-20}$ \\
Milli & m & $10^{-3}$ & $2^{-10}$ \\
Kilo & k & $10^{3}$ & $2^{10}$ \\
Mega & M & $10^{6}$ & $2^{20}$ \\
Giga & G & $10^{9}$ & $2^{30}$ \\
Tera & T & $10^{12}$ & $2^{40}$ \\
\hline
\end{tabular}
\end{center}
\end{minipage}

%%%%%%%%%%%%%%%%%%%%%%%%%%%%%%%%%%%%%%%%%%%%%%%%%%%%%%
%                       End                          %
%%%%%%%%%%%%%%%%%%%%%%%%%%%%%%%%%%%%%%%%%%%%%%%%%%%%%%
\end{document}

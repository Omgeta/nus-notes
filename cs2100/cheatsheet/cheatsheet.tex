\documentclass[12pt, a4paper]{article}

\usepackage[utf8]{inputenc}
\usepackage[mathscr]{euscript}
\let\euscr\mathscr \let\mathscr\relax
\usepackage[scr]{rsfso}
\usepackage{amssymb,amsmath,amsthm,amsfonts}
\usepackage[shortlabels]{enumitem}
\usepackage{multicol,multirow}
\usepackage{lipsum}
\usepackage{balance}
\usepackage{calc}
\usepackage[colorlinks=true,citecolor=blue,linkcolor=blue]{hyperref}
\usepackage{import}
\usepackage{xifthen}
\usepackage{pdfpages}
\usepackage{transparent}
\usepackage{tabularx}

\newcommand{\incfig}[2][1.0]{
    \def\svgwidth{#1\columnwidth}
    \import{./figures/}{#2.pdf_tex}
}
\newcommand{\incimg}[2][1.0]{
  \includegraphics[width=#1\columnwidth]{./figures/#2}
}


\usepackage{ifthen}
\usepackage[landscape]{geometry}
\usepackage[shortlabels]{enumitem}

\ifthenelse{\lengthtest { \paperwidth = 11in}}
    { \geometry{top=.5in,left=.5in,right=.5in,bottom=.5in} }
	{\ifthenelse{ \lengthtest{ \paperwidth = 297mm}}
		{\geometry{top=1cm,left=1cm,right=1cm,bottom=1cm} }
		{\geometry{top=1cm,left=1cm,right=1cm,bottom=1cm} }
	}

\pagestyle{empty}
\makeatletter
\renewcommand\thesection{\arabic{section}.}
\renewcommand{\section}{\@startsection{section}{1}{0mm}%
                                {-1ex plus -.5ex minus -.2ex}%
                                {0.05ex}%x
                                {\normalfont\normalsize\bfseries}}
\renewcommand{\subsection}{\@startsection{subsection}{2}{0mm}%
                                {-1ex plus -.5ex minus -.2ex}%
                                {0.05ex}%
                                {\normalfont\small\bfseries}}
\renewcommand{\subsubsection}{\@startsection{subsubsection}{3}{0mm}%
                                {-1ex plus -.5ex minus -.2ex}%
                                {0.05ex}%
                                {\normalfont\footnotesize\bfseries}}
\newcommand{\colbreak}{\vfill\null\columnbreak}
\makeatother
\setcounter{secnumdepth}{1}
\setlength{\parindent}{0pt}
\setlength{\parskip}{0.7em}

\setlist[itemize]{itemsep=0.6ex, topsep=-2pt, partopsep=0pt, parsep=0pt}
\setlist[enumerate]{itemsep=0.6ex, topsep=-2pt, partopsep=0pt, parsep=0pt}

\input{letterfonts}

\newcommand{\mytitle}{CS2100 Computer Organisation}
\newcommand{\myauthor}{github/omgeta}
\newcommand{\mydate}{AY 24/25 Sem 2}

\begin{document}
\raggedright
\footnotesize
\begin{multicols*}{3}
\setlength{\premulticols}{1pt}
\setlength{\postmulticols}{1pt}
\setlength{\multicolsep}{1pt}
\setlength{\columnsep}{2pt}

{\normalsize{\textbf{\mytitle}}} \\
{\footnotesize{\mydate\hspace{2pt}\textemdash\hspace{2pt}\myauthor}}
%%%%%%%%%%%%%%%%%%%%%%%%%%%%%%%%%%%%%%%%%%%%%%%%%%%%%%
%                      Begin                         %
%%%%%%%%%%%%%%%%%%%%%%%%%%%%%%%%%%%%%%%%%%%%%%%%%%%%%%

\section{Number Systems}

Weighted-positional number systems with base-$R$, has each position weighted in powers of $R$.
\begin{enumerate}[\roman*.]
  \item $N$ positions give $R^N$ values
  \item $M$ values need $\ceil{\log_RM}$ positions
\end{enumerate}

\subsection{Conversion}

Base-$R$$\rightarrow$Decimal: repeated multiply by $R^{n-1}$

Decimal$\rightarrow$Binary (Generalised for Base-$R$):
\begin{enumerate}[\roman*.]
  \item Whole numbers: repeated division by 2, $LSB \rightarrow MSB$
  \item Fractional: repeated multiplication by 2, $MSB \rightarrow LSB$
\end{enumerate}

Binary$\rightarrow$Octal: partition in groups of 3

Octal$\rightarrow$Binary: convert each digit to 3 bits

Binary$\rightarrow$Hexadecimal: partition in groups of 4

Hexadecimal$\rightarrow$Binary: convert each digit to 4 bits

\subsection{C Programming}

ASCII Chars are represented with $7$ bits and $1$ parity bit

\lstinline|char| is 1 byte, \lstinline|int + float| are 4 bytes, \lstinline|long + double| are 8 bytes.

Functions only modify \lstinline|struct| if passed as pointer or in array with: \lstinline|(*object_p).a = 1;|, \lstinline|object_p->a = 1;|

Arrays in \lstinline|struct|s are deep-copied (but not pointers to arrays).

\colbreak
\subsection{Negative Numbers}

Sign-and-Magnitude:
\begin{enumerate}[\roman*.]
  \item MSB is the sign bit ($0$ is positive)
  \item Range: $-(2^{N-1}-1)$ to $2^{N-1}-1$
  \item Negation: Flip MSB
\end{enumerate}

1s Complement (useful for arithmetic):
\begin{enumerate}[\roman*.]
  \item Negated $x$, $-x = 2^N- x -1$
  \item Range: $-(2^{N-1}-1)$ to $2^{N-1}-1$
  \item Negation: Invert bits
  \item Overflow: add carry to result, wrap around
  \item $(b-1)s$: $-x = b^N- x -1$
\end{enumerate}

2s Complement (useful for arithmetic):
\begin{enumerate}[\roman*.]
  \item Negated $x$, $-x = 2^N- x$
  \item Range: $-2^{N-1}$ to $2^{N-1}-1$
  \item Negation: Invert bits, then add $1$
  \item Overflow: truncate 
  \item $bs$: $-x = b^N- x$
\end{enumerate}

Excess-$M$ (useful for comparisons):
\begin{enumerate}[\roman*.]
  \item Start at $-M = -2^{N-1}$, for $N$ bits.
  \item Range: $-2^{N-1}$ to $2^{N-1}-1$
  \item Express $x$ in Excess-$N$: $x + N$
\end{enumerate}

\subsection{Real Numbers}

Fixed-Point (limited range):
\begin{enumerate}[\roman*.]
  \item Reserve bits for whole numbers and for fraction, converting with 1s or 2s complement
\end{enumerate}

IEEE 754 Floating-Point (more complex):
\begin{enumerate}[\roman*.]
  \item Sign 1-bit (0 is positive)
  \item Exponent 8-bits (excess-127)
  \item Mantissa 23-bits (normalised to $1.m \times 2^{exp}$)
\end{enumerate}
\incimg[0.75]{ieee754}
\colbreak

\section{ISA}

Instruction Set Architecture (ISA) defines instructions for how software controls the hardware. 

von Neumann Architecture:
\begin{enumerate}[\roman*.]
  \item Processor: Perform computations.
  \item Memory: Stores code and data (stored memory).
  \item Bus: Bridge between processor and memory.
\end{enumerate}

Storage Architectures:
\begin{enumerate}[\roman*.]
  \item Stack: Operands are implicitly on top of the stack.
  \item Accumulator: One operand is implicitly in the accumulator.
  \item Memory-Memory: All operands in memory.
  \item Register-Register: All operands in registers.
\end{enumerate}

Endianness (ordering of bytes in a word):
\begin{enumerate}[\roman*.]
  \item Big Endian: MSB in lowest address
  \item Little Endian: LSB in lowest address
\end{enumerate}

Instruction Length:
\begin{enumerate}[\roman*.]
  \item Fixed-Length: easy fetch and decode, simplified pipelining, instruction bits are scarce
  \item Variable-Length: more flexible
\end{enumerate}

Instruction Encoding, under fixed-length instructions involves extending opcode for instruction types with unused bits:
\begin{enumerate}
  \item Minimise: maximise opcode range of instruction type with least opcode bits
  \item Maximise: minimise opcode range of instruction type with least opcode bits
\end{enumerate}

\colbreak

\section{MIPS}

MIPS Assembly Language:
\begin{enumerate}[\roman*.]
  \item Mainly Register-Register, Fixed-Length
  \item $32$ registers, each $32$-bits ($4$-byte) long
  \item $2^{30}$ words contains $32$-bits ($4$-byte) long 
  \item Memory addresses are $32$-bits long
\end{enumerate}

Arithmetic Instructions:
\begin{enumerate}[\roman*.]
  \item $C16, C5$ are $16, 5$ bit patterns
  \item $C16_{2S}$ is a sign-extended 2's complement
  \item NOT: \lstinline|nor| with \lstinline|$zero|, or \lstinline|xor| with all 1s
  \item Large constants: \lstinline|lui| (31:16 bits) + \lstinline|ori| (15:0 bits)
\end{enumerate}
\incimg{mips_basic}
Memory Instructions:
\begin{enumerate}[\roman*.]
  \item \lstinline|lw $rt, offset($rs)|: load contents to \lstinline|$rt|
  \item \lstinline|sw $rt, offset($rs)|: store contents to \lstinline|$rs|
  \item Use \lstinline|lb, sb| for loading bytes (like chars)
\end{enumerate}

Control Instructions:
\begin{enumerate}[\roman*.]
  \item \lstinline|beq/bne $rs, $rt, label/imm|: jump if $=$/$\neq$.
  \item Branch: adds \lstinline|imm x 4| to \lstinline|$PC = $PC + 4| on branch. Range: $\pm 2^{15}$ instructions
  \item \lstinline|j label/imm|: jump unconditionally.
  \item Pseudo-direct: takes 4 MSBs from \lstinline|$PC + 4|, and 2 LSB omitted since instructions are word-aligned. Range: $256$-byte boundary of \lstinline|$PC + 4| 4 MSBs
  \item \lstinline|slt $rd, $rs, $rt|: set \lstinline|$rd| to 1 if \lstinline|$rs < $rt|
\end{enumerate}
\colbreak

\subsection{Datapath}
\incimg{datapath_cycle}

Clock:
\begin{enumerate}[\roman*.]
  \item Single-Cycle: Read + Compute are done within clock period. PC Write on rising clock edge. Time taken depends on longest instruction.
\incimg[0.7]{clock}
  \item Multicycle: Break up execution into multiple steps (e.g. fetch, decode, alu, memory, writeback), with each step taking one cycle. Time taken depends on slowest step.
\end{enumerate}

Critical Paths:
\begin{enumerate}[\roman*.]
  \item R-type/Immediate:\\
    Inst Mem$\rightarrow$RegFile$\rightarrow$MUX (ALUSrc)\\$\rightarrow$ALU$\rightarrow$MUX (MemToReg)$\rightarrow$RegFile
  \item \lstinline|lw/sw|:\\
    Inst Mem$\rightarrow$RegFile$\rightarrow$ALU$\rightarrow$DataMem$\rightarrow$MUX (MemToReg)$\rightarrow$RegFile
  \item \lstinline|beq|:\\
    Inst Mem$\rightarrow$RegFile$\rightarrow$MUX (ALUSrc)$\rightarrow$ALU$\rightarrow$AND$\rightarrow$MUX (PCSrc)
\end{enumerate}

\colbreak
\subsection{Control}
Control Signals:
\begin{enumerate}[\roman*.]
  \item \lstinline|RegDst| @Decode: Selects destination register.\\
    (0) $\rightarrow$ \lstinline|$rt|
    (1) $\rightarrow$ \lstinline|$rd|
  \item \lstinline|RegWrite| @Decode: Enable register write.\\
    (0) $\rightarrow$ no write
    (1) $\rightarrow$ \lstinline|WD| written to \lstinline|WR|
  \item \lstinline|ALUSrc| @ALU: Second 2nd ALU operand.\\ 
    (0) $\rightarrow$ \lstinline|RD2|
    (1) $\rightarrow$ \lstinline|SignExt(Imm)|
  \item \lstinline|ALUcontrol| @ALU: Second ALU operation.\\
    2-bit \lstinline|ALUop| from opcode + funct for 4-bits
  \item \lstinline|MemRead| @Memory: Enable memory read.\\
    (0) $\rightarrow$ no read
    (1) $\rightarrow$ read \lstinline|Addr.| into \lstinline|Read Data|
  \item \lstinline|MemWrite| @Memory: Enable memory write.\\
    (0) $\rightarrow$ no write
    (1) $\rightarrow$ write \lstinline|Write Data| into \lstinline|Addr.|
  \item \lstinline|MemToReg| @Writeback: Select writeback result.\\
    (0) $\rightarrow$ \lstinline|ALU result|
    (1) $\rightarrow$ Memory \lstinline|Read data|
  \item \lstinline|Branch| @Memory: Select next \lstinline|$PC|.\\
    (0) $\rightarrow$ \lstinline|$PC + 4|
    (1) $\rightarrow$ \lstinline|($PC + 4) + (imm x 4)|
\end{enumerate}

\incimg[0.9]{alucontrolunit}

\subsubsection{1-bit ALU}
\incimg{1bitalu}

\end{multicols*}
\section*{Reference Sheet}
\incimg{datapath}

\begin{minipage}{0.40\columnwidth}
  \incimg{controlsig}
\end{minipage}
\begin{minipage}{0.58\columnwidth}
  \incimg{instcontrol}

  \incimg{alucontrol}
\end{minipage}

\incimg[0.98]{ascii}

\incimg[0.90]{cprecedence}
%%%%%%%%%%%%%%%%%%%%%%%%%%%%%%%%%%%%%%%%%%%%%%%%%%%%%%
%                       End                          %
%%%%%%%%%%%%%%%%%%%%%%%%%%%%%%%%%%%%%%%%%%%%%%%%%%%%%%
\end{document}

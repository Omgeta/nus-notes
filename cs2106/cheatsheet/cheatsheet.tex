\documentclass[12pt, a4paper]{article}

\usepackage[utf8]{inputenc}
\usepackage[mathscr]{euscript}
\let\euscr\mathscr \let\mathscr\relax
\usepackage[scr]{rsfso}
\usepackage{amssymb,amsmath,amsthm,amsfonts}
\usepackage[shortlabels]{enumitem}
\usepackage{multicol,multirow}
\usepackage{lipsum}
\usepackage{balance}
\usepackage{calc}
\usepackage[colorlinks=true,citecolor=blue,linkcolor=blue]{hyperref}
\usepackage{import}
\usepackage{xifthen}
\usepackage{pdfpages}
\usepackage{transparent}
\usepackage{listings}

\newcommand{\incfig}[2][1.0]{
    \def\svgwidth{#1\columnwidth}
    \import{./figures/}{#2.pdf_tex}
}

\newlist{enumproof}{enumerate}{4}
\setlist[enumproof,1]{label=\arabic*., parsep=1em}
\setlist[enumproof,2]{label=\arabic{enumproofi}.\arabic*., parsep=1em}
\setlist[enumproof,3]{label=\arabic{enumproofi}.\arabic{enumproofii}.\arabic*., parsep=1em}
\setlist[enumproof,4]{label=\arabic{enumproofi}.\arabic{enumproofii}.\arabic{enumproofiii}.\arabic*., parsep=1em}

\renewcommand{\qedsymbol}{\ensuremath{\blacksquare}}

\lstdefinestyle{mystyle}{
  language=C, % Set the language to C
  commentstyle=\color{codegray}, % Color for comments
  keywordstyle=\color{orange}, % Color for basic keywords
  stringstyle=\color{mauve}, % Color for strings
  basicstyle={\ttfamily\footnotesize}, % Basic font style
  breakatwhitespace=false,         
  breaklines=true,                 
  captionpos=b,                    
  keepspaces=true,                 
  numbers=none,                    
  tabsize=2,
  morekeywords=[2]{\#include, \#define, \#ifdef, \#ifndef, \#endif, \#pragma, \#else, \#elif}, % Preprocessor directives
  keywordstyle=[2]\color{codegreen}, % Style for preprocessor directives
  morekeywords=[3]{int, char, float, double, void, struct, union, enum, const, volatile, static, extern, register, inline, restrict, _Bool, _Complex, _Imaginary, size_t, ssize_t, FILE}, % C standard types and common identifiers
  keywordstyle=[3]\color{identblue}, % Style for types and common identifiers
  morekeywords=[4]{printf, scanf, fopen, fclose, malloc, free, calloc, realloc, perror, strtok, strncpy, strcpy, strcmp, strlen}, % Standard library functions
  keywordstyle=[4]\color{cyan}, % Style for library functions
}

\usepackage{ifthen}
\usepackage[landscape]{geometry}
\usepackage[shortlabels]{enumitem}

\ifthenelse{\lengthtest { \paperwidth = 11in}}
    { \geometry{top=.5in,left=.5in,right=.5in,bottom=.5in} }
	{\ifthenelse{ \lengthtest{ \paperwidth = 297mm}}
		{\geometry{top=1cm,left=1cm,right=1cm,bottom=1cm} }
		{\geometry{top=1cm,left=1cm,right=1cm,bottom=1cm} }
	}

\pagestyle{empty}
\makeatletter
\renewcommand\thesection{\arabic{section}.}
\renewcommand{\section}{\@startsection{section}{1}{0mm}%
                                {-1ex plus -.5ex minus -.2ex}%
                                {0.05ex}%x
                                {\normalfont\normalsize\bfseries}}
\renewcommand{\subsection}{\@startsection{subsection}{2}{0mm}%
                                {-1ex plus -.5ex minus -.2ex}%
                                {0.05ex}%
                                {\normalfont\small\bfseries}}
\renewcommand{\subsubsection}{\@startsection{subsubsection}{3}{0mm}%
                                {-1ex plus -.5ex minus -.2ex}%
                                {0.05ex}%
                                {\normalfont\footnotesize\bfseries}}
\newcommand{\colbreak}{\vfill\null\columnbreak}
\makeatother
\setcounter{secnumdepth}{1}
\setlength{\parindent}{0pt}
\setlength{\parskip}{0.7em}

\setlist[itemize]{itemsep=0.6ex, topsep=-2pt, partopsep=0pt, parsep=0pt}
\setlist[enumerate]{itemsep=0.6ex, topsep=-2pt, partopsep=0pt, parsep=0pt}

% Things Lie
\newcommand{\kb}{\mathfrak b}
\newcommand{\kg}{\mathfrak g}
\newcommand{\kh}{\mathfrak h}
\newcommand{\kn}{\mathfrak n}
\newcommand{\ku}{\mathfrak u}
\newcommand{\kz}{\mathfrak z}
\DeclareMathOperator{\Ext}{Ext} % Ext functor
\DeclareMathOperator{\Tor}{Tor} % Tor functor
\newcommand{\gl}{\opname{\mathfrak{gl}}} % frak gl group
\renewcommand{\sl}{\opname{\mathfrak{sl}}} % frak sl group chktex 6

% More script letters etc.
\newcommand{\SA}{\mathcal A}
\newcommand{\SB}{\mathcal B}
\newcommand{\SC}{\mathcal C}
\newcommand{\SF}{\mathcal F}
\newcommand{\SG}{\mathcal G}
\newcommand{\SH}{\mathcal H}
\newcommand{\OO}{\mathcal O}

\newcommand{\SCA}{\mathscr A}
\newcommand{\SCB}{\mathscr B}
\newcommand{\SCC}{\mathscr C}
\newcommand{\SCD}{\mathscr D}
\newcommand{\SCE}{\mathscr E}
\newcommand{\SCF}{\mathscr F}
\newcommand{\SCG}{\mathscr G}
\newcommand{\SCH}{\mathscr H}

% Mathfrak primes
\newcommand{\km}{\mathfrak m}
\newcommand{\kp}{\mathfrak p}
\newcommand{\kq}{\mathfrak q}

% number sets
\newcommand{\RR}[1][]{\ensuremath{\ifstrempty{#1}{\mathbb{R}}{\mathbb{R}^{#1}}}}
\newcommand{\NN}[1][]{\ensuremath{\ifstrempty{#1}{\mathbb{N}}{\mathbb{N}^{#1}}}}
\newcommand{\ZZ}[1][]{\ensuremath{\ifstrempty{#1}{\mathbb{Z}}{\mathbb{Z}^{#1}}}}
\newcommand{\QQ}[1][]{\ensuremath{\ifstrempty{#1}{\mathbb{Q}}{\mathbb{Q}^{#1}}}}
\newcommand{\CC}[1][]{\ensuremath{\ifstrempty{#1}{\mathbb{C}}{\mathbb{C}^{#1}}}}
\newcommand{\PP}[1][]{\ensuremath{\ifstrempty{#1}{\mathbb{P}}{\mathbb{P}^{#1}}}}
\newcommand{\HH}[1][]{\ensuremath{\ifstrempty{#1}{\mathbb{H}}{\mathbb{H}^{#1}}}}
\newcommand{\FF}[1][]{\ensuremath{\ifstrempty{#1}{\mathbb{F}}{\mathbb{F}^{#1}}}}
% expected value
\newcommand{\EE}{\ensuremath{\mathbb{E}}}
\newcommand{\charin}{\text{ char }}
\DeclareMathOperator{\sign}{sign}
\DeclareMathOperator{\Aut}{Aut}
\DeclareMathOperator{\Inn}{Inn}
\DeclareMathOperator{\Syl}{Syl}
\DeclareMathOperator{\Gal}{Gal}
\DeclareMathOperator{\GL}{GL} % General linear group
\DeclareMathOperator{\SL}{SL} % Special linear group

%---------------------------------------
% BlackBoard Math Fonts :-
%---------------------------------------

%Captital Letters
\newcommand{\bbA}{\mathbb{A}}	\newcommand{\bbB}{\mathbb{B}}
\newcommand{\bbC}{\mathbb{C}}	\newcommand{\bbD}{\mathbb{D}}
\newcommand{\bbE}{\mathbb{E}}	\newcommand{\bbF}{\mathbb{F}}
\newcommand{\bbG}{\mathbb{G}}	\newcommand{\bbH}{\mathbb{H}}
\newcommand{\bbI}{\mathbb{I}}	\newcommand{\bbJ}{\mathbb{J}}
\newcommand{\bbK}{\mathbb{K}}	\newcommand{\bbL}{\mathbb{L}}
\newcommand{\bbM}{\mathbb{M}}	\newcommand{\bbN}{\mathbb{N}}
\newcommand{\bbO}{\mathbb{O}}	\newcommand{\bbP}{\mathbb{P}}
\newcommand{\bbQ}{\mathbb{Q}}	\newcommand{\bbR}{\mathbb{R}}
\newcommand{\bbS}{\mathbb{S}}	\newcommand{\bbT}{\mathbb{T}}
\newcommand{\bbU}{\mathbb{U}}	\newcommand{\bbV}{\mathbb{V}}
\newcommand{\bbW}{\mathbb{W}}	\newcommand{\bbX}{\mathbb{X}}
\newcommand{\bbY}{\mathbb{Y}}	\newcommand{\bbZ}{\mathbb{Z}}

%---------------------------------------
% MathCal Fonts :-
%---------------------------------------

%Captital Letters
\newcommand{\mcA}{\mathcal{A}}	\newcommand{\mcB}{\mathcal{B}}
\newcommand{\mcC}{\mathcal{C}}	\newcommand{\mcD}{\mathcal{D}}
\newcommand{\mcE}{\mathcal{E}}	\newcommand{\mcF}{\mathcal{F}}
\newcommand{\mcG}{\mathcal{G}}	\newcommand{\mcH}{\mathcal{H}}
\newcommand{\mcI}{\mathcal{I}}	\newcommand{\mcJ}{\mathcal{J}}
\newcommand{\mcK}{\mathcal{K}}	\newcommand{\mcL}{\mathcal{L}}
\newcommand{\mcM}{\mathcal{M}}	\newcommand{\mcN}{\mathcal{N}}
\newcommand{\mcO}{\mathcal{O}}	\newcommand{\mcP}{\mathcal{P}}
\newcommand{\mcQ}{\mathcal{Q}}	\newcommand{\mcR}{\mathcal{R}}
\newcommand{\mcS}{\mathcal{S}}	\newcommand{\mcT}{\mathcal{T}}
\newcommand{\mcU}{\mathcal{U}}	\newcommand{\mcV}{\mathcal{V}}
\newcommand{\mcW}{\mathcal{W}}	\newcommand{\mcX}{\mathcal{X}}
\newcommand{\mcY}{\mathcal{Y}}	\newcommand{\mcZ}{\mathcal{Z}}

%---------------------------------------
% Bold Math Fonts :-
%---------------------------------------

%Captital Letters
\newcommand{\bmA}{\boldsymbol{A}}	\newcommand{\bmB}{\boldsymbol{B}}
\newcommand{\bmC}{\boldsymbol{C}}	\newcommand{\bmD}{\boldsymbol{D}}
\newcommand{\bmE}{\boldsymbol{E}}	\newcommand{\bmF}{\boldsymbol{F}}
\newcommand{\bmG}{\boldsymbol{G}}	\newcommand{\bmH}{\boldsymbol{H}}
\newcommand{\bmI}{\boldsymbol{I}}	\newcommand{\bmJ}{\boldsymbol{J}}
\newcommand{\bmK}{\boldsymbol{K}}	\newcommand{\bmL}{\boldsymbol{L}}
\newcommand{\bmM}{\boldsymbol{M}}	\newcommand{\bmN}{\boldsymbol{N}}
\newcommand{\bmO}{\boldsymbol{O}}	\newcommand{\bmP}{\boldsymbol{P}}
\newcommand{\bmQ}{\boldsymbol{Q}}	\newcommand{\bmR}{\boldsymbol{R}}
\newcommand{\bmS}{\boldsymbol{S}}	\newcommand{\bmT}{\boldsymbol{T}}
\newcommand{\bmU}{\boldsymbol{U}}	\newcommand{\bmV}{\boldsymbol{V}}
\newcommand{\bmW}{\boldsymbol{W}}	\newcommand{\bmX}{\boldsymbol{X}}
\newcommand{\bmY}{\boldsymbol{Y}}	\newcommand{\bmZ}{\boldsymbol{Z}}
%Small Letters
\newcommand{\bma}{\boldsymbol{a}}	\newcommand{\bmb}{\boldsymbol{b}}
\newcommand{\bmc}{\boldsymbol{c}}	\newcommand{\bmd}{\boldsymbol{d}}
\newcommand{\bme}{\boldsymbol{e}}	\newcommand{\bmf}{\boldsymbol{f}}
\newcommand{\bmg}{\boldsymbol{g}}	\newcommand{\bmh}{\boldsymbol{h}}
\newcommand{\bmi}{\boldsymbol{i}}	\newcommand{\bmj}{\boldsymbol{j}}
\newcommand{\bmk}{\boldsymbol{k}}	\newcommand{\bml}{\boldsymbol{l}}
\newcommand{\bmm}{\boldsymbol{m}}	\newcommand{\bmn}{\boldsymbol{n}}
\newcommand{\bmo}{\boldsymbol{o}}	\newcommand{\bmp}{\boldsymbol{p}}
\newcommand{\bmq}{\boldsymbol{q}}	\newcommand{\bmr}{\boldsymbol{r}}
\newcommand{\bms}{\boldsymbol{s}}	\newcommand{\bmt}{\boldsymbol{t}}
\newcommand{\bmu}{\boldsymbol{u}}	\newcommand{\bmv}{\boldsymbol{v}}
\newcommand{\bmw}{\boldsymbol{w}}	\newcommand{\bmx}{\boldsymbol{x}}
\newcommand{\bmy}{\boldsymbol{y}}	\newcommand{\bmz}{\boldsymbol{z}}

%---------------------------------------
% Scr Math Fonts :-
%---------------------------------------

\newcommand{\sA}{{\mathscr{A}}}   \newcommand{\sB}{{\mathscr{B}}}
\newcommand{\sC}{{\mathscr{C}}}   \newcommand{\sD}{{\mathscr{D}}}
\newcommand{\sE}{{\mathscr{E}}}   \newcommand{\sF}{{\mathscr{F}}}
\newcommand{\sG}{{\mathscr{G}}}   \newcommand{\sH}{{\mathscr{H}}}
\newcommand{\sI}{{\mathscr{I}}}   \newcommand{\sJ}{{\mathscr{J}}}
\newcommand{\sK}{{\mathscr{K}}}   \newcommand{\sL}{{\mathscr{L}}}
\newcommand{\sM}{{\mathscr{M}}}   \newcommand{\sN}{{\mathscr{N}}}
\newcommand{\sO}{{\mathscr{O}}}   \newcommand{\sP}{{\mathscr{P}}}
\newcommand{\sQ}{{\mathscr{Q}}}   \newcommand{\sR}{{\mathscr{R}}}
\newcommand{\sS}{{\mathscr{S}}}   \newcommand{\sT}{{\mathscr{T}}}
\newcommand{\sU}{{\mathscr{U}}}   \newcommand{\sV}{{\mathscr{V}}}
\newcommand{\sW}{{\mathscr{W}}}   \newcommand{\sX}{{\mathscr{X}}}
\newcommand{\sY}{{\mathscr{Y}}}   \newcommand{\sZ}{{\mathscr{Z}}}


%---------------------------------------
% Math Fraktur Font
%---------------------------------------

%Captital Letters
\newcommand{\mfA}{\mathfrak{A}}	\newcommand{\mfB}{\mathfrak{B}}
\newcommand{\mfC}{\mathfrak{C}}	\newcommand{\mfD}{\mathfrak{D}}
\newcommand{\mfE}{\mathfrak{E}}	\newcommand{\mfF}{\mathfrak{F}}
\newcommand{\mfG}{\mathfrak{G}}	\newcommand{\mfH}{\mathfrak{H}}
\newcommand{\mfI}{\mathfrak{I}}	\newcommand{\mfJ}{\mathfrak{J}}
\newcommand{\mfK}{\mathfrak{K}}	\newcommand{\mfL}{\mathfrak{L}}
\newcommand{\mfM}{\mathfrak{M}}	\newcommand{\mfN}{\mathfrak{N}}
\newcommand{\mfO}{\mathfrak{O}}	\newcommand{\mfP}{\mathfrak{P}}
\newcommand{\mfQ}{\mathfrak{Q}}	\newcommand{\mfR}{\mathfrak{R}}
\newcommand{\mfS}{\mathfrak{S}}	\newcommand{\mfT}{\mathfrak{T}}
\newcommand{\mfU}{\mathfrak{U}}	\newcommand{\mfV}{\mathfrak{V}}
\newcommand{\mfW}{\mathfrak{W}}	\newcommand{\mfX}{\mathfrak{X}}
\newcommand{\mfY}{\mathfrak{Y}}	\newcommand{\mfZ}{\mathfrak{Z}}
%Small Letters
\newcommand{\mfa}{\mathfrak{a}}	\newcommand{\mfb}{\mathfrak{b}}
\newcommand{\mfc}{\mathfrak{c}}	\newcommand{\mfd}{\mathfrak{d}}
\newcommand{\mfe}{\mathfrak{e}}	\newcommand{\mff}{\mathfrak{f}}
\newcommand{\mfg}{\mathfrak{g}}	\newcommand{\mfh}{\mathfrak{h}}
\newcommand{\mfi}{\mathfrak{i}}	\newcommand{\mfj}{\mathfrak{j}}
\newcommand{\mfk}{\mathfrak{k}}	\newcommand{\mfl}{\mathfrak{l}}
\newcommand{\mfm}{\mathfrak{m}}	\newcommand{\mfn}{\mathfrak{n}}
\newcommand{\mfo}{\mathfrak{o}}	\newcommand{\mfp}{\mathfrak{p}}
\newcommand{\mfq}{\mathfrak{q}}	\newcommand{\mfr}{\mathfrak{r}}
\newcommand{\mfs}{\mathfrak{s}}	\newcommand{\mft}{\mathfrak{t}}
\newcommand{\mfu}{\mathfrak{u}}	\newcommand{\mfv}{\mathfrak{v}}
\newcommand{\mfw}{\mathfrak{w}}	\newcommand{\mfx}{\mathfrak{x}}
\newcommand{\mfy}{\mathfrak{y}}	\newcommand{\mfz}{\mathfrak{z}}


\newcommand{\mytitle}{CS2106 Intro. to Operating Systems}
\newcommand{\myauthor}{github/omgeta}
\newcommand{\mydate}{AY 25/26 Sem 1}

\begin{document}
\raggedright
\footnotesize
\begin{multicols*}{3}
\setlength{\premulticols}{1pt}
\setlength{\postmulticols}{1pt}
\setlength{\multicolsep}{1pt}
\setlength{\columnsep}{2pt}

{\normalsize{\textbf{\mytitle}}}\\
{\footnotesize{\mydate\hspace{2pt}\textemdash\hspace{2pt}\myauthor}}
%%%%%%%%%%%%%%%%%%%%%%%%%%%%%%%%%%%%%%%%%%%%%%%%%%%%%%
%                      Begin                         %
%%%%%%%%%%%%%%%%%%%%%%%%%%%%%%%%%%%%%%%%%%%%%%%%%%%%%%
\section{Introduction}
Operating systems are software intermediaries between users and hardware, allocating resources, providing services and controlling program execution. 
\begin{enumerate}[\roman*.]
  \item Kernel mode: OS access to hardware resources
  \item User mode: Software has limited or controlled access to hardware resources
\end{enumerate}

Structures:
\begin{enumerate}[\roman*.]
  \item \textbf{Monolithic}: Kernel is big and highly coupled.\\Well-understood and performant, but complicated. 
    \begin{center}
      \hspace{-1em}\incimg[0.9]{monolithic}
    \end{center}
  \item \textbf{Microkernel}: Kernel is small with only essential features; high-level services are userspace server processes communicating with kernel via IPC.\\Robust, extensible and isolates kernel from high-level services, but lower performance.
    \begin{center}
      \hspace{-1em}\incimg[0.9]{microkernel}
    \end{center}
  \item \textbf{Layered}
  \item \textbf{Client-Server}
\end{enumerate}

\colbreak
Types:
\begin{enumerate}[\roman*.]
  \item \textbf{None}: Program directly interact with hardware;\\ minimal overhead but not portable and inefficient.
  \item \textbf{Batch}: Execute jobs with minimal user interaction;\\efficient for many jobs but CPU idles on IO.
  \item \textbf{Timesharing}: Multiple users interact using terminals, user job scheduler illudes concurrency. 
  \item \textbf{Real Time}: Hard real time has timing requirements strictly respected, soft real time allows some time constraints to be missed.
  \item \textbf{Embedded}: Specialized for devices;\\considers power, memory and real-time constraints but not general-purpose.
  \item \textbf{Distributed}: Manages systems on network;\\improves resource sharing and fault tolerance but adds complexity and reliability concerns.
\end{enumerate}

\subsection{Virtual Machines}
Virtual machines provide a virtualization of underlying hardware that runs an operating system and applications independently on a host system. 

Hypervisors are software which manage virtualization.
\begin{enumerate}[\roman*.]
  \item \textbf{Type 1 Hypervisor}: Run directly on hardware.
    \begin{center}
      \incimg[0.65]{type1}
    \end{center}
  \item \textbf{Type 2 Hypervisor}: Run as an app on host OS.
    \begin{center}
      \incimg[0.65]{type2}
    \end{center}
\end{enumerate}

\colbreak
\subsection{System Calls}
System calls are API calls to the OS for kernel services, changing from user mode to kernel mode via special instructions called TRAPs.

Mechanism:
\begin{enumerate}[\roman*.]
  \item User invokes library call 
  \item Library call (usually assembly code) places system call number in special register 
  \item Library call executes TRAP to enter kernel mode 
  \item System call dispatcher calls appropriate system call handler based on system call number 
  \item System call handler executed 
  \item OS switches to user mode and returns control to library call
  \item Library call returns to user using function return
\end{enumerate}

\subsection{Exceptions and Interrupts}
Exceptions are synchronous, caused by program execution errors, and cause the exception handler to execute.

Interrupts are assynchronous, caused by external events, and cause the interrupt handler to execute.

\colbreak
\section{Processes and Threads}
\subsection{Processes}
Process is an abstraction for a running program.\\
\begin{enumerate}[\roman*.]
  \item Process ID (PID): uniquely identifies a process 
  \item Process State: indicates execution status
  \item Process Control Block (PCB): stores process execution context (registers, address space, etc.)
  \item Process Table: maintains PCBs in kernel
  \item UNIX processes stores parent PID, CPU time, etc. with \lstinline|init| as root process
\end{enumerate}
{\centering
  \incimg[0.75]{processtable}\\
  \vspace{1em}
  \incimg[0.8]{processstates}
\par}

POSIX System Calls:
\begin{enumerate}[\roman*.]
  \item \lstinline|fork()|: duplicate process and PCB to new PID; memory is copy-on-write (shared until modified)
  \item \lstinline|execl()|: replace core image of process 
  \item \lstinline|exit()|: end process and return exit status
  \item \lstinline|wait()|: block until a child exits, then cleanup;\\\lstinline|waitpid()|: wait for specific process\\\vspace{2pt}
    \begin{itemize}[leftmargin=*]
    \item Zombie: child exited without cleanup 
    \item Orphan: parent exited before child;\\ \lstinline|init| becomes parent and cleans up
  \end{itemize}
\end{enumerate}
\vspace{-1em}
\colbreak
\subsection{Scheduling}
Schedulers order process execution with policy types: 
\begin{enumerate}[\roman*.]
  \item Non-Preemptive: processes run until they block or they voluntary yield.
  \item Preemptive: processes have fixed time quota.
  \vspace{2pt}
  \begin{itemize}
    \item Interval of Timer Interrupt (ITI):\\Scheduler trigger time ($1-10ms$)
    \item Time Quantum: Max time quota;\\must be multiple of ITI ($5-100ms$)
  \end{itemize}
\end{enumerate}
and algorithms dependent on environment and criteria:
\begin{enumerate}[\roman*.]
  \item Fairness: processes have a fair share of CPU time (per process or per user basis) with no starvation.
  \item Utilization: all parts of computer should be used.
\end{enumerate}

\subsubsection{Batch Scheduling Algorithms}
Criteria:
\begin{enumerate}[\roman*.]
  \item Throughput: Jobs completed per unit time
  \item CPU Utilization: Percentage of time CPU busy
  \item Turnaround Time: finish time $-$ arrival time;\\ relating to waiting time: time waiting for CPU
\end{enumerate}

\textbf{First-Come First-Serve (FCFS)} uses FIFO queue for jobs based on arrival time.
\begin{enumerate}[$+$]
  \item Starvation-Free (assuming all tasks complete)
\end{enumerate}\vspace{-1pt}
\begin{enumerate}[$-$]
  \item Convoy Effect: Long task at start can greatly increase turnaround time for shorter tasks later
\end{enumerate}

\textbf{Shortest Job First (SJF)} selects job with smallest total CPU time (guess if info not available).
\begin{enumerate}[$+$]
  \item Minimises Average Waiting Time
\end{enumerate}\vspace{-1pt}
\begin{enumerate}[$-$]
  \item Starvation Possible: Long jobs may never run
\end{enumerate}\vspace{-1pt}
\begin{enumerate}[\roman*.]
  \item CPU time: $pred_{n+1} = \alpha\cdot actual_{n}+(1-\alpha)\cdot pred_{n}$;\\
    where $\alpha$ is weight placed on recent event
\end{enumerate}

\textbf{Shortest Remaining Time (SRT)} selects job with smallest remaining time.
\begin{enumerate}[$+$]
  \item Preemptive: Short jobs can preempt current job; provides good service for short jobs even when late
\end{enumerate}
\colbreak
\subsubsection{Interactive Scheduling Algorithms}
Criteria:
\begin{enumerate}[\roman*.]
  \item Response time: time between request and response
  \item Predictability: variation in response time
\end{enumerate}

\textbf{Round Robin (RR)} uses FIFO queue for jobs, each run with fixed time quantum then moved to end of queue if ready. Blocked tasks are queued elsewhere until ready.
\begin{enumerate}[\roman*.]
  \item Response Time Guarantee: $n^{th}$ task at $(n-1)q$
  \item Choice of Time Quantum: If big, better CPU utilization but longer waiting time. If small, more context switch overhead but shorter waiting time. 
\end{enumerate}

\textbf{Priority Scheduling} chooses task with highest priority. If preemptive, higher priority tasks preempt lower priority.
\begin{enumerate}[$+$]
  \item Handles Priority 
\end{enumerate}\vspace{-1pt}
\begin{enumerate}[$-$]
  \item Starvation Possible: Low priority tasks may never run, especially if preemptive. Solutions:
    \begin{itemize}[leftmargin=*]
      \item Decrease priority of current task after every time quantum 
      \item Give current process a time quantum
    \end{itemize}
  \item Priority Inversion: Lower priority tasks preempt higher priority task (e.g. $A > B$ but $A$ cannot preempt $B$ because $A$ depends on locked file)
\end{enumerate}

\textbf{Multi-Level Feedback Queue (MLFQ)} uses priority scheduling, with RR for same priorities. New tasks have highest priority. Tasks fully using time quantum reduce priority; while yield/block early retains priority. 
\begin{enumerate}[$+$]
  \item Adaptive: Adjusts to actual behaviour
  \item Minimises: Response time for IO bound tasks, and turnaround time for CPU bound tasks
\end{enumerate}\vspace{-1pt}
\begin{enumerate}[$-$]
  \item Abuse Possible: If process keeps yielding early
\end{enumerate}

\textbf{Lottery Scheduling} gives out "lottery tickets" to tasks, with a random task chosen during scheduling.
\begin{enumerate}[$+$]
  \item Responsive: New tasks can participate immediately
  \item Level of Control: Tickets can be distributed to child processes; more tickets for higher priority; each resource can have its own ticket distribution
  \item Simple to Implement
\end{enumerate}
\vspace{-1em}
\colbreak
\subsection{Inter-Process Communication}

\textbf{Shared-Memory} involves a process creating a shared memory region which is attached to by another process.
\begin{enumerate}[$+$]
  \item Efficient: only create and attach involve OS
  \item Ease of Use: shared region behaves as normal
\end{enumerate}\vspace{-1pt}
\begin{enumerate}[$-$]
  \item Synchronization Necessary
  \item Difficult to Implement
\end{enumerate}\vspace{-1pt}
\begin{enumerate}[\roman*.]
  \item \lstinline|shmget()|: create/get shared memory segment
  \item \lstinline|shmat()|: attach shared memory to address space
  \item \lstinline|shmdt()|: detach shared memory
  \item \lstinline|shmctl()|: control shared memory (e.g. delete)
\end{enumerate}

\begin{lstlisting}
#include <stdio.h>
#include <sys/shm.h>

// Create
int shmid = shmget(IPC_PRIVATE, 40, IPC_CREAT | 6000);

// Attach
int *shm = shmat(shmid, NULL, 0);

// Read/Write 
int x = shm[0];
shm[0] = 10;

// Detach and delete
shmdt((char*) shm);
shmctl(shmid, IPC_RMID, 0);
\end{lstlisting}

\textbf{Message Passing} involves a process sending a message to a receiving process. Messages can be explicitly named to sender/receiver or to shared mailbox. Primitives can be blocking (synchronous) or non-blocking (asynchronous).
\begin{enumerate}[$+$]
  \item Portable 
  \item Implicit Synchronization (if blocking primitives)
\end{enumerate}\vspace{-1pt}
\begin{enumerate}[$-$]
  \item Inefficient: usually require OS intervention
  \item Hard to Use: messages limited in size/format
\end{enumerate}\vspace{-1pt}
\begin{enumerate}[\roman*.]
  \item \lstinline|SEND(r, msg)| and \lstinline|RECV(s, msg)|
\end{enumerate}

\colbreak
\textbf{UNIX Pipes} provide a half-duplex (unidirectional)/ full-duplex (bidirectional) byte stream from writing process to reading process(es).
\begin{enumerate}[$+$]
  \item Implicit Synchronization: writer blocks if full, reader blocks if empty
\end{enumerate}\vspace{-1pt}
\begin{enumerate}[\roman*.]
  \item \lstinline|pipe()|: scan read and write fds into \lstinline|int[2]| 
  \item \lstinline|dup()|: duplicate argument fd into lowest unused fd 
  \item \lstinline|dup2()|: duplicate argument fd into specified fd 
  \item \lstinline|read()|, \lstinline|write()|: read/write to fd with data
  \item \lstinline|close()|: close a file descriptor
\end{enumerate}
\begin{lstlisting}
#include <stdio.h>
#include <unistd.h>
#include <string.h>

#define READ_END 0 
#define WRITE_END 1 

int fd[2];
char buffer[20];
char *str = "test";

pipe(fd);

if (fork() > 0) { // Parent
  close(fd[READ_END]);
  write(fd[WRITE_END],
        str,
        strlen(str) + 1);
} else {          // Child
  close(fd[WRITE_END]);
  read(fd[READ_END],
       buffer,
       sizeof(buffer));
  close(fd[READ_END]);
  printf("Received: %s\n", buffer);
}
\end{lstlisting}

\colbreak
\textbf{UNIX Signals} are asynchronous notifications sent to processes to inform of events. Signals must be explicitly handled by default handler or user defined handler*. 
\begin{enumerate}[\roman*.]
  \item Common Signals:
      \begin{itemize}[leftmargin=*]
        \item SIGINT: interrupt (Ctrl-C)
        \item SIGTERM: terminate request
        \item SIGSEGV: segmentation fault
        \item SIGKILL: force termination (no handling*)
        \item SIGSTOP: pause execution (no handling*)
        \item SIGCONT: resume a stopped process
      \end{itemize}
  \item \lstinline|signal()|: assign a handler to a signal
  \item \lstinline|kill()|: send SIGKILL
  \item \lstinline|raise(sig)|: send signal to current process
\end{enumerate}

\begin{lstlisting}
#include <stdio.h>
#include <stdlib.h>
#include <signal.h> 

void seg_handler(int signo) {
  if (signo == SIGSEGV) {
    printf("Caught SIGSEGV\n");
    exit(1);
  }
}

int main() {
  int *test = NULL;
  if (signal(SIGSEGV, seg_handler) == SIG_ERR)
    printf("Registration failed\n");
  *test = 10;
}
\end{lstlisting}

\colbreak
\subsection{Threads}
Threads are lightweight processes that share memory with other threads in the same process, managed by thread tables.
\begin{enumerate}[$+$]
  \item Economical: cheaper creation and context switching
  \item Resource Sharing: process resources can be shared
  \item Concurrent: multithreading improves responsiveness and can benefit from multiple CPUs
\end{enumerate}\vspace{-1pt}
\begin{enumerate}[$-$]
  \item Parallel System Calls may be called and must be checked for correctness
  \item Process Operations (e.g. \lstinline|fork()|) must be consistent when executed with threads
\end{enumerate}

Resource Distribution:
\begin{enumerate}[\roman*.]
  \item Shared: Text, Data, Heap, Open Files 
  \item Private: Thread ID, Registers, Stack
\end{enumerate}

POSIX System Calls:
\begin{enumerate}[\roman*.]
  \item \lstinline|pthread_create()|: create thread
  \item \lstinline|pthread_exit()|: exit calling thread
  \item \lstinline|pthread_join()|: wait for thread to exit
\end{enumerate}

\subsubsection{Implementations}
\textbf{User Threads} implements threads per-process as library. 
\begin{enumerate}[$+$]
  \item Portable: implementable on all OSes 
  \item Ease of Use: just library calls 
  \item Configurable: user can self-manage synchronisation
\end{enumerate}\vspace{-1pt}
\begin{enumerate}[$-$]
  \item Threads blocking will block whole process
\end{enumerate}

\textbf{Kernel Threads} implements threads in the OS.
\begin{enumerate}[$+$]
  \item Parallel: threads of same process can run across many CPUs 
\end{enumerate}\vspace{-1pt}
\begin{enumerate}[$-$]
  \item Slower: thread operations are slower system calls
  \item Less Flexible
\end{enumerate}

\textbf{Hybrid Threads} map a number of user threads to a number of kernel threads.
\begin{enumerate}[$+$]
  \item Maintains flexibility, efficiency and concurrency
\end{enumerate}

\colbreak
\subsection{Synchronization}
Synchronization is vital at critical sections (CS): memory regions modified concurrently by processes that are subject to errors caused by race conditions.

Correctness Properties:
\begin{enumerate}[\roman*.]
  \item Mutual Exclusion: at most $1$ process in CS
  \item Progress: if no process is in CS, a waiting process should be granted access 
  \item Bounded Wait: after process $P_i$ requests to enter CS, there exists upper bound of times other processes can enter before $P_i$
  \item Independence: process not executing in CS should not block other process 
\end{enumerate}

Symptoms of Incorrect Synchronization:
\begin{enumerate}[\roman*.]
  \item Deadlock: all processes are blocked
  \item Livelock: due to deadlock avoidance, processes are not blocked but not making progress 
  \item Starvation: a process is unable to access CS 
\end{enumerate}

\subsubsection{Implementations} \textbf{Test and Set Lock (TSL)}: atomic assembly instruction that loads value at memory location into the register and changes the value at memory location to 1.\\Variants: Compare and Swap, Atomic Swap (XCHG), Load Link / Store Conditional 
\begin{enumerate}[$+$]
  \item Atomic 
\end{enumerate}\vspace{-1pt}
\begin{enumerate}[$-$]
  \item Busy Waiting: wastes processing power
\end{enumerate}
\vspace{2pt}
\begin{lstlisting}
enter_region: TSL REG, LOCK 
              JNE REG, $0, enter_region 
              RET 

leave_region: MOV LOCK, $0 
              RET
\end{lstlisting}

\colbreak

\textbf{Peterson's Algorithm}: 2-process software lock using flags and a turn variable.
\begin{enumerate}[$-$]
  \item Busy Waiting: wastes processing power
  \item Low Level: error prone
  \item Not General: limited to just mutual exclusion 
\end{enumerate}
\begin{lstlisting}
// Process i
flag[i] = 1;
turn = j;
while (flag[j] && turn == j);
// critical section
flag[i] = false;
\end{lstlisting}

\textbf{Semaphores}: protected integer accessed only via two atomic operations \lstinline|WAIT(S)| and \lstinline|SIGNAL(S)|.\\Binary semaphore or mutex has values $0$ or $1$.
\begin{enumerate}[\roman*.]
  \item \lstinline|WAIT(S)|: if $S \leq 0$ block; else $S \leftarrow S-1$
  \item \lstinline|SIGNAL(S)|: $S \leftarrow S+1$, unblock one process if any
  \item Invariant: $S_{curr} = S_{init} + \#SIGNAL(S) - \#WAIT(S)$; s.t. $\#SIGNAL(S)$ is number of \lstinline|SIGNAL(S)| executed and $\#WAIT(S)$ is number of \lstinline|WAIT(S)| completed
  \item \lstinline|pthread_mutex_lock()|, \lstinline|pthread_mutex_unlock()|
\end{enumerate}

\textbf{Conditional Variables}: allow a process to wait for a condition and be signaled when condition occurs.
\begin{enumerate}[\roman*.]
  \item \lstinline|pthread_cond_wait()|, \lstinline|pthread_cond_signal()|, \lstinline|pthread_cond_broadcast()|
\end{enumerate}

\colbreak
\subsubsection{Producer-Consumer}
Processes share a bounded buffer of fixed size $K$, where producers add items until buffer full and consumers remove items when not empty.

Busy Waiting Solution:\\
\begin{minipage}{0.45\columnwidth}
  \begin{lstlisting}
// Producer
while (1) {
  x = produce();

  while (!can_prod);

  wait(mutex);
  if (K > count) {
    buf[in] = x;
    in = (in+1)%K;
    count++;
    can_cons = 1;
  } else {
    can_prod = 0;
  }
  signal(mutex);
}
  \end{lstlisting}
\end{minipage}
\begin{minipage}{0.5\columnwidth}
  \begin{lstlisting}
// Consumer
while (1) {
  while (!can_cons);

  wait(mutex);
  if (count > 0) {
    x = buf[out];
    out = (out+1)%K;
    count--;
    can_prod = 1;
  } else {
    can_cons = 0;
  }
  signal(mutex);

  consume(item);
}

  \end{lstlisting}
\end{minipage}

Blocking Solution:\\
\begin{minipage}{0.45\columnwidth}
  \begin{lstlisting}
// Producer
while (1) {
  x = produce();

  wait(not_full);
  wait(mutex);
  buf[in] = x;
  in = (in+1)%K;
  count++;
  signal(mutex);
  signal(not_emp);
}
  \end{lstlisting}
\end{minipage}
\begin{minipage}{0.5\columnwidth}
  \begin{lstlisting}
// Consumer
while (1) {
  wait(not_emp);
  wait(mutex);
  x = buf[out];
  out = (out+1)%K;
  count--;
  signal(mutex);
  signal(not_full);

  consume(item);
}
  \end{lstlisting}
\end{minipage}

\colbreak
\subsubsection{Reader-Writer}
Processes share a critical region, where readers can read simultaneously but writers must have exclusive access.

Solution:\\
\begin{minipage}{0.45\columnwidth}
  \begin{lstlisting}
// Writer
while (1) {
  wait(empty);
  write();
  signal(empty);
}
  \end{lstlisting}
\end{minipage}
\begin{minipage}{0.5\columnwidth}
  \begin{lstlisting}
// Reader 
while (1) {
  wait(mutex);
  readers++;
  if (readers == 1)
    wait(empty);
  signal(mutex);

  read();

  wait(mutex);
  readers--;
  if (readers == 0)
    signal(empty);
  signal(mutex);
}
  \end{lstlisting}
\end{minipage}
\begin{enumerate}[\roman*.]
  \item \lstinline|mutex|: syncs readers 
  \item \lstinline|empty|: sync readers and writers 
  \item Only the first reader to enter and last reader to exit will update \lstinline|empty|
\end{enumerate}
\colbreak
\subsubsection{Dining Philosophers}
$N$ philosophers around a circular table, with a single chopstick between. $2$ chopsticks are needed to eat.

Tanenbaum's Solution:\\
\begin{minipage}{0.48\columnwidth}
  \begin{lstlisting}
// Philosopher
void p(int i) {
  while (1) {
    // think
    take(i);
    eat();
    put(i);
  }
}

void safe_to_eat (int i) {
  if (state[i] == HUNGRY && state[LEFT] != EATING && state[RIGHT] != EATING) {
    state[i] = EATING;
    signal(s[i]);
  }
}
  \end{lstlisting}
\end{minipage}
\begin{minipage}{0.45\columnwidth}
  \begin{lstlisting}
// Take Chopstick
void take(int i) {
  wait(mutex);
  state[i] = HUNGRY;
  safe_to_eat(i);
  signal(mutex);
  wait(s[i]);
}

// Put Chopstick
void put(int i) {
  wait(mutex);
  state[i] = THINK;
  safe_to_eat(LEFT);
  safe_to_eat(RIGHT)
  signal(mutex);
}
  \end{lstlisting}
  \vfill
\end{minipage}
Limited Eater Solution, when at most $N-1$ philosophers eat concurrently, it's guaranteed at least one can eat:
\begin{lstlisting}
void philosopher(int i) {
  while (1) {
    // think 
    wait(seats);
    wait(chopstick[LEFT]);
    wait(chopstick[RIGHT]);
    eat();
    signal(chopstick[RIGHT]);
    signal(chopstick[LEFT]);
    signal(seats);
  }
}
\end{lstlisting}
\section{Memory Management}
Memory management ensures that processes in main memory can run concurrently, safely, and efficiently.
\begin{enumerate}[\roman*.]
  \item \textbf{Static Relocation}: all memory references modified by offset at load time. 
    \begin{enumerate}[$-$]
      \item Slow loading time
      \item Difficult to identify memory references
    \end{enumerate}
  \item \textbf{Dynamic Relocation}: special base and limit registers store start and size of process.
    \begin{enumerate}[$-$]
      \item Slow runtime access (add base + limit check)
    \end{enumerate}
  \item \textbf{Logical Addresses}: programs have independent virtual address space which Memory Management Unit (MMU) maps to physical addresses.
\end{enumerate}

\subsection{Contiguous Memory Allocation}
Processes allocated contiguous blocks of physical memory. 
\subsubsection{Swapping}
Inactive or blocked processes are moved to secondary storage to free main memory until they are needed again.

\textbf{Fixed Partitioning} divides memory into fixed-size partitions.
\begin{enumerate}[$+$]
  \item Easy to manage
  \item Fast allocation (all free partitions are the same)
\end{enumerate}\vspace{-1pt}
\begin{enumerate}[$-$]
  \item Internal fragmentation
  \item Size must be large enough for largest process
\end{enumerate}

\textbf{Variable Partitioning} divides memory into process-dependent partition sizes. Allocation requires OS to maintain a list/bitmap of partitions and holes.
\begin{enumerate}[$+$]
  \item No internal fragmentation
\end{enumerate}\vspace{-1pt}
\begin{enumerate}[$-$]
  \item Complex to manage
  \item External fragmentation, requires time-consuming compaction to consolidate holes
  \item Slower allocation to find space:
    \begin{itemize}[leftmargin=*]
    \vspace{3pt}
    \item First Fit: use first hole large enough
    \item Best Fit: find smallest hole large enough
    \item Worst Fit: find largest hole
  \end{itemize}
\end{enumerate}
\vspace{-1em}
\colbreak
\subsection{Disjoint Memory Allocation}
Processes can occupy non-contiguous memory regions.\\ OS maintains maps between logical and physical memory.

\subsubsection{Paging}
Splits logical memory into fixed-size pages mapped by page table into physical memory with equal-sized frames.
\begin{enumerate}[$+$]
  \item No external fragmentation
  \item Protection with page table entry bits:
    \vspace{2pt}
    \begin{itemize}[leftmargin=*]
      \item Access Rights: write, read, exec
      \item Valid: if process active; memory access checked against this bit to catch out-of-range access 
    \end{itemize}
\end{enumerate}\vspace{-1pt}
\begin{enumerate}[$-$]
  \item Internal fragmentation (last page may not be full)
  \item Require two memory accesses per memory reference (lookup page table entry $+$ access physical address) 
\end{enumerate}

Logical addresses are grouped into pages by their higher-order page bits and mapped to a frame:
\begin{enumerate}[\roman*.]
  \item $2^N$ byte page size $\implies$ $N$ offset bits
\end{enumerate}
{\centering\incimg[0.6]{page}\par}

Translation Lookaside Buffer (TLB) caches page table entries to avoid repeated memory lookups. On context switch, TLB entries are flushed.
\begin{enumerate}[\roman*.]
  \item Hit: frame number used to find physical address
  \item Miss: retrieve page table entry from page table, then use frame number to find physical address
\end{enumerate}

Page sharing involves multiple processes's entries to map to the same frame (e.g. shared libraries, \lstinline|fork()|)
\begin{enumerate}[\roman*.]
  \item Copy-on-Write: initially \lstinline|write| bit $= 0$; on write, OS duplicates frame and updates page table with new frame number, \lstinline|write| $= 1$, and modifies frame
\end{enumerate}
\vspace{-1em}
\colbreak
\subsubsection{Segmentation}
Splits logical memory into contiguous variable-sized segments, each with base and limit in a segment table.
\begin{enumerate}[$+$]
  \item Segments can grow/shrink independently 
  \item Segments can be protected/shared independently
\end{enumerate}\vspace{-1pt}
\begin{enumerate}[$-$]
  \item External fragmentation possible
\end{enumerate}

Logical addresses are grouped by segment ID and mapped to a segment:
\begin{enumerate}[\roman*.]
  \item Logical Addr. $=$ $<$\lstinline|SegID, Offset|$>$; where \lstinline|SegID| maps a segment's $<$\lstinline|Base, Limit|$>$ in segment table
  \item Physical Addr. $=$ \lstinline|Base + Offset|;\\where \lstinline|Offset| $<$ \lstinline|Limit| if valid
\end{enumerate}
{\centering\incimg[0.8]{segmentation}\par}
\vspace{-1em}
\subsubsection{Segmentation with Paging}
Segments are divided into fixed-size pages instead of being stored contiguously. Each segment has its own page table. Segments grow/shrink by adding or removing pages.

{\centering\incimg[0.8]{segmentationwithpaging}\par}
\vspace{-1em}
\colbreak
\subsection{Virtual Memory}
Virtual memory enables larger programs and more programs to run on limited main memory.
\begin{enumerate}[\roman*.]
  \item Pages are stored in secondary storage (e.g. disk)
  \item Pages are loaded to main memory on-demand
  \item Page table uses \lstinline|isResident| bit to track if pages are memory resident (in physical memory) or non-memory resident (in secondary storage)
\end{enumerate}

CPU can only access memory resident pages.\\Attempt to access non-memory resident causes page fault:
\begin{enumerate}[\roman*.]
  \item Hardware traps to OS (page fault interrupt)
  \item OS locates page on disk 
  \item Run page replacement and update page table
  \item Restart instruction that caused page fault
\end{enumerate}

\subsubsection{Page Table Structures}
\textbf{Direct Paging} keeps all page table entries in a single table.
\begin{enumerate}[$-$]
  \item Too large for big address spaces\\(e.g. $2^{20}$ entries for 32-bit addresses, 4KB pages)
\end{enumerate}\vspace{-1pt}
\begin{enumerate}[\roman*.]
  \item $M$-bit address and $2^N$ byte page size\\$\implies N$ offset bits and $M-N$ page number bits 
\end{enumerate}

\textbf{2-Level Paging} divides page table into smaller page tables indexed via a page directory.
\begin{enumerate}[$+$]
  \item Space-efficient: only allocates page subtables for used regions
\end{enumerate}\vspace{-1pt}
\begin{enumerate}[\roman*.]
  \item $2^P$ direct paged table entries and $2^K$ subtables\\$\implies K$ index bits and $2^{P-K}$ entries per subtable
\end{enumerate}
%{\centering\incimg[0.7]{2level}\par}

\textbf{Inverted Page Table} keeps one entry per frame, each entry mapped to the current to the current occupying process and page's $<$PID, PageNum$>$.
\begin{enumerate}[$+$]
  \item Space-efficient: only allocate for smaller number of physical page frames
  \item Frame management is easier and faster
\end{enumerate}\vspace{-1pt}
\begin{enumerate}[$-$]
  \item Slower lookup; requires searching or hashing
\end{enumerate}

\colbreak
\subsubsection{Page Replacement Algorithms}

Page replacement occurs when there is no free physical frame during a page fault, and page eviction is required. 

Evicted Page Write Policy:
\begin{enumerate}[\roman*.]
  \item Clean Page: no need to write back
  \item Dirty Page: write back to secondary memory
\end{enumerate}

Time for memory access, with probability of page fault $p$:
\vspace{-3pt}
\begin{align*}
  T_{access} = (1-p)\cdot T_{mem} + p\times T_{page fault}
\end{align*}
\textbf{Optimal (OPT)} evicts page not reused for longest time.
\begin{enumerate}[$+$]
  \item Guarantees minimal page fault rate
\end{enumerate}\vspace{-1pt}
\begin{enumerate}[$-$]
  \item Not implementable (needs future knowledge)
\end{enumerate}

\textbf{FIFO} evicts oldest loaded page.
\begin{enumerate}[$+$]
  \item Easy implementation with queue
\end{enumerate}\vspace{-1pt}
\begin{enumerate}[$-$]
  \item Belady's Anomaly: more frames can increase faults (because not exploiting temporal locality)
\end{enumerate}

\textbf{LRU} evicts least recently used page.
\begin{enumerate}[$+$]
  \item Approximates OPT, giving good results
  \item Does not suffer from Belady's Anomaly
\end{enumerate}\vspace{-1pt}
\begin{enumerate}[$-$]
  \item Hard to implement: needs counter or stack
\end{enumerate}

\textbf{Second Chance (CLOCK)} evists oldest loaded page after giving a second chance to referenced page.
\begin{enumerate}[\roman*.]
  \item Store pages in a circular queue with reference bits
  \item On page load, set reference bit to $0$
  \item On page access, set reference bit to $1$
  \item During page replacement, if oldest page reference bit is $1$, unset and requeue, else replace it 
\end{enumerate}

\colbreak
\subsubsection{Working Set Model}
Working Set Model reduces page faults by ensuring each process has enough frames to hold its active pages.
\begin{enumerate}[label=\roman*., leftmargin=*]
  \item $W(t, \Delta)$ = active pages in last $\Delta$ references at time $t$
  \item Allocate $\geq |W(t, \Delta)|$ frames to avoid excessive faults
\end{enumerate}

Choosing $\Delta$:
\begin{enumerate}[\roman*.]
  \item Too small $\Rightarrow$ working set misses some active pages
  \item Too large $\Rightarrow$ includes pages from different locality
\end{enumerate}

Thrashing is when a process causes frequent page faults (e.g. every few instructions), typically because its working set exceeds its allocated frame count. 

\subsubsection{Frame Allocation}

Proportion:
\begin{enumerate}[\roman*.]
  \item Equal: all processes get equal share of frames
  \item Proportional: frames distributed by process size
\end{enumerate}

Scope:
\begin{enumerate}[\roman*.]
  \item Local: replace page of same process on page fault
    \vspace{3pt}
    \begin{enumerate}[$+$]
      \item Frames per process are constant every run 
    \end{enumerate}\vspace{3pt}
    \begin{enumerate}[$-$]
      \item If frames per process are not enough, progress of process is hindered
      \item Thrashing can be limited to a process, but it will hog I/O and degrade others' performance
    \end{enumerate}
  \item Global: replace page of any process
    \vspace{3pt}
    \begin{enumerate}[$+$]
      \item Allow self-adjustment between processes 
    \end{enumerate}\vspace{3pt}
    \begin{enumerate}[$-$]
      \item Badly behaving processes can affect others 
      \item Frames per process can be different every run
      \item Cascading Thrashing: thrashing process steals pages, causing other processes to thrash
    \end{enumerate}
\end{enumerate}

\colbreak
\section{File Systems}
File systems manage data on secondary storage in a safe, persistent, self-contained and efficient manner. 

\subsection{Files}
Files represent data created by process, and its metadata: 
\begin{enumerate}[\roman*.]
  \item Name: human-readable reference to the file;\\extensions denote filetype (magic numbers can too) 
  \item Identifier: unique ID used by file system
  \item Type:\vspace{2pt}
    \begin{itemize}[leftmargin=*]
      \item Regular File: ASCII/ Binary data 
      \item Directory 
      \item Special File: UNIX Character/ Block files
    \end{itemize}
  \item Size: file size 
  \item Protection: access rights (e.g. read/write/execute)\vspace{2pt} 
    \begin{itemize}[leftmargin=*]
      \item UNIX Permission Bits: \lstinline|rwx| for owner, group, all 
      \item Access Control List: granular permissions on per-user or per-group basis
    \end{itemize}
  \item Time, Data and Owner Information 
  \item Table of Content: determines access method 
\end{enumerate}

Data Structures:
\begin{enumerate}[\roman*.]
  \item Byte Sequence: just raw bytes\vspace{3pt}
    \begin{enumerate}[$+$]
      \item Most flexible
    \end{enumerate}
  \item Fixed-Length Records: array of records 
    \begin{enumerate}[$+$]
      \item Easy to jump to any record:\vspace{3pt}
      \begin{itemize}[leftmargin=*]
        \item Offset of $n^{th}$ = size of record * ($N-1$)
      \end{itemize}
    \end{enumerate}
  \item Variable-Length Records\vspace{2pt}
    \begin{enumerate}[$+$]
      \item Flexible
    \end{enumerate}\vspace{3pt}
    \begin{enumerate}[$-$]
      \item Harder to find records
    \end{enumerate}
\end{enumerate}

Access Methods:
\begin{enumerate}[\roman*.]
  \item Sequential: read in order, cannot skip but can rewind
  \item Random: read in any order using either \lstinline|read(offset)| or \lstinline|seek(offset)|
  \item Direct: random access for records
\end{enumerate}

\colbreak
Two-Table Approach:
\begin{enumerate}[\roman*.]
  \item System-wide: one entry to file/ inode per open file 
  \item Per-process: one file descriptor to system-wide table entries per open file by process\vspace{1pt}
    \begin{itemize}[leftmargin=*]
      \item Point to two entries if opened by two processes
      \item Point to same entry if after \lstinline|fork()| or \lstinline|dup()|
    \end{itemize}
\end{enumerate}

POSIX System Calls:
\begin{enumerate}[\roman*.]
  \item \lstinline|open()|: open a file and return file descriptor
  \item \lstinline|read()|, \lstinline|write()|: read/write with file descriptor
  \item \lstinline|lseek()|: move pointer of file descriptor
  \item \lstinline|close()|: release file descriptor
\end{enumerate}

\subsubsection{Implementations}

\textbf{Contiguous Allocation} allocates consecutive blocks. 
\begin{enumerate}[$+$]
  \item Easy tracking: files just need start + length
  \item Fast access: just need to seek first block
\end{enumerate}\vspace{-1pt}
\begin{enumerate}[$-$]
  \item External fragmentation: gaps after deletion
  \item File size must be specified in advance
\end{enumerate}

\textbf{Linked List (LL)} stores each disk block in nodes, with first and last node per file.
\begin{enumerate}[$+$]
  \item No external fragmentation
\end{enumerate}\vspace{-1pt}
\begin{enumerate}[$-$]
  \item Slow random access: must traverse linked list
  \item Extra pointer per block
\end{enumerate}

\textbf{File Allocation Table (FAT)} is same as LL but stores all pointers in a FAT in memory.
\begin{enumerate}[$+$]
  \item Faster access than LL: traversal now in memory
\end{enumerate}\vspace{-1pt}
\begin{enumerate}[$-$]
  \item Takes up space: FAT tracks all disk blocks
\end{enumerate}

\textbf{Indexed Allocation (inode)} uses one block per file to store pointers to all other blocks.
\begin{enumerate}[$+$]
  \item Less memory overhead: only index block of opened files need to be in memory 
  \item Fast direct access
\end{enumerate}\vspace{-1pt}
\begin{enumerate}[$-$]
  \item Limited max file size: number of blocks bounded by max array size of index block 
  \item Index block overhead
\end{enumerate}
\vspace{-1em}
\colbreak
\subsection{Directories}
Directories organise and manage files in a system.

Structures:
\begin{enumerate}[\roman*.]
  \item Single-Level: all files in one root directory.
  \item Tree: hierarchical tree of directories and files.
  \item DAG: possible when files are shared \vspace{3pt}
    \begin{itemize}[leftmargin=*]
      \item Hard Link: directory entries share same i-node
        \begin{enumerate}[$+$]
          \item Low overhead: just represent with pointers 
        \end{enumerate}\vspace{3pt}
        \begin{enumerate}[$-$]
          \item Deletion problems: when to delete file; resolve with storing count of references
        \end{enumerate}
      \item Symbolic Link: symbolic link entry is a special link to target filepath which must be followed 
        \begin{enumerate}[$+$]
          \item Simple deletion: only delete file when owner deletes; symbolic links stay but don't work
        \end{enumerate}\vspace{3pt}
        \begin{enumerate}[$-$]
          \item Larger overhead: special link file takes space 
        \end{enumerate}
    \end{itemize}
  \item General: possible when symbolic links create cycle\vspace{1pt} 
    \begin{enumerate}[$-$]
      \item Hard to traverse 
      \item Hard to know when to remove a file/directory
    \end{enumerate}
\end{enumerate}

POSIX System Calls:
\begin{enumerate}[label=\roman*., leftmargin=*]
  \item \lstinline|mkdir(),rmdir()|: create/remove a directory
  \item \lstinline|chdir(),getcwd()|: change/ get current working dir
  \item \lstinline|link(),symlink()|: create hard/ symbolic link
  \item \lstinline|unlink()|: remove link from the filesystem 
\end{enumerate}
\vspace{-3pt}
\subsubsection{Implementations}

\textbf{Linked List} stores directory entries sequentially.
\begin{enumerate}[$+$]
  \item Simple to implement
\end{enumerate}\vspace{-1pt}
\begin{enumerate}[$-$]
  \item Slow search: must traverse entries linearly
\end{enumerate}

\textbf{Hash Table} maps filenames to files using hash function.
\begin{enumerate}[$+$]
  \item Fast lookup
\end{enumerate}\vspace{-1pt}
\begin{enumerate}[$-$]
  \item Hash table has limited size
  \item Depends on good hash function
\end{enumerate}

File information storage:
\begin{enumerate}[\roman*.]
  \item Store everything in directory entry 
  \item Store name in directory entry, and point to a different structure for other data
\end{enumerate}
\vspace{-2em}
\colbreak
\subsection{I/O}
Disk I/O has high latency due to seek and rotation delays; scheduling improves performance.

Disk Structure:
\begin{enumerate}[\roman*.]
  \item Track: one ring around the disk has many tracks of different radii 
  \item Sector: sector of a track 
  \item Disk head: point moving above the disk which transforms its magnetic field into electric current or vice versa 
  \item By rotating, we change sector 
  \item By shifting the disk head between the center and the circumference, we change track
\end{enumerate}

Scheduling Algorithms:
\begin{enumerate}[\roman*.]
  \item FCFS (First Come First Serve)
  \item SSTF (Shortest Seek Time First)
  \item SCAN (Elevator): services in one direction, then reverses
  \item C-SCAN (Circular SCAN): only services in one direction
  \item Deadline: ensures deadline-based read/write ordering
  \item CFQ (Completely Fair Queuing): assigns time slices to processes
  \item BFQ (Budget Fair Queuing): shares I/O based on number of sectors requested
\end{enumerate}

\colbreak
\subsection{Disk Organisation}

In BIOS:
\begin{enumerate}[\roman*.]
  \item Master Boot Record (MBR) at sector 0 contains boot code and partition table.
  \item Partitions can contain independent file systems
\end{enumerate}

{\centering\incimg[0.8]{diskalloc}\par}

\subsection{Free Space Management}
Tracks free disk blocks on each partition to support file allocation. 

\textbf{Bitmap} represents each disk block by 1 bit.
\begin{enumerate}[$+$]
  \item Easy to manipulate: use bit operations to find first free block or n-consecutive free blocks. 
\end{enumerate}\vspace{-1pt}
\begin{enumerate}[$-$]
  \item Need to keep in memory for efficiency reasons 
\end{enumerate}

\textbf{Linked List} maintains a list of free blocks by their indices.
\begin{enumerate}[$+$]
  \item Easy to find free block: take any block 
  \item Only first pointer needed (but others can be cached for efficiency)
\end{enumerate}\vspace{-1pt}
\begin{enumerate}[$-$]
  \item High overhead: mitigate by storing list itself in a free block
\end{enumerate}

\colbreak
\subsection{Case Studies}
\subsubsection{Microsoft File Allocation Table (FAT)}
FAT is used in early DOS and removable devices.

{\centering\incimg[0.6]{fat}\par}
Directory is represented as a special file, where each directory entry is fixed 32-bytes storing:
\begin{enumerate}[\roman*.]
  \item Filename: max. 8+3 characters; first byte may have special meaning
  \item Attributes (e.g. read-only, hidden)
  \item Creation Time: years $1980-2017$, time $\pm 2$s
  \item First Disk Block Index: X-byte length for FATX 
  \item File Size
\end{enumerate}

{\centering\incimg[0.7]{fatdir}\par}

FAT entries contains FREE (block unused), Block number of next block, EOF (NULL ptr) or BAD (unusable)

{\centering\incimg[0.65]{fatfile}\par}
\vspace{-1em}
\subsubsection{Extended-2 File System (Ext2)}
Ext2 is a UNIX file system used in early Linux distros.

{\centering\incimg[0.8]{ext2}\par}

Block Group Structure:
\begin{enumerate}[\roman*.]
  \item Superblock: describe whole file system (e.g. total inodes, disk blocks) and duplicated in each block group for redundance 
  \item Group Descriptor Table: describe each block group (e.g. number of free disk bocks, free inodes, location of bitmaps) and duplicated in each block group
  \item Block Bitmap: tracks used/free blocks
  \item inode Bitmap: tracks used/free inodes
  \item inode Table: stores inodes
  \item Data Blocks
\end{enumerate}
\colbreak 

inode Structure of 128 bytes:\\
{\centering\incimg[0.7]{inode}\par}

Multilevel Data Blocks:
\begin{enumerate}[\roman*.]
  \item 12 direct: point to actual data blocks
  \item 1 single indirect: points to block with pointers to data blocks
  \item 1 double indirect: points to block that points to blocks of pointers
  \item 1 triple indirect: points to block that points to blocks of blocks of pointers
\end{enumerate}

Directories are files containing directory entries:
\begin{enumerate}[\roman*.]
  \item inode Number: 0 indicates unused entry 
  \item Size of Entry: for traversal to next entry 
  \item Length of Name 
  \item Type: regular file, directory or special 
  \item Name of File: max. 255 characters
\end{enumerate}

{\centering\incimg[0.8]{ext2dict}\par}

\colbreak
Accessing a file (e.g. \lstinline|/sub/file|):
\begin{enumerate}[\roman*.]
  \item Get root directory inode (usually fixed at 2)
  \item If next slug is subdirectory, locate directory entry, read the inode, change working directory and repeat after step 1 
  \item Else if its a file, locate directory entry and read the inode terminating
\end{enumerate}

{\centering\incimg[0.75]{ext2file}\par}

Open:
\begin{enumerate}[\roman*.]
  \item Trace. If file is not found, throw error.
  \item Load file information with new entry in the system-wide table.
  \item Create a file descriptor in process's table to point to the new entry
  \item Return fd of this entry.
\end{enumerate}

Delete:
\begin{enumerate}[\roman*.]
  \item Remove its directory entry by pointing previous entry to next entry. If its the first entry, make it a blank record.
  \item Update inode bitmap and mark the inode as free.
  \item Update block bitmap and mark blocks as free.
\end{enumerate}

Hard Link:
\begin{enumerate}[\roman*.]
  \item Create new directory entry with same inode target.
  \item Update inode reference count.
  \item For deletion, decrement reference count; if it hits 0 perform actual deletion.
\end{enumerate}

Symbolic Link:
\begin{enumerate}[\roman*.]
  \item Create new file and corresponding directory entry 
  \item Store pathname of target file in new file 
\end{enumerate}

%%%%%%%%%%%%%%%%%%%%%%%%%%%%%%%%%%%%%%%%%%%%%%%%%%%%%%
%                       End                          %
%%%%%%%%%%%%%%%%%%%%%%%%%%%%%%%%%%%%%%%%%%%%%%%%%%%%%%
\end{multicols*}
\end{document}

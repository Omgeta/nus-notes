\documentclass[12pt, a4paper]{article}

\usepackage[utf8]{inputenc}
\usepackage[mathscr]{euscript}
\let\euscr\mathscr \let\mathscr\relax
\usepackage[scr]{rsfso}
\usepackage{amssymb,amsmath,amsthm,amsfonts}
\usepackage[shortlabels]{enumitem}
\usepackage{multicol,multirow}
\usepackage{lipsum}
\usepackage{balance}
\usepackage{calc}
\usepackage[colorlinks=true,citecolor=blue,linkcolor=blue]{hyperref}
\usepackage{import}
\usepackage{xifthen}
\usepackage{pdfpages}
\usepackage{transparent}
\usepackage{tabularx}

\newcommand{\incfig}[2][1.0]{
    \def\svgwidth{#1\columnwidth}
    \import{./figures/}{#2.pdf_tex}
}
\newcommand{\incimg}[2][1.0]{
  \includegraphics[width=#1\columnwidth]{./figures/#2}
}


\usepackage{ifthen}
\usepackage[landscape]{geometry}
\usepackage[shortlabels]{enumitem}

\ifthenelse{\lengthtest { \paperwidth = 11in}}
    { \geometry{top=.5in,left=.5in,right=.5in,bottom=.5in} }
	{\ifthenelse{ \lengthtest{ \paperwidth = 297mm}}
		{\geometry{top=1cm,left=1cm,right=1cm,bottom=1cm} }
		{\geometry{top=1cm,left=1cm,right=1cm,bottom=1cm} }
	}

\pagestyle{empty}
\makeatletter
\renewcommand\thesection{\arabic{section}.}
\renewcommand{\section}{\@startsection{section}{1}{0mm}%
                                {-1ex plus -.5ex minus -.2ex}%
                                {0.05ex}%x
                                {\normalfont\normalsize\bfseries}}
\renewcommand{\subsection}{\@startsection{subsection}{2}{0mm}%
                                {-1ex plus -.5ex minus -.2ex}%
                                {0.05ex}%
                                {\normalfont\small\bfseries}}
\renewcommand{\subsubsection}{\@startsection{subsubsection}{3}{0mm}%
                                {-1ex plus -.5ex minus -.2ex}%
                                {0.05ex}%
                                {\normalfont\footnotesize\bfseries}}
\newcommand{\colbreak}{\vfill\null\columnbreak}
\makeatother
\setcounter{secnumdepth}{1}
\setlength{\parindent}{0pt}
\setlength{\parskip}{0.7em}

\setlist[itemize]{itemsep=0.6ex, topsep=-2pt, partopsep=0pt, parsep=0pt}
\setlist[enumerate]{itemsep=0.6ex, topsep=-2pt, partopsep=0pt, parsep=0pt}

\input{letterfonts}

\newcommand{\mytitle}{CS2109S Intro. to AI \& Machine Learning}
\newcommand{\myauthor}{github/omgeta}
\newcommand{\mydate}{AY 25/26 Sem 1}

\begin{document}
\raggedright
\footnotesize
\begin{multicols*}{3}
\setlength{\premulticols}{1pt}
\setlength{\postmulticols}{1pt}
\setlength{\multicolsep}{1pt}
\setlength{\columnsep}{2pt}

{\normalsize{\textbf{\mytitle}}}\\
{\footnotesize{\mydate\hspace{2pt}\textemdash\hspace{2pt}\myauthor}}
%%%%%%%%%%%%%%%%%%%%%%%%%%%%%%%%%%%%%%%%%%%%%%%%%%%%%%
%                      Begin                         %
%%%%%%%%%%%%%%%%%%%%%%%%%%%%%%%%%%%%%%%%%%%%%%%%%%%%%%
\vspace{-0.5em}\section{Intelligent Agents}
Agents receive precepts from the environment through sensors and perform actions through actuators.

\subsection{PEAS Framework}
PEAS defines problems in terms of Performance Measure, Environment, Actuators and Sensors:
\begin{enumerate}[\roman*.]
  \item Rational Agent: chooses actions that maximise performance measure
\end{enumerate}
{\centering\incimg[0.55]{structure}\par}

Properties of Task Environment:
\begin{enumerate}[\roman*.]
  \item Fully Observable (vs. Partially Observable):\\sensors give access to the complete state of the environment at each point in time. 
  \item Single-agent (vs. Multi-agent):\\agent operates by itself in an environment.
  \item Deterministic (vs. Stochastic):\\next state of the environment is completely determined by current state and action by agent.
    \begin{itemize}[leftmargin=*]\vspace{3pt}
      \item Strategic: if environment is dependent on actions of other unpredictable (not "dumb") agents 
    \end{itemize}
  \item Episodic (vs. Sequential):\\ agent's experience is divided into atomic "episodes", and choice of action in each episode depends only on the episode itself.
  \item Static (vs. dynamic):\\ environment is unchanged while agent deliberates.
    \begin{itemize}[leftmargin=*]\vspace{3pt}
      \item Semi-dynamic: if environment does not change with time, but agent's performance score does.
    \end{itemize}
  \item Discrete (vs. Continuous):\\a limited number of distinct percepts and actions.
\end{enumerate}\vspace{-1em}
\colbreak

\subsubsection{Agent Structures}
Agents are completely specified by the agent function:
\begin{enumerate}[\roman*.]
  \item Agent Function: $f: \mcP \rightarrow \mcA$ maps from precept histories $\mcP$ to actions $\mcA$
  \item Agent Program: implements the agent function
\end{enumerate}

Common Agent Structures:
\begin{enumerate}[\roman*.]
  \item Simple Reflex:\\ chooses action only based on current percept, ignoring precept history (follow if-then rules).
  \item Goal-based:\\select actions to achieve a given tracked goal.
  \item Utility-based:\\selects actions to maximise a utility function which assigns a score to any precept sequence.\\If the utility function aligns with performance measure, the agent is rational.
  \item Learning:\\improves performance over time with experience.
    \begin{itemize}[leftmargin=*]\vspace{3pt}
      \item Performance Element: selects actions.
      \item Learning Element: updates knowledge from feedback. 
      \item Critic: provides feedback on performance relative to a fixed performance standard.
      \item Problem Generator: suggest exploratory actions
    \end{itemize}
    %\vspace{1em}{\centering\incimg[0.7]{learningagent}\par}
\end{enumerate}

Exploration-Exploitation Dilemma:
\begin{enumerate}[\roman*.]
  \item Explore: learn more about the world.
  \item Exploit: maximize gain from current knowledge.
\end{enumerate}

\section{Systematic Search}

Systematic search problems are fully-observable, deterministic, static, discrete and defined by:
\begin{enumerate}[\roman*.]
  \item States: representation of problem state 
  \item Initial State 
  \item Goal State/Test 
  \item Actions: possible operations on a state
  \item Transition Model: function of action on a state
  \item Action Cost Function
\end{enumerate}
\vspace{-1em}
\colbreak
\subsection{Uninformed Search}
Uninformed search explores a problem space without domain-specific knowledge or heuristics to guide searching.
\begin{enumerate}[\roman*.]
  \item Branching factor $b$: max successors of any node 
  \item Optimal solution depth $d$, Maximum depth $m$ 
\end{enumerate}

Breadth-First Search (BFS) expands nodes level-by-level using a queue:
\begin{enumerate}[\roman*.]
  \item Time: $O(b^{d+1})$ 
  \item Space: $O(b^d)$
  \item Complete: Yes (if $b$ is finite)
  \item Optimal: Yes (if all same step cost)
\end{enumerate}

Uniform-Cost Search (UCS) expands least-cost node using a priority queue, for optimal solution cost $C^*$, levels $C^* /\epsilon$:
\begin{enumerate}[\roman*.]
  \item Time: $O(b^{C^*/ \epsilon})$ 
  \item Space: $O(b^{C^* /\epsilon})$
  \item Complete: Yes (if step costs $> 0$ $\land$ finite total cost)
  \item Optimal: Yes (if step costs $> 0$)
\end{enumerate}

Depth-First Search (DFS) expands deepest unexpanded nodes using a stack:
\begin{enumerate}[\roman*.]
  \item Time: $O(b^{m})$ 
  \item Space: $O(bm)$
  \item Complete: No (if infinite depth $\lor$ cyclic)
  \item Optimal: No 
\end{enumerate}

Depth-Limited Search (DLS) limits search depth to $\ell$:
\begin{enumerate}[\roman*.]
  \item Time: $O(b^{\ell})$ 
  \item Space: $O(b\ell)$ (with DFS)
  \item Complete: No (if $\ell < d$)
  \item Optimal: No (with DFS)
\end{enumerate}

Iterative Deepening Search (IDS) runs DLS with increasing depth limit until solution found:
\begin{enumerate}[\roman*.]
  \item Time: $O(b^d)$ (with additional overhead)
  \item Space: $O(bd)$ (with DFS)
  \item Complete: Yes (if $b$ finite)
  \item Optimal: Yes (if all same step cost)
\end{enumerate}
\colbreak
\subsection{Informed Search}
Informed search uses heuristics to guide searching, where heuristic $h(n)$ estimates optimal cost from a state to goal:
\begin{enumerate}[\roman*.]
  \item All heuristics must be non-negative $\land$ $h(goal) = 0$
\end{enumerate}

Usefulness Properties:
\begin{enumerate}[\roman*.]
  \item Admissible: $\forall$  node $n$, $h(n) \leq h^*(n)$ where $h^*(n)$ is optimal path cost to reach goal state from $n$\\ ($h(n)$ never over-estimates true cost to goal)
    \begin{itemize}[leftmargin=*]\vspace{3pt}
      \item Theorem: if $h(n)$ is admissible, $A^*$ search (without visited memory) is optimal
    \end{itemize}
  \item Consistent: $\forall$ node $n$, $h(n) \leq c(n,a,n') + h(n')$ (triangle inequality)
    \begin{itemize}[leftmargin=*]\vspace{3pt}
      \item Theorem: if $h(n)$ is consistent, $A^*$ search\\(with visited memory) is optimal
      \item Theorem: if $h(n)$ is consistent, $h(n)$ is admissible
    \end{itemize}
  \item Dominance: $\forall$ nodes $n$, $h_1(n) \geq h_2(n)$ implies that $h_1$ dominates $h_2$ (more informed) 
    \begin{itemize}[leftmargin=*]\vspace{3pt}
      \item Theorem: if $h_1$ is admissible, $h_1$ is better for search, expanding no more nodes than $h_2$
    \end{itemize}
\end{enumerate}

Admissible heuristics can be found using optimal cost of a relaxed problem (fewer restrictions on actions):
\begin{enumerate}[Ex.]
  \item $h_{SLD}$ straight-line distance if agent can fly
  \item $h = 0$ if agent can teleport
\end{enumerate}

Best-First Search expands nodes with evaluation $f(n) = h(n)$ using a priority queue:
\begin{enumerate}[\roman*.]
  \item Time: $O(b^m)$ 
  \item Space: $O(b^m)$ 
  \item Complete: No (may get stuck in loops).
  \item Optimal: No (heuristic-only may mislead).
\end{enumerate}

A* expands nodes with evaluation $f(n) = g(n) + h(n)$, where $g(n)$ is step cost, using a priority queue:
\begin{enumerate}[\roman*.]
  \item Time: $O(b^d)$ (good heuristic can improve)
  \item Space: $O(b^d)$ (keeps all nodes in memory).
  \item Complete: Yes (if step costs $> 0$ $\land$ finite $b$).
  \item Optimal: Depends (if $h$ admissible in tree search,\\\hfill or consistent in graph search).
\end{enumerate}
\vspace{-1em}
\colbreak 
\section{Local Search}
Local search problems are defined by:
\begin{enumerate}[\roman*.]
  \item States: representation of candidate solution, may not map to actual problem state
  \item Initial State 
  \item Goal State/Test (optional) 
  \item Successor Function: generate neighbour states by applying modifications to current state
\end{enumerate}

Local search explores large, otherwise intractable problem spaces by considering only locally reachable states, guided by an evaluation function:
\begin{enumerate}[\roman*.]
  \item Evaluation Function: assesses quality of a state;\\we either minimize or maximise this function
\end{enumerate}

State Space Landscape:
\begin{enumerate}[\roman*.]
  \item Global Maximum: optimal solution 
  \item Local Maximum: local optimum solution 
  \item Shoulder: region with flat evaluation function, difficult for algorithm to move past
\end{enumerate}

Hill-Climbing searches for local optimum, generating successors from current state and picking the best using heuristic:
\begin{enumerate}[\roman*.]
  \item Any-time: More time gives better solutions
  \item Space: $O(b)$
  \item Complete: No 
  \item Optimal: Not guaranteed
  \item Variants:
    \begin{itemize}[leftmargin=*]\vspace{3pt}
      \item Simulated Annealing: allow some bad moves, gradually decreasing frequency 
      \item Beam Search: perform parallel $k$ hill-climbs 
      \item Genetic Algorithm: successor generated by combining 2 parent states 
      \item Random-Restart: escape local optimum by restarting search
    \end{itemize}
\end{enumerate}


\colbreak
\section{Adversarial Search}
Adversarial search problems are fully-observable, deterministic-strategic, static, discrete and defined by:
\begin{enumerate}[\roman*.]
  \item States: representation of candidate solution, may not map to actual problem state
  \item Initial State 
  \item Terminal States: where game ends (e.g. win)
  \item Actions: possible operations on a state
  \item Transition Model: function of action on a state
  \item Utility Function: output value of state from the perspective of our agent
\end{enumerate}

Minimax assumes both players play optimally, expanding game tree in DFS with alternating MAX and MIN levels:
\begin{enumerate}[\roman*.]
  \item Time: $O(b^m)$
  \item Space: $O(bm)$ 
  \item Complete: Yes (for finite games)
  \item Optimal: Yes (for both, if both play optimally)
    \begin{itemize}[leftmargin=*]\vspace{3pt}
      \item Theorem: if opponent plays sub-optimally, utility obtained by agent is never less than utility obtained against an optimal opponent
    \end{itemize}
\end{enumerate}

Alpha-Beta Pruning reduces nodes evaluated without altering minimax value and optimal move for root node:
\begin{enumerate}[\roman*.]
  \item Maintains $\alpha$ = best value , $\beta$ = worst value and pruning subtree if $\alpha \geq \beta$ 
  \item Time: $O(b^{m/2})$ (with perfect ordering)
  \item Space: $O(bm)$ (same as minimax)
  \item Complete \& Optimal: same as minimax
\end{enumerate}

Cutoff Strategy halts search in the middle and estimates value of mid-game states with an evaluation function:
\begin{enumerate}[\roman*.]
  \item Evaluation Function: if terminal, use utility function, else use a heuristic 
  \item Heuristic Function: estimates utility of a state
    \begin{itemize}[leftmargin=*]\vspace{3pt}
      \item A heuristic must be between max and min utility
      \item Admissibility and consistency are not needed
    \end{itemize}
  \item Theorem: returns a move that is optimal with respect to the evaluation function at the cutoff; may be suboptimal with respect to true minimax
\end{enumerate}
\vspace{-1em}
\colbreak
\section{Machine Learning}
Machine learning develops learning agents with data.
\begin{enumerate}[\roman*.]
  \item Data, $D = \{1 \leq i \leq n : (x^{(i)}, y^{(i)})\}$ of $n$ samples, for $x^{(i)} \in \RR^d$ feature vector, $y^{(i)}$ label of $i^{th}$ data point. 
  \item Model/Hypothesis, $h: X \rightarrow Y$ is a predictor of outputs which is learned by a learning algorithm.
  \item Performance Measure evaluates $h$ map $x^{(i)} \rightarrow y^{(i)}$.
    \begin{itemize}[leftmargin=*]\vspace{3pt}
      \item ${MAE}  = \frac{1}{n}\sum^n_{i=1} \left|\hat{y}^{(i)}-y^{(i)}\right|$ (outlier robust)
      \item ${Accuracy}  = \frac{1}{n}\sum^n_{i=1} \mathbbm{1}_{\hat{y}^{(i)}=y^{(i)}}$ (classification) 
    \end{itemize}
  \item Loss Function measures $\hat{y}^{(i)}$ error for optimization
    \begin{itemize}[leftmargin=*]\vspace{3pt}
      \item ${MSE}  = \frac{1}{n}\sum^n_{i=1} \left(\hat{y}^{(i)}-y^{(i)}\right)^2$ (outlier sensitive)
    \end{itemize}
\end{enumerate}

Binary Metrics:
\begin{enumerate}[\roman*.]
  \item Accuracy = $\frac{\text{TP + TN}}{\text{TP + FN + FP + TN}}$
  \item Precision, $P = \frac{\text{TP}}{\text{TP + FP}}$\hfill(if FP are costly)
  \item Recall, $R = \frac{\text{TP}}{\text{TP + FN}}$\hfill(if FN are costly)
  \item F1 Score, $F1 = \frac{2}{\frac{1}{P} + \frac{1}{R}}$\hfill(balance precision \& recall)
\end{enumerate}

\subsection{Decision Trees}
Decision trees split data using features to predict outputs.
\begin{enumerate}[\roman*.]
  \item Representational Completeness: any decision tree can fit any finite consistent labelled data exactly 
\end{enumerate}

Entropy: $\displaystyle H(Y)= -\sum_i P(y_i)\log_2P(y_i)$\\
Cond. Entropy: $\displaystyle H(Y\mid X)=\sum_x P(X=x)H(Y\mid X=x)$\\
Information Gain:\;\;$IG(Y;X)=H(Y)-H(Y\mid X)$

DTL grows tree top-down, greedily choosing feature $X$ with highest IG, and recurses. At leaves:
\begin{enumerate}[\roman*.]
  \item If no more data, return default 
  \item If data has same classification, return classification 
  \item If no more features, return majority class
\end{enumerate}

Pruning reduces overfitting, giving simpler hypothesis by limiting representational capacity, removing outliers:
\begin{enumerate}[\roman*.]
  \item Max-depth: limit max depth of decision tree.
  \item Min-sample leaves: set min samples for leaf nodes. 
\end{enumerate}
\vspace{-1em}
\colbreak
\section{Supervised Learning}
Supervised Learning learns from labelled data to learn a mapping from inputs to outputs.

\subsection{Linear Regression}
Linear regression creates a linear model to predict $y \in \RR$:
\begin{align*}
  h_w(x)= w^Tx = w_0 + w_1x_1 + \cdots + w_dx_d
\end{align*}
where $w_0, \cdots, w_d$ are weights, using MSE loss function:
\begin{align*}
  J_{MSE}(w) = \frac{1}{2n}\sum^n_{i=1}(h_w(x^{(i)}) - y^{(i)})^2
\end{align*}
Learning uses $\frac{\partial J(w)}{\partial w_j} = \frac{1}{n}\sum^n_{i=1}(h_w(x^{(i)}) - y^{(i)})x_j^{(i)}$ via:
\begin{enumerate}[\roman*.]
  \item Normal Equation: $w = (X^{\top}X)^{-1}X^{\top}Y$
    \begin{itemize}[leftmargin=*]\vspace{3pt}
      \item Closed form, assumes invertible, $O(d^3)$ inversion
    \end{itemize}
  \item Gradient Descent: $w_j \leftarrow w_j - \gamma \frac{\partial J(w)}{\partial w_j}$ repeated
    \begin{itemize}[leftmargin=*]\vspace{3pt}
      \item Converge on global min., for appropriate $\gamma > 0$ 
      \item If features linearly indep., global min is unique
      \item May need feature scaling/ different $\gamma_j$ for weights
      \item Stochastic/ Mini-batch faster by using less data
    \end{itemize}
\end{enumerate}

\subsection{Logistic Regression}
Logistic regression creates a model to classify $y \in \{0,1\}$:
\begin{align*}
  h_w(x) = \sigma(w^Tx),\quad \sigma(z) = \frac{1}{1+e^{-z}}
\end{align*}
where $w$ is the weight vector and $\sigma(z)$ is sigmoid function with $\sigma'(z) = \sigma(z)(1-\sigma(z))$, using BCE loss function:
\begin{gather*}
  J_{BCE} = \frac{1}{n}\sum^n_{i=1}BCE(y^{(i)}, h_w(x^{(i)}))\\
  BCE(y, p) = -y\log(p) - (1-y)\log(1-p)
\end{gather*}
Model $h_w(x)$ can be interpreted as $P(y=1\mid x)$, where we classify $y=1$ if $h_w(x) \geq \tau$, decision boundary $h_w(x) = \tau$

Learning only via Gradient Descent with same properties.

Feature Transformation $x \in \RR^d \rightarrow \phi(x) \in \RR^M$ allows linear and logistic regression on non-linear relationships.

\colbreak
\subsubsection{Regularisation}
Regularisation augments constraints to prevent overfitting:
\begin{align*}
  J_{reg}(w) = J(w) + \lambda R(w),\quad\text{where }\lambda> 0
\end{align*}
\begin{enumerate}[\roman*.]
  \item Generally, $L_p$-norm, $R_{L_p} = \|w\|_p = (\sum^d_{j=0}|w_j|^p)^\frac{1}{p}$
  \item $L_1$-norm, $R_{L_1} = \|w\|_1 = \sum^d_{j=0}|w_j|$
    \begin{itemize}[leftmargin=*]\vspace{2pt}
      \item $\frac{\partial J^{MSE}_{L_1}(w)}{\partial w_j} = \frac{1}{n}\sum^n_{i=1}(w^\top x^{(i)} - y^{(i)})x_j^{(i)} + \lambda \frac{\partial |w_j|}{\partial w_j}$
      \item Theorem: features with non-zero weight will have constant penalty of $\lambda$  
      \item Theorem: $L_1$ induces sparsity, selecting features with $J_{MSE}(w) = \lambda$
    \end{itemize}
  \item $L_2$-norm, $R_{L_2} = \|w\|_2 = \sqrt{\sum^d_{j=0}w_j^2}$ or $\frac{1}{2}\sum^d_{j=0}w_j^2$
    \begin{itemize}[leftmargin=*]\vspace{2pt}
      \item $\frac{\partial J^{MSE}_{L_2}(w)}{\partial w_j} = \frac{1}{n}\sum^n_{i=1}(w^\top x^{(i)} - y^{(i)})x_j^{(i)} + \lambda w_j$
      \item Theorem: $X^\top X + \lambda I$ is invertible with unique, closed-form $w = (X^\top X + \lambda I)^{-1}X^\top Y$
      \item Theorem: $L_2$ induces weights shrinkage
    \end{itemize}
  \item $L_\infty$-norm, $R_{L_\infty} = \|w\|_\infty = \max_j|w_j|$ 
\end{enumerate}

\subsubsection{Dual Formulation and Kernel Method}
Dual formulation re-parameterises $w$ using training data. Representer Theorem for optimal $w$ of $J_{L_2}^{MSE}(w)$:
\begin{gather*}
  w = \sum^n_{j=1}\alpha_jx^{(j)},\quad h_\alpha(x) = \sum^n_{j=1}\alpha_jx^{(j)\top} x = \sum^n_{j=1}\alpha_j \langle x^{(j)}, x\rangle\\
  J^{MSE}_{L_2}(\alpha) = \frac{1}{2}\sum^n_{i=1}(\sum^n_{j=1} \alpha_j x^{(j)\top} x^{(i)} - y^{(i)})^2 + \lambda\sum^n_{i=0}(\sum^n_{j=1}\alpha_jx_i^{(j)})^2
\end{gather*}

Kernel function $k_\phi(u, v) = \langle \phi(u), \phi(v)\rangle$ exists for any feature mapping $\phi$, giving cheaper computation:
\begin{align*}
  h_\alpha^\phi(x) = \sum^n_{j=1}\alpha_j k_\phi(x^{j}, x)
\end{align*}
Common kernels:
\begin{enumerate}[\roman*.]
  \item Polynomial deg. $k$: $k_{pk}(u, v) = (u^\top v)^k$
  \item Gaussian var. $s^2$: $k_{RBF}(u,v) = \operatorname{exp}(-\frac{\|u-v\|^2}{2s^2})$
\end{enumerate}

\colbreak
\subsection{Support Vector Machines (SVMs)}
SVMs are models to classify $y \in \{-1, 1\}$:
\begin{align*}
  h_w(x) &= \operatorname{sign}(w^T x),\quad
  h_\alpha(x) &= \operatorname{sign}(\sum_{x^{(i)}\in S} \alpha_ik(x^{(i)}, x))
\end{align*}\vspace{-1em}\\
where $w$ is normal vector of the zero-offset hyperplane decision boundary $w^\top x_{\in S} = 0$, maximising objective:\vspace{-2pt}
\begin{gather*}
  \max_w \frac{1}{\|w\|}=\min_w \frac{\|w\|^2}{2},\quad y^{(i)}(w^{\top}x^{(i)}) \geq 1\\
  \max_\alpha \sum^n_{i=1}\alpha_i - \frac{1}{2}\sum^n_{i=j=1} \alpha_i\alpha_j y^{(i)}y^{(j)}k_{\phi}(x^{(i)}, x^{(j)}),\sum^n_{i=1}\alpha_iy^{(i)}=0
\end{gather*}\vspace{-1em}
\begin{enumerate}[\roman*.]
  \item Distance to point $x$: $\frac{|w^{\top}x|}{\|w\|}$, Margin: $\frac{1}{\|w\|}$
  \item Support vectors $S$ lie exactly on margin boundary
    \begin{itemize}[leftmargin=*]\vspace{2pt}
      \item $\alpha_i > 0 \iff x^{(i)} \in S$, $\alpha_i = 0 \iff x^{(i)} \not\in S$
      \item Theorem: for linearly separable data of $r \leq d$ effective dimension, $|S| \geq r+1$\\(i.e. support vectors much less than data points)
    \end{itemize}
\end{enumerate}

\subsubsection{Applications}
Multi-class Classification ($y \in \{1, \dots, K\}$):
\begin{enumerate}[\roman*.]
  \item One-One: $\binom K2$ classifiers vote between $I, J$ 
  \item One-Rest: $K$ classifiers choose between $I, I^c$
\end{enumerate}

Multi-label Classification ($y \in \{0, 1\}^K$):
\begin{enumerate}[\roman*.]
  \item Binary Relevance: indep. binary classifiers
    \begin{itemize}[leftmargin=*]\vspace{2pt}
      \item ignores label correlations and constraints
    \end{itemize}
  \item Classifier Chains: chain pred. of binary classifiers
    \begin{itemize}[leftmargin=*]\vspace{2pt}
      \item captures dependencies; order-sensitive
    \end{itemize}
  \item Label Powerset
    \begin{itemize}[leftmargin=*]\vspace{2pt}
      \item preserve correlations; classes explode; data-sparse
    \end{itemize}
\end{enumerate}

Generalizability:
\begin{enumerate}[\roman*.]
  \item Dataset: relevance, noise, balance (classification)
  \item Model Complexity: 
    \begin{itemize}[leftmargin=*]\vspace{2pt}
      \item Low: underfits for complex data, high bias, low variance on retrains on different data 
      \item High: overfits for scarce data, low bias with enough data, high variance on retrains on different data 
    \end{itemize}
\end{enumerate}


\section{Unsupervised Learning}
Unsupervised Learning finds patterns in unlabelled data.
\subsection{K-Means Clustering}
K-means Clustering groups $n$ data points into $K$ groups with centroids $\mu_1, \dots, \mu_K$, where centroid $\mu_j$ of $n_j$ points is the average of points $x^{(i)}$ in the cluster:
\begin{align*}
  \mu_j = \frac{1}{n_j}\sum^{n_j}_{i=1}x^{(i)}
\end{align*}
with evaluation function:
\begin{align*}
  J(c^{(1)},\dots,c^{(n)}, \mu_1, \dots, \mu_K) = \frac{1}{n}\sum^n_{i=1}\|x^{(i)}-\mu_{c^{i}}\|^2
\end{align*}
\begin{enumerate}[\arabic*.]
  \item Randomly initialize $K$ centroids $\mu_1, \dots, \mu_K$
  \item Repeat until convergence:
    \begin{enumerate}[2.\arabic*.]\vspace{2pt}
      \item For $i=1,\dots, n$: $c^{(i)} \leftarrow$ $k$ of closest $\mu_k$
      \item For $k=1,\dots,K$: $\mu_k \leftarrow$ centroid of data points $x^{(i)}$ assigned to cluster $k$
    \end{enumerate}
\end{enumerate}\vspace{-2pt}
\begin{enumerate}[\roman*.]
  \item Theorem: each iteration never increases distortion; need deterministic tie-break for conv. to local opt.
  \item Choose $K$ by business need or using elbow method (stop after decrease in $J$ with $K$ slows suddenly)
  \item Can be used for classification.
\end{enumerate}
\subsection{Principal Component Analysis (PCA)}
PCA preserves variance while reducing features to $r < d$:
\begin{enumerate}[\roman*.]
  \item SVD of $d \times n$ $X^\top$: $X^\top = U\Sigma V^\top$
    \begin{itemize}[leftmargin=*]\vspace{2pt}
      \item $U$ is $d\times d$ orthonormal columns and rows
      \item $\Sigma$ is $d\times n$ diagonal with ordered $\sigma_j \geq 0$
      \item $V$ is $n\times n$ orthonormal columns and rows
    \end{itemize}
  \item Truncating to $\sigma_r$ yields $d \times r$ $\tilde{U}$, $r \times n$ $\tilde{\Sigma}$
  \item Compression of $X^\top$: $r\times n$ $Z = \tilde{U}^\top X^\top$
  \item Reconstruction: $\hat{X}^\top \approx \tilde{U}\tilde{U}^\top X^\top$ since $\tilde{U}\tilde{U}^T \approx I$ with approximation error dependent on $r$
  \item Theorem: in mean-centered $\hat{X}^\top$, $\frac{\sigma_j^2}{n-1}$ are variance of data in the basis of $u^{(j)}$
  \item Var: $\frac{\sum^r_{i=1}\sigma_i^2}{\sum^n_{i=1}\sigma^2_i} \geq 0.99 \iff \frac{\sum^n_{i=1}\|\hat{x}^{(i)} - \tilde{x}^{(i)}\|^2}{\sigma^n_{i=1}\|\hat{x}^{(i)}\|^2} \leq 0.01$
\end{enumerate}
\vspace{-1em}
\colbreak
\section{Neural Networks}
Neural networks are layered models of neurons that learn feature transformations for prediction tasks.

Neuron computes activation function on a weighted sum:
\begin{align*}
  \hat{y} = g(z),\quad z = \sum^d_{j=0}w_jx_j = w^\top x
\end{align*}\vspace{-2em}
\begin{enumerate}[\roman*.]
  \item Common activation functions:
    \begin{itemize}[leftmargin=*]\vspace{2pt}
      \item Identity: $g(z) = z$
      \item Sigmoid: $g(z) = \sigma(z) = \frac{1}{1 + e^{-z}}$
      \item Tanh: $g(z) = \operatorname{tanh}(z) = 2\sigma(2x) - 1 = \frac{e^z-e^{-z}}{e^z + e^{-z}}$
      \item Relu: $g(z) = max(0, z)$
      \item Leaky Relu: $g(z) = max(az, z)$
      \item Softmax: $g(z_i) = \frac{e^{z_i}}{\sum^K_{j=1}e^{z_j}} \in [0, 1]$
    \end{itemize}
  \item Output Layer: chosen to match task target
  \item Hidden Layer: any layer between input and output
\end{enumerate}

Forward Propagation is the process of input data passing through neural network to output layer:
\begin{enumerate}[\roman*.]
  \item Weights of layer $i$, $W^{[i]}$, where $W^{[i]}_{jk}$ are weights from previous neuron $j$ to layer neuron $k$
  \item Activation function of layer $i$, $g^{[i]}$ 
  \item Output of layer $i$, $\hat{y}^{[i]} = g^{[i]}(W^{[i]\top} y^{[i-1]})$
\end{enumerate}

Backpropagation is gradient descent in the network:
\begin{enumerate}[\roman*.]
  \item Compute loss $L(\hat{y}, y)$ at output layer
  \item Apply chain rule layer-by-layer backwards for $\frac{\partial L}{\partial w_j} = \frac{dL}{d\hat{y}} \frac{d\hat{y}}{dz} \frac{\partial z}{\partial w_j}$ and $\frac{dL}{dx} = \frac{\partial L}{\partial z_1}\frac{\partial z_1}{dx} + \frac{\partial L}{\partial z_2}\frac{dz_2}{dx}$
  \item Algorithm: Compute $\frac{dL}{d\hat{y}}$, convert $g(z)$ to $g'(z)$, reverse graph and compute gradients 
\end{enumerate}

Tasks:
\begin{enumerate}[\roman*.]
  \item Binary Classification: 1 sigmoid output neuron
  \item Multi-class Classification: $K$ softmax output neurons for $\hat{y} = [\hat{y}_1, \dots, \hat{y}_K]^\top$ as one-hot vector
  \item Single Regression: 1 any output neuron
  \item Multi Regression: $K$ any output neurons for $\hat{y} = [\hat{y}_1, \dots, \hat{y}_K]^\top$ 
\end{enumerate}
\vspace{-1em}
\colbreak
\subsection{Convolutional Neural Networks (CNNs)}
CNNs apply shared filters over local regions of inputs:
\begin{enumerate}[\roman*.]
  \item Convolution Layer: slide kernel over input and compute weighted sums to produce feature maps
    \begin{itemize}[leftmargin=*]\vspace{2pt}
      \item Kernel/Filter: matrix of weights
      \item Stride: step size 
      \item Padding: add $k$ padding around border of input
      \item Output: $N' = \lfloor \frac{N -K + 2P}{S}\rfloor + 1$
    \end{itemize}
  \item Pooling Layer: downsample feature maps to reduce resolution and parameters
    \begin{itemize}[leftmargin=*]\vspace{2pt}
      \item Max-Pool: keep max value in window 
      \item Average-Pool: compute average value in window
    \end{itemize}
  \item Architecture:
    \begin{itemize}[leftmargin=*]\vspace{2pt}
      \item Feature Transformation: stack of Conv and Pools 
      \item Prediction: fully-connected output layer
    \end{itemize}
\end{enumerate}
usually using CE loss between predicted $\hat{y}$ and actual $y$:
\begin{gather*}
  J_{CE} = \frac{1}{n}\sum^n_{i=1}CE(y^{(i)}, \hat{y}^{(i)})\\
  CE(y^{(i)}, \hat{y}^{(i)}) = -\sum^K_{j=1}y_j^{(i)}\log(\hat{y}_j^{(i)})
\end{gather*}\vspace{-1em}
\subsection{Recurrent Neural Networks (RNNs)}
RNNs model sequences using hidden state, where for layer $j$ with hidden state $h^{[t]}$ at time $t$:
\begin{align*}
  h^{[t]} = \hat{y}^{[j]} = g^{[j]}(W^{[ij]\top} x^{[t]} + W^{[hj]\top} h^{[t-1]})
\end{align*}
Tasks:
\begin{enumerate}[\roman*.]
  \item Many-Many (e.g. sequence tagging, translation):\\1 input and 1 output neuron for $t \leq T_x = T_y$ 
  \item Many-Many (e.g. QnA):\\1 input for $t \leq T_x$, "BEGIN" at $t = T_x + 1$, and 1 output neuron for $T_x < t \leq T_y$
  \item Many-One (e.g. sentiment analysis):\\1 input for $t \leq T_x$, 1 output at final $t = T_y = T_x$
  \item One-Many (e.g. image captioning):\\1 input at start $t = T_x = 1$, 1 output for $t \leq T_y$
\end{enumerate}
\colbreak
\subsection{Attention Neural Networks (ANNs)}
Attention lets a model selectively focus on relevant parts of a sequence using vectors derived from $d\times T$ input $X$:
\begin{enumerate}[\roman*.]
  \item Query, $d_q\times T$ $Q$: represent current element query 
  \item Key, $d_k\times T$ $K$: contain keys to search information 
  \item Value, $d_v \times T$ $V$: actual element information
\end{enumerate}

Attention Layers:
\begin{enumerate}[\roman*.]
  \item Self-Attention: $Q, K, V$ from same sequence 
    \begin{itemize}[leftmargin=*]\vspace{2pt}
      \item $q^{[t]} = W^qx^{[t]} \iff Q = W^qX$,\hfill$W^q$ is $d_q \times d$
      \item $k^{[t]} = W^kx^{[t]} \iff K = W^kX$,\hfill$W^k$ is $d_k \times d$
      \item $v^{[t]} = W_vx^{[t]} \iff V = W^vX$,\hfill$W^v$ is $d_v \times d$
      \item $\displaystyle\alpha_{it} = \frac{(k^{[i]})^\top q^{[t]}}{\sqrt{d_k}} \iff A = \frac{K^\top Q}{\sqrt{d_k}}$,\hfill$A$ is $T\times T$
      \item $\displaystyle\alpha'_{it} = \operatorname{softmax}(a_{it}) \iff A' = \operatorname{softmax(A)}$ col-w
      \item $\displaystyle h^{[t]} = \sum^T_{i=1}a'_{it}v^{[i]}\iff H = VA'$,\hfill$H$ is $d_v\times T$
    \end{itemize}
  \item Masked Self-Attention: restrict to only $\leq t$
    \begin{itemize}[leftmargin=*]\vspace{2pt}
      \item $\displaystyle a_{it} = \begin{cases}a_{it}, &\text{if } i \leq t\\ -\infty, &\text{if } i > t\end{cases}$
    \end{itemize}
  \item Cross-Attention: $Q$ from decoder sequence, $K, V$ from encoder sequence
    \begin{itemize}[leftmargin=*]\vspace{2pt}
      \item $k^{[t]} = W^kz^{[t]}$ i.e. $K = W^kZ$
      \item $v^{[t]} = W_vz^{[t]}$ i.e. $V = W^vZ$
    \end{itemize}
\end{enumerate}

Tasks:
\begin{enumerate}[\roman*.]
  \item Many-Many: self-attention layer with BCE loss
    \begin{itemize}[leftmargin=*]\vspace{2pt}
      \item Positional Encoding (distinct, consistent with $T$, bounded): $x'^{[t]} = x^{[t]} + PE^{[t]}$ where for pos. $k$: $PE^{[t]}_k = \begin{cases}
    \sin(t/C^{\frac{k}{d}}), &\text{if }k\%2==0\\
    \cos(t/C^{\frac{k-1}{d}}), &\text{if }k\%2==1
    \end{cases}$
    \end{itemize}
  \item Many-Many (e.g. QnA): self-attention for question, masked self-attention and cross-attention for answer 
  \item Many-One (e.g. sentiment): self-attention layer with additional $"CLS"$ input for $h^{[CLS]}$ to capture summary information of input sequence
  \item One-Many (e.g. image caption): masked self-attention layer with additional "BEGIN" input
\end{enumerate}
\vspace{-1em}
\colbreak
Teacher Forcing is used in training for sequence prediction:
\begin{enumerate}[\roman*.]
  \item Input for next step is true target, not models own prediction from current step
\end{enumerate}\vspace{-1pt}
\begin{enumerate}[$+$]
  \item All output can be generated simultaneously, no need to wait for previous word
  \item In early training, predictions are nearly random, which may make the model lost 
\end{enumerate}
\begin{enumerate}[$-$]\vspace{-1pt}
  \item During testing, teacher forcing is not available, leading to train-test mismatch
  \item Mitigation: sometimes feed true word, sometimes prediction, gradually reducing teacher forcing
\end{enumerate}

\subsubsection{Transformers}
Transformers are deep attention neural networks:
\begin{enumerate}[\roman*.]
  \item Stack of Encoder Blocks; self-attention $\rightarrow$ feed-forward
  \item Stack of Decoder Blocks; masked self-attention $\rightarrow$ cross-attention $\rightarrow$ feed-forward
\end{enumerate}

\subsection{Unsupervised Learning with Neural Networks}
Autoencoders use neural networks for dimensionality reduction and representation learning:
\begin{enumerate}[\roman*.]
  \item Encoder: converts high-dimensional $d$ input to $k < d$ dimensional compressed data.
  \item Decoder: reconstructs original data from compressed data.
  \item Reconstruction Loss: measures difference between reconstructed output and original input.
  \item After Training: discard decoder, use encoder to generate compressed representation for each input.
\end{enumerate}

Model Pre-training initialises feature transformation weights based on tasks which are solvable with unlabelled data, challenging enough to force understanding of task, and yield useful learned understanding:
\begin{enumerate}[\roman*.]
  \item Image Rotation Prediction
  \item Contrastive Learning (positive, negative pairs) 
  \item Image Inpainting
  \item Next-word Prediction
\end{enumerate}
\vspace{-1em}
\colbreak
\subsection{Problems with Neural Networks}
Common problems when training deep neural networks:
\begin{enumerate}[\roman*.]
  \item Overfitting:
    \begin{itemize}[leftmargin=*]\vspace{2pt}
      \item Model fits training data too well and performs poorly on unseen data.
      \item Early Stopping: stop training when performance on validation set begins to worsen.
      \item Dropout: during training, randomly set some neurons’ output to $0$; prevents overfitting by making the network less reliant on specific neurons.
    \end{itemize}
  \item Vanishing / Exploding Gradient:
    \begin{itemize}[leftmargin=*]\vspace{2pt}
      \item Vanishing gradient: gradients become very small, causing very slow learning in deep networks.
      \item Exploding gradient: gradients become very large, causing unstable updates.
      \item Gradient Clipping: clip gradients within range $[-\text{clip\_value}, \text{clip\_value}]$ to control exploding gradients.
    \end{itemize}
\end{enumerate}

\subsubsection{Perceptrons}
Perceptron is a single neuron for binary classification:
\begin{align*}
  \hat{y} = \operatorname{sign}(w^\top x + b)
\end{align*}
\begin{enumerate}[\roman*.]
  \item Represents a linear decision boundary (hyperplane)
  \item Perceptron update rule for misclassified $i$: $w' \leftarrow w + \lambda (y^{(i)} - \hat{y}^{(i)})x^{(i)}$
  \item Multi-Layer Perceptron (MLP): stack of fully-connected layers with non-linear activations.
\end{enumerate}

\subsubsection{Receptive Fields in CNNs}
Receptive field of a neuron is the region of input that can affect its activation.

Formualae for receptive field $r_i$ for a neuron in layer $i$:
\begin{align*}
  r_i &= r_{i-1} + (K_i - 1) \times j_{i-1}\\
  j_i &= j_{i-1} \times S_i
\end{align*}
where $K_i$ is kernel size at $i$, $S_i$ is stride at $i$, $r_0 = j_0 = 1$
\colbreak
\incimg[0.7]{alphabeta}

a-B pruning Steps:
\begin{enumerate}[\roman*.]
  \item Start at root $\alpha = -\infty, \beta = \infty$
  \item When going down, pass down the values of $\alpha, \beta$
  \item When going up, pass up the value of $\alpha$ if MAX, or $\beta$ if MIN
  \item When taking a value up, store $ \alpha = \max(\alpha, value)$ if MAX, or $\beta = \min(\beta, value)$ if MIN
  \item At any point, if there is $\alpha \geq \beta$ prune all other children
\end{enumerate}
\end{multicols*}
\incimg[0.9]{search}
%%%%%%%%%%%%%%%%%%%%%%%%%%%%%%%%%%%%%%%%%%%%%%%%%%%%%%
%                       End                          %
%%%%%%%%%%%%%%%%%%%%%%%%%%%%%%%%%%%%%%%%%%%%%%%%%%%%%%
\end{document}

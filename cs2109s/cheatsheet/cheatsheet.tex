\documentclass[12pt, a4paper]{article}

\usepackage[utf8]{inputenc}
\usepackage[mathscr]{euscript}
\let\euscr\mathscr \let\mathscr\relax
\usepackage[scr]{rsfso}
\usepackage{amssymb,amsmath,amsthm,amsfonts}
\usepackage[shortlabels]{enumitem}
\usepackage{multicol,multirow}
\usepackage{lipsum}
\usepackage{balance}
\usepackage{calc}
\usepackage[colorlinks=true,citecolor=blue,linkcolor=blue]{hyperref}
\usepackage{import}
\usepackage{xifthen}
\usepackage{pdfpages}
\usepackage{transparent}
\usepackage{tabularx}

\newcommand{\incfig}[2][1.0]{
    \def\svgwidth{#1\columnwidth}
    \import{./figures/}{#2.pdf_tex}
}
\newcommand{\incimg}[2][1.0]{
  \includegraphics[width=#1\columnwidth]{./figures/#2}
}


\usepackage{ifthen}
\usepackage[landscape]{geometry}
\usepackage[shortlabels]{enumitem}

\ifthenelse{\lengthtest { \paperwidth = 11in}}
    { \geometry{top=.5in,left=.5in,right=.5in,bottom=.5in} }
	{\ifthenelse{ \lengthtest{ \paperwidth = 297mm}}
		{\geometry{top=1cm,left=1cm,right=1cm,bottom=1cm} }
		{\geometry{top=1cm,left=1cm,right=1cm,bottom=1cm} }
	}

\pagestyle{empty}
\makeatletter
\renewcommand\thesection{\arabic{section}.}
\renewcommand{\section}{\@startsection{section}{1}{0mm}%
                                {-1ex plus -.5ex minus -.2ex}%
                                {0.05ex}%x
                                {\normalfont\normalsize\bfseries}}
\renewcommand{\subsection}{\@startsection{subsection}{2}{0mm}%
                                {-1ex plus -.5ex minus -.2ex}%
                                {0.05ex}%
                                {\normalfont\small\bfseries}}
\renewcommand{\subsubsection}{\@startsection{subsubsection}{3}{0mm}%
                                {-1ex plus -.5ex minus -.2ex}%
                                {0.05ex}%
                                {\normalfont\footnotesize\bfseries}}
\newcommand{\colbreak}{\vfill\null\columnbreak}
\makeatother
\setcounter{secnumdepth}{1}
\setlength{\parindent}{0pt}
\setlength{\parskip}{0.7em}

\setlist[itemize]{itemsep=0.6ex, topsep=-2pt, partopsep=0pt, parsep=0pt}
\setlist[enumerate]{itemsep=0.6ex, topsep=-2pt, partopsep=0pt, parsep=0pt}

\input{letterfonts}

\newcommand{\mytitle}{CS2109S Intro. to AI \& Machine Learning}
\newcommand{\myauthor}{github/omgeta}
\newcommand{\mydate}{AY 25/26 Sem 1}

\begin{document}
\raggedright
\footnotesize
\begin{multicols*}{3}
\setlength{\premulticols}{1pt}
\setlength{\postmulticols}{1pt}
\setlength{\multicolsep}{1pt}
\setlength{\columnsep}{2pt}

{\normalsize{\textbf{\mytitle}}}\\
{\footnotesize{\mydate\hspace{2pt}\textemdash\hspace{2pt}\myauthor}}
%%%%%%%%%%%%%%%%%%%%%%%%%%%%%%%%%%%%%%%%%%%%%%%%%%%%%%
%                      Begin                         %
%%%%%%%%%%%%%%%%%%%%%%%%%%%%%%%%%%%%%%%%%%%%%%%%%%%%%%
\vspace{-0.5em}\section{Intelligent Agents}
Agents receive precepts from the environment through sensors and perform actions through actuators.

\subsection{PEAS Framework}
PEAS defines problems in terms of Performance Measure, Environment, Actuators and Sensors:
\begin{enumerate}[\roman*.]
  \item Rational Agent: chooses actions that maximise performance measure
\end{enumerate}
{\centering\incimg[0.55]{structure}\par}

Properties of Task Environment:
\begin{enumerate}[\roman*.]
  \item Fully Observable (vs. Partially Observable):\\sensors give access to the complete state of the environment at each point in time. 
  \item Single-agent (vs. Multi-agent):\\agent operates by itself in an environment.
  \item Deterministic (vs. Stochastic):\\next state of the environment is completely determined by current state and action by agent.
    \begin{itemize}[leftmargin=*]\vspace{3pt}
      \item Strategic: if environment is dependent on actions of other unpredictable (not "dumb") agents 
    \end{itemize}
  \item Episodic (vs. Sequential):\\ agent's experience is divided into atomic "episodes", and choice of action in each episode depends only on the episode itself.
  \item Static (vs. dynamic):\\ environment is unchanged while agent deliberates.
    \begin{itemize}[leftmargin=*]\vspace{3pt}
      \item Semi-dynamic: if environment does not change with time, but agent's performance score does.
    \end{itemize}
  \item Discrete (vs. Continuous):\\a limited number of distinct percepts and actions.
\end{enumerate}\vspace{-1em}
\colbreak

\subsubsection{Agent Structures}
Agents are completely specified by the agent function:
\begin{enumerate}[\roman*.]
  \item Agent Function: $f: \mcP \rightarrow \mcA$ maps from precept histories $\mcP$ to actions $\mcA$
  \item Agent Program: implements the agent function
\end{enumerate}

Common Agent Structures:
\begin{enumerate}[\roman*.]
  \item Simple Reflex:\\ chooses action only based on current percept, ignoring precept history (follow if-then rules).
  \item Goal-based:\\select actions to achieve a given tracked goal.
  \item Utility-based:\\selects actions to maximise a utility function which assigns a score to any precept sequence.\\If the utility function aligns with performance measure, the agent is rational.
  \item Learning:\\improves performance over time with experience.
    \begin{itemize}[leftmargin=*]\vspace{3pt}
      \item Performance Element: selects actions.
      \item Learning Element: updates knowledge from feedback. 
      \item Critic: provides feedback on performance relative to a fixed performance standard.
      \item Problem Generator: suggest exploratory actions
    \end{itemize}
    %\vspace{1em}{\centering\incimg[0.7]{learningagent}\par}
\end{enumerate}

Exploration-Exploitation Dilemma:
\begin{enumerate}[\roman*.]
  \item Explore: learn more about the world.
  \item Exploit: maximize gain from current knowledge.
\end{enumerate}

\section{Systematic Search}

Systematic search problems are fully-observable, deterministic, static, discrete and defined by:
\begin{enumerate}[\roman*.]
  \item States: representation of problem state 
  \item Initial State 
  \item Goal State/Test 
  \item Actions: possible operations on a state
  \item Transition Model: function of action on a state
  \item Action Cost Function
\end{enumerate}
\vspace{-1em}
\colbreak
\subsection{Uninformed Search}
Uninformed search explores a problem space without domain-specific knowledge or heuristics to guide searching.
\begin{enumerate}[\roman*.]
  \item Branching factor $b$: max successors of any node 
  \item Optimal solution depth $d$, Maximum depth $m$ 
\end{enumerate}

Breadth-First Search (BFS) expands nodes level-by-level using a queue:
\begin{enumerate}[\roman*.]
  \item Time: $O(b^{d+1})$ 
  \item Space: $O(b^d)$
  \item Complete: Yes (if $b$ is finite)
  \item Optimal: Yes (if all same step cost)
\end{enumerate}

Uniform-Cost Search (UCS) expands least-cost node using a priority queue, for optimal solution cost $C^*$, levels $C^* /\epsilon$:
\begin{enumerate}[\roman*.]
  \item Time: $O(b^{C^*/ \epsilon})$ 
  \item Space: $O(b^{C^* /\epsilon})$
  \item Complete: Yes (if step costs $> 0$ $\land$ finite total cost)
  \item Optimal: Yes (if step costs $> 0$)
\end{enumerate}

Depth-First Search (DFS) expands deepest unexpanded nodes using a stack:
\begin{enumerate}[\roman*.]
  \item Time: $O(b^{m})$ 
  \item Space: $O(bm)$
  \item Complete: No (if infinite depth $\lor$ cyclic)
  \item Optimal: No 
\end{enumerate}

Depth-Limited Search (DLS) limits search depth to $\ell$:
\begin{enumerate}[\roman*.]
  \item Time: $O(b^{\ell})$ 
  \item Space: $O(b\ell)$ (with DFS)
  \item Complete: No (if $\ell < d$)
  \item Optimal: No (with DFS)
\end{enumerate}

Iterative Deepening Search (IDS) runs DLS with increasing depth limit until solution found:
\begin{enumerate}[\roman*.]
  \item Time: $O(b^d)$ (with additional overhead)
  \item Space: $O(bd)$ (with DFS)
  \item Complete: Yes (if $b$ finite)
  \item Optimal: Yes (if all same step cost)
\end{enumerate}
\colbreak
\subsection{Informed Search}
Informed search uses heuristics to guide searching, where heuristic $h(n)$ estimates optimal cost from a state to goal:
\begin{enumerate}[\roman*.]
  \item All heuristics must be non-negative $\land$ $h(goal) = 0$
\end{enumerate}

Usefulness Properties:
\begin{enumerate}[\roman*.]
  \item Admissible: $\forall$  node $n$, $h(n) \leq h^*(n)$ where $h^*(n)$ is optimal path cost to reach goal state from $n$\\ ($h(n)$ never over-estimates true cost to goal)
    \begin{itemize}[leftmargin=*]\vspace{3pt}
      \item Theorem: if $h(n)$ is admissible, $A^*$ search (without visited memory) is optimal
    \end{itemize}
  \item Consistent: $\forall$ node $n$, $h(n) \leq c(n,a,n') + h(n')$ (triangle inequality)
    \begin{itemize}[leftmargin=*]\vspace{3pt}
      \item Theorem: if $h(n)$ is consistent, $A^*$ search\\(with visited memory) is optimal
      \item Theorem: if $h(n)$ is consistent, $h(n)$ is admissible
    \end{itemize}
  \item Dominance: $\forall$ nodes $n$, $h_1(n) \geq h_2(n)$ implies that $h_1$ dominates $h_2$ (more informed) 
    \begin{itemize}[leftmargin=*]\vspace{3pt}
      \item Theorem: if $h_1$ is admissible, $h_1$ is better for search, expanding no more nodes than $h_2$
    \end{itemize}
\end{enumerate}

Admissible heuristics can be found using optimal cost of a relaxed problem (fewer restrictions on actions):
\begin{enumerate}[Ex.]
  \item $h_{SLD}$ straight-line distance if agent can fly
  \item $h = 0$ if agent can teleport
\end{enumerate}

Best-First Search expands nodes with evaluation $f(n) = h(n)$ using a priority queue:
\begin{enumerate}[\roman*.]
  \item Time: $O(b^m)$ 
  \item Space: $O(b^m)$ 
  \item Complete: No (may get stuck in loops).
  \item Optimal: No (heuristic-only may mislead).
\end{enumerate}

A* expands nodes with evaluation $f(n) = g(n) + h(n)$, where $g(n)$ is step cost, using a priority queue:
\begin{enumerate}[\roman*.]
  \item Time: $O(b^d)$ (good heuristic can improve)
  \item Space: $O(b^d)$ (keeps all nodes in memory).
  \item Complete: Yes (if step costs $> 0$ $\land$ finite $b$).
  \item Optimal: Depends (if $h$ admissible in tree search,\\\hfill or consistent in graph search).
\end{enumerate}
\vspace{-1em}
\colbreak 
\section{Local Search}
Local search problems are defined by:
\begin{enumerate}[\roman*.]
  \item States: representation of candidate solution, may not map to actual problem state
  \item Initial State 
  \item Goal State/Test (optional) 
  \item Successor Function: generate neighbour states by applying modifications to current state
\end{enumerate}

Local search explores large, otherwise intractable problem spaces by considering only locally reachable states, guided by an evaluation function:
\begin{enumerate}[\roman*.]
  \item Evaluation Function: assesses quality of a state;\\we either minimize or maximise this function
\end{enumerate}

State Space Landscape:
\begin{enumerate}[\roman*.]
  \item Global Maximum: optimal solution 
  \item Local Maximum: local optimum solution 
  \item Shoulder: region with flat evaluation function, difficult for algorithm to move past
\end{enumerate}

Hill-Climbing searches for local optimum, generating successors from current state and picking the best using heuristic:
\begin{enumerate}[\roman*.]
  \item Any-time: More time gives better solutions
  \item Space: $O(b)$
  \item Complete: No 
  \item Optimal: Not guaranteed
  \item Variants:
    \begin{itemize}[leftmargin=*]\vspace{3pt}
      \item Simulated Annealing: allow some bad moves, gradually decreasing frequency 
      \item Beam Search: perform parallel $k$ hill-climbs 
      \item Genetic Algorithm: successor generated by combining 2 parent states 
      \item Random-Restart: escape local optimum by restarting search
    \end{itemize}
\end{enumerate}


\colbreak
\section{Adversarial Search}
Adversarial search problems are fully-observable, deterministic-strategic, static, discrete and defined by:
\begin{enumerate}[\roman*.]
  \item States: representation of candidate solution, may not map to actual problem state
  \item Initial State 
  \item Terminal States: where game ends (e.g. win)
  \item Actions: possible operations on a state
  \item Transition Model: function of action on a state
  \item Utility Function: output value of state from the perspective of our agent
\end{enumerate}

Minimax assumes both players play optimally, expanding game tree in DFS with alternating MAX and MIN levels:
\begin{enumerate}[\roman*.]
  \item Time: $O(b^m)$
  \item Space: $O(bm)$ 
  \item Complete: Yes (for finite games)
  \item Optimal: Yes (for both, if both play optimally)
    \begin{itemize}[leftmargin=*]\vspace{3pt}
      \item Theorem: if opponent plays sub-optimally, utility obtained by agent is never less than utility obtained against an optimal opponent
    \end{itemize}
\end{enumerate}

Alpha-Beta Pruning reduces nodes evaluated without altering minimax value and optimal move for root node:
\begin{enumerate}[\roman*.]
  \item Maintains $\alpha$ = best value , $\beta$ = worst value and pruning subtree if $\alpha \geq \beta$ 
  \item Time: $O(b^{m/2})$ (with perfect ordering)
  \item Space: $O(bm)$ (same as minimax)
  \item Complete \& Optimal: same as minimax
\end{enumerate}

Cutoff Strategy halts search in the middle and estimates value of mid-game states with an evaluation function:
\begin{enumerate}[\roman*.]
  \item Evaluation Function: if terminal, use utility function, else use a heuristic 
  \item Heuristic Function: estimates utility of a state
    \begin{itemize}[leftmargin=*]\vspace{3pt}
      \item A heuristic must be between max and min utility
      \item Admissibility and consistency are not needed
    \end{itemize}
  \item Theorem: returns a move that is optimal with respect to the evaluation function at the cutoff; may be suboptimal with respect to true minimax
\end{enumerate}
\vspace{-1em}
\colbreak
\section{Machine Learning}
Machine learning develops learning agents with data.
\begin{enumerate}[\roman*.]
  \item Data, $D = \{1 \leq i \leq n : (x^{(i)}, y^{(i)})\}$ of $n$ samples, for $x^{(i)} \in \RR^d$ feature vector, $y^{(i)}$ label of $i^{th}$ data point. 
  \item Model/Hypothesis, $h: X \rightarrow Y$ is a predictor of outputs which is learned by a learning algorithm.
  \item Performance Measure evaluates $h$ map $x^{(i)} \rightarrow y^{(i)}$.
    \begin{itemize}[leftmargin=*]\vspace{3pt}
      \item ${MAE}  = \frac{1}{n}\sum^n_{i=1} \left|\hat{y}^{(i)}-y^{(i)}\right|$ (outlier robust)
      \item ${Accuracy}  = \frac{1}{n}\sum^n_{i=1} \mathbbm{1}_{\hat{y}^{(i)}=y^{(i)}}$ (classification) 
    \end{itemize}
  \item Loss Function measures $\hat{y}^{(i)}$ error for optimization
    \begin{itemize}[leftmargin=*]\vspace{3pt}
      \item ${MSE}  = \frac{1}{n}\sum^n_{i=1} \left(\hat{y}^{(i)}-y^{(i)}\right)^2$ (outlier sensitive)
    \end{itemize}
\end{enumerate}

Binary Metrics:
\begin{enumerate}[\roman*.]
  \item Accuracy = $\frac{\text{TP + TN}}{\text{TP + FN + FP + TN}}$
  \item Precision, $P = \frac{\text{TP}}{\text{TP + FP}}$\hfill(if FP are costly)
  \item Recall, $R = \frac{\text{TP}}{\text{TP + FN}}$\hfill(if FN are costly)
  \item F1 Score, $F1 = \frac{2}{\frac{1}{P} + \frac{1}{R}}$\hfill(balance precision \& recall)
\end{enumerate}

\subsection{Decision Trees}
Decision trees split data using features to predict outputs.
\begin{enumerate}[\roman*.]
  \item Representational Completeness: any decision tree can fit any finite consistent labelled data exactly 
\end{enumerate}

Entropy: $\displaystyle H(Y)= -\sum_i P(y_i)\log_2P(y_i)$\\
Cond. Entropy: $\displaystyle H(Y\mid X)=\sum_x P(X=x)H(Y\mid X=x)$\\
Information Gain:\;\;$IG(Y;X)=H(Y)-H(Y\mid X)$

DTL grows tree top-down, greedily choosing feature $X$ with highest IG, and recurses. At leaves:
\begin{enumerate}[\roman*.]
  \item If no more data, return default 
  \item If data has same classification, return classification 
  \item If no more features, return majority class
\end{enumerate}

Pruning reduces overfitting, giving simpler hypothesis by limiting representational capacity, removing outliers:
\begin{enumerate}[\roman*.]
  \item Max-depth: limit max depth of decision tree.
  \item Min-sample leaves: set min samples for leaf nodes. 
\end{enumerate}
\vspace{-1em}
\colbreak
\section{Supervised Learning}
Supervised Learning learns from labelled data to learn a mapping from inputs to outputs.

\subsection{Linear Regression}
Linear regression creates a linear model to predict $y \in \RR$:
\begin{align*}
  h_w(x)= w^Tx = w_0 + w_1x_1 + \cdots + w_dx_d
\end{align*}
where $w_0, \cdots, w_d$ are weights, using MSE loss function:
\begin{align*}
  J_{MSE}(w) = \frac{1}{2n}\sum^n_{i=1}(h_w(x^{(i)}) - y^{(i)})^2
\end{align*}
Learning uses $\frac{\partial J(w)}{\partial w_j} = \frac{1}{n}\sum^n_{i=1}(h_w(x^{(i)}) - y^{(i)})x_j^{(i)}$ via:
\begin{enumerate}[\roman*.]
  \item Normal Equation: $w = (X^{\top}X)^{-1}X^{\top}Y$
    \begin{itemize}[leftmargin=*]\vspace{3pt}
      \item Closed form, assumes invertible, $O(d^3)$ inversion
    \end{itemize}
  \item Gradient Descent: $w_j \leftarrow w_j - \gamma \frac{\partial J(w)}{\partial w_j}$ repeated
    \begin{itemize}[leftmargin=*]\vspace{3pt}
      \item Converge on global min., for appropriate $\gamma > 0$ 
      \item If features linearly indep., global min is unique
      \item May need feature scaling/ different $\gamma_j$ for weights
      \item Stochastic/ Mini-batch faster by using less data
    \end{itemize}
\end{enumerate}

\subsection{Logistic Regression}
Logistic regression creates a model to classify $y \in \{0,1\}$:
\begin{align*}
  h_w(x) = \sigma(w^Tx),\quad \sigma(z) = \frac{1}{1+e^{-z}}
\end{align*}
where $w$ is the weight vector and $\sigma(z)$ is sigmoid function with $\sigma'(z) = \sigma(z)(1-\sigma(z))$, using BCE loss function:
\begin{gather*}
  J_{BCE} = \frac{1}{n}\sum^n_{i=1}BCE(y^{(i)}, h_w(x^{(i)}))\\
  BCE(y, p) = -y\log(p) - (1-y)\log(1-p)
\end{gather*}
Model $h_w(x)$ can be interpreted as $P(y=1\mid x)$, where we classify $y=1$ if $h_w(x) \geq \tau$, decision boundary $h_w(x) = \tau$

Learning only via Gradient Descent with same properties.

Feature Transformation $x \in \RR^d \rightarrow \phi(x) \in \RR^M$ allows linear and logistic regression on non-linear relationships.

\colbreak

\subsubsection{Multi-class Classification ($y\in \{1,\cdots,K\}$)}
One-vs-One classifies every pair of classes: 
\begin{enumerate}[\roman*.]
  \item $\binom K2$ classifiers choosing between class $I$ and $J$
  \item Each classifier votes for a class, most votes wins
\end{enumerate}

One-vs-Rest classifies per class vs. all others:
\begin{enumerate}[\roman*.]
  \item $K$ classifiers choosing between class $I$ and not-$I$
  \item Classifier with highest probability determins class
\end{enumerate}

\subsubsection{Multi-label Classification ($y\in \{0,1\}^K$)}

Binary Relevance uses independent label binary classifiers:
\begin{enumerate}[\roman*.]
  \item Ignores label correlations and constraints (e.g. mutual exclusive)
\end{enumerate}

Classifier Chains feeds predictions of previous binary labels classifiers as features for subsequent labels:
\begin{enumerate}[\roman*.]
  \item Captures label dependencies; sensitive to chain order
\end{enumerate}

Label Powerset treats each unique label-set as a single multi-class label: 
\begin{enumerate}[\roman*.]
  \item Preserves correlations; can explode in classes and be data-sparse
\end{enumerate}

\subsection{Generalization}
Generalization is the model's performance on unseen data.

Dataset Factors:
\begin{enumerate}[\roman*.]
  \item Relevance: features should be relevant to problem
  \item Noise: outliers can hinder learning, generalization 
  \item Balance (for classification): ensures all classes are fairly represented for generalization over classes
\end{enumerate}

Model Complexity (optimal complexity minimises error):
\begin{enumerate}[\roman*.]
  \item Low Complexity: underfits for complex data, high bias; low variance on retrains across different training sets 
  \item High Complexity: overfits for scarce data; low bias with enough data; high variance across retrains on different training sets 
\end{enumerate}
\vspace{-1em}
\colbreak
\subsection{Support Vector Machines}
\section{Neural Networks}
\section{Unsupervised Learning}
Unsupervised Learning learns from unlabelled data to find patterns or structure.
\subsection{K-Means Clustering}
\subsection{Hierarchical Clustering}
\subsection{Dimensionality Reduction}
\section{Ethics}
\colbreak
\incimg[0.7]{alphabeta}

a-B pruning Steps:
\begin{enumerate}[\roman*.]
  \item Start at root $\alpha = -\infty, \beta = \infty$
  \item When going down, pass down the values of $\alpha, \beta$
  \item When going up, pass up the value of $\alpha$ if MAX, or $\beta$ if MIN
  \item When taking a value up, store $ \alpha = \max(\alpha, value)$ if MAX, or $\beta = \min(\beta, value)$ if MIN
  \item At any point, if there is $\alpha \geq \beta$ prune all other children
\end{enumerate}

\incimg[0.9]{search}
%%%%%%%%%%%%%%%%%%%%%%%%%%%%%%%%%%%%%%%%%%%%%%%%%%%%%%
%                       End                          %
%%%%%%%%%%%%%%%%%%%%%%%%%%%%%%%%%%%%%%%%%%%%%%%%%%%%%%
\end{multicols*}
\end{document}

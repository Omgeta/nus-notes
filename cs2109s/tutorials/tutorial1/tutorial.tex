\documentclass[12pt, a4paper]{article}

\usepackage[a4paper, margin=1in]{geometry}

\usepackage[utf8]{inputenc}
\usepackage[mathscr]{euscript}
\let\euscr\mathscr \let\mathscr\relax
\usepackage[scr]{rsfso}
\usepackage{amssymb,amsmath,amsthm,amsfonts}
\usepackage[shortlabels]{enumitem}
\usepackage{multicol,multirow}
\usepackage{lipsum}
\usepackage{balance}
\usepackage{calc}
\usepackage[colorlinks=true,citecolor=blue,linkcolor=blue]{hyperref}
\usepackage{import}
\usepackage{xifthen}
\usepackage{pdfpages}
\usepackage{transparent}
\usepackage{tabularx}

\newcommand{\incfig}[2][1.0]{
    \def\svgwidth{#1\columnwidth}
    \import{./figures/}{#2.pdf_tex}
}
\newcommand{\incimg}[2][1.0]{
  \includegraphics[width=#1\columnwidth]{./figures/#2}
}


\input{letterfonts}

\newcommand{\mytitle}{CS2109S Tutorial 1}
\newcommand{\myauthor}{github/omgeta}
\newcommand{\mydate}{AY 25/26 Sem 1}

\begin{document}
\raggedright
\footnotesize
\begin{center}
{\normalsize{\textbf{\mytitle}}} \\
{\footnotesize{\mydate\hspace{2pt}\textemdash\hspace{2pt}\myauthor}}
\end{center}
\setlist{topsep=-1em, itemsep=-1em, parsep=2em}
%%%%%%%%%%%%%%%%%%%%%%%%%%%%%%%%%%%%%%%%%%%%%%%%%%%%%%
%                      Begin                         %
%%%%%%%%%%%%%%%%%%%%%%%%%%%%%%%%%%%%%%%%%%%%%%%%%%%%%%
\begin{enumerate}[\Alph*.]
  \item 
    \begin{enumerate}[\arabic*.]
      \item Performance Measure: Fitness improvement, user consistency in following programme\\
        Environment: Body health metrics (e.g. weight, fat\%), device, Internet\\
        Actuators: Notifications, suggestions, workout plans, voice guidance\\
        Sensors: Motion sensors, accelerometers, camera, heart rate monitor, user feedback
    \end{enumerate}

  \item 
    \begin{enumerate}[\arabic*.]
      \item Tuple of $n$ stacks, e.g. \lstinline|([3, 2, 1], [], [])|

      \item Each stack must have disks in decreasing order, where the largest is at the bottom and the smallest at the top. 

      \item Initial: \lstinline|([n, ..., 1], [], [])|\\
        Goal: \lstinline|([], [], [n, ..., 1])|

      \item Action: \lstinline|(x, y)| move top disk from non-empty stack $x$ to stack $y$ if ordering is not violated

      \item Transition: Pop top disk from $x$, and push onto $y$. 

      \item DLS; optimal depth $d= 2^n-1$ is known in advance, avoiding infinite loop issues of DFS while keeping polynomial memory $O(bd)$ and similar time complexity of $O(b^d)$ as BFS. 

      \item With $k$ pegs we choose a source and destination peg which are unique, leading to $b = \binom k2$ 
    \end{enumerate}

  \item 
    \begin{enumerate}[\arabic*.]
      \item DFS; minimises memory use and is not optimal, but doesn't matter for this question since we don't need the shortest path. IDS which is similar would have additional overhead.

      \item IDS; guarantees complete and optimal path but uses less (polynomial) memory than BFS (exponential memory) without visited memory. It's also more efficient with shallower solutions.

      \item BFS with depth limit of 50; complete and optimal, and terminates if no solution is found within 50 moves. IDS would have more overhead and the memory savings are negligible. DLS won't return shortest path.
    \end{enumerate}

  \item 
    \begin{enumerate}[\arabic*.]
      \item Yes; trivially $h_3(n) = h_1(n) + h_2(n) \geq h_2(n)$

      \item $h_{2}(n)$; $h_2(n)$ dominates $h_1(n)$ by virtue of Manhattan distance to further pellet always being $\geq$ to nearest pellet. $h_1(n), h_2(n)$ are also admissible as they never overestimate the true cost and satisfy the triangle inequality. $h_3(n)$ is inadmissible, e.g. when furthest pellet is 1 away from nearest, $h_3(n) = MD(near) + MD(far) \geq MD(near) + 1 = h^*(n)$

      \item Number of pellets left.
    \end{enumerate}
\end{enumerate}
%%%%%%%%%%%%%%%%%%%%%%%%%%%%%%%%%%%%%%%%%%%%%%%%%%%%%%
%                       End                          %
%%%%%%%%%%%%%%%%%%%%%%%%%%%%%%%%%%%%%%%%%%%%%%%%%%%%%%

\end{document}

\documentclass[12pt, a4paper]{article}

\usepackage[a4paper, margin=1in]{geometry}

\usepackage[utf8]{inputenc}
\usepackage[mathscr]{euscript}
\let\euscr\mathscr \let\mathscr\relax
\usepackage[scr]{rsfso}
\usepackage{amssymb,amsmath,amsthm,amsfonts}
\usepackage[shortlabels]{enumitem}
\usepackage{multicol,multirow}
\usepackage{lipsum}
\usepackage{balance}
\usepackage{calc}
\usepackage[colorlinks=true,citecolor=blue,linkcolor=blue]{hyperref}
\usepackage{import}
\usepackage{xifthen}
\usepackage{pdfpages}
\usepackage{transparent}
\usepackage{tabularx}

\newcommand{\incfig}[2][1.0]{
    \def\svgwidth{#1\columnwidth}
    \import{./figures/}{#2.pdf_tex}
}
\newcommand{\incimg}[2][1.0]{
  \includegraphics[width=#1\columnwidth]{./figures/#2}
}


\input{letterfonts}

\newcommand{\mytitle}{CS2109S Tutorial 3}
\newcommand{\myauthor}{github/omgeta}
\newcommand{\mydate}{AY 25/26 Sem 1}

\begin{document}
\raggedright
\footnotesize
\begin{center}
{\normalsize{\textbf{\mytitle}}} \\
{\footnotesize{\mydate\hspace{2pt}\textemdash\hspace{2pt}\myauthor}}
\end{center}
\setlist{topsep=-1em, itemsep=-1em, parsep=2em}
%%%%%%%%%%%%%%%%%%%%%%%%%%%%%%%%%%%%%%%%%%%%%%%%%%%%%%
%                      Begin                         %
%%%%%%%%%%%%%%%%%%%%%%%%%%%%%%%%%%%%%%%%%%%%%%%%%%%%%%
\begin{enumerate}[\Alph*.]
  \item 
    \begin{enumerate}[\arabic*.]
      \item At root:
        \begin{align*}
          H(Y) &= 1\\
          H(Y\mid Education) &= \frac{4}{10}H(\frac{1}{4}, \frac{3}{4}) + \frac{3}{10}H(\frac{1}{3}, \frac{2}{3}) + \frac{3}{10}H(\frac{3}{3}, 0) = 0.6\\
          \implies IG(Y ; Education) &= H(Y) - H(Y \mid Education) = 1 - 0.4 = 0.6\\ 
          H(Y\mid Age) &= \frac{5}{10}H(\frac{3}{5}, \frac{2}{5}) + \frac{5}{10}H(\frac{2}{5}, \frac{3}{5}) = 0.971\\
          \implies IG(Y; Age) &= H(Y) - H(Y\mid Age) = 1- 0.971 = 0.029\\ 
          H(Y\mid Experience) &= \frac{6}{10}H(\frac{3}{6}, \frac{3}{6}) + \frac{4}{10}H(\frac{2}{4}, \frac{2}{4}) = 1\\
          \implies IG(Experience) &= H(Y) - H(Y\mid Experience) = 0
        \end{align*}
        Since Education has the highest information gain, we choose it as root. Since all $Education = PhD$ have same classification, we only build a new subtree for $Education = Masters, Bachelors$. For $Education = Masters$:
        \begin{align*}
          H(Y\mid Education = Masters) &= H(\frac{1}{3}, \frac{2}{3}) = 0.918\\
          H(Y\mid Education = Masters, Experience) &= \frac{2}{3}H(\frac{1}{2}, \frac{1}{2}) + \frac{1}{3}H(1, 0) = 0.667\\
          \implies IG(Y\mid Education = Masters; Experience) &= 0.918 - 0.667 = 0.251\\
          H(Y\mid Education = Masters, Age) &= \frac{1}{3}H(\frac{1}{1}, \frac{0}{1}) + \frac{2}{3}H(\frac{0}{2}, \frac{2}{2}) = 0\\
          \implies IG(Y\mid Education = Masters; Age) &= 0.918 - 0 = 0.918\\
        \end{align*}
        Since Age has the highest information gain, we choose it as subroot. For $Education = Bachelors$:
        \begin{align*}
          H(Y\mid Education = Bachelors) &= H(\frac{1}{4}, \frac{3}{4}) = 0.811\\
          H(Y\mid Education = Bachelors, Experience) &= \frac{2}{4}H(\frac{2}{2}, \frac{0}{2}) + \frac{2}{4}H(\frac{1}{2}, \frac{1}{2}) = 0.5\\
          \implies IG(Y\mid Education = Bachelors; Experience) &= 0.811 - 0.5 = 0.311\\
          H(Y\mid Education = Bachelors, Age) &= \frac{2}{4}H(\frac{0}{2}, \frac{2}{2}) + \frac{2}{4}H(\frac{1}{2}, \frac{1}{2}) = 0.5\\
          \implies IG(Y\mid Education = Bachelors; Age) &= 0.811 - 0.5 = 0.311\\
        \end{align*}
        With equal information gain, we tiebreak choosing Experience as subroot. We are left with splitting by Age in $Experience = High$
        \begin{center}
        \incimg[0.4]{a1}
        \end{center}

      \item Outliers would be the applicant with Bachelors, High Experience and Age $\geq$ 30, as well as the applicant with Masters and age $< 3$
        \begin{center}
        \incimg[0.4]{a2}
        \end{center}

      \item Based on the predicted and actual decisions, $TP = 3, TN = 1, FP = 1, FN = 2$ so that\\
        $Accuracy = \frac{4}{7}$, $Precision = \frac{3}{4}$, $Recall = \frac{3}{5}$, $F1 = 0.667$ where $F1 = 0.667 > 0.6$ which indicates the model achieves a stronger balance than expected providing more reliable predictions for the positive class.
 
    \end{enumerate}

  \item 
    \begin{enumerate}[\arabic*.]
      \item Construct $X = \left(\begin{array}{cccccc} 1 & 2 & 1 & 4 & 2 & 1\\ 1 & 3 & 2 & 9 & 6 & 4\\ 1 & 5 & 3 & 25 & 15 & 9\\ 1 & 7 & 4 & 49 & 28 & 16\\ 1 & 8 & 5 & 64 & 40 & 25\\ 1 & 9 & 6 & 81 & 54 & 36 \end{array}\right)$ then $w = \left(\begin{array}{c} 7.5\\ -4\\ 6.5\\ -9.5\\ 33.5\\ -28 \end{array}\right)$, giving $\hat{y} = 7.5 - 4x_1 + 6.5x_2 - 9.5x_1^2 + 33.5x_1x_2 - 28x_2^2$

      \item Column $x_3 = x_1^2+2x_1x_2+x_2^2$ is a linear combination of the existing columns\\$\implies X$ loses full column rank $\implies X^TX$ is singular. We can drop one of the dependent columns.
    \end{enumerate}

  \item 
    \begin{enumerate}[\arabic*.]
      \item For $\hat{y} = 2$, $L_{MSE} = 0.005, L_{MAE} = 0.05$;\\For $\hat{y} = 4$, $L_{MSE} = 0.405, L_{MAE} = 0.45$

      \item MSE magnifies large outliers
    \end{enumerate}

  \pagebreak
  \item 
    \begin{enumerate}[\arabic*.]
      \item Given $y = x^2$, $\frac{dy}{dx} = 2x$, then $x_{t+1} = x_t - \gamma 2 x_t = (1-2\gamma)x_t$\\\vspace{1em}
        \begin{tabular}{c r r r}
        \hline
        $\gamma$ & $t$ & $x_t$ & $y_t=x_t^2$ \\
        \hline
        all   & 0 & 5         & 25 \\
        \hline
        & 1 & -95       & 9025 \\
        & 2 & 1805      & 3258025 \\
        10   & 3 & -34295    & 1176147025 \\
        & 4 & 651605    & 424589076025 \\
        & 5 & -12380495 & 153276656445025 \\
        \hline
        & 1 & -5  & 25 \\
        & 2 & 5   & 25 \\
        1    & 3 & -5  & 25 \\
        & 4 & 5   & 25 \\
        & 5 & -5  & 25 \\
        \hline
        & 1 & 4      & 16 \\
        & 2 & 3.2    & 10.24 \\
        0.1  & 3 & 2.56   & 6.5536 \\
        & 4 & 2.048  & 4.1943 \\
        & 5 & 1.6384 & 2.6844 \\
        \hline
         & 1 & 4.9    & 24.01 \\
         & 2 & 4.802  & 23.0592 \\
        0.01 & 3 & 4.706  & 22.1461 \\
         & 4 & 4.6118 & 21.2691 \\
         & 5 & 4.5196 & 20.4268 \\
        \hline
        \end{tabular}
        $\therefore \gamma = 0.1$ converges the fastest

      \item Add learning rate decay or an adaptive learning rate
    \end{enumerate}
\end{enumerate}
%%%%%%%%%%%%%%%%%%%%%%%%%%%%%%%%%%%%%%%%%%%%%%%%%%%%%%
%                       End                          %
%%%%%%%%%%%%%%%%%%%%%%%%%%%%%%%%%%%%%%%%%%%%%%%%%%%%%%

\end{document}

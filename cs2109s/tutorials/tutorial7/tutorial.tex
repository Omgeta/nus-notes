\documentclass[12pt, a4paper]{article}

\usepackage[a4paper, margin=1in]{geometry}

\usepackage[utf8]{inputenc}
\usepackage[mathscr]{euscript}
\let\euscr\mathscr \let\mathscr\relax
\usepackage[scr]{rsfso}
\usepackage{amssymb,amsmath,amsthm,amsfonts}
\usepackage[shortlabels]{enumitem}
\usepackage{multicol,multirow}
\usepackage{lipsum}
\usepackage{balance}
\usepackage{calc}
\usepackage[colorlinks=true,citecolor=blue,linkcolor=blue]{hyperref}
\usepackage{import}
\usepackage{xifthen}
\usepackage{pdfpages}
\usepackage{transparent}
\usepackage{tabularx}

\newcommand{\incfig}[2][1.0]{
    \def\svgwidth{#1\columnwidth}
    \import{./figures/}{#2.pdf_tex}
}
\newcommand{\incimg}[2][1.0]{
  \includegraphics[width=#1\columnwidth]{./figures/#2}
}


\input{letterfonts}

\newcommand{\mytitle}{CS2109S Tutorial 7}
\newcommand{\myauthor}{github/omgeta}
\newcommand{\mydate}{AY 25/26 Sem 1}

\begin{document}
\raggedright
\footnotesize
\begin{center}
{\normalsize{\textbf{\mytitle}}} \\
{\footnotesize{\mydate\hspace{2pt}\textemdash\hspace{2pt}\myauthor}}
\end{center}
\setlist{topsep=-1em, itemsep=-1em, parsep=2em}
%%%%%%%%%%%%%%%%%%%%%%%%%%%%%%%%%%%%%%%%%%%%%%%%%%%%%%
%                      Begin                         %
%%%%%%%%%%%%%%%%%%%%%%%%%%%%%%%%%%%%%%%%%%%%%%%%%%%%%%
\begin{enumerate}[\Alph*.]
  \item 
    \begin{enumerate}[\arabic*.]
      \item $a = ReLU((W^{[1]})^{\top}\begin{bmatrix}1\\x\end{bmatrix}) = ReLU(\begin{bmatrix}0.8\\-0.7\end{bmatrix}) = \begin{bmatrix}0.8\\0\end{bmatrix}$ 
      \item $\hat{y} = ReLU((W^{[2]})^{\top}\begin{bmatrix}1\\a\end{bmatrix}) = ReLU(\begin{bmatrix}0.5\\-0.38\end{bmatrix}) = \begin{bmatrix}0.5\\0\end{bmatrix}$ 
      \item $L(\hat{y}, y) = \frac{1}{2}\sum^2_{i=1} (\hat{y}_i - y_i)^2 = 0.485$
    \end{enumerate}

  \item 
    \begin{enumerate}[\arabic*.]
      \item $f(x) = |x - 1|$

      \item No; identity will not perform any transformation, sigmoid is incompatible with the output range, and ReLU here with a single neuron only can achieve one half of the transformation 

      \item Hidden layer ReLU: $W^{[1]} = \begin{bmatrix}-1&1\\1&-1\end{bmatrix}$\\
        Output layer Identity: $W^{[2]} = \begin{bmatrix}0\\1\\1\end{bmatrix}$

      \item They are needed to represent non-linear relationships 
    \end{enumerate}

  \item 
    \begin{enumerate}[\arabic*.]
      \item 
        \begin{enumerate}[(\alph*.)]
          \item $\frac{\partial J_{BCE}(W)}{\partial \hat{y}} = -y \frac{\partial}{\partial \hat{y}}\log(\hat{y}) + (1-y) \frac{\partial}{\partial \hat{y}}(1-\hat{y}) = -\frac{y}{\hat{y}} + \frac{1-y}{1-\hat{y}}$

          \item $\frac{\partial J_{BCE}(W)}{\partial \hat{f}} = \frac{\partial J_{BCE}(W)}{\partial \hat{y}} \frac{\partial \hat{y}}{\partial f} = \left(-\frac{y}{\hat{y}} + \frac{1-y}{1-\hat{y}}\right) \cdot \hat{y}(1-\hat{y}) = \hat{y} - y$ 

          \item Since $\frac{\partial f}{\partial W_2} = x_2$, then $\frac{\partial J_{BCE}(W)}{\partial W_2} = \frac{\partial J_{BCE}(W)}{\partial f} x_2$
        \end{enumerate}

      \item They are needed to deal with imbalanced data so not to be dominated by the majority class. They should be set such that $\alpha |A| \approx \beta |B|$ so $\alpha = 5.5, \beta = 0.55$
    \end{enumerate}

  \item 
    \begin{enumerate}[\arabic*.]
      \item 
        \begin{enumerate}[(\alph*.)]
          \item Sigmoid on early layers squashes values mostly to 0, so the gradient per layer also becomes almost zero.

          \item Use ReLU instead for bigger gradients.
        \end{enumerate}

      \item 
        \begin{enumerate}[(\alph*.)]
          \item Gradients can be negative so the neurons can adjust back out. 

          \item When $a = 1$, LeakyReLU degenerates into the identify function. 
        \end{enumerate}
    \end{enumerate}
\end{enumerate}
%%%%%%%%%%%%%%%%%%%%%%%%%%%%%%%%%%%%%%%%%%%%%%%%%%%%%%
%                       End                          %
%%%%%%%%%%%%%%%%%%%%%%%%%%%%%%%%%%%%%%%%%%%%%%%%%%%%%%

\end{document}

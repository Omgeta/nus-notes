\documentclass[12pt, a4paper]{article}

\usepackage[utf8]{inputenc}
\usepackage[mathscr]{euscript}
\let\euscr\mathscr \let\mathscr\relax
\usepackage[scr]{rsfso}
\usepackage{amssymb,amsmath,amsthm,amsfonts}
\usepackage[shortlabels]{enumitem}
\usepackage{multicol,multirow}
\usepackage{lipsum}
\usepackage{balance}
\usepackage{calc}
\usepackage[colorlinks=true,citecolor=blue,linkcolor=blue]{hyperref}
\usepackage{import}
\usepackage{xifthen}
\usepackage{pdfpages}
\usepackage{transparent}
\usepackage{tabularx}

\newcommand{\incfig}[2][1.0]{
    \def\svgwidth{#1\columnwidth}
    \import{./figures/}{#2.pdf_tex}
}
\newcommand{\incimg}[2][1.0]{
  \includegraphics[width=#1\columnwidth]{./figures/#2}
}


\usepackage{ifthen}
\usepackage[landscape]{geometry}
\usepackage[shortlabels]{enumitem}

\ifthenelse{\lengthtest { \paperwidth = 11in}}
    { \geometry{top=.5in,left=.5in,right=.5in,bottom=.5in} }
	{\ifthenelse{ \lengthtest{ \paperwidth = 297mm}}
		{\geometry{top=1cm,left=1cm,right=1cm,bottom=1cm} }
		{\geometry{top=1cm,left=1cm,right=1cm,bottom=1cm} }
	}

\pagestyle{empty}
\makeatletter
\renewcommand\thesection{\arabic{section}.}
\renewcommand{\section}{\@startsection{section}{1}{0mm}%
                                {-1ex plus -.5ex minus -.2ex}%
                                {0.05ex}%x
                                {\normalfont\normalsize\bfseries}}
\renewcommand{\subsection}{\@startsection{subsection}{2}{0mm}%
                                {-1ex plus -.5ex minus -.2ex}%
                                {0.05ex}%
                                {\normalfont\small\bfseries}}
\renewcommand{\subsubsection}{\@startsection{subsubsection}{3}{0mm}%
                                {-1ex plus -.5ex minus -.2ex}%
                                {0.05ex}%
                                {\normalfont\footnotesize\bfseries}}
\newcommand{\colbreak}{\vfill\null\columnbreak}
\makeatother
\setcounter{secnumdepth}{1}
\setlength{\parindent}{0pt}
\setlength{\parskip}{0.7em}

\setlist[itemize]{itemsep=0.6ex, topsep=-2pt, partopsep=0pt, parsep=0pt}
\setlist[enumerate]{itemsep=0.6ex, topsep=-2pt, partopsep=0pt, parsep=0pt}

\input{letterfonts}

\newcommand{\mytitle}{CS3210 Parallel Computing}
\newcommand{\myauthor}{github/omgeta}
\newcommand{\mydate}{AY 25/26 Sem 2}

\begin{document}
\raggedright
\footnotesize
\begin{multicols*}{3}
\setlength{\premulticols}{1pt}
\setlength{\postmulticols}{1pt}
\setlength{\multicolsep}{1pt}
\setlength{\columnsep}{2pt}

{\normalsize{\textbf{\mytitle}}}\\
{\footnotesize{\mydate\hspace{2pt}\textemdash\hspace{2pt}\myauthor}}
%%%%%%%%%%%%%%%%%%%%%%%%%%%%%%%%%%%%%%%%%%%%%%%%%%%%%%
%                      Begin                         %
%%%%%%%%%%%%%%%%%%%%%%%%%%%%%%%%%%%%%%%%%%%%%%%%%%%%%%
\section{Introduction}
Parallel Computing is the simultaneous use of multiple processing units to solve problems efficiently. 

Program Parallelization Steps:
\begin{enumerate}[\roman*.]
  \item Decomposition: split problem into parallel tasks
    \begin{itemize}[leftmargin=*]\vspace{2pt}
      \item Granularity: size of task
    \end{itemize}
  \item Scheduling: assign tasks to processes/threads
    \begin{itemize}[leftmargin=*]\vspace{2pt}
      \item Orchestration: imposed synchronisation and communication of tasks to satisfy dependencies
    \end{itemize}
  \item Mapping: bind processes/threads to hardware processing units (e.g. CPU cores)
\end{enumerate}

\subsection{Processes}
Process is an abstraction for a running program:
\begin{enumerate}[\roman*.]
  \item Process ID (PID): uniquely identifies a process 
  \item Process State: indicates execution status
  \item Process Control Block: stores execution context (registers, resources, exclusive address space)
\end{enumerate}
\begin{enumerate}[$-$]\vspace{-2pt}
  \item Costly: syscall overhead, data structures must be allocated, communication goes through OS
\end{enumerate}\vspace{-1pt}
{\centering
  \incimg[.9]{process}
\par}
Interprocess Communication (IPC):
\begin{enumerate}[\roman*.]
  \item Shared Memory (e.g. locks, semaphores)
  \item Message Passing 
  \item UNIX Pipes and Signals
\end{enumerate}
\colbreak

\subsection{Threads}
Threads are independent execution flows within a process:
\begin{enumerate}[\roman*.]
  \item Shared: Resources, Address Space 
  \item Private: Thread ID, Registers, "Stack" (diff. SP)
\end{enumerate}\vspace{-1pt}
\begin{enumerate}[$+$]
  \item Efficient: cheaper creation and context switching
  \item Resource Sharing: process resources can be shared
  \item Concurrent: multithreading on different cores
\end{enumerate}\vspace{-1pt}
{\centering
  \incimg[0.8]{thread}
\par}
Implementations:
\begin{enumerate}[\roman*.]
  \item User-Threads are managed by a userspace library:
    \begin{enumerate}[$+$]\vspace{2pt}
    \item Fast: context switches have low overhead
  \end{enumerate}
  \begin{enumerate}[$-$]\vspace{6pt}
    \item No Parallelism: OS cannot map threads to different execution units 
    \item Threads blocking will block whole process
  \end{enumerate}
\item Kernel-Threads are implemented in the OS:
  \begin{enumerate}[$+$]\vspace{2pt}
    \item Parallel: threads can run across many CPUs 
  \end{enumerate}
  \begin{enumerate}[$-$]\vspace{6pt}
    \item Slower: thread operations are system calls
  \end{enumerate}
\end{enumerate}

Mapping Models:
\begin{enumerate}[\roman*.]
  \item Many-to-One:
    \begin{itemize}[leftmargin=*]\vspace{2pt}
      \item All user-level threads are mapped to one process
      \item User library schedules user-threads
    \end{itemize}
  \item One-to-One:
    \begin{itemize}[leftmargin=*]\vspace{2pt}
      \item Each user-thread mapped to one kernel-thread
      \item No user library scheduler needed
    \end{itemize}
  \item Many-to-Many:
    \begin{itemize}[leftmargin=*]\vspace{2pt}
      \item User-threads mapped to set of kernel-threads
      \item Library scheduler may move user-threads to different kernel-threads
    \end{itemize}
\end{enumerate}
\vspace{-1em}
\colbreak 
\subsection{Synchronisation}
Synchronisation is required by concurrent threads to coordinate access to shared resources in critical sections.

Critical Section Properties:
\begin{enumerate}[\roman*.]
  \item Safety: nothing bad happens
    \begin{itemize}[leftmargin=*]\vspace{3pt}
      \item Mutual Exclusion: at most $1$ thread in CS
    \end{itemize}
  \item Liveness: something good happens 
    \begin{itemize}[leftmargin=*]\vspace{3pt}
      \item Progress: if no thread is in CS, a waiting thread should be granted access 
      \item Bounded Wait: a waiting thread requesting to enter the CS, will eventually enter
    \end{itemize}
  \item Performance: overhead of entering/exiting CS is small w.r.t work done within it 
\end{enumerate}

Symptoms of Incorrect Synchronization:
\begin{enumerate}[\roman*.]
  \item Deadlock: all threads blocked; iff all conditions met
    \begin{itemize}[leftmargin=*]\vspace{3pt}
      \item Mutual Exclusion: $\geq 1$ resource held exclusively
      \item Hold \& Wait: $\geq 1$ process holding one resource while waiting for another
      \item No Pre-emption: resources must be yielded
      \item Circular Wait: set of processes waiting in circles
    \end{itemize}
  \item Livelock: due to deadlock avoidance, no progress 
  \item Starvation: a thread is unable to access CS 
  \item Race Condition: bug from program outcome depending on unpredictable access order
\end{enumerate}

Mechanisms:
\begin{enumerate}[\roman*.]
  \item Lock: primitive \lstinline|acquire()|, \lstinline|release()|
    \begin{itemize}[leftmargin=*]\vspace{3pt}
      \item Spinlock: busy-wait; wastes cycles
      \item Mutex: blocking
      \item Uses hardware atomics (\lstinline|TSL|) or toggle interrupts
    \end{itemize}
  \item Semaphore: atomic counter $\geq 0$; \lstinline|wait()|, \lstinline|signal()|   
    \begin{itemize}[leftmargin=*]\vspace{3pt}
      \item Binary/Mutex: represents single access 
      \item Counting/General: number of threads in CS $\leq N$
      \item Can be manipulated by different thread 
      \item No relation to data being controlled
    \end{itemize}
  \item Monitors: thread-safe high-level data structures
  \item Messages
\end{enumerate}
\vspace{-1em}
\colbreak
\subsubsection{Producer-Consumer}
Processes share a bounded buffer of fixed size $K$, where producers add items until buffer full and consumers remove items when not empty.

Blocking Solution (init \lstinline|not_full = K, not_emp = 0|):\\
\begin{minipage}{0.45\columnwidth}
  \begin{lstlisting}
// Producer
x = produce();

wait(not_full);
wait(mutex);
buffer.add(x);
signal(mutex);
signal(not_emp);
  \end{lstlisting}
\end{minipage}
\begin{minipage}{0.5\columnwidth}
  \begin{lstlisting}
// Consumer
wait(not_emp);
wait(mutex);
x = buffer.get();
signal(mutex);
signal(not_full);

consume(item);
  \end{lstlisting}
\end{minipage}

Busy Waiting Solution:\\
\begin{minipage}{0.45\columnwidth}
  \begin{lstlisting}
// Producer
while(!can_prod);
x = produce();

wait(mutex);
if (K > count) {
  buf[in] = x;
  in = (in+1)%K;
  count++;
  can_cons = 1;
} else {
  can_prod = 0;
}
signal(mutex);
  \end{lstlisting}
\end{minipage}
\begin{minipage}{0.5\columnwidth}
  \begin{lstlisting}
// Consumer
while(!can_cons);

wait(mutex);
if (count > 0) {
  x = buf[out];
  out = (out+1)%K;
  count--;
  can_prod = 1;
} else {
  can_cons = 0;
}
signal(mutex);

consume(x);
  \end{lstlisting}
\end{minipage}

\colbreak
\subsubsection{Reader-Writer}
Processes share a critical region, where readers can read simultaneously but writers must have exclusive access.
\vspace{-1em}
\begin{minipage}{0.45\columnwidth}
  \begin{lstlisting}
// Writer
wait(empty);
write();
signal(empty);
  \end{lstlisting}
\end{minipage}
\begin{minipage}{0.5\columnwidth}
  \begin{lstlisting}
// Reader 
wait(mutex);
readers++;
if (readers == 1)
  wait(empty);
signal(mutex);

read();

wait(mutex);
readers--;
if (readers == 0)
  signal(empty);
signal(mutex);
  \end{lstlisting}
\end{minipage}
\begin{enumerate}[$-$]
  \item Writer Starvation: possible if readers keep entering and \lstinline|empty| is never signalled. Solutions:
    \begin{itemize}[leftmargin=*]\vspace{3pt}
      \item Queue: all pass through \lstinline|wait(queue)| 
      \item Writer-Priority: readers entering must \lstinline|wait(readTry)|; but first waiting writer blocks new readers with \lstinline|wait(readTry)| and does \lstinline|signal(readTry)| when done 
    \end{itemize}
\end{enumerate}

\subsubsection{Barrier}
\begin{lstlisting}
int i_am_last = 0;
wait(mutex);
count++;
if (count == N) {
  i_am_last = 1;
  signal(barrier); 
}
signal(mutex);

wait(barrier);
if (!i_am_last) 
  signal(barrier);
\end{lstlisting}
\colbreak
\subsubsection{Dining Philosophers}
$N$ philosophers around a circular table, with a single chopstick between. $2$ chopsticks are needed to eat.
\begin{minipage}{0.48\columnwidth}
  \begin{lstlisting}
// Philosopher
void p(int i) {
  while (1) {
    // think
    take(i);
    eat();
    put(i);
  }
}

void safe_to_eat (int i) {
  if (state[i] == HUNGRY && state[LEFT] != EATING && state[RIGHT] != EATING) {
    state[i] = EATING;
    signal(s[i]);
  }
}
  \end{lstlisting}
\end{minipage}
\begin{minipage}{0.45\columnwidth}
  \begin{lstlisting}
// Take Chopstick
void take(int i) {
  wait(mutex);
  state[i] = HUNGRY;
  safe_to_eat(i);
  signal(mutex);
  wait(s[i]);
}

// Put Chopstick
void put(int i) {
  wait(mutex);
  state[i] = THINKING;
  safe_to_eat(LEFT);
  safe_to_eat(RIGHT)
  signal(mutex);
}
  \end{lstlisting}
\end{minipage}
Limited Eater Solution, when at most $N-1$ philosophers eat concurrently, it's guaranteed at least one can eat:
\begin{lstlisting}
void philosopher(int i) {
  while (1) {
    // think 
    wait(seats);
    wait(chopstick[LEFT]);
    wait(chopstick[RIGHT]);
    eat();
    signal(chopstick[RIGHT]);
    signal(chopstick[LEFT]);
    signal(seats);
  }
}
\end{lstlisting}
\colbreak
\section{Architecture}
Levels of Parallelism:
\begin{enumerate}[\roman*.]
  \item Bit-Level: operate on wider words
  \item Instruction-Level:
    \begin{itemize}[leftmargin=*]\vspace{2pt}
      \item Pipeline: split execution into stages; multiple instructions can occupy different stages in same cycle (if no dependency hazards)
      \item Superscalar: duplicate pipelines; issue multiple independent instructions per cycle
    \end{itemize}
  \item Thread-Level: execute multiple threads on the same core concurrently
  \item Processor-Level: execute multiple threads/processes in parallel on multiple cores/processors
\end{enumerate}

Flynn Taxonomy of Parallel Architectures:
\begin{enumerate}[\roman*.]
  \item Single Instruction Single Data (SISD)
  \item Single Instruction Multiple Data (SIMD)
  \item Multiple Instruction Single Data (MISD)
  \item Multiple Instruction Multiple Data (MIMD)
\end{enumerate}

\subsection{Distributed-Memory}
Distributed-Memory Systems are multicomputer systems:
\begin{enumerate}[\roman*.]
  \item Node: indep. unit of processor and private memory 
  \item Interconnection Network: supports message passing for data exchange between nodes
\end{enumerate}
\begin{center}
  \incimg[0.7]{distributedmemory}
\end{center}
\colbreak

\subsection{Shared-Memory}
Shared-Memory Systems are multiprocessor systems:
\begin{enumerate}[\roman*.]
  \item Shared Memory Provider: maintains shared address space abstraction across processors
\end{enumerate}
\begin{enumerate}[$+$]\vspace{-2pt }
  \item No need to partition code or data
  \item Efficient communication: no need to physically move data between processors
\end{enumerate}
\begin{enumerate}[$-$]\vspace{-2pt }
  \item Special synchonisation constructs required 
  \item Limited scalability due to contention 
\end{enumerate}

Uniform Memory Access (UMA):
\begin{enumerate}[\roman*.]
  \item Same memory access latency
\end{enumerate}
\begin{enumerate}[$+$]\vspace{-2pt }
  \item Suitable with few processors due to contention
\end{enumerate}
{\centering
  \incimg[0.7]{uma}
\par}

Non-Uniform Memory Access (NUMA):
\begin{enumerate}[\roman*.]
  \item Different memory access latency for every processor (e.g. local memory faster than remote)
  \item Cache Coherent NUMA (ccNUMA):\\each node has coherent cache to reduce contention
\end{enumerate}
{\centering
  \incimg[0.7]{numa}
\par}

Cache Only Memory Access (COMA):
\begin{enumerate}[\roman*.]
  \item Memory blocks behave as caches 
  \item Data migrates dynamically and continuously according to cache coherence scheme
\end{enumerate}
{\centering
  \incimg[0.7]{coma}
\par}
\colbreak
\subsection{Cache Coherence}
Caches reduce memory access latency and provide high bandwidth data transfer to CPU: 
\begin{enumerate}[\roman*.]
  \item Cache Size: larger cache reduces misses but increases access time (addressing complexity) 
  \item Block Size: unit of transfer; larger blocks improve spatial locality, but longer transfer time
\end{enumerate}

Write Policy:
\begin{enumerate}[\roman*.]
  \item Write-through: write to both cache and memory
    \begin{enumerate}[$+$]\vspace{3pt}
      \item Always get newest value of memory block 
    \end{enumerate}
    \begin{enumerate}[$-$]\vspace{6pt}
      \item Slower due to many memory accesses; mitigate with write buffer
    \end{enumerate}
  \item Write-back: write to cache only and set dirty bit; write to memory on cache block eviction
    \begin{enumerate}[$+$]\vspace{3pt}
      \item Fewer write operations to memory
    \end{enumerate}
    \begin{enumerate}[$-$]\vspace{6pt}
      \item Memory may contain invalid entries 
    \end{enumerate}
\end{enumerate}

Cache Coherence ensures all processors have consistent view of memory through their local cache. Properties:
\begin{enumerate}[\roman*.]
  \item Program Order: if $P$ writes to $X$, then n.f.w to $X$, $P$ should read same value from $X$ 
  \item Write Propagation: if $P_1$ writes to $X$, n.f.w to $X$, $P_2$ should read same value from $X$
  \item Write Serialization: if $V_1 \rightarrow X$, then $V_2 \rightarrow X$,\\all $P_i$ should never read $V_2$ then $V_1$ from $X$
\end{enumerate}

Implementations: 
\begin{enumerate}[\roman*.]
  \item Software-based: use page-fault to propagate writes
  \item Snooping-based: no centralized directory; caches snoop a shared bus and react to transactions
  \item Directory-based: centralized directory records sharing status; common in NUMA
\end{enumerate}

Implications:
\begin{enumerate}[\roman*.]
  \item "Increase" memory latency and lower cache hit rate
  \item Cache Ping-Pong: multiple processors repeatedly read/modify same shared variable
  \item False Sharing: processors write to different addresses which map to same cache line
\end{enumerate}
\vspace{-1em}
\colbreak
\subsection{Memory Consistency}
Memory Consistency constrains the in which order of memory operations by a thread become visible to other threads across different memory locations. 
\begin{enumerate}[\roman*.]
  \item Programmer: helps reason about correctness and program behaviour 
  \item System/Compiler: decide what memory reordering is allowed (to hide write latencies)
\end{enumerate}
Constraints:
\begin{enumerate}[\roman*.]
  \item $W \rightarrow R$: write to $X$ completes before read from $Y$
  \item $R \rightarrow R$: read from $X$ completes before read from $Y$
  \item $R \rightarrow W$: read from $X$ completes before write to $Y$
  \item $W \rightarrow W$: write to $X$ completes before write to $Y$
\end{enumerate}

Sequential Consistency Model (SC):
\begin{enumerate}[\roman*.]
  \item Processors issue mem. operations in program order
  \item Result is as if all operations were interleaved in one sequential order seen consistenly by all processors
\end{enumerate}
\begin{enumerate}[$+$]\vspace{-2pt }
  \item Intuitive 
  \item Maintains all constraints
\end{enumerate}
\begin{enumerate}[$-$]\vspace{-2pt }
  \item Can result in loss of performance
\end{enumerate}

Relaxed Consistency Model:
\begin{enumerate}[\roman*.]
  \item Relax ordering if data dependencies allow\\(i.e. memory operations to same memory location):
    \begin{itemize}[leftmargin=*]\vspace{3pt}
      \item $R\rightarrow W$: anti-dependence (WAR)
      \item $W\rightarrow W$: output dependence (WAW)
      \item $W\rightarrow R$: flow dependence (RAW)
    \end{itemize}
  \item Allow overriding mechanism for programmers
  \item Write-to-Read Program Order:
    \begin{itemize}[leftmargin=*]\vspace{3pt}
      \item Total Store Ordering (TSO): return value written earlier without waiting for serialization
      \item Processor Consistency (PC): return value of any write before propagation or serializion
    \end{itemize}
  \item Write-to-Write Program Order:
    \begin{itemize}[leftmargin=*]\vspace{3pt}
      \item Partial Store Ordering (PSO): later write can propagate/serialize before earlier write 
    \end{itemize}
\end{enumerate}
\begin{enumerate}[$+$]\vspace{-2pt }
  \item Hide latencies by overlapping indep. operations
\end{enumerate}
\vspace{-1em}
\colbreak
\subsection{Interconnection Networks}
Interconnection forms the backbone of communication between processors, memories/caches, and I/O devices.

\subsection{Topology}
Topology is the geometrical shape of the connection.

Direct Interconnections (or Static, P2P) have endpoints connected directly, usually of same type (e.g. core--core). Metrics: 
\begin{enumerate}[\roman*.]
  \item Diameter, $\delta(G)$: maximum shortest-path distance between any pair of nodes
    \begin{itemize}[leftmargin=*]\vspace{3pt}
      \item Small diameter $\Rightarrow$ shorter message transmission
    \end{itemize}
  \item Degree, $g(v)$: no. of nodes adjacent to node $v$; $g(G)$ is the maximum node degree in network $G$
    \begin{itemize}[leftmargin=*]\vspace{3pt}
      \item Small degree $\Rightarrow$ lower node hardware overhead.
    \end{itemize}
  \item Bisection Width, $B(G)$: minimum no. of edges removed to split the network into two equal halves
    \begin{itemize}[leftmargin=*]\vspace{3pt}
      \item Bisection bandwidth $BW(G)$: total bandwidth between the two bisections 
      \item Measures capacity under simult. transmission
    \end{itemize}
  \item Node Connectivity, $nc(G)$: minimum no. of nodes whose failure disconnects the network
    \begin{itemize}[leftmargin=*]\vspace{3pt}
      \item Determines robustness of the network
    \end{itemize}
  \item Edge Connectivity, $ec(G)$: minimum no. of edges whose failure disconnects the network
    \begin{itemize}[leftmargin=*]\vspace{3pt}
      \item Determines number of independent paths between any pair of nodes
    \end{itemize}
\end{enumerate}

\renewcommand{\arraystretch}{1.05}
\setlength{\tabcolsep}{3pt}
\begin{tabular}{|c|c|c|c|c|c|}
\hline
& $n$ & $g(G)$ & $\delta(G)$ & $ec(G)$ & $B(G)$ \\\hline
Complete Graph & $n$ & $n-1$ & $1$ & $n-1$ & $(\frac{n}{2})^2$\\\hline 
Linear Array & $n$ & $2$ & $n-1$ & $1$ & $1$\\\hline
Ring & $n$ & $2$ & $\lfloor \frac{n}{2}\rfloor$ & $2$ & $2$\\\hline
$d$-dim Mesh & $r^d$ & $2d$ & $d(\sqrt[d]{n} - 1)$ & $d$ & $n^{\frac{d-1}{d}}$\\\hline
$d$-dim Torus & $r^d$ & $2d$ & $d\lfloor \frac{\sqrt[d]{n}}{2}\rfloor$ & $2d$ & $2n^{\frac{d-1}{d}}$\\\hline
$k$-dim Hypercube & $2^k$ & $\log n$ & $\log n$ & $\log n$ & $\frac{n}{2}$\\\hline
$k$-dim CCC & $k2^k$ & $3$ & $2k-1 + \lfloor \frac{k}{2}\rfloor$ & $3$ & $\frac{n}{2k}$\\\hline
Complete BT & $2^k-1$ & $3$ & $2\log \frac{n+1}{2}$ & $1$ & $1$\\\hline
$k$-ary $d$-cube & $k^d$ & $2d$ & $d\lfloor \frac{k}{2}\rfloor$ & $2d$ & $2k^{d-1}$\\\hline
\end{tabular}

\colbreak
Indirect Interconnection (or Dynamic) have endpoints interconnect through switches configured dynamically:
\begin{enumerate}[$+$]
  \item Reduce hardware costs by sharing switches \& links
\end{enumerate}
Metrics:
\begin{enumerate}[\roman*.]
  \item Cost: number of switches and links 
  \item Concurrent connections
\end{enumerate}

Types:
\begin{enumerate}[\roman*.]
  \item Bus: set of wires carry data from sender to receiver
    \begin{itemize}[leftmargin=*]\vspace{3pt}
      \item Only a pair of devices can communicate at a time 
      \item Bus arbiter used for coordination 
      \item Typically used for small number of processors
    \end{itemize}
  \item Crossbar: switch matrix (of $n\times m$ switches) from $n$ inputs to $m$ outputs
    \begin{itemize}[leftmargin=*]\vspace{3pt}
      \item Switch states: straight or direction change 
      \item Hardware costly $\rightarrow$ small number of processors
    \end{itemize}
  \item Multistage Switching: intermediate switches with connecting wires between neighbouring stages
    \begin{itemize}[leftmargin=*]\vspace{3pt}
      \item Goal: obtain small distance for any pair of input and output devices
    \end{itemize}
  \item Omega: unique path from each input to each output ($n\times n$)
    \begin{itemize}[leftmargin=*]\vspace{3pt}
      \item Construction: switch $(\alpha, i) \rightarrow (\beta,i+1)$ where $\beta$ is cyclic-left-shift of $\alpha$, and cyclic-left-shift + inversion of LSB
      \item $\log n$ stages of $\frac{n}{2}$ switches per stage
    \end{itemize}
  \item Butterfly:
    \begin{itemize}[leftmargin=*]\vspace{3pt}
      \item Construction: switch $(\alpha, i) \rightarrow (\alpha,i+1), (\alpha', i+1)$ where $\alpha, \alpha'$ differ in the $(i+1)^{th}$ bit from the left
    \end{itemize}
  \item Baseline:
    \begin{itemize}[leftmargin=*]\vspace{3pt}
      \item Construction: switch $(\alpha, i) \rightarrow (\beta,i+1)$ where $\beta$ is cyclic-right-shift of last ($k-i$) bits of $\alpha$, and inversion of LSB + same cyclic-right-shift
    \end{itemize}
\end{enumerate}

\colbreak

\subsection{Routing}
Routing determines path(s) from source to destination within a given topology.
Algorithm Classification:
\begin{enumerate}[\roman*.]
  \item Path Length: 
    \begin{itemize}[leftmargin=*]\vspace{3pt}
      \item Minimal: shortest-path always chosen 
      \item Non-minimal: shortest-path not necessary
    \end{itemize}
  \item Adaptivity: 
    \begin{itemize}[leftmargin=*]\vspace{3pt}
      \item Deterministic: always use same path for same pair of source and destination nodes 
      \item Adaptive: take into account network status\\(e.g. avoid congested path, avoid dead nodes)
    \end{itemize}
\end{enumerate}

Deterministic Algorithms:
\begin{enumerate}[\roman*.]
  \item XY Routing for 2D Mesh: 
    \begin{itemize}[leftmargin=*]\vspace{3pt}
      \item Move in $X$ direction until $X_{src} == X_{dst}$
      \item Move in $Y$ direction until $Y_{src} == Y_{dst}$
    \end{itemize}
  \item E-Cube Routing for Hypercube:
    \begin{itemize}[leftmargin=*]\vspace{3pt}
      \item Bit difference in source and target node address (hamming distance) $=$ number of hops
      \item From MSB$\rightarrow$LSB (or vice versa), find first different bit and go to neighbour node with bit corrected; at most $n$ hops
    \end{itemize}
  \item XOR-Tag Routing for Omega Network:
    \begin{itemize}[leftmargin=*]\vspace{3pt}
      \item Let $T = $ Source ID $\oplus$ Destination ID
      \item At stage $k$, go straight if bit $k = 0$, else crossover
    \end{itemize}
\end{enumerate}

\subsection{Additional}
More Questions:
\begin{enumerate}[\roman*.]
  \item Switching: how to transfer messages along a path 
  \item Flow Control: how to handle concurrent messaging 
\end{enumerate}

\subsection{Current Trends}
Ethernet: supports P2P and broadcast; common topologies include bus, and star with a logical bus.

InfiniBand: supports P2P and multicast; common topologies include fat-tree and torus.

\colbreak
\section{Programming Models}

Types of Parallelism:
\begin{enumerate}[\roman*.]
  \item Data Parallelism: same op. on different data elements (loop parallelism; SIMD/SPMD)
  \item Task Parallelism: independent tasks executed concurrently (statements, loops, function calls)
    \begin{itemize}[leftmargin=*]\vspace{3pt}
      \item Task Dependency Graph: DAG representing control dependency between tasks
        \begin{itemize}[leftmargin=*]\vspace{3pt}
          \item Critical Path Length: slowest completion
          \item Deg. Concurrency: $\displaystyle \frac{\text{Total Work}}{\text{Critical Path Length}}$
        \end{itemize}
    \end{itemize}
\end{enumerate}

Representation of Parallelism:
\begin{enumerate}[\roman*.]
  \item Implicit Parallelism: Haskell
  \item Implicit Scheduling: OpenMP
  \item Explicit Communication \& Sync: MPI, Pthreads
\end{enumerate}

Models of Coordination:
\begin{enumerate}[\roman*.]
  \item Shared Address Space:
    \begin{itemize}[leftmargin=*]\vspace{2pt}
      \item Communicate via shared variables; mutual exclusion via locks
      \item Matches shared-memory (UMA/NUMA); requires HW support; costly to scale 
    \end{itemize}
  \item Data Parallel:
    \begin{itemize}[leftmargin=*]\vspace{2pt}
      \item Map a function onto a large collection of data (historic: SIMD/vector; modern: CUDA)
      \item Rigid structure enables parallel scheduling; ideally no communication among invocations
    \end{itemize}
  \item Message Passing:
    \begin{itemize}[leftmargin=*]\vspace{2pt}
      \item Private address spaces; explicit send/receive 
      \item Matches distributed memory
      \item Point-to-point vs Global communication
      \item Blocking vs Non-Blocking 
      \item Buffered vs Non-Buffered 
      \item Synchronous vs Asynchronous
    \end{itemize}
\end{enumerate}

\colbreak

\subsection{Foster's Design Methodology}
\begin{enumerate}[\roman*.]
  \item Partitioning: problem into pieces/data 
    \begin{itemize}[leftmargin=*]\vspace{2pt}
      \item $\geq 10\times$ primitive tasks than cores, of similar size
      \item Minimize redundant computations and data
      \item Number of tasks increases with problem size
    \end{itemize}
  \item Communication: pass needed data among tasks 
    \begin{itemize}[leftmargin=*]\vspace{2pt}
      \item Balance communication among tasks
      \item Perform communication in parallel 
      \item Overlap computation with communication
      \item Communicate with small number of neighbours
    \end{itemize}
  \item Agglomeration: combine tasks into larger tasks 
    \begin{itemize}[leftmargin=*]\vspace{2pt}
      \item Number of tasks $\geq$ number of cores
      \item Reduce cost of task creation + communication
      \item Maintain scalability, simplify code
    \end{itemize}
  \item Mapping: tasks to execution units
    \begin{itemize}[leftmargin=*]\vspace{2pt}
      \item Maximize processor utilization; minimize IPC
      \item Optimal mapping is NP-hard $\Rightarrow$ heuristics 
      \item Consider static vs dynamic allocation;\\if static, tasks:cores $\ge 10{:}1$;\\if dynamic, allocator must not bottleneck
    \end{itemize}
\end{enumerate}

\subsection{Parallel Programming Patterns}
Patterns:
\begin{enumerate}[\roman*.]
  \item Fork--Join: \lstinline|fork| child tasks, then \lstinline|join|
  \item Parbegin--Parend: mark statements to execute in parallel, then sync at end
  \item SIMD: single instruction stream over multiple data elements (synchronous)
  \item SPMD: same program on multiple threads; different data via IDs (asynchronous)
  \item Master--Worker: master distributes tasks to workers
  \item Client--Server (MPMD): request-response model
  \item Task Pools: shared pool of tasks; workers dynamically dequeue tasks
  \item Producer--Consumer: produce tasks into buffer for consumers to retrieve; needs sync 
  \item Pipelining: split compute into stages for data stream; throughput improves via stage-overlap
\end{enumerate}
\vspace{-1em}
\colbreak
\subsection{OpenMP}
Shared-Memory Programming Model:
\begin{enumerate}[\roman*.]
  \item Threads communicate via shared variables; may also have private variables
  \item Avoid race conditions using mutual exclusion
\end{enumerate} 

\subsubsection{Parallel For (Data Parallel)}
\begin{lstlisting}
// Parallel outer loop; shared data + private indices
#pragma omp parallel for num_threads(8)
shared(a,b,result) private(i,j,k)
for (i = 0; i < size; i++)
  for (j = 0; j < size; j++)
    for (k = 0; k < size; k++)
      result[i][j] += a[i][k] * b[k][j];
\end{lstlisting}

\subsubsection{Mutex with OpenMP Lock}
\begin{lstlisting}
int count = 0;
omp_lock_t lock;
omp_init_lock(&lock);

#pragma omp parallel
{
  omp_set_lock(&lock);
  count = count + 1;
  omp_unset_lock(&lock);
}
\end{lstlisting}

\colbreak
\subsection{NVIDIA CUDA}
CUDA Programming Model:
\begin{enumerate}[\roman*.]
  \item Heterogeneous: Host (CPU) + Device (GPU)
  \item Kernel: function that runs on the device
  \item Thread: local registers and memory; \lstinline|threadIdx|
  \item Warp: execution unit of 32 threads scheduled together (SIMT); one instruction at a time
  \item SM (Streaming Multiprocessor): schedules warps with resources (register file, shared memory); thread blocks execute concurrently on an SM
\end{enumerate}

Function Qualifiers:\\\vspace{2pt}
{\centering
\begin{tabular}{|c|c|c|}
\hline
\textbf{Qualifier} & \textbf{Scope} & \textbf{Usage} \\
\hline
\lstinline|__host__|   & CPU & local to host \\
\hline
\lstinline|__device__| & GPU & local to device \\
\hline
\lstinline|__global__| & GPU & launch from host \\
\hline
\end{tabular}
\par}

Variable Qualifiers:\\\vspace{2pt} 
{\centering
\begin{tabular}{|c|c|c|c|c|}
\hline
\textbf{Qualifier} & \textbf{R/W} & \textbf{Cache} & \textbf{Scope} & \textbf{Lifetime} \\
\hline
\lstinline|__device__|   & R/W & No  & grid  & static \\
\hline
\lstinline|__constant__| & R & Yes & grid  & static \\
\hline
\lstinline|__shared__|   & R/W & No  & block & block \\
\hline
unqualified     & R/W & --  & thread & thread \\
\hline
\lstinline|__managed__|  & R/W & --  & global & static \\
\hline
\end{tabular}
\par}

Optimizations:
\begin{enumerate}[\roman*.]
  \item Reduce transfers: minimize host$\leftrightarrow$device copies; batch small copies; use pinned memory; overlap copy/compute with \lstinline|cudeMemcpyAsync| + streams
  \item Global memory: coalesce accesses; minimize global loads/stores; use \lstinline|__shared__|; avoid bank conflicts
  \item Occupancy: many warps to hide latency; threads/block multiple of 32; balance registers/shared-mem so $\ge 1$ block/SM
  \item Control flow: minimize warp divergence (avoid thread-ID dependent branches)
  \item Compute: prefer high-throughput ops; single precision faster; avoid int div/mod (use shifts/bitwise)
  \item Avoid multi-context: do not create multiple CUDA contexts per GPU within one application
\end{enumerate}

\subsubsection{Matrix Addition (Device)}
\begin{lstlisting}[basicstyle=\ttfamily\scriptsize]
__global__ 
void addMatrixG(float *a,float *b,float *c,int N) 
{
  int i = blockIdx.x * blockDim.x + threadIdx.x;
  int j = blockIdx.y * blockDim.y + threadIdx.y;
  int index = i + j * N;
  if (i < N && j < N) 
    c[index] = a[index] + b[index];
}
\end{lstlisting}

\subsubsection{Matrix Multiplication (Device)}
\begin{lstlisting}[basicstyle=\ttfamily\scriptsize]
__global__ 
void addMatrixG(float *a,float *b,float *c,int N) 
{
  int i = blockIdx.x * blockDim.x + threadIdx.x;
  int j = blockIdx.y * blockDim.y + threadIdx.y;
  int index = i + j * N;
  if (i < N && j < N) 
    c[index] = a[index] + b[index];
}
\end{lstlisting}

\subsubsection{Host Skeleton}
\begin{lstlisting}[basicstyle=\ttfamily\scriptsize]
void main(){
  // Assume host arrays: h_A, h_B, h_C; device pointers: d_A, d_B, d_C
  size_t bytes = (size_t)N * sizeof(float);
  cudaMalloc((void**)&d_A, bytes);
  cudaMalloc((void**)&d_B, bytes);
  cudaMalloc((void**)&d_C, bytes);

  // Copy inputs Host -> Device
  cudaMemcpy(d_A, h_A, bytes, cudaMemcpyHostToDevice);
  cudaMemcpy(d_B, h_B, bytes, cudaMemcpyHostToDevice);

  // Configure grid/block and launch kernel
  dim3 dimBlk(BLOCK_X, BLOCK_Y);    // e.g. (16,16)
  dim3 dimGrd((NX + BLOCK_X - 1)/BLOCK_X,
              (NY + BLOCK_Y - 1)/BLOCK_Y);
  kernel<<<dimGrd, dimBlk>>>(d_A, d_B, d_C, ...);

  // Copy result Device -> Host
  cudaMemcpy(h_C, d_C, bytes, cudaMemcpyDeviceToHost);

  // Free device memory
  cudaFree(d_A); cudaFree(d_B); cudaFree(d_C);
}
\end{lstlisting}

\colbreak
\subsection{MPI}
Message Passing Model:
\begin{enumerate}[\roman*.]
  \item Distributed memory 
  \item Explicitly represent parallelism with send/receive
\end{enumerate}

MPI Program Overview:
\begin{enumerate}[\roman*.]
  \item Initialize communications
  \item Communicate to coordinate computation
  \item Exit from message-passing system
\end{enumerate}

\subsubsection{Communication Semantics}
Protocol Choice:
\begin{enumerate}[\roman*.]
  \item Buffered: sender returns after copying data into a communication buffer (less idling; buffer overhead)
  \item Non-buffered: sender blocks until matching receive is encountered (idling + deadlock risks if mismatch)
\end{enumerate}

Local View:
\begin{enumerate}[\roman*.]
  \item Blocking: return $\Rightarrow$ safe to reuse resources/buffers
  \item Non-blocking: may return before completion; programmer must ensure completion (poll/check)
\end{enumerate}

Global View:
\begin{enumerate}[\roman*.]
  \item Synchronous: operation does not complete before both sides have started
  \item Asynchronous: sender can proceed without coordination with receiver
\end{enumerate}

Operations:
\begin{enumerate}[\roman*.]
  \item Sync + Blocking: \lstinline|MPI_Ssend|
  \item Async + Blocking: \lstinline|MPI_Send| / \lstinline|MPI_Recv|
  \item Sync + Non-blocking: \lstinline|MPI_Issend|
  \item Async + Non-blocking: \lstinline|MPI_Isend| / \lstinline|MPI_Irecv|
\end{enumerate}
* Note: blocking and non-blocking ops can be mixed

\colbreak

\subsubsection{Library Functions}
Initi, Finalize, Abort Calls:
\begin{enumerate}[\roman*.]
  \item \lstinline|int MPI_Init(int* argc, char** argv[])|
  \item \lstinline|int MPI_Finalize(void)|
  \item \lstinline|int MPI_Abort(MPI_Comm comm, int errorCode)|
\end{enumerate}

Point-to-Point Messaging Calls:
\begin{enumerate}[\roman*.]
  \item \lstinline|int MPI_Send(void* buf, int count, MPI_Datatype dt, int dst, int tag, MPI_Comm c)| 
  \item \lstinline|int MPI_Recv(void* buf, int count, MPI_Datatype dt, int src, int tag, MPI_Comm c, MPI_Status *status)|
\end{enumerate}

Message Format:
\begin{enumerate}[\roman*.]
  \item Data: start-buffer + count + datatype
  \item Envelope: destination/source (rank) + tag + communicator
  \item Receive buffer must be \(\ge\) message length
  \item Wildcards: \lstinline|MPI_ANY_SOURCE|, \lstinline|MPI_ANY_TAG|
  \item \lstinline|MPI_Status|: \lstinline|MPI_SOURCE|, \lstinline|MPI_TAG|, \lstinline|MPI_ERROR|
\end{enumerate}

Ordering:
\begin{enumerate}[\roman*.]
  \item 1 sender, 1 receiver: messages delivered in order
  \item \(>2\) processes: message delivery order undefined
\end{enumerate}

\subsubsection{Deadlocks}
Deadlock Types: 
\begin{enumerate}[\roman*.]
  \item Message Order: both sides wait on receives
  \item Buffer: no/small buffers $\Rightarrow$ sends cannot complete
\end{enumerate}

Secure MPI program: correctness does not depend on assumptions about MPI runtime properties.
\begin{itemize}
  \item Specify send/receive order\\(e.g. even: send$\rightarrow$recv, odd: recv$\rightarrow$send) 
\end{itemize}

\colbreak
\subsubsection{Collective Communication}

Process Groups are ordered sets of processes:
\begin{enumerate}[\roman*.]
  \item Unique rank identifier 
  \item Processes may be in multiple groups
  \item Communicator: communication domain for a group
    \begin{itemize}[leftmargin=*]\vspace{2pt}
      \item Intra-communicator: collectives in a group; default \lstinline|MPI_COMM_WORLD|
      \item Inter-communicator: P2P between two groups
    \end{itemize}
\end{enumerate}

Virtual Topology is a Cartesian-style communicator: 
\begin{enumerate}[\roman*.]
  \item \lstinline|MPI_Cart_create|, \lstinline|MPI_Cart_get|, \lstinline|MPI_Cartdim_get|,
\lstinline|MPI_Cart_coords|, \lstinline|MPI_Cart_rank|, \lstinline|MPI_Cart_shift|
\end{enumerate}

Collectives involve all processes in a communicator: 
\begin{enumerate}[\roman*.]
  \item Total Exchange: \lstinline|MPI_Alltoall|
  \item Multi-broadcast: \lstinline|MPI_Allgather|
  \item Multi-accumulation: \lstinline|MPI_Reduce_scatter|
  \item Scatter/Gather: \lstinline|MPI_Scatter| / \lstinline|MPI_Gather|
  \item Single Broadcast: \lstinline|MPI_Broadcast|
  \item Single Accumulation: \lstinline|MPI_Reduce|
  \item Barrier: \lstinline|MPI_Barrier|
\end{enumerate}

\subsubsection{Hello World (Master Sends)}
\begin{lstlisting}[basicstyle=\ttfamily\scriptsize]
MPI_Init(&argc, &argv);
MPI_Comm_size(MPI_COMM_WORLD, &size);
MPI_Comm_rank(MPI_COMM_WORLD, &rank);
tag = 100;
if (rank == 0) {
  strcpy(message, "Hello World 2");
  for (i=1; i<size; i++)
    MPI_Send(message, 14, MPI_CHAR, i, tag, MPI_COMM_WORLD);
} else {
  MPI_Recv(message, 14, MPI_CHAR, 0, tag, MPI_COMM_WORLD, &status);
}
printf("node %d : %.13s\n", rank, message);
MPI_Finalize();
\end{lstlisting}

\colbreak
\section{Performance}
Performance is measured by primary goals:
\begin{enumerate}[\roman*.]
  \item Response Time: wall-clock duration of an execution
    \begin{itemize}[leftmargin=*]\vspace{3pt}
      \item User CPU Time: time spent on program 
      \item System CPU Time: time spent on OS routines 
      \item Waiting Time: time spent waiting for I/O and execution of other programs (timesharing) 
    \end{itemize}
  \item Throughput: average work completed per unit time
\end{enumerate}

\subsection{Sequential Programs}
User CPU Time:
\begin{gather*}
  T_{user} = N_{cycle}\times T_{cycle}
\end{gather*}
\begin{enumerate}[\roman*.]
  \item $\displaystyle T_{cycle} = \frac{1}{\text{clock rate}}$
  \item $\displaystyle N_{cycle} = \sum n_i\times CPI_i \;\approx\; N_{instr}\times CPI$
\end{enumerate}\vspace{1em}

User CPU Time (Memory Access; One-level Cache):
\begin{gather*}
  T_{user} = \bigl(N_{cycle} + N_{mm\_cycle}\bigr)\times T_{cycle}
\end{gather*}
\begin{enumerate}[\roman*.]
  \item $\displaystyle N_{mm\_cycle} = N_{read\_cycle} + N_{write\_cycle}$
  \item $\displaystyle N_{read\_cycle} = N_{reads}\times R_{read\_miss}\times N_{miss\_cycles}$
\end{enumerate}\vspace{1em}

Average Memory Access Time (Two-level Cache):
\begin{gather*}
  T_{read} = T_{hit} + R_{miss}\times T_{miss}
\end{gather*}
\begin{enumerate}[\roman*.]
  \item $\displaystyle T_{read} = T^{L1}_{hit} + R^{L1}_{miss}\times T^{L1}_{miss}$
  \item $\displaystyle T^{L1}_{miss} = T^{L2}_{hit} + R^{L2}_{miss}\times T^{L2}_{miss}$
  \item Global miss rate: $\displaystyle R^{L1}_{miss}\times R^{L2}_{miss}$
\end{enumerate}\vspace{1em}

Throughput:
\begin{enumerate}[\roman*.]
  \item $\displaystyle MIPS = \frac{N_{instr}}{T_{user} \times 10^6} = \frac{clock\_freq}{CPI \times 10^6}$
  \item $\displaystyle MFLOPS = \frac{N_{fl\_ops}}{T_{user} \times 10^6}$
\end{enumerate}

\colbreak
\subsection{Parallel Programs}
Parallel Execution Time on $p$ processors, $T_p$ consists:
\begin{enumerate}[\roman*.]
  \item Time for executing local computations
  \item Time for exchange of data between processors
  \item Time for synchronization between processors
  \item Waiting time
\end{enumerate}\vspace{1em}

Cost:
\begin{gather*}
  C_p = p \times T_p
\end{gather*}
\begin{enumerate}[\roman*.]
  \item Measures total work performed by all processors 
\item Cost-optimal: parallel program executes same total number of operations as fastest sequential program
\end{enumerate}\vspace{1em}

Speedup:
\begin{gather*}
  S_p = \frac{T_{*}}{T_p}
\end{gather*}
\begin{enumerate}[\roman*.]
  \item Measures the benefit of parallelism compared to best sequential algorithm time $T_*$ 
  \item Theoretically, $S_p \le p$ always holds
  \item In practice, $S_p > p$ (superlinear speedup) can occur (e.g. problem working set ``fits'' in the cache)
\end{enumerate}\vspace{1em}

Efficiency:
\begin{gather*}
  E_p = \frac{T_{*}}{C_p} = \frac{S_p}{p} = \frac{T_{*}}{p \times T_p}
\end{gather*}
\begin{enumerate}[\roman*.]
  \item Measures actual degree of speedup performance achieved compared to maximum
  \item Ideal speedup $S_p=p \;\Rightarrow\; E_p=1$
\end{enumerate}\vspace{1em}

\colbreak
\subsection{Scalability}
Scalability is the interaction between problem size and parallel machine size:
\begin{enumerate}[\roman*.]
  \item Impacts: load balancing, overhead, arithmetic intensity, locality (application dependent)
  \item Fixed problem size:
    \begin{itemize}[leftmargin=*]\vspace{3pt}
      \item Small $n$: overheads dominate $\Rightarrow$ suitable for small machine, poor on large machine
      \item Large $n$: working set may not fit on small machine (thrashing / exceeds cache / can't run)
    \end{itemize}
  \item Constraints:
    \begin{itemize}[leftmargin=*]
      \item Application: particles/processor (N-body), transactions/processor (distributed DB) 
      \item Resources: time-limited, memory-limited, problem-limited (same problem faster)
    \end{itemize}
\end{enumerate}

Amdahl's Law:
\begin{gather*}
  S_p(n) = \frac{T_*(n)}{f \cdot T_*(n) + \frac{1-f}{p}T_*(n)}
        = \frac{1}{f + \frac{1-f}{p}}
        \le \frac{1}{f}
\end{gather*}
\begin{enumerate}[\roman*.]
  \item Sequential fraction: $0 \leq f \leq 1$ limits speedup
  \item Discourage from making large parallel computers, focus on reducing sequential fraction $f$
  \item Rebuttal: circumvented for large problem size 
    \begin{itemize}[leftmargin=*]\vspace{3pt}
      \item Commonly $f$ is a function of $n$, $f(n)$
      \item Efficient parallel algorithm: $\lim_{n\rightarrow\infty} f(n)=0$
      \item Thus, $\lim_{n\rightarrow\infty}S_p = \frac{p}{1 + (p-1)f(n)} = p$ 
    \end{itemize}
\end{enumerate}

Gustafson's Law:
\begin{gather*}
  S_p(n) = \frac{\tau_f + \tau_v(n,1)}{\tau_f + \tau_v(n,p)}
\end{gather*}
\begin{enumerate}[\roman*.]
  \item Encouraged to use more processors to solve larger problem in the same time (in CPU-bound apps)
  \item $\tau_f$: constant sequential execution time 
  \item $\tau_v(n,p)$: parallel execution time on $p$ processors;\\if perfectly parallelizable (no overheads):
    \begin{itemize}[leftmargin=*]\vspace{3pt}
      \item $\tau_v(n,1)=T^*(n)-\tau_f$
      \item $\tau_v(n,p)=\frac{T^*(n)-\tau_f}{p}$
    \end{itemize}
  \item If $T^*(n)$ increases strongly monotonically with $n$, then $\lim_{n\rightarrow\infty} S_p(n)=p$
\end{enumerate}

\subsection{Communication}
Communication Performance Measures:
\begin{enumerate}[\roman*.]
  \item Bandwidth, $B$: maximum data rate (bytes/s)
  \item Throughput: effective bandwidth (bytes/s)
  \item Time of Flight, $T_{delay}$: time for first bit to arrive
  \item Byte Transfer Time, $t_B$: time to transmit one byte
  \item Transmission Time: time to transmit message
  \item Transport Latency: time to transfer a message (time of flight + transmission time)
  \item Sender Overhead, $O_{send}$: time to compute checksum, append header, and execute routing
  \item Receiver Overhead, $O_{recv}$: time to compare checksum and generate acknowledgement 
\end{enumerate}\vspace{1em}

Communication Time (message size $m$):
\begin{align*}
  T(m) &= O_{send} + T_{delay} + \frac{m}{B} + O_{recv}\\
       &= T_{overhead} + \frac{m}{B}\\
       &= T_{overhead} + t_B \cdot m
\end{align*}
where:
\begin{enumerate}[\roman*.]
  \item $T_{overhead}=O_{send}+T_{delay}+O_{recv}$ is independent of message size $m$
  \item Byte Transfer Time, $t_B=\frac{1}{B}$ 
  \item Assume no checksum error, and no network contention / congestion
\end{enumerate}

\subsection{Performance Analysis}
Practical Workflow:
\begin{enumerate}[\roman*.]
  \item Measure first; define target metric\\(latency vs throughput vs speedup)
  \item Localise the bottleneck\\(CPU vs memory vs sync vs I/O vs network)
  \item Change \textit{one} factor at a time; keep workload and environment fixed
  \item Validate: speedup should be explained by reduced dominant component
\end{enumerate}

\colbreak

\subsection{Instrumentation}
Performance Instrumentation finds performance metrics:
\begin{enumerate}[\roman*.]
  \item Latency: time to service a request 
  \item Response Time: wall-clock duration of an execution
  \item Throughput: average work completed per unit time
  \item IOPS: I/O operations per second
  \item Utilization: fraction of time a resource is busy
  \item Saturation: degree to which a resource has queued work it cannot service 
\end{enumerate}

Perspectives:
\begin{enumerate}[\roman*.]
  \item Resource Analysis: focus on utilization
    \begin{itemize}[leftmargin=*]\vspace{3pt}
      \item Time-based: average time the resource was busy 
      \item Capacity-based: resource throughput
    \end{itemize}
  \item Workload Analysis: focus on throughput, latency
\end{enumerate}

Methodologies:
\begin{enumerate}[\roman*.]
  \item Anti-Methodologies: undeliberate, street light (look for obvious issues), drunk man (tune at random)
  \item Problem Statement
  \item USE (per resource): Utilization, Saturation, Errors
  \item Monitoring: record metrics over time
\end{enumerate}

Tool Types \& Counters:
\begin{enumerate}[\roman*.]
  \item Observability: watch activity under workload (timing statements, performance counters)
  \item Static: examine system at rest
  \item Benchmarking: load test (caution: contention/production risk)
  \item Tuning: change defaults (danger: regressions now/later under load)
  \item Categorization:
    \begin{itemize}[leftmargin=*]\vspace{2pt}
      \item System-wide vs per-process
      \item Fixed counters (kernel-maintained metrics) vs event-based counters (enabled as needed)
      \item Profiling: samples/snapshots \quad vs \quad Tracing: record every event occurrence
    \end{itemize}
\end{enumerate}\vspace{0.5em}

\colbreak

\incimg[0.9]{linux_observability_tools}
\incimg[0.9]{linux_static_tools}
\incimg[0.9]{linux_benchmarking_tools}

\colbreak
\section{Energy Efficiency}
Energy-Efficiency Motivation:
\begin{enumerate}[\roman*.]
  \item Power dissipation has become a limiting factor (\textit{``power wall''})
  \item Higher performance $\Rightarrow$ more/faster computers $\Rightarrow$ more power/heat/cooling/space/cost
\end{enumerate}\vspace{0.5em}

\subsection{Mobile Computing}
Trends:
\begin{enumerate}[\roman*.]
  \item Energy-efficiency: decrease power consumption in hardware and increase performance
  \item Increase battery capacity
  \item ARM processing systems:
    \begin{itemize}[leftmargin=*]\vspace{3pt}
      \item Designed for power-efficiency 
      \item Closing in on x64 performance
      \item Highly customizable architecture
    \end{itemize}
\end{enumerate}\vspace{0.5em}

ARM big.LITTLE (low-power approach):
\begin{enumerate}[\roman*.]
  \item Big CPU: high performance for intensive workloads
  \item Little CPU: low power for majority workloads
  \item Switch between cores depending on demand
\end{enumerate}\vspace{0.5em}

\subsection{Heterogeneous Computing}
Heterogeneous Platforms:
\begin{enumerate}[\roman*.]
  \item Processors: brawny vs wimpy, big--little
  \item Supercomputers: accelerators 
  \item Data Centers: mixed server generations
  \item Cloud: heterogeneous resources with different price--performance
\end{enumerate}\vspace{0.5em}

Challenges:
\begin{enumerate}[\roman*.]
  \item Use heterogeneity to reduce power while maintaining performance
  \item Energy-efficient configuration for parallel applications is complex (scheduling)
  \item Burden shifts to programmers for portable code
\end{enumerate}\vspace{0.5em}

Data Centers:
\begin{enumerate}[\roman*.]
  \item Large-scale data centers consume significant electricity and emit substantial CO$_2$
  \item Cooling overhead can be comparable to computation energy
  \item Power Use Efficiency (PUE) = $\frac{\text{Total Energy}}{\text{IT Energy}}$: measures energy efficiency in data centers
\end{enumerate}\vspace{0.5em}

Google (energy-efficiency practices):
\begin{enumerate}[\roman*.]
  \item Continuously measure efficiency:
    \begin{itemize}[leftmargin=*]\vspace{3pt}
      \item Computation (IT) energy
      \item Overhead: cooling and conversions
    \end{itemize}
  \item Build custom, highly-efficient servers:
    \begin{itemize}[leftmargin=*]\vspace{3pt}
      \item Minimize AC/DC conversion losses
      \item Remove unnecessary parts 
      \item Strategic rack placement 
      \item Tune fan speeds for cooling
    \end{itemize}
  \item Extend equipment lifecycle
    \begin{itemize}[leftmargin=*]\vspace{3pt}
      \item Reuse, resell components
    \end{itemize}
  \item Control equipment temperature:
    \begin{itemize}[leftmargin=*]\vspace{3pt}
      \item Raise operating temperature (e.g. 26$^\circ$C) 
      \item Thermal modeling
    \end{itemize}
  \item Cooling with water instead of chillers:
    \begin{itemize}[leftmargin=*]\vspace{3pt}
      \item Evaporative cooling, sea water, recycled water
    \end{itemize}
\end{enumerate}\vspace{0.5em}

Reducing Energy Consumption:
\begin{enumerate}[\roman*.]
  \item Move less data:
    \begin{itemize}[leftmargin=*]\vspace{3pt}
      \item Reduce transfers to/from memory
      \item Exploit locality
      \item Use compression
    \end{itemize}
  \item Use specialized processing:
    \begin{itemize}[leftmargin=*]\vspace{3pt}
      \item Compute less (no parallel if it increases work)
      \item CPU-like cores + GPU-like throughput-optimized cores
      \item FPGAs (programmable hardware)
    \end{itemize}
\end{enumerate}

\section{Appendix}
\begin{minipage}{0.48\linewidth}
\centering
\begin{tabular}{|l c c|}
\hline
\textbf{name} & \textbf{prefix} & \textbf{multiplier} \\
\hline
exa  & E  & $10^{18}$ \\
peta & P  & $10^{15}$ \\
tera & T  & $10^{12}$ \\
giga & G  & $10^{9}$  \\
mega & M  & $10^{6}$  \\
kilo & K  & $10^{3}$  \\
\hline
milli & m  & $10^{-3}$ \\
micro & $\mu$ & $10^{-6}$ \\
nano  & n  & $10^{-9}$ \\
pico  & p  & $10^{-12}$ \\
\hline
\end{tabular}
\end{minipage}\hfill
\begin{minipage}{0.48\linewidth}
\centering
\begin{tabular}{|l c c|}
\hline
\textbf{name} & \textbf{prefix} & \textbf{multiplier} \\
\hline
exbi & Ei & $2^{60}$ \\
pebi & Pi & $2^{50}$ \\
tebi & Ti & $2^{40}$ \\
gibi & Gi & $2^{30}$ \\
mebi & Mi & $2^{20}$ \\
kibi & Ki & $2^{10}$ \\
\hline
\end{tabular}
\end{minipage}
%%%%%%%%%%%%%%%%%%%%%%%%%%%%%%%%%%%%%%%%%%%%%%%%%%%%%%
%                       End                          %
%%%%%%%%%%%%%%%%%%%%%%%%%%%%%%%%%%%%%%%%%%%%%%%%%%%%%%
\end{multicols*}
\end{document}

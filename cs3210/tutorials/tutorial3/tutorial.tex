\documentclass[12pt, a4paper]{article}

\usepackage[a4paper, margin=1in]{geometry}

\usepackage[utf8]{inputenc}
\usepackage[mathscr]{euscript}
\let\euscr\mathscr \let\mathscr\relax
\usepackage[scr]{rsfso}
\usepackage{amssymb,amsmath,amsthm,amsfonts}
\usepackage[shortlabels]{enumitem}
\usepackage{multicol,multirow}
\usepackage{lipsum}
\usepackage{balance}
\usepackage{calc}
\usepackage[colorlinks=true,citecolor=blue,linkcolor=blue]{hyperref}
\usepackage{import}
\usepackage{xifthen}
\usepackage{pdfpages}
\usepackage{transparent}
\usepackage{tabularx}

\newcommand{\incfig}[2][1.0]{
    \def\svgwidth{#1\columnwidth}
    \import{./figures/}{#2.pdf_tex}
}
\newcommand{\incimg}[2][1.0]{
  \includegraphics[width=#1\columnwidth]{./figures/#2}
}


\input{letterfonts}

\newcommand{\mytitle}{CS3210 Tutorial 3}
\newcommand{\myauthor}{github/omgeta}
\newcommand{\mydate}{AY 25/26 Sem 2}

\begin{document}
\raggedright
\normalsize
\begin{center}
{\normalsize{\textbf{\mytitle}}} \\
{\footnotesize{\mydate\hspace{2pt}\textemdash\hspace{2pt}\myauthor}}
\end{center}
\setlist{topsep=-1em, itemsep=-1em, parsep=2em}
%%%%%%%%%%%%%%%%%%%%%%%%%%%%%%%%%%%%%%%%%%%%%%%%%%%%%%
%                      Begin                         %
%%%%%%%%%%%%%%%%%%%%%%%%%%%%%%%%%%%%%%%%%%%%%%%%%%%%%%
\begin{enumerate}[Q\arabic*.]
  \item Instruction-Level Parallelism: pipeline, branch prediction, superscalar\\ 
    Thread-Level Parallelism: Simultaneous Multi-Threading (SMT)\\ 
    Processor-Level Parallelism: multi-core running in parallel, UPI shared memory, Omni-Path distributed memory\\
    IO Parallelism: Multiple DDR4 channels and PCIe lanes

  \item 
    \begin{enumerate}[(\alph*)]
      \item SISD: single core with single instruction stream on each memory address 

      \item MIMD: parallel instruction streams on different cores can act on different data 

      \item SIMD: vector instructions act over multiple memory locations 

      \item SISD: single core with single instruction stream, pipelined instructions still act on a single memory address at a specific time step 

      \item MISD: same paper (data) operated on by different students (instructions)
    \end{enumerate}

  \item 
    \begin{enumerate}[(\alph*)]
      \item False; Shared-Memory can be UMA or NUMA

      \item False; data locality affects latency

      \item True 
    \end{enumerate}

  \item 
    \begin{enumerate}[(\alph*)]
      \item False/True; binary semaphores emulate mutual exclusion but code depending on mutex ownership and semantics may be affected

      \item False; it depends on overheads, synchronisation and workload 
    \end{enumerate}

  \pagebreak
  \item 
    \begin{enumerate}[(\alph*)]
      \item Fragment 1:\\\incimg[0.15]{5a1}\\ 
        Fragment 2:\\\incimg[0.15]{5a2}

      \item Fragment 1: \\\incimg[0.25]{5b1}\\ 
        Max Concurrency $= 4$,\quad Average Concurrency $= \frac{59}{28} = 2.1$\\ 
        Fragment 2: \\\incimg[0.25]{5b2}\\ 
        Max Concurrency $= 4$,\quad Average Concurrency $= \frac{59}{33} = 1.78$

      \item Fragment 1 Speedup $= \frac{59}{28} = 2.1$\\ 
        Fragment 2 Speedup = $\frac{59}{33} = 1.78$

      \item Fragment 1: \\\incimg[0.25]{5d1}\\ 
        Speedup $= \frac{59}{32} = 1.84 < 2.1$\\
        Fragment 2: \\\incimg[0.25]{5d2}\\ 
        Speedup $= \frac{59}{35} = 1.69 < 1.78$
    \end{enumerate}

  \pagebreak
  \item 
    \begin{enumerate}[(\alph*)]
      \item Data Parallelism; SIMD; each independent output element $c_i$ is an independent dot product of the vector $b$ with row $a_i$

      \item Parbegin-Parend
    \end{enumerate}

  \item 
    \begin{enumerate}[(\alph*)]
      \item Task Parallelism: each stage is a task\\ 
        Pipelining: best for cases where each stage takes a similar time

      \item Task Parallelism: each stage is a task\\ 
      Producer--Consumer: producer reads from socket and processes, consumer writes back to socket and there are equal number of producers as consumers

      \item Task Parallelism: each stage is a task\\ 
        Producer--Consumer: same as (b) but more producers than consumers 

      \item Task Parallelism: each stage is a task\\ 
        Task Pool: pool of tasks are ready for jobs and assigned as necessary

      \item Same; but need explicit communication
    \end{enumerate}

  \item Average CPI (Translation 1) $= \frac{10}{5} = 2$\\
    Average CPI (Translation 1) $= \frac{9}{6} = 1.5$

  \item Execution Time ($A_1$) $= \frac{(5 + 1\cdot 2 + 1\cdot 3) \times 10^9}{2 \times 10^9} = 5s$\\
    Execution Time ($A_1$) $= \frac{(10 + 1\cdot 2 + 1\cdot 3) \times 10^9}{2 \times 10^9} = 7.5s$

    Conclusion: MIPS is not an accurate measurement of performance

  \item 
    \begin{enumerate}[(\alph*)]
      \item Sequential Time: $\frac{(100 + 100^2) \times 2}{10^9} = 20200 ns$\\
        Parallel Time ($p = 10$): $\frac{(100 + \frac{100^2}{10}) \times 2}{10^9} = 2200 ns$ ($9.18\times$ speedup)\\
        Parallel Time ($p = 100$): $\frac{(100 + \frac{100^2}{100}) \times 2}{10^9} = 400 ns$ ($50.6\times$ speedup)\\
        Parallel Time ($p = \infty$): $\frac{(100 + 0) \times 2}{10^9} = 200 ns$ ($101\times$ speedup)\\

      \item From (a), $T_1 = 20200ns$\\ 
        For $p=10$, $\displaystyle \frac{(N + \frac{N_{2}}{10})\times 2}{10^9} = 20200ns \implies N_{p=10} = 312 \implies S_{10}(312) = 9.669\times$\\
        For $p=100$, $\displaystyle \frac{(N + \frac{N_{2}}{100})\times 2}{10^9} = 20200ns \implies N_{p=100} = 956 \implies S_{100}(312) = 90.58\times$
    \end{enumerate}
\end{enumerate}
%%%%%%%%%%%%%%%%%%%%%%%%%%%%%%%%%%%%%%%%%%%%%%%%%%%%%%
%                       End                          %
%%%%%%%%%%%%%%%%%%%%%%%%%%%%%%%%%%%%%%%%%%%%%%%%%%%%%%

\end{document}

\documentclass[12pt, a4paper]{article}

\usepackage[utf8]{inputenc}
\usepackage[mathscr]{euscript}
\let\euscr\mathscr \let\mathscr\relax
\usepackage[scr]{rsfso}
\usepackage{amssymb,amsmath,amsthm,amsfonts}
\usepackage[shortlabels]{enumitem}
\usepackage{multicol,multirow}
\usepackage{lipsum}
\usepackage{balance}
\usepackage{calc}
\usepackage[colorlinks=true,citecolor=blue,linkcolor=blue]{hyperref}
\usepackage{import}
\usepackage{xifthen}
\usepackage{pdfpages}
\usepackage{transparent}
\usepackage{tabularx}

\newcommand{\incfig}[2][1.0]{
    \def\svgwidth{#1\columnwidth}
    \import{./figures/}{#2.pdf_tex}
}
\newcommand{\incimg}[2][1.0]{
  \includegraphics[width=#1\columnwidth]{./figures/#2}
}


\usepackage{ifthen}
\usepackage[landscape]{geometry}
\usepackage[shortlabels]{enumitem}

\ifthenelse{\lengthtest { \paperwidth = 11in}}
    { \geometry{top=.5in,left=.5in,right=.5in,bottom=.5in} }
	{\ifthenelse{ \lengthtest{ \paperwidth = 297mm}}
		{\geometry{top=1cm,left=1cm,right=1cm,bottom=1cm} }
		{\geometry{top=1cm,left=1cm,right=1cm,bottom=1cm} }
	}

\pagestyle{empty}
\makeatletter
\renewcommand\thesection{\arabic{section}.}
\renewcommand{\section}{\@startsection{section}{1}{0mm}%
                                {-1ex plus -.5ex minus -.2ex}%
                                {0.05ex}%x
                                {\normalfont\normalsize\bfseries}}
\renewcommand{\subsection}{\@startsection{subsection}{2}{0mm}%
                                {-1ex plus -.5ex minus -.2ex}%
                                {0.05ex}%
                                {\normalfont\small\bfseries}}
\renewcommand{\subsubsection}{\@startsection{subsubsection}{3}{0mm}%
                                {-1ex plus -.5ex minus -.2ex}%
                                {0.05ex}%
                                {\normalfont\footnotesize\bfseries}}
\newcommand{\colbreak}{\vfill\null\columnbreak}
\makeatother
\setcounter{secnumdepth}{1}
\setlength{\parindent}{0pt}
\setlength{\parskip}{0.7em}

\setlist[itemize]{itemsep=0.6ex, topsep=-2pt, partopsep=0pt, parsep=0pt}
\setlist[enumerate]{itemsep=0.6ex, topsep=-2pt, partopsep=0pt, parsep=0pt}

\input{letterfonts}

\newcommand{\mytitle}{CS3211 Parallel and Concurrent Prog.}
\newcommand{\myauthor}{github/omgeta}
\newcommand{\mydate}{AY 25/26 Sem 2}

\begin{document}
\raggedright
\footnotesize
\begin{multicols*}{3}
\setlength{\premulticols}{1pt}
\setlength{\postmulticols}{1pt}
\setlength{\multicolsep}{1pt}
\setlength{\columnsep}{2pt}

{\normalsize{\textbf{\mytitle}}}\\
{\footnotesize{\mydate\hspace{2pt}\textemdash\hspace{2pt}\myauthor}}
%%%%%%%%%%%%%%%%%%%%%%%%%%%%%%%%%%%%%%%%%%%%%%%%%%%%%%
%                      Begin                         %
%%%%%%%%%%%%%%%%%%%%%%%%%%%%%%%%%%%%%%%%%%%%%%%%%%%%%%
\section{Introduction}
Parallel Computing is the simultaneous use of multiple processing units to solve problems efficiently. 

Concurrency vs. Parallelism:
\begin{enumerate}[\roman*.]
  \item Concurrency: tasks overlap in time\\(may interleave on a single core)
  \item Parallelism: tasks run at the exact same time\\(simultaneously on different cores)
    \begin{itemize}[leftmargin=*]\vspace{3pt}
      \item Multiple cores, with each core supporting multiple hardware threads (SMT)
      \item Depends on available hardware threads
    \end{itemize}
\end{enumerate}

Program Parallelization Steps:
\begin{enumerate}[\roman*.]
  \item Decomposition: split problem into parallel tasks
    \begin{itemize}[leftmargin=*]\vspace{2pt}
      \item Granularity: size of task
    \end{itemize}
  \item Scheduling: assign tasks to processes/threads
    \begin{itemize}[leftmargin=*]\vspace{2pt}
      \item Orchestration: imposed synchronisation and communication of tasks to satisfy dependencies
    \end{itemize}
  \item Mapping: bind processes/threads to hardware processing units (e.g. CPU cores)
\end{enumerate}

Work Distribution Patterns:
\begin{enumerate}[\roman*.]
  \item Task Parallelism: divide work into different tasks; threads become specialists for specific tasks 
  \item Data Parallelism: divide data among threads, runing the same operation on different partitions
\end{enumerate}

\colbreak
\subsection{Processes}
Process is an abstraction for a running program:
\begin{enumerate}[$-$]\vspace{-2pt}
  \item High system call overhead 
  \item OS allocates and initializes data structures 
  \item Communication goes through the OS
\end{enumerate}\vspace{-1pt}
{\centering
  \incimg[.8]{process}
\par}
\subsection{Threads}
Threads are independent execution flows within a process:
\begin{enumerate}[\roman*.]
  \item Shared: Resources, Address Space 
  \item Private: Thread ID, Registers, "Stack" (diff. SP)
\end{enumerate}\vspace{-1pt}
\begin{enumerate}[$+$]
  \item Efficient: cheaper creation and context switching
  \item Resource Sharing: process resources can be shared
\end{enumerate}\vspace{-1pt}
{\centering
  \incimg[0.7]{thread}
\par}
Implementations:
\begin{enumerate}[\roman*.]
  \item User-Threads are managed by a userspace library:
    \begin{enumerate}[$+$]\vspace{3pt}
    \item Fast: context switches have low overhead
  \end{enumerate}
  \begin{enumerate}[$-$]\vspace{5pt}
    \item Cannot parallelize to different cores 
    \item Threads blocking will block whole process
  \end{enumerate}
\item Kernel-Threads are implemented in the OS:
  \begin{enumerate}[$+$]\vspace{3pt}
    \item Parallel: threads can run across many CPUs 
  \end{enumerate}
  \begin{enumerate}[$-$]\vspace{5pt}
    \item Slower: thread operations are system calls
  \end{enumerate}
\end{enumerate}
\vspace{-1em}
\colbreak 
\subsection{Synchronisation}
Synchronisation is required by concurrent threads to coordinate access to shared resources in critical sections.

Race Condition is when two concurrent threads access a shared resource without synchronisation, and at least one thread modifies the shared resource.

Data Race is when two concurrent memory accesses target the same location, with both not being reads or sync operations.

Critical Section Properties:
\begin{enumerate}[\roman*.]
  \item Safety: nothing bad happens
    \begin{itemize}[leftmargin=*]\vspace{3pt}
      \item Mutual Exclusion: at most $1$ thread in CS
    \end{itemize}
  \item Liveness: something good happens 
    \begin{itemize}[leftmargin=*]\vspace{3pt}
      \item Progress: if no thread is in CS, a waiting thread should be granted access 
      \item Bounded Wait: a waiting thread requesting to enter the CS, will eventually enter
    \end{itemize}
  \item Performance: overhead of entering/exiting CS is small w.r.t work done within it 
\end{enumerate}

Symptoms of Incorrect Synchronization:
\begin{enumerate}[\roman*.]
  \item Deadlock: all threads blocked; iff all conditions met
    \begin{itemize}[leftmargin=*]\vspace{3pt}
      \item Mutual Exclusion: $\geq 1$ resource held exclusively
      \item Hold \& Wait: $\geq 1$ process holding one resource while waiting for another
      \item No Pre-emption: resources must be yielded
      \item Circular Wait: set of processes waiting in circles
    \end{itemize}
  \item Livelock: due to deadlock avoidance, no progress 
  \item Starvation: a thread is unable to access CS 
\end{enumerate}

Mechanisms:
\begin{enumerate}[\roman*.]
  \item Lock: primitive \lstinline|acquire()|, \lstinline|release()|
  \item Semaphore: atomic counter $\geq 0$; \lstinline|wait()|, \lstinline|signal()|   
  \item Monitors: thread-safe high-level data structures
  \item Messages: explicit sync with atomic data transfer
\end{enumerate}
\vspace{-1em}
\colbreak
\section{C++}
\lstset{style=cpp}
\subsection{Threads}

Operations:
\begin{enumerate}[\roman*.]
  \item Construct:
    \begin{itemize}[leftmargin=*]\vspace{3pt}
      \item Function: \lstinline|std::thread t(f, args...)|
      \item Callable object: \lstinline|std::thread t(F{}, args...)|
      \item Lambda: \lstinline|std::thread t([]{ ... })|
    \end{itemize}
  \item Wait Once: \lstinline|t.join()|
    \begin{enumerate}[leftmargin=*, label=$\bullet$]\vspace{3pt}
      \item Check with \lstinline|t.joinable()|
    \end{enumerate}
  \item Detach: \lstinline|t.detach()|
    \begin{enumerate}[leftmargin=*, label=$-$]\vspace{3pt}
      \item Extra care with local variables: detached thread may outlive their scope (dangling references).
    \end{enumerate}
  \item Pass Args By Reference: \lstinline|std::ref(x)|
  \item Move Ownership (cannot copy): \lstinline|m2 = move(m)|
\end{enumerate}

\subsection{Mutexes}

Operations:
\begin{enumerate}[\roman*.]
  \item Construct: \lstinline|std::mutex m|
  \item Explicit: \lstinline|m.lock()|, \lstinline|m.unlock()|
  \begin{itemize}[leftmargin=*]\vspace{3pt}
    \item Multiple (avoid deadlocks):\\\lstinline|std::lock(m1, m2, ...)| 
  \end{itemize}
  \begin{enumerate}[leftmargin=*, label=$-$]\vspace{5pt}
    \item Must unlock on every exit path, incl. exceptions
  \end{enumerate}
  \item RAII: \lstinline|std::lock_guard<std::mutex> g(m)| 
  \begin{itemize}[leftmargin=*]\vspace{3pt}
    \item Adopt Existing Locked Mutex: \lstinline|std::lock_guard g(m, std::adopt_lock)|
    \item Multiple (avoid deadlocks):\\\lstinline|std::scoped_lock lk(m1, m2, ...)| 
  \end{itemize}
  \begin{enumerate}[leftmargin=*, label=$+$]\vspace{5pt}
    \item Locks on construction and unlocks on scope exit
  \end{enumerate}
  \item Flexible: \lstinline|std::unique_lock<std::mutex> lk(m)| 
  \begin{itemize}[leftmargin=*]\vspace{3pt}
    \item Adopt Existing Locked Mutex: \lstinline|std::unique_lock lk(m, std::adopt_lock)|
    \item Defer Lock Until Later: \lstinline|std::unique_lock lk(m, std::defer_lock)|
  \end{itemize}
  \begin{enumerate}[leftmargin=*, label=$+$]\vspace{5pt}
    \item RAII unlocks on scope exit if mutex owning 
    \item Allows multiple lock/unlock within a scope 
    \item Needed for condition variables
  \end{enumerate}
\end{enumerate}
\vspace{-1em}
\colbreak
\subsection{Condition Variables}

Condition variables sync actions without busy-waiting.

Operations:
\begin{enumerate}[\roman*.]
  \item {Construct: \lstinline|std::condition_variable cv|} 
    \begin{itemize}[leftmargin=*]\vspace{3pt}
      \item Works only on \lstinline|std::unique_lock<std::mutex>|
      \item Generic (any mutex-like, less efficient):\\\lstinline|std::condition_variable_any cva|
    \end{itemize}
  \item Wait: \lstinline|cv.wait(lk)| / \lstinline|cv.wait(lk, pred)| 
    \begin{itemize}[leftmargin=*]\vspace{3pt}
      \item Unlocks while blocking, then locks before return
    \end{itemize}
    \begin{enumerate}[leftmargin=*, label=$-$]\vspace{5pt}
      \item Spurious Wakeups: predicate may be checked any number of times; needs no side effects
    \end{enumerate}
  \item Wake One Thread: \lstinline|cv.notify_one()|
  \item Wake All Threads: \lstinline|cv.notify_all()|
\end{enumerate}

\subsection{Monitors}
Monitors encapsulate shared data and operations enforced by a mutex and condition variables.
\begin{enumerate}[\roman*.]
  \item One thread executes monitor operations at a time 
  \item Threads wait for conditions and state changes
\end{enumerate}
Example:
\begin{lstlisting}[basicstyle=\ttfamily\scriptsize]
class BoundedBuffer {
  std::mutex m;
  std::condition_variable cv;
  std::queue<int> q;
  size_t K;

public:
  void push(int x) {
    std::unique_lock<std::mutex> lk(m);
    cv.wait(lk, [&]{ return q.size() < K; });
    q.push(x);
    cv.notify_one();
  }

  int pop() {
    std::unique_lock<std::mutex> lk(m);
    cv.wait(lk, [&]{ return !q.empty(); });
    int x = q.front(); q.pop();
    cv.notify_one();
    return x;
  }
};
\end{lstlisting}
\colbreak
\subsection{Memory Model}

Memory model is the contract to programmers about how memory operations will be reordered by the compiler.
\begin{enumerate}[\roman*.]
  \item Indep. read/write may be reordered to optimize
  \item Ordering Relations:
    \begin{itemize}[leftmargin=*]\vspace{3pt}
      \item Sequenced-Before: order within a single thread
      \item Synchronizes-With: cross-thread with sync ops 
      \item Happens-Before: transitive visibility relation of Sequenced-Before $\cup$ Synchronizes-With 
      \item Modification Order: order of writes per atomic
    \end{itemize}
  \item SC-for-DRF: modern compilers guarantee data race free programs behave as if sequentially consistent
    \begin{itemize}[leftmargin=*]\vspace{3pt}
      \item Guaranteed if every atomic uses \lstinline|seq_cst|
    \end{itemize}
\end{enumerate}

Atomics (\lstinline|std::atomic<T>|):
\begin{enumerate}[\roman*.]
  \item Load: \lstinline|a.load(std::memory_order)|
  \item Store: \lstinline|a.store(val, std::memory_order)| 
  \item May use internal locks (not always faster) 
    \begin{itemize}[leftmargin=*]\vspace{3pt}
      \item \lstinline|a.is_lock_free()|/ \lstinline|T::is_always_lock_free|
    \end{itemize}
\end{enumerate}

Memory Models:
\begin{enumerate}[\roman*.]
  \item Sequentially Consistent (\lstinline|seq_cst|)
  \item Relaxed (\lstinline|relaxed|):
    \begin{itemize}[leftmargin=*]\vspace{3pt}
      \item No \emph{synchronizes-with}; minimal cross-thread ordering guarantees
      \item Still atomic; participates in modification order for that atomic variable
    \end{itemize}
  \item Acquire-Release (\lstinline|acquire|, \lstinline|release|, \lstinline|acq_rel|)
    \begin{itemize}[leftmargin=*]\vspace{3pt}
      \item If an acquire load reads-from a release store on the same atomic $\Rightarrow$ \emph{synchronizes-with}
      \item Writes before the release become visible after the acquire (in the acquiring thread)
    \end{itemize}
\end{enumerate}

Fences (\lstinline|std::atomic_thread_fence(memory_order)|):
\begin{enumerate}[\roman*.]
  \item Enforces ordering without modifying data
  \item Typically constrain reorder around relaxed atomics
\end{enumerate}

\colbreak
\subsection{Concurrent Data Structures}
\lstset{style=cpp}
Concurrent data structures let multiple threads access it safely without race conditions or broken invariants.

Granularity:
\begin{enumerate}[\roman*.]
  \item Coarse: one mutex protects the whole structure
    \begin{itemize}[leftmargin=*]\vspace{3pt}
      \item Simple + safe, but serializes all operations
    \end{itemize}
  \item Fine: multiple mutexes protect indep. parts
    \begin{itemize}[leftmargin=*]\vspace{3pt}
      \item More true concurrency, but higher complexity
    \end{itemize}
  \item Reader/Writer Split: allow concurrent reads, exclusive writes (e.g. \lstinline|std::shared_mutex|).
\end{enumerate}

Lock-based Examples:
\begin{enumerate}[\roman*.]
  \item Stack (one global mutex):
    \begin{enumerate}[leftmargin=*, label=$+$]\vspace{3pt}
      \item Safe for concurrent calls, exception-safe
    \end{enumerate}
    \begin{enumerate}[leftmargin=*, label=$-$]\vspace{5pt}
      \item Work is serialized
      \item No built-in waiting for items: caller must poll \lstinline|empty()| or catch \lstinline|empty_stack| exceptions
    \end{enumerate}
 \item Queue (one mutex + condvar):
    \begin{enumerate}[leftmargin=*, label=$+$]\vspace{3pt}
      \item \lstinline|wait_and_pop()| blocks until an item exists
    \end{enumerate}
    \begin{enumerate}[leftmargin=*, label=$-$]\vspace{5pt}
      \item Issue: \lstinline|notify_one()| on thread can throw
        \begin{itemize}[leftmargin=*]\vspace{3pt}
          \item Fixes: \lstinline|notify_all()| (costly), \lstinline|notify_one()| on exception, use shared pointers
        \end{itemize}
    \end{enumerate}
  \item Queue (\lstinline|std::shared_ptr<T>| node):
    \begin{enumerate}[leftmargin=*, label=$+$]\vspace{3pt}
      \item Ensures data deleted when no longer needed
      \item Better exception safety; shorter lock hold time
    \end{enumerate}
    \begin{enumerate}[leftmargin=*, label=$-$]\vspace{5pt}
      \item Still serialized by one global mutex
    \end{enumerate}
  \item Queue (fine; dummy node + separate locks):
    \begin{enumerate}[leftmargin=*, label=$\bullet$]\vspace{3pt}
      \item Add dummy node so if empty \lstinline|front == back|
      \item Avoid race on the same node: ensure \lstinline|front->next| and \lstinline|back->next| are not the same
    \end{enumerate}
    \begin{enumerate}[leftmargin=*, label=$+$]\vspace{3pt}
      \item Separate \lstinline|front_mutex| / \lstinline|back_mutex| $\Rightarrow$ concurrent \lstinline|push()| and \lstinline|pop()|
    \end{enumerate}
    \begin{enumerate}[leftmargin=*, label=$-$]\vspace{5pt}
      \item More complex invariants and locking discipline
    \end{enumerate}

  \item Queue (super-fine; per-node locks):  
    \begin{enumerate}[leftmargin=*, label=$+$]\vspace{3pt}
      \item Synchronizes-with push and pop threads  
    \end{enumerate}
    \begin{enumerate}[leftmargin=*, label=$-$]\vspace{5pt}
      \item Correctness complexity increases (avoid UAF; lock ordering becomes subtle)
    \end{enumerate}
\end{enumerate}

Nonblocking Data Structures use no blocking library calls:
\begin{enumerate}[\roman*.]
  \item Obstruction-free: any given thread completes in bounded steps if others are paused
  \item Lock-free: if multiple threads are operating, some thread completes in bounded steps
    \begin{itemize}[leftmargin=*]\vspace{3pt}
      \item Threads need not do same operations 
      \item If a thread is suspended, other threads must still be able to complete without waiting
    \end{itemize}
    \begin{enumerate}[leftmargin=*, label=$+$]\vspace{5pt}
      \item Maximum concurrency: some thread makes progress every step
      \item Robust: if a thread dies mid-operation, only that thread's data is lost; others proceed
    \end{enumerate}
    \begin{enumerate}[leftmargin=*, label=$-$]\vspace{5pt}
      \item Livelock possible: two threads repeatedly restart because each sees the other's changes
      \item Can reduce overall performance even if individual waiting time drops
      \item Atomics can be much slower than non-atomic ops (hardware sync on shared atomics)
      \item Memory contention / write propagation $\Rightarrow$ cache ping-pong under contention
      \item False sharing can also cause cache ping-pong (harder to identify)
    \end{enumerate}
  \item Wait-free: if multiple threads are operating, every thread completes in bounded steps
    \begin{itemize}[leftmargin=*]\vspace{3pt}
      \item Must be starvation-free
      \item Algorithms with unbounded retries because of clashes with other threads are not wait-free 
    \end{itemize}
\end{enumerate}

Guidelines:
\begin{enumerate}[\roman*.]
  \item Prototype with \lstinline|std::memory_order_seq_cst|
  \item Use a lock-free memory reclamation scheme:
  \begin{itemize}[leftmargin=*]\vspace{3pt}
    \item Track how many threads reference an object; delete when no longer referenced
    \item Recycle nodes
  \end{itemize}
  \item Watch out for ABA:
  \begin{itemize}[leftmargin=*]\vspace{3pt}
    \item Include an ABA counter alongside the variable
  \end{itemize}
  \item Identify busy-wait loops and help the other thread
\end{enumerate}

\colbreak
\colbreak
\section{Go}
\lstset{style=go}

\subsection{Goroutines}
Goroutines are lightweight function tasks running in the same address space.

Operations:
\begin{enumerate}[\roman*.]
  \item Start: \lstinline|go f(args...)|
  \item Lifetime:
    \begin{itemize}[leftmargin=*]\vspace{3pt}
      \item Program exits when \lstinline|main| returns 
      \item Not GC'd: avoid goroutine leaks
    \end{itemize}
  \item Blocks only block the thread, no other goroutines
  \item Preemptable: by Go's runtime
\end{enumerate}

\subsection{Channels}
Channels coordinate goroutines by communication.

Operations:
\begin{enumerate}[\roman*.]
  \item Make: \lstinline|ch := make(chan T)| / \lstinline|make(chan T, C)|
    \begin{itemize}[leftmargin=*]\vspace{3pt}
      \item Directional:\\\lstinline|chan<- T| (send-only), \lstinline|<-chan T| (recv-only)
    \end{itemize}
  \item Send: \lstinline|ch <- v|
    \begin{itemize}[leftmargin=*]\vspace{3pt}
      \item Blocks if channel is full (buffered) /\\until a receiver (unbuffered)
    \end{itemize}
  \item Receive: \lstinline|v, ok := <-ch|
    \begin{itemize}[leftmargin=*]\vspace{3pt}
      \item Blocks if channel is empty (can cause deadlocks)
      \item \lstinline|ok| false $\iff$ \lstinline|v| is default from closed channel
    \end{itemize}
  \item Close: \lstinline|close(ch)|
    \begin{itemize}[leftmargin=*]\vspace{3pt}
      \item Allowed from closed channel any number of times
      \item Sending to a closed channel panics
    \end{itemize}
  \item Range: \lstinline|for v := range ch { ... }| runs until channel is closed and drained
\end{enumerate}

Ownership Pattern:
\begin{enumerate}[\roman*.]
  \item Owner goroutine makes, writes, and closes channel
  \item Expose only \lstinline|<-chan T| to consumers
\end{enumerate}

\colbreak
\subsection{select}
\lstinline|select| composes channel operations:
\begin{enumerate}[\roman*.]
  \item Blocks if no case is ready (unless \lstinline|default|)
  \item If multiple cases are ready: runtime picks pseudo-randomly (uniform over ready cases)
  \item Stopping:
    \begin{itemize}[leftmargin=*]\vspace{3pt}
      \item \lstinline|time.After(d)|: channel that fires after \lstinline|d|
      \item \lstinline|done| channel to signal stop
    \end{itemize}
\end{enumerate}

For-select Loop: 
\begin{lstlisting}[basicstyle=\ttfamily\scriptsize]
for {
  select {
  case <-done:
    return
  default:
    // do work
  }
}
\end{lstlisting}

\subsection{sync Package}
Use mostly in small scopes (e.g., inside a \lstinline|struct|); prefer higher-level coordination with channels when possible.
\begin{enumerate}[\roman*.]
  \item WaitGroup: \lstinline|var wg sync.WaitGroup| 
    \begin{itemize}[leftmargin=*]\vspace{3pt}
      \item \lstinline|wg.Add(n)|, \lstinline|wg.Done()|, \lstinline|wg.Wait()|, \lstinline|wg.Go(f)|
    \end{itemize}
  \item Mutex: \lstinline|var mu sync.Mutex|:
    \begin{itemize}[leftmargin=*]\vspace{3pt}
      \item Locks: \lstinline|mu.Lock()|, \lstinline|mu.Unlock()|
    \end{itemize}
  \item Read-Write Mutex: \lstinline|var rw sync.Mutex|:
    \begin{itemize}[leftmargin=*]\vspace{3pt}
      \item Reads: \lstinline|rw.RLock()|, \lstinline|rw.RUnlock()|
      \item Writes: \lstinline|rw.Lock()|, \lstinline|rw.Unlock()|
    \end{itemize}
  \item Cond: \lstinline|c := sync.NewCond(&mu)|
    \begin{itemize}[leftmargin=*]\vspace{3pt}
      \item Wait: \lstinline|c.Wait()| unlocks \& blocks, then re-locks before returning
      \item Wake: \lstinline|c.Signal()| / \lstinline|c.Broadcast()|
    \end{itemize}
  \item Once: \lstinline|once.Do(f)| runs \lstinline|f| at most once
  \item Pool: \lstinline|p.Get()| / \lstinline|p.Put(x)|
\end{enumerate}

\colbreak
\subsection{Memory Model}
Go's memory model defines when a read in one goroutine is guaranteed to observe a write from another.
\begin{enumerate}[\roman*.]
  \item Happens-Before: transitive visibility relation of Sequenced-Before $\cup$ Synchronizes-With 
  \item Within goroutine: observe program order (Sequenced-Before)
  \item Across goroutines: observed order may differ without synchronization
  \item Guarantee a read $r$ of variable $v$ observes write $w$ if:
  \begin{itemize}[leftmargin=*]\vspace{3pt}
    \item $w$ happens before $r$
    \item Other write to $v$ happens before $w$ or after $r$
  \end{itemize}
\end{enumerate}

\subsection{Concurrency Patterns}
Patterns:
\begin{enumerate}[\roman*.]
  \item Confinement:
    \begin{itemize}[leftmargin=*]\vspace{3pt}
      \item Ad-hoc: only one goroutine mutates shared data
      \item Lexical: restrict access to shared locations\\(e.g., expose only \lstinline|<-chan T|, slice of array)
    \end{itemize}
  \item Prevent goroutine leaks:
    \begin{itemize}[leftmargin=*]\vspace{3pt}
      \item Creator ensures termination when work done, unrecoverable error, or when told to stop working 
      \item Beware nil channels: read/write blocks forever
    \end{itemize}
  \item Error handling:
    \begin{itemize}[leftmargin=*]\vspace{3pt}
      \item Return result + error along the same channel
      \item Centralize decisions in a state-goroutine
    \end{itemize}
  \item Pipeline:
    \begin{itemize}[leftmargin=*]\vspace{3pt}
      \item Stages connected by channels; each stage is a group of goroutines running the same function
    \end{itemize}
  \item Fan-out / Fan-in:
    \begin{itemize}[leftmargin=*]\vspace{3pt}
      \item Fan-out: replicate a slow stage across goroutines (independent work; order not guaranteed)
      \item Fan-in: multiplex multiple input channels into one output channel
    \end{itemize}
\item Load balancing:
  \begin{itemize}[leftmargin=*]\vspace{3pt}
    \item Pool of workers reads from a shared work channel
    \item Each worker emits results; collect with fan-in
  \end{itemize}
\end{enumerate}
\vspace{-1em}
\colbreak
\section{Classical Synchronisation Problems}
\lstset{style=c}
\subsection{Producer-Consumer}
Processes share a bounded buffer of fixed size $K$, where producers add items until buffer full and consumers remove items when not empty.

Blocking Solution (init \lstinline|not_full = K, not_emp = 0|):\\
\begin{minipage}{0.45\columnwidth}
  \begin{lstlisting}
// Producer
x = produce();

wait(not_full);
wait(mutex);
buffer.add(x);
signal(mutex);
signal(not_emp);
  \end{lstlisting}
\end{minipage}
\begin{minipage}{0.5\columnwidth}
  \begin{lstlisting}
// Consumer
wait(not_emp);
wait(mutex);
x = buffer.get();
signal(mutex);
signal(not_full);

consume(item);
  \end{lstlisting}
\end{minipage}

\subsection{Sleeping Barber}
Barber sleeps until a customer wakes him. If barber is busy and chairs are free, the customer sits or will leave if there are no chairs.

\begin{minipage}{0.48\columnwidth}
  \begin{lstlisting}
// Customer
wait(mutex);
if (customers==n){
  signal(mutex);
  exit();
}
customers += 1;
signal(mutex);

signal(customer);
wait(barber);
getHairCut();
signal(custDone);
wait(barbDone);

wait(mutex);
customers -= 1;
signal(mutex);
  \end{lstlisting}
\end{minipage}
\begin{minipage}{0.48\columnwidth}
  \begin{lstlisting}
// Barber
while (1) {
  wait(customer);
  signal(barber);
  cutHair();
  wait(custDone);
  signal(barbDone);
}
  \end{lstlisting}
\end{minipage}


\colbreak
\subsection{Reader-Writer}
Processes share a critical region, where readers can read simultaneously but writers must have exclusive access.
\vspace{-1em}
\begin{minipage}{0.45\columnwidth}
  \begin{lstlisting}
// Writer
wait(empty);
write();
signal(empty);
  \end{lstlisting}
\end{minipage}
\begin{minipage}{0.5\columnwidth}
  \begin{lstlisting}
// Reader 
wait(mutex);
readers++;
if (readers == 1)
  wait(empty);
signal(mutex);

read();

wait(mutex);
readers--;
if (readers == 0)
  signal(empty);
signal(mutex);
  \end{lstlisting}
\end{minipage}
\begin{enumerate}[$-$]
  \item Writer Starvation: possible if readers keep entering and \lstinline|empty| is never signalled. Solutions:
    \begin{itemize}[leftmargin=*]\vspace{3pt}
      \item Queue: all pass through \lstinline|wait(queue)| 
      \item Writer-Priority: readers entering must \lstinline|wait(readTry)|; but first waiting writer blocks new readers with \lstinline|wait(readTry)| and does \lstinline|signal(readTry)| when done 
    \end{itemize}
\end{enumerate}

\subsection{Barrier}
\begin{lstlisting}
int i_am_last = 0;
wait(mutex);
count++;
if (count == N) {
  i_am_last = 1;
  signal(barrier); 
}
signal(mutex);

wait(barrier);
if (!i_am_last) 
  signal(barrier);
\end{lstlisting}
\colbreak
\subsection{Dining Philosophers}
$N$ philosophers around a circular table, with a single chopstick between. $2$ chopsticks are needed to eat.
\begin{minipage}{0.48\columnwidth}
  \begin{lstlisting}
// Philosopher
void p(int i) {
  while (1) {
    // think
    take(i);
    eat();
    put(i);
  }
}

void safe_to_eat (int i) {
  if (state[i] == HUNGRY && state[LEFT] != EATING && state[RIGHT] != EATING) {
    state[i] = EATING;
    signal(s[i]);
  }
}
  \end{lstlisting}
\end{minipage}
\begin{minipage}{0.45\columnwidth}
  \begin{lstlisting}
// Take Chopstick
void take(int i) {
  wait(mutex);
  state[i] = HUNGRY;
  safe_to_eat(i);
  signal(mutex);
  wait(s[i]);
}

// Put Chopstick
void put(int i) {
  wait(mutex);
  state[i] = THINKING;
  safe_to_eat(LEFT);
  safe_to_eat(RIGHT)
  signal(mutex);
}
  \end{lstlisting}
\end{minipage}
Limited Eater Solution, when at most $N-1$ philosophers eat concurrently, it's guaranteed at least one can eat:
\begin{lstlisting}
void philosopher(int i) {
  while (1) {
    // think 
    wait(seats);
    wait(chopstick[LEFT]);
    wait(chopstick[RIGHT]);
    eat();
    signal(chopstick[RIGHT]);
    signal(chopstick[LEFT]);
    signal(seats);
  }
}
\end{lstlisting}
\colbreak
\section{Rust}
\lstset{style=rust}

\subsection{Ownership and Borrowing}
Rust enforces memory-safety at compile-time without GC, preventing data races and enabling concurrency.
\begin{enumerate}[\roman*.]
  \item Each value has exactly one owner 
\end{enumerate}
\begin{enumerate}[$+$]\vspace{-1pt}
  \item Ownership prevents double-free; owner frees 
  \item Borrowing prevents use-after-free (UAF)
  \item Segfault-free
\end{enumerate}

Operations:
\begin{enumerate}[\roman*.]
  \item Move: \lstinline|T|
  \begin{itemize}[leftmargin=*]\vspace{3pt}
    \item Transfers ownership; old binding cannot be used
    \item Deep Copy: \lstinline|clone()| is explicit (expensive)
  \end{itemize}
  \item Shared Borrow: \lstinline|&T|
  \begin{itemize}[leftmargin=*]\vspace{3pt}
    \item Both bindings are read-only while shared
  \end{itemize}
  \item Mutable Borrow: \lstinline|&mut T|
  \begin{itemize}[leftmargin=*]\vspace{3pt}
    \item Old binding is read-only while shared 
    \item New binding is mutable while shared
  \end{itemize}
\end{enumerate}

\subsection{Threads}

Operations (\lstinline|std::thread|):
\begin{enumerate}[\roman*.]
  \item Spawn: \lstinline|let h = spawn(move| $\mid\mid$ \lstinline|{...})|
  \begin{itemize}[leftmargin=*]\vspace{3pt}
    \item \lstinline|move|: moves captured values into thread
  \end{itemize}
  \item Join: \lstinline|h.join()|
  \item Marker Traits:
    \begin{itemize}[leftmargin=*]\vspace{3pt}
    \item \lstinline|Copy|: safe to memcpy (i.e. \lstinline|u32, f32|, no String) 
    \item \lstinline|Send|: safe to transfer to another thread 
    \item \lstinline|Sync|: safe to share reference between threads;\\
      \lstinline|T| is \lstinline|Sync| $\iff$ \lstinline|&T| is \lstinline|Send|
  \end{itemize}
  \item Shared Ownership:
    \begin{itemize}[leftmargin=*]\vspace{3pt}
      \item \lstinline|Rc<T>|: within a thread
      \item \lstinline|Arc<T>|: across threads; has \lstinline|Send| trait
  \end{itemize}
\end{enumerate}



\colbreak
\subsection{Mutexes}

Operations (\lstinline|std::sync::Mutex|):
\begin{enumerate}[\roman*.]
  \item Construct: \lstinline|let m = Mutex::new(x)|
  \item Lock-Guard: \lstinline|let mut g Guard<T> = m.lock()|
  \begin{itemize}[leftmargin=*]\vspace{3pt}
    \item Guard controls access: \lstinline|*g += 1|
    \item When guard goes out of scope, mutex unlocks
  \end{itemize}
\end{enumerate}

\subsection{Atomics}
Atomics are guaranteed lock-free indivisible types.

Operations (\lstinline|std::sync::atomic|):
\begin{enumerate}[\roman*.]
  \item Construct: \lstinline|let a = AtomicUsize::new(v)|
  \begin{itemize}[leftmargin=*]\vspace{3pt}
    \item \lstinline|AtomicBool|, \lstinline|AtomicIsize|, \lstinline|AtomicI16|
  \end{itemize}
  \item Load: \lstinline|a.load(SeqCst)|
  \item Store: \lstinline|a.store(v, SeqCst)|
  \item Read-Modify-Write:\\\lstinline|a.fetch_add(1, SeqCst)|, \lstinline|a.swap(v, SeqCst)|
\end{enumerate}

\subsection{Multi-Producer, Single Consumer (MPSC)}
MPSC provides a FIFO communication queue.

Operations (\lstinline|std::sync::mpsc|):
\begin{enumerate}[\roman*.]
  \item Create: \lstinline|let (tx, rx) = channel()|
  \item Clone Sender: \lstinline|let tx2 = tx.clone()|
  \item Send: \lstinline|tx.send(v)|
  \item Receive: \lstinline|rx.recv()|
\end{enumerate}

\colbreak
\subsection{crossbeam Package}
Crossbeam extends beyond \lstinline|std|.

Operations:
\begin{enumerate}[\roman*.]
  \item Scoped Threads: \lstinline|crossbeam::scope(|\dots\lstinline|)|
  \begin{itemize}[leftmargin=*]\vspace{3pt}
    \item Threads within borrow only from the scope
    \item All spawned threads must end before scope exits
  \end{itemize}
  \item MPMC Channels: \lstinline|crossbeam::channel()|
  \item Bounded Channels: \lstinline|crossbeam::bounded(C)|
  \item Backoff: \lstinline|crossbeam::utils::Backoff::new()|
  \begin{itemize}[leftmargin=*]\vspace{3pt}
    \item Exponentially slow retry reduces contention, spin 
  \end{itemize}
\end{enumerate}

\begin{lstlisting}[basicstyle=\ttfamily\scriptsize]
let mut v = vec![1, 2, 3, 4];
thread::scope(|s| {
  let (a, b) = v.split_at_mut(2); // disjoint 
  s.spawn(|_| { for x in a { *x += 1; } });
  s.spawn(|_| { for x in b { *x += 1; } });
}).unwrap();
println!("{:?}", v); // safe: both returned
\end{lstlisting}

\subsection{rayon Package}
Rayon provides data-parallelism operations.

Operations (\lstinline|rayon::prelude::*|):
\begin{enumerate}[\roman*.]
  \item Iterators:
  \begin{itemize}[leftmargin=*]\vspace{3pt}
    \item Read-Only: \lstinline|xs.par_iter()|
    \item Mutable: \lstinline|xs.par_iter_mut()|
  \end{itemize}
  \item Combinators:
  \begin{itemize}[leftmargin=*]\vspace{3pt}
    \item \lstinline|map(...)|, \lstinline|filter(...)|, \lstinline|for_each(...)|
    \item Aggregation: \lstinline|reduce(...)|/ \lstinline|sum()| 
  \end{itemize}
  \item Sort: \lstinline|xs.par_sort()|/ \lstinline|xs.par_sort_unstable()|
\end{enumerate}

\begin{lstlisting}[basicstyle=\ttfamily\scriptsize]
let max = AtomicI64::new(MIN);
vec.par_iter().for_each(|n| {
  loop {
    let old = max.load(SeqCst);
    if *n <= old { break; }
    let r = max.compare_and_swap(old, *n, SeqCst);
    if r == old {
      // swapped 
      break;
    }
  }
})
\end{lstlisting}

\colbreak
\subsection{Non-blocking I/O}
Non-blocking I/O enables concurrency with one thread:
\begin{enumerate}[\roman*.]
  \item Blocking I/O: thread sleeps until data arrives
  \item \lstinline|epoll|: kernel reports ready file descriptors
  \item Event Loop: wait for ready fds $\rightarrow$ handle I/O
\end{enumerate}

\subsection{Futures}
Futures represent work that will produce a value later: 
\begin{enumerate}[\roman*.]
  \item \lstinline|Future|: trait driven by \lstinline|poll(Context)|:
  \begin{itemize}[leftmargin=*]\vspace{3pt}
    \item Runs until it can no longer make progress
    \item Returns \lstinline|Poll::Ready(T)| or \lstinline|Poll::Pending|
  \end{itemize}
  \item \lstinline|Context| provides \lstinline|wake()| (notify executor to repoll)
  \item Combinators: \lstinline|.then()|
\end{enumerate}

Executors are user-space schedulers that drive futures:
\begin{enumerate}[\roman*.]
  \item \lstinline|Pending| $\Rightarrow$ future arranges a \lstinline|wake()| when it can progress again
  \item If nothing can progress: sleep until a \lstinline|wake()|
  \item Threading with Tokio:
  \begin{itemize}[leftmargin=*]\vspace{3pt}
    \item Single-thread: one OS thread polls all futures
    \item Multi-thread: futures may run on multiple cores
    \item Shared data still needs synchronization
  \end{itemize}
\end{enumerate}

\subsection{Async/Await}
Async/Await are syntactic sugar for composing Futures.
\begin{enumerate}[\roman*.]
  \item \lstinline|async fn| returns a \lstinline|Future|
  \item \lstinline|.await| waits for a future (only inside \lstinline|async|)
\end{enumerate}
\begin{enumerate}[$+$]\vspace{-1pt}
  \item Zero-cost Abstraction: compiler lowers \lstinline|async fn| into a state machine implementing \lstinline|poll()|
  \item Lower memory use with stackless coroutines
  \item Lower scheduling overhead from context switches
  \item Scales to very high concurrency (esp. I/O bound)
\end{enumerate}
\begin{enumerate}[$-$]\vspace{-1pt}
  \item Block inside async sleeps the executor thread
  \item Cooperative: tasks yield only at \lstinline|.await|
  \item CPU-heavy work can starve other tasks 
  \item No recursion; future size fixed at compile time
\end{enumerate}
\vspace{-1em}
\colbreak
\section{Testing}
\lstset{style=c}
Bug Types:
\begin{enumerate}[\roman*.]
\item Unwanted Blocking: deadlock, livelock, IO
\item Race Conditions:
  \begin{itemize}[leftmargin=*]\vspace{3pt}
    \item Data races $\Rightarrow$ undefined behavior (unsynchronized concurrent access)
    \item Broken invariants, partial updates observed
    \item Dangling pointers / use-after-free across threads
    \item Memory corruption, double-free
    \item Lifetime issues (thread outlives data it accesses)
  \end{itemize}
\end{enumerate}

Techniques:
\begin{enumerate}[\roman*.]
\item Review the code:
\begin{itemize}[leftmargin=*]\vspace{3pt}
\item Identify shared data and how it is protected
\item Track held mutexes and find lock ordering issues
\item Check cross-thread ordering enforcement
\item Validate lifetimes\\(e.g. \lstinline|join()| should not skip on exceptions)
\end{itemize}
\item Testing:
\begin{itemize}[leftmargin=*]\vspace{3pt}
\item Run the smallest amount of code that can demonstrate the problem
\item Remove concurrency to confirm it is concurrency-related
\item Run on multicore and (when useful) single-core
\end{itemize}
\end{enumerate}

Tools:
\begin{enumerate}[\roman*.]
\item Valgrind memcheck (memory error; high overhead): \lstinline|valgrind --tool=memcheck <prog ...>|
\item Helgrind/DRD (thread error detectors):
\begin{itemize}[leftmargin=*]\vspace{3pt}
\item Detects misuses of pthreads, potential deadlocks (lock ordering), data races
\item Uses lock-order graphs, happens-before reasoning
\end{itemize}
\item Sanitizers (compile-instrumentation; mid overhead):
\begin{itemize}[leftmargin=*]\vspace{3pt}
\item ASan: \lstinline|-fsanitize=address| (addressability)
\item TSan: \lstinline|-fsanitize=thread| (data races, thread/lifetime)
\end{itemize}
\end{enumerate}

\subsection{Model Checking}

Check Formal Specifications:
\begin{enumerate}[\roman*.]
  \item Build a model in a special DSL / new language
  \item Check: constraints, unexpected behaviour, deadlock
  \item Why spend time:
  \begin{itemize}[leftmargin=*]\vspace{3pt}
    \item Check designs before costly implementation
    \item Prove properties for existing code
    \item Enable aggressive optimizations without breaking correctness
  \end{itemize}
\end{enumerate}
\begin{enumerate}[$+$]\vspace{-1pt}
  \item Rigorous, check all traces exhaustively 
  \item Produces trace violating requirement
\end{enumerate}
\begin{enumerate}[$-$]\vspace{-1pt}
  \item Tedious to define correct specification
\end{enumerate}

Approaches:
\begin{enumerate}[\roman*.]
  \item Write a formal specification and check it\\(with model checker/ proof assistant)
  \item Use specification to write minimal code:
  \item Add invariants as comments to assist checker:
\end{enumerate}

Model Checkers:
\begin{enumerate}[\roman*.]
  \item TLA+ (TLC): temporal properties; good for concurrent/distributed systems
  \begin{itemize}[leftmargin=*]\vspace{3pt}
    \item Write a TLA+ specification; use TLC to exhaustively explore the state space
    \item Check safety (invariants) and liveness (temporal)
    \item Reports errors with a counterexample trace 
    \item Model Structure:\\\lstinline|Init| (initial states) + \lstinline|Next| (state transitions)
    \item Uses fairness to rule out infinite ``stuttering'' (non-progress) executions
    \end{itemize}
  \item Coq: interactive proofs; generate code (OCaml/Haskell/Scheme)
  \item Alloy: relational logic; good for modeling structures
\end{enumerate}

Challenges of Distributed Systems:
\begin{enumerate}[\roman*.]
  \item Reliability: in case of failures 
  \item No global clock or ordering: need memory model 
  \item Consistency 
  \item Consensus
\end{enumerate}
%%%%%%%%%%%%%%%%%%%%%%%%%%%%%%%%%%%%%%%%%%%%%%%%%%%%%%
%                       End                          %
%%%%%%%%%%%%%%%%%%%%%%%%%%%%%%%%%%%%%%%%%%%%%%%%%%%%%%
\end{multicols*}
\end{document}

\documentclass[12pt, a4paper]{article}

\usepackage[utf8]{inputenc}
\usepackage[mathscr]{euscript}
\let\euscr\mathscr \let\mathscr\relax
\usepackage[scr]{rsfso}
\usepackage{amssymb,amsmath,amsthm,amsfonts}
\usepackage[shortlabels]{enumitem}
\usepackage{multicol,multirow}
\usepackage{lipsum}
\usepackage{balance}
\usepackage{calc}
\usepackage[colorlinks=true,citecolor=blue,linkcolor=blue]{hyperref}
\usepackage{import}
\usepackage{xifthen}
\usepackage{pdfpages}
\usepackage{transparent}
\usepackage{listings}

\newcommand{\incfig}[2][1.0]{
    \def\svgwidth{#1\columnwidth}
    \import{./figures/}{#2.pdf_tex}
}

\newlist{enumproof}{enumerate}{4}
\setlist[enumproof,1]{label=\arabic*., parsep=1em}
\setlist[enumproof,2]{label=\arabic{enumproofi}.\arabic*., parsep=1em}
\setlist[enumproof,3]{label=\arabic{enumproofi}.\arabic{enumproofii}.\arabic*., parsep=1em}
\setlist[enumproof,4]{label=\arabic{enumproofi}.\arabic{enumproofii}.\arabic{enumproofiii}.\arabic*., parsep=1em}

\renewcommand{\qedsymbol}{\ensuremath{\blacksquare}}

\lstdefinestyle{mystyle}{
  language=C, % Set the language to C
  commentstyle=\color{codegray}, % Color for comments
  keywordstyle=\color{orange}, % Color for basic keywords
  stringstyle=\color{mauve}, % Color for strings
  basicstyle={\ttfamily\footnotesize}, % Basic font style
  breakatwhitespace=false,         
  breaklines=true,                 
  captionpos=b,                    
  keepspaces=true,                 
  numbers=none,                    
  tabsize=2,
  morekeywords=[2]{\#include, \#define, \#ifdef, \#ifndef, \#endif, \#pragma, \#else, \#elif}, % Preprocessor directives
  keywordstyle=[2]\color{codegreen}, % Style for preprocessor directives
  morekeywords=[3]{int, char, float, double, void, struct, union, enum, const, volatile, static, extern, register, inline, restrict, _Bool, _Complex, _Imaginary, size_t, ssize_t, FILE}, % C standard types and common identifiers
  keywordstyle=[3]\color{identblue}, % Style for types and common identifiers
  morekeywords=[4]{printf, scanf, fopen, fclose, malloc, free, calloc, realloc, perror, strtok, strncpy, strcpy, strcmp, strlen}, % Standard library functions
  keywordstyle=[4]\color{cyan}, % Style for library functions
}

\usepackage{ifthen}
\usepackage[landscape]{geometry}
\usepackage[shortlabels]{enumitem}

\ifthenelse{\lengthtest { \paperwidth = 11in}}
    { \geometry{top=.5in,left=.5in,right=.5in,bottom=.5in} }
	{\ifthenelse{ \lengthtest{ \paperwidth = 297mm}}
		{\geometry{top=1cm,left=1cm,right=1cm,bottom=1cm} }
		{\geometry{top=1cm,left=1cm,right=1cm,bottom=1cm} }
	}

\pagestyle{empty}
\makeatletter
\renewcommand\thesection{\arabic{section}.}
\renewcommand{\section}{\@startsection{section}{1}{0mm}%
                                {-1ex plus -.5ex minus -.2ex}%
                                {0.05ex}%x
                                {\normalfont\normalsize\bfseries}}
\renewcommand{\subsection}{\@startsection{subsection}{2}{0mm}%
                                {-1ex plus -.5ex minus -.2ex}%
                                {0.05ex}%
                                {\normalfont\small\bfseries}}
\renewcommand{\subsubsection}{\@startsection{subsubsection}{3}{0mm}%
                                {-1ex plus -.5ex minus -.2ex}%
                                {0.05ex}%
                                {\normalfont\footnotesize\bfseries}}
\newcommand{\colbreak}{\vfill\null\columnbreak}
\makeatother
\setcounter{secnumdepth}{1}
\setlength{\parindent}{0pt}
\setlength{\parskip}{0.7em}

\setlist[itemize]{itemsep=0.6ex, topsep=-2pt, partopsep=0pt, parsep=0pt}
\setlist[enumerate]{itemsep=0.6ex, topsep=-2pt, partopsep=0pt, parsep=0pt}

% Things Lie
\newcommand{\kb}{\mathfrak b}
\newcommand{\kg}{\mathfrak g}
\newcommand{\kh}{\mathfrak h}
\newcommand{\kn}{\mathfrak n}
\newcommand{\ku}{\mathfrak u}
\newcommand{\kz}{\mathfrak z}
\DeclareMathOperator{\Ext}{Ext} % Ext functor
\DeclareMathOperator{\Tor}{Tor} % Tor functor
\newcommand{\gl}{\opname{\mathfrak{gl}}} % frak gl group
\renewcommand{\sl}{\opname{\mathfrak{sl}}} % frak sl group chktex 6

% More script letters etc.
\newcommand{\SA}{\mathcal A}
\newcommand{\SB}{\mathcal B}
\newcommand{\SC}{\mathcal C}
\newcommand{\SF}{\mathcal F}
\newcommand{\SG}{\mathcal G}
\newcommand{\SH}{\mathcal H}
\newcommand{\OO}{\mathcal O}

\newcommand{\SCA}{\mathscr A}
\newcommand{\SCB}{\mathscr B}
\newcommand{\SCC}{\mathscr C}
\newcommand{\SCD}{\mathscr D}
\newcommand{\SCE}{\mathscr E}
\newcommand{\SCF}{\mathscr F}
\newcommand{\SCG}{\mathscr G}
\newcommand{\SCH}{\mathscr H}

% Mathfrak primes
\newcommand{\km}{\mathfrak m}
\newcommand{\kp}{\mathfrak p}
\newcommand{\kq}{\mathfrak q}

% number sets
\newcommand{\RR}[1][]{\ensuremath{\ifstrempty{#1}{\mathbb{R}}{\mathbb{R}^{#1}}}}
\newcommand{\NN}[1][]{\ensuremath{\ifstrempty{#1}{\mathbb{N}}{\mathbb{N}^{#1}}}}
\newcommand{\ZZ}[1][]{\ensuremath{\ifstrempty{#1}{\mathbb{Z}}{\mathbb{Z}^{#1}}}}
\newcommand{\QQ}[1][]{\ensuremath{\ifstrempty{#1}{\mathbb{Q}}{\mathbb{Q}^{#1}}}}
\newcommand{\CC}[1][]{\ensuremath{\ifstrempty{#1}{\mathbb{C}}{\mathbb{C}^{#1}}}}
\newcommand{\PP}[1][]{\ensuremath{\ifstrempty{#1}{\mathbb{P}}{\mathbb{P}^{#1}}}}
\newcommand{\HH}[1][]{\ensuremath{\ifstrempty{#1}{\mathbb{H}}{\mathbb{H}^{#1}}}}
\newcommand{\FF}[1][]{\ensuremath{\ifstrempty{#1}{\mathbb{F}}{\mathbb{F}^{#1}}}}
% expected value
\newcommand{\EE}{\ensuremath{\mathbb{E}}}
\newcommand{\charin}{\text{ char }}
\DeclareMathOperator{\sign}{sign}
\DeclareMathOperator{\Aut}{Aut}
\DeclareMathOperator{\Inn}{Inn}
\DeclareMathOperator{\Syl}{Syl}
\DeclareMathOperator{\Gal}{Gal}
\DeclareMathOperator{\GL}{GL} % General linear group
\DeclareMathOperator{\SL}{SL} % Special linear group

%---------------------------------------
% BlackBoard Math Fonts :-
%---------------------------------------

%Captital Letters
\newcommand{\bbA}{\mathbb{A}}	\newcommand{\bbB}{\mathbb{B}}
\newcommand{\bbC}{\mathbb{C}}	\newcommand{\bbD}{\mathbb{D}}
\newcommand{\bbE}{\mathbb{E}}	\newcommand{\bbF}{\mathbb{F}}
\newcommand{\bbG}{\mathbb{G}}	\newcommand{\bbH}{\mathbb{H}}
\newcommand{\bbI}{\mathbb{I}}	\newcommand{\bbJ}{\mathbb{J}}
\newcommand{\bbK}{\mathbb{K}}	\newcommand{\bbL}{\mathbb{L}}
\newcommand{\bbM}{\mathbb{M}}	\newcommand{\bbN}{\mathbb{N}}
\newcommand{\bbO}{\mathbb{O}}	\newcommand{\bbP}{\mathbb{P}}
\newcommand{\bbQ}{\mathbb{Q}}	\newcommand{\bbR}{\mathbb{R}}
\newcommand{\bbS}{\mathbb{S}}	\newcommand{\bbT}{\mathbb{T}}
\newcommand{\bbU}{\mathbb{U}}	\newcommand{\bbV}{\mathbb{V}}
\newcommand{\bbW}{\mathbb{W}}	\newcommand{\bbX}{\mathbb{X}}
\newcommand{\bbY}{\mathbb{Y}}	\newcommand{\bbZ}{\mathbb{Z}}

%---------------------------------------
% MathCal Fonts :-
%---------------------------------------

%Captital Letters
\newcommand{\mcA}{\mathcal{A}}	\newcommand{\mcB}{\mathcal{B}}
\newcommand{\mcC}{\mathcal{C}}	\newcommand{\mcD}{\mathcal{D}}
\newcommand{\mcE}{\mathcal{E}}	\newcommand{\mcF}{\mathcal{F}}
\newcommand{\mcG}{\mathcal{G}}	\newcommand{\mcH}{\mathcal{H}}
\newcommand{\mcI}{\mathcal{I}}	\newcommand{\mcJ}{\mathcal{J}}
\newcommand{\mcK}{\mathcal{K}}	\newcommand{\mcL}{\mathcal{L}}
\newcommand{\mcM}{\mathcal{M}}	\newcommand{\mcN}{\mathcal{N}}
\newcommand{\mcO}{\mathcal{O}}	\newcommand{\mcP}{\mathcal{P}}
\newcommand{\mcQ}{\mathcal{Q}}	\newcommand{\mcR}{\mathcal{R}}
\newcommand{\mcS}{\mathcal{S}}	\newcommand{\mcT}{\mathcal{T}}
\newcommand{\mcU}{\mathcal{U}}	\newcommand{\mcV}{\mathcal{V}}
\newcommand{\mcW}{\mathcal{W}}	\newcommand{\mcX}{\mathcal{X}}
\newcommand{\mcY}{\mathcal{Y}}	\newcommand{\mcZ}{\mathcal{Z}}

%---------------------------------------
% Bold Math Fonts :-
%---------------------------------------

%Captital Letters
\newcommand{\bmA}{\boldsymbol{A}}	\newcommand{\bmB}{\boldsymbol{B}}
\newcommand{\bmC}{\boldsymbol{C}}	\newcommand{\bmD}{\boldsymbol{D}}
\newcommand{\bmE}{\boldsymbol{E}}	\newcommand{\bmF}{\boldsymbol{F}}
\newcommand{\bmG}{\boldsymbol{G}}	\newcommand{\bmH}{\boldsymbol{H}}
\newcommand{\bmI}{\boldsymbol{I}}	\newcommand{\bmJ}{\boldsymbol{J}}
\newcommand{\bmK}{\boldsymbol{K}}	\newcommand{\bmL}{\boldsymbol{L}}
\newcommand{\bmM}{\boldsymbol{M}}	\newcommand{\bmN}{\boldsymbol{N}}
\newcommand{\bmO}{\boldsymbol{O}}	\newcommand{\bmP}{\boldsymbol{P}}
\newcommand{\bmQ}{\boldsymbol{Q}}	\newcommand{\bmR}{\boldsymbol{R}}
\newcommand{\bmS}{\boldsymbol{S}}	\newcommand{\bmT}{\boldsymbol{T}}
\newcommand{\bmU}{\boldsymbol{U}}	\newcommand{\bmV}{\boldsymbol{V}}
\newcommand{\bmW}{\boldsymbol{W}}	\newcommand{\bmX}{\boldsymbol{X}}
\newcommand{\bmY}{\boldsymbol{Y}}	\newcommand{\bmZ}{\boldsymbol{Z}}
%Small Letters
\newcommand{\bma}{\boldsymbol{a}}	\newcommand{\bmb}{\boldsymbol{b}}
\newcommand{\bmc}{\boldsymbol{c}}	\newcommand{\bmd}{\boldsymbol{d}}
\newcommand{\bme}{\boldsymbol{e}}	\newcommand{\bmf}{\boldsymbol{f}}
\newcommand{\bmg}{\boldsymbol{g}}	\newcommand{\bmh}{\boldsymbol{h}}
\newcommand{\bmi}{\boldsymbol{i}}	\newcommand{\bmj}{\boldsymbol{j}}
\newcommand{\bmk}{\boldsymbol{k}}	\newcommand{\bml}{\boldsymbol{l}}
\newcommand{\bmm}{\boldsymbol{m}}	\newcommand{\bmn}{\boldsymbol{n}}
\newcommand{\bmo}{\boldsymbol{o}}	\newcommand{\bmp}{\boldsymbol{p}}
\newcommand{\bmq}{\boldsymbol{q}}	\newcommand{\bmr}{\boldsymbol{r}}
\newcommand{\bms}{\boldsymbol{s}}	\newcommand{\bmt}{\boldsymbol{t}}
\newcommand{\bmu}{\boldsymbol{u}}	\newcommand{\bmv}{\boldsymbol{v}}
\newcommand{\bmw}{\boldsymbol{w}}	\newcommand{\bmx}{\boldsymbol{x}}
\newcommand{\bmy}{\boldsymbol{y}}	\newcommand{\bmz}{\boldsymbol{z}}

%---------------------------------------
% Scr Math Fonts :-
%---------------------------------------

\newcommand{\sA}{{\mathscr{A}}}   \newcommand{\sB}{{\mathscr{B}}}
\newcommand{\sC}{{\mathscr{C}}}   \newcommand{\sD}{{\mathscr{D}}}
\newcommand{\sE}{{\mathscr{E}}}   \newcommand{\sF}{{\mathscr{F}}}
\newcommand{\sG}{{\mathscr{G}}}   \newcommand{\sH}{{\mathscr{H}}}
\newcommand{\sI}{{\mathscr{I}}}   \newcommand{\sJ}{{\mathscr{J}}}
\newcommand{\sK}{{\mathscr{K}}}   \newcommand{\sL}{{\mathscr{L}}}
\newcommand{\sM}{{\mathscr{M}}}   \newcommand{\sN}{{\mathscr{N}}}
\newcommand{\sO}{{\mathscr{O}}}   \newcommand{\sP}{{\mathscr{P}}}
\newcommand{\sQ}{{\mathscr{Q}}}   \newcommand{\sR}{{\mathscr{R}}}
\newcommand{\sS}{{\mathscr{S}}}   \newcommand{\sT}{{\mathscr{T}}}
\newcommand{\sU}{{\mathscr{U}}}   \newcommand{\sV}{{\mathscr{V}}}
\newcommand{\sW}{{\mathscr{W}}}   \newcommand{\sX}{{\mathscr{X}}}
\newcommand{\sY}{{\mathscr{Y}}}   \newcommand{\sZ}{{\mathscr{Z}}}


%---------------------------------------
% Math Fraktur Font
%---------------------------------------

%Captital Letters
\newcommand{\mfA}{\mathfrak{A}}	\newcommand{\mfB}{\mathfrak{B}}
\newcommand{\mfC}{\mathfrak{C}}	\newcommand{\mfD}{\mathfrak{D}}
\newcommand{\mfE}{\mathfrak{E}}	\newcommand{\mfF}{\mathfrak{F}}
\newcommand{\mfG}{\mathfrak{G}}	\newcommand{\mfH}{\mathfrak{H}}
\newcommand{\mfI}{\mathfrak{I}}	\newcommand{\mfJ}{\mathfrak{J}}
\newcommand{\mfK}{\mathfrak{K}}	\newcommand{\mfL}{\mathfrak{L}}
\newcommand{\mfM}{\mathfrak{M}}	\newcommand{\mfN}{\mathfrak{N}}
\newcommand{\mfO}{\mathfrak{O}}	\newcommand{\mfP}{\mathfrak{P}}
\newcommand{\mfQ}{\mathfrak{Q}}	\newcommand{\mfR}{\mathfrak{R}}
\newcommand{\mfS}{\mathfrak{S}}	\newcommand{\mfT}{\mathfrak{T}}
\newcommand{\mfU}{\mathfrak{U}}	\newcommand{\mfV}{\mathfrak{V}}
\newcommand{\mfW}{\mathfrak{W}}	\newcommand{\mfX}{\mathfrak{X}}
\newcommand{\mfY}{\mathfrak{Y}}	\newcommand{\mfZ}{\mathfrak{Z}}
%Small Letters
\newcommand{\mfa}{\mathfrak{a}}	\newcommand{\mfb}{\mathfrak{b}}
\newcommand{\mfc}{\mathfrak{c}}	\newcommand{\mfd}{\mathfrak{d}}
\newcommand{\mfe}{\mathfrak{e}}	\newcommand{\mff}{\mathfrak{f}}
\newcommand{\mfg}{\mathfrak{g}}	\newcommand{\mfh}{\mathfrak{h}}
\newcommand{\mfi}{\mathfrak{i}}	\newcommand{\mfj}{\mathfrak{j}}
\newcommand{\mfk}{\mathfrak{k}}	\newcommand{\mfl}{\mathfrak{l}}
\newcommand{\mfm}{\mathfrak{m}}	\newcommand{\mfn}{\mathfrak{n}}
\newcommand{\mfo}{\mathfrak{o}}	\newcommand{\mfp}{\mathfrak{p}}
\newcommand{\mfq}{\mathfrak{q}}	\newcommand{\mfr}{\mathfrak{r}}
\newcommand{\mfs}{\mathfrak{s}}	\newcommand{\mft}{\mathfrak{t}}
\newcommand{\mfu}{\mathfrak{u}}	\newcommand{\mfv}{\mathfrak{v}}
\newcommand{\mfw}{\mathfrak{w}}	\newcommand{\mfx}{\mathfrak{x}}
\newcommand{\mfy}{\mathfrak{y}}	\newcommand{\mfz}{\mathfrak{z}}


\newcommand{\mytitle}{CS3230 Design and Analysis of Algorithms}
\newcommand{\myauthor}{github/omgeta}
\newcommand{\mydate}{AY 25/26 Sem 1}

\begin{document}
\raggedright
\footnotesize
\begin{multicols*}{3}
\setlength{\premulticols}{1pt}
\setlength{\postmulticols}{1pt}
\setlength{\multicolsep}{1pt}
\setlength{\columnsep}{2pt}

{\normalsize{\textbf{\mytitle}}} \\
{\footnotesize{\mydate\hspace{2pt}\textemdash\hspace{2pt}\myauthor}}
\vspace{-0.5em}
%%%%%%%%%%%%%%%%%%%%%%%%%%%%%%%%%%%%%%%%%%%%%%%%%%%%%%
%                      Begin                         %
%%%%%%%%%%%%%%%%%%%%%%%%%%%%%%%%%%%%%%%%%%%%%%%%%%%%%%
\section{Algorithms}

Algorithms are sequences of instructions to solve problems.

Runtime analysis for $T(n)$ is usually done for worst-case (maximum time for any input) or average-case (expected time over all inputs)


\subsection{Correctness Proofs}

Iterative Algorithms:
\begin{enumerate}[\roman*.]
  \item Loop Invariant: define $I$ for every call or loop.
  \item Initialization: show $I$ holds before first iteration.
  \item Maintenance: assuming $I$ true at start, show $I$ holds true at start of next iteration.
  \item Termination: show when the loop exits, $I$ together with the exit condition implies correctness.
\end{enumerate}

Recursive Algorithms:
\begin{enumerate}[\roman*.]
  \item Base Case: show algorithm is correct for base cases
  \item Inductive Step: assuming correctness for input smaller than $n$, show algorithm is correct for any input of size $n$
\end{enumerate}

\subsection*{Mathematical Properties}
Logarithm Rules:
\begin{enumerate}[\roman*.]
  \item $a = b^{\log_ba}$\hfill(Inverse)
  \item $\log(ab) = \log a + \log b$\hfill(Product) 
  \item $\log(\frac{a}{b}) = \log a - \log b,\quad\log(\frac{1}{a}) = -\log a$\hfill(Quotient)
  \item $\log_ba = \frac{\log_ca}{\log_cb},\quad\log_ba = \frac{1}{\log_ab}$\hfill(Change of Base)
  \item $\log(a^k) = k\log a$\hfill(Power)
\end{enumerate}

Exponential Rules:
\begin{enumerate}[\roman*.]
  \item $a^m a^n = a^{m+n}$\hfill(Product) 
  \item $\frac{a^m}{a^n} = a^{m-n}$\hfill(Quotient)
  \item $(a^m)^n = a^{mn} = (a^n)^m$\hfill(Power)
\end{enumerate}

\colbreak
\section{Asymptotic Analysis}

Asymptotic bounds for functions $f(n), g(n)$:
\begin{enumerate}[\roman*.]
  \item $f(n) \in O(g(n)) \iff f(n) \leq cg(n)$,\\ where $\exists c, n_0, \forall n \geq n_0$\hfill(Upper)
  \item $f(n) \in \Omega(g(n)) \iff f(n) \geq cg(n)$,\\ where $\exists c, n_0, \forall n \geq n_0$\hfill(Lower)
  \item $f(n) \in \Theta(g(n)) \iff c_1g(n) \leq f(n) \leq c_2g(n)$,\\ where $\exists c_1, c_2, n_0, \forall n \geq n_0$\hfill(Tight)
  \item $f(n) \in o(g(n)) \iff f(n) < cg(n)$,\\ where $\forall c > 0, \exists n_0, \forall n \geq n_0$\hfill(Strict Upper)
  \item $f(n) \in \omega(g(n)) \iff f(n) > cg(n)$,\\ where $\forall c > 0, \exists n_0, \forall n \geq n_0$\hfill(Strict Lower)
\end{enumerate}

Limits assuming $f(n), g(n) > 0$:
\begin{enumerate}[\roman*.]
  \item $\displaystyle\lim_{n\rightarrow\infty}$$\frac{f(n)}{g(n)} < \infty \implies f(n) \in O(g(n))$
  \item $\displaystyle\lim_{n\rightarrow\infty}$$\frac{f(n)}{g(n)} > 0 \implies f(n) \in \Omega(g(n))$
  \item $0 < \displaystyle\lim_{n\rightarrow\infty}$$\frac{f(n)}{g(n)} < \infty \implies f(n) \in \Theta(g(n))$
  \item $\displaystyle\lim_{n\rightarrow\infty}$$\frac{f(n)}{g(n)} = 0 \implies f(n) \in o(g(n))$
  \item $\displaystyle\lim_{n\rightarrow\infty}$$\frac{f(n)}{g(n)} = \infty \implies f(n) \in \omega(g(n))$
\end{enumerate}

Properties:
\begin{enumerate}[\roman*.]
  \item Transitivity - for $O, \Omega, \Theta, o, \omega$:
    \begin{itemize}[leftmargin=*]\vspace{2pt }
      \item $f(n) = O(g(n)) \land g(n) = O(h(n))$\\$\implies f(n) = O(h(n))$
    \end{itemize}
  \item Reflexivity - for $O, \Omega, \Theta$:
    \begin{itemize}[leftmargin=*]\vspace{2pt}
      \item $f(n) = O(f(n))$
    \end{itemize}
  \item Symmetry - for $\Theta$:
    \begin{itemize}[leftmargin=*]\vspace{2pt}
      \item $f(n) = \Theta(g(n)) \iff g(n) = \Theta(f(n))$
    \end{itemize}
  \item Complementarity - for $O, \Omega, o, \omega$:
    \begin{itemize}[leftmargin=*]\vspace{2pt}
      \item $f(n) = O(g(n)) \iff g(n) = \Omega(f(n))$
      \item $f(n) = o(g(n)) \iff g(n) = \omega(f(n))$
    \end{itemize}
\end{enumerate}

Order of growth is:\\
$O(1) < O(\frac{1}{k}^n) < O(\log\log n) < O(\log n) < O(\log^k n) < O(n^{\frac{1}{k}}) < O(n) < O(n\log n) < O(n^k) < O(k^n) < O(n!)$


\colbreak

\section{Divide \& Conquer}

Divide \& Conquer involves:
\begin{enumerate}[\roman*.]
  \item Divide: split problem into smaller subproblems
  \item Conquer: solve subproblems recursively
  \item Combine: merge subresults to form a total solution
\end{enumerate}

\subsection{Problems}

MergeSort divides the array into halves, recursively sorts each half, and merges the sorted halves.
\begin{enumerate}[\roman*.]
  \item Locally sorted prefix in power of 2
  \item $T(n) = 2T(n /2) + \Theta(n) = \Theta(n\log n)$
  \item Out-of-place and stable
\end{enumerate}

Exponentiation by squaring computes $x^n$.\\If $n$ even: compute $(x^{n/2})^2$; if $n$ odd: return $x \cdot (x^{\lfloor n/2\rfloor})^2$. 
\begin{enumerate}[\roman*.]
  \item $T(n)=T(\floor{n/2})+\Theta(1)=\Theta(\log n)$.
\end{enumerate}

Fibonacci finds $F(n)$ uses $F(2k)=F(k)\big(2F(k{+}1)-F(k)\big)$ and
$F(2k{+}1)=F(k{+}1)^2+F(k)^2$:
\begin{enumerate}[\roman*.]
  \item $T(n)=T(\floor{n/2})+\Theta(1)=\Theta(\log n)$
  \item Naïve: $T(n)=T(n-1)+T(n-2)+\Theta(1)=\Theta(\varphi^{\,n})$
\end{enumerate}

Matrix Multiplication partitions input $A, B$ into quadrants of size $n /2$ aand recombine via $8$ submatrix products and $4$ additions into $C$:
\begin{enumerate}[\roman*.]
  \item $T(n)=8T(n/2)+\Theta(n^{2})=\Theta(n^{3})$
  \item Pad non-square input matrices with zeroes
\end{enumerate}

Strassen’s Algorithm computes $7$ block products of input $A,B$ and recombines via linear combinations to form $C$:
\begin{enumerate}[\roman*.]
  \item $T(n)=7T(n/2)+\Theta(n^{2})=\Theta\!\big(n^{\log 7}\big)\approx \Theta(n^{2.807})$.
\end{enumerate}
\colbreak

\section{Recurrences}

Common recurrences:
\begin{enumerate}[\roman*.]
  \item $T(n) = T(n/2) + O(1) = O(\log n)$
  \item $T(n) = T(n/2) + O(n) = O(n)$
  \item $T(n) = 2T(n/2) + O(1) = O(n)$
  \item $T(n) = 2T(n/2) + O(n) = O(n\log n)$
  \item $T(n) = T(n - 1) + O(1) = O(n)$
  \item $T(n) = T(n - 1) + O(n) = O(n^2)$
  \item $T(n) = 2T(n - 1) + O(1) = O(2^n)$
\end{enumerate}

\subsection{Telescoping Method}

Given recurrence $T(n) = aT(\frac{n}{b}) + f(n)$, divide expression to get $\frac{T(n)}{g(n)} = \frac{T(n /b)}{g(n /b)} + \frac{f(n)}{g(n)}$
\begin{enumerate}[\roman*.]
  \item Cancel out common terms to solve for $T(n)$
\end{enumerate}

\subsection{Recursion Tree}

Given recurrence, draw recurrence tree:
\begin{enumerate}[\roman*.]
  \item Total $=$ depth $\times$ work done per level
\end{enumerate}

\incimg{recursiontree}

\colbreak

\subsection{Master Theorem}

Dividing function $T(n) = aT(n/b) + f(n)$ where $a>1, b>1$ has cases:
\begin{enumerate}[\roman*.]
  \item $f(n) \in O(n^{d-\varepsilon}) \implies T(n)=\Theta(n^{\log_ba})$
  \item $f(n) \in \Theta(n^d\log^p n)$ \\
    $\hspace{3em}\land\hspace{0.25em}p>-1 \implies T(n) = \Theta(n^k\log^{p+1}n)$\\
    \hspace{3em}$\land\hspace{0.25em}p=-1 \implies T(n) = \Theta(n^k\log\log n)$\\
    \hspace{3em}$\land\hspace{0.25em}p<-1 \implies T(n) = \Theta(n^k)$
  \item $f(n) \in \Omega(n^{d+\varepsilon}) \land\hspace{0.25em}af(n /b) \leq cf(n)$, for $c < 1$\\$\hspace{7em}\implies T(n)=\Theta(f(n))$
\end{enumerate}

Decreasing function $T(n) = aT(n-b) + f(n)$ where $a>0, b>0$ has cases:
\begin{enumerate}[\roman*.]
  \item $a < 1 \implies T(n)=O(f(n))$
  \item $a = 1 \implies T(n) = O(n\cdot f(n))$
  \item $a > 1 \implies T(n) = O(a^{n/b}\cdot f(n))$
\end{enumerate}



\subsection{Substitution Method}

Guess $T(n) = O(f(n))$ and verify by induction:
\begin{enumerate}[\roman*.]
  \item Choose values of $c > 0, n_0$ from recurrence
  \item Base case: verify $T(n_0) \leq cf(n)$
  \item Inductive step: assuming $T(k) \leq cf(k)$ for $n > k \geq n_0$, prove $T(n) \leq cf(n)$ by subbing $T(k)$
\end{enumerate}

\subsubsection{Example}

Prove $T(n) = 4T(n /2) + n \in O(n^2)$:\\
\begin{enumproof}
\item Induction hypothesis: $T(n) \leq (c+1)n^2-n$
\item Base case: If $n = 1, T(n) = c \leq (c+1)n^2-n$
\item Inductive step: 
  \begin{enumproof}
  \item By strong induction, assume $T(k) \leq (c+1)k^2-k$ for all $n > k \geq 1$
  \item $T(n) = 4T(n /2) + n$
  \item $\quad\quad \leq 4(c+1)(n /2)^2 - 4(n /2) + n$
  \item $\quad\quad = (c+1)n^2-n$
  \end{enumproof}
\item $\therefore T(n) \in O(n^2)$
\end{enumproof}

\colbreak
\section{Probabilistic Analysis}
Average-case runtime $A(n)$ is the expected running time over the distribution of possible inputs (uniformly over $n!$ permutations for a uniform random permutation).

QuickSort rearranges the array by $\leq$ and $>$ a chosen pivot in $\Theta(n)$, then recurses on both partitions.
\begin{enumerate}[\roman*.]
  \item Elements before pivot are $\leq$, after pivot are $>$
  \item $T(n) = T(j-1)+T(n-j)+\Theta(n)$,\\ where pivot is $j^{th}$ smallest element.
  \item Worst: $T(n) = T(0) + T(n-1) + cn = \Theta(n^2)$,\\ when $j=1 \lor j=n$ 
  \item Average: $A(n) = \frac{1}{n} \sum^n_{j=1}[A(j-1)+A(n-j)+cn]$\\$= cn + \frac{2}{n}\sum^{n-1}_{j=0}A(j) = O(n\log n)$ (by telescoping)
  \item In-place but unstable
\end{enumerate}

\section{Randomized Algorithms} 
Randomized algorithms are dependent on random bits:
\begin{enumerate}[label=\roman*.]
  \item Las Vegas: always correct, running time is random
  \item Monte Carlo: may be wrong, running time is finite
\end{enumerate}

\subsection{Tools}
Union Bound can upper bound probability that bad event $\varepsilon = \varepsilon_1 \cup \dots \cup \varepsilon_n$ occurs: $\Pr[\varepsilon] \leq \Pr[\varepsilon_1] + \cdots + \Pr[\varepsilon_n]$
\begin{enumerate}[\roman*.]
  \item $\therefore \Pr[\varepsilon_i] \leq \frac{f}{n}$, $\forall i \in [n] \implies \Pr[\varepsilon] \leq f$ 
\end{enumerate}

Markov Inequality can turn a Las Vegas into Monte Carlo: $X \geq 0 \land a > 0 \Rightarrow \Pr[X\geq a \bbE[X]] \leq \frac{1}{a}$

Indicator Random Variable $\mathbf{1}_\varepsilon$ is a binary RV for event $\varepsilon$:
\begin{enumerate}[\roman*.]
  \item $\bbE[\mathbf{1}_\varepsilon] = \Pr(\varepsilon)$
\end{enumerate}

Linearity of Expectation: $\bbE[A+B] = \bbE[A] + \bbE[B]$

\colbreak
\subsection{Problems} %TODO: add analysis

Randomized QuickSort chooses a random pivot:
\begin{enumerate}[\roman*.]
  \item Average: $A(n) = \bbE[T(n)] = O(n\log n)$ (LoE)
\end{enumerate}

Approximate Median is a pivot selection algorithm. Pick a random pivot with probability $ \geq \frac{1}{2}$ to be an approximate median of rank $[\frac{n}{4}, \frac{3n}{4}]$, repeating $k$ times for boosting.
\begin{enumerate}[\roman*.]
  \item $A(n) = O(n)$ (QuickSelect)
  \item Error Rate: $2^{-k}$, less than $\frac{1}{n^2}$ if $k=1+10\log n$
\end{enumerate}


Freivalds' Algorithm verifies if matrices $AB = C$ by picking random bit vector $\vec{r}$ and check if $AB \vec{r} - C \vec{r} == \vec{0}$, repeating $k$ times with accepting if all successes:
\begin{enumerate}[\roman*.]
  \item $A(n) = O(kn^2)$ (3 matrix-vector multiplications)
  \item Error Rate: $2^{-k}$ when $AB \neq C$
\end{enumerate}
\begin{lstlisting}
FREIVALDS(A, B, C, k)    
  n = rows(A)               
  for i = 1 to k
    r = RANDOM-01-VECTOR(n) 
    ABr = DOT(A, DOT(B, r))
    Cr = DOT(C, r)
    if ABr != Cr then
      return FALSE   
  return TRUE
\end{lstlisting}

Balls-and-Bins deals with placing $m$ balls into $n$ bins:
\begin{enumerate}[\roman*.]
  \item For a fixed bin, $\Pr[\text{empty}]=(1-\tfrac1n)^m \le e^{-m/n}$\\(using $1+x\le e^x$)
  \item $\Pr[\exists\ \text{empty bin}] \le n\,e^{-m/n}$ (union bound) 
  \item Taking $m \ge 2n\ln n$ makes $\Pr[\exists\ \text{empty bin}] \leq \frac{1}{n}$\\(so all bins nonempty with probability $\ge 1-\tfrac1n$).
\end{enumerate}
\vspace{-1em}
\colbreak
\section{Dynamic Programming (DP)} % TODO: add optimal substructure proofs, and code examples

Dynamic Programming involves:
\begin{enumerate}[\roman*.]
  \item Divide: split problem into smaller overlapping subproblems
  \item Conquer: solve subproblems recursively with memoization/bottom-up approach
  \item Combine: merge subresults to form a total solution
  \item Optimal Substructure: optimal solution can be constructed from optimal solutions of subproblems
\end{enumerate}

Cut-and-paste Proof: If "optimal" solution has suboptimal part, replacing it with an optimal subsolution strictly improves the objective $\rightarrow$ contradiction, justifying DP. 

\subsection{Problems}

Longest Common Subsequence (LCS) of $A[:i]$ and $B[:j]$:
\begin{enumerate}[\roman*.]
  \item $LCS(i,j)=\begin{cases}
        0, &i=0\lor j=0\\
        LCS(i-1, j-1)+1,&A[i]=B[j]\\
        \max\{LCS(i-1, j), LCS(i, j-1)\},&A[i]\neq B[j]
      \end{cases}$
  \item $T(n) = \Theta(nm)$
  \item Longest Palindromic Subsequence (LPS) in $A[i:j]$ is just LCS of $A$ and $B = reversed(A)$
\end{enumerate}
\begin{lstlisting}
LCS(A, B)
 m = |A|
 n = |B|
 alloc 2D array L[0..m][0..n] = 0
 for i = 1 to m
   for j = 1 to n
     if A[i] = B[j] then
      L[i][j] = L[i-1][j-1] + 1
     else
      L[i][j] = max(L[i-1][j],L[i][j-1])
 return L[m][n]
\end{lstlisting}

\colbreak
Knapsack finds maximum value $v$ achievable given $items[:i]$ with ($v_i, w_i$) and maximum weight $W$:
\begin{enumerate}[\roman*.]
  \item $dp(i, j)=\begin{cases}
        0, &i=0\lor j=0\\
        \max\left\{\scriptstyle{dp(i-1, j), dp(i-1, j-w_i) + v_i} \right\},& w_i \leq j\\
        dp(i-1, j), &\text{otherwise}
      \end{cases}$
  \item $T(n) = \Theta(nW)$
  \item If item counts are unbounded, loop through items $dp(j)=\max_{i: w_i\le j}\{\,dp(j-w_i)+v_i\,\}$ 
\end{enumerate}
\begin{lstlisting}
KNAPSACK-01(v, w, W)
 n = |v|
 alloc table DP[0..n][0..W] = 0
 for i = 1 to n
   for j = 0 to W
     if w[i] <= j then
       DP[i][j] = max(DP[i-1][j],
          DP[i-1][j-w[i]] + v[i])
     else
       DP[i][j] = DP[i-1][j]
 return DP[n][W]
\end{lstlisting}

Coin Change finds fewest coins to make $n$ using $k$ coins $d_i$:
\begin{enumerate}[\roman*.]
  \item $dp(n)=\begin{cases}
        0, &n=0\\
        1 + \min_{i\in[k]}\{dp(j-d_i)\}
      \end{cases}$
  \item $T(n) = O(nk)$
\end{enumerate}
\begin{lstlisting}
COIN-CHANGE(D, N) 
 alloc array C[0..N] = INFINITY
 C[0] = 0
 for v = 1 to N
   for coin in D
     if coin <= v
   C[v] = min(C[v], C[v-coin] + 1)
 return C[N]
\end{lstlisting}

\colbreak


\section{Greedy Algorithms} % TODO: add optimal substructure proofs, and code examples
Greedy algorithms solve only one subproblem at each step, a locally optimal choice hoping it's globally optimal. It outperforms DP, and D\&C when it works. It involves:
\begin{enumerate}[\roman*.]
  \item Greedy Choice Property: a locally optimal choice must be globally optimal
  \item Optimal Substructure
\end{enumerate}

\subsection{Problems}

Fractional Knapsack allows taking fractions of a item:
\begin{enumerate}[\roman*.]
  \item Greedy by sorted maximum value/kg $v_i, w_i$, fill capacity, then take next item
  \item $T(n) = O(n \log n)$
  \item Greedy Choice Property: if $j^*$ is item with maximum value/kg $v_j /w_j$, there exists optimal knapsack containing $\min\{w_h, W\}$ kg of item $h^*$
\end{enumerate}
\begin{lstlisting}
FRACTIONAL-KNAPSACK(I, W)
  sort I by (v_i/ w_i) desending
  totalValue = 0
  for each (v, w) in I
      if W = 0
          return totalValue
      take = min(w, W)          
      totalValue += take * (v / w)
      W -= take
  return totalValue
\end{lstlisting}

\colbreak
Huffman Code is an optimal variable length prefix coding, where no prefix $\gamma(x)$ is prefix of $\gamma(y)$, and average bit length $ABL(\gamma) = \sum_{x\in A} freq(x)|\gamma(x)|$ is minimised:\\\vspace{1em}
{\centering
\incimg[0.9]{huffmancode}
\par}
\begin{enumerate}[\roman*.]
  \item Using a min-heap, repeatedly extract two least frequent $a, b$, create parent $z$ with weight $freq(a) + freq(b)$, and reinsert $z$. Final tree is an optimal code, where left branch adds $0$, and right branch adds $1$ to the code. 
  \item $T(n)=O(n\log n)$ (using heap)
  \item Greedy Choice Property: any merge not using two smallest nodes increases cost no less than merging the two smallest at every step 
\end{enumerate}
\begin{lstlisting}
HUFFMAN(C)
  n = |C|
  Q = C
  for i = 1 to n - 1
    allocate a new node z
    x = EXTRACT-MIN(Q)
    y = EXTRACT-MIN(Q)
    z.left = x
    z.right = y
    z.freq = x.freq + y.freq
    INSERT(Q, z)
  return EXTRACT-MIN(Q)
\end{lstlisting}

\colbreak
\section{Amortized Analysis}

Amortized analysis is used over sequence of operations to show that average cost per operation is small, even if some operations are expensive.
\begin{enumerate}[\roman*.]
  \item Guarantee average cost over $k$ operations $\leq k T(n)$ in the worst-case
\end{enumerate}

Methods:
\begin{enumerate}[\roman*.]
  \item Aggregate: $\frac{1}{k} \sum^k_{i=1}t(i)$; sum of costs divided by $k$ 
  \item Accounting: Overcharge cheap operations to "pay" for expensive operations later
  \item Potential: pick $\phi(0)=0$, $\phi(i)\ge 0$;\\ am. cost$ = t(i)+\phi(i)-\phi(i-1)$ such that\\ $\sum \text{am. cost of op $i$} \geq \text{actual cost of n ops} = \sum t(i)$ 
\end{enumerate}

\subsection{Problems}

Increment $k-$bit Binary Counter, counting bit flips:
\begin{enumerate}[\roman*.]
  \item Aggregate: bit $j$ flips $\frac{n}{2^j} \Rightarrow T(n)<2n \Rightarrow$ am. $O(1)$
  \item Accounting: charge \$2 per $0\!\to\!1$, use saved \$1 per $1\!\to\!0$ reset; banked $=$ no. of $1$s $\geq 0$ $\Rightarrow$ am. $O(1)$
  \item Potential: let $\phi(i)=$ \# of 1s after op $i$;\\ am. cost $=(\ell_i+1)+(-\ell_i+1)=2 = O(1)$
\end{enumerate}

Dynamic Table, when full, alloc size $2n$ and copy $n$ items:
\begin{enumerate}[\roman*.]
  \item Aggregate: $T(n) \leq n + \sum^{\log(n-1)}_{j=0}2^j \leq 3n$\\$\Rightarrow$ am. $O(1)$
  \item Accounting: charge \$2/insert, use saved \$1/copy per item; bank never negative $\implies$ am. $O(1)$
  \item Potential: let $\phi(i)=2i-\mathrm{size}(T)$;\\ full case am. $= i + (3-i) = 3 = O(1)$;\\ not-full case am. $= 1 + 2 = 3 = O(1)$
\end{enumerate}

\colbreak
\section{Problem Reduction}

Problem $A$ reduces to problem $B$ if we can:  
\begin{enumerate}[\arabic*.]
  \item convert an instance $\alpha$ of $A$ to an instance $\beta$ of $B$, where an instance denotes input 
  \item solve $\beta$ with $B$ to obtain solution $B(\beta)$,  
  \item convert $B(\beta)$ back into a solution $A(\alpha)$ for $\alpha$.  
\end{enumerate}

\subsection{$p(n)$-time Reduction}
Polynomial-time reduction from $A$ to $B$, $A \preduce B$, exists if:
\begin{enumerate}[\roman*.]
  \item Converting $\alpha \rightarrow \beta$, $B(\beta) \rightarrow A(\alpha)$ take $\leq p(n)$ time, and new $\beta$ takes $\leq p(n)$ size, where $p(n) \in O(n^c)$
  \item Consequence:
    \begin{itemize}\vspace{2pt}
      \item $B$ is easy ($p(n)$-time) $\implies A$ is easy
      \item $A$ is hard (no $p(n)$-time) $\implies B$ is hard
    \end{itemize}
  \item Length encoding: $n=|\alpha|$ is measured in length of bits, e.g. $\ell = \ceil{\log n}$ bits for numeric inputs.  
  \item Pseudo-polynomial time: if running time is $p(n)$ in numeric value (e.g. $O(n)$ iFib, $O(nW)$ knapsack) but exponential in the bit-length (e.g. $O(2^\ell)$).  
\end{enumerate}

Running-time Composition:
\begin{enumerate}[\roman*.]
  \item If $A \preduce B$ and $B$ can be solved in $T(n)$ time, then  
        $A$ can be solved in $\left(T(p(n)) + O(p(n))\right)$ time.
\end{enumerate}

\subsection*{Problems}
Longest Palindromic Subsequence $\preduce$ LCS:
\begin{enumerate}[\roman*.]
  \item $\text{LPS}(\alpha) = \text{LCS}(\alpha, reversed(\alpha))$
  \item Reverse and call LCS in $O(|\alpha|) \implies $ polynomial
\end{enumerate}

Matrix Square $\preduce$ Matrix Multiplication:
\begin{enumerate}[\roman*.]
  \item $\text{MAT-SQR}(C) =\text{MAT-MULTI}(C, C)$
  \item Copying $C$ in $O(n^{2})$ + one mult $\implies$ polynomial
\end{enumerate}

Matrix Multiplication $\preduce$ Matrix Square:
\begin{enumerate}[\roman*.]
  \item $\text{MAT-MULTI}(A, B) = \text{MAT-SQR}(\begin{bmatrix}
      0 & A\\
      B & 0
      \end{bmatrix})_{1, 1}$
  \item Build $M$ and find result in $O(n^{2})$ $\implies $polynomial.
\end{enumerate}


\colbreak

\section{NP-Completeness}

Decision problems map an instance space $I$ into the solution set \{YES, NO\}.
\begin{enumerate}[\roman*.]
  \item Given decision problems $A \preduce B$,\\ $\alpha$ is YES-instance for $A \Leftrightarrow \beta$ is YES-instance for $B$
\end{enumerate}


Complexity classes for decision problems in a deterministic Turing Machine:
\begin{enumerate}[\roman*.]
  \item P: solvable in $p(n)$ time; $P \subseteq NP$
  \item NP: certificate verifiable in $p(n)$ time
    \begin{itemize}[leftmargin=*]\vspace{2pt}
      \item Proof: give $p(n)$-time verifier
    \end{itemize}
  \item NP-Hard: if for every problem $ X \in$ NP, $X \preduce A$
    \begin{itemize}[leftmargin=*]\vspace{2pt}
      \item Proof: can show $p(n)$-time reduction to known NP-Complete problem
    \end{itemize}
  \item NP-Complete: in NP and NP-Hard
    \begin{itemize}[leftmargin=*]\vspace{2pt}
      \item Proof: prove both NP and NP-Hard 
    \end{itemize}
\end{enumerate}

\subsection*{NP-Complete Problems}
CSAT checks if DAG with AND, OR, NOT gate nodes and $n$ binary inputs can output 1.
\begin{enumerate}[\roman*]
  \item NP: evaluate certificate with DAG in $O(n)$
  \item NP-Hard: any $p(n)$-time NP verifier $Q$ can be built into circuit $C$ in $\Theta(p(n)^2)$ and solved by CSAT, hence all NP problems $\preduce$ CSAT  
\end{enumerate}

CNF-SAT checks satisfiability of CNF (product-of-sums):
\begin{enumerate}[\roman*]
  \item NP-Hard: convert each gate of circuit $C$ to a new var + $O(1)$ clauses. Hence, CSAT $\preduce$ CNF-SAT 
\end{enumerate}

3-SAT checks satisfiability of CNF with 3 literals/clause:
\begin{enumerate}[\roman*]
  \item NP-Hard: for any unrestricted CNF formula $\phi$, split every clause with $>3$ literals into a chain of 3-literal clauses with auxiliary variables in $O(n)$. Hence, CNF-SAT $\preduce$ 3-SAT
\end{enumerate}

Independent Set (IS) checks if $\leq k$ nodes in graph $G=(V, E)$ can share no edges:
\begin{enumerate}[\roman*.]
  \item IS$(G, k) =$ VC$(G, |V| - k) \implies$ IS $\preduce$ VC
  \item NP: reject if any edge has both ends in $X$ in $O(|E|)$
  \item NP-Hard: given a 3-SAT formula, create a graph with one node per literal, connecting pairs in the same clause and pairs of complementary literals. Hence, 3-SAT $\preduce$ IS 
\end{enumerate}

Vertex Cover (VC) checks if $\leq k$ nodes in graph $G=(V, E)$ can cover every edge:
\begin{enumerate}[\roman*.]
  \item NP: ensure all edges are adjacent to $X$ in $\Theta(|E|)$
  \item NP-Hard: VC$(G, k) =$ IS$(G, |V| - k) \Rightarrow$ VC $\preduce$ IS
\end{enumerate}

Hitting-Set (HS) checks if set $H$ of $\leq k$ elements can have non-empty intersection with all $S_i$ in $S =\{S_1,\cdots,S_n\}$:
\begin{enumerate}[\roman*]
  \item NP: $\forall S_i$ ensure $H\cap S_i\neq\varnothing$ in $\Theta\bigl(\sum_i|S_i|\bigr)$.
  \item NP-Hard: VC$(G, k) =$ HS$(S = \{ \{e\} : e \in E\} , k)$, where reduction is $\Theta(n)$. Hence, VC $\preduce$ HS.
\end{enumerate}

\mbox{CLIQUE checks if $\geq k$ nodes in $G = (V, E)$ can form clique:}
\begin{enumerate}[\roman*.]
  \item NP: check all nodes are adjacent in $\Theta(k^2)$
  \item NP-Hard: CLIQUE$(G, k) =$ IS$(\overline{G}, k)$, where reduction is $\Theta(|E|)$. Hence, CLIQUE $\preduce$ IS
\end{enumerate}

%%%%%%%%%%%%%%%%%%%%%%%%%%%%%%%%%%%%%%%%%%%%%%%%%%%%%%
%                       End                          %
%%%%%%%%%%%%%%%%%%%%%%%%%%%%%%%%%%%%%%%%%%%%%%%%%%%%%%
\end{multicols*}
\end{document}

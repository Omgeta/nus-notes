\documentclass[12pt, a4paper]{article}

\usepackage[a4paper, margin=1in]{geometry}

\usepackage[utf8]{inputenc}
\usepackage[mathscr]{euscript}
\let\euscr\mathscr \let\mathscr\relax
\usepackage[scr]{rsfso}
\usepackage{amssymb,amsmath,amsthm,amsfonts}
\usepackage[shortlabels]{enumitem}
\usepackage{multicol,multirow}
\usepackage{lipsum}
\usepackage{balance}
\usepackage{calc}
\usepackage[colorlinks=true,citecolor=blue,linkcolor=blue]{hyperref}
\usepackage{import}
\usepackage{xifthen}
\usepackage{pdfpages}
\usepackage{transparent}
\usepackage{tabularx}

\newcommand{\incfig}[2][1.0]{
    \def\svgwidth{#1\columnwidth}
    \import{./figures/}{#2.pdf_tex}
}
\newcommand{\incimg}[2][1.0]{
  \includegraphics[width=#1\columnwidth]{./figures/#2}
}


\input{letterfonts}

\newcommand{\mytitle}{CS3230 Tutorial 3}
\newcommand{\myauthor}{github/omgeta}
\newcommand{\mydate}{AY 25/26 Sem 1}

\begin{document}
\raggedright
\footnotesize
\begin{center}
{\normalsize{\textbf{\mytitle}}} \\
{\footnotesize{\mydate\hspace{2pt}\textemdash\hspace{2pt}\myauthor}}
\end{center}
\setlist{topsep=-1em, itemsep=-1em, parsep=2em}
%%%%%%%%%%%%%%%%%%%%%%%%%%%%%%%%%%%%%%%%%%%%%%%%%%%%%%
%                      Begin                         %
%%%%%%%%%%%%%%%%%%%%%%%%%%%%%%%%%%%%%%%%%%%%%%%%%%%%%%
\begin{enumerate}[Q\arabic*).]
  \item 
    \begin{enumerate}[(\alph*)]
      \item At start of iteration $i$, array $A[0..i-1]$ is sorted % with the same elements as the original array

      \item Initialization: At start of iteration $i=1$, array $A[0..0]$ is trivially sorted.\\
        Maintenance: Assuming $A[0..i-1]$ is sorted, the inner loops looks for the correct position to insert $A[i]$, leaving $A[0..i]$ sorted for the next iteration.\\
        Termination: Loop ends after $i=N-1$, leaving $A[0..N-1]$ sorted.\\
        (Optional) Inner Loop Invariant: At start of iteration $j$, array $A[j+1..i]$ consists of the original elements $\geq X$ shifted right by one for $X$ to be inserted.
    \end{enumerate}

  \item 
    \begin{enumerate}[(\alph*)]
      \item 
        \begin{enumproof}
        \item Base Case: if $n\leq 2$, swap if necessary and resultant array is trivially sorted
        \item Inductive Step: Suppose StoogeSort correct sorts arrays of size $< n$.\\Let $X, Y, Z$ denote the 3 thirds of the array.
          \begin{enumproof}
          \item Sorting first $\ceil{2n /3}$ sorts $X, Y$ by IH, so all elements in $Y$ are $\geq$ elements in $X$.
          \item Sorting last $\ceil{2n /3}$ sorts $Y, Z$ by IH, so all elements in $Z$ are $\geq$ all elements in $X, Y$. $Z$ is now finished.
          \item Sorting first $\ceil{2n /3}$ sorts remaining elements in $X, Y$ by IH, and we're done.
          \end{enumproof}
        \end{enumproof}

      \item $T(n) = 3T(\ceil{2n /3}) + \Theta(1) \in O(n^2)$;\\since $d = \log_{3/2}{3} \approx 2.71$ and $f(n) \in O(n^{\log_{3 /2}+\epsilon})$ for $\epsilon = 0.000001$,\\$\therefore$ by case 1 of Master Theorem, $T(n) \in \Theta(n^{\log_{1.5}3}) = \Theta(n^{2.71})$
    \end{enumerate}

  \item Take the maximum of any 2D array. By definition, the maximum is $\geq$ its neighbours, therefore there is always a peak which is the maximum.

  \item $T(m, n) = 2T(m, \floor{n /2}) + \Theta(m)$. There are $\log n$ levels, $2^k$ subproblems at each level and $\Theta(m)$ work for each subproblem. So overall we have total $\sum^{\log n}_{i=1}2^i = n$ subproblems so $\Theta(mn)$ time. 

  \item 
    \begin{enumproof}
    \item Base Case: if $n = 1$, maximal of column is trivially a peak because there are no left-right neighbours, and its $\geq$ up-down neighbours  
    \item Inductive Step: Assume there is a peak for all widths $< n$.
      \begin{enumproof}
      \item If maximum of $C_m$ is a peak, we're done
      \item Else, there must be a left-right neighbour greater. WLOG, if the left neighbour is greater, recurse on the left half. Then by IH, there must be a peak in the left half.
      \end{enumproof}
    \end{enumproof}
    To optimize, we only recurse on the half with the larger neighbour of the current maximal element, giving $T(n) = T(m, \floor{n /2}) + \Theta(m) \in \Theta(m\log n)$ 

\end{enumerate}
%%%%%%%%%%%%%%%%%%%%%%%%%%%%%%%%%%%%%%%%%%%%%%%%%%%%%%
%                       End                          %
%%%%%%%%%%%%%%%%%%%%%%%%%%%%%%%%%%%%%%%%%%%%%%%%%%%%%%

\end{document}

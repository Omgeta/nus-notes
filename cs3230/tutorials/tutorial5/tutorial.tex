\documentclass[12pt, a4paper]{article}

\usepackage[a4paper, margin=1in]{geometry}

\usepackage[utf8]{inputenc}
\usepackage[mathscr]{euscript}
\let\euscr\mathscr \let\mathscr\relax
\usepackage[scr]{rsfso}
\usepackage{amssymb,amsmath,amsthm,amsfonts}
\usepackage[shortlabels]{enumitem}
\usepackage{multicol,multirow}
\usepackage{lipsum}
\usepackage{balance}
\usepackage{calc}
\usepackage[colorlinks=true,citecolor=blue,linkcolor=blue]{hyperref}
\usepackage{import}
\usepackage{xifthen}
\usepackage{pdfpages}
\usepackage{transparent}
\usepackage{tabularx}

\newcommand{\incfig}[2][1.0]{
    \def\svgwidth{#1\columnwidth}
    \import{./figures/}{#2.pdf_tex}
}
\newcommand{\incimg}[2][1.0]{
  \includegraphics[width=#1\columnwidth]{./figures/#2}
}


\input{letterfonts}

\newcommand{\mytitle}{CS3230 Tutorial 5}
\newcommand{\myauthor}{github/omgeta}
\newcommand{\mydate}{AY 25/26 Sem 1}

\begin{document}
\raggedright
\footnotesize
\begin{center}
{\normalsize{\textbf{\mytitle}}} \\
{\footnotesize{\mydate\hspace{2pt}\textemdash\hspace{2pt}\myauthor}}
\end{center}
\setlist{topsep=-1em, itemsep=-1em, parsep=2em}
%%%%%%%%%%%%%%%%%%%%%%%%%%%%%%%%%%%%%%%%%%%%%%%%%%%%%%
%                      Begin                         %
%%%%%%%%%%%%%%%%%%%%%%%%%%%%%%%%%%%%%%%%%%%%%%%%%%%%%%
\begin{enumerate}[Q\arabic*).]
  \item Take $\displaystyle A = [1], B = [1]$ so $AB = [1]$ and $C = [0]$ so $C \neq AB$. Thus, we have $AB \vec{r} = r$  and $C \vec{r} = 0$, which means that when we pick $r = 1$ with probability $\frac{1}{2}$, we get $AB \neq C$ (correct) and when we pick $r = 0$ with probability $\frac{1}{2}$ we get $AB = C$ (false positive). 

  \item Case 1 ($S_A = S_B$): trivially $S_A \mod p = S_B \mod p$

    Case 2 ($S_A \neq S_B$): Let $D = S_A - S_B$, then $S_A \mod p = S_B \mod p \implies p \mid D$\\
    Since $S_A, S_B \leq 2^n$, then $D \leq 2^n$\\
    Also since by prime factorisation theorem, $D$ can be written uniquely as $D = p_1p_2\cdots p_k$ and all prime numbers are $\geq 2$, then $D \geq 2\cdot 2 \ldots \cdot 2 = 2^k$\\
    Now, $2^k \leq D \leq 2^n \implies 2^k \leq 2^n \implies k \leq n \iff D$ has atmost $n$ prime factors 

    $Pr[\text{failure}] = Pr[p \mid D] = \frac{\text{\# distinct prime factors in $D$}}{\text{\# of prime numbers }p} \leq \frac{k}{|S|} \leq \frac{n}{n^2} = \frac{1}{n}$\\
    $\therefore Pr[\text{success}] \geq 1 - \frac{1}{n}$

  \item Suppose not, that we send $m < n$ bits and test for equality.\\
    If $S_A$ was $n$ bits we would have $2^{n}$ different messages, but with $m$ bits we only have $2^{m} < 2^{n}$ different messages.\\
    By Pigeonhole Principle, two different $S_A$ now correspond to the same encoded message then Bob cannot distinguish between the 2 original messages.

  \item For each edge $e$, denote indicator random variable $X_e = \begin{cases}1&\text{if $e$ crosses the cut}\\ 0&\text{if $e$ does not cross the cut}\end{cases}$\\
    $\therefore$ Total edges crossing the cut $= \displaystyle\sum_{e\in E}X_e$\\
    By fixing the cut, $Pr[X_e = 1] = \frac{1}{2}$ so $E[X_e] = 1 \cdot P[X_e = 1] + 0\cdot P[X_e = 0] = \frac{1}{2}$

    Therefore, $\displaystyle E[\text{total edges crossing cut}] = E[\sum_{e\in E}X_e] = \sum_{e\in E} E[X_e] = \sum_{e\in E}\frac{1}{2} = \frac{|E|}{2}$

  \item We don't guarantee $V_1, V_2$ are non empty sets. To fix this, we fix our original cut across the first edge $e^*$.

    Therefore, $\displaystyle E[\sum_{e\in E}X_e] = \sum_{e\in E}E[X_e] = E[X_{e^*}] + \sum_{e\in E \backslash\{e^*\}} E[X_e] = 1 + \frac{|E| - 1}{2} = \frac{|E| + 1}{2} > \frac{|E|}{2}$
\end{enumerate}
%%%%%%%%%%%%%%%%%%%%%%%%%%%%%%%%%%%%%%%%%%%%%%%%%%%%%%
%                       End                          %
%%%%%%%%%%%%%%%%%%%%%%%%%%%%%%%%%%%%%%%%%%%%%%%%%%%%%%

\end{document}

\documentclass[12pt, a4paper]{article}

\usepackage[utf8]{inputenc}
\usepackage[mathscr]{euscript}
\let\euscr\mathscr \let\mathscr\relax
\usepackage[scr]{rsfso}
\usepackage{amssymb,amsmath,amsthm,amsfonts}
\usepackage[shortlabels]{enumitem}
\usepackage{multicol,multirow}
\usepackage{lipsum}
\usepackage{balance}
\usepackage{calc}
\usepackage[colorlinks=true,citecolor=blue,linkcolor=blue]{hyperref}
\usepackage{import}
\usepackage{xifthen}
\usepackage{pdfpages}
\usepackage{transparent}
\usepackage{tabularx}

\newcommand{\incfig}[2][1.0]{
    \def\svgwidth{#1\columnwidth}
    \import{./figures/}{#2.pdf_tex}
}
\newcommand{\incimg}[2][1.0]{
  \includegraphics[width=#1\columnwidth]{./figures/#2}
}


\usepackage{ifthen}
\usepackage[landscape]{geometry}
\usepackage[shortlabels]{enumitem}

\ifthenelse{\lengthtest { \paperwidth = 11in}}
    { \geometry{top=.5in,left=.5in,right=.5in,bottom=.5in} }
	{\ifthenelse{ \lengthtest{ \paperwidth = 297mm}}
		{\geometry{top=1cm,left=1cm,right=1cm,bottom=1cm} }
		{\geometry{top=1cm,left=1cm,right=1cm,bottom=1cm} }
	}

\pagestyle{empty}
\makeatletter
\renewcommand\thesection{\arabic{section}.}
\renewcommand{\section}{\@startsection{section}{1}{0mm}%
                                {-1ex plus -.5ex minus -.2ex}%
                                {0.05ex}%x
                                {\normalfont\normalsize\bfseries}}
\renewcommand{\subsection}{\@startsection{subsection}{2}{0mm}%
                                {-1ex plus -.5ex minus -.2ex}%
                                {0.05ex}%
                                {\normalfont\small\bfseries}}
\renewcommand{\subsubsection}{\@startsection{subsubsection}{3}{0mm}%
                                {-1ex plus -.5ex minus -.2ex}%
                                {0.05ex}%
                                {\normalfont\footnotesize\bfseries}}
\newcommand{\colbreak}{\vfill\null\columnbreak}
\makeatother
\setcounter{secnumdepth}{1}
\setlength{\parindent}{0pt}
\setlength{\parskip}{0.7em}

\setlist[itemize]{itemsep=0.6ex, topsep=-2pt, partopsep=0pt, parsep=0pt}
\setlist[enumerate]{itemsep=0.6ex, topsep=-2pt, partopsep=0pt, parsep=0pt}

\input{letterfonts}

\newcommand{\mytitle}{EL1101E Nature of Language}
\newcommand{\myauthor}{github/omgeta}
\newcommand{\mydate}{AY 25/26 Sem 2}

\begin{document}
\raggedright
\footnotesize
\begin{multicols*}{3}
\setlength{\premulticols}{1pt}
\setlength{\postmulticols}{1pt}
\setlength{\multicolsep}{1pt}
\setlength{\columnsep}{2pt}

{\normalsize{\textbf{\mytitle}}}\\
{\footnotesize{\mydate\hspace{2pt}\textemdash\hspace{2pt}\myauthor}}
%%%%%%%%%%%%%%%%%%%%%%%%%%%%%%%%%%%%%%%%%%%%%%%%%%%%%%
%                      Begin                         %
%%%%%%%%%%%%%%%%%%%%%%%%%%%%%%%%%%%%%%%%%%%%%%%%%%%%%%
\section{Introduction}
Linguistics is the descriptive scientific study of language as a system and as a social phenomenon. It studies:
\begin{enumerate}[\roman*.]
  \item How language is acquired
  \item How it is processed in the brain
  \item How it can be processed by computers
  \item How it changes over time
  \item How it varies by situation
  \item How it functions in society
\end{enumerate}

Big Questions:
\begin{enumerate}[\roman*.]
  \item Do all languages share an underlying structure?
  \item Is the way you think shaped by languages spoken?
  \item Is language learned and processed in the brain differently from other skills?
\end{enumerate}

Qualtitative and Quantitive Research Methods:
\begin{enumerate}[\roman*.]
  \item Analytical Reasoning: observation of data and identification of patterns 
  \item Brain Imaging
  \item Acoustic Analysis 
  \item Statistical Analysis of Corpora
  \item Ethnographic Fieldwork
\end{enumerate}

Knowledge of Language:
\begin{enumerate}[\roman*.]
  \item Function: knowing how to communicate
  \item Form: knowing the words of that language and the rules for putting them together
\end{enumerate}

Key Features of Language:
\begin{enumerate}[\roman*.]
  \item Arbitariness: of units to concepts
  \item Discreteness: discrete units with levels of structure 
  \item Compositionality: of larger units by smaller units
  \item Productivity: using finite grammar to compose infinite number of utterances
  \item Rule-governedness: set of conventional rules
\end{enumerate}

\colbreak
\section{Phonetics}
Phonetics is the study of speech sounds. Areas:
\begin{enumerate}[\roman*.]
  \item Articulatory: how speech sounds are produced
  \item Acoustic: acoustic properties of the speech signal
  \item Auditory: listeners' perception of speech sounds
\end{enumerate}

Abstractions:
\begin{enumerate}[\roman*.]
  \item Speech Stream: speech as a continuous stream 
  \item Speech Chain: production/perception is a chain 
    \begin{itemize}[leftmargin=*]\vspace{3pt}
      \item linguistic $\rightarrow$ physiological $\rightarrow$ acoustic $\rightarrow$ physiological $\rightarrow$ linguistic
    \end{itemize} 
\end{enumerate}

Speech Production:
\begin{enumerate}[\roman*.]
  \item Voice is powered by air from the lungs
  \item Speech is the molding of air by the vocal tract as it travels to escape through our mouth and nose
  \item Voiced/Voiceless: vocal folds vibrate/ don't vibrate
  \item Nasal/Oral: velum lowered/ raised
\end{enumerate}
{\centering
  \incimg[0.6]{vocal_tract}
\par}\vspace{-1em}
Consonants: airflow via oral cavity is obstructed
\begin{enumerate}[\roman*.]
  \item Voicing (voiced vs.\ voiceless)
  \item Place of articulation (location of obstruction)
  \item Manner of articulation (cause of obstruction)
\end{enumerate}
Vowels: airflow via oral cavity is unobstructed
\begin{enumerate}[\roman*.]
  \item Rounding: rounded vs. unrounded lips
  \item Height: tongue height (high, mid, low)
  \item Frontness: tongue position (front, central, back)
  \item Tenseness: effort (tense, lax)
  \item Monophthongs vs.\ diphthongs (steady, transient)
\end{enumerate}

Places of Articulation (English):
\begin{enumerate}[\roman*.]
  \item Bilabial: obstruction made with two lips
  \item Labiodental: lower lip against upper teeth
  \item Interdental: tongue between teeth
  \item Alveolar: tongue at/near alveolar ridge
  \item Post-alveolar: tongue just behind alveolar ridge
  \item Palatal: tongue body against/near hard palate
  \item Velar: tongue body against/near velum
  \item Glottal: constriction at glottis
\end{enumerate}

Manners of Articulation (English):
\begin{enumerate}[\roman*.]
  \item Stops: complete closure of oral cavity
    \begin{itemize}[leftmargin=*]\vspace{3pt}
      \item Plosive: followed by release of closure
      \item Nasal: airflow redirected through the nose
    \end{itemize}
  \item Fricatives: partial constriction causing turbulence 
  \item Affricates: stop + fricative as a single segment
  \item Approximants: narrowing but no turbulence 
    \begin{itemize}[leftmargin=*]\vspace{3pt}
      \item Liquid: relatively open constriction 
      \item Glide: very vowel-like
    \end{itemize}
\end{enumerate}

IPA (International Phonetic Alphabet):
\begin{enumerate}[\roman*.]
  \item Characters + diacritics representing speech sounds 
  \item Consistent representation across all languages
  \item Narrow (specific) vs. broad (general) transcriptions
\end{enumerate}

Lexical Stress:
\begin{enumerate}[\roman*.]
  \item Stress: emphasis placed on a specific linguistic unit
  \item Lexical Stress: emphasis placed on a specific syllable within a word
  \item Stressed syllables are higher, louder, and longer; unstressed vowels are likely to be reduced
  \item Primary stress: one syllable receives primary stress (optionally, mark with [\textipa{\textprimstress}])
  \item Secondary stress: longer words sometimes have secondary stress (optionally, mark with [\textipa{\textsecstress}])
\end{enumerate}

\colbreak
\section{Phonology}
Phonology is the study of the structures and patterns of speech sounds within languages.

Representations:
\begin{enumerate}[\roman*.]
  \item Phone: basic unit of speech sound 
  \item Phoneme: phonological units that contrast in a language, i.e. replacement forms different word
    (e.g. /t/)
  \item Allophone: possible phone realization of a phoneme
    (e.g. [t], [\textipa{\textfishhookr}], [t$^{\text{h}}$], etc.)
  \item Language-specific: allophones vary
  \item Phones can realise more than one phoneme
\end{enumerate}

Phone Distributions:
\begin{enumerate}[\roman*.]
  \item Contrastive: two phones can occur in the same phonological environment $\Rightarrow$ different phonemes
    \begin{itemize}[leftmargin=*]\vspace{3pt}
      \item Minimal pairs: words differ by only one sound
    \end{itemize}
  \item Complementary: two phones never occur in the same phonological environment $\Rightarrow$ likely allophones
    \begin{itemize}[leftmargin=*]\vspace{3pt}
      \item Must also be phonetically similar (share features)
    \end{itemize}
  \item Free: two phones can occur in the same phonological environment as alternate pronunciation of same word $\Rightarrow$ allophones
\end{enumerate}

Phonological Rules describe when a phoneme is realised as a particular allophone:
\begin{enumerate}[\roman*.]
  \item Maximise Parsimony: simple, broad coverage
  \item Natural Classes: group of phones defined by phonological similarity (e.g. "voiced plosive")
  \item Format: $A \rightarrow B$ / [env. 1] \_ [env. 2]
    \begin{itemize}[leftmargin=*]\vspace{3pt}
      \item $A$ realised as $B$ when it occurs after env. 1 and before env. 2; else, realized by default as $A$
      \item Features: used for natural classes 
      \item IPA: used for specific segments
    \end{itemize}
  \item Environmental Symbols:
    \begin{itemize}[leftmargin=*]\vspace{3pt}
      \item \$: syllable boundary
      \item \#: word boundary
      \item V, C: vowel, consonant 
      \item $\varnothing$: deletion
    \end{itemize}
\end{enumerate}

Common Phonological Patterns:
\begin{enumerate}[\roman*.]
  \item Assimilation: nearby sounds become more similar
  \item Dissimilation: nearby sounds become less similar
  \item Epenthesis: insert a phone
  \item Deletion: underlying phoneme is not realised on the surface phonetic level
  \item Metathesis: switch sounds
\end{enumerate}

Phonotactics describe constraints on where phones can appear in a language:
\begin{enumerate}[\roman*.]
  \item Language-specific: a sequence can use existing phonemes but still be ill-formed
  \item Syllables: primary prosodic unit
    \begin{itemize}[leftmargin=*]\vspace{3pt}
      \item Prosody: rhythm, stress, intonation 
      \item Structure: Onset + Rime (Nucleus + Coda)
      \item Must have a nucleus; onset and coda are optional
      \item Dots in IPA indicate syllable boundaries
    \end{itemize}
  \item Sonority Hierarchy:
    \begin{itemize}[leftmargin=*]\vspace{3pt}
      \item Relative loudness of phones 
      \item vowels $>$ approximants $>$ nasals $>$ fricatives $>$ affricates $>$ plosives
      \item More sonorous phones tend to be closer to the syllable nucleus
    \end{itemize}
\end{enumerate}

Example Phonotactic constraints (English):
\begin{enumerate}[\roman*.]
  \item If the first consonant in a complex onset is not an /s/, the second must be a liquid or glide
    \begin{itemize}[leftmargin=*]\vspace{3pt}
      \item OK: \textit{trap, glum, break, please, cute, quake}; \textit{stare, sphere, scare, spare}
      \item Ill-formed: *\textit{ksare}, *\textit{tfare}
    \end{itemize}
  \item If the second consonant in a complex coda is voiced, so is the first
    \begin{itemize}[leftmargin=*]\vspace{3pt}
      \item OK: \textit{bend, bent, best}
      \item Ill-formed: *\textit{besd}
    \end{itemize}
\end{enumerate}

\colbreak
\section{Morphology}
Morphology is the study of words and their structure.

Morpheme is the smallest meaningful unit in a language:
\begin{enumerate}[\roman*.]
  \item Bound: cannot stand as an independent word
  \item Free: can stand on its own as a simple word
\end{enumerate}

Affixes form most bound morphemes:
\begin{enumerate}[\roman*.]
  \item Occur more than once in the lexicon
  \item Have identifiable meaning/ grammatical function
  \item Be added to modify core meaning
  \item Types: prefix, suffix, infix, circumfix
\end{enumerate}

Productive vs.\ Unproductive Affixes:
\begin{enumerate}[\roman*.]
  \item Productive: new words are commonly made using it
  \item Unproductive: falls out of use over time
\end{enumerate}

Inflectional vs.\ Derivational Affixes:
\begin{enumerate}[\roman*.]
  \item Derivational: creates a new word with a different meaning (often different part of speech)
    \begin{itemize}[leftmargin=*]\vspace{3pt}
      \item Class-changing (e.g.\ V$\rightarrow$N; ADJ$\rightarrow$ADV)
      \item Class-maintaining (same class in $\rightarrow$ out)
    \end{itemize}
  \item Inflectional: expresses grammatical information (e.g.\ tense, number); always class-maintaining
  \item Function over Form: identify morphemes by how they are used, not just by shape
\end{enumerate}

Non-affix Bound Morphemes 
\begin{enumerate}[\roman*.]
  \item Cranberry Morphemes: bound morphemes with no clear meaning (often due to language change/ borrowing; e.g. cran-, twi-)
  \item Bound Roots: roots that carry core meaning but cannot occur alone (common in some languages)
\end{enumerate}

Grammaticalization:
\begin{enumerate}[\roman*.]
  \item Process where a free morpheme gradually serves a fixed grammatical function over time
  \item Involves phonological reduction + loss of syntactic freedom, sometimes into a bound morpheme\\(e.g. hope-full $\rightarrow$ hopeful, happy-like $\rightarrow$ happily)
\end{enumerate}
\colbreak
Roots, Stems, Compounds:
\begin{enumerate}[\roman*.]
  \item Root: primary indivisble morpheme 
    \begin{itemize}[leftmargin=*]\vspace{3pt}
      \item In English, almost always free morphemes
    \end{itemize}
  \item Stem: unit that an affix attaches onto 
    \begin{itemize}[leftmargin=*]\vspace{3pt}
      \item Can be root alone, or root + affix(es)
    \end{itemize}
  \item Compound Word: multiple stems combined 
    \begin{itemize}[leftmargin=*]\vspace{3pt}
      \item Orthography is irrelevant (e.g. hyphens)
    \end{itemize}
  \item Stress test:
    \begin{itemize}[leftmargin=*]\vspace{3pt}
      \item Compounds: stress first stem
      \item Two-word sequences: stress both words
    \end{itemize}
\end{enumerate}

Word is the smallest meaningful unit that can occur alone:
\begin{enumerate}[\roman*.]
  \item Orthographic Word: "stuff between spaces"\\(useful, but not sufficient cross-linguistically)
  \item Word Boundary Criterion:
    \begin{itemize}[leftmargin=*]\vspace{3pt}
      \item Orthography (e.g. spaces)
      \item Phonology (stress patterns; word-internal rules)
      \item Divisibility (can you insert more material?)
      \item Pause (can speakers pause here?)
    \end{itemize}
\end{enumerate}

Word Structure:
\begin{enumerate}[\roman*.]
  \item Affixes have constraints on what classes they attach to and what they create
  \item Multiple affixation can force an order of formation (intermediate forms must be well-formed)
  \item Word structure is hierarchical:
    \begin{itemize}[leftmargin=*]\vspace{3pt}
      \item Binary-branching
      \item Each node has a label
      \item Each stage must be a well-formed word
    \end{itemize}
  \item Ambiguity arises if more than one tree is possible
\end{enumerate}

\incimg{word_structure}
\colbreak

\section{Word Classes}
Word Classes categorise words based on grammatical behaviour, not just meaning:
\begin{enumerate}[\roman*.]
  \item Noun (incl.\ pronoun): entities
  \item Verb: actions
  \item Adjective: descriptors for nouns
  \item Adverb: descriptors for verbs
  \item Preposition: indicate time/place/direction/manner
  \item Determiner: specify nouns
  \item Conjunction: link words/phrases/clauses
  \item Auxiliary Verb: helping verbs
\end{enumerate}

Content vs. Function Words:
\begin{enumerate}[\roman*.]
  \item Content Words:
    \begin{itemize}[leftmargin=*]\vspace{3pt}
      \item Incl. nouns, verbs, adjectives, adverbs 
      \item Open-class: can easily add new words 
      \item Contentful: easily defined meaning
    \end{itemize}
  \item Function Words:
    \begin{itemize}[leftmargin=*]\vspace{3pt}
      \item Incl. determiners, pronouns, conjunctions, prepositions, auxilaries 
      \item Closed-class: lexicon relatively stable
      \item Serve grammatical functions, relationships between content words
    \end{itemize}
\end{enumerate}

Word Formation:
\begin{enumerate}[\roman*.]
  \item Affixation
  \item Compounding
  \item Conversion: change word class with no new morphemes added
  \item Backformation: from reanalysis of structure
  \item Clipping: cutting off part of a word
  \item Blending: combining clippings into a blend 
  \item Acronym: initials pronounced as a word
  \item Initialism: pronounced as letters
\end{enumerate}
\colbreak
\section{Syntax}
Syntax is the study of rules underlying sentence structure.
\begin{enumerate}[\roman*.]
  \item Grammaticality does not depend on acceptability/meaningfulness of the sentence
  \item Approaches:
    \begin{itemize}[leftmargin=*]\vspace{3pt}
      \item Top-down: how can we parse a sentence?
      \item Bottom-up: what rules determine how words fit?
    \end{itemize}
\end{enumerate}

Constituents are chunks that operate as a single unit:
\begin{enumerate}[\roman*.]
  \item Types: words, phrases, sentences
  \item Constituency Tests: 
    \begin{itemize}[leftmargin=*]\vspace{3pt}
      \item Topicalization: move a chunk to the front
      \begin{itemize}[leftmargin=*]\vspace{2pt}
        \item \textit{In linguistics, Kunmei tutors us.}
      \end{itemize}
      \item Clefts: \textit{It is X that Y}
      \begin{itemize}[leftmargin=*]\vspace{2pt}
        \item \textit{It is in linguistics that Kunmei tutors us.}
      \end{itemize}
      \item Pseudoclefts: \textit{X is what Y}
      \begin{itemize}[leftmargin=*]\vspace{2pt}
        \item \textit{Tutors us in linguistics is what Kunmei does.}
      \end{itemize}
      \item Substitution: replace chunk with a pronoun / known constituent (e.g.\ \textit{it}, \textit{do so})
      \begin{itemize}[leftmargin=*]\vspace{2pt}
        \item \textit{Yunbo tutors us in linguistics and Kunmei does, too.}
      \end{itemize}
      \item Deletion: if the chunk can be removed
      \begin{itemize}[leftmargin=*]\vspace{2pt}
        \item \textit{Kunmei tutors us.}
      \end{itemize}
    \end{itemize}
    \begin{enumerate}[leftmargin=*, label=$-$]\vspace{5pt}
      \item Not 100\% reliable, but passing $>1$ test is strong evidence for constituency
    \end{enumerate}
\end{enumerate}

Phrase is an intermediate between words and sentences:
\begin{enumerate}[\roman*.]
  \item Head: determines the phrase's syntactic class:
    \begin{itemize}[leftmargin=*]\vspace{3pt}
      \item Noun Phrase (NP)
      \item Verb Phrase (VP)
      \item Prepositional Phrase (PP)
    \end{itemize}
  \item Head Test: ``A Y is a type of X''
    \begin{itemize}[leftmargin=*]\vspace{3pt}
      \item ``The yellow box'' is a type of ``box''
      \item ``ate an apple'' is a type of eating
      \item ``in the little box'' is a type of in-position
    \end{itemize}
\end{enumerate}

\colbreak
Phrase Structure:
\begin{enumerate}[\roman*.]
  \item Rules indicate what types of constituents can make up different classes of phrases. In English:
    \begin{itemize}[leftmargin=*]\vspace{3pt}
      \item NP $\rightarrow$ (Det) (Adj)$^*$ N (PP)
      \item PP $\rightarrow$ P NP
      \item VP $\rightarrow$ V (NP) (NP) (PP)
      \item S $\rightarrow$ NP VP\hfill(Complete Sentence)
    \end{itemize}
  \item Syntax Trees: represent hierarchical structure 
    \begin{itemize}[leftmargin=*]\vspace{3pt}
      \item A string is a constituent $\Leftrightarrow$ there is a node that exclusively dominates that sequence
    \end{itemize}
\end{enumerate}
{\centering\incimg[0.3]{syntax_tree}\par}\vspace{-1em}
Sentence Constituents:
\begin{enumerate}[\roman*.]
  \item Subject (NP)
  \item Predicate (VP)
    \begin{itemize}[leftmargin=*]\vspace{3pt}
      \item Main: verbs differ in how many and what types of arguments they take
      \begin{itemize}[leftmargin=*]\vspace{3pt}
        \item Intransitive: subject only (no object)
        \item Transitive: subject + object
        \item Ditransitive: subject + two objects 
      \end{itemize}
    \item Auxiliary: modify the grammatical function and meaning of the main verb
    \item Argument: necessary for predicates' meaning
    \item Adjunct: optional; adds information about how something occurred
    \end{itemize}
\end{enumerate} 

Recursion:
\begin{enumerate}[\roman*.]
  \item Nesting: rules can generate infinitely long sentences
  \item Coordination: link constituents of the same syntactic class as sisters
  \item Subordination: add a subordinate constituent as the daughter of another constituent
\end{enumerate}
\colbreak

Syntactic Ambiguity arises from multiple possible syntactic structures:
\begin{enumerate}[\roman*.]
  \item Example: \textit{She saw the man with the telescope} 
    \begin{itemize}[leftmargin=*]\vspace{3pt}
      \item \text{[VP saw [NP the man] [PP with the telescope]]} - used a telescope to see the man

      \item \text{[VP saw [NP the man [PP with the telescope]]]} - saw a man carrying a telescope
    \end{itemize}
  \item Garden-path Sentence: difficult to parse initially due to ambiguity
    \begin{itemize}[leftmargin=*]\vspace{3pt}
      \item \textit{The horse raced past the barn fell.} 
    \end{itemize}
  \item Demonstrates parsing strategies and how syntax interacts with processing.
\end{enumerate}

Word Order:
\begin{enumerate}[\roman*.]
  \item Basic Syntactic Roles: subject, verb, object.
  \item Word Order Typology:
    \begin{itemize}[leftmargin=*]\vspace{3pt}
      \item SVO: English, Chinese, Malay, French
      \item SOV: Japanese, Tamil, Korean
      \item VSO: Tagalog, Irish Gaelic, Arabic
      \item VOS/OVS rare: Malagasy 
    \end{itemize}
  \item Generalisation: subjects tend to precede objects
\end{enumerate}

Morphology-Syntax Tradeoff:
\begin{enumerate}[\roman*.]
  \item Inflectional Morphology: changes on a word mark grammatical information (case, number, agreement)
  \item Analytic Languages: minimal inflectional morphology, stricter word order 
    \begin{itemize}[leftmargin=*]\vspace{3pt}
      \item English: \textit{Rebecca gave Melody a dog}\\$\neq$ \textit{Melody gave Rebecca a dog}
      \item Roles are largely signalled by position (subject/object/indirect object)
    \end{itemize}
  \item Synthetic Languages: rich inflectional morphology, freer word order
    \begin{itemize}[leftmargin=*]\vspace{3pt}

      \item Latin: dominus (nominative), domine (vocative)\\Troia est in Asia = Troia in Asia est 
    \end{itemize}
  \item Generalisation: languages converge to balance overall grammatical complexity
\end{enumerate}
\colbreak
\section{Semantics}
Semantics is the study of meaning in language.
\begin{enumerate}[\roman*.]
  \item Referent Approach: meaning is tied to the referent
    \begin{itemize}[leftmargin=*]\vspace{3pt}
      \item Referent: the object/entity a word refers to; can be constant or variable
    \end{itemize}
    \begin{enumerate}[leftmargin=*, label=$-$]\vspace{5pt}
      \item Frege's Puzzle: same referent can lead to different truth in different belief contexts
        \begin{itemize}[leftmargin=*]\vspace{3pt}
          \item Solution: Sense: way term refers to referent
      \end{itemize}
    \end{enumerate}
  \item Non-referring Expressions: meaningful with sense
    \begin{itemize}
      \item Truth-value gap: neither true nor false\\(no referent to evaluate against)
    \end{itemize}
\end{enumerate}

Word Meaning Relationships:
\begin{enumerate}[\roman*.]
  \item Similar Form, Different Meanings: 
    \begin{itemize}[leftmargin=*]\vspace{3pt}
      \item Homophone: same pronunciation, diff. spelling
      \item Homograph: same spelling, diff. pronunciation
      \item Homonym: same spelling and pronunciation 
      \item Polyseme: same word used in diff. related sense
    \end{itemize}
  \item Hyponomy: if $A \subseteq B$
    \begin{itemize}[leftmargin=*]\vspace{3pt}
      \item $A$ is a hyponym of $B$
      \item $B$ is a hypernym of $A$
    \end{itemize}
  \item Similarity/Opposition:
    \begin{itemize}[leftmargin=*]\vspace{3pt}
      \item Synonyms: words with the same/similar meaning
      \item Antonyms: words with opposite meaning
      \begin{itemize}[leftmargin=*]\vspace{5pt}
        \item Gradable: ends of a continuous scale 
        \item Complementary: no middle ground\\(can be used gradably metaphorically)
        \item Relational: opposite roles in a relationship\\(``if there is an X, there must be a Y'')
      \end{itemize}
    \end{itemize}
\end{enumerate}

\colbreak

Semantic Shift/Drift is the change of meaning in words over time. Patterns:
\begin{enumerate}[\roman*.]
  \item Narrowing: meaning becomes more specific\\(e.g.\ \textit{girl}: child of either sex $\rightarrow$ female child)
  \item Broadening: meaning becomes more general\\(incl.\ genericization; e.g.\ \textit{Kleenex, Hoover, Google})
  \item Upgrading: meaning becomes more positive\\(e.g.\ \textit{nice}: stupid $\rightarrow$ kind or pleasant)
  \item Downgrading: meaning becomes more negative\\(e.g.\ many terms for \textit{women}: \textit{wench, tart, bitch})
\end{enumerate}

Sentence Meaning Relationships:
\begin{enumerate}[\roman*.]
  \item Entailment: if $A$ is true, $B$ must be true
    \begin{itemize}[leftmargin=*]\vspace{3pt}
      \item \textit{Taylor bought a poodle} $\Rightarrow$ \textit{Taylor bought a dog}
    \end{itemize}
  \item Contradiction: if $A$ is true, $B$ must be false
    \begin{itemize}[leftmargin=*]\vspace{3pt}
      \item  \textit{Taylor bought a dog} $\Rightarrow\Leftarrow$ \textit{Taylor didn't buy a dog}
    \end{itemize}
  \item Paraphrase: $A$ and $B$ entail each other
    \begin{itemize}[leftmargin=*]\vspace{3pt}
      \item  \textit{He killed it} $\Leftrightarrow$ \textit{It was killed by him}
    \end{itemize}
  \item Presupposition: If $A$ presupposes $B$, then both $A$ and \textit{not $A$} still assume $B$\\(background assumption that survives negation)
    \begin{itemize}[leftmargin=*]\vspace{3pt}
      \item  \textit{Lee is (not) the current Prime Minister of Singapore} $\Rightarrow$ \textit{Singapore has a Prime Minister}
    \item Presupposition Triggers:
      \begin{itemize}[leftmargin=*]\vspace{3pt}
        \item Factive Verbs: \textit{I (didn't) realize she was sick} 
        \item Clefts: \textit{It (wasn't) my phone that exploded} 
        \item Temporal Clauses: \textit{She (didn't) call me before she went to dinner}
        \item Change of State: \textit{It (hasn't) stopped raining}
      \end{itemize}
    \end{itemize}
\end{enumerate}

\colbreak
\section{Pragmatics}
Pragmatics is the study of meaning in context.

Speech Acts:
\begin{enumerate}[\roman*.]
  \item Utterances convey meaning and make listeners do specific things 
    \begin{itemize}[leftmargin=*]\vspace{3pt}
      \item Not all utterances have truth values
      \item Saying something is doing something
      \item Different words can achieve the same outcome
      \item Utterances can be the same but with different intent or effect depending on context
    \end{itemize}
  \item Components:
    \begin{itemize}[leftmargin=*]\vspace{3pt}
      \item Locution: utterance
      \item Illocution: intention
      \item Perlocution: effect
    \end{itemize}
  \item Performatives: not true/false; change social reality under particular (felicity) conditions 
    \begin{itemize}[leftmargin=*]\vspace{3pt}
      \item \textit{I now pronounce you man and wife.}
      \item ``Hereby'' test: \textit{I hereby apologize} vs.\\\ *\textit{I hereby bought a phone}
    \end{itemize}
  \item Searle's Speech Act Classification:
    \begin{itemize}[leftmargin=*]\vspace{3pt}
      \item Representative: express beliefs (true/false)
      \item Directive: get the addressee to do something (incl. questions as a subtype)
      \item Commissive: commit speaker to future action
      \item Expressive: express emotional state
      \item Declaration: bring about a state of affairs
    \end{itemize}
\end{enumerate}

\colbreak
Conversational Implicature:
\begin{enumerate}[\roman*.]
  \item Cooperative Principle: safe assumption that interlocutors contribute appropriately to the accepted purpose/direction of the exchange
  \item Gricean Maxims:
    \begin{itemize}[leftmargin=*]\vspace{3pt}
      \item Quality: be truthful
      \item Quantity: be as informative as required
      \item Relation: be relevant
      \item Manner: be clear and orderly
    \end{itemize}
  \item Flouting Maxims:
    \begin{itemize}[leftmargin=*]\vspace{3pt}
      \item Generates implications via cooperative principle
      \begin{itemize}[leftmargin=*]\vspace{3pt}
        \item Context-dependent 
        \item Cancellable 
      \end{itemize}
      \item Quantity (under-informative):
        \begin{itemize}[leftmargin=*]\vspace{2pt}
          \item A: \textit{Did you contact the prof and the tutor?}\\
                B: \textit{I contacted the tutor.}
          \item Implicature: B did not contact the prof 
        \end{itemize}
      \item Quality (sarcasm):
        \begin{itemize}[leftmargin=*]\vspace{2pt}
          \item A: \textit{Do I look okay in this dress?}\\
                B: \textit{No, it's so hideous.}
          \item Implicature: B thinks A knows perfectly she looks great (B's true views are obvious to A)
        \end{itemize}
      \item Relation (surface irrelevance):
        \begin{itemize}[leftmargin=*]\vspace{2pt}
          \item A: \textit{Can we meet up on Friday?}\\
                B: \textit{My sister wants me to go shopping with her.}
          \item Implicature: B can't meet up on Friday
        \end{itemize}
      \item Manner (obscuring / lack of clarity):
        \begin{itemize}[leftmargin=*]\vspace{2pt}
          \item Journalist: \textit{...what have you got wrong, so that you get it right next time?}\\
                Boris: \textit{I think, Laura, when you look back...}
          \item Implicature: the speaker does not wish to cooperate with the narrative implied
        \end{itemize}
    \end{itemize}
  \item Tests:
    \begin{itemize}[leftmargin=*]\vspace{3pt}
      \item ``and not'' test: implicatures are cancellable
        \begin{itemize}[leftmargin=*]\vspace{2pt}
          \item A: \textit{My sister has been to Taiwan, and I have too.}
        \end{itemize}
      \item ``but'' test: can make the implicature explicit
        \begin{itemize}[leftmargin=*]\vspace{2pt}
          \item \textit{My sister has been to Taiwan, but I haven't.}
        \end{itemize}
    \end{itemize}
\end{enumerate}
\colbreak

Politeness Theory:
\begin{enumerate}[\roman*.]
  \item Politeness: interacting harmoniously; avoiding conflict/offense (culture- and context-dependent)
  \item Politeness Strategies:
    \begin{itemize}[leftmargin=*]\vspace{3pt}
      \item Positive Politeness: show friendliness/closeness
      \item Negative Politeness: avoid offense with deference 
    \end{itemize}
  \item Face Wants:
    \begin{itemize}[leftmargin=*]\vspace{3pt}
      \item Positive Face: want to be liked/approved of
      \item Negative Face: want freedom of action; freedom from imposition
    \end{itemize}
  \item Face-Threatening Acts (FTAs)
    \begin{itemize}[leftmargin=*]\vspace{3pt}
      \item Threaten addressee: criticism (positive face), orders (negative face)
      \item Threaten speaker: apologizing (positive face), thanking (negative face)
    \end{itemize}
  \item Mitigating FTAs (examples):
    \begin{itemize}[leftmargin=*]\vspace{3pt}
      \item No mitigation: \textit{You've got dirt on your nose.}
      \item Positive politeness: \textit{This is really interesting, but I have to get going.}
      \item Negative politeness: \textit{I'm so sorry to interrupt, but I have to get going.}
      \item Off-record hint: \textit{Wow, look at the time.}
    \end{itemize}
  \item Strategy Choice Factors:
    \begin{itemize}[leftmargin=*]\vspace{3pt}
      \item Social distance
      \item Power difference
      \item Cost of imposition
    \end{itemize}
\end{enumerate}
\colbreak
\section{Sociolinguistics}
Sociolinguistics is the study of how language functions in society.

Variation in Language:
\begin{enumerate}[\roman*.]
  \item Accent: difference in pronunciation
  \item Dialect: difference in pronunciation-related features, lexicon and grammatical structure
  \item Mutual Intelligibility:
    \begin{itemize}[leftmargin=*]\vspace{3pt}
      \item Not straightforward: can be asymmetric; \\shaped by experience and attitudes
      \item Sociopolitical factors often determine whether varieties are treated as ``languages'' vs.\ ``dialects''
      \item ``A language is a dialect with an army and navy.''
    \end{itemize}
  \item Sociolinguistic Knowledge: native speakers know how to use language appropriately in different situations 
  \item Language Use Factors:
    \begin{itemize}[leftmargin=*]\vspace{3pt}
      \item Linguistic: feature environment 
      \item Social: difference in speakers
      \item Stylistic: choice across situations
    \end{itemize}
\end{enumerate}

Non-standard Language:
\begin{enumerate}[\roman*.]
  \item Language Attitudes:
    \begin{itemize}[leftmargin=*]\vspace{3pt}
      \item Status: standard speakers rated ``intelligent/wealthy/educated''
      \item Solidarity: non-standard speakers rated ``friendly/trustworthy/kind''
    \end{itemize}
  \item Regional features can index local identity and become commodified
  \item Standard varieties are not inherently better:
    \begin{itemize}[leftmargin=*]\vspace{3pt}
      \item Misconception: non-standard language is ``random/illogical/simplified''
      \item All varieties are equally rule-governed, expressive
      \item Complexity is not owned by the standard
      \item Stigma is social: prestige tracks the social status of the people who use a variant
    \end{itemize}
\end{enumerate}
\colbreak

Identity and Style:
\begin{enumerate}[\roman*.]
  \item Co-constitutive: language reflects identity \textit{and} language helps construct identity
  \item Speaker-focused approach: speakers actively construct styles using linguistic + other resources
  \item Personae: recognizable community identities with conventional attributes
    \begin{itemize}[leftmargin=*]\vspace{3pt}
      \item Ex. SG: Ah Beng, Taxi Uncle, XMM, Auntie
    \end{itemize}
  \item Indexing and style-building:
    \begin{itemize}[leftmargin=*]\vspace{3pt}
      \item Features become linked with social meanings; a single feature can index many meanings
      \item People mix-and-match features to create new styles
    \end{itemize}
\end{enumerate}

Language Change:
\begin{enumerate}[\roman*.]
  \item Diachronic Linguistics: language change across time
  \item Synchronic Linguistics: language at a time-slice
  \item Languages change incrementally; studied via:
    \begin{itemize}[leftmargin=*]\vspace{3pt}
      \item Real-time: compare data across time periods
      \item Apparent-time: compare age groups at one time
    \end{itemize}
\end{enumerate}

Language Contact:
\begin{enumerate}[\roman*.]
  \item Borrowing: lexical/structural influence
  \item Code-switching: structured alternation in discourse
  \item Contact Languages:
    \begin{itemize}[leftmargin=*]\vspace{3pt}
      \item Pidgin: reduced system for communication between groups without shared language;\\no native speakers
      \item Creole: stable language that develops from a pidgin and becomes a native language
    \end{itemize}
\end{enumerate}

Language and Justice:
\begin{enumerate}[\roman*.]
  \item Linguistics can be applied to injustice (e.g.\ forensic linguistics; discrimination; institutional bias)
  \item Accent-based discrimination: differential treatment by accent (e.g.\ housing enquiries)
  \item Language and police bias: body-cam analysis used to study differences in officer speech
\end{enumerate}
\colbreak
\section{Psycholinguistics}

Psycholinguistics is the study of the psychology of language.

Language Acquisition:
\begin{enumerate}[\roman*.]
  \item Studying child language is hard $\Rightarrow$ specialized methods (e.g.\ head-turn preference)
  \item Early perception:
    \begin{itemize}[leftmargin=*]\vspace{3pt}
      \item Newborns: familiar with rhythms/sounds of caregiver language(s) (womb exposure)
      \item Perceptual narrowing: perception becomes specialized to native phonology
        \begin{itemize}[leftmargin=*]\vspace{2pt}
          \item 6 mo: can hear non-native contrasts (e.g.\ /r/ vs.\ /l/ for Japanese)
          \item 12 mo: reduced sensitivity to non-contrastive differences
        \end{itemize}
    \end{itemize}
  \item Milestones (typical):
    \begin{itemize}[leftmargin=*]\vspace{3pt}
      \item 6 mo: babbling; responds to name
      \item 12 mo: one-word/holophrastic; understands simple instructions
      \item 18--24 mo: rapid vocab growth; two-word combinations
      \item 3--5 yrs: multiword sentences; large receptive/expressive vocab; all vowels + most consonants
      \item 12 yrs: receptive $\sim$20k--50k
    \end{itemize}
  \item Learning problems/strategies:
    \begin{itemize}[leftmargin=*]\vspace{3pt}
      \item Mapping problem: many words/day; few directly taught
      \item Whole object assumption (``gavagai'')
      \item Comprehension $>$ production: \textit{fis} phenomenon
    \end{itemize}
\end{enumerate}

\colbreak
Acquisition Evidence:
\begin{enumerate}[\roman*.]
  \item Error patterns:
    \begin{itemize}[leftmargin=*]\vspace{3pt}
      \item Overregularization: irregulars $\rightarrow$ rule over-application (go $\rightarrow$ \textit{goed}; mouse $\rightarrow$ \textit{mouses})
      \item Suggests active rule-building (Go + \textit{-ed} = \textit{goed})
    \end{itemize}
  \item Universal Grammar + poverty of the stimulus:
    \begin{itemize}[leftmargin=*]\vspace{3pt}
      \item Claim: input lacks enough evidence 
      \item Example:
        \begin{itemize}[leftmargin=*]\vspace{2pt}
          \item \textit{Anyone who is interested can see me later.}
          \item \textit{Can anyone who is interested see me later?}
          \item *\textit{Is anyone who interested can see me later?}
        \end{itemize}
    \end{itemize}
  \item A critical period?
    \begin{itemize}[leftmargin=*]\vspace{3pt}
      \item Idea: sensitive window (duck imprinting)
      \item Evidence: age-of-arrival vs.\ grammar performance (e.g.\ Johnson \& Newport 1989)
      \item Critics: gradual decline; children still take years; adults have less time
    \end{itemize}
\end{enumerate}

Language in the Brain:
\begin{enumerate}[\roman*.]
  \item Methods:
    \begin{itemize}[leftmargin=*]\vspace{3pt}
      \item Atypical/impaired brains (e.g.\ stroke)
      \item Typical brains (direct imaging, e.g.\ fMRI)
      \item Indirect evidence from cognition (e.g.\ speech errors: \textit{``Three cheers for our queer old dean!''})
    \end{itemize}
  \item Localization (left hemisphere):
    \begin{itemize}[leftmargin=*]\vspace{3pt}
      \item Broca's area: important for speech production
      \item Broca's aphasia: comprehension relatively intact; difficulty producing speech/syntax
      \item Wernicke's area: associated with comprehension
      \item Wernicke's aphasia: fluent syntax but often nonsensical; frequent word substitutions
    \end{itemize}
  \item Beyond Broca/Wernicke:
    \begin{itemize}[leftmargin=*]\vspace{3pt}
      \item Language mostly left-lateralized, but right hemisphere crucial for some tasks (e.g.\ intonation/emotion)
      \item Reading Chinese characters involves both hemispheres more than alphabetic reading
      \item Many additional areas contribute\\(e.g.\ switching between languages)
    \end{itemize}
\end{enumerate}

Language Production and Processing:
\begin{enumerate}[\roman*.]
  \item Automatic processing can interfere with responses:
    \begin{itemize}[leftmargin=*]\vspace{3pt}
      \item Stroop effect: word meaning interferes with naming ink color
    \end{itemize}
  \item Multimodal integration:
    \begin{itemize}[leftmargin=*]\vspace{3pt}
      \item McGurk effect: audio [ba] + visual [ga] $\Rightarrow$ many hear [da]
    \end{itemize}
  \item Modeling Production with connectionist networks 
    \begin{itemize}[leftmargin=*]\vspace{3pt}
      \item Nodes + spreading activation
      \item Most activated node is selected as output
      \item Ex.: semantic features of \textit{cat} $\rightarrow$ lexical CAT (vs.\ DOG) $\rightarrow$ phonological /k/ /a/ /t/
    \end{itemize}
\end{enumerate}

Language and Thought:
\begin{enumerate}[\roman*.]
  \item Competing views:
    \begin{itemize}[leftmargin=*]\vspace{3pt}
      \item Linguistic determinism (strong Sapir-Whorf): can only conceive what language can describe
      \item Linguistic relativism (weak Sapir-Whorf): some cognition is influenced by language
      \item Linguistic universalism: thinking is not affected by language differences (universal metalanguage)
    \end{itemize}
  \item Thinking for speaking: speakers attend to distinctions their language requires
    \begin{itemize}[leftmargin=*]\vspace{3pt}
      \item Color terms:
        \begin{itemize}[leftmargin=*]\vspace{2pt}
          \item Japanese: one term can cover ``blue''/``green''
          \item Russian: green vs.\ light blue vs.\ dark blue
        \end{itemize}
    \end{itemize}
  \item Metaphorical Mappings:
    \begin{itemize}[leftmargin=*]\vspace{3pt}
      \item Space $\rightarrow$ time mappings vary; influence cognition
      \item Horizontal Metaphors (English/Chinese):\\\textit{looking back/behind} vs.\ \textit{looking forward/ahead}
      \item Vertical Metaphors (Chinese): (up month = last month), (down month = next month)
      \item Boroditsky (2001): English faster after horizontal tasks; Chinese faster after horizontal \textit{or} vertical tasks
      \item Trainability: English speakers trained with ``Monday is above Tuesday'' shift toward the Mandarin-like pattern
    \end{itemize}
\end{enumerate}
\end{multicols*}
%%%%%%%%%%%%%%%%%%%%%%%%%%%%%%%%%%%%%%%%%%%%%%%%%%%%%%
%                       End                          %
%%%%%%%%%%%%%%%%%%%%%%%%%%%%%%%%%%%%%%%%%%%%%%%%%%%%%%
\end{document}

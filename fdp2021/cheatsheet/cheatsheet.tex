\documentclass[12pt, a4paper]{article}

\usepackage[utf8]{inputenc}
\usepackage[mathscr]{euscript}
\let\euscr\mathscr \let\mathscr\relax
\usepackage[scr]{rsfso}
\usepackage{amssymb,amsmath,amsthm,amsfonts}
\usepackage[shortlabels]{enumitem}
\usepackage{multicol,multirow}
\usepackage{lipsum}
\usepackage{balance}
\usepackage{calc}
\usepackage[colorlinks=true,citecolor=blue,linkcolor=blue]{hyperref}
\usepackage{import}
\usepackage{xifthen}
\usepackage{pdfpages}
\usepackage{transparent}
\usepackage{listings}

\newcommand{\incfig}[2][1.0]{
    \def\svgwidth{#1\columnwidth}
    \import{./figures/}{#2.pdf_tex}
}

\newlist{enumproof}{enumerate}{4}
\setlist[enumproof,1]{label=\arabic*., parsep=1em}
\setlist[enumproof,2]{label=\arabic{enumproofi}.\arabic*., parsep=1em}
\setlist[enumproof,3]{label=\arabic{enumproofi}.\arabic{enumproofii}.\arabic*., parsep=1em}
\setlist[enumproof,4]{label=\arabic{enumproofi}.\arabic{enumproofii}.\arabic{enumproofiii}.\arabic*., parsep=1em}

\renewcommand{\qedsymbol}{\ensuremath{\blacksquare}}

\lstdefinestyle{mystyle}{
  language=C, % Set the language to C
  commentstyle=\color{codegray}, % Color for comments
  keywordstyle=\color{orange}, % Color for basic keywords
  stringstyle=\color{mauve}, % Color for strings
  basicstyle={\ttfamily\footnotesize}, % Basic font style
  breakatwhitespace=false,         
  breaklines=true,                 
  captionpos=b,                    
  keepspaces=true,                 
  numbers=none,                    
  tabsize=2,
  morekeywords=[2]{\#include, \#define, \#ifdef, \#ifndef, \#endif, \#pragma, \#else, \#elif}, % Preprocessor directives
  keywordstyle=[2]\color{codegreen}, % Style for preprocessor directives
  morekeywords=[3]{int, char, float, double, void, struct, union, enum, const, volatile, static, extern, register, inline, restrict, _Bool, _Complex, _Imaginary, size_t, ssize_t, FILE}, % C standard types and common identifiers
  keywordstyle=[3]\color{identblue}, % Style for types and common identifiers
  morekeywords=[4]{printf, scanf, fopen, fclose, malloc, free, calloc, realloc, perror, strtok, strncpy, strcpy, strcmp, strlen}, % Standard library functions
  keywordstyle=[4]\color{cyan}, % Style for library functions
}

\usepackage{ifthen}
\usepackage[landscape]{geometry}
\usepackage[shortlabels]{enumitem}

\ifthenelse{\lengthtest { \paperwidth = 11in}}
    { \geometry{top=.5in,left=.5in,right=.5in,bottom=.5in} }
	{\ifthenelse{ \lengthtest{ \paperwidth = 297mm}}
		{\geometry{top=1cm,left=1cm,right=1cm,bottom=1cm} }
		{\geometry{top=1cm,left=1cm,right=1cm,bottom=1cm} }
	}

\pagestyle{empty}
\makeatletter
\renewcommand\thesection{\arabic{section}.}
\renewcommand{\section}{\@startsection{section}{1}{0mm}%
                                {-1ex plus -.5ex minus -.2ex}%
                                {0.05ex}%x
                                {\normalfont\normalsize\bfseries}}
\renewcommand{\subsection}{\@startsection{subsection}{2}{0mm}%
                                {-1ex plus -.5ex minus -.2ex}%
                                {0.05ex}%
                                {\normalfont\small\bfseries}}
\renewcommand{\subsubsection}{\@startsection{subsubsection}{3}{0mm}%
                                {-1ex plus -.5ex minus -.2ex}%
                                {0.05ex}%
                                {\normalfont\footnotesize\bfseries}}
\newcommand{\colbreak}{\vfill\null\columnbreak}
\makeatother
\setcounter{secnumdepth}{1}
\setlength{\parindent}{0pt}
\setlength{\parskip}{0.7em}

\setlist[itemize]{itemsep=0.6ex, topsep=-2pt, partopsep=0pt, parsep=0pt}
\setlist[enumerate]{itemsep=0.6ex, topsep=-2pt, partopsep=0pt, parsep=0pt}

% Things Lie
\newcommand{\kb}{\mathfrak b}
\newcommand{\kg}{\mathfrak g}
\newcommand{\kh}{\mathfrak h}
\newcommand{\kn}{\mathfrak n}
\newcommand{\ku}{\mathfrak u}
\newcommand{\kz}{\mathfrak z}
\DeclareMathOperator{\Ext}{Ext} % Ext functor
\DeclareMathOperator{\Tor}{Tor} % Tor functor
\newcommand{\gl}{\opname{\mathfrak{gl}}} % frak gl group
\renewcommand{\sl}{\opname{\mathfrak{sl}}} % frak sl group chktex 6

% More script letters etc.
\newcommand{\SA}{\mathcal A}
\newcommand{\SB}{\mathcal B}
\newcommand{\SC}{\mathcal C}
\newcommand{\SF}{\mathcal F}
\newcommand{\SG}{\mathcal G}
\newcommand{\SH}{\mathcal H}
\newcommand{\OO}{\mathcal O}

\newcommand{\SCA}{\mathscr A}
\newcommand{\SCB}{\mathscr B}
\newcommand{\SCC}{\mathscr C}
\newcommand{\SCD}{\mathscr D}
\newcommand{\SCE}{\mathscr E}
\newcommand{\SCF}{\mathscr F}
\newcommand{\SCG}{\mathscr G}
\newcommand{\SCH}{\mathscr H}

% Mathfrak primes
\newcommand{\km}{\mathfrak m}
\newcommand{\kp}{\mathfrak p}
\newcommand{\kq}{\mathfrak q}

% number sets
\newcommand{\RR}[1][]{\ensuremath{\ifstrempty{#1}{\mathbb{R}}{\mathbb{R}^{#1}}}}
\newcommand{\NN}[1][]{\ensuremath{\ifstrempty{#1}{\mathbb{N}}{\mathbb{N}^{#1}}}}
\newcommand{\ZZ}[1][]{\ensuremath{\ifstrempty{#1}{\mathbb{Z}}{\mathbb{Z}^{#1}}}}
\newcommand{\QQ}[1][]{\ensuremath{\ifstrempty{#1}{\mathbb{Q}}{\mathbb{Q}^{#1}}}}
\newcommand{\CC}[1][]{\ensuremath{\ifstrempty{#1}{\mathbb{C}}{\mathbb{C}^{#1}}}}
\newcommand{\PP}[1][]{\ensuremath{\ifstrempty{#1}{\mathbb{P}}{\mathbb{P}^{#1}}}}
\newcommand{\HH}[1][]{\ensuremath{\ifstrempty{#1}{\mathbb{H}}{\mathbb{H}^{#1}}}}
\newcommand{\FF}[1][]{\ensuremath{\ifstrempty{#1}{\mathbb{F}}{\mathbb{F}^{#1}}}}
% expected value
\newcommand{\EE}{\ensuremath{\mathbb{E}}}
\newcommand{\charin}{\text{ char }}
\DeclareMathOperator{\sign}{sign}
\DeclareMathOperator{\Aut}{Aut}
\DeclareMathOperator{\Inn}{Inn}
\DeclareMathOperator{\Syl}{Syl}
\DeclareMathOperator{\Gal}{Gal}
\DeclareMathOperator{\GL}{GL} % General linear group
\DeclareMathOperator{\SL}{SL} % Special linear group

%---------------------------------------
% BlackBoard Math Fonts :-
%---------------------------------------

%Captital Letters
\newcommand{\bbA}{\mathbb{A}}	\newcommand{\bbB}{\mathbb{B}}
\newcommand{\bbC}{\mathbb{C}}	\newcommand{\bbD}{\mathbb{D}}
\newcommand{\bbE}{\mathbb{E}}	\newcommand{\bbF}{\mathbb{F}}
\newcommand{\bbG}{\mathbb{G}}	\newcommand{\bbH}{\mathbb{H}}
\newcommand{\bbI}{\mathbb{I}}	\newcommand{\bbJ}{\mathbb{J}}
\newcommand{\bbK}{\mathbb{K}}	\newcommand{\bbL}{\mathbb{L}}
\newcommand{\bbM}{\mathbb{M}}	\newcommand{\bbN}{\mathbb{N}}
\newcommand{\bbO}{\mathbb{O}}	\newcommand{\bbP}{\mathbb{P}}
\newcommand{\bbQ}{\mathbb{Q}}	\newcommand{\bbR}{\mathbb{R}}
\newcommand{\bbS}{\mathbb{S}}	\newcommand{\bbT}{\mathbb{T}}
\newcommand{\bbU}{\mathbb{U}}	\newcommand{\bbV}{\mathbb{V}}
\newcommand{\bbW}{\mathbb{W}}	\newcommand{\bbX}{\mathbb{X}}
\newcommand{\bbY}{\mathbb{Y}}	\newcommand{\bbZ}{\mathbb{Z}}

%---------------------------------------
% MathCal Fonts :-
%---------------------------------------

%Captital Letters
\newcommand{\mcA}{\mathcal{A}}	\newcommand{\mcB}{\mathcal{B}}
\newcommand{\mcC}{\mathcal{C}}	\newcommand{\mcD}{\mathcal{D}}
\newcommand{\mcE}{\mathcal{E}}	\newcommand{\mcF}{\mathcal{F}}
\newcommand{\mcG}{\mathcal{G}}	\newcommand{\mcH}{\mathcal{H}}
\newcommand{\mcI}{\mathcal{I}}	\newcommand{\mcJ}{\mathcal{J}}
\newcommand{\mcK}{\mathcal{K}}	\newcommand{\mcL}{\mathcal{L}}
\newcommand{\mcM}{\mathcal{M}}	\newcommand{\mcN}{\mathcal{N}}
\newcommand{\mcO}{\mathcal{O}}	\newcommand{\mcP}{\mathcal{P}}
\newcommand{\mcQ}{\mathcal{Q}}	\newcommand{\mcR}{\mathcal{R}}
\newcommand{\mcS}{\mathcal{S}}	\newcommand{\mcT}{\mathcal{T}}
\newcommand{\mcU}{\mathcal{U}}	\newcommand{\mcV}{\mathcal{V}}
\newcommand{\mcW}{\mathcal{W}}	\newcommand{\mcX}{\mathcal{X}}
\newcommand{\mcY}{\mathcal{Y}}	\newcommand{\mcZ}{\mathcal{Z}}

%---------------------------------------
% Bold Math Fonts :-
%---------------------------------------

%Captital Letters
\newcommand{\bmA}{\boldsymbol{A}}	\newcommand{\bmB}{\boldsymbol{B}}
\newcommand{\bmC}{\boldsymbol{C}}	\newcommand{\bmD}{\boldsymbol{D}}
\newcommand{\bmE}{\boldsymbol{E}}	\newcommand{\bmF}{\boldsymbol{F}}
\newcommand{\bmG}{\boldsymbol{G}}	\newcommand{\bmH}{\boldsymbol{H}}
\newcommand{\bmI}{\boldsymbol{I}}	\newcommand{\bmJ}{\boldsymbol{J}}
\newcommand{\bmK}{\boldsymbol{K}}	\newcommand{\bmL}{\boldsymbol{L}}
\newcommand{\bmM}{\boldsymbol{M}}	\newcommand{\bmN}{\boldsymbol{N}}
\newcommand{\bmO}{\boldsymbol{O}}	\newcommand{\bmP}{\boldsymbol{P}}
\newcommand{\bmQ}{\boldsymbol{Q}}	\newcommand{\bmR}{\boldsymbol{R}}
\newcommand{\bmS}{\boldsymbol{S}}	\newcommand{\bmT}{\boldsymbol{T}}
\newcommand{\bmU}{\boldsymbol{U}}	\newcommand{\bmV}{\boldsymbol{V}}
\newcommand{\bmW}{\boldsymbol{W}}	\newcommand{\bmX}{\boldsymbol{X}}
\newcommand{\bmY}{\boldsymbol{Y}}	\newcommand{\bmZ}{\boldsymbol{Z}}
%Small Letters
\newcommand{\bma}{\boldsymbol{a}}	\newcommand{\bmb}{\boldsymbol{b}}
\newcommand{\bmc}{\boldsymbol{c}}	\newcommand{\bmd}{\boldsymbol{d}}
\newcommand{\bme}{\boldsymbol{e}}	\newcommand{\bmf}{\boldsymbol{f}}
\newcommand{\bmg}{\boldsymbol{g}}	\newcommand{\bmh}{\boldsymbol{h}}
\newcommand{\bmi}{\boldsymbol{i}}	\newcommand{\bmj}{\boldsymbol{j}}
\newcommand{\bmk}{\boldsymbol{k}}	\newcommand{\bml}{\boldsymbol{l}}
\newcommand{\bmm}{\boldsymbol{m}}	\newcommand{\bmn}{\boldsymbol{n}}
\newcommand{\bmo}{\boldsymbol{o}}	\newcommand{\bmp}{\boldsymbol{p}}
\newcommand{\bmq}{\boldsymbol{q}}	\newcommand{\bmr}{\boldsymbol{r}}
\newcommand{\bms}{\boldsymbol{s}}	\newcommand{\bmt}{\boldsymbol{t}}
\newcommand{\bmu}{\boldsymbol{u}}	\newcommand{\bmv}{\boldsymbol{v}}
\newcommand{\bmw}{\boldsymbol{w}}	\newcommand{\bmx}{\boldsymbol{x}}
\newcommand{\bmy}{\boldsymbol{y}}	\newcommand{\bmz}{\boldsymbol{z}}

%---------------------------------------
% Scr Math Fonts :-
%---------------------------------------

\newcommand{\sA}{{\mathscr{A}}}   \newcommand{\sB}{{\mathscr{B}}}
\newcommand{\sC}{{\mathscr{C}}}   \newcommand{\sD}{{\mathscr{D}}}
\newcommand{\sE}{{\mathscr{E}}}   \newcommand{\sF}{{\mathscr{F}}}
\newcommand{\sG}{{\mathscr{G}}}   \newcommand{\sH}{{\mathscr{H}}}
\newcommand{\sI}{{\mathscr{I}}}   \newcommand{\sJ}{{\mathscr{J}}}
\newcommand{\sK}{{\mathscr{K}}}   \newcommand{\sL}{{\mathscr{L}}}
\newcommand{\sM}{{\mathscr{M}}}   \newcommand{\sN}{{\mathscr{N}}}
\newcommand{\sO}{{\mathscr{O}}}   \newcommand{\sP}{{\mathscr{P}}}
\newcommand{\sQ}{{\mathscr{Q}}}   \newcommand{\sR}{{\mathscr{R}}}
\newcommand{\sS}{{\mathscr{S}}}   \newcommand{\sT}{{\mathscr{T}}}
\newcommand{\sU}{{\mathscr{U}}}   \newcommand{\sV}{{\mathscr{V}}}
\newcommand{\sW}{{\mathscr{W}}}   \newcommand{\sX}{{\mathscr{X}}}
\newcommand{\sY}{{\mathscr{Y}}}   \newcommand{\sZ}{{\mathscr{Z}}}


%---------------------------------------
% Math Fraktur Font
%---------------------------------------

%Captital Letters
\newcommand{\mfA}{\mathfrak{A}}	\newcommand{\mfB}{\mathfrak{B}}
\newcommand{\mfC}{\mathfrak{C}}	\newcommand{\mfD}{\mathfrak{D}}
\newcommand{\mfE}{\mathfrak{E}}	\newcommand{\mfF}{\mathfrak{F}}
\newcommand{\mfG}{\mathfrak{G}}	\newcommand{\mfH}{\mathfrak{H}}
\newcommand{\mfI}{\mathfrak{I}}	\newcommand{\mfJ}{\mathfrak{J}}
\newcommand{\mfK}{\mathfrak{K}}	\newcommand{\mfL}{\mathfrak{L}}
\newcommand{\mfM}{\mathfrak{M}}	\newcommand{\mfN}{\mathfrak{N}}
\newcommand{\mfO}{\mathfrak{O}}	\newcommand{\mfP}{\mathfrak{P}}
\newcommand{\mfQ}{\mathfrak{Q}}	\newcommand{\mfR}{\mathfrak{R}}
\newcommand{\mfS}{\mathfrak{S}}	\newcommand{\mfT}{\mathfrak{T}}
\newcommand{\mfU}{\mathfrak{U}}	\newcommand{\mfV}{\mathfrak{V}}
\newcommand{\mfW}{\mathfrak{W}}	\newcommand{\mfX}{\mathfrak{X}}
\newcommand{\mfY}{\mathfrak{Y}}	\newcommand{\mfZ}{\mathfrak{Z}}
%Small Letters
\newcommand{\mfa}{\mathfrak{a}}	\newcommand{\mfb}{\mathfrak{b}}
\newcommand{\mfc}{\mathfrak{c}}	\newcommand{\mfd}{\mathfrak{d}}
\newcommand{\mfe}{\mathfrak{e}}	\newcommand{\mff}{\mathfrak{f}}
\newcommand{\mfg}{\mathfrak{g}}	\newcommand{\mfh}{\mathfrak{h}}
\newcommand{\mfi}{\mathfrak{i}}	\newcommand{\mfj}{\mathfrak{j}}
\newcommand{\mfk}{\mathfrak{k}}	\newcommand{\mfl}{\mathfrak{l}}
\newcommand{\mfm}{\mathfrak{m}}	\newcommand{\mfn}{\mathfrak{n}}
\newcommand{\mfo}{\mathfrak{o}}	\newcommand{\mfp}{\mathfrak{p}}
\newcommand{\mfq}{\mathfrak{q}}	\newcommand{\mfr}{\mathfrak{r}}
\newcommand{\mfs}{\mathfrak{s}}	\newcommand{\mft}{\mathfrak{t}}
\newcommand{\mfu}{\mathfrak{u}}	\newcommand{\mfv}{\mathfrak{v}}
\newcommand{\mfw}{\mathfrak{w}}	\newcommand{\mfx}{\mathfrak{x}}
\newcommand{\mfy}{\mathfrak{y}}	\newcommand{\mfz}{\mathfrak{z}}


\newcommand{\mytitle}{FDP2021 Special Physics Class 1, 2}
\newcommand{\myauthor}{github/omgeta}
\newcommand{\mydate}{AY 24/25 Sem 2 - 25/26 Sem 1}

\begin{document}
\raggedright
\footnotesize
\begin{multicols*}{3}
\setlength{\premulticols}{1pt}
\setlength{\postmulticols}{1pt}
\setlength{\multicolsep}{1pt}
\setlength{\columnsep}{2pt}

{\normalsize{\textbf{\mytitle}}} \\
{\footnotesize{\mydate\hspace{2pt}\textemdash\hspace{2pt}\myauthor}}
%%%%%%%%%%%%%%%%%%%%%%%%%%%%%%%%%%%%%%%%%%%%%%%%%%%%%%
%                      Begin                         %
%%%%%%%%%%%%%%%%%%%%%%%%%%%%%%%%%%%%%%%%%%%%%%%%%%%%%%
\vspace{-0.8em}
\section{Vector Algebra}
\subsection{Differential Calculus}
\textbf{Gradient} of scalar function $f$, $\nabla f$, is a vector rate of change of $f$ with maximum increase in the direction $\nabla f$:
\begin{enumerate}[\roman*.]
  \item $\nabla(fg) = f(\nabla g) + g(\nabla f)$
  \item $\nabla (\vec{v} \cdot \vec{w}) = \vec{v} \times (\nabla \times \vec{w}) + \vec{w} \times (\nabla \times \vec{v}) + (\vec{v}\cdot\nabla)\vec{w} + (\vec{w}\cdot \nabla)\vec{v}$
\end{enumerate}

\textbf{Divergence} of vector function $\vec{v}$, $\nabla \cdot \vec{v}$, is a scalar of how much $\vec{v}$ spreads out:
\begin{enumerate}[\roman*.]
  \item $\nabla \cdot (f \vec{v}) = f(\nabla \cdot \vec{v}) + \vec{v}\cdot(\nabla f)$
  \item $\nabla \cdot (\vec{v} \times \vec{w}) = \vec{w}\cdot(\nabla\times \vec{v}) - \vec{v}\cdot(\nabla\times \vec{w})$
\end{enumerate}

\textbf{Curl} of vector function $\vec{v}$, $\nabla \times \vec{v}$, is a vector of how much $\vec{v}$ curls around:
\begin{enumerate}[\roman*.]
  \item $\nabla \times (f \vec{v}) = f(\nabla \times \vec{v}) - \vec{v} \times (\nabla f)$
  \item $\nabla \times (\vec{v}\times \vec{w}) = (\vec{w}\cdot\nabla)\vec{v} - (\vec{v}\cdot\nabla)\vec{w} + \vec{v}(\nabla\cdot \vec{w})- \vec{w}(\nabla\cdot \vec{v})$
\end{enumerate}

\textbf{Laplacian} of scalar function $f$, $\nabla^2 f = \nabla \cdot \nabla f$, is a scalar. Other second derivatives are:
\begin{enumerate}[\roman*.]
  \item $\nabla \cdot (\nabla \times \vec{v}) = 0$
  \item $\nabla \times (\nabla f) = \vec{0}$
  \item $\nabla \times (\nabla \times \vec{v}) = \nabla(\nabla \cdot \vec{v}) - \nabla^2 \vec{v}$
\end{enumerate}
\subsection{Integral Calculus}

Line Integral: $\displaystyle\int^b_a \vec{v}\cdot d\vec{\ell}$, for small $d \vec{\ell}$ along line\\
Surface Integral: $\displaystyle\int_\mcS \vec{v}\cdot d\vec{a}$, for small $d \vec{a}$ surface normal\\
Volume Integral: $\displaystyle\int_\mcV f\,d\tau$, for small $d\tau$ volume

Fundamental Theorems:
\begin{enumerate}[\roman*.]
  \item $\displaystyle\int^{\vec{w}}_{\vec{v}} (\nabla f)\cdot d\vec{\ell} = f(\vec{w}) - f(\vec{v})$\hfill(Gradient)
  \item $\displaystyle\int(\nabla\cdot \vec{v})\,dV = \oint \vec{v}\cdot d \vec{a}$\hfill(Divergence/ Gauss's)
  \item $\displaystyle\int(\nabla\times \vec{v})\cdot d \vec{S} = \oint \vec{v}\cdot d \vec{\ell}$\hfill(Curl/ Stokes')
\end{enumerate}
\vspace{-1em}
\colbreak
\subsection{Coordinate Systems}

\textbf{Cartesian} ($x, y, z$):
\begin{align*}
  d\vec{\ell} &= \hat{x}\,dx + \hat{y}\,dy + \hat{z}\,dz,\quad d\tau = dx\,dy\,dz\\
  \nabla f &= \frac{\partial f}{\partial x}\hat{x} + \frac{\partial f}{\partial y}\hat{y} + \frac{\partial f}{\partial z}\hat{z}\\
  \nabla \cdot \vec{v} &= \frac{\partial v_x}{\partial x} + \frac{\partial v_y}{\partial y} + \frac{\partial v_z}{\partial z}\\
  \nabla \times \vec{v} &= \hat{x}(\frac{\partial v_z}{\partial y} - \frac{\partial v_y}{\partial z}) + \hat{y}(\frac{\partial v_x}{\partial z} - \frac{\partial v_z}{\partial x}) + \hat{z}(\frac{\partial v_y}{\partial x} - \frac{\partial v_x}{\partial y})\\
    \nabla^2f &= \frac{\partial^2f}{\partial x^2} + \frac{\partial^2f}{\partial y^2} + \frac{\partial^2f}{\partial z^2}
\end{align*}

\textbf{Spherical} ($r, \theta, \phi$) -- origin radius $r$, $z$-angle $\theta$, $xy$-angle $\phi$: 
\begin{align*}
  d\vec{\ell} &= \hat{r}\,dr 
        + \hat{\theta}\,(r\,d\theta)
        + \hat{\phi}\,(r\sin\theta\,d\phi),\hspace{0.5em} d\tau = r^{2}\sin\theta \;dr\,d\theta\,d\phi\\
  \nabla f &= \frac{\partial f}{\partial r}\hat{r} + \frac{1}{r}\frac{\partial f}{\partial \theta} \hat{\theta} + \frac{1}{r\sin\theta}\frac{\partial f}{\partial \phi}\hat{\phi}\\
  \nabla \cdot \vec{v} &= \frac{1}{r^2}\frac{\partial}{\partial r}(r^2v_r) + \frac{1}{r\sin\theta}\frac{\partial}{\partial \theta}(\sin\theta\,v_\theta) + \frac{1}{r\sin\theta}\frac{\partial v_{\phi}}{\partial\phi} \\
  \nabla \times \vec{v} &= \frac{1}{r\sin\theta}\left[\frac{\partial}{\partial\theta}(\sin\theta\,v_\phi)-\frac{\partial v_\theta}{\partial\phi} \right]\hat{r}\\
                        &\quad + \frac{1}{r}\left[\frac{1}{\sin\theta}\frac{\partial v_r}{\partial \phi} - \frac{\partial}{\partial r}(rv_\phi) \right] \hat{\theta} + \frac{1}{r}\left[ \frac{\partial}{\partial r}(rv_\theta) - \frac{\partial v_r}{\partial\theta} \right] \hat{\phi}\\ 
    \nabla^2f &= \frac{1}{r^2}\frac{\partial}{\partial r}(r^2 \frac{\partial f}{\partial r}) + \frac{1}{r^2\sin\theta} \frac{\partial}{\partial \theta}(\sin\theta \frac{\partial f}{\partial \theta}) + \frac{1}{r^2\sin^2\theta}\frac{\partial^2 f}{\partial\phi^2}
\end{align*}

\textbf{Cylindrical} ($s, \phi, z$) -- $z$-radius $s$, $xy$-angle $\phi$, height $z$:
\begin{align*}
  d\vec{\ell} &= \hat{s}\,ds 
        + \hat{\phi}\,(s\,d\phi)
        + \hat{z}\,dz,\quad d\tau = s \; ds\,d\phi\,dz\\
  \nabla f &= \frac{\partial f}{\partial s}\hat{s}
           + \frac{1}{s}\frac{\partial f}{\partial \phi}\hat{\phi}
           + \frac{\partial f}{\partial z}\hat{z}\\
  \nabla \cdot \vec{v} &= \frac{1}{s}\frac{\partial}{\partial s}(s\, v_{s})
                       + \frac{1}{s}\frac{\partial v_{\phi}}{\partial \phi}
                       + \frac{\partial v_{z}}{\partial z}\\
  \nabla \times \vec{v} &= \left[\frac{1}{s}\frac{\partial v_{z}}{\partial \phi}
                          - \frac{\partial v_{\phi}}{\partial z}\right]\hat{s}
                        + \left[\frac{\partial v_{s}}{\partial z}
                          - \frac{\partial v_{z}}{\partial s}\right]\hat{\phi}\\
                        &\quad + \left[\frac{1}{s}\frac{\partial}{\partial s}(s\, v_{\phi})
                          - \frac{1}{s}\frac{\partial v_{s}}{\partial \phi}\right]\hat{z}\\
  \nabla^{2} f &= \frac{1}{s}\frac{\partial}{\partial s}\!\left(s\,\frac{\partial f}{\partial s}\right)
               + \frac{1}{s^{2}}\frac{\partial^{2} f}{\partial \phi^{2}}
               + \frac{\partial^{2} f}{\partial z^{2}}
\end{align*}

\colbreak
\subsection{Standard Derivatives}
{\centering
\begin{tabular}{|c|c|}
\hline
$\mathbf{f(x)}$ & $\mathbf{f'(x)}$ \\ \hline
$\tan(g(x))$ & $g'(x)\sec^2(g(x))$ \\ \hline
$\sec(g(x))$ & $g'(x)\sec(g(x))\tan(g(x))$ \\ \hline
$\csc(g(x))$ & $-g'(x)\csc(g(x))\cot(g(x))$ \\ \hline
$\cot(g(x))$ & $-g'(x)\csc^2(g(x))$ \\ \hline
$\sin^{-1}(g(x))$ & $\frac{g'(x)}{\sqrt{1-g(x)^2}}$ \\ \hline
$\cos^{-1}(g(x))$ & $-\frac{g'(x)}{\sqrt{1-g(x)^2}}$ \\ \hline
$\tan^{-1}(g(x))$ & $\frac{g'(x)}{1+g(x)^2}$ \\ \hline
$\cot^{-1}(g(x))$ & $-\frac{g'(x)}{1+g(x)^2}$ \\ \hline
$\sec^{-1}(g(x))$ & $\frac{g'(x)}{|g(x)|\sqrt{g(x)^2 - 1}}, |g(x)|>1$ \\ \hline
$\csc^{-1}(g(x))$ & $-\frac{g'(x)}{|g(x)|\sqrt{g(x)^2 - 1}}, |g(x)|>1$ \\ \hline
$a^x$ & $a^x\ln(a)$ \\ \hline
\end{tabular}
\par}

\subsection{Standard Integrals}
{\centering
\begin{tabular}{|c|c|}
\hline
$\mathbf{f(x)}$ & $\mathbf{F(x) - C}$ \\ \hline
$[f(x)]^n,$ $n\neq -1$ & $\frac{[f(x)]^{n+1}}{(n+1)f'(x)} $\\ \hline
$\tan(f(x))$ & $\frac{1}{f'(x)}\ln|\sec(f(x))| $ \\ \hline
$\sec(f(x))$ & $\frac{1}{f'(x)}\ln|\sec(f(x)) + \tan(f(x))| $ \\ \hline
$\csc(f(x))$ & $-\frac{1}{f'(x)}\ln|\csc(f(x)) + \cot(f(x))| $ \\ \hline
$\cot(f(x))$ & $-\frac{1}{f'(x)}\ln|\csc(f(x))| $ \\ \hline
$\sec^2(f(x))$ & $\frac{1}{f'(x)}\tan(f(x)) $ \\ \hline
$\csc^2(f(x))$ & $-\frac{1}{f'(x)}\cot(f(x)) $ \\ \hline
$\sec(f(x))\tan(f(x))$ & $\frac{1}{f'(x)}\sec(f(x)) $ \\ \hline
$\csc(f(x))\cot(f(x))$ & $-\frac{1}{f'(x)}\csc(f(x)) $ \\ \hline
$\frac{1}{a^2+[f(x)]^2}$ & $\frac{1}{af'(x)}\tan^{-1}(\frac{f(x)}{a})$ \\ \hline
$\frac{1}{\sqrt{a^2-[f(x)]^2}}$ & $\frac{1}{f'(x)}\sin^{-1}(\frac{f(x)}{a})$ \\ \hline
$-\frac{1}{\sqrt{a^2-[f(x)]^2}}$ & $\frac{1}{f'(x)}\cos^{-1}(\frac{f(x)}{a})$ \\ \hline
$\frac{1}{a^2-[f(x)]^2}$ & $\frac{1}{2af'(x)}\ln|\frac{f(x)+a}{f(x)-a}|$ \\ \hline
$\frac{1}{[f(x)]^2-a^2}$ & $\frac{1}{2af'(x)}\ln|\frac{f(x)-a}{f(x)+a}|$ \\ \hline
$\frac{1}{\sqrt{[f(x)]^2+a^2}}$ & $\frac{1}{f'(x)}\ln|f(x)+\sqrt{[f(x)]^2+a^2}|$ \\ \hline
$\frac{1}{\sqrt{[f(x)]^2-a^2}}$ & $\frac{1}{f'(x)}\ln|f(x)+\sqrt{[f(x)]^2-a^2}|$ \\ \hline
$\sqrt{a^2-x^2}$ & $\frac{x}{2}\sqrt{a^2-x^2}+\frac{a^2}{2}\sin^{-1}(\frac{x}{a})$ \\ \hline
$\sqrt{x^2-a^2}$ & $\frac{x}{2}\sqrt{x^2-a^2}+\frac{a^2}{2}\ln|x+\sqrt{x^2-a^2}|$ \\ \hline
\end{tabular}
\par}

\colbreak
\section{Electrostatics}
Electrostatics involves stationary source charges and their properties. 

\subsection{Coulomb's Law}

Force $\vec{F}$ on test charge $Q$ at $\vec{r}$ by source charge $q$ at $\vec{r}\,'$ with separation vector $\vec{\sep} = \vec{r} - \vec{r}\,'$ is:
\begin{align*}
  \vec{F} = \frac{1}{4\pi\epsilon_0} \frac{qQ}{\sep^2} \hat{\sep}
\end{align*}
where $\epsilon_0 = 8.85 \times 10^{-12} \frac{C^2}{N\cdot m^2}$ is permittivity of free space.

By Principle of Superposition, interactions between any two charges are unaffected by presence of any others;\\$\therefore$ Force on $Q$ by point charges $q_1,\cdots ,q_n$ at $\vec{r}_1,\cdots,\vec{r}_n$ is:
\begin{align*}
  \vec{F} = \vec{F}_1 + \cdots + \vec{F}_n &= \frac{Q}{4\pi\epsilon_0}\sum^n_{i=1} \frac{q_i}{\sep_i^2} \hat{\sep}_i = Q \vec{E}
\end{align*}
where $\vec{E}(\vec{r})$ is the electric field of source charges denoting force per unit charge exerted on a test charge at $\vec{r}$. 

Line Charge: $\displaystyle \vec{E}(\vec{r}) = \frac{1}{4\pi\epsilon_0}\int \frac{\lambda (\vec{r}\,')}{\sep^2}\hat{\sep} \, d\ell'$\\
Surface Charge: $\displaystyle \vec{E}(\vec{r}) = \frac{1}{4\pi\epsilon_0}\int \frac{\sigma (\vec{r}\,')}{\sep^2}\hat{\sep} \, da'$\\
Volume Charge: $\displaystyle \vec{E}(\vec{r}) = \frac{1}{4\pi\epsilon_0}\int \frac{\rho (\vec{r}\,')}{\sep^2}\hat{\sep} \, d\tau'$

\subsection{Gauss's Law}

Integral: $\displaystyle\oint \vec{E}\cdot d \vec{a} = \frac{Q_{enc}}{\epsilon_0}$ for enclosed charge $Q_{enc}$\\
Differential: $\displaystyle \nabla \cdot \vec{E} = \frac{\rho}{\epsilon_0}$

\subsubsection{Symmetry}
Spherical (total $Q$): $\vec{E}(r)=\frac{1}{4\pi\epsilon_0}\frac{Q}{r^2}\hat r$ (out); $\vec{E}=\vec{0}$ (inside)\\
Cylindrical (line $\lambda$): $\vec{E}(\rho)=\frac{\lambda}{2\pi\epsilon_0\,\rho}\hat\rho$\\
Planar (plane $\sigma$): $|\vec{E}|=\frac{\sigma}{2\epsilon_0}$ on each side, normal.

\colbreak
\subsection{Potentials}
Potential $V$ at an electric field $\vec{E}$ is given by:
\begin{align*}
  V(\vec{r}) \equiv -\int^{\vec{r}}_{\infty} \vec{E} \cdot d \vec{\ell} \quad\text{or}\quad \vec{E} = -\nabla V
\end{align*}
and potential difference between $\vec{a}$ and $\vec{b}$ is:
\begin{align*}
  V(\vec{b}) - V(\vec{a}) = -\int^b_a \vec{E}\cdot d \vec{\ell}
\end{align*}
where principle of superposition applies, and in closed contour $\displaystyle\oint \vec{E}\cdot d \vec{\ell} = 0$ by conservative circulation.

Line Charge: $\displaystyle V(\vec{r}) = \frac{1}{4\pi\epsilon_0}\int \frac{\lambda (\vec{r}\,')}{\sep} \, d\ell'$\\
Surface Charge: $\displaystyle V(\vec{r}) = \frac{1}{4\pi\epsilon_0}\int \frac{\sigma (\vec{r}\,')}{\sep} \, da'$\\
Volume Charge: $\displaystyle V(\vec{r}) = \frac{1}{4\pi\epsilon_0}\int \frac{\rho (\vec{r}\,')}{\sep} \, d\tau'$\\

\subsubsection{Poisson's and Laplace's Equations}

Poisson's Equation: $\displaystyle \nabla^2 V = \frac{-\rho}{\epsilon_0}$

Laplace's Equation: $\displaystyle \nabla^2 V = 0$ (for region of $\rho = 0$ charge)

\subsection{Equipotential Surfaces}

Equipotential surface is surface with constant potential $V$:
\begin{enumerate}[\roman*.]
  \item Field $\vec{E}$ follows direction of decreasing potentials
  \item Equipotential surfaces are (by definition of gradient), orthogonal to field lines
  \item In particular, a plane of antisymmetry is always an equipotential surface
\end{enumerate}
{\centering
\incimg[0.7]{equipotential}
\par}
\colbreak
\subsection{Boundary Conditions}

Electric field $\vec{E}$ is discontinuous over a surface charge $\sigma$ by:
\begin{align*}
  \vec{E}_{\text{above}} - \vec{E}_{\text{below}} = \frac{\sigma}{\epsilon_0} \hat{n}
\end{align*}
where $\hat{n}$ is the normal to the surface boundary, since
\begin{align*}
    \vec{E}^{\perp}_{\text{above}} - \vec{E}^{\perp}_{\text{below}} = \frac{\sigma}{\epsilon_0} \land \vec{E}^{\parallel}_{\text{above}} = \vec{E}^{\parallel}_{\text{below}}
\end{align*}

Potential $V$ is continuous over a boundary since:
\begin{align*}
  V_{\text{above}} - V_{\text{below}} = -\int^b_a \vec{E}\cdot d \vec{\ell} \rightarrow 0 \text{ as } \ell \rightarrow 0
\end{align*}
but gradient of $V$ inherits discontinuity from $\vec{E} = -\nabla V$:
\begin{align*}
  \nabla V_{\text{above}} - \nabla V_{\text{below}} = -\frac{\sigma}{\epsilon_0} \hat{n} 
\end{align*}

\subsection{Work and Energy}
Work $W$ taken to move test charge $Q$ from $\vec{a}$ to $\vec{b}$ is:
\begin{align*}
  W &= \int^b_a \vec{F}\cdot d \vec{\ell} = -Q\int^b_a \vec{E}\cdot d\vec{\ell} = Q[V(\vec{b})- V(\vec{a})]
\end{align*}
or for system of point charges $q_1,\cdots,q_n$:
\begin{align*}
  W = \frac{1}{2}\sum^n_{i=1}q_i V(\vec{r}_i)
\end{align*}
Volume Charge: $\displaystyle W = \frac{1}{2}\int\rho V\, d\tau = \frac{\epsilon_0}{2}\int_{\text{all space}}E^2\, d\tau$

\subsection{Capacitors}
Capacitance $C$ measured in farads (F) is coloumb-per-volt of electric charge stored, given by:
\begin{align*}
  C = \frac{Q}{V}
\end{align*}
Parallel-Plate Capacitor: $\displaystyle C = \frac{A\epsilon_0}{d}$ for area $A$, distance $d$

\colbreak
\subsection{Electric Dipole}
Electric dipoles consist of charges $\pm q$ separated by $\vec{d}$ characterised by dipole moment, $\vec{p}$, given by:
\begin{align*}
  \vec{p} = q \vec{d}
\end{align*}
Potential $V_{\text{dip}}$ of a dipole is given by:
\begin{align*}
  V_{\text{dip}}(\vec r) = \frac{q}{4\pi\epsilon_0}\left(\frac{1}{\big|\vec r-\frac{\vec d}{2}\big|} -\frac{1}{\big|\vec r+\frac{\vec{d}}{2}\big|}\right) 
  \;\underset{r\gg d}{\simeq}\; \frac{\vec{p}\cdot \hat{r}}{4\pi\epsilon_0r^2}
\end{align*}
Electric field $\vec{E}_{\text{dip}}$ of a dipole is given by:
\begin{align*}
  \vec E_{\text{dip}}(\vec r) = \frac{q}{4\pi\epsilon_0}\left(
\frac{\vec r- \frac{\vec{d}}{2}}{\big|\vec r-\frac{\vec{d}}{2}\big|^{3}}
-\frac{\vec r+ \frac{\vec{d}}{2}}{\big|\vec r+ \frac{\vec{d}}{2}\big|^{3}}
\right)
\underset{r\gg d}{\simeq}\; \frac{3(\vec p\!\cdot\!\vec{r})\,\hat r-\vec p}{4\pi\epsilon_0r^4}
\end{align*}

\subsubsection{Under External Fields}
Torque $\vec{N}$ on a dipole $\vec{p}$ by uniform external field $\vec{E}$ is:
\begin{align*}
  \vec{N} = \vec{p} \times \vec{E} 
\end{align*}
Force $\vec{F}$ on a dipole $\vec{p}$ by non-uniform external field $\vec{E}$ is:
\begin{align*}
  \vec{F} = \nabla(\vec{p}\cdot \vec{E})
\end{align*}
Energy $U$ on a dipole $\vec{p}$ by external field $\vec{E}$ is:
\begin{align*}
  U = -\vec{p}\cdot \vec{E}
\end{align*}

\subsubsection{Induced Dipole}
Atoms can be polarized by external field $\vec{E}$ into a tiny dipole moment $\vec{p}$ given by:
\begin{align*}
  \vec{p} = \alpha \vec{E}
\end{align*}
where constant $\alpha$ is atomic polarizability of the atom.

\subsubsection{Multipole Expansion}
For general charge distributions, potential of a multipole is:
\begin{align*}
  V(\vec{r}) = \frac{1}{4\pi\epsilon_0}\left(\frac{q_0}{r}+ \frac{\vec{p}\cdot \hat{r}}{r^2} + \frac{Q_{ij}r_ir_j}{r^3}+\cdots\right)
\end{align*}
where $q_0$ is total charge, $\vec{p}$ is dipole moment, and $Q_{ij}$ is quadrupole moment tensor

\colbreak
\subsection{Examples}

\colbreak
\colbreak
\section{Magnetostatics}
Magnetostatics involves steady currents and their properties.

\subsection{Lorentz Force Law}
Force $\vec{F}_{\text{mag}}$ on charge $Q$ moving with velocity $\vec v$ in magnetic field $\vec B$ is:
\begin{align*}
  \vec{F}_{\text{mag}} = Q(\vec{v} \times \vec{B})
\end{align*}
Line Current: $\displaystyle \vec{F}_{\text{mag}} = \int \vec{I} \times \vec{B} \, d\ell = I \int d \vec{\ell} \times \vec{B}$, for $\vec{I} = \lambda \vec{v}$\\ 
Surface Current: $\displaystyle \vec{F}_{\text{mag}} = \int \vec{K} \times \vec{B} \, da$,\quad for $\vec{K} = \sigma \vec{v}$\\ 
Volume Current: $\displaystyle \vec{F}_{\text{mag}} = \int \vec{J} \times \vec{B} \, d\tau$,\quad for $\vec{J} = \rho \vec{v}$

Net force $\vec{F}$ on $Q$, in the presence of both electric and magnetic fields, is:
\begin{align*}
  \vec{F} = Q[\vec{E} + (\vec{v}\times \vec{B})]
\end{align*}


\subsection{Biot-Savart Law}

Line Current: $\displaystyle \vec{B}(\vec{r}) = \frac{\mu_0}{4\pi} \int \frac{\vec{I}(\vec{r}') \times \hat{\sep}}{\sep^2} d\ell'$\\ 
Surface Current: $\displaystyle \vec{B}(\vec{r}) = \frac{\mu_0}{4\pi} \int \frac{\vec{K}(\vec{r}') \times \hat{\sep}}{\sep^2} da'$\\
Volume Current: $\displaystyle \vec{B}(\vec{r}) = \frac{\mu_0}{4\pi} \int \frac{\vec{J}(\vec{r}') \times \hat{\sep}}{\sep^2} d\tau'$ 


\subsection{Amp\`ere's Law}

Differential: $\nabla \times \vec{B} = \mu_0 \vec{J}$

Integral: $\displaystyle\oint \vec{B}\cdot d \vec{\ell} = \mu_0 I_{enc}$ for enclosed current $I_{enc}$ 

\colbreak

\subsection{Vector Potentials}

Vector potential $\vec{A}$ at a magnetic field $\vec{B}$ is given by:
\begin{align*}
  \vec{B} = \nabla \times \vec{A}
\end{align*}
with the Coloumb gauge choice $\nabla \cdot \vec{A} = 0$ 

Line Current: $\displaystyle \vec{A}(\vec{r}) = \frac{\mu_0}{4\pi}\int \frac{\vec{I}(\vec{r})}{\sep} d\ell' = \frac{\mu_0}{4\pi I}\int \frac{1}{\sep} d\vec{\ell}\,'$\\
Surface Current: $\displaystyle \vec{A}(\vec{r}) = \frac{\mu_0}{4\pi}\int \frac{\vec{K}(\vec{r})}{\sep} da'$\\
Surface Current: $\displaystyle \vec{A}(\vec{r}) = \frac{\mu_0}{4\pi}\int \frac{\vec{J}(\vec{r})}{\sep} d\tau'$\\

\subsection{Poisson's Equation}

Poisson's Equation: $\nabla^2 \vec{A} = -\mu_0 \vec{J}$

\subsection{Boundary Conditions}

Magnetic field $\vec{B}$ is discontinuous over a surface current density $\vec{K}$ by:
\begin{align*}
  \vec{B}_{\text{above}} - \vec{B}_{\text{below}} = \mu_0(\vec{K} \times \hat{n})
\end{align*}
where $\hat{n}$ is the normal to the surface boundary, since
\begin{align*}
    \vec{B}^{\perp}_{\text{above}} = \vec{B}^{\perp}_{\text{below}} \land 
    \vec{B}^{\parallel}_{\text{above}} - \vec{B}^{\parallel}_{\text{below}} 
    = \mu_0 \vec{K} \times \hat{n}
\end{align*}

Vector potential $\vec{A}$ is continuous over a boundary since:
\begin{align*}
  \vec{A}_{\text{above}} - \vec{A}_{\text{below}} 
  = \int^b_a \vec{B}\cdot d \vec{S} \rightarrow 0 \text{ as } S \rightarrow 0
\end{align*}
and $\nabla \cdot \vec{A} = 0$ and $\nabla \times \vec{A} = \vec{B}$ guarantees the normal and tangential components are continuous respectively but derivative of $\vec{A}$ inherits discontinuity from $\vec{B}$:
\begin{align*}
  \frac{\partial \vec{A}_{\text{above}}}{\partial n} - \frac{\partial \vec{A}_{\text{below}}}{\partial n} = -\mu_0 \vec{K}
\end{align*}
\colbreak

\subsection{Magnetic Dipole}
Magnetic dipoles consist of a current loop of $I$ and area vector $\vec{a}$ characterised by dipole moment, $\vec{m}$, given by:
\begin{align*}
  \vec{m} = I\int d \vec{a} = I \vec{a}
\end{align*}
Vector potential $\vec{A}_{\text{dip}}$ of a dipole is given by:
\begin{align*}
  \vec{A}_{\text{dip}}(\vec r) = \frac{\mu_0(\vec{m}\times \vec{r})}{4\pi r^3}
\end{align*}
Magnetic field $\vec{B}_{\text{dip}}$ of a dipole is given by:
\begin{align*}
  \vec{B}_{\text{dip}}(\vec r) = \nabla \times \vec{A}_{\text{dip}}(\vec r)
  = \frac{\mu_0}{4\pi r^3}[3(\vec m\!\cdot\!\hat r)\hat r - \vec m]
\end{align*}

\subsubsection{Under External Fields}
Torque $\vec{N}$ on a dipole $\vec{m}$ by uniform external field $\vec{B}$ is:
\begin{align*}
  \vec{N} = \vec{m} \times \vec{B}
\end{align*}
Force $\vec{F}$ on a dipole $\vec{m}$ by non-uniform external field $\vec{B}$ is:
\begin{align*}
  \vec{F} = \nabla(\vec{m}\cdot \vec{B})
\end{align*}
Energy $U$ on a dipole $\vec{m}$ by external field $\vec{B}$ is:
\begin{align*}
  U = -\vec{m}\cdot \vec{B}
\end{align*}

\subsubsection{Induced Dipole}
Atoms can be magnetized by an external field $\vec{B}$ into a tiny dipole moment $\vec{m}$ given by:
\begin{align*}
  \vec{m} = \chi \vec{B}
\end{align*}
where constant $\chi$ is the magnetic susceptibility.

\subsubsection{Multipole Expansion}
For general current distributions, vector potential admits a multipole expansion:
\begin{align*}
  \vec{A}(\vec{r}) = \frac{\mu_0}{4\pi}\left(\frac{\vec{m}\times \hat r}{r^2} 
  + \frac{Q_{ij}^{(m)} r_j}{r^3} + \cdots\right)
\end{align*}
where $\vec{m}$ is dipole moment and $Q_{ij}^{(m)}$ is magnetic quadrupole tensor.

\colbreak
\section{Electrodynamics}
\subsection{Maxwell's Equations}
%%%%%%%%%%%%%%%%%%%%%%%%%%%%%%%%%%%%%%%%%%%%%%%%%%%%%%
%                       End                          %
%%%%%%%%%%%%%%%%%%%%%%%%%%%%%%%%%%%%%%%%%%%%%%%%%%%%%%
\end{multicols*}
\end{document}

\documentclass[12pt, a4paper]{article}

\usepackage[utf8]{inputenc}
\usepackage[mathscr]{euscript}
\let\euscr\mathscr \let\mathscr\relax
\usepackage[scr]{rsfso}
\usepackage{amssymb,amsmath,amsthm,amsfonts}
\usepackage[shortlabels]{enumitem}
\usepackage{multicol,multirow}
\usepackage{lipsum}
\usepackage{balance}
\usepackage{calc}
\usepackage[colorlinks=true,citecolor=blue,linkcolor=blue]{hyperref}
\usepackage{import}
\usepackage{xifthen}
\usepackage{pdfpages}
\usepackage{transparent}
\usepackage{tabularx}

\newcommand{\incfig}[2][1.0]{
    \def\svgwidth{#1\columnwidth}
    \import{./figures/}{#2.pdf_tex}
}
\newcommand{\incimg}[2][1.0]{
  \includegraphics[width=#1\columnwidth]{./figures/#2}
}


\usepackage{ifthen}
\usepackage[landscape]{geometry}
\usepackage[shortlabels]{enumitem}

\ifthenelse{\lengthtest { \paperwidth = 11in}}
    { \geometry{top=.5in,left=.5in,right=.5in,bottom=.5in} }
	{\ifthenelse{ \lengthtest{ \paperwidth = 297mm}}
		{\geometry{top=1cm,left=1cm,right=1cm,bottom=1cm} }
		{\geometry{top=1cm,left=1cm,right=1cm,bottom=1cm} }
	}

\pagestyle{empty}
\makeatletter
\renewcommand\thesection{\arabic{section}.}
\renewcommand{\section}{\@startsection{section}{1}{0mm}%
                                {-1ex plus -.5ex minus -.2ex}%
                                {0.05ex}%x
                                {\normalfont\normalsize\bfseries}}
\renewcommand{\subsection}{\@startsection{subsection}{2}{0mm}%
                                {-1ex plus -.5ex minus -.2ex}%
                                {0.05ex}%
                                {\normalfont\small\bfseries}}
\renewcommand{\subsubsection}{\@startsection{subsubsection}{3}{0mm}%
                                {-1ex plus -.5ex minus -.2ex}%
                                {0.05ex}%
                                {\normalfont\footnotesize\bfseries}}
\newcommand{\colbreak}{\vfill\null\columnbreak}
\makeatother
\setcounter{secnumdepth}{1}
\setlength{\parindent}{0pt}
\setlength{\parskip}{0.7em}

\setlist[itemize]{itemsep=0.6ex, topsep=-2pt, partopsep=0pt, parsep=0pt}
\setlist[enumerate]{itemsep=0.6ex, topsep=-2pt, partopsep=0pt, parsep=0pt}

\input{letterfonts}

\newcommand{\mytitle}{GEA1000 Quant. Reasoning with Data}
\newcommand{\myauthor}{github/omgeta}
\newcommand{\mydate}{AY 24/25 Sem 2}

\begin{document}
\raggedright
\footnotesize
\begin{multicols*}{3}
\setlength{\premulticols}{1pt}
\setlength{\postmulticols}{1pt}
\setlength{\multicolsep}{1pt}
\setlength{\columnsep}{2pt}

{\normalsize{\textbf{\mytitle}}} \\
{\footnotesize{\mydate\hspace{2pt}\textemdash\hspace{2pt}\myauthor}}\vspace{-1pt}
%%%%%%%%%%%%%%%%%%%%%%%%%%%%%%%%%%%%%%%%%%%%%%%%%%%%%%
%                      Begin                         %
%%%%%%%%%%%%%%%%%%%%%%%%%%%%%%%%%%%%%%%%%%%%%%%%%%%%%%
\section{Studying Data}
Population is a group of interest we have interest in. Population parameter is a numerical fact of the population. Census is a study of the complete population.

Sample is a subset of a population from a sampling frame. Sample statistic is a numeric fact of the sample. \\Estimates infer pop. parameters from sample statistics. 

Selection bias is caused by flawed sampling frame or non-probability sampling. Non-response bias is caused by systematic exclusion of subjects by unwillingness.

Probability sampling:
\begin{enumerate}[\roman*.]
  \item \textbf{Simple random sampling}.
  \item \textbf{Systematic sampling}: $k^{\text{th}}$ subject of each size $r$ component chosen.
  \item \textbf{Stratified sampling}: sampling from each strata, a subdivision of similar characteristic.
  \item \textbf{Cluster sampling}: sampling all members of sampled clusters, a natural subdivision.
\end{enumerate}

Non-probability sampling:
\begin{enumerate}[\roman*.]
  \item \textbf{Convenience sampling}: subjects chosen by convenience, introducing bias.
  \item \textbf{Volunteer sampling}: self-selected sample, usually with subjects off strong opinions, introducing bias.
\end{enumerate}

Study types:
\begin{enumerate}[\roman*.]
  \item \textbf{Experimental study}: observe dependent variable after direct manipulation of independent variable. Random treatment and control groups are similar. \\Shows cause-effect relationship.
  \item \textbf{Observational study}: observe variable of interest without manipulation of variables. \\Shows association, not necessarily cause-effect. 
\end{enumerate}

Generalizability depends on sampling frame size $\geq$ population, probability sampling, large sample size and little bias.
\colbreak
\section{Categorical Data Analysis}
Categorical variables are ordinal (naturally ordered) or nominal (no natural order).

\subsection{Rates}

When variables $A$, $B$ are not associated:
\begin{enumerate}[\roman*.]
  \item $\rate{A\mid B} = \rate{A\mid B'}$
\end{enumerate}

When variables $A$, $B$ are associated:
\begin{enumerate}[\roman*.]
  \item $\rate{A\mid B} > \rate{A\mid B'}\text{ and}$\\$\rate{A'\mid B'} > \rate{A'\mid B}$\hfill($+$ve)
  \item $\rate{A\mid B} < \rate{A\mid B'}\text{ and}$\\$\rate{A'\mid B'} < \rate{A'\mid B}$\hfill($-$ve)
\end{enumerate}

Symmetry Rules:
\begin{enumerate}[\roman*.]
  \item $\rate{A\mid B} > \rate{A\mid B'}$ \\$\iff \rate{B\mid A} > \rate{B\mid A'}$
  \item $\rate{A\mid B} < \rate{A\mid B'}$ \\$\iff \rate{B\mid A} < \rate{B\mid A'}$
  \item $\rate{A\mid B} = \rate{A\mid B'}$ \\$\iff \rate{B\mid A} = \rate{B\mid A'}$
\end{enumerate}

Basic Rule on Rates:
\begin{enumerate}[\roman*.]
  \item $\rate{A}$ lies between $\rate{A\mid B}$ and $\rate{A\mid B'}$
  \item As $\rate{B} \rightarrow 100\%$, $\rate{A} \rightarrow \rate{A\mid B}$
  \item $\rate{B} = 50\%$ \\$\implies \rate{A} = \frac{1}{2}[\rate{A\mid B} + \rate{A\mid B'}]$
  \item $\rate{A\mid B} = \rate{A\mid B'}$\\$\implies \rate{A} = \rate{A\mid B} = \rate{A\mid B'}$
\end{enumerate}

\subsection{Simpson's Paradox}
Simpson's paradox is the observation that a trend appearing in majority of the groups of the data disappears/reverses when the groups are combined.

\subsubsection{Confounders}
Confounder is a third variable associated with both the independent and dependent variable being investigated. Randomised assignment can help to remove an association, in order to remove the confounder.

\section{Numerical Data Analysis}
Numerical variables are discrete or continuous.

\subsection{Summary Statistics}
Mean, $\overline{x}$, is the average of variable $x$.\\
Mode is the most common element in variable $x$.\\
$Q_1$, Median, $Q_3$ are the ordered  $1^{\text{st}}$, $2^{\text{st}}$, $3^{\text{rd}}$ quarter element of variable $x$.

Sample variance, Var, of variable $x$ is given by: 
\begin{align*}
  \text{Var }= \frac{1}{n-1}\sum (x_i-\overline{x})^2
\end{align*}
Standard derivation, $s_x$, of variable $x$ is given by:
\begin{align*}
  s_x = \sqrt{\text{Var}}
\end{align*}
Coefficient of variance, $\displaystyle \frac{s_x}{\overline{x}}$, measures variance between different variables and has no units.

Median with $IQR = Q_3 - Q_1$ is preferred for asymmetrical distributions or when there are outliers.

Outliers satisfy one of the conditions:
\begin{enumerate}[\roman*.]
  \item $x > Q_3 + 1.5 \times IQR$
  \item $x < Q_1 - 1.5 \times IQR$
\end{enumerate}

\subsection{Univariate EDA}
\subsubsection{Histograms}
Histograms show data distribution, are better at greater frequencies and represent data points better.
Distributions with $n$ peaks are called $n$-modal.

Unimodal distribution shapes can be:
\begin{enumerate}[\roman*.]
  \item Symmetrical \hfill(mean = mode = median)
  \item Left-skewed \hfill(mean $<$ mode $<$ median)
  \item Right-skewed \hfill(mean $>$ mode $>$ median)
\end{enumerate}

Bell distributions are symmetrical with spread:
\begin{enumerate}[\roman*.]
  \item 68\% of data within 1 S.D.
  \item 95\% of data within 2 S.D.
\end{enumerate}
\colbreak
\subsubsection{Boxplots}
Boxplots side-by-side help compare distributions of different data sets, and are better to identify outliers.
They consist of minimum, $Q_1$, median, $Q_3$ and maximum.

Boxplot shapes can be:
\begin{enumerate}[\roman*.]
  \item Symmetrical \hfill($Q_1, Q_3$ equidistant to median)
  \item Left-skewed \hfill($Q_1$ closer to median)
  \item Right-skewed \hfill($Q_3$ closer to median)
\end{enumerate}

Boxplot spread for middle 50\% is given by $IQR$.

\subsection{Bivariate EDA}
Determinististic relationships determine exactly a variable given the value of the other variable.\\Association is a statistical relation describing average value of a variable given the value of the other variables

Correlation coefficient, $r$, is given by:
\begin{gather*}
  r = \frac{\text{Pop. covariance}}{\text{Pop. SD}_x \times \text{Pop. SD}_y} = \frac{\sum(x_i-\overline{x})(y_i-\overline{y})}{\sqrt{\sum(x_i-\overline{x})^2\cdot\sum(y_i-\overline{y})^2}}\\
  \text{*unaffected by swapping $x,y$ or adding/scaling by constant}
\end{gather*}
Direction, form and magnitude can be summarized by $r$:
\begin{enumerate}[\roman*.]
  \item $r > 0$\hfill($+$ve direction)
  \item $r < 0$\hfill($-$ve direction)
  \item $r=0$\hfill(Non-linear form)
  \item $0 < |r| < 0.3$\hfill(Weak association)
  \item $0.3 < |r| < 0.7$\hfill(Moderate association)
  \item $0.7 < |r| < 1$\hfill(Strong association)
\end{enumerate}

\subsection{Linear Regression}
Linear regression between variables believed to be linearly associated predicts the average value of the dependent variable given the independent variable.

Least squares regression line for predicting variable $Y$ given $X$ (and not vice versa) is given by:
\begin{align*}
  Y = mX + b,\quad m=\frac{s_Y}{s_X}r
\end{align*}
\colbreak

\section{Statistical Inference}
Probability of event $E$ in sample space $S$, $P(E)$, is given by:
\begin{enumerate}[\roman*.]
  \item $P(E) = \frac{|E|}{|S|}$, where $0 \leq P(E) \leq 1$
  \item $P(E') = 1 - P(E)$\hfill(Complement)
\end{enumerate}

Conditional probability of $B$ given $A$ is given by:
\begin{align*}
  P(B\mid A) = \displaystyle\frac{P(A\cap B)}{P(A)} = \frac{P(A\mid B)P(B)}{P(A)}
\end{align*}
Mutually exclusive events $A,B$ have special results:
\begin{enumerate}[\roman*.]
  \item $P(A\cap B) = 0$\hfill(Intersection)
  \item $P(A\cup B) = P(A) + P(B)$\hfill(Union)
  \item $A \cup B = S$\hfill(Total probability)\\$\implies P(C) = P(C\mid A)P(A) + P(C\mid B)P(B)$
\end{enumerate}

Independent events $A,B$ have special results:
\begin{enumerate}[\roman*.]
  \item $P(A\cap B) = P(A)\cdot P(B)$\hfill(Intersection)
  \item $P(A\mid B) = P(A)$\hfill(Conditional)
\end{enumerate}
\begin{align*}
  \text{Sensitivity} = \frac{\text{True Positives}}{\text{True Positives} + \text{False Negatives}}\\
  \text{Specificity} = \frac{\text{True Negatives}}{\text{True Negatives} + \text{False Positives}}
\end{align*}

\subsection{Fallacies}
Distribution fallacies:
\begin{enumerate}[\roman*.]
  \item \textbf{Ecological fallacy}: when we use aggregate level correlation to draw conclusions on individual data. 
  \item \textbf{Atomistic fallacy}: when we use individual level correlation to draw conclusions on group data. 
\end{enumerate}

Probability fallacies:
\begin{enumerate}[\roman*.]
  \item \textbf{Conjunction fallacy}: probability of two events occurring together is always less than of either event occurring alone.
  \item \textbf{Base rate fallacy}: ignoring the base rate of an event when calculating its probability.
\end{enumerate}
\colbreak
Relation between sample statistic and population parameter is given by:
\begin{align*}
  \text{Sample statistic = pop. parameter + bias + random error}
\end{align*}

\subsection{Confidence Intervals}
Confidence interval is a range of values likely to contain a population parameter at a certain confidence level.

Given a sample proportion $p^*$ and sample size $n$, confidence interval for population proportion is given by:
\begin{align*}
  p^* \pm z^* \times \sqrt{\frac{p^*(1-p^*)}{n}}
\end{align*}
where $z^*$ is the $z$-value for desired confidence level.

Given a sample mean $\overline{x}$, sample SD $s_x$ and sample size $n$, confidence interval for population mean is given by:
\begin{align*}
  \overline{x} \pm t^* \times \frac{s_x}{\sqrt{n}}
\end{align*}
where $t^*$ is the $t$-value for desired confidence level.

\subsection{Hypothesis Testing}
Hypothesis tests can be used for population proportion, population mean, and association, given a null hypothesis $H_0$, alternative hypothesis $H_1$, and significance value $a$.
For hypothesis test on association, we take:
\begin{enumerate}[\roman*.]
  \item $H_0$ there is no association
  \item $H_1$: there is an association.  
\end{enumerate}

$p$-value can be defined as:
\begin{enumerate}[\roman*.]
  \item Probability of obtaining a sample statistic as extreme or more extreme than the observed statistic, assuming $H_0$ is true.
  \item Smallest level of significance at which $H_0$ is rejected, assuming $H_0$ is true
\end{enumerate}
where we reject $H_0$ in favour of $H_1$ when $p$-value $< a$\\
and not reject $H_0$ (not implying truth) when $p$-value $\geq a$

%%%%%%%%%%%%%%%%%%%%%%%%%%%%%%%%%%%%%%%%%%%%%%%%%%%%%%
%                       End                          %
%%%%%%%%%%%%%%%%%%%%%%%%%%%%%%%%%%%%%%%%%%%%%%%%%%%%%%
\end{multicols*}
\end{document}

\documentclass[12pt, a4paper]{article}

\usepackage[utf8]{inputenc}
\usepackage[mathscr]{euscript}
\let\euscr\mathscr \let\mathscr\relax
\usepackage[scr]{rsfso}
\usepackage{amssymb,amsmath,amsthm,amsfonts}
\usepackage[shortlabels]{enumitem}
\usepackage{multicol,multirow}
\usepackage{lipsum}
\usepackage{balance}
\usepackage{calc}
\usepackage[colorlinks=true,citecolor=blue,linkcolor=blue]{hyperref}
\usepackage{import}
\usepackage{xifthen}
\usepackage{pdfpages}
\usepackage{transparent}
\usepackage{listings}

\newcommand{\incfig}[2][1.0]{
    \def\svgwidth{#1\columnwidth}
    \import{./figures/}{#2.pdf_tex}
}

\newlist{enumproof}{enumerate}{4}
\setlist[enumproof,1]{label=\arabic*., parsep=1em}
\setlist[enumproof,2]{label=\arabic{enumproofi}.\arabic*., parsep=1em}
\setlist[enumproof,3]{label=\arabic{enumproofi}.\arabic{enumproofii}.\arabic*., parsep=1em}
\setlist[enumproof,4]{label=\arabic{enumproofi}.\arabic{enumproofii}.\arabic{enumproofiii}.\arabic*., parsep=1em}

\renewcommand{\qedsymbol}{\ensuremath{\blacksquare}}

\lstdefinestyle{mystyle}{
  language=C, % Set the language to C
  commentstyle=\color{codegray}, % Color for comments
  keywordstyle=\color{orange}, % Color for basic keywords
  stringstyle=\color{mauve}, % Color for strings
  basicstyle={\ttfamily\footnotesize}, % Basic font style
  breakatwhitespace=false,         
  breaklines=true,                 
  captionpos=b,                    
  keepspaces=true,                 
  numbers=none,                    
  tabsize=2,
  morekeywords=[2]{\#include, \#define, \#ifdef, \#ifndef, \#endif, \#pragma, \#else, \#elif}, % Preprocessor directives
  keywordstyle=[2]\color{codegreen}, % Style for preprocessor directives
  morekeywords=[3]{int, char, float, double, void, struct, union, enum, const, volatile, static, extern, register, inline, restrict, _Bool, _Complex, _Imaginary, size_t, ssize_t, FILE}, % C standard types and common identifiers
  keywordstyle=[3]\color{identblue}, % Style for types and common identifiers
  morekeywords=[4]{printf, scanf, fopen, fclose, malloc, free, calloc, realloc, perror, strtok, strncpy, strcpy, strcmp, strlen}, % Standard library functions
  keywordstyle=[4]\color{cyan}, % Style for library functions
}

\usepackage{ifthen}
\usepackage[landscape]{geometry}
\usepackage[shortlabels]{enumitem}

\ifthenelse{\lengthtest { \paperwidth = 11in}}
    { \geometry{top=.5in,left=.5in,right=.5in,bottom=.5in} }
	{\ifthenelse{ \lengthtest{ \paperwidth = 297mm}}
		{\geometry{top=1cm,left=1cm,right=1cm,bottom=1cm} }
		{\geometry{top=1cm,left=1cm,right=1cm,bottom=1cm} }
	}

\pagestyle{empty}
\makeatletter
\renewcommand\thesection{\arabic{section}.}
\renewcommand{\section}{\@startsection{section}{1}{0mm}%
                                {-1ex plus -.5ex minus -.2ex}%
                                {0.05ex}%x
                                {\normalfont\normalsize\bfseries}}
\renewcommand{\subsection}{\@startsection{subsection}{2}{0mm}%
                                {-1ex plus -.5ex minus -.2ex}%
                                {0.05ex}%
                                {\normalfont\small\bfseries}}
\renewcommand{\subsubsection}{\@startsection{subsubsection}{3}{0mm}%
                                {-1ex plus -.5ex minus -.2ex}%
                                {0.05ex}%
                                {\normalfont\footnotesize\bfseries}}
\newcommand{\colbreak}{\vfill\null\columnbreak}
\makeatother
\setcounter{secnumdepth}{1}
\setlength{\parindent}{0pt}
\setlength{\parskip}{0.7em}

\setlist[itemize]{itemsep=0.6ex, topsep=-2pt, partopsep=0pt, parsep=0pt}
\setlist[enumerate]{itemsep=0.6ex, topsep=-2pt, partopsep=0pt, parsep=0pt}

% Things Lie
\newcommand{\kb}{\mathfrak b}
\newcommand{\kg}{\mathfrak g}
\newcommand{\kh}{\mathfrak h}
\newcommand{\kn}{\mathfrak n}
\newcommand{\ku}{\mathfrak u}
\newcommand{\kz}{\mathfrak z}
\DeclareMathOperator{\Ext}{Ext} % Ext functor
\DeclareMathOperator{\Tor}{Tor} % Tor functor
\newcommand{\gl}{\opname{\mathfrak{gl}}} % frak gl group
\renewcommand{\sl}{\opname{\mathfrak{sl}}} % frak sl group chktex 6

% More script letters etc.
\newcommand{\SA}{\mathcal A}
\newcommand{\SB}{\mathcal B}
\newcommand{\SC}{\mathcal C}
\newcommand{\SF}{\mathcal F}
\newcommand{\SG}{\mathcal G}
\newcommand{\SH}{\mathcal H}
\newcommand{\OO}{\mathcal O}

\newcommand{\SCA}{\mathscr A}
\newcommand{\SCB}{\mathscr B}
\newcommand{\SCC}{\mathscr C}
\newcommand{\SCD}{\mathscr D}
\newcommand{\SCE}{\mathscr E}
\newcommand{\SCF}{\mathscr F}
\newcommand{\SCG}{\mathscr G}
\newcommand{\SCH}{\mathscr H}

% Mathfrak primes
\newcommand{\km}{\mathfrak m}
\newcommand{\kp}{\mathfrak p}
\newcommand{\kq}{\mathfrak q}

% number sets
\newcommand{\RR}[1][]{\ensuremath{\ifstrempty{#1}{\mathbb{R}}{\mathbb{R}^{#1}}}}
\newcommand{\NN}[1][]{\ensuremath{\ifstrempty{#1}{\mathbb{N}}{\mathbb{N}^{#1}}}}
\newcommand{\ZZ}[1][]{\ensuremath{\ifstrempty{#1}{\mathbb{Z}}{\mathbb{Z}^{#1}}}}
\newcommand{\QQ}[1][]{\ensuremath{\ifstrempty{#1}{\mathbb{Q}}{\mathbb{Q}^{#1}}}}
\newcommand{\CC}[1][]{\ensuremath{\ifstrempty{#1}{\mathbb{C}}{\mathbb{C}^{#1}}}}
\newcommand{\PP}[1][]{\ensuremath{\ifstrempty{#1}{\mathbb{P}}{\mathbb{P}^{#1}}}}
\newcommand{\HH}[1][]{\ensuremath{\ifstrempty{#1}{\mathbb{H}}{\mathbb{H}^{#1}}}}
\newcommand{\FF}[1][]{\ensuremath{\ifstrempty{#1}{\mathbb{F}}{\mathbb{F}^{#1}}}}
% expected value
\newcommand{\EE}{\ensuremath{\mathbb{E}}}
\newcommand{\charin}{\text{ char }}
\DeclareMathOperator{\sign}{sign}
\DeclareMathOperator{\Aut}{Aut}
\DeclareMathOperator{\Inn}{Inn}
\DeclareMathOperator{\Syl}{Syl}
\DeclareMathOperator{\Gal}{Gal}
\DeclareMathOperator{\GL}{GL} % General linear group
\DeclareMathOperator{\SL}{SL} % Special linear group

%---------------------------------------
% BlackBoard Math Fonts :-
%---------------------------------------

%Captital Letters
\newcommand{\bbA}{\mathbb{A}}	\newcommand{\bbB}{\mathbb{B}}
\newcommand{\bbC}{\mathbb{C}}	\newcommand{\bbD}{\mathbb{D}}
\newcommand{\bbE}{\mathbb{E}}	\newcommand{\bbF}{\mathbb{F}}
\newcommand{\bbG}{\mathbb{G}}	\newcommand{\bbH}{\mathbb{H}}
\newcommand{\bbI}{\mathbb{I}}	\newcommand{\bbJ}{\mathbb{J}}
\newcommand{\bbK}{\mathbb{K}}	\newcommand{\bbL}{\mathbb{L}}
\newcommand{\bbM}{\mathbb{M}}	\newcommand{\bbN}{\mathbb{N}}
\newcommand{\bbO}{\mathbb{O}}	\newcommand{\bbP}{\mathbb{P}}
\newcommand{\bbQ}{\mathbb{Q}}	\newcommand{\bbR}{\mathbb{R}}
\newcommand{\bbS}{\mathbb{S}}	\newcommand{\bbT}{\mathbb{T}}
\newcommand{\bbU}{\mathbb{U}}	\newcommand{\bbV}{\mathbb{V}}
\newcommand{\bbW}{\mathbb{W}}	\newcommand{\bbX}{\mathbb{X}}
\newcommand{\bbY}{\mathbb{Y}}	\newcommand{\bbZ}{\mathbb{Z}}

%---------------------------------------
% MathCal Fonts :-
%---------------------------------------

%Captital Letters
\newcommand{\mcA}{\mathcal{A}}	\newcommand{\mcB}{\mathcal{B}}
\newcommand{\mcC}{\mathcal{C}}	\newcommand{\mcD}{\mathcal{D}}
\newcommand{\mcE}{\mathcal{E}}	\newcommand{\mcF}{\mathcal{F}}
\newcommand{\mcG}{\mathcal{G}}	\newcommand{\mcH}{\mathcal{H}}
\newcommand{\mcI}{\mathcal{I}}	\newcommand{\mcJ}{\mathcal{J}}
\newcommand{\mcK}{\mathcal{K}}	\newcommand{\mcL}{\mathcal{L}}
\newcommand{\mcM}{\mathcal{M}}	\newcommand{\mcN}{\mathcal{N}}
\newcommand{\mcO}{\mathcal{O}}	\newcommand{\mcP}{\mathcal{P}}
\newcommand{\mcQ}{\mathcal{Q}}	\newcommand{\mcR}{\mathcal{R}}
\newcommand{\mcS}{\mathcal{S}}	\newcommand{\mcT}{\mathcal{T}}
\newcommand{\mcU}{\mathcal{U}}	\newcommand{\mcV}{\mathcal{V}}
\newcommand{\mcW}{\mathcal{W}}	\newcommand{\mcX}{\mathcal{X}}
\newcommand{\mcY}{\mathcal{Y}}	\newcommand{\mcZ}{\mathcal{Z}}

%---------------------------------------
% Bold Math Fonts :-
%---------------------------------------

%Captital Letters
\newcommand{\bmA}{\boldsymbol{A}}	\newcommand{\bmB}{\boldsymbol{B}}
\newcommand{\bmC}{\boldsymbol{C}}	\newcommand{\bmD}{\boldsymbol{D}}
\newcommand{\bmE}{\boldsymbol{E}}	\newcommand{\bmF}{\boldsymbol{F}}
\newcommand{\bmG}{\boldsymbol{G}}	\newcommand{\bmH}{\boldsymbol{H}}
\newcommand{\bmI}{\boldsymbol{I}}	\newcommand{\bmJ}{\boldsymbol{J}}
\newcommand{\bmK}{\boldsymbol{K}}	\newcommand{\bmL}{\boldsymbol{L}}
\newcommand{\bmM}{\boldsymbol{M}}	\newcommand{\bmN}{\boldsymbol{N}}
\newcommand{\bmO}{\boldsymbol{O}}	\newcommand{\bmP}{\boldsymbol{P}}
\newcommand{\bmQ}{\boldsymbol{Q}}	\newcommand{\bmR}{\boldsymbol{R}}
\newcommand{\bmS}{\boldsymbol{S}}	\newcommand{\bmT}{\boldsymbol{T}}
\newcommand{\bmU}{\boldsymbol{U}}	\newcommand{\bmV}{\boldsymbol{V}}
\newcommand{\bmW}{\boldsymbol{W}}	\newcommand{\bmX}{\boldsymbol{X}}
\newcommand{\bmY}{\boldsymbol{Y}}	\newcommand{\bmZ}{\boldsymbol{Z}}
%Small Letters
\newcommand{\bma}{\boldsymbol{a}}	\newcommand{\bmb}{\boldsymbol{b}}
\newcommand{\bmc}{\boldsymbol{c}}	\newcommand{\bmd}{\boldsymbol{d}}
\newcommand{\bme}{\boldsymbol{e}}	\newcommand{\bmf}{\boldsymbol{f}}
\newcommand{\bmg}{\boldsymbol{g}}	\newcommand{\bmh}{\boldsymbol{h}}
\newcommand{\bmi}{\boldsymbol{i}}	\newcommand{\bmj}{\boldsymbol{j}}
\newcommand{\bmk}{\boldsymbol{k}}	\newcommand{\bml}{\boldsymbol{l}}
\newcommand{\bmm}{\boldsymbol{m}}	\newcommand{\bmn}{\boldsymbol{n}}
\newcommand{\bmo}{\boldsymbol{o}}	\newcommand{\bmp}{\boldsymbol{p}}
\newcommand{\bmq}{\boldsymbol{q}}	\newcommand{\bmr}{\boldsymbol{r}}
\newcommand{\bms}{\boldsymbol{s}}	\newcommand{\bmt}{\boldsymbol{t}}
\newcommand{\bmu}{\boldsymbol{u}}	\newcommand{\bmv}{\boldsymbol{v}}
\newcommand{\bmw}{\boldsymbol{w}}	\newcommand{\bmx}{\boldsymbol{x}}
\newcommand{\bmy}{\boldsymbol{y}}	\newcommand{\bmz}{\boldsymbol{z}}

%---------------------------------------
% Scr Math Fonts :-
%---------------------------------------

\newcommand{\sA}{{\mathscr{A}}}   \newcommand{\sB}{{\mathscr{B}}}
\newcommand{\sC}{{\mathscr{C}}}   \newcommand{\sD}{{\mathscr{D}}}
\newcommand{\sE}{{\mathscr{E}}}   \newcommand{\sF}{{\mathscr{F}}}
\newcommand{\sG}{{\mathscr{G}}}   \newcommand{\sH}{{\mathscr{H}}}
\newcommand{\sI}{{\mathscr{I}}}   \newcommand{\sJ}{{\mathscr{J}}}
\newcommand{\sK}{{\mathscr{K}}}   \newcommand{\sL}{{\mathscr{L}}}
\newcommand{\sM}{{\mathscr{M}}}   \newcommand{\sN}{{\mathscr{N}}}
\newcommand{\sO}{{\mathscr{O}}}   \newcommand{\sP}{{\mathscr{P}}}
\newcommand{\sQ}{{\mathscr{Q}}}   \newcommand{\sR}{{\mathscr{R}}}
\newcommand{\sS}{{\mathscr{S}}}   \newcommand{\sT}{{\mathscr{T}}}
\newcommand{\sU}{{\mathscr{U}}}   \newcommand{\sV}{{\mathscr{V}}}
\newcommand{\sW}{{\mathscr{W}}}   \newcommand{\sX}{{\mathscr{X}}}
\newcommand{\sY}{{\mathscr{Y}}}   \newcommand{\sZ}{{\mathscr{Z}}}


%---------------------------------------
% Math Fraktur Font
%---------------------------------------

%Captital Letters
\newcommand{\mfA}{\mathfrak{A}}	\newcommand{\mfB}{\mathfrak{B}}
\newcommand{\mfC}{\mathfrak{C}}	\newcommand{\mfD}{\mathfrak{D}}
\newcommand{\mfE}{\mathfrak{E}}	\newcommand{\mfF}{\mathfrak{F}}
\newcommand{\mfG}{\mathfrak{G}}	\newcommand{\mfH}{\mathfrak{H}}
\newcommand{\mfI}{\mathfrak{I}}	\newcommand{\mfJ}{\mathfrak{J}}
\newcommand{\mfK}{\mathfrak{K}}	\newcommand{\mfL}{\mathfrak{L}}
\newcommand{\mfM}{\mathfrak{M}}	\newcommand{\mfN}{\mathfrak{N}}
\newcommand{\mfO}{\mathfrak{O}}	\newcommand{\mfP}{\mathfrak{P}}
\newcommand{\mfQ}{\mathfrak{Q}}	\newcommand{\mfR}{\mathfrak{R}}
\newcommand{\mfS}{\mathfrak{S}}	\newcommand{\mfT}{\mathfrak{T}}
\newcommand{\mfU}{\mathfrak{U}}	\newcommand{\mfV}{\mathfrak{V}}
\newcommand{\mfW}{\mathfrak{W}}	\newcommand{\mfX}{\mathfrak{X}}
\newcommand{\mfY}{\mathfrak{Y}}	\newcommand{\mfZ}{\mathfrak{Z}}
%Small Letters
\newcommand{\mfa}{\mathfrak{a}}	\newcommand{\mfb}{\mathfrak{b}}
\newcommand{\mfc}{\mathfrak{c}}	\newcommand{\mfd}{\mathfrak{d}}
\newcommand{\mfe}{\mathfrak{e}}	\newcommand{\mff}{\mathfrak{f}}
\newcommand{\mfg}{\mathfrak{g}}	\newcommand{\mfh}{\mathfrak{h}}
\newcommand{\mfi}{\mathfrak{i}}	\newcommand{\mfj}{\mathfrak{j}}
\newcommand{\mfk}{\mathfrak{k}}	\newcommand{\mfl}{\mathfrak{l}}
\newcommand{\mfm}{\mathfrak{m}}	\newcommand{\mfn}{\mathfrak{n}}
\newcommand{\mfo}{\mathfrak{o}}	\newcommand{\mfp}{\mathfrak{p}}
\newcommand{\mfq}{\mathfrak{q}}	\newcommand{\mfr}{\mathfrak{r}}
\newcommand{\mfs}{\mathfrak{s}}	\newcommand{\mft}{\mathfrak{t}}
\newcommand{\mfu}{\mathfrak{u}}	\newcommand{\mfv}{\mathfrak{v}}
\newcommand{\mfw}{\mathfrak{w}}	\newcommand{\mfx}{\mathfrak{x}}
\newcommand{\mfy}{\mathfrak{y}}	\newcommand{\mfz}{\mathfrak{z}}


\newcommand{\mytitle}{GEA1000 Quant. Reasoning with Data}
\newcommand{\myauthor}{github/omgeta}
\newcommand{\mydate}{AY 24/25 Sem 2}

\begin{document}
\raggedright
\footnotesize
\begin{multicols*}{3}
\setlength{\premulticols}{1pt}
\setlength{\postmulticols}{1pt}
\setlength{\multicolsep}{1pt}
\setlength{\columnsep}{2pt}

{\normalsize{\textbf{\mytitle}}} \\
{\footnotesize{\mydate\hspace{2pt}\textemdash\hspace{2pt}\myauthor}}\vspace{-1pt}
%%%%%%%%%%%%%%%%%%%%%%%%%%%%%%%%%%%%%%%%%%%%%%%%%%%%%%
%                      Begin                         %
%%%%%%%%%%%%%%%%%%%%%%%%%%%%%%%%%%%%%%%%%%%%%%%%%%%%%%
\section{Studying Data}
Population is the entire group of interest.\\ Population parameter is a population's numerical fact. Census is an attempted survey of full population.

Sample is a subset of a population from a sampling frame. Sample statistic is a numeric fact of the sample. \\Estimates infer pop. parameters from sample statistics. 

Selection bias is caused by flawed sampling frame or non-probability sampling. Non-response bias is caused by systematic exclusion of subjects by unwillingness.

Probability sampling:
\begin{enumerate}[\roman*.]
  \item \textbf{Simple random}.
  \item \textbf{Systematic}: $k^{\text{th}}$ subject of each size-$r$ component.
  \item \textbf{Stratified}: Divide into strata sharing a characteristic, then SRS within each stratum. 
  \item \textbf{Cluster}: Divide into natural clusters, then SRS including all subjects within selected clusters.
\end{enumerate}

Non-probability sampling:
\begin{enumerate}[\roman*.]
  \item \textbf{Convenience sampling}: subjects chosen by convenience; selection bias.
  \item \textbf{Volunteer sampling}: self-selected sample, usually with subjects off strong opinions; selection bias.
\end{enumerate}

Study types:
\begin{enumerate}[\roman*.]
  \item \textbf{Experimental study}: observe dependent variable after direct manipulation of independent variable. Random treatment and control groups are similar. \\Shows cause-effect relationship.
  \item \textbf{Observational study}: observe variable of interest without manipulation of variables. \\Shows association, not necessarily cause-effect. 
\end{enumerate}

Generalizability: frame size $\geq$ population, probability sampling, large sample size and minimal bias.
\colbreak
\section{Categorical Data Analysis}
Categorical variables are ordinal (naturally ordered) or nominal (no natural order).

\subsection{Rates}

When variables $A$, $B$ are not associated:
\begin{enumerate}[\roman*.]
  \item $\rate{A\mid B} = \rate{A\mid B'}$
\end{enumerate}

When variables $A$, $B$ are associated:
\begin{enumerate}[\roman*.]
  \item $\rate{A\mid B} > \rate{A\mid B'}\text{ and}$\\$\rate{A'\mid B'} > \rate{A'\mid B}$\hfill($+$ve)
  \item $\rate{A\mid B} < \rate{A\mid B'}\text{ and}$\\$\rate{A'\mid B'} < \rate{A'\mid B}$\hfill($-$ve)
\end{enumerate}

Symmetry Rules:
\begin{enumerate}[\roman*.]
  \item $\rate{A\mid B} > \rate{A\mid B'}$ \\$\iff \rate{B\mid A} > \rate{B\mid A'}$
  \item $\rate{A\mid B} < \rate{A\mid B'}$ \\$\iff \rate{B\mid A} < \rate{B\mid A'}$
  \item $\rate{A\mid B} = \rate{A\mid B'}$ \\$\iff \rate{B\mid A} = \rate{B\mid A'}$
\end{enumerate}

Basic Rule on Rates:
\begin{enumerate}[\roman*.]
  \item $\rate{A}$ lies between $\rate{A\mid B}$ and $\rate{A\mid B'}$
  \item As $\rate{B} \rightarrow 100\%$, $\rate{A} \rightarrow \rate{A\mid B}$
  \item $\rate{B} = 50\%$ \\$\implies \rate{A} = \frac{1}{2}[\rate{A\mid B} + \rate{A\mid B'}]$
  \item $\rate{A\mid B} = \rate{A\mid B'}$\\$\implies \rate{A} = \rate{A\mid B} = \rate{A\mid B'}$
\end{enumerate}

\subsection{Simpson's Paradox}
Simpson's paradox is the observation that a trend appearing in majority of the groups of the data disappears/reverses when the groups are combined.

\subsubsection{Confounders}
Confounder is a third variable associated with both the independent and dependent variable being investigated. Randomised assignment can help to remove associations, removing the confounder in experimental studies.

\section{Numerical Data Analysis}
Numerical variables are discrete or continuous.

\subsection{Summary Statistics}
Mean, $\overline{x}$, is the average of variable $x$.\\
Mode is the most common element in variable $x$.\\
$Q_1$, Median, $Q_3$ are the ordered  $1^{\text{st}}$, $2^{\text{nd}}$, $3^{\text{rd}}$ quarter element of variable $x$.

Sample variance, Var, and standard deviation, $s_x$, of variable $x$ are given by: 
\begin{align*}
  \text{Var }&= \frac{\sum (x_i-\overline{x})^2}{n-1}\\
  s_x &= \sqrt{\text{Var}}
\end{align*}
Coefficient of variance, $\displaystyle \frac{s_x}{\overline{x}}$, measures spread relative to mean between different variables and has no units.

Median with $IQR = Q_3 - Q_1$ is preferred for asymmetrical distributions or when there are outliers.

Outliers satisfy one of the conditions:
\begin{enumerate}[\roman*.]
  \item $x > Q_3 + 1.5 \times IQR$
  \item $x < Q_1 - 1.5 \times IQR$
\end{enumerate}

\subsection{Univariate EDA}
\subsubsection{Histograms}
Histograms show data distribution, are better at greater frequencies and represent data points better.
Distributions with $n$ peaks are called $n$-modal.

Unimodal distribution shapes can be:
\begin{enumerate}[\roman*.]
  \item Symmetrical \hfill(mean = mode = median)
  \item Left-skewed \hfill(mean $<$ mode $<$ median)
  \item Right-skewed \hfill(mean $>$ mode $>$ median)
\end{enumerate}

Bell distributions are symmetrical with spread:
\begin{enumerate}[\roman*.]
  \item 68\% of data within 1 S.D.
  \item 95\% of data within 2 S.D.
\end{enumerate}
\colbreak
\subsubsection{Boxplots}
Boxplots side-by-side help compare distributions of different data sets, and are better to identify outliers.
They consist of minimum, $Q_1$, median, $Q_3$ and maximum.

Boxplot shapes can be:
\begin{enumerate}[\roman*.]
  \item Symmetrical \hfill($Q_1, Q_3$ equidistant to median)
  \item Left-skewed \hfill($Q_1$ closer to median)
  \item Right-skewed \hfill($Q_3$ closer to median)
\end{enumerate}

Boxplot spread for middle 50\% is given by $IQR$.

\subsection{Bivariate EDA}
Determinististic relationships determine exactly a variable given the value of the other variable.\\Association is a statistical relation describing average value of a variable given the value of the other variables

Correlation coefficient, $r$, is given by:
\begin{gather*}
  r = \frac{\text{Pop. covariance}}{\text{Pop. SD}_x \times \text{Pop. SD}_y} = \frac{\sum(x_i-\overline{x})(y_i-\overline{y})}{\sqrt{\sum(x_i-\overline{x})^2\cdot\sum(y_i-\overline{y})^2}}\\
  \text{*unaffected by swapping $x,y$ or adding/scaling by constant}
\end{gather*}
Direction, form and magnitude can be summarized by $r$:
\begin{enumerate}[\roman*.]
  \item $r > 0$\hfill($+$ve direction)
  \item $r < 0$\hfill($-$ve direction)
  \item $r=0$\hfill(Non-linear form)
  \item $0 < |r| < 0.3$\hfill(Weak association)
  \item $0.3 < |r| < 0.7$\hfill(Moderate association)
  \item $0.7 < |r| < 1$\hfill(Strong association)
\end{enumerate}

\subsection{Linear Regression}
Linear regression between variables believed to be linearly associated predicts the average value of the dependent variable given the independent variable.

Least squares regression line for predicting variable $Y$ given $X$ (and not vice versa) is given by:
\begin{align*}
  Y = mX + b,\quad m=\frac{s_Y}{s_X}r
\end{align*}
\colbreak

\section{Statistical Inference}
Probability of event $E$ in sample space $S$, $P(E)$, is given by:
\begin{enumerate}[\roman*.]
  \item $P(E) = \frac{|E|}{|S|}$, where $0 \leq P(E) \leq 1$
  \item $P(E') = 1 - P(E)$\hfill(Complement)
\end{enumerate}

Conditional probability of $B$ given $A$ is given by:
\begin{align*}
  P(B\mid A) = \displaystyle\frac{P(A\cap B)}{P(A)} = \frac{P(A\mid B)P(B)}{P(A)}
\end{align*}
Mutually exclusive events $A,B$ have special results:
\begin{enumerate}[\roman*.]
  \item $P(A\cap B) = 0$\hfill(Intersection)
  \item $P(A\cup B) = P(A) + P(B)$\hfill(Union)
  \item $A \cup B = S$\hfill(Total probability)\\$\implies P(C) = P(C\mid A)P(A) + P(C\mid B)P(B)$
\end{enumerate}

Independent events $A,B$ have special results:
\begin{enumerate}[\roman*.]
  \item $P(A\cap B) = P(A)\cdot P(B)$\hfill(Intersection)
  \item $P(A\mid B) = P(A)$\hfill(Conditional)
\end{enumerate}
\begin{align*}
  \text{Sensitivity} = \frac{\text{True Positives}}{\text{True Positives} + \text{False Negatives}}\\
  \text{Specificity} = \frac{\text{True Negatives}}{\text{True Negatives} + \text{False Positives}}
\end{align*}

\subsection{Fallacies}
Distribution fallacies:
\begin{enumerate}[\roman*.]
  \item \textbf{Ecological fallacy}: when we generalise group-level correlation to individuals. 
  \item \textbf{Atomistic fallacy}: when we generalise individual-level relations to groups. 
\end{enumerate}

Probability fallacies:
\begin{enumerate}[\roman*.]
  \item \textbf{Conjunction fallacy}: probability of two events occurring together is always less than of either event occurring alone.
  \item \textbf{Base rate fallacy}: ignoring the base rate of an event when calculating its probability.
\end{enumerate}
\colbreak
Relation between sample statistic and population parameter is given by:
\begin{align*}
  \text{Sample statistic = pop. parameter + bias + random error}
\end{align*}

\subsection{Confidence Intervals}
Confidence interval is a range of values likely to contain a population parameter at a certain confidence level.

Given a sample proportion $p^*$ and sample size $n$, confidence interval for population proportion is given by:
\begin{align*}
  p^* \pm z^* \times \sqrt{\frac{p^*(1-p^*)}{n}}
\end{align*}
where $z^*$ is the $z$-value for desired confidence level.

Given a sample mean $\overline{x}$, sample SD $s_x$ and sample size $n$, confidence interval for population mean is given by:
\begin{align*}
  \overline{x} \pm t^* \times \frac{s_x}{\sqrt{n}}
\end{align*}
where $t^*$ is the $t$-value for desired confidence level.

\subsection{Hypothesis Testing}
Hypothesis tests can be used for population proportion, population mean, and association, given a null hypothesis $H_0$, alternative hypothesis $H_1$, and significance value $\alpha$.
For hypothesis test on association, we take:
\begin{enumerate}[\roman*.]
  \item $H_0$ there is no association
  \item $H_1$: there is an association.  
\end{enumerate}

$p$-value can be defined as:
\begin{enumerate}[\roman*.]
  \item Probability of obtaining a sample statistic as extreme or more extreme than the observed statistic, assuming $H_0$ is true.
  \item Smallest level of significance at which $H_0$ is rejected, assuming $H_0$ is true
\end{enumerate}
where we reject $H_0$ in favour of $H_1$ when $p$-value $< \alpha$\\
or not reject $H_0$ (doesn't imply $H_0$ true) when $p$-value $\geq \alpha$

%%%%%%%%%%%%%%%%%%%%%%%%%%%%%%%%%%%%%%%%%%%%%%%%%%%%%%
%                       End                          %
%%%%%%%%%%%%%%%%%%%%%%%%%%%%%%%%%%%%%%%%%%%%%%%%%%%%%%
\end{multicols*}
\end{document}

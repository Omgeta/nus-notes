\documentclass[12pt, a4paper]{article}

\usepackage[utf8]{inputenc}
\usepackage[mathscr]{euscript}
\let\euscr\mathscr \let\mathscr\relax
\usepackage[scr]{rsfso}
\usepackage{amssymb,amsmath,amsthm,amsfonts}
\usepackage[shortlabels]{enumitem}
\usepackage{multicol,multirow}
\usepackage{lipsum}
\usepackage{balance}
\usepackage{calc}
\usepackage[colorlinks=true,citecolor=blue,linkcolor=blue]{hyperref}
\usepackage{import}
\usepackage{xifthen}
\usepackage{pdfpages}
\usepackage{transparent}
\usepackage{listings}

\newcommand{\incfig}[2][1.0]{
    \def\svgwidth{#1\columnwidth}
    \import{./figures/}{#2.pdf_tex}
}

\newlist{enumproof}{enumerate}{4}
\setlist[enumproof,1]{label=\arabic*., parsep=1em}
\setlist[enumproof,2]{label=\arabic{enumproofi}.\arabic*., parsep=1em}
\setlist[enumproof,3]{label=\arabic{enumproofi}.\arabic{enumproofii}.\arabic*., parsep=1em}
\setlist[enumproof,4]{label=\arabic{enumproofi}.\arabic{enumproofii}.\arabic{enumproofiii}.\arabic*., parsep=1em}

\renewcommand{\qedsymbol}{\ensuremath{\blacksquare}}

\lstdefinestyle{mystyle}{
  language=C, % Set the language to C
  commentstyle=\color{codegray}, % Color for comments
  keywordstyle=\color{orange}, % Color for basic keywords
  stringstyle=\color{mauve}, % Color for strings
  basicstyle={\ttfamily\footnotesize}, % Basic font style
  breakatwhitespace=false,         
  breaklines=true,                 
  captionpos=b,                    
  keepspaces=true,                 
  numbers=none,                    
  tabsize=2,
  morekeywords=[2]{\#include, \#define, \#ifdef, \#ifndef, \#endif, \#pragma, \#else, \#elif}, % Preprocessor directives
  keywordstyle=[2]\color{codegreen}, % Style for preprocessor directives
  morekeywords=[3]{int, char, float, double, void, struct, union, enum, const, volatile, static, extern, register, inline, restrict, _Bool, _Complex, _Imaginary, size_t, ssize_t, FILE}, % C standard types and common identifiers
  keywordstyle=[3]\color{identblue}, % Style for types and common identifiers
  morekeywords=[4]{printf, scanf, fopen, fclose, malloc, free, calloc, realloc, perror, strtok, strncpy, strcpy, strcmp, strlen}, % Standard library functions
  keywordstyle=[4]\color{cyan}, % Style for library functions
}

\usepackage{ifthen}
\usepackage[landscape]{geometry}
\usepackage[shortlabels]{enumitem}

\ifthenelse{\lengthtest { \paperwidth = 11in}}
    { \geometry{top=.5in,left=.5in,right=.5in,bottom=.5in} }
	{\ifthenelse{ \lengthtest{ \paperwidth = 297mm}}
		{\geometry{top=1cm,left=1cm,right=1cm,bottom=1cm} }
		{\geometry{top=1cm,left=1cm,right=1cm,bottom=1cm} }
	}

\pagestyle{empty}
\makeatletter
\renewcommand\thesection{\arabic{section}.}
\renewcommand{\section}{\@startsection{section}{1}{0mm}%
                                {-1ex plus -.5ex minus -.2ex}%
                                {0.05ex}%x
                                {\normalfont\normalsize\bfseries}}
\renewcommand{\subsection}{\@startsection{subsection}{2}{0mm}%
                                {-1ex plus -.5ex minus -.2ex}%
                                {0.05ex}%
                                {\normalfont\small\bfseries}}
\renewcommand{\subsubsection}{\@startsection{subsubsection}{3}{0mm}%
                                {-1ex plus -.5ex minus -.2ex}%
                                {0.05ex}%
                                {\normalfont\footnotesize\bfseries}}
\newcommand{\colbreak}{\vfill\null\columnbreak}
\makeatother
\setcounter{secnumdepth}{1}
\setlength{\parindent}{0pt}
\setlength{\parskip}{0.7em}

\setlist[itemize]{itemsep=0.6ex, topsep=-2pt, partopsep=0pt, parsep=0pt}
\setlist[enumerate]{itemsep=0.6ex, topsep=-2pt, partopsep=0pt, parsep=0pt}

% Things Lie
\newcommand{\kb}{\mathfrak b}
\newcommand{\kg}{\mathfrak g}
\newcommand{\kh}{\mathfrak h}
\newcommand{\kn}{\mathfrak n}
\newcommand{\ku}{\mathfrak u}
\newcommand{\kz}{\mathfrak z}
\DeclareMathOperator{\Ext}{Ext} % Ext functor
\DeclareMathOperator{\Tor}{Tor} % Tor functor
\newcommand{\gl}{\opname{\mathfrak{gl}}} % frak gl group
\renewcommand{\sl}{\opname{\mathfrak{sl}}} % frak sl group chktex 6

% More script letters etc.
\newcommand{\SA}{\mathcal A}
\newcommand{\SB}{\mathcal B}
\newcommand{\SC}{\mathcal C}
\newcommand{\SF}{\mathcal F}
\newcommand{\SG}{\mathcal G}
\newcommand{\SH}{\mathcal H}
\newcommand{\OO}{\mathcal O}

\newcommand{\SCA}{\mathscr A}
\newcommand{\SCB}{\mathscr B}
\newcommand{\SCC}{\mathscr C}
\newcommand{\SCD}{\mathscr D}
\newcommand{\SCE}{\mathscr E}
\newcommand{\SCF}{\mathscr F}
\newcommand{\SCG}{\mathscr G}
\newcommand{\SCH}{\mathscr H}

% Mathfrak primes
\newcommand{\km}{\mathfrak m}
\newcommand{\kp}{\mathfrak p}
\newcommand{\kq}{\mathfrak q}

% number sets
\newcommand{\RR}[1][]{\ensuremath{\ifstrempty{#1}{\mathbb{R}}{\mathbb{R}^{#1}}}}
\newcommand{\NN}[1][]{\ensuremath{\ifstrempty{#1}{\mathbb{N}}{\mathbb{N}^{#1}}}}
\newcommand{\ZZ}[1][]{\ensuremath{\ifstrempty{#1}{\mathbb{Z}}{\mathbb{Z}^{#1}}}}
\newcommand{\QQ}[1][]{\ensuremath{\ifstrempty{#1}{\mathbb{Q}}{\mathbb{Q}^{#1}}}}
\newcommand{\CC}[1][]{\ensuremath{\ifstrempty{#1}{\mathbb{C}}{\mathbb{C}^{#1}}}}
\newcommand{\PP}[1][]{\ensuremath{\ifstrempty{#1}{\mathbb{P}}{\mathbb{P}^{#1}}}}
\newcommand{\HH}[1][]{\ensuremath{\ifstrempty{#1}{\mathbb{H}}{\mathbb{H}^{#1}}}}
\newcommand{\FF}[1][]{\ensuremath{\ifstrempty{#1}{\mathbb{F}}{\mathbb{F}^{#1}}}}
% expected value
\newcommand{\EE}{\ensuremath{\mathbb{E}}}
\newcommand{\charin}{\text{ char }}
\DeclareMathOperator{\sign}{sign}
\DeclareMathOperator{\Aut}{Aut}
\DeclareMathOperator{\Inn}{Inn}
\DeclareMathOperator{\Syl}{Syl}
\DeclareMathOperator{\Gal}{Gal}
\DeclareMathOperator{\GL}{GL} % General linear group
\DeclareMathOperator{\SL}{SL} % Special linear group

%---------------------------------------
% BlackBoard Math Fonts :-
%---------------------------------------

%Captital Letters
\newcommand{\bbA}{\mathbb{A}}	\newcommand{\bbB}{\mathbb{B}}
\newcommand{\bbC}{\mathbb{C}}	\newcommand{\bbD}{\mathbb{D}}
\newcommand{\bbE}{\mathbb{E}}	\newcommand{\bbF}{\mathbb{F}}
\newcommand{\bbG}{\mathbb{G}}	\newcommand{\bbH}{\mathbb{H}}
\newcommand{\bbI}{\mathbb{I}}	\newcommand{\bbJ}{\mathbb{J}}
\newcommand{\bbK}{\mathbb{K}}	\newcommand{\bbL}{\mathbb{L}}
\newcommand{\bbM}{\mathbb{M}}	\newcommand{\bbN}{\mathbb{N}}
\newcommand{\bbO}{\mathbb{O}}	\newcommand{\bbP}{\mathbb{P}}
\newcommand{\bbQ}{\mathbb{Q}}	\newcommand{\bbR}{\mathbb{R}}
\newcommand{\bbS}{\mathbb{S}}	\newcommand{\bbT}{\mathbb{T}}
\newcommand{\bbU}{\mathbb{U}}	\newcommand{\bbV}{\mathbb{V}}
\newcommand{\bbW}{\mathbb{W}}	\newcommand{\bbX}{\mathbb{X}}
\newcommand{\bbY}{\mathbb{Y}}	\newcommand{\bbZ}{\mathbb{Z}}

%---------------------------------------
% MathCal Fonts :-
%---------------------------------------

%Captital Letters
\newcommand{\mcA}{\mathcal{A}}	\newcommand{\mcB}{\mathcal{B}}
\newcommand{\mcC}{\mathcal{C}}	\newcommand{\mcD}{\mathcal{D}}
\newcommand{\mcE}{\mathcal{E}}	\newcommand{\mcF}{\mathcal{F}}
\newcommand{\mcG}{\mathcal{G}}	\newcommand{\mcH}{\mathcal{H}}
\newcommand{\mcI}{\mathcal{I}}	\newcommand{\mcJ}{\mathcal{J}}
\newcommand{\mcK}{\mathcal{K}}	\newcommand{\mcL}{\mathcal{L}}
\newcommand{\mcM}{\mathcal{M}}	\newcommand{\mcN}{\mathcal{N}}
\newcommand{\mcO}{\mathcal{O}}	\newcommand{\mcP}{\mathcal{P}}
\newcommand{\mcQ}{\mathcal{Q}}	\newcommand{\mcR}{\mathcal{R}}
\newcommand{\mcS}{\mathcal{S}}	\newcommand{\mcT}{\mathcal{T}}
\newcommand{\mcU}{\mathcal{U}}	\newcommand{\mcV}{\mathcal{V}}
\newcommand{\mcW}{\mathcal{W}}	\newcommand{\mcX}{\mathcal{X}}
\newcommand{\mcY}{\mathcal{Y}}	\newcommand{\mcZ}{\mathcal{Z}}

%---------------------------------------
% Bold Math Fonts :-
%---------------------------------------

%Captital Letters
\newcommand{\bmA}{\boldsymbol{A}}	\newcommand{\bmB}{\boldsymbol{B}}
\newcommand{\bmC}{\boldsymbol{C}}	\newcommand{\bmD}{\boldsymbol{D}}
\newcommand{\bmE}{\boldsymbol{E}}	\newcommand{\bmF}{\boldsymbol{F}}
\newcommand{\bmG}{\boldsymbol{G}}	\newcommand{\bmH}{\boldsymbol{H}}
\newcommand{\bmI}{\boldsymbol{I}}	\newcommand{\bmJ}{\boldsymbol{J}}
\newcommand{\bmK}{\boldsymbol{K}}	\newcommand{\bmL}{\boldsymbol{L}}
\newcommand{\bmM}{\boldsymbol{M}}	\newcommand{\bmN}{\boldsymbol{N}}
\newcommand{\bmO}{\boldsymbol{O}}	\newcommand{\bmP}{\boldsymbol{P}}
\newcommand{\bmQ}{\boldsymbol{Q}}	\newcommand{\bmR}{\boldsymbol{R}}
\newcommand{\bmS}{\boldsymbol{S}}	\newcommand{\bmT}{\boldsymbol{T}}
\newcommand{\bmU}{\boldsymbol{U}}	\newcommand{\bmV}{\boldsymbol{V}}
\newcommand{\bmW}{\boldsymbol{W}}	\newcommand{\bmX}{\boldsymbol{X}}
\newcommand{\bmY}{\boldsymbol{Y}}	\newcommand{\bmZ}{\boldsymbol{Z}}
%Small Letters
\newcommand{\bma}{\boldsymbol{a}}	\newcommand{\bmb}{\boldsymbol{b}}
\newcommand{\bmc}{\boldsymbol{c}}	\newcommand{\bmd}{\boldsymbol{d}}
\newcommand{\bme}{\boldsymbol{e}}	\newcommand{\bmf}{\boldsymbol{f}}
\newcommand{\bmg}{\boldsymbol{g}}	\newcommand{\bmh}{\boldsymbol{h}}
\newcommand{\bmi}{\boldsymbol{i}}	\newcommand{\bmj}{\boldsymbol{j}}
\newcommand{\bmk}{\boldsymbol{k}}	\newcommand{\bml}{\boldsymbol{l}}
\newcommand{\bmm}{\boldsymbol{m}}	\newcommand{\bmn}{\boldsymbol{n}}
\newcommand{\bmo}{\boldsymbol{o}}	\newcommand{\bmp}{\boldsymbol{p}}
\newcommand{\bmq}{\boldsymbol{q}}	\newcommand{\bmr}{\boldsymbol{r}}
\newcommand{\bms}{\boldsymbol{s}}	\newcommand{\bmt}{\boldsymbol{t}}
\newcommand{\bmu}{\boldsymbol{u}}	\newcommand{\bmv}{\boldsymbol{v}}
\newcommand{\bmw}{\boldsymbol{w}}	\newcommand{\bmx}{\boldsymbol{x}}
\newcommand{\bmy}{\boldsymbol{y}}	\newcommand{\bmz}{\boldsymbol{z}}

%---------------------------------------
% Scr Math Fonts :-
%---------------------------------------

\newcommand{\sA}{{\mathscr{A}}}   \newcommand{\sB}{{\mathscr{B}}}
\newcommand{\sC}{{\mathscr{C}}}   \newcommand{\sD}{{\mathscr{D}}}
\newcommand{\sE}{{\mathscr{E}}}   \newcommand{\sF}{{\mathscr{F}}}
\newcommand{\sG}{{\mathscr{G}}}   \newcommand{\sH}{{\mathscr{H}}}
\newcommand{\sI}{{\mathscr{I}}}   \newcommand{\sJ}{{\mathscr{J}}}
\newcommand{\sK}{{\mathscr{K}}}   \newcommand{\sL}{{\mathscr{L}}}
\newcommand{\sM}{{\mathscr{M}}}   \newcommand{\sN}{{\mathscr{N}}}
\newcommand{\sO}{{\mathscr{O}}}   \newcommand{\sP}{{\mathscr{P}}}
\newcommand{\sQ}{{\mathscr{Q}}}   \newcommand{\sR}{{\mathscr{R}}}
\newcommand{\sS}{{\mathscr{S}}}   \newcommand{\sT}{{\mathscr{T}}}
\newcommand{\sU}{{\mathscr{U}}}   \newcommand{\sV}{{\mathscr{V}}}
\newcommand{\sW}{{\mathscr{W}}}   \newcommand{\sX}{{\mathscr{X}}}
\newcommand{\sY}{{\mathscr{Y}}}   \newcommand{\sZ}{{\mathscr{Z}}}


%---------------------------------------
% Math Fraktur Font
%---------------------------------------

%Captital Letters
\newcommand{\mfA}{\mathfrak{A}}	\newcommand{\mfB}{\mathfrak{B}}
\newcommand{\mfC}{\mathfrak{C}}	\newcommand{\mfD}{\mathfrak{D}}
\newcommand{\mfE}{\mathfrak{E}}	\newcommand{\mfF}{\mathfrak{F}}
\newcommand{\mfG}{\mathfrak{G}}	\newcommand{\mfH}{\mathfrak{H}}
\newcommand{\mfI}{\mathfrak{I}}	\newcommand{\mfJ}{\mathfrak{J}}
\newcommand{\mfK}{\mathfrak{K}}	\newcommand{\mfL}{\mathfrak{L}}
\newcommand{\mfM}{\mathfrak{M}}	\newcommand{\mfN}{\mathfrak{N}}
\newcommand{\mfO}{\mathfrak{O}}	\newcommand{\mfP}{\mathfrak{P}}
\newcommand{\mfQ}{\mathfrak{Q}}	\newcommand{\mfR}{\mathfrak{R}}
\newcommand{\mfS}{\mathfrak{S}}	\newcommand{\mfT}{\mathfrak{T}}
\newcommand{\mfU}{\mathfrak{U}}	\newcommand{\mfV}{\mathfrak{V}}
\newcommand{\mfW}{\mathfrak{W}}	\newcommand{\mfX}{\mathfrak{X}}
\newcommand{\mfY}{\mathfrak{Y}}	\newcommand{\mfZ}{\mathfrak{Z}}
%Small Letters
\newcommand{\mfa}{\mathfrak{a}}	\newcommand{\mfb}{\mathfrak{b}}
\newcommand{\mfc}{\mathfrak{c}}	\newcommand{\mfd}{\mathfrak{d}}
\newcommand{\mfe}{\mathfrak{e}}	\newcommand{\mff}{\mathfrak{f}}
\newcommand{\mfg}{\mathfrak{g}}	\newcommand{\mfh}{\mathfrak{h}}
\newcommand{\mfi}{\mathfrak{i}}	\newcommand{\mfj}{\mathfrak{j}}
\newcommand{\mfk}{\mathfrak{k}}	\newcommand{\mfl}{\mathfrak{l}}
\newcommand{\mfm}{\mathfrak{m}}	\newcommand{\mfn}{\mathfrak{n}}
\newcommand{\mfo}{\mathfrak{o}}	\newcommand{\mfp}{\mathfrak{p}}
\newcommand{\mfq}{\mathfrak{q}}	\newcommand{\mfr}{\mathfrak{r}}
\newcommand{\mfs}{\mathfrak{s}}	\newcommand{\mft}{\mathfrak{t}}
\newcommand{\mfu}{\mathfrak{u}}	\newcommand{\mfv}{\mathfrak{v}}
\newcommand{\mfw}{\mathfrak{w}}	\newcommand{\mfx}{\mathfrak{x}}
\newcommand{\mfy}{\mathfrak{y}}	\newcommand{\mfz}{\mathfrak{z}}


\newcommand{\mytitle}{GE1040 A Culture of Sustainability}
\newcommand{\myauthor}{github/omgeta}
\newcommand{\mydate}{AY 25/26 Sem 1}

\begin{document}
\raggedright
\footnotesize
\begin{multicols*}{3}
\setlength{\premulticols}{1pt}
\setlength{\postmulticols}{1pt}
\setlength{\multicolsep}{1pt}
\setlength{\columnsep}{2pt}

{\normalsize{\textbf{\mytitle}}} \\
{\footnotesize{\mydate\hspace{2pt}\textemdash\hspace{2pt}\myauthor}}
\vspace{-0.5em}
%%%%%%%%%%%%%%%%%%%%%%%%%%%%%%%%%%%%%%%%%%%%%%%%%%%%%%
%                      Begin                         %
%%%%%%%%%%%%%%%%%%%%%%%%%%%%%%%%%%%%%%%%%%%%%%%%%%%%%%
\section{Unsustainable Development}
Unsustainable development harms the ability of future generations to meet their needs due to the degradation of our climate, life-support systems and resources.
\begin{enumerate}[\roman*.]
  \item Global Environmental Indicators: Planetary boundaries have already been exceeded (climate change, biodiversity loss, nitrogen cycle).
  \item Stability Landscape: Ecosystems may lose resilience - once thresholds are crossed (valley basins), systems may not return to their original state.
  \item Changing Climate: We are living in a changing climate. We need to mitigate impacts on us and ecosystem, or otherwise become more adaptable
\end{enumerate}

\subsubsection{Ecological Footprint}

Terminology:
\begin{enumerate}[\roman*.]
  \item Bioproductivity: amount and rate of production occuring in an ecosystem over a period of time.
  \item Biocapacity $=$ Area $\times$ Biocapacity: quantifies nature's capacity to produce renewable resources, provide land for built-up areas and provide waste absorption services such as carbon uptake.
  \item Ecological Footprint (gha) $=$ Population $\times$ Consumption/Person $\times$ Footprint Intensity: quantifies biological productive area needed for provision of renewable resources, or require absorption of $\text{CO}_2$ waste.
  \item Footprint and biocapacity can be ranked by countries, enabling them to learn from each other. 
  \item Ecological Deficit: ecological footprint$>$biocapacity (currently exceeded by 50\%).
\end{enumerate}

Earth Overshoot Day is the date when annual footprint exceeds annual biocapacity:
\begin{enumerate}[\roman*.]
  \item Occurs earlier each year (first at 1970)
  \item COVID-19 caused ecological footprint to contract
  \item \#MoveTheDate: push overshoot to December via sustainable cities, resilient systems, etc.
\end{enumerate}
\vspace{-1em}
\colbreak
\subsection{Causes}
Overpopulation due to exponential population growth:
\begin{enumerate}[\roman*.]
  \item 8.1b (2025) $\rightarrow$ 9.5b (2050) $\rightarrow$ 11b (2100) 
  \item Pressure on land, soil degredation, and biodiversity 
  \item Prior to Industrial Revolution, growth was resource-limited but now largely unchecked 
  \item Various demands by billions for a comfortable life, but everyone should have the equal right to health 
\end{enumerate}

Unsustainable Resource Use:
\begin{enumerate}[\roman*.]
  \item Affluenza: oversumption in affluent societies
  \item Developing countries fixated on GDP as sole success metric ignoring environmental costs
\end{enumerate}

Poverty links to environment in a downward spiral:
\begin{enumerate}[\roman*.]
  \item Direct reliance on food, water, fuel for survival
  \item Degenerate forests, soil, grasslands and wildlife causing environmental degradation
  \item Degraded environment further impoverishes people
\end{enumerate}

Excluding Environmental Costs:
\begin{enumerate}[\roman*.]
  \item Market prices ignore externalities such as ecosystem loss, health impacts and pollution 
  \item Ex. Timber companies pay to clear forests but not for environmental degradation and loss of habitat 
  \item Ex. Fishing companies pay to catch fish but not for depletion of fish stocks 
  \item Taxes and fines aim to fix this but not enough
\end{enumerate}

\subsubsection{Tragedy of the Commons}

Tragedy of the commons is the overuse of a common property or free-access resource causing depletion for all.
\begin{enumerate}[\roman*.]
  \item Mentality of "If I do not use it, someone else will. The little bit I use or pollute doesn't matter".
  \item Solutions:
    \begin{itemize}[leftmargin=*]\vspace{2pt}
      \item Responsible usage of shared renewed resources at rates well below sustainable yields
      \item Convert open-access renewable resources to private ownership
    \end{itemize}
\end{enumerate}

\colbreak
\subsubsection{IPAT Model}
IPAT quantifies environmental impact as $I = P \times A \times T$
where $P$ is population, $A$ is affluence, $T$ is technology:
\begin{enumerate}[\roman*.]
  \item Population: not dominant factor 
  \item Affluence: $\displaystyle\frac{\text{Goods \& Services}}{\text{Person}}$ can harm through high consumption, pollution and resource wastage but also produce funding for innovative R\&D (e.g. Denmark, India's ethanol-blended gasoline)
  \item Technology: $\displaystyle\frac{\text{Impact}}{\text{Goods \& Services}}$ reduces impact 
  \item Ex.: Gasoline = cars $\times$ miles/car $\times$ gasoline/mile
\end{enumerate}

\subsection{Pollution}
Pollution is the introduction of contaminants into the natural environment that adversely affects a resource. 
\begin{enumerate}[\roman*.]
  \item Point Source: single, identifiable source (e.g. smokestack, drainpipes)
  \item Nonpoint Source: dispersed and difficult to identify (e.g. fertilizer and pesticide runoff into lakes - first flush effect after dryspell)
\end{enumerate}

Health Effects:
\begin{enumerate}[\roman*.]
  \item Headache and Fatigue
  \item Respiratory Illness
  \item Cardiovascular Illness
  \item Cancer Risk
  \item Nausea and Gastroenteritis
  \item Skin Irritation
\end{enumerate}

Management Methods:
\begin{enumerate}[\roman*.]
  \item Cleanup (end-of-pipe): clean/dillute contaminants
    \begin{enumerate}[leftmargin=*, label=$-$]\vspace{1pt}
      \item Temporary; growth in consumption may offset pollution control tech
      \item Often relocates pollutants to another area 
      \item Costly to clean dispersed pollutants
    \end{enumerate}
  \item Prevention (front-of-pipe): reduce/stop production
\end{enumerate}

\colbreak
\subsection{Environmental Viewpoints}
Planetary Management:
\begin{enumerate}[\roman*.]
  \item View: We are apart from the rest of nature and can manage it to meet our increasing demands.
  \item Resources: We will not run out, due to our ingenuity and technology.
  \item Economy: Potential for economic growth is essentially unlimited.
  \item Success: Depends on how well we manage the earth's life-support systems for our benefit.
\end{enumerate}

Stewardship:
\begin{enumerate}[\roman*.]
  \item View: We have an ethical responsibility to be caring stewards of the earth.
  \item Resources: We will probably not run out, but they should not be wasted.
  \item Economy: Encourage environmentally friendly economic growth and discourage harmful forms.
  \item Success: Depends on our managing the earth's life-support systems for our and nature's benefit. 
\end{enumerate}

Environmental Wisdom:
\begin{enumerate}[\roman*.]
  \item View: We are part of and totally dependent on nature, and nature exists for all species. 
  \item Resources: Limited and should not be wasted.
  \item Economy: Encourage earth-sustaining economic growth and discourage earth-degrading forms.
  \item Success: Depends on learning how nature sustains itself, integrating them into how we think and act.
\end{enumerate}

\colbreak
\section{Principles and Practical Applications of Sustainability}
Environmentally sustainable societies meet present needs without compromising future generations' own needs:
\begin{enumerate}[\roman*.]
  \item Without destroying the environment
  \item Without endangering the future welfare of the planet and its people 
  \item In a just and equitable manner
\end{enumerate}

Sustainability is the ability of Earth's natural systems, cultural systems and economies to survive and adapt to changing environmental conditions indefinitely. 3 Pillars:
\begin{enumerate}[\roman*.]
  \item Environment (ecological integrity)
  \item Economy (economic viability)
  \item Society (equity)
\end{enumerate}

\subsubsection{Sustainable Development Goals (SDGs)}

UN Sustainable Development Goals (SDGs) were adopted in 2015 by 2500 scientists from 190 nations, providing a global framework to steer towards a safe and just operating space for society to thrive in until 2030:
\begin{enumerate}[\arabic*.]
  \item No Poverty 
  \item Zero Hunger
  \item Good Health and Well-being 
  \item Quality Education 
  \item Gender Equality
  \item Clean Water and Sanitation
  \item Affordable and Clean Energy
  \item Decent Work and Economic Growth 
  \item Industry, Innovation and Infrastructure
  \item Reduce Inequality
  \item Sustainable Cities and Communities
  \item Responsible Consumption and Production 
  \item Climate Action 
  \item Life below Water 
  \item Life on Land 
  \item Peace and Justice Strong Institutions
  \item Partnerships to achieve the Goal
\end{enumerate}
\vspace{-1em}
\colbreak
\subsubsection{International Spillovers}
Spillovers are transboundary negative impacts generated by one country on others, which can undermine their ability to achieve the SDGs, measured by Spillover Score. Types of spillovers:
\begin{enumerate}[\roman*.]
  \item Environmental: use of natural resources and pollution, including transboundary effects embodied in trade, and direct cross-border flows in air and water
  \item Economic/Financial/Governance: international development finance, unfair tax competition, banking secrecy, and labor standards
  \item Security: negative externalities such as arms trade and organized crime destabilizing poorer countries; positive spillovers include conflict-prevention and peacekeeping investments
\end{enumerate}

\subsubsection{Natural Capital}
Natural capital (natural resources $+$ natural services) refers to the stock of natural resources and ecosystem services that sustain human life :  
\begin{enumerate}[\roman*.]
  \item Natural Resources: includes air, water, soil, land, life (biodiversity), nonrenewable resources, renewable energy and nonrenewable energy
  \item Natural Services: includes air purification, water purification, water storage, soil renewal, nutrient recycling, food production, conservation of biodiversity, wildlife habitat, forest renewal, waste treatment, climate control, population control and pest control
  \item Degradation of natural capital undermines long-term sustainability
  \item Preserving natural capital is essential for intergenerational equity
\end{enumerate}


\colbreak
\subsection{Sustainability Concepts}
Shifting Emphasis:
\begin{enumerate}[\roman*.]
  \item Pollution cleanup $\rightarrow$ Pollution prevention
  \item Waste disposal $\rightarrow$ Waste prevention and reduction
  \item Species protection $\rightarrow$ Habitat protection
  \item Environ. degradation $\rightarrow$ Environ. restoration
  \item Increased resource use $\rightarrow$ Less wasteful resource use
  \item Population growth $\rightarrow$ Population stabilization by decreasing birth rates
  \item Depleting and degrading natural capital $\rightarrow$ Protecting natural capital, living off bio-interest
\end{enumerate}

Lessons from Nature:
\begin{enumerate}[\roman*.]
  \item Runs on renewable solar energy $\rightarrow$ Rely mostly on renewable solar energy
  \item Recycles nutrients and wastes (little waste) $\rightarrow$ Prevent/reduce pollution, recycle \& reuse resources
  \item Uses biodiversity to maintain and adapt to environmental change $\rightarrow$ Preserve biodiversity by protecting ecosystem services, habitats and species
  \item Controls species' population size and resource use $\rightarrow$ Reduce births and wasteful resource use to prevent environmental overload, and depletion and degradation of resources 
\end{enumerate}

Challenges:
\begin{enumerate}[\roman*.]
  \item Depletion of finite resources (fossil fuels, soil, minerals, species)
  \item Overuse of renewable resources (forests, fish \& wildlife, soil fertility, public funds)
  \item Pollution (air, water, soil)
  \item Inequity (economic, political, social, gender)
  \item Species loss (endangered species and spaces)
\end{enumerate}

Solutions:
\begin{enumerate}[\roman*.]
  \item Cyclical use of resources (emulate nature; 3R's)
  \item Safe reliable energy (conservation, renewable energy, subtitution, interim measures)
  \item Human well-being interests (health, creativity, learning, cultural and spiritual development)
\end{enumerate}

\subsection{Research \& Development (R\&D)}
Examples of sustainability-oriented research and indigenous technology development:  
\begin{enumerate}[\roman*.]
  \item Power Generation: Biogas integrated with waste management/Co-Gen systems
  \item Construction Materials: Local, non-toxic, reusable materials; water as material for thermal walls
  \item Water Supply: Rainwater harvesting for groundwater recharge and building cooling
  \item Water Treatment: Natural biomaterials for turbidity removal; UV disinfection from sunlight; fabric filtration for point-of-use treatment
  \item Storm-water Management: Green roofs for runoff reduction, reduced energy use, and cooling effect
\item Building Design: Passive solar design, right-sized homes, cost-effective ventilation, maximize storage and comfort with minimal energy
\end{enumerate}

Green Roofs are intensive (thick substrate, shrubs/trees) or extensive (thin substrate, smaller plants):
\begin{enumerate}[\roman*.]
  \item Aesthetically pleasing
  \item Reduce storm-water runoff 
  \item Reduce urban heat island effects 
  \item Reduce air conditioning costs 
  \item Negate acid rain effects 
  \item Reduce $\text{CO}_2$ impact 
  \item Create habitats for certain plants and animals
  \item Cooling Effect: protects from solar radiation, stabilizes roof temperature, cools building interiors
  \item Water Quality: depends on substrate layer, vegetation type, fertilisation quality, roof age, surrounding area type, local pollution sources
\end{enumerate}

Key Carbon Considerations:  
\begin{enumerate}[\roman*.]
  \item Embodied: emissions from construction materials
  \item Operational: emissions from running processes
\end{enumerate}
\colbreak
\subsection{Case Study: NUS Sustainability Initiative}
Current Practices:
\begin{enumerate}[\roman*.]
  \item Integrated into education, research, campus operations, and leadership
  \item Contributes directly to multiple SDGs such as climate action, clean energy, and sustainable cities
\end{enumerate}

Future:
\begin{enumerate}[\roman*.]
  \item Driving sustainability via collaboration with other sectors (internal and external partners)
  \item Building climate resilience by linking research across climate, urban, economic, and social areas
\end{enumerate}

\subsection{Case Study: Punggol Digital District (PDD)}
Current Practices:
\begin{enumerate}[\roman*.]
  \item 17 Green Mark Platinum and 3 Super Low Energy Buildings (e.g. Mass Engineered Timber with 98\% lower embodied carbon)
  \item Smart Energy Grid: real-time data management optimisation saves 1,700 tonnes CO$_2$ annually; rooftop PV panels generate 3,000 MWh annually
  \item Open Digital Platform (ODP): collect and analyse environmental and building data to improve energy efficiency, reduce costs and minimise impact
  \item Digital Twin: real-time planning-simulation model
  \item Centralised Cooling System: cooling towers with underground distribution reduce energy use by $30\%$ and 3,700--4,000 tonnes CO$_2$ annually
  \item Carlite District: prioritises low-emission transport
\end{enumerate}

Future:
\begin{enumerate}[\roman*.]
  \item Environmental Modelling
  \item Use of recycled materials for construction
  \item Fuel Cell System for Lifts 
  \item Regenerative Lift converts motions into energy
  \item Centralized Chutes for Recyclables
  \item "Develop an eco-town with a human settlement that enables its residents to live a
good quality of life while using minimal natural resources."
\end{enumerate}

\colbreak
\subsection{Case Study: Singapore Story}
Vision of "Clean, Green and Good Living Environment" for present and future generations, balancing economic development, social progress and environmental protection (strong commitment from the top).

Fundamental Principles:
\begin{enumerate}[\roman*.]
  \item Control pollution at source
  \item "Polluter Pays" principle
  \item Pre-empt and take early action
  \item Innovation and technology
    \begin{itemize}[leftmargin=*]\vspace{2pt}
      \item Ex. Semakau Landfill as first man-made offshore landfill, Ulu Pandan NEWater Plant
    \end{itemize}
  \item Resource conservation 
    \begin{itemize}[leftmargin=*]\vspace{2pt}
      \item Ex. Waste recycling, energy efficient energy 
    \end{itemize}
  \item Environmental ownership
    \begin{itemize}[leftmargin=*]\vspace{3pt}
      \item Ex. Involvment of 3P (private, pubic, people) with communication, engagement and empowerment
    \end{itemize}
\end{enumerate}

Key Strategies:
\begin{enumerate}[\roman*.]
  \item Integrated land-use planning and development control (e.g., Singapore River clean-up)
  \item Environmental infrastructure (e.g., integrated solid waste; water and wastewater treatment)
  \item Environmental legislation and enforcement (pragmatic, progressive controls)
  \item Environmental monitoring (assess level of pollution, track trends, inform standards and planning)
  \item Environmental education (instil awareness via campaigns, training, dialogues)
\end{enumerate}

Future Challenges:
\begin{enumerate}[\roman*.]
  \item Small tropical island, high population density, limited natural resources
  \item Rising consumerism and environmental expectations; need to sustain standards
\end{enumerate}
\colbreak
\section{Circular Economy}
Circular economy is a regenerative system where resource input, waste, emissions, and energy leakage are minimised by slowing, closing, and narrowing resource loops.  
\begin{enumerate}[\roman*.]
  \item Replaces the end-of-life concept with long-lasting design, maintenance, repair, reuse, remanufacturing, refurbishing, and recycling
  \item Regenerate natural systems by returning valuable nutrients to ecosystems
  \item Supports economic, environmental, and social dimensions of sustainable development
  \item Mimics nature's cyclical systems 
\end{enumerate}

Economy Progression:
\begin{enumerate}[\roman*.]
  \item Linear Economy: take $\rightarrow$ make $\rightarrow$ dispose (single-use, wasteful)
  \item Recycling Economy: partial recovery through recycling but still wasteful
  \item Circular Economy: design for closed loops, resources kept in circulation, waste designed out
\end{enumerate}

\subsection{Energy and Matter}

High-quality energy is concentrated, good for useful work.
Low-quality energy is dispersed, bad for useful work.

Three Big Ideas:
\begin{enumerate}[\roman*.]
  \item There is no "away": matter cannot be destroyed, only converted\hfill(Law of Conservation)
  \item Cannot get something for nothing; cannot get out more energy than in\hfill(1st Law of Thermodynamics)
  \item Cannot break even; energy conversion from one form to another reduces energy quality or less usable energy\hfill(2nd Law of Thermodynamics)
\end{enumerate}

Economy Types:
\begin{enumerate}[\roman*.]
  \item High-throughput economy: unsustainable, high-waste, promotes pollution
  \item Low-throughput economy: matter-recycling and reuse; mimics nature and reduces pollutants but may be insufficient for growing populations
\end{enumerate}

\colbreak
\subsection{Industrial Symbiosis}

Industrial symbiosis is a business relationship focused on sharing resources between industrial facilities/companies where wastes or byproducts from one become raw materials for another.
\begin{enumerate}[\roman*.]
  \item Subset of Industrial Ecology: shift from linear to cyclical (closed-loop) systems
  \item Mutually beneficial cooperation which reduces environmental impact while improving business competitiveness
  \item Key enabling business model for advancing the move to a circular economy
  \item Ex. Kalundborg, Denmark: cooperation among 8 industries sharing energy, water, and materials 

\end{enumerate}

\subsection{Applications and Innovations}
\begin{enumerate}[\roman*.]
  \item Bioplastics: key enabler of low-carbon circular economy; allow closed resource cycles and cascading reuse especially when being reused or recycled
  \item Design for longevity: durability, easy repair, remanufacturing
  \item Closed resource cycles: minimize virgin resource extraction via reuse and recovery
  \item Resource Recovery: convert waste streams into valuable resources (e.g., waste-to-energy, nutrient recovery, material cascading)
\end{enumerate}
\colbreak

\section{Sustainable Urbanization}

Urbanization is inevitable as societies advance, but it poses significant environmental and social challenges. Sustainable cities aim to balance growth with ecological integrity, equity, and efficiency.  
\begin{enumerate}[\roman*.]
  \item $>50\%$ of the world's population lives in cities, with further growth projected esp. developing countries
  \item Urban living $\rightarrow$ higher literacy, education, health, social services, cultural and political participation.
  \item Trends:\vspace{2pt}
    \begin{itemize}[leftmargin=*]
      \item Proportion of urban global population growing
      \item Number and sizes of urban areas mushrooming
      \item Rapid increase in urban populations in developing countries
      \item Urban growth slower in developed nations
      \item Increasing poverty and inequality (75\% of cities more unequal now than 20 years ago)
    \end{itemize}
\end{enumerate}

Disadvantages of Urbanization:
\begin{enumerate}[\roman*.]
  \item Unsustainable systems
  \item Lack of vegetation
  \item Water problems 
  \item Pollution and health problems 
  \item Noise pollution 
  \item Climate and artificial light 
  \item Urban heat island
  \item Light pollution
\end{enumerate}

City Expansions:
\begin{enumerate}[\roman*.]
  \item Compact Cities: Limited land area with high population density, thus growing vertically. Most people get around by walking, biking or public transport.
    \begin{itemize}\vspace{2pt}
      \item Ex:"" Hong Kong, Singapore, Tokyo
    \end{itemize}
  \item Dispersed Cities: Ample land area available for outward expansion. Residents mostly depend on motor vehicles for transportation.
    \begin{itemize}\vspace{2pt}
      \item Ex: Australia, Canada, United States
    \end{itemize}
\end{enumerate}

\colbreak
\subsection{Urban Sprawl}
Urban sprawl is the uncontrolled, low-density outward expansion of cities into surrounding undeveloped or agricultural land, often characterised by low-density housing, single-use zoning and increasing reliance on private automobiles. 

Causes:
\begin{enumerate}[\roman*.]
  \item Prosperity
  \item Ample and Affordable Land
  \item Automobile 
  \item Cheap Gasoline 
  \item Poor Urban Planning
\end{enumerate}

Problems (Natural Capital Degradation):
\begin{enumerate}[\roman*.]
  \item Land and Biodiversity: loss of cropland, forests, grasslands, wetlands and habitats
  \item Water: increased use/pollution of surface and groundwater, runoff and flooding, decreased natural sewage treatment
  \item Energy, Air and Climate: increased energy waste, air pollution, greenhouse gas emissions 
  \item Economic: decline of downtown business districts, increased unemployment in central, loss of tax base in central
\end{enumerate}

Regulating:
\begin{enumerate}[\roman*.]
  \item Smart Growth: policies promoting compact, high-density, mixed-use development with access to mass transit
  \item Greenbelts: surrounding large cities with open space for recreation, forestry, or sustainable uses
  \item Urban Growth Boundaries: strict limits beyond which urban development is prohibited
  \item Cluster Development: concentrate high-density housing on part of land while preserving shared open space on the rest of the land (often 40-50\%)
  \item Satellite Towns: smaller metropolitan areas outside the main city, reducing pressure on the core
\end{enumerate}

\colbreak
\subsection{Megacities}
Megacities are metropolitan regions with populations exceeding 10 million people. They represent areas of high risk due to their complexity, scale, and vulnerability to environmental, economic, and social stresses.  

\begin{enumerate}[\roman*.]
  \item Growth: By 2030, 41--53 megacities are projected worldwide. Asia alone is expected to host 30 megacities by 2025
  \item Sustainability Risks:\vspace{2pt}
    \begin{itemize}[leftmargin=*]
      \item Rapid urban expansion without adequate planning drives resource depletion
      \item Environmental degradation and loss of ecosystem services
      \item Intensified greenhouse gas emissions from dense populations
    \end{itemize}
\end{enumerate}

Challenges:
\begin{enumerate}[\roman*.]
  \item Enormous demand for water, energy, waste management, and housing
  \item High risks of congestion, pollution, and infrastructure strain
  \item Extreme levels of poverty, social inequality, and vulnerability
  \item Fragmentation across economic, societal, and geopolitical dimensions
\end{enumerate}

Solutions:
\begin{enumerate}[\roman*.]
  \item Promote resource efficiency through policy (e.g., compact cities, smart growth)
  \item Invest in sustainable infrastructure, mass transit, and renewable energy
  \item Develop strong institutional frameworks for governance and resilience
  \item Encourage inclusive urban planning to reduce inequality
\end{enumerate}

\colbreak
\subsection{Sustainable City Models}

Environmentally Sustainable Cities:
\begin{enumerate}[\roman*.]
  \item Centralize the population within a given area 
  \item Use renewable energy as much as possible 
  \item Build and design people-oriented cities 
  \item Use energy and matter efficiently 
  \item Prevent pollution and reduce waste 
  \item Recycle, reuse, and compost 
  \item Protect and encourage biodiversity 
  \item Promote urban gardens and farmers markets 
  \item Zone for environmentally stable population levels
\end{enumerate}

Eco-cities model self-reliant resilient ecosystems:
\begin{enumerate}[\roman*.]
  \item Includes inhabitants and their ecological impacts 
  \item Focus: eliminate carbon waste, run on 100\% renewable energy 
  \item Goal: incorporate environment into the city, reduce poverty, higher population densities, improve health
\end{enumerate}

Smart cities use information and communication technologies:
\begin{enumerate}[\roman*.]
  \item Focus: increase operational efficiency, share information with public
  \item Goal: improve quality of government services and citizen welfare
\end{enumerate}

\colbreak
\subsection{Case Study: Vauban, Freiburg (Germany)}
Current Practices:
\begin{enumerate}[\roman*.]
  \item Connected, efficient, green transport: trams near every home, pedestrian and bicycle paths, trains every 7.5 minutes with subsidised tickets
  \item Combined Heat and Power (CHP) plant fueled by woodchips for district heating
  \item Buildings designed to low-energy and positive-energy standards 
  \item Anaerobic digestion of organic waste: ecological sewage treatment from household waste creates biogas for cooking
  \item Green infrastructure: 600 hectares of parks, 3,800 garden plots, and strong local food economy of farmships, wineries, butcheries, bakeries, etc.
  \item Renewable energy production encouraged with federal tax credits and regional utility subsidies
\end{enumerate}

\subsection{Case Study: Stockholm, Sweden}
Current Practices:
\begin{enumerate}[\roman*.]
  \item Green roofs bind rainwater, solar cells convert solar energy and heat water 
  \item Ecological fashion for environmentally aware 
  \item Household refuse is sucked into automatic underground waste collection systems
  \item Heat exchangers in water treatment 
  \item Street rainwater is treated locally and flows into lakes instead of treatment plants 
  \item Combustible waste is used for district heating and electricity; organic waste turned into biogas 
  \item Low-flushing toilets and tap aerators reduce water consumption by half
\end{enumerate}

\colbreak
\subsection{Case Study: Singapore Green Plan 2030} 
Aims:
\begin{enumerate}[\roman*.]
  \item City in Nature: plant 1 millon more trees by 2030, develop green spaces, adding new 1000ha by 2035
  \item Green Government: public sector leads sustainablty
  \item Sustainable Living: strengthen green efforts in schools, development of nationwide car-lite transport system, reduce waste to landfill per capta and household water consumption
  \item Energy Reset: clear-energy vehicles and EV-ready towns, achieve 80\% gren buildings by 2030, reduce energy consumtpion in HDBs, $4\times$ solar energy 
  \item Green Economy: incentivise sustainable industries, develop carbon services, seek investments in sustainability, and create jobs
  \item Resilient Future: safeguard coastlines against rising sea levels, improve food security, reduce urban heat
\end{enumerate}

\subsection{Case Study: CO$_2$ Accounting in China}
Findings:
\begin{enumerate}[\roman*.]
  \item {Shanghai:} highest per capita GDP and consumption-based CO$_2$ emissions
  \item {Beijing \& Tianjin:} similar per capita consumption-based emissions despite Beijing's local per-capita territorial emissions
  \item {Chongqing:} lowest per capita consumption-based emissions due to lower development
  \item Capital formation is the largest contributor to emissions, driven by rapid growth, infrastructure, and government policy
  \item Household consumption ranks second, with a smaller share than in developed countries
  \item Beijing has large government expenditure as capital
  \item Imports outweigh exports in embodied emissions, dominated by construction
  \item Food imports are high in Shanghai, Beijing, and Tianjin, but minimal in Chongqing
  \item Emissions rise with infrastructure demand, transport networks, and higher incomes
\end{enumerate}
\vspace{-1em}
\colbreak
\section{Carbon Management}

Carbon management in built environment is critical to reduce greenhouse gas emissions, since the building and construction sector accounts ~36\% of final energy use and ~39\% of energy- and process-related \text{CO$_2$} emissions in 2018. 
\begin{enumerate}[\roman*.]
  \item Operational Carbon: emissions from building use such as heating, cooling, lighting, and appliances.
  \item Embodied Carbon: emissions from materials and construction, including extraction, manufacturing, transport, and assembly.
  \item Life Cycle Stages: raw materials, transport, manufacturing, construction, operation, and end-of-life (reuse, recycling, landfill, incineration).
  \item Footprint: the total carbon impact is the sum of embodied and operational emissions.
\end{enumerate}

\subsection{Energy-Efficient Buildings}
{Passive Design} reduces energy demand:
\begin{enumerate}[\roman*.]
  \item Natural Ventilation 
  \item Sunlight Shading and Daylighting
  \item Dynamic Facade
\end{enumerate}

{Active Design} improves system efficiency:
\begin{enumerate}[\roman*.]
  \item Energy Efficient Chiller 
  \item Energy Recovery System 
  \item Evaporative Cooling 
  \item LED Lights
\end{enumerate}

{Smart Energy Management} uses technology to monitor, control, and optimize energy use:
\begin{enumerate}[\roman*.]
  \item Intelligent Building Management System (BMS)
\end{enumerate}

{Renewable Energy Integration} generates power on-site, often combined with other strategies to maximise energy generation:
\begin{enumerate}[\roman*.]
  \item Photovoltaic Panels 
  \item Wind Energy
  \item Tidal Energy 
  \item Geothermal Energy
\end{enumerate}

Types of Buildings:
\begin{enumerate}[\roman*.]
  \item Zero Energy Buildings (ZEB): produce as much energy as they consume annually.
  \item Super Low Energy Buildings (SLEB): operate with much lower energy use than conventional buildings.
  \item Positive Energy Buildings (PEB): generate surplus energy that can be stored, shared, or sold.
\end{enumerate}

Surplus Energy from PEBs can be:
\begin{enumerate}[\roman*.]
  \item Stored: in batteries for later use 
  \item Shared/Sold: exported to the grid or shared with others; a Renewable Energy Certificate (REC) is a tradeable proof of 1 MWh of renewable energy fed into the grid 
  \item Integrated: in smart grids for efficient distribution
\end{enumerate}

\subsection{Design for Manufacture and Assembly (DfMA)}
Fundamentals of DfMA:
\begin{enumerate}[\roman*.]
  \item Simplifies design for efficient manufacture and assembly.
  \item Identifies and eliminates waste and inefficiency.
  \item Enables end-of-life pathways such as reuse and recycling through design for disassembly.
\end{enumerate}

EOL Scenarios:
\begin{enumerate}[\roman*.]
  \item Landfill: frame is disposed, contributing to waste and environmental impact 
  \item Downcycle: frame is recycled into lower-value products, extending life but reducing utility 
  \item Incinerate: frame is burned, generating energy but releasing emissions and reducing material recovery 
  \item Re-use: frame is directly reused in new projects, preserving value and minimising waste
\end{enumerate}

Implementation:
\begin{enumerate}[\roman*.]
  \item Minimize number of components
  \item Modular construction (e.g. prefab units, modules)
  \item Simplify joints (e.g. plug-in connections)
  \item Top-down vertical assembly with self-aligning parts
\end{enumerate}

\subsubsection{Prefabricated Prefinished Volumetric Construction (PPVC)}
PPVC modules are manufactured and finished off-site before being transported and assembled on-site.
\begin{enumerate}[\roman*.]
  \item Improves construction productivity and efficiency
  \item Ensures higher quality through controlled factory conditions
  \item Reduces dust, noise, and on-site disruption
  \item Requires fewer workers on-site, enhancing safety
  \item Speeds up project timelines through parallel off-site fabrication and on-site preparation
\end{enumerate}

\subsection{Reducing Embodied Carbon}
Concrete and steel are major sources of embodied emissions. Reductions can be achieved through:
\begin{enumerate}[\roman*.]
  \item Material substitution:
    \begin{itemize}[leftmargin=*]\vspace{3pt}
      \item GGBS - 30--50\% replacement of cement
      \item PFA - 15--30\% replacement of cement
      \item RCA - 10\% replacement (coarse fraction)
      \item WCS - 10\% replacement (fine fraction)
    \end{itemize}
  \item Optimizing material use via efficient structural design
  \item Procuring low- or zero-carbon alternatives 
  \item Designing for re-use and recycling at end-of-life
\end{enumerate}

Life Cycle Assessment (LCA) of Materials:
\begin{enumerate}[\roman*.]
  \item Embodied carbon is released at the start of a building's life, ``locking in'' emissions for decades
  \item Early-phase reductions are critical to avoid long-term carbon lock-in
  \item Operational carbon accumulates over the use phase and can be reduced through retrofits and energy efficiency upgrades
\end{enumerate}

\colbreak
\section{Air Quality}
Air quality is linked to earth’s climate and ecosystems.
\begin{enumerate}[\roman*.]
  \item Ambient Air: surrounds us outdoors and to which we are constantly exposed.
\end{enumerate}

Atmosphere Layers:
\begin{enumerate}[\roman*.]
  \item Thermosphere ($\sim$500km): contains $>$99.9\% of atmospheric gases
  \item Mesosphere ($\sim$100km)
  \item Stratosphere ($\sim$50km): contains 90\% of atmospheric ozone, which acts a protective layer against the sun's harmful UV radiation
  \item Troposphere ($\sim$10km): weather conditions are mainly controlled by physical processes here
\end{enumerate}

\subsection{Air Pollutants}
Air pollutants are substances which are unnatural or have higher than normal concentration. They can be transferred to other sectors of the environment\\(i.e. biosphere, hydrosphere, lithosphere).

Physical forms:
\begin{enumerate}[\roman*.]
  \item Particulate matter (e.g. ash, dust, smoke)
  \item Gases (e.g. sulfur dioxide, carbon monoxide)
\end{enumerate}

Classification of Air Pollutants:
\begin{enumerate}[\roman*.]
  \item Primary Pollutants: found in atmosphere in the same chemical form as when it was emitted.
    \begin{itemize}[leftmargin=*]\vspace{3pt}
      \item Ex. carbon monoxide, nitric oxide, hydrogen sulfide, sulfur dioxide, halogen compounds
    \end{itemize}
  \item Secondary Pollutants: formed in the air as result of chemical transformation of primary pollutants. 
    \begin{itemize}[leftmargin=*]\vspace{3pt}
      \item Ex. nitrogen dioxide from nitric oxide, ozone from photochemical reactions of nitrogen oxides, sulfuric acid droplets from sulfur dioxide
    \end{itemize}
\end{enumerate}

\colbreak
Sources of Air Pollutants:
\begin{enumerate}[\roman*.]
  \item Fires release harmful chemicals and haze: 
    \begin{itemize}[leftmargin=*]\vspace{3pt}
      \item Ex. forest fires, deforestation, agri. burning
    \end{itemize}
  \item Industrialization and modernization have increased air pollution due to increased demand for power 
    \begin{itemize}[leftmargin=*]\vspace{3pt}
      \item Ex. Bhopal, India: Gas leak on 2 Dec. 1984 at UCIL Pesticide plant which exposed $>$500,000 people to MIC gas and other chemicals. Approximately 5,200 deaths and several thousand permanent or partial disabilities. 
      \item Ex. Chernobyl, Ukraine: Nuclear accident on 26 Apr. 1986 releasing large quantities of radioactive particles and a radioactive cloud, killing 30 workers from acute radiation poisoning and exposing $>$6000,000 recovery workers.
    \end{itemize}
  \item Mobile sources 
    \begin{itemize}[leftmargin=*]\vspace{3pt}
      \item Ex. automobiles, diesel trucks, buses, planes
      \item Ex. New Delhi, India: most polluted city - WHO; vehicles account for $75\%$ of pollution
    \end{itemize}
  \item Stationary sources 
    \begin{itemize}[leftmargin=*]\vspace{3pt}
      \item Ex. industrial and power plants 
    \end{itemize}
  \item Air pollution episodes and accidents
  \item Thermal inversion occurs when warm air settles over cooler air near the ground, holding down the cool air, stopping pollutants rising and scattering
    \begin{itemize}[leftmargin=*]\vspace{3pt}
      \item Ex. Donora, Pennsylvania: temperature inversion on Oct. 1948 caused a wall of smog killing 20 and sickening 7,000, due to fluoride emissions from zinc and steel plants.
      \item Ex. Great Smog of London: in Dec. 1952 caused by black smoke from homes and factories killing 12,000 and bringing transport to a standstill.
    \end{itemize}
\end{enumerate}

\subsubsection{Clean Air Act}

US Environmental Protection Agency (EPA) in 1970 addresses environmental problems; Clean Air Act (CAA) passed to safeguard public health by regulating emissions.
\begin{enumerate}[\roman*.]
  \item EPA is authorized to set standards protecting public health and welfare, and to regulate emissions
  \item CAA is one of most comprehensive air quality laws
\end{enumerate}
\vspace{-1em}
\colbreak
\subsection{Criteria Pollutants}

US National Ambient Air Quality Standards (NAAQS) sets limits for 6 common air pollutants which are most prevalent and harmful to health and the environment, if concentration in ambient air is above certain levels.

Carbon Monoxide (CO) is a colorless, odorless gas from incomplete combustion of carbon:
\begin{enumerate}[\roman*.]
  \item Sources: cigarette smoking; incomplete burning of fossil fuels; $\sim$77\% (up to 95\% in cities) from motor-vehicle exhaust.
  \item Health: binds to hemoglobin reducing O$_2$ delivery; impairs perception/reflexes; headaches, drowsiness; can trigger angina/heart attacks; harms fetal/child development; aggravates chronic bronchitis, emphysema, anemia;  coma, brain damage, death.
\end{enumerate}

Nitrogen Dioxide (NO$_2$) is a reddish-brown irritant, converts to nitric acid (HNO$_3$) acid rain component:
\begin{enumerate}[\roman*.]
  \item Sources: fossil-fuel burning in motor vehicles (49\%) power plants \& industries (46\%).
  \item Health: lung irritation/damage; aggravates asthma \& chronic bronchitis; increases susceptibility to respiratory infections (children/elderly).
  \item Environmental: reduces visibility; HNO$_3$ damages trees, soils, aquatic life; corrodes metals/stone; NO$_2$ damages fabrics.
\end{enumerate}

Sulfur Dioxide (SO$_2$) is a colorless irritant mainly from burning sulfur-containing fuel, converts to sulfuric acid (H$_2$SO$_4$) acid rain component:
\begin{enumerate}[\roman*.]
  \item Sources: coal-burning power plants (88\%); industrial processes (10\%).
  \item Health: breathing problems; airway restriction in asthmatics; chronic exposure cause bronchitis-like
  \item Environmental/Property: visibility loss; H$_2$SO$_4$ damages trees, soils, lakes; SO$_2$ damages paint, paper, leather; both corrode metals, erode stone.
\end{enumerate}
\colbreak
Particulate Matter (PM) are particles and droplets (aerosols) light enough to remain suspended; cause smoke, dust, haze:
\begin{enumerate}[\roman*.]
  \item Sources: coal burning (40\%); diesel/other fuels in vehicles (17\%); agriculture (plowing, field burning), unpaved roads, construction.
  \item Health: nose/throat irritation, lung damage, bronchitis; aggravates bronchitis/asthma; shortens life; toxic particulates (Pb, Cd, dioxins) can cause mutations, reproductive problems, cancer; damage severity depends on particle size, inhaled number, and individual health; fine/ultrafine penetrate deep lungs (some to bloodstream).
  \item Environmental: reduces visibility; H$_2$SO$_4$ droplets harm trees/soils/aquatic life; corrodes metal; soils/discolors buildings, clothes, fabrics, paints; affects photosynthesis and precipitation (more condensation nuclei).
\end{enumerate}

Ground-level Ozone (O$_3$) is highly reactive irritant and major component of smog:
\begin{enumerate}[\roman*.]
  \item Sources: Nitrogen oxides and volatile organic compounds (mostly cars/industry) react in sunlight
  \item Health: breathing problems; coughing; eye/nose/throat irritation; aggravates asthma, bronchitis, emphysema, heart disease; reduces resistance to colds/pneumonia; speeds lung aging.
  \item Environmental: damages plants more than many other pollutants; smog reduces visibility; damages rubber, fabrics, paints; large mortality globally.
\end{enumerate}

Lead (Pb) is a toxic metal; emitted to air attached to particulate matter:
\begin{enumerate}[\roman*.]
  \item Sources: old-house paint, smelters (metal refineries), lead manufacture, storage batteries, (legacy) leaded gasoline.
  \item Health: bioaccumulates; brain/nervous-system damage and mental retardation (esp. children); digestive/other health problems; carcinogenic.
  \item Enivornmental: can harm wildlife
\end{enumerate}

Mathematical dispersion modeling estimates concentration of pollutants at various distances from a source.
\vspace{-1em}
\colbreak
\subsubsection{Acid Deposition (Acid Rain)}
Acid deposition is the accumulation of acids in land, water, or in vegatation as a result of acid rain or direct absorption from atmosphere.
\begin{enumerate}[\roman*.]
  \item Acid rain increases the concentration of Al$^{3+}$ in groundwater, adversely affecting plant growth. 
  \item Large sections of established forests have been severely damaged in eastern US.
\end{enumerate}

\subsubsection{Hazardous Air Pollutants (HAPs)}
Toxic pollutants known or suspected to cause cancer or other serious health effects: 
\begin{enumerate}[\roman*.]
  \item Highest levels closest to source
  \item Standards set based on availability of control tech  
  \item Ex. asbestos, vinyl chloride, benzene, arsenic
\end{enumerate}

\subsubsection{Indoor Air Pollution}
Indoor air pollution is ubiquitous, taking many forms, especially in developing countries and in mixtures of volatile compounds in modern buildings:
\begin{enumerate}[\roman*.]
  \item Sources: solid-fuel cooking (e.g. wood) in poorly ventilated homes causes high pollutant levels.
  \item Health: irritation of eyes, nose, throat; headaches, dizziness, fatigue; respiratory diseases; heart disease; cancer; predominantly targets women and young children
  \item Ex. Outbreak of Legionnaires’ disease at Philadelphia, Pennsylvania
\end{enumerate}

\subsubsection{Greenhouse Gases (GHGs)}

Greenhouse gas accumulation can also lead to adverse effects on the global climate:
\begin{enumerate}[\roman*.]
  \item Concerns arise over polar ice caps, coastal flooding, and blocking of sun's rays triggering mini ice age
  \item Kyoto Protocol, adopted in Kyoto, Japan, on 11 December 1997, proposed that industrialized nations must reduce their collective greenhouse gas emissions by 5.2\% between 2008 to 2012.
\end{enumerate}

\colbreak
\subsubsection{Pollution Prevention}
Pollution prevention hierarchy (least to most preferable): dispose $\rightarrow$ treat $\rightarrow$ recover $\rightarrow$ recycle $\rightarrow$ reuse $\rightarrow$ reduce $\rightarrow$ avoid

Pollution reduction or prevention come with a price and may require process alteration, change of fuel, marketting by-products, and temporary plant shutdowns.


\subsection{Good Ozone Depletion}
Gradual thinning of Earth’s stratospheric ozone layer caused by human-released gases containing chlorine and other halogens.
\begin{enumerate}[\roman*.]
  \item {Causes:} Chlorofluorocarbons (CFCs) and other ozone-depleting substances (ODS) such as HCFCs, halons, methyl chloroform, and carbon tetrachloride.
  \item {Ozone Hole (Antarctica):} Dramatic springtime loss in the lower stratosphere, most pronounced over the polar regions (Antarctica).
    \begin{itemize}[leftmargin=*]\vspace{2pt}
      \item Largest observed on 24 Sep 2006 (areas with $>\!50\%$ ozone decrease).
      \item Polar vortex: isolated, concentric flow around Antarctica with strong westerlies and little meridional transport.
      \item Springtime phenomenon observed in satellite records (e.g., TOMS/OMI daily minima).
      \item Polar stratospheric clouds form at very cold temperatures and enable reactive chlorine chemistry.
    \end{itemize}
  \item {Impacts:} More UV reaches Earth’s surface $\rightarrow$ increased skin cancer, cataracts, and genetic/immune system damage; broader environmental effects.
  \item {Policy \& Recovery:} Montreal Protocol (1987; amended 1990, 1992) phases out production/consumption of ODS (targeted by 2000); also yields climate benefits since many ODS are potent greenhouse gases; global ozone shows recovery stages.
\end{enumerate}

\colbreak
\section{Climate Change}
Climate is the statistical description of weather conditions and their variations. Climate change is a shift in the average pattern of weather over long periods. 
\begin{enumerate}[\roman*.]
  \item Greenhouse gases (GHGs) cause the greenhouse effect, trapping infrared radiation from the sun 
    \begin{itemize}[leftmargin=*]\vspace{3pt}
      \item Ex. water vapour, carbon dioxide, methane, nitrous oxide, and some industrial gases such as chloroflurocarbons (CFCs)
    \end{itemize}
  \item IPCC Report: human activity has increased global surface temperature ($1.09^{\circ}$C comparing 2011-20 vs. 1850–1900; $1.5^{\circ}$C limit in Paris Agreement likely breached in few decades, with overshoot temporary only if emission cuts are immediate \& sustained.
  \item CO$_2$ is the major driver of recent warming (observed rise in CO$_2$ and global temperature, 1880–2010).
  \item Urbanization: by 2050, $\sim$70\% of people will live in cities; cities already consume $\sim$80\% of global material/energy and produce $>$70\% of CO$_2$. 
\end{enumerate}

\subsection{OCBC Climate Index}

OCBC Climate Index was developed to capture local climate attitudes and actions:
\begin{enumerate}[\roman*.]
  \item Singaporeans have high awareness of environmental issues, but adoption levels not on par 
  \item Cost and inconvenience cited as top reasons for low adoption; could be due to needs of an individual according to life stages 
  \item Singaporeans love food, struggling most in terms of adoption here but likely to advocate green practices 
  \item Change in mindset is needed to kill bad habits
\end{enumerate}

Lessons Learnt:
\begin{enumerate}[\roman*.]
  \item Transport: prefer public transport, Park Connector Networks and safe cycling routes
  \item Home: use AC responsibly between $25-27^{\circ}$C
  \item Food: buy local vegetables, eat less meat
  \item Goods: think before purchasing, prefer quality long-lasting items, repair if possible
\end{enumerate}
\vspace{-1em}
\colbreak
\subsection{Effects}
Project and Observed Impacts:
\begin{enumerate}[\roman*.]
  \item Sea Level Rise: saltwater intrusion raises groundwater salinity; amplifies storm-surge/high-wave impacts on coasts.
  \item Extreme Weather Events:: more frequent/severe events in recent decades.
  \item More Heat Extremes: Small mean warming $\rightarrow$ large increase in heat extremes; fewer cold extremes.
  \item Heavier Rain: warmer air holds more exponentially vapor; when it rains, more falls at once.
  \item Drought: higher water evaporation not offset by precipitation (e.g. western US), snowpack down, runoff spikes from intense events.
  \item Wildfire: increasing area burned; fires now \#1 PM$_{2.5}$ source in western US; offsets air-quality gain.
  \item Human Health: serious threats to well-being from climate crisis.
\end{enumerate}

Future:
\begin{enumerate}[\roman*.]
  \item Continued CO$_2$ growth will lead to dangerous temperature increases; aggressive decarbonisation is needed to avoid 2 degrees of danger, requiring cutting emissions by $>$ half.
  \item 2023 CO$_2$ emissions $+1.1\%$ (IEA); 2020-23 increase driven by Global South; coal $\sim$65\% of rise.
\end{enumerate}
\subsection{Mitigation}
\begin{enumerate}[\roman*.]
  \item Energy Efficiency: buildings (design, insulation, A/C setpoints $25$-$27^{\circ}$C), industry (heat recovery), transport (EVs, mode shift).
  \item Clean Power: solar, wind, hydro, geothermal; retire inefficient coal; flexible grids and storage.
  \item Fuel Switching: coal $\rightarrow$ natural gas (as a transition), electrification of end-uses.
  \item Non-CO$_2$ Gases: methane leak detection/repair (LDAR), agricultural CH$_4$/N$_2$O reductions, refrigerant management.
  \item Nature-Based: afforestation/reforestation, peatland and mangrove restoration, improved soils.
\end{enumerate}
\vspace{-1em}
\colbreak

\subsection{Carbon Capture, Utilisation \& Storage}
Pipeline:
\begin{enumerate}[\roman*.]
  \item Capture: selective removal of CO$_2$ from point sources (power, cement, steel) or air (DAC).
  \item Transport: compression and movement via pipelines (regional) or ships (long distance).
  \item Utilisation: CO$_2$ for enhanced oil/gas/coal-bed recovery (EOR/EGR/ECBM), synthetic fuels/chemicals, mineralisation.
  \item Storage: secure geologic formations with physical/chemical trapping; monitoring, reporting, verification (MRV).
\end{enumerate}

\subsection{Government Role}
Strategies:
\begin{enumerate}[\roman*.]
  \item Regulatory: set performance standards for CO$_2$/CH$_4$; phase out inefficient coal plants.
  \item Carbon Pricing: carbon taxes or cap-and-trade; recycle revenues to households/innovation.
  \item Subsidy Shifts: phase out fossil subsidies; support efficiency, renewables, clean industry, sustainable agriculture.
  \item Finance \& R\&D: scale public/private investment; de-risk projects; accelerate technology transfer.
  \item Land Use: protect forests; monitor and finance anti-deforestation; sustainable food systems.
\end{enumerate}

\subsection{Adaptation}
\begin{enumerate}[\roman*.]
  \item Coasts: raise/flood-proof assets; restore dunes/mangroves; managed retreat in hotspots.
  \item Heat: cool roofs/urban greening; early-warning systems; heat-health action plans.
  \item Water: diversify sources (reuse, desalination), conserve, improve storage; flood-resilient drainage.
  \item Food: climate-smart agriculture, drought/heat-tolerant crops; reduce loss/waste.
  \item Health: surveillance of climate-sensitive diseases; protect vulnerable populations.
\end{enumerate}

\colbreak
\section{Nonrenewable Energy}

Nonrenewable Energy (NRE) comes from finite sources and is consumed much faster than can be naturally replenished.
\begin{enumerate}[\roman*.]
  \item Nonrenewable energy fuels $\sim \frac{3}{4}$ of the world's commercial energy.
  \item Global energy demand is projected to increase 28\% between 2015-40
  \item Net energy is the only energy that counts: usable energy minus energy automatically wasted in finding, processing and transporting it to users.
\end{enumerate}

\subsection{Coal}
Coal is the world's most abundant fossil fuel:
\begin{enumerate}[\roman*.]
  \item Generates $\sim$40\% of global electricity at inexpensive cost, with potential to cover global energy needs into 22$^{nd}$ century.
  \item Well-developed technology, with air pollution reducible by improved technology
  \item Burned industrially for iron, steel and other products
  \item Major CO$_2$, SO$_2$ emitter, along with soot, and toxic industructible mercury and radioactive materials
  \item Major factor in development of emerging and developing economies.
  \item Process: heat generated boils water to produce steam which spins a turbine to product electricity. Steam can be recooled, condensed and reused. 
\end{enumerate}

Synfuels burn cleaner than coal, but lower net energy and higher cost per unit of energy than conventional coal:
\begin{enumerate}[\roman*.]
  \item Gasification: forms syngas (of H$_2$, CO, CO$_2$). 
  \item Liquefaction: forms methanol/synthetic gasoline.
\end{enumerate}

\colbreak
\subsection{Oil}
Crude oil is a black liquid of different combustible hydrocarbons, trappened under crust or under the seafloor, requiring drilling and refining.
\begin{enumerate}[\roman*.]
  \item Crude oil is biggest source of commercial energy (85 million barrels of oil /day), oil is consued 4$\times$ faster than being discovered, crude oil will be depleted over 80\% between 2050-2100.
  \item High net energy yield, easy to transport, low land use, and well developed tech
  \item Need to find substitutes within 50 years, with fluctuating market price and high pollution
  \item Organisation of Petroleum Exporting Countries (OPEC): 13 countries holding ~78\% of proven reserves; long-time supply influence. 
  \item Process: Fractional distillation by boiling points; yields fuels \& {petrochemicals} (plastics, fibers, solvents, medicines, etc.). 
\end{enumerate}

Unconventional Oil is found in poorly connected poors, or is too viscious/heavy so buoyant forces are insufficient to expel them from the reservoir and requires more technology:
\begin{enumerate}[\roman*.]
  \item Oil Sands: extra heavy, viscious crude oil trapped in unconsolidated sand; most in Canada and Venezuela; expensive and emission intensive
  \item Oil Shales: fine grained sedimentary rocks containing kerogen from which shale oil and gas are produced; mainly in US, Australia, China, Brazil, Estonia
  \item Tight Oil: light crude oil in formations of low permeability; mainly in Russia, US, China, Argentia, Libya
  \item Hydraulic Fracturing: extracts tight oil/gas; carries environmental risks (water, seismicity, fugitive emissions). 
\end{enumerate}

\colbreak
\subsection{Natural Gas}
Natural gas is a mixture of gases (50-90\% methane) and other hydrocarbons (e.g. propane, butane) often found above crude oil.
\begin{enumerate}[\roman*.]
  \item Known reserves should last 62-125 years, with high net energy yield, abundance, transport and usage breakthroughts, energy security concerns
  \item Low net energy density complicates handling
  \item Process: refine to remove impurities/NGLs
  \item Liquefied Petroleum Gas (LPG): when natural gas is tapped, propane and butane gases are liquefied and removed
\end{enumerate}

Unconvential Gas:
\begin{enumerate}[\roman*.]
  \item Tight Gas: trapped in low-permeability sandstones/limestones; mainly US or North Sea.
  \item Shale Gas: trapped in shale (acts as source and reservoir); produced via horizontal drilling + hydraulic fracturing.
  \item Coal Bed Methane (CBM): methane adsorbed in coal seams; co-produces significant water which must be removed (adds cost/handling).
  \item Methane Hydrates: methane trapped in ice lattices under seafloor/permafrost; large potential but technical \& environmental risks.
\end{enumerate}

\subsubsection{CO$_2$ Emissions}
CO$_2$ emissions per unit of electrical energy:\\Coal-fired electricity $>$ Coal synfuel $>$ Coal $>$ Tar sand $>$ Oil $>$ Natural gas $>$ Nuclear $>$ Geothermal
\colbreak

\subsection{Nuclear Power}
Nuclear power is the only nonrenewable clean energy, formed from nuclear fission:
\begin{enumerate}[\roman*.]
  \item Role: large-scale baseload with \emph{no direct} CO$_2$ at the plant; constrained by cost, low net energy yield (full fuel cycle), radioactive wastes, safety/security concerns.
  \item Basics: induced fission chain reaction in a controlled reactor; moderators and control rods sustain steady power.
  \item Fuel Cycle: natural uranium (99.28\% $^{238}$U, 0.71\% $^{235}$U) enriched to $\sim$3.5--5\% $^{235}$U $\rightarrow$ UO$_2$ pellets $\rightarrow$ fuel rods/assemblies.
  \item Process: reactor heat $\rightarrow$ steam $\rightarrow$ turbine; cooling via once-through water or cooling towers.
  \item Accidents: Three Mile Island (1979), Chernobyl (1986), Fukushima (2011) led to upgraded safety standards and regulatory scrutiny.
\end{enumerate}

\subsection{Energy Efficiency}
A large share of commercial energy is wasted; roughly half is avoidable through better tech/operations.

Techniques:
\begin{enumerate}[\roman*.]
  \item Industry: cogeneration/ combined heat and power (CHP), replacing energy-wasting electri motors with variable speed motors, recycling materials, swithing to LED lighting 
  \item Transport: hybrids/EVs, lightweighting, aerodynamics, public transit \& mode shift; hydrogen fuel cells.
  \item Buildings: solar heating, superinsulation for heating, high-efficiency HVAC, green roofs, daylighting, sustainable building materials, straw bale houses.
\end{enumerate}

Waste still exists due to artificially cheap fossil fuels (due to subsidies and exclusion of environmental costs), few incentives and inadequate energy-efficiency building codes and appliance standards.

\colbreak
\section{Renewable Energy}
Renewable Energy (RE) is replenished by natural processes at a rate equal or exceeding rate of use.
\begin{enumerate}[\roman*.]
  \item Drivers: growing energy demand, environmental concerns, economic benefits, energy security, cost competitiveness 
  \item Needs: grid flexibility (storage, demand response, transmission), modern market design, land and water management, biodiversity trade-offs.
\end{enumerate}

\subsection{Solar Energy}
Solar is the largest energy source, inexhaustible, and captured as heat (absorption by gases/liquids/solids) or photoreaction (photon-driven electron flow).

Photovoltaic (PV)/ Solar cells generate electricity through the photovoltaic effect; sunlight on semiconductor releases electrons, and current flows through an external circuit:
\begin{enumerate}[\roman*.]
  \item Cosine Effect: solar irradiance is maximal when sun is directly overhead, and reduced at anglessolar 
  \item Variance: solar irradiance varies with latitudes (high variance at high altitudes, low in tropics); dependent on season and weather
  \item Tilt: face the equator; summer tilt $<$ latitude to maximize capture; winter tilt $>$ latitude for higher need months
  \item Applications: rooftop (residential/commercial), utility-scale ground-mount, building-integrated PV, floating PV on reservoirs (dual use, red. evaporatn)
    \begin{itemize}[leftmargin=*]\vspace{2pt}
      \item {Topaz Solar Farm}: 550\,MW, powers $\sim$160{,}000 homes; avoids $\sim$377{,}000 t\,CO$_2$/yr.
      \item {Marina Barrage Solar Park}: 405 modules; $\sim$76{,}000\,kWh/yr for onsite loads. 
      \item {Floating PV}: Huainan 40-150\,MW; Yamakura Dam 13.7\,MW; Sanshan 8.5\,MW.
    \end{itemize}
\end{enumerate}\vspace{-1pt}
\begin{enumerate}[$+$]
  \item High net energy yield, quick installation, easily expanded, no CO$_2$, long lasting
\end{enumerate}\vspace{-1pt}
\begin{enumerate}[$-$]
  \item Need access to sun, need electricity storage system or backup, environmental costs not included, expensive, high land use, must convert DC to AC
\end{enumerate}
\vspace{-1em}
\colbreak
Concentrating Solar Power (CSP) is the generation of electricity through optical concentration of solar energy, producing high temperature fluids or materials:
\begin{enumerate}[\roman*.]
  \item Resultant output is used to drive heat engines and electrical generators 
\end{enumerate}

Solar Thermal uses solar energy for heating purporses:
\begin{enumerate}[\roman*.]
  \item Solar panels of evacuated tubes or plate-plate collectors heat up water tanks, or for space heating
\end{enumerate}

Solar Fuels involve conversion of sunlight into storable chemical fuels (e.g. hydrogen or synthetic hydrocarbons) through photochemical/photobiological, artificial photosynthesis or thermochemical:
\begin{enumerate}[\roman*.]
  \item Artificial Photosynthesis: converts law materials like water and CO$_2$ into clean fuels and value-added chemicals (e.g. H$_2$, and hydrocarbons) 
  \item Photocatalytic water splitting converts water into hydrogen ions and oxygen, and is an active research area in artificial photosynthesis.
  \item Can also be used as feedstock in (the chemical) industry.
  \item Combined with fuel cell tech which convert fuel to electricity and heat, can power a building or small community
\end{enumerate}
\colbreak

\subsection{Hydropower}
Converts potential/kinetic energy of water to electricity.
\begin{enumerate}[$+$]
  \item Moderate-high net energy, high efficiency, large untapped potential, low-cost, long lifespan, no CO$_2$ emissions in temperatee areas, can provide flood control, provides irrigation water
\end{enumerate}\vspace{-1pt}
\begin{enumerate}[$-$]
  \item High construction cost, environmental impacts (flooding, fish), danger of collapse, social displacement, silting reduces efficiency 
\end{enumerate}

\subsection{Wind Energy}
Converts kinetic energy of air into electricity via lift on rotating blades in wind turbines.
\begin{enumerate}[\roman*.]
  \item Ex. Gansu, China: largest wind farm; 6,000 MW as of 2012, goal of 20,000 MW by 2020
  \item Ex. Londay Array: biggest offshore farm; 630 MW
\end{enumerate}\vspace{-1pt}
\begin{enumerate}[$+$]
  \item Moderate-high net energy, high efficiency, low environmental impact, no CO$_2$ emissions, easy expansion, can be located at sea 
\end{enumerate}\vspace{-1pt}
\begin{enumerate}[$-$]
  \item Variable power suuply, needs backup systems, plastic components produced from oil, high land use, visual pollution, noise, can kill birds and disrupt migratory patterns 
\end{enumerate}


\subsection{Geothermal Energy}
Converts generated steam to drive turbines:
\begin{enumerate}[\roman*.]
  \item Direct Uses: district heating, industry, greenhouses; ground-source heat pumps for efficient HVAC.
\end{enumerate}\vspace{-1pt}
\begin{enumerate}[$+$]
  \item Moderate net energy at accessible sites, very high efficiency, moderate environmental impact, lower CO$_2$ emissions, low land use, low cost at some sites
\end{enumerate}\vspace{-1pt}
\begin{enumerate}[$-$]
  \item Scarcity of suitable sites, can be depleted, moderate to high local air pollution, noise and odor, usually high cost
\end{enumerate}

\colbreak
\subsection{Ocean Energy}
Harnesses waves, tidal range (barrages), tidal currents, ocean currents, ocean thermal energy (OTEC), and salinity gradients (osmosis).
\begin{enumerate}[\roman*.]
  \item Ex. oscillating water columns, point absorbers, attenuators (wave); barrages/lagoon schemes and in-stream turbines (tidal); warm-cold cycle heat engines (OTEC).
\end{enumerate}\vspace{-1pt}
\begin{enumerate}[$+$]
  \item Predictable tides, proximity to coastal loads, large potential; complements solar/wind
\end{enumerate}\vspace{-1pt}
\begin{enumerate}[$-$]
  \item High cost, harsh marine environment, ecological constraints
\end{enumerate}

\subsection{Bioenergy}
Energy from biomass/biogas and liquid biofuels.
\begin{enumerate}[\roman*.]
  \item Biomass: residues/wood pellets for heat/power.
  \item Biogas: anaerobic digestion (wastewater, landfills, agri/food waste) $\rightarrow$ CH$_4$ for CHP or upgrading to biomethane.
  \item Biofuels: ethanol (sugar/starch), biodiesel (oils/fats), advanced drop-in fuels (cellulosic, SAF).
\end{enumerate}\vspace{-1pt}
\begin{enumerate}[$+$]
  \item Valorises wastes, cut landfill methane; lower net CO$_2$ if sustainably sourced; rural jobs and energy
\end{enumerate}\vspace{-1pt}
\begin{enumerate}[$-$]
  \item Water use, fertilizer runoff, competition for food, land availability, destruction of forests, conversion technology issues 
\end{enumerate}

\subsection{Hydrogen Energy}
Hydrogen is a fuel produced by using energy, storable in fuel cells:
\begin{enumerate}[$+$]
  \item Producible with water, low environmental impact, renewable, no CO$_2$ from water production, good substitute for oil, competitive price considering env. cost, easy storage, safety, nontoxic, high efficiency 
\end{enumerate}\vspace{-1pt}
\begin{enumerate}[$-$]
  \item Not found in nature, energy is needed to produce it, negative net energy, will need generated by renewable fuels, technology immature, excessive H$_2$ leaks hurt ozone
\end{enumerate}

\colbreak
\subsection{Grid Integration \& Storage}
Flexibility tools to match variable RE with demand.
\begin{enumerate}[\roman*.]
  \item Storage: batteries (short-duration), pumped hydro (long), thermal (molten salt, chilled/heat), emerging hydrogen storage.
  \item System tools: demand response, forecasting, interconnection/transmission, curtailment, smart inverters (volt/VAR, grid-forming).
  \item Hybrids: solar+storage, wind+storage, RE+diesel (remote), virtual power plants (aggregate distributed resources).
\end{enumerate}\vspace{-1pt}
\begin{enumerate}[$+$]
  \item Improves reliability and RE penetration; provides ancillary services; lowers curtailment; enables resilient microgrids
\end{enumerate}\vspace{-1pt}
\begin{enumerate}[$-$]
  \item Capital cost and siting (batteries, pumped hydro, lines); permitting/interconnection delays; operational complexity at high RE shares
\end{enumerate}

\subsection{Economics \& Policy}
Enablers for scaling RE deployment.
\begin{enumerate}[\roman*.]
  \item Costs \& learning: PV/wind/batteries show strong cost declines; at high shares, BOS/soft costs dominate.
  \item Markets \& finance: competitive auctions/PPAs reduce WACC; rooftop export rules (net billing) shape adoption.
  \item Standards \& incentives: RPS/RECs, FITs, tax credits; streamlined permitting and grid interconnection.
  \item Externalities: carbon/pollution pricing aligns private with social costs; just transition and workforce policies.
\end{enumerate}\vspace{-1pt}
\begin{enumerate}[$+$]
  \item Faster, cheaper scale-up; investment certainty; innovation and domestic industry growth
\end{enumerate}\vspace{-1pt}
\begin{enumerate}[$-$]
  \item Policy risk and inconsistency; equity/affordability concerns if design is poor; rebound or land-use conflicts without safeguards
\end{enumerate}

\colbreak
\section{Sustainable Water Resources}
\section{Zero Waste}
\section{Environmental Hazards and Health}

%%%%%%%%%%%%%%%%%%%%%%%%%%%%%%%%%%%%%%%%%%%%%%%%%%%%%%
%                       End                          %
%%%%%%%%%%%%%%%%%%%%%%%%%%%%%%%%%%%%%%%%%%%%%%%%%%%%%%
\end{multicols*}
\end{document}

\documentclass[12pt, a4paper]{article}

\usepackage[utf8]{inputenc}
\usepackage[mathscr]{euscript}
\let\euscr\mathscr \let\mathscr\relax
\usepackage[scr]{rsfso}
\usepackage{amssymb,amsmath,amsthm,amsfonts}
\usepackage[shortlabels]{enumitem}
\usepackage{multicol,multirow}
\usepackage{lipsum}
\usepackage{balance}
\usepackage{calc}
\usepackage[colorlinks=true,citecolor=blue,linkcolor=blue]{hyperref}
\usepackage{import}
\usepackage{xifthen}
\usepackage{pdfpages}
\usepackage{transparent}
\usepackage{tabularx}

\newcommand{\incfig}[2][1.0]{
    \def\svgwidth{#1\columnwidth}
    \import{./figures/}{#2.pdf_tex}
}
\newcommand{\incimg}[2][1.0]{
  \includegraphics[width=#1\columnwidth]{./figures/#2}
}


\usepackage{ifthen}
\usepackage[landscape]{geometry}
\usepackage[shortlabels]{enumitem}

\ifthenelse{\lengthtest { \paperwidth = 11in}}
    { \geometry{top=.5in,left=.5in,right=.5in,bottom=.5in} }
	{\ifthenelse{ \lengthtest{ \paperwidth = 297mm}}
		{\geometry{top=1cm,left=1cm,right=1cm,bottom=1cm} }
		{\geometry{top=1cm,left=1cm,right=1cm,bottom=1cm} }
	}

\pagestyle{empty}
\makeatletter
\renewcommand\thesection{\arabic{section}.}
\renewcommand{\section}{\@startsection{section}{1}{0mm}%
                                {-1ex plus -.5ex minus -.2ex}%
                                {0.05ex}%x
                                {\normalfont\normalsize\bfseries}}
\renewcommand{\subsection}{\@startsection{subsection}{2}{0mm}%
                                {-1ex plus -.5ex minus -.2ex}%
                                {0.05ex}%
                                {\normalfont\small\bfseries}}
\renewcommand{\subsubsection}{\@startsection{subsubsection}{3}{0mm}%
                                {-1ex plus -.5ex minus -.2ex}%
                                {0.05ex}%
                                {\normalfont\footnotesize\bfseries}}
\newcommand{\colbreak}{\vfill\null\columnbreak}
\makeatother
\setcounter{secnumdepth}{1}
\setlength{\parindent}{0pt}
\setlength{\parskip}{0.7em}

\setlist[itemize]{itemsep=0.6ex, topsep=-2pt, partopsep=0pt, parsep=0pt}
\setlist[enumerate]{itemsep=0.6ex, topsep=-2pt, partopsep=0pt, parsep=0pt}

\input{letterfonts}

\newcommand{\mytitle}{GE1040 A Culture of Sustainability}
\newcommand{\myauthor}{github/omgeta}
\newcommand{\mydate}{AY 25/26 Sem 1}

\begin{document}
\raggedright
\footnotesize
\begin{multicols*}{3}
\setlength{\premulticols}{1pt}
\setlength{\postmulticols}{1pt}
\setlength{\multicolsep}{1pt}
\setlength{\columnsep}{2pt}

{\normalsize{\textbf{\mytitle}}} \\
{\footnotesize{\mydate\hspace{2pt}\textemdash\hspace{2pt}\myauthor}}
\vspace{-0.5em}
%%%%%%%%%%%%%%%%%%%%%%%%%%%%%%%%%%%%%%%%%%%%%%%%%%%%%%
%                      Begin                         %
%%%%%%%%%%%%%%%%%%%%%%%%%%%%%%%%%%%%%%%%%%%%%%%%%%%%%%
\section{Unsustainable Development}
Unsustainable development harms the ability of future generations to meet their needs due to the degradation of our climate, life-support systems and resources.
\begin{enumerate}[\roman*.]
  \item Global Environmental Indicators: Planetary boundaries have already been exceeded (climate change, biodiversity loss, nitrogen cycle).
  \item Stability Landscape: Ecosystems may lose resilience — once thresholds are crossed (valley basins), systems may not return to their original state.
  \item Changing Climate: We are living in a changing climate. We need to mitigate impacts on us and ecosystem, or otherwise become more adaptable
\end{enumerate}

\subsubsection{Ecological Footprint}

Terminology:
\begin{enumerate}[\roman*.]
  \item Bioproductivity: amount and rate of production occuring in an ecosystem over a period of time.
  \item Biocapacity $=$ Area $\times$ Biocapacity: quantifies nature's capacity to produce renewable resources, provide land for built-up areas and provide waste absorption services such as carbon uptake.
  \item Ecological Footprint (gha) $=$ Population $\times$ Consumption/Person $\times$ Footprint Intensity: quantifies biological productive area needed for provision of renewable resources, or require absorption of $\text{CO}_2$ waste.
  \item Footprint and biocapacity can be ranked by countries, enabling them to learn from each other. 
  \item Ecological Deficit: ecological footprint$>$biocapacity (currently exceeded by 50\%).
\end{enumerate}

Earth Overshoot Day is the date when annual footprint exceeds annual biocapacity:
\begin{enumerate}[\roman*.]
  \item Occurs earlier each year (first at 1970)
  \item COVID-19 caused ecological footprint to contract
  \item \#MoveTheDate: push overshoot to December via sustainable cities, resilient systems, etc.
\end{enumerate}
\vspace{-1em}
\colbreak
\subsection{Causes}
Overpopulation due to exponential population growth:
\begin{enumerate}[\roman*.]
  \item 8.1b (2025) $\rightarrow$ 9.5b (2050) $\rightarrow$ 11b (2100) 
  \item Pressure on land, soil degredation, and biodiversity 
  \item Prior to Industrial Revolution, growth was resource-limited but now largely unchecked 
  \item Various demands by billions for a comfortable life, but everyone should have the equal right to health 
\end{enumerate}

Unsustainable Resource Use:
\begin{enumerate}[\roman*.]
  \item Affluenza: oversumption in affluent societies
  \item Developing countries fixated on GDP as sole success metric ignoring environmental costs
\end{enumerate}

Poverty links to environment in a downward spiral:
\begin{enumerate}[\roman*.]
  \item Direct reliance on food, water, fuel for survival
  \item Degenerate forests, soil, grasslands and wildlife causing environmental degradation
  \item Degraded environment further impoverishes people
\end{enumerate}

Excluding Environmental Costs:
\begin{enumerate}[\roman*.]
  \item Market prices ignore externalities such as ecosystem loss, health impacts and pollution 
  \item Ex. Timber companies pay to clear forests but not for environmental degradation and loss of habitat 
  \item Ex. Fishing companies pay to catch fish but not for depletion of fish stocks 
  \item Taxes and fines aim to fix this but not enough
\end{enumerate}

\subsubsection{Tragedy of the Commons}

Tragedy of the commons is the overuse of a common property or free-access resource causing depletion for all.
\begin{enumerate}[\roman*.]
  \item Mentality of "If I do not use it, someone else will. The little bit I use or pollute doesn't matter".
  \item Solutions:
    \begin{itemize}[leftmargin=*]\vspace{2pt}
      \item Responsible usage of shared renewed resources at rates well below sustainable yields
      \item Convert open-access renewable resources to private ownership
    \end{itemize}
\end{enumerate}

\colbreak
\subsubsection{IPAT Model}
IPAT quantifies environmental impact as $I = P \times A \times T$
where $P$ is population, $A$ is affluence, $T$ is technology:
\begin{enumerate}[\roman*.]
  \item Population: not dominant factor 
  \item Affluence: $\displaystyle\frac{\text{Goods \& Services}}{\text{Person}}$ can harm through high consumption, pollution and resource wastage but also produce funding for innovative R\&D (e.g. Denmark, India's ethanol-blended gasoline)
  \item Technology: $\displaystyle\frac{\text{Impact}}{\text{Goods \& Services}}$ reduces impact 
  \item Ex.: Gasoline = cars $\times$ miles/car $\times$ gasoline/mile
\end{enumerate}

\subsection{Pollution}
Pollution is the introduction of contaminants into the natural environment that adversely affects a resource. 
\begin{enumerate}[\roman*.]
  \item Point Source: single, identifiable source (e.g. smokestack, drainpipes)
  \item Nonpoint Source: dispersed and difficult to identify (e.g. fertilizer and pesticide runoff into lakes - first flush effect after dryspell)
\end{enumerate}

Health Effects:
\begin{enumerate}[\roman*.]
  \item Headache and Fatigue
  \item Respiratory Illness
  \item Cardiovascular Illness
  \item Cancer Risk
  \item Nausea and Gastroenteritis
  \item Skin Irritation
\end{enumerate}

Management Methods:
\begin{enumerate}[\roman*.]
  \item Cleanup (end-of-pipe): clean/dillute contaminants
    \begin{enumerate}[leftmargin=*, label=$-$]\vspace{1pt}
      \item Temporary; growth in consumption may offset pollution control tech
      \item Often relocates pollutants to another area 
      \item Costly to clean dispersed pollutants
    \end{enumerate}
  \item Prevention (front-of-pipe): reduce/stop production
\end{enumerate}

\colbreak
\subsection{Environmental Viewpoints}
Planetary Management:
\begin{enumerate}[\roman*.]
  \item View: We are apart from the rest of nature and can manage it to meet our increasing demands.
  \item Resources: We will not run out, due to our ingenuity and technology.
  \item Economy: Potential for economic growth is essentially unlimited.
  \item Success: Depends on how well we manage the earth's life-support systems for our benefit.
\end{enumerate}

Stewardship:
\begin{enumerate}[\roman*.]
  \item View: We have an ethical responsibility to be caring stewards of the earth.
  \item Resources: We will probably not run out, but they should not be wasted.
  \item Economy: Encourage environmentally friendly economic growth and discourage harmful forms.
  \item Success: Depends on our managing the earth's life-support systems for our and nature's benefit. 
\end{enumerate}

Environmental Wisdom:
\begin{enumerate}[\roman*.]
  \item View: We are part of and totally dependent on nature, and nature exists for all species. 
  \item Resources: Limited and should not be wasted.
  \item Economy: Encourage earth-sustaining economic growth and discourage earth-degrading forms.
  \item Success: Depends on learning how nature sustains itself, integrating them into how we think and act.
\end{enumerate}

\colbreak
\section{Principles and Practical Applications of Sustainability}
Environmentally sustainable societies meet present needs without compromising future generations' own needs:
\begin{enumerate}[\roman*.]
  \item Without destroying the environment
  \item Without endangering the future welfare of the planet and its people 
  \item In a just and equitable manner
\end{enumerate}

Sustainability is the ability of Earth's natural systems, cultural systems and economies to survive and adapt to changing environmental conditions indefinitely. 3 Pillars:
\begin{enumerate}[\roman*.]
  \item Environment (ecological integrity)
  \item Economy (economic viability)
  \item Society (equity)
\end{enumerate}

\subsubsection{Sustainable Development Goals (SDGs)}

UN Sustainable Development Goals (SDGs) were adopted in 2015 by 2500 scientists from 190 nations, providing a global framework to steer towards a safe and just operating space for society to thrive in until 2030:
\begin{enumerate}[\arabic*.]
  \item No Poverty 
  \item Zero Hunger
  \item Good Health and Well-being 
  \item Quality Education 
  \item Gender Equality
  \item Clean Water and Sanitation
  \item Affordable and Clean Energy
  \item Decent Work and Economic Growth 
  \item Industry, Innovation and Infrastructure
  \item Reduce Inequality
  \item Sustainable Cities and Communities
  \item Responsible Consumption and Production 
  \item Climate Action 
  \item Life below Water 
  \item Life on Land 
  \item Peace and Justice Strong Institutions
  \item Partnerships to achieve the Goal
\end{enumerate}
\vspace{-1em}
\colbreak
\subsubsection{International Spillovers}
Spillovers are transboundary negative impacts generated by one country on others, which can undermine their ability to achieve the SDGs, measured by Spillover Score. Types of spillovers:
\begin{enumerate}[\roman*.]
  \item Environmental: use of natural resources and pollution, including transboundary effects embodied in trade, and direct cross-border flows in air and water
  \item Economic/Financial/Governance: international development finance, unfair tax competition, banking secrecy, and labor standards
  \item Security: negative externalities such as arms trade and organized crime destabilizing poorer countries; positive spillovers include conflict-prevention and peacekeeping investments
\end{enumerate}

\subsubsection{Natural Capital}
Natural capital (natural resources $+$ natural services) refers to the stock of natural resources and ecosystem services that sustain human life :  
\begin{enumerate}[\roman*.]
  \item Natural Resources: includes air, water, soil, land, life (biodiversity), nonrenewable resources, renewable energy and nonrenewable energy
  \item Natural Services: includes air purification, water purification, water storage, soil renewal, nutrient recycling, food production, conservation of biodiversity, wildlife habitat, forest renewal, waste treatment, climate control, population control and pest control
  \item Degradation of natural capital undermines long-term sustainability
  \item Preserving natural capital is essential for intergenerational equity
\end{enumerate}


\colbreak
\subsection{Sustainability Concepts}
Shifting Emphasis:
\begin{enumerate}[\roman*.]
  \item Pollution cleanup $\rightarrow$ Pollution prevention
  \item Waste disposal $\rightarrow$ Waste prevention and reduction
  \item Species protection $\rightarrow$ Habitat protection
  \item Environ. degradation $\rightarrow$ Environ. restoration
  \item Increased resource use $\rightarrow$ Less wasteful resource use
  \item Population growth $\rightarrow$ Population stabilization by decreasing birth rates
  \item Depleting and degrading natural capital $\rightarrow$ Protecting natural capital, living off bio-interest
\end{enumerate}

Lessons from Nature:
\begin{enumerate}[\roman*.]
  \item Runs on renewable solar energy $\rightarrow$ Rely mostly on renewable solar energy
  \item Recycles nutrients and wastes (little waste) $\rightarrow$ Prevent/reduce pollution, recycle \& reuse resources
  \item Uses biodiversity to maintain and adapt to environmental change $\rightarrow$ Preserve biodiversity by protecting ecosystem services, habitats and species
  \item Controls species' population size and resource use $\rightarrow$ Reduce births and wasteful resource use to prevent environmental overload, and depletion and degradation of resources 
\end{enumerate}

Challenges:
\begin{enumerate}[\roman*.]
  \item Depletion of finite resources (fossil fuels, soil, minerals, species)
  \item Overuse of renewable resources (forests, fish \& wildlife, soil fertility, public funds)
  \item Pollution (air, water, soil)
  \item Inequity (economic, political, social, gender)
  \item Species loss (endangered species and spaces)
\end{enumerate}

Solutions:
\begin{enumerate}[\roman*.]
  \item Cyclical use of resources (emulate nature; 3R's)
  \item Safe reliable energy (conservation, renewable energy, subtitution, interim measures)
  \item Human well-being interests (health, creativity, learning, cultural and spiritual development)
\end{enumerate}

\subsection{Research \& Development (R\&D)}
Examples of sustainability-oriented research and indigenous technology development:  
\begin{enumerate}[\roman*.]
  \item Power Generation: Biogas integrated with waste management/Co-Gen systems
  \item Construction Materials: Local, non-toxic, reusable materials; water as material for thermal walls
  \item Water Supply: Rainwater harvesting for groundwater recharge and building cooling
  \item Water Treatment: Natural biomaterials for turbidity removal; UV disinfection from sunlight; fabric filtration for point-of-use treatment
  \item Storm-water Management: Green roofs for runoff reduction, reduced energy use, and cooling effect
\item Building Design: Passive solar design, right-sized homes, cost-effective ventilation, maximize storage and comfort with minimal energy
\end{enumerate}

Green Roofs are intensive (thick substrate, shrubs/trees) or extensive (thin substrate, smaller plants):
\begin{enumerate}[\roman*.]
  \item Aesthetically pleasing
  \item Reduce storm-water runoff 
  \item Reduce urban heat island effects 
  \item Reduce air conditioning costs 
  \item Negate acid rain effects 
  \item Reduce $\text{CO}_2$ impact 
  \item Create habitats for certain plants and animals
  \item Cooling Effect: protects from solar radiation, stabilizes roof temperature, cools building interiors
  \item Water Quality: depends on substrate layer, vegetation type, fertilisation quality, roof age, surrounding area type, local pollution sources
\end{enumerate}

Key Carbon Considerations:  
\begin{enumerate}[\roman*.]
  \item Embodied: emissions from construction materials
  \item Operational: emissions from running processes
\end{enumerate}
\colbreak
\subsection{Case Study: NUS Sustainability Initiative}
Current Practices:
\begin{enumerate}[\roman*.]
  \item Integrated into education, research, campus operations, and leadership
  \item Contributes directly to multiple SDGs such as climate action, clean energy, and sustainable cities
\end{enumerate}

Future:
\begin{enumerate}[\roman*.]
  \item Driving sustainability via collaboration with other sectors (internal and external partners)
  \item Building climate resilience by linking research across climate, urban, economic, and social areas
\end{enumerate}

\subsection{Case Study: Punggol Digital District (PDD)}
Current Practices:
\begin{enumerate}[\roman*.]
  \item 17 Green Mark Platinum and 3 Super Low Energy Buildings (e.g. Mass Engineered Timber with 98\% lower embodied carbon)
  \item Smart Energy Grid: real-time data management optimisation saves 1,700 tonnes CO$_2$ annually; rooftop PV panels generate 3,000 MWh annually
  \item Open Digital Platform (ODP): collect and analyse environmental and building data to improve energy efficiency, reduce costs and minimise impact
  \item Digital Twin: real-time planning-simulation model
  \item Centralised Cooling System: cooling towers with underground distribution reduce energy use by $30\%$ and 3,700--4,000 tonnes CO$_2$ annually
  \item Carlite District: prioritises low-emission transport
\end{enumerate}

Future:
\begin{enumerate}[\roman*.]
  \item Environmental Modelling
  \item Use of recycled materials for construction
  \item Fuel Cell System for Lifts 
  \item Regenerative Lift converts motions into energy
  \item Centralized Chutes for Recyclables
  \item "Develop an eco-town with a human settlement that enables its residents to live a
good quality of life while using minimal natural resources."
\end{enumerate}

\colbreak
\section{Circular Economy}
Circular economy is a regenerative system where resource input, waste, emissions, and energy leakage are minimised by slowing, closing, and narrowing resource loops.  
\begin{enumerate}[\roman*.]
  \item Replaces the end-of-life concept with long-lasting design, maintenance, repair, reuse, remanufacturing, refurbishing, and recycling
  \item Regenerate natural systems by returning valuable nutrients to ecosystems
  \item Supports economic, environmental, and social dimensions of sustainable development
  \item Mimics nature's cyclical systems 
\end{enumerate}

Economy Progression:
\begin{enumerate}[\roman*.]
  \item Linear Economy: take $\rightarrow$ make $\rightarrow$ dispose (single-use, wasteful)
  \item Recycling Economy: partial recovery through recycling but still wasteful
  \item Circular Economy: design for closed loops, resources kept in circulation, waste designed out
\end{enumerate}

\subsection{Energy and Matter}

High-quality energy is concentrated, good for useful work.
Low-quality energy is dispersed, bad for useful work.

Three Big Ideas:
\begin{enumerate}[\roman*.]
  \item There is no "away": matter cannot be destroyed, only converted\hfill(Law of Conservation)
  \item Cannot get something for nothing; cannot get out more energy than in\hfill(1st Law of Thermodynamics)
  \item Cannot break even; energy conversion from one form to another reduces energy quality or less usable energy\hfill(2nd Law of Thermodynamics)
\end{enumerate}

Economy Types:
\begin{enumerate}[\roman*.]
  \item High-throughput economy: unsustainable, high-waste, promotes pollution
  \item Low-throughput economy: matter-recycling and reuse; mimics nature and reduces pollutants but may be insufficient for growing populations
\end{enumerate}

\colbreak
\subsection{Industrial Symbiosis}

Industrial symbiosis is a business relationship focused on sharing resources between industrial facilities/companies where wastes or byproducts from one become raw materials for another.
\begin{enumerate}[\roman*.]
  \item Subset of Industrial Ecology: shift from linear to cyclical (closed-loop) systems
  \item Mutually beneficial cooperation which reduces environmental impact while improving business competitiveness
  \item Key enabling business model for advancing the move to a circular economy
  \item Ex. Kalundborg, Denmark: cooperation among 8 industries sharing energy, water, and materials 

\end{enumerate}

\subsection{Applications and Innovations}
\begin{enumerate}[\roman*.]
  \item Bioplastics: key enabler of low-carbon circular economy; allow closed resource cycles and cascading reuse especially when being reused or recycled
  \item Design for longevity: durability, easy repair, remanufacturing
  \item Closed resource cycles: minimize virgin resource extraction via reuse and recovery
  \item Resource Recovery: convert waste streams into valuable resources (e.g., waste-to-energy, nutrient recovery, material cascading)
\end{enumerate}
\colbreak
\section{Sustainable Urbanisation}
\subsection{Case Study: Stockholm}
\subsection{Case Study: Singapore}
\colbreak
\section{Sustainable Infrastructure}
\section{Air Quality}
\section{Climate Change}
\section{Nonrenewable Energy}
\section{Renewable Energy}
\section{Sustainable Water Resources}
\section{Zero Waste}
\section{Environmental Hazards and Health}

%%%%%%%%%%%%%%%%%%%%%%%%%%%%%%%%%%%%%%%%%%%%%%%%%%%%%%
%                       End                          %
%%%%%%%%%%%%%%%%%%%%%%%%%%%%%%%%%%%%%%%%%%%%%%%%%%%%%%
\end{multicols*}
\end{document}

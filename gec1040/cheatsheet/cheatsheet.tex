\documentclass[12pt, a4paper]{article}

\usepackage[utf8]{inputenc}
\usepackage[mathscr]{euscript}
\let\euscr\mathscr \let\mathscr\relax
\usepackage[scr]{rsfso}
\usepackage{amssymb,amsmath,amsthm,amsfonts}
\usepackage[shortlabels]{enumitem}
\usepackage{multicol,multirow}
\usepackage{lipsum}
\usepackage{balance}
\usepackage{calc}
\usepackage[colorlinks=true,citecolor=blue,linkcolor=blue]{hyperref}
\usepackage{import}
\usepackage{xifthen}
\usepackage{pdfpages}
\usepackage{transparent}
\usepackage{tabularx}

\newcommand{\incfig}[2][1.0]{
    \def\svgwidth{#1\columnwidth}
    \import{./figures/}{#2.pdf_tex}
}
\newcommand{\incimg}[2][1.0]{
  \includegraphics[width=#1\columnwidth]{./figures/#2}
}


\usepackage{ifthen}
\usepackage[landscape]{geometry}
\usepackage[shortlabels]{enumitem}

\ifthenelse{\lengthtest { \paperwidth = 11in}}
    { \geometry{top=.5in,left=.5in,right=.5in,bottom=.5in} }
	{\ifthenelse{ \lengthtest{ \paperwidth = 297mm}}
		{\geometry{top=1cm,left=1cm,right=1cm,bottom=1cm} }
		{\geometry{top=1cm,left=1cm,right=1cm,bottom=1cm} }
	}

\pagestyle{empty}
\makeatletter
\renewcommand\thesection{\arabic{section}.}
\renewcommand{\section}{\@startsection{section}{1}{0mm}%
                                {-1ex plus -.5ex minus -.2ex}%
                                {0.05ex}%x
                                {\normalfont\normalsize\bfseries}}
\renewcommand{\subsection}{\@startsection{subsection}{2}{0mm}%
                                {-1ex plus -.5ex minus -.2ex}%
                                {0.05ex}%
                                {\normalfont\small\bfseries}}
\renewcommand{\subsubsection}{\@startsection{subsubsection}{3}{0mm}%
                                {-1ex plus -.5ex minus -.2ex}%
                                {0.05ex}%
                                {\normalfont\footnotesize\bfseries}}
\newcommand{\colbreak}{\vfill\null\columnbreak}
\makeatother
\setcounter{secnumdepth}{1}
\setlength{\parindent}{0pt}
\setlength{\parskip}{0.7em}

\setlist[itemize]{itemsep=0.6ex, topsep=-2pt, partopsep=0pt, parsep=0pt}
\setlist[enumerate]{itemsep=0.6ex, topsep=-2pt, partopsep=0pt, parsep=0pt}

\input{letterfonts}

\newcommand{\mytitle}{GE1040 A Culture of Sustainability}
\newcommand{\myauthor}{github/omgeta}
\newcommand{\mydate}{AY 25/26 Sem 1}

\begin{document}
\raggedright
\footnotesize
\begin{multicols*}{3}
\setlength{\premulticols}{1pt}
\setlength{\postmulticols}{1pt}
\setlength{\multicolsep}{1pt}
\setlength{\columnsep}{2pt}

{\normalsize{\textbf{\mytitle}}} \\
{\footnotesize{\mydate\hspace{2pt}\textemdash\hspace{2pt}\myauthor}}
\vspace{-0.5em}
%%%%%%%%%%%%%%%%%%%%%%%%%%%%%%%%%%%%%%%%%%%%%%%%%%%%%%
%                      Begin                         %
%%%%%%%%%%%%%%%%%%%%%%%%%%%%%%%%%%%%%%%%%%%%%%%%%%%%%%
\section{Unsustainable Development}
In a changing climate

\subsection{Causes}
Overpopulation is caused by exponential population growth, with prediction to reach 8.1bil by 2025, 9.5bil by mid-century and 11bil by 2100.
\begin{align*}
  \item 
\end{align*}
Climate has already changed -> we are living in a changing climate 
Mitigate impacts on us and ecosystem
If mitigation is not possible or practical, can we make ourselves more adaptable?
Fertilisers not being used for intendedd purposes (promote agricultural protection) due to rain, agricultural runoff affecting 
Biodiversity loss due to Urbanisation
Valley basins, lack of resilience, items cannot return to their original states

Companies were generally profit-driven, less caring about Sustainability
Now, carbon-tax, green-washing encourage companies to be green

Overpopulation, 10 bn people competing to improve their own living standards
Prior to industrial revolution, growth waws limited
All people should have the equal right to protect their own Health
Environment should be capable of assimilating our waste generated 
Various demands for a comfortable life

Some countries have more biocapacity than ecological footprint, neutral or opposite (can be ranked)
Different countries can learn from each other

Ecological deficit = ecological footprint > biocapacity (currently exceeded by 50\%)
Need to take more proactive measures
Overconsumption across the world, developing countries competing to boost their economies (healthy competition) by fixating on economic growth as the sole indicator of success causes environmental damage

COVID-19 caused contractions, making overshoot day decreased

\#MoveTheDate to end of december ideally: make cities more sustainable, climate resilient, liveable, etc

Tragedy of the commons: if everyone collectively prioritised their own lifestyles, it will lead to catastrophe
Responsibility: use shared renewed resources at rates WELL below sustainable yields 
Convert open-access resources to private ownership


Quantitative: I= PAT 
Total Human Impact = Population x Affluence x Technology (unsustainable but cheap used for economic progress)
Population is not the dominating factor
E.g. Gasoline = cars x miles driven per car x gasline per miles
In India now, EX gasoline which mixes with ethanol reduces greatly the impact of gasoline
Affluence can also be used for benefit goals (e.g. R\&D as in Denmark)
Poverty = Income Inequality also contribute 

Pollution:
First Flush Effect after dryspell caused sudden pollution event -> kill fish coming from non-point sources 
\subsection{Environmental Viewpoints}

Planetary Management:
\begin{enumerate}[\roman*.]
  \item View: we are apart from the rest of nature and can manage it to meet our increasing demands.
  \item Resources: we will not run out, due to our ingenuity and technology.
  \item Economy: potential for economic growth is essentially unlimited.
  \item Success: depends on how well we manage the earth's life-support systems for our benefit.
\end{enumerate}

Stewardship:
\begin{enumerate}[\roman*.]
  \item View: We have an ethical responsibility to be caring stewards of the earth.
  \item Resources: we will probably not run out, but they should not be wasted.
  \item Economy: encourage environmentally friendly economic growth and discourage environmentally harmful forms.
  \item Success: depends on how well we manage the earth's life-support systems for our benefit and for the rest of nature.
\end{enumerate}

Stewardship:
\begin{enumerate}[\roman*.]
  \item View: We are part of and totally dependent on nature, and nature exists for all species. 
  \item Resources: limited and should not be wasted.
  \item Economy: encourage earth-sustaining economic growth and discourage earth-degrading forms.
  \item Success: depends on learning how nature sustains itself and integrating them into the ways we think and act.
\end{enumerate}
\colbreak
\section{Principles and Practical Applications of Sustainability}

SDGs by UN IPCC 2500 scientists from 190 nations across the world
Working together for policy changes to tackle global climate change

Explore nature-based solutions for everything: e.g. mangroves for seawalls, roadside hedges to absorb polluats / vegetation barrier against traffic 

Embodied carbon - as a result of construction materials
Operational carbon - from running of processes

Punggol:
Digital Twin - Virtual object, helps to simulate environmental impacts and helps planning, better preparation
Cooling System - Centralised cooling tower and underground distribution, supplies cooling water to buildings for operations - reduce current by 30\% and consequently 4000 tonnes of CO\_2
Carlite District - reduce emissions
Need locally generated knowledge -> need to solve problems by our own, external knowledge can catalyze change but eventually need to be independent

Vision/Experiments:
Fuel cells with hydrogen -> clean energy
Fluorescent lamps/LED -> 2/4x more energy efficient

Fertilizers can cause washoff


Long-term planning - every 5 years new master plan 

\section{Circular Economy}

Sustainability is the movement to preserve life w
Natural capital = natural resources + ecosystem services

\section{Sustainable Urbanisation}
\section{Sustainable Infrastructure}
\section{Air Quality}
\section{Climate Change}
\section{Nonrenewable Energy}
\section{Renewable Energy}
\section{Sustainable Water Resources}
\section{Zero Waste}
\section{Environmental Hazards and Health}

\section{Introduction}
Interdisciplinnary, consequences of socio-economic trends -> earth-system trends, if cities can transform we will be better positioned

Economy bad during COVID -> unemployment -> less carbon emissions
Sustainable cities, renewable carbon-free energy 
Reduce import of foods even with lack of space
Who overconsumes in the population
Shift overshoot date earlier
Need to maintain or improve liveability
Urbanization is inevitable, is it possible without pollution? Possible provided 100\% renewable energy

\colbreak
%%%%%%%%%%%%%%%%%%%%%%%%%%%%%%%%%%%%%%%%%%%%%%%%%%%%%%
%                       End                          %
%%%%%%%%%%%%%%%%%%%%%%%%%%%%%%%%%%%%%%%%%%%%%%%%%%%%%%
\end{multicols*}
\end{document}

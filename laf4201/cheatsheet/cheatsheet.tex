\documentclass[12pt, a4paper]{article}

\usepackage[utf8]{inputenc}
\usepackage[mathscr]{euscript}
\let\euscr\mathscr \let\mathscr\relax
\usepackage[scr]{rsfso}
\usepackage{amssymb,amsmath,amsthm,amsfonts}
\usepackage[shortlabels]{enumitem}
\usepackage{multicol,multirow}
\usepackage{lipsum}
\usepackage{balance}
\usepackage{calc}
\usepackage[colorlinks=true,citecolor=blue,linkcolor=blue]{hyperref}
\usepackage{import}
\usepackage{xifthen}
\usepackage{pdfpages}
\usepackage{transparent}
\usepackage{tabularx}

\newcommand{\incfig}[2][1.0]{
    \def\svgwidth{#1\columnwidth}
    \import{./figures/}{#2.pdf_tex}
}
\newcommand{\incimg}[2][1.0]{
  \includegraphics[width=#1\columnwidth]{./figures/#2}
}


\usepackage{ifthen}
\usepackage[landscape]{geometry}
\usepackage[shortlabels]{enumitem}

\ifthenelse{\lengthtest { \paperwidth = 11in}}
    { \geometry{top=.5in,left=.5in,right=.5in,bottom=.5in} }
	{\ifthenelse{ \lengthtest{ \paperwidth = 297mm}}
		{\geometry{top=1cm,left=1cm,right=1cm,bottom=1cm} }
		{\geometry{top=1cm,left=1cm,right=1cm,bottom=1cm} }
	}

\pagestyle{empty}
\makeatletter
\renewcommand\thesection{\arabic{section}.}
\renewcommand{\section}{\@startsection{section}{1}{0mm}%
                                {-1ex plus -.5ex minus -.2ex}%
                                {0.05ex}%x
                                {\normalfont\normalsize\bfseries}}
\renewcommand{\subsection}{\@startsection{subsection}{2}{0mm}%
                                {-1ex plus -.5ex minus -.2ex}%
                                {0.05ex}%
                                {\normalfont\small\bfseries}}
\renewcommand{\subsubsection}{\@startsection{subsubsection}{3}{0mm}%
                                {-1ex plus -.5ex minus -.2ex}%
                                {0.05ex}%
                                {\normalfont\footnotesize\bfseries}}
\newcommand{\colbreak}{\vfill\null\columnbreak}
\makeatother
\setcounter{secnumdepth}{1}
\setlength{\parindent}{0pt}
\setlength{\parskip}{0.7em}

\setlist[itemize]{itemsep=0.6ex, topsep=-2pt, partopsep=0pt, parsep=0pt}
\setlist[enumerate]{itemsep=0.6ex, topsep=-2pt, partopsep=0pt, parsep=0pt}

\input{letterfonts}

\newcommand{\mytitle}{LAF4201 French 5}
\newcommand{\myauthor}{github/omgeta}
\newcommand{\mydate}{AY 24/25 Sem 1}

\begin{document}
\raggedright
\footnotesize
\begin{multicols*}{3}
\setlength{\premulticols}{1pt}
\setlength{\postmulticols}{1pt}
\setlength{\multicolsep}{1pt}
\setlength{\columnsep}{2pt}

{\normalsize{\textbf{\mytitle}}} \\
{\footnotesize{\mydate\hspace{2pt}\textemdash\hspace{2pt}\myauthor}}

%%%%%%%%%%%%%%%%%%%%%%%%%%%%%%%%%%%%%%%%%%%%%%%%%%%%%%
%                      Begin                         %
%%%%%%%%%%%%%%%%%%%%%%%%%%%%%%%%%%%%%%%%%%%%%%%%%%%%%%
\section{Le temps}
\subsection{Présent de l'indicatif}
\begin{center}
\begin{tabular}{|c|c|c|c|}
  \hline
  & \textbf{-er} & \textbf{-ir} & \textbf{-re}\\\hline
  \textbf{Je} & parle & finis & vends \\\hline
  \textbf{Tu} & parles & finis & vends \\\hline
  \textbf{Il} & parle & finit & vend \\\hline
  \textbf{Nous} & parlons & finissons & vendons \\\hline
  \textbf{Vous} & parlez & finissez & vendez \\\hline
  \textbf{Ils} & parlent & finissent & vendent \\\hline
\end{tabular}
\end{center}

\begin{center}
\begin{tabular}{|c|c|c|c|c|}
  \hline
  & \textbf{avoir} & \textbf{être} & \textbf{faire} & \textbf{aller}\\\hline
  \textbf{Je} & ai & suis & fais & vais\\\hline
  \textbf{Tu} & as & es & fais & vas\\\hline
  \textbf{Il} & a & est & fait & va\\\hline
  \textbf{Nous} & avons & sommes & faisons & allons\\\hline
  \textbf{Vous} & avez & êtes & faites & allez \\\hline
  \textbf{Ils} & ont & sont & font & vont \\\hline
\end{tabular}
\end{center}
\begin{center}
\begin{tabular}{|c|c|c|c|c|}
  \hline
  & \textbf{voir} & \textbf{devoir} & \textbf{pouvoir} & \textbf{vouloir}\\\hline
  \textbf{Je} & vois & dois & peux & veux\\\hline
  \textbf{Tu} & vois & dois & peux & veux\\\hline
  \textbf{Il} & voit & doit & peut & veut\\\hline
  \textbf{Nous} & voyons & devons & pouvons & voulons\\\hline
  \textbf{Vous} & voyez & devez & pouvez & voulez \\\hline
  \textbf{Ils} & voient & doivent & peuvent & veulent \\\hline
\end{tabular}
\end{center}
\colbreak

\subsection{Passé composé}

\begin{center}
\underline{\textbf{avoir/être + participe passé}}
\begin{tabular}{|c|c|c|}
  \hline
  & \textbf{avec avoir} & \textbf{avec être}\\\hline
  \textbf{Je} & ai + pp & suis + pp \\\hline
  \textbf{Tu} & as + pp & es + pp \\\hline
  \textbf{Il} & a + pp & est + pp \\\hline
  \textbf{Nous} & avons + pp & sommes + pp \\\hline
  \textbf{Vous} & avez + pp & êtes + pp \\\hline
  \textbf{Ils} & ont + pp & sont + pp \\\hline
\end{tabular}
\end{center}

On utilise «avoir» pour la plupart des verbes.\\Le participe passé s'accorde avec le COD lorsque celui-ci est placé avant le verbe.
\begin{align*}
  \text{-er} &\rightarrow \text{-é}\\
  \text{-ir} &\rightarrow \text{-i}\\
  \text{-re} &\rightarrow \text{-u}
\end{align*}
On utilise «être» pour les verbes pronominaux et les exceptions des verbes de mouvement.\\Le participe passé s'accorde avec le sujet toujours.
\begin{align*}
  \text{Devenir} \rightarrow \text{Devenu} &\quad \text{Descendre} \rightarrow \text{Descendu}\\
  \text{Revenir} \rightarrow \text{Revenu} &\quad \text{Entrer} \rightarrow \text{Entré}\\
  \text{Monter} \rightarrow \text{Monté} &\quad \text{Rentrer} \rightarrow \text{Rentré}\\
  \text{Rester} \rightarrow \text{Resté} &\quad \text{Tomber} \rightarrow \text{Tombé}\\
  \text{Sortir} \rightarrow \text{Sorti} &\quad \text{Retourner} \rightarrow \text{Retourné}\\
  \text{Venir} \rightarrow \text{Venu} &\quad \text{Arriver} \rightarrow \text{Arrivé}\\
  \text{Aller} \rightarrow \text{Allé} &\quad \text{Mourir} \rightarrow \text{Mort}\\
  \text{Naître} \rightarrow \text{Né} &\quad \text{Partir} \rightarrow \text{Parti}
\end{align*}
\colbreak

\subsection{Imparfait}
\begin{center}
\underline{\textbf{présent «nous» -ons + la terminaison}}
\begin{tabular}{|c|c|c|c|}
  \hline
  & \textbf{-er} & \textbf{-ir} & \textbf{-re}\\\hline
  \textbf{Je} & parlais & finissais & vendais \\\hline
  \textbf{Tu} & parlais & finisais & vendais \\\hline
  \textbf{Il} & parlait & finissait & vendait \\\hline
  \textbf{Nous} & parlions & finissions & vendions \\\hline
  \textbf{Vous} & parliez & finissiez & vendiez \\\hline
  \textbf{Ils} & parlaient & finissaient & vendaient \\\hline
\end{tabular}
\end{center}

\subsection{Plus-que-parfait}
\begin{center}
\underline{\textbf{avoir/être (imparfait) + participe passé}}
\begin{tabular}{|c|c|c|}
  \hline
  & \textbf{avec avoir} & \textbf{avec être}\\\hline
  \textbf{Je} & avais + pp & étais + pp \\\hline
  \textbf{Tu} & avais + pp & étais + pp \\\hline
  \textbf{Il} & avait + pp & était + pp \\\hline
  \textbf{Nous} & avions + pp & étions + pp \\\hline
  \textbf{Vous} & aviez + pp & étiez + pp \\\hline
  \textbf{Ils} & avaient + pp & étaient + pp \\\hline
\end{tabular}
\end{center}

On utilise «être» pour les mêmes exceptions que pour le passé composé.
\colbreak

\subsection{Futur simple}
\begin{center}
\underline{\textbf{infinitif -e + la terminaison}}
\begin{tabular}{|c|c|c|c|}
  \hline
  & \textbf{-er} & \textbf{-ir} & \textbf{-re}\\\hline
  \textbf{Je} & parlerai & finirai & vendrai \\\hline
  \textbf{Tu} & parleras & finiras & vendras \\\hline
  \textbf{Il} & parlera & finira & vendra \\\hline
  \textbf{Nous} & parlerons & finirons & vendrons \\\hline
  \textbf{Vous} & parlerez & finirez & vendrez \\\hline
  \textbf{Ils} & parleront & finiront & vendront \\\hline
\end{tabular}
\end{center}

Les exceptions:
\begin{align*}
  \text{avoir} &\rightarrow \text{aur-}\\ 
  \text{être} &\rightarrow \text{ser-}\\ 
  \text{aller} &\rightarrow \text{ir-}\\ 
  \text{faire} &\rightarrow \text{fer-}\\ 
  \text{voir} &\rightarrow \text{verr-}\\ 
  \text{devoir} &\rightarrow \text{devr-}\\ 
  \text{pouvoir} &\rightarrow \text{pourr-}\\ 
  \text{vouloir} &\rightarrow \text{voudr-}\\ 
  \text{venir} &\rightarrow \text{viendr-}\\ 
  \text{savoir} &\rightarrow \text{saur-}\\ 
\end{align*}

\subsection{Plus-que-parfait}
\begin{center}
\underline{\textbf{avoir/être (futur simple) + participe passé}}
\begin{tabular}{|c|c|c|}
  \hline
  & \textbf{avec avoir} & \textbf{avec être}\\\hline
  \textbf{Je} & aurai + pp & serai + pp \\\hline
  \textbf{Tu} & auras + pp & seras + pp \\\hline
  \textbf{Il} & aura + pp & sera + pp \\\hline
  \textbf{Nous} & aurons + pp & serons + pp \\\hline
  \textbf{Vous} & aurez + pp & serez + pp \\\hline
  \textbf{Ils} & auront + pp & seront + pp \\\hline
\end{tabular}
\end{center}

On utilise «être» pour les mêmes exceptions que pour le passé composé.
\colbreak

\subsection{Conditionnel présent}
\begin{center}
\underline{\textbf{infinitif -e + la terminaison (de l'imparfait)}}
\begin{tabular}{|c|c|c|c|}
  \hline
  & \textbf{-er} & \textbf{-ir} & \textbf{-re}\\\hline
  \textbf{Je} & parlerais & finirais & vendrais \\\hline
  \textbf{Tu} & parlerais & finirais & vendrais \\\hline
  \textbf{Il} & parlerait & finirait & vendrait \\\hline
  \textbf{Nous} & parlerions & finirions & vendrions \\\hline
  \textbf{Vous} & parleriez & finiriez & vendriez \\\hline
  \textbf{Ils} & parleraient & finiraient & vendraient \\\hline
\end{tabular}
\end{center}

Les exceptions:
\begin{align*}
  \text{avoir} &\rightarrow \text{aur-}\\ 
  \text{être} &\rightarrow \text{ser-}\\ 
  \text{aller} &\rightarrow \text{ir-}\\ 
  \text{faire} &\rightarrow \text{fer-}\\ 
  \text{voir} &\rightarrow \text{verr-}\\ 
  \text{devoir} &\rightarrow \text{devr-}\\ 
  \text{pouvoir} &\rightarrow \text{pourr-}\\ 
  \text{vouloir} &\rightarrow \text{voudr-}\\ 
  \text{venir} &\rightarrow \text{viendr-}\\ 
  \text{savoir} &\rightarrow \text{saur-}\\ 
\end{align*}

\subsection{Conditionnel passé}
\begin{center}
\underline{\textbf{avoir/être (conditionnel) + participe passé}}
\begin{tabular}{|c|c|c|}
  \hline
  & \textbf{avec avoir} & \textbf{avec être}\\\hline
  \textbf{Je} & aurais + pp & serais + pp \\\hline
  \textbf{Tu} & aurais + pp & serais + pp \\\hline
  \textbf{Il} & aurait + pp & serait + pp \\\hline
  \textbf{Nous} & aurions + pp & serions + pp \\\hline
  \textbf{Vous} & auriez + pp & seriez + pp \\\hline
  \textbf{Ils} & auraient + pp & seraient + pp \\\hline
\end{tabular}
\end{center}

On utilise «être» pour les mêmes exceptions que pour le passé composé.
\colbreak

\subsection{Subjonctif présent}
\begin{center}
\underline{\textbf{présent «ils» -ent + la terminaison}}
\begin{tabular}{|c|c|c|c|}
  \hline
  & \textbf{-er} & \textbf{-ir} & \textbf{-re}\\\hline
  \textbf{Que je} & parle & finisse & vende \\\hline
  \textbf{Que tu} & parles & finisses & vendes \\\hline
  \textbf{Qu'il} & parle & finisse & vende \\\hline
  \textbf{Que nous} & parlions & finissions & vendions \\\hline
  \textbf{Que vois} & parliez & finissiez & vendiez \\\hline
  \textbf{Qu'ils} & parlent & finissent & vendent \\\hline
\end{tabular}
\end{center}
\begin{center}
\begin{tabular}{|c|c|c|c|c|}
  \hline
  & \textbf{avoir} & \textbf{être} & \textbf{faire} & \textbf{aller}\\\hline
  \textbf{Que je} & aie & sois & fasse & aille\\\hline
  \textbf{Que tu} & aies & sois & fasses & ailles\\\hline
  \textbf{Qu'il} & ait & soit & fasse & aille\\\hline
  \textbf{Que nous} & ayons & soyons & fassions & allions\\\hline
  \textbf{Que vous} & ayez & soyez & fassiez & alliez \\\hline
  \textbf{Qu'ils} & aient & soient & fassent & aillent \\\hline
\end{tabular}
\end{center}
\begin{center}
\begin{tabular}{|c|c|c|c|c|}
  \hline
  & \textbf{voir} & \textbf{devoir} & \textbf{pouvoir} & \textbf{vouloir}\\\hline
  \textbf{Que je} & voie & doive & puisse & veuille\\\hline
  \textbf{Que tu} & voies & doives & puisses & veuilles\\\hline
  \textbf{Qu'il} & voie & doive & puisse & veuille\\\hline
  \textbf{Que nous} & voyions &  devions & puissions & voulions\\\hline
  \textbf{Que vous} & voyiez & deviez & puissiez & vouliez \\\hline
  \textbf{Qu'ils} & voient & doivent & puissent & veuillent \\\hline
\end{tabular}
\end{center}

\subsection{Subjonctif passé}
\begin{center}
\underline{\textbf{avoir/être (subjonctif) + participe passé}}
\begin{tabular}{|c|c|c|}
  \hline
  & \textbf{avec avoir} & \textbf{avec être}\\\hline
  \textbf{Que je} & aie + pp & sois + pp \\\hline
  \textbf{Que tu} & aies + pp & sois + pp \\\hline
  \textbf{Qu'il} & ait + pp & soit + pp \\\hline
  \textbf{Que nous} & ayons + pp & soyons + pp \\\hline
  \textbf{Que vous} & ayez + pp & soyez + pp \\\hline
  \textbf{Qu'ils} & aient+ pp & soient + pp \\\hline
\end{tabular}
\end{center}

On utilise «être» pour les mêmes exceptions que pour le passé composé.

\colbreak
\section{Les pronoms relatifs}
«Qui» utilisé pour le sujet de la preposition relative.\\
E.g. «L'homme \textbf{qui} ici est mon ami»

«Que» utilisé pour le COD de la preposition relative.\\
E.g. «L'homme \textbf{que} j'ai rencontré est mon ami»

«Dont» utilisé pour indiquer la possession ou pour remplacer des phrases introduites par «de».\\
E.g. «L'homme \textbf{dont} je parle est mon ami»

«Où» utilisé pour des lieux ou des moments.\\
E.g. «La ville \textbf{où} je suis né est ma ville natale»

«Lequel/Laquelle/Lesquels/Lesquelles» utilisé pour les objets de prépositions (plus formel).\\ 
E.g. «L'homme pour \textbf{lequel} je travaille est mon ami»

\section{Les pronoms compléments}
«le/la/les» remplacent un nom qui est le COD.\\
E.g. «Je lis le livre» $\rightarrow$ «Je \textbf{le} lis»

«lui/leur» remplacent un nom qui est le COI.\\
E.g. «Je parle à mon frère» $\rightarrow$ «Je \textbf{lui} parle»

«y» remplace une phrase introduite par «à» (pour les lieux ou les choses).\\
E.g. «Je vais à Paris» $\rightarrow$ «J'y vais»

«en» remplace une phrase introduite par «de» (pour les quantités ou les objets).\\
E.g. «Je mange des pommes» $\rightarrow$ «J'en  mange»

\section{Les négations complexes}
«ne $\ldots$ jamais» pour dire "never".\\
E.g. «Il ne mange jamais de légumes»

«ne $\ldots$ plus» pour dire "no longer".\\
E.g. «Il ne travaille plus ici»

«ne $\ldots$ que» pour dire "only".\\
E.g. «Il n'ai que deux livres»

«ne $\ldots$ aucun(e)» pour dire "none".\\
E.g. «Il n'ai aucun ami dans la ville»
\section{Les prépositions}
«à» indique la destination, l'heure ou l'utilisation.\\
E.g. «Je vais à l'école à 8h»

«de» indique l'origine ou la possession.\\
E.g. «Je viens de Singapour»

«chez» indique le lieu chez une personne ou un organisation.\\
E.g. «Je vais chez mon ami»

«en» indique un pays féminin ou un moyen de transport.\\
E.g. «Je vais en France en avion»

«par» indique un moyen ou un passage.\\
E.g. «Je passe par la porte»

\section{Les nombres}
\begin{alignat*}{3}
  &\text{0 - zero}\quad\quad&&\text{10 - dix}\quad\quad&&\text{20 - vingt}\\
  &\text{1 - un}\quad\quad&&\text{11 - onze}\quad\quad&&\text{30 - trente}\\
  &\text{2 - deux}\quad\quad&&\text{12 - douze}\quad\quad&&\text{40 - quarante}\\
  &\text{3 - trois}\quad\quad&&\text{13 - treize}\quad\quad&&\text{50 - cinquante}\\
  &\text{4 - quatre}\quad\quad&&\text{14 - quatorze}\quad\quad&&\text{60 - soixante}\\
  &\text{5 - cinq}\quad\quad&&\text{15 - quinze}\quad\quad&&\text{70 - soixante-dix}\\
  &\text{6 - six}\quad\quad&&\text{16 - seize}\quad\quad&&\text{80 - quatre-vingt}\\
  &\text{7 - sept}\quad\quad&&\text{17 - dix-sept}\quad\quad&&\text{90 - quatre-vingt-dix}\\
  &\text{8 - huit}\quad\quad&&\text{18 - dix-huit}\quad\quad&&\text{100 - cent}\\
  &\text{9 - neuf}\quad\quad&&\text{19 - dix-neuf}\quad\quad&&\text{1000 - mille}\\
\end{alignat*}
\colbreak
\colbreak

\section{Les connecteurs}
\subsection{Connecteurs de contraste}
\vspace{-10pt}
\begin{align*}
  &\text{Mais} \rightarrow \text{But}\\
  &\text{Bien que}\rightarrow\text{Although}\\
  &\text{Cependant} \rightarrow \text{However}\\
  &\text{Toutefois} \rightarrow \text{However}\\
  &\text{Néanmoins} \rightarrow \text{Nevertheless}\\
  &\text{Par contre} \rightarrow \text{On the contrary}\\
  &\text{Malgré}\rightarrow\text{Despite}\\
  &\text{Pourtant} \rightarrow \text{Yet}\\
  &\text{En revanche} \rightarrow \text{On the other hand}\\
\end{align*}
\vspace{-40pt}
\subsection{Connecteurs d'addition}
\vspace{-10pt}
\begin{align*}
  &\text{Et} \rightarrow \text{And}\\
  &\text{Aussi} \rightarrow \text{Also}\\
  &\text{Puis} \rightarrow \text{Then}\\
  &\text{De plus} \rightarrow \text{Furthermore}\\
  &\text{De même} \rightarrow \text{Likewise}\\
  &\text{En plus} \rightarrow \text{In addition}\\
  &\text{Par ailleurs} \rightarrow \text{Besides}\\
\end{align*}
\vspace{-40pt}

\subsection{Connecteurs de cause et conséquence}
\vspace{-10pt}
\begin{align*}
  &\text{Parce que} \rightarrow \text{Because}\\
  &\text{Donc} \rightarrow \text{Therefore}\\
  &\text{Car} \rightarrow \text{For}\\
  &\text{Puisque} \rightarrow \text{Since}\\
  &\text{Ainsi} \rightarrow \text{Thus}\\
  &\text{Par conséquent} \rightarrow \text{Consequently}\\
  &\text{C'est pourquoi} \rightarrow \text{That’s why}\\
  &\text{D'où} \rightarrow \text{Hence}\\
  &\text{Alors} \rightarrow \text{So}\\
\end{align*}
\vspace{-40pt}
\subsection{Connecteurs d’exemple}
\vspace{-10pt}
\begin{align*}
  &\text{Par exemple} \rightarrow \text{For example}\\
  &\text{C'est-à-dire} \rightarrow \text{That is to say}\\
  &\text{À savoir} \rightarrow \text{Namely}\\
  &\text{Notamment} \rightarrow \text{Notably}\\
\end{align*}
\vspace{-40pt}


\subsection{Connecteurs de temps}
\vspace{-10pt}
\begin{align*}
  &\text{Quand} \rightarrow \text{When}\\
  &\text{Lorsque} \rightarrow \text{When}\\
  &\text{Avant que} \rightarrow \text{Before}\\
  &\text{Après que} \rightarrow \text{After}\\
  &\text{Dès que} \rightarrow \text{As soon as}\\
  &\text{Pendant que} \rightarrow \text{While}\\
  &\text{Tandis que} \rightarrow \text{While}\\
  &\text{Soudain} \rightarrow \text{Suddenly}\\
  &\text{Ensuite} \rightarrow \text{Next}\\
\end{align*}
\vspace{-40pt}

\subsection{Connecteurs de condition}
\vspace{-10pt}
\begin{align*}
  &\text{Si} \rightarrow \text{If}\\
  &\text{Sinon} \rightarrow \text{If not}\\
  &\text{À condition que} \rightarrow \text{Provided that}\\
  &\text{Sauf si} \rightarrow \text{Unless}\\
  &\text{Au cas où} \rightarrow \text{In case}\\
  &\text{Pourvu que} \rightarrow \text{Provided that}\\
  &\text{Dans le cas où} \rightarrow \text{In the event that}\\
\end{align*}
\vspace{-40pt}

\subsection{Connecteurs de but}
\vspace{-10pt}
\begin{align*}
  &\text{Pour que} \rightarrow \text{So that}\\
  &\text{Afin que} \rightarrow \text{In order that}\\
  &\text{Pour} \rightarrow \text{To}\\
  &\text{Dans le but de} \rightarrow \text{With the aim of}\\
  &\text{En vue de} \rightarrow \text{In order to}\\
\end{align*}
\vspace{-40pt}

\subsection{Connecteurs de comparaison}
\vspace{-10pt}
\begin{align*}
  &\text{Comme} \rightarrow \text{Like/As}\\
  &\text{Ainsi que} \rightarrow \text{As well as}\\
  &\text{De la même manière} \rightarrow \text{In the same way}\\
  &\text{Plus que} \rightarrow \text{More than}\\
  &\text{Moins que} \rightarrow \text{Less than}\\
  &\text{Comparé à} \rightarrow \text{Compared to}\\
\end{align*}
\vspace{-40pt}

\subsection{Connecteurs de conclusion}
\vspace{-10pt}
\begin{align*}
  &\text{En conclusion} \rightarrow \text{In conclusion}\\
  &\text{Finalement} \rightarrow \text{Finally}\\
  &\text{Pour résumer} \rightarrow \text{To summarize}\\
  &\text{En somme} \rightarrow \text{In summary}\\
  &\text{Bref} \rightarrow \text{In short}\\
  &\text{Ainsi} \rightarrow \text{Thus}\\
  &\text{En définitive} \rightarrow \text{Ultimately}\\
\end{align*}
\vspace{-40pt}

\subsection{Connecteurs de confirmation}
\vspace{-10pt}
\begin{align*}
  &\text{En effet} \rightarrow \text{Indeed}\\
  &\text{Vraiment} \rightarrow \text{Really}\\
  &\text{Certainement} \rightarrow \text{Certainly}\\
  &\text{Assurément} \rightarrow \text{Surely}\\
\end{align*}

%%%%%%%%%%%%%%%%%%%%%%%%%%%%%%%%%%%%%%%%%%%%%%%%%%%%%%
%                       End                          %
%%%%%%%%%%%%%%%%%%%%%%%%%%%%%%%%%%%%%%%%%%%%%%%%%%%%%%

\end{multicols*}
\end{document}

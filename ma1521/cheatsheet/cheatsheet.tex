\documentclass[12pt, a4paper]{article}

\usepackage[utf8]{inputenc}
\usepackage[mathscr]{euscript}
\let\euscr\mathscr \let\mathscr\relax
\usepackage[scr]{rsfso}
\usepackage{amssymb,amsmath,amsthm,amsfonts}
\usepackage[shortlabels]{enumitem}
\usepackage{multicol,multirow}
\usepackage{lipsum}
\usepackage{balance}
\usepackage{calc}
\usepackage[colorlinks=true,citecolor=blue,linkcolor=blue]{hyperref}
\usepackage{import}
\usepackage{xifthen}
\usepackage{pdfpages}
\usepackage{transparent}
\usepackage{tabularx}

\newcommand{\incfig}[2][1.0]{
    \def\svgwidth{#1\columnwidth}
    \import{./figures/}{#2.pdf_tex}
}
\newcommand{\incimg}[2][1.0]{
  \includegraphics[width=#1\columnwidth]{./figures/#2}
}


\usepackage{ifthen}
\usepackage[landscape]{geometry}
\usepackage[shortlabels]{enumitem}

\ifthenelse{\lengthtest { \paperwidth = 11in}}
    { \geometry{top=.5in,left=.5in,right=.5in,bottom=.5in} }
	{\ifthenelse{ \lengthtest{ \paperwidth = 297mm}}
		{\geometry{top=1cm,left=1cm,right=1cm,bottom=1cm} }
		{\geometry{top=1cm,left=1cm,right=1cm,bottom=1cm} }
	}

\pagestyle{empty}
\makeatletter
\renewcommand\thesection{\arabic{section}.}
\renewcommand{\section}{\@startsection{section}{1}{0mm}%
                                {-1ex plus -.5ex minus -.2ex}%
                                {0.05ex}%x
                                {\normalfont\normalsize\bfseries}}
\renewcommand{\subsection}{\@startsection{subsection}{2}{0mm}%
                                {-1ex plus -.5ex minus -.2ex}%
                                {0.05ex}%
                                {\normalfont\small\bfseries}}
\renewcommand{\subsubsection}{\@startsection{subsubsection}{3}{0mm}%
                                {-1ex plus -.5ex minus -.2ex}%
                                {0.05ex}%
                                {\normalfont\footnotesize\bfseries}}
\newcommand{\colbreak}{\vfill\null\columnbreak}
\makeatother
\setcounter{secnumdepth}{1}
\setlength{\parindent}{0pt}
\setlength{\parskip}{0.7em}

\setlist[itemize]{itemsep=0.6ex, topsep=-2pt, partopsep=0pt, parsep=0pt}
\setlist[enumerate]{itemsep=0.6ex, topsep=-2pt, partopsep=0pt, parsep=0pt}

\input{letterfonts}

\newcommand{\mytitle}{MA1521 Calculus for Computing}
\newcommand{\myauthor}{github/omgeta}
\newcommand{\mydate}{AY 24/25 Sem 1}

\begin{document}
\raggedright
\footnotesize
\begin{multicols*}{3}
\setlength{\premulticols}{1pt}
\setlength{\postmulticols}{1pt}
\setlength{\multicolsep}{1pt}
\setlength{\columnsep}{2pt}

{\normalsize{\textbf{\mytitle}}} \\
{\footnotesize{\mydate\hspace{2pt}\textemdash\hspace{2pt}\myauthor}}
%%%%%%%%%%%%%%%%%%%%%%%%%%%%%%%%%%%%%%%%%%%%%%%%%%%%%%
%                      Begin                         %
%%%%%%%%%%%%%%%%%%%%%%%%%%%%%%%%%%%%%%%%%%%%%%%%%%%%%%
\section{Limits}
Function $f(x)$ is continuous at $x=c$ if and only if it is differentiable at $c$ or $\displaystyle \lim_{x\rightarrow c}f(x)$ exists and $\displaystyle \lim_{x\rightarrow c}f(x) = f(c)$.

Laws of Limits:
\begin{enumerate}[\roman*.]
  \item $\displaystyle \lim_{x\rightarrow c}(f(x) \pm g(x)) = \lim_{x\rightarrow c}f(x) \pm \lim_{x\rightarrow c}g(x)$
  \item $\displaystyle \lim_{x\rightarrow c}kf(x) = k\lim_{x\rightarrow c}f(x)$
  \item $\displaystyle \lim_{x\rightarrow c}(f(x)g(x)) = (\lim_{x\rightarrow c}f(x))(\lim_{x\rightarrow c}g(x))$
  \item $\displaystyle \lim_{x\rightarrow c} \frac{f(x)}{g(x)} = \frac{\lim_{x\rightarrow c}f(x)}{\lim_{x\rightarrow c}g(x)}$
  \item $g$ is continuous at $x=b \land \displaystyle\lim_{x\rightarrow c}f(x)=b $\\$\implies \lim_{x\rightarrow c}g(f(x)) = g(b) = g(\lim_{x\rightarrow c}f(x))$
\end{enumerate}
\textbf{Squeeze Theorem:}\\
{\centering
  $g(x) \leq f(x) \leq h(x) \land \displaystyle \lim_{x\rightarrow c}g(x) = \lim_{x\rightarrow c}h(x) = L $\\$\implies \displaystyle \lim_{x\rightarrow c}f(x) = L$
\par}

\textbf{Intermediate Value Theorem:}\\
{\centering
  $f$ is continuous on $[a,b] \land k$ is between $f(a)$ and $f(b)$\\
  $\implies f(c) = k$ for some $c \in [a,b]$
\par}

\textbf{Trignometric Identities:}\\
{\centering
  $\displaystyle\lim_{x\rightarrow c}g(x) = 0$\\
  $\implies \displaystyle \lim_{x\rightarrow c} \frac{g(x)}{\sin(g(x))} = \lim_{x\rightarrow c} \frac{\sin(g(x))}{g(x)} = 1$
  $\implies \displaystyle \lim_{x\rightarrow c} \frac{g(x)}{\tan(g(x))} = \lim_{x\rightarrow c} \frac{\tan(g(x))}{g(x)} = 1$
\par}

\textbf{Exponential Trick}\\
{\centering
  $\displaystyle \lim_{x\rightarrow c}\ln f(x) = L \implies \lim_{x\rightarrow c}f(x) = e^{L}$
\par}

\textbf{L'Hôpital's Rule}\\
{\centering
  $\displaystyle \lim_{x\rightarrow c}\frac{f(x)}{g(x)} = \frac{0}{0} \text{ or } \frac{\infty}{\infty}\implies \lim_{x\rightarrow c}\frac{f(x)}{g(x)} = \lim_{x\rightarrow c}\frac{f'(x)}{g'(x)}$
\par}
\colbreak

\section{Differentiation}
Derivative of a function $f$ is given by:\\
{\centering
  $\displaystyle f'(x) = \lim_{\Delta x\rightarrow 0} \frac{f(x+\Delta x)-f(x)}{\Delta x}$
\par}
\vspace{-0.5em}
Derivative of a parametric function in $t$ is given by:\\
{\centering
  $\displaystyle \frac{dy}{dx} = \frac{dy}{dt} \div \frac{dx}{dt},\quad \frac{dy^2}{dx^2} = \frac{d}{dt}(\frac{dy}{dx}) \div \frac{dx}{dt}$
\par}

Critical points at $x=c$ of function $f$ are non-endpoints where $f'(c)$ is $0$ or does not exist.

\textbf{First Derivative Test:}
\begin{enumerate}[\roman*.]
  \item $f'(c^-) > 0 \land f'(c^+)<0$\hfill(Local maximum)
  \item $f'(c^-) < 0 \land f'(c^+)>0$\hfill(Local minimum)
  \item Otherwise\hfill(Point of inflection)
\end{enumerate}

\textbf{Second Derivative Test:}
\begin{enumerate}[\roman*.]
  \item $f''(c)<0$\hfill(Local maximum)
  \item $f''(c)>0$\hfill(Local minimum)
\end{enumerate}

\textbf{Rolle's Theorem:}\\
{\centering
  $f$ continuous on $[a,b]$, differentiable on $(a,b) \land f(a) = f(b)$\\
  $\implies f'(c) = 0$ for some $c \in [a,b]$
\par}

\textbf{Mean Value Theorem:}\\
{\centering
  $f$ is continuous on $[a,b] \land f$ is differentiable on $(a,b)$\\
  $\implies f'(c) = 0$ for some $c \in [a,b]$
\par}

\subsection{Standard Derivatives}
{\centering
\begin{tabular}{|c|c|}
\hline
$\mathbf{f(x)}$ & $\mathbf{f'(x)}$ \\ \hline
$\tan(g(x))$ & $g'(x)\sec^2(g(x))$ \\ \hline
$\sec(g(x))$ & $g'(x)\sec(g(x))\tan(g(x))$ \\ \hline
$\cosec(g(x))$ & $-g'(x)\cosec(g(x))\cot(g(x))$ \\ \hline
$\cot(g(x))$ & $-g'(x)\cosec^2(g(x))$ \\ \hline
$\sin^{-1}(g(x))$ & $\frac{g'(x)}{\sqrt{1-g(x)^2}}$ \\ \hline
$\cos^{-1}(g(x))$ & $-\frac{g'(x)}{\sqrt{1-g(x)^2}}$ \\ \hline
$\tan^{-1}(g(x))$ & $\frac{g'(x)}{1+g(x)^2}$ \\ \hline
$\cot^{-1}(g(x))$ & $-\frac{g'(x)}{1+g(x)^2}$ \\ \hline
$\sec^{-1}(g(x))$ & $\frac{g'(x)}{|g(x)|\sqrt{g(x)^2 - 1}}, |g(x)|>1$ \\ \hline
$\cosec^{-1}(g(x))$ & $-\frac{g'(x)}{|g(x)|\sqrt{g(x)^2 - 1}}, |g(x)|>1$ \\ \hline
$a^x$ & $a^x\ln(a)$ \\ \hline
\end{tabular}
\par}
\section{Integration}
Definite integrals of function $f$ have Riemann Sum:\\
{\centering
  $\displaystyle \int^b_a f(x)dx = \lim_{n\rightarrow\infty}\Sigma^n_{i=1} \frac{b-a}{n}f\left(a+(b-a)\frac{i}{n}\right)$
\par}

Integration by substitution involves choosing $u=g(x)$ and replacing all original variables, limits and $dx$.

Integration by parts for $\int f(x)g(x)dx$ involves choosing $u$ and $\frac{dv}{dx}$ ($u$ by LIATE) so $\int u \frac{dv}{dx}dx = uv - \int v \frac{du}{dx} dx$

Volume of revolution about same axis, in a disk:\\
{\centering
  $V = \pi\int^b_a [f(x)]^2dx,\quad V=\pi\int^d_c[g(y)]^2dy$
\par}
\vspace{-0.5em}
Volume of revolution about diff. axis, in a cylindrical shell:\\
{\centering
  $V = 2\pi\int^b_a x|f(x)|dx,\quad V = 2\pi\int^d_c y|g(y)|dy$
\par}

Arc length of a curve measured along $x$ or $y$:\\
{\centering
  $l = \int^b_a \sqrt{1+[f'(x)]^2}dx,\quad l=\int^d_c \sqrt{1+[g'(y)]^2}dy$
\par}

\subsection{Standard Integrals}
{\centering
\begin{tabular}{|c|c|}
\hline
$\mathbf{f(x)}$ & $\mathbf{F(x) - C}$ \\ \hline
$[f(x)]^n,$ $n\neq -1$ & $\frac{[f(x)]^{n+1}}{(n+1)f'(x)} $\\ \hline
$\tan(f(x))$ & $\frac{1}{f'(x)}\ln|\sec(f(x))| $ \\ \hline
$\sec(f(x))$ & $\frac{1}{f'(x)}\ln|\sec(f(x)) + \tan(f(x))| $ \\ \hline
$\cosec(f(x))$ & $-\frac{1}{f'(x)}\ln|\cosec(f(x)) + \cot(f(x))| $ \\ \hline
$\cot(f(x))$ & $-\frac{1}{f'(x)}\ln|\cosec(f(x))| $ \\ \hline
$\sec^2(f(x))$ & $\frac{1}{f'(x)}\tan(f(x)) $ \\ \hline
$\cosec^2(f(x))$ & $-\frac{1}{f'(x)}\cot(f(x)) $ \\ \hline
$\sec(f(x))\tan(f(x))$ & $\frac{1}{f'(x)}\sec(f(x)) $ \\ \hline
$\cosec(f(x))\cot(f(x))$ & $-\frac{1}{f'(x)}\cosec(f(x)) $ \\ \hline
$\frac{1}{a^2+[f(x)]^2}$ & $\frac{1}{af'(x)}\tan^{-1}(\frac{f(x)}{a})$ \\ \hline
$\frac{1}{\sqrt{a^2-[f(x)]^2}}$ & $\frac{1}{f'(x)}\sin^{-1}(\frac{f(x)}{a})$ \\ \hline
$-\frac{1}{\sqrt{a^2-[f(x)]^2}}$ & $\frac{1}{f'(x)}\cos^{-1}(\frac{f(x)}{a})$ \\ \hline
$\frac{1}{a^2-[f(x)]^2}$ & $\frac{1}{2af'(x)}\ln|\frac{f(x)+a}{f(x)-a}|$ \\ \hline
$\frac{1}{[f(x)]^2-a^2}$ & $\frac{1}{2af'(x)}\ln|\frac{f(x)-a}{f(x)+a}|$ \\ \hline
$\frac{1}{\sqrt{[f(x)]^2+a^2}}$ & $\frac{1}{f'(x)}\ln|f(x)+\sqrt{[f(x)]^2+a^2}|$ \\ \hline
$\frac{1}{\sqrt{[f(x)]^2-a^2}}$ & $\frac{1}{f'(x)}\ln|f(x)+\sqrt{[f(x)]^2-a^2}|$ \\ \hline
$\sqrt{a^2-x^2}$ & $\frac{x}{2}\sqrt{a^2-x^2}+\frac{a^2}{2}\sin^{-1}(\frac{x}{a})$ \\ \hline
$\sqrt{x^2-a^2}$ & $\frac{x}{2}\sqrt{x^2-a^2}+\frac{a^2}{2}\ln|x+\sqrt{x^2-a^2}|$ \\ \hline
\end{tabular}
\par}
\section{Sequences and Series}

\mbox{\textbf{$n^{\text{th}}$ Term:} $\displaystyle\lim_{n\rightarrow\infty} a_n\neq 0 \Rightarrow \sum^{\infty}_{n=1}a_n$ diverges}

\textbf{Integral Test} for $a_n = f(n)$, where $f$ is continuous, positive, decreasing for $x \geq 1$:\vspace{-0.5em}
\begin{align*}
  \displaystyle \int^{\infty}_1 f(x)\text{ converges} \iff \sum^{\infty}_{n=1}a_n\text{ converges}
\end{align*}
\mbox{\textbf{$p$-series:} $\displaystyle \sum^{\infty}_{n=1} \frac{1}{n^p}\text{ converges} \iff p > 1$}

\textbf{Comparison Test} for $0 \leq a_n \leq b_n$:\vspace{-0.5em}
\begin{align*}
  \displaystyle \sum^{\infty}_{n=1}b_n\text{ converges}\implies\sum^{\infty}_{n=1}a_n\text{ converges}\\
  \displaystyle \sum^{\infty}_{n=1}a_n\text{ diverges}\implies\sum^{\infty}_{n=1}b_n\text{ diverges}
\end{align*}

\textbf{Absolute Convergence:}\vspace{-0.5em} 
\begin{align*}
\displaystyle\sum^{\infty}_{n=0} |a_n| \text{ converges}\implies \sum^{\infty}_{n=0} a_n \text{ converges}
\end{align*}

\textbf{Ratio/Root Test:}\\
\vspace{0.5em}
{\centering 
  $\displaystyle \lim_{n\rightarrow\infty}\left|\frac{a_{n+1}}{a_n}\right| = L\text{ or }\lim_{n\rightarrow\infty}\sqrt[n]{|a_n|}=L$
\par}
\begin{enumerate}[\roman*.]
  \item $0\leq L<1$\hfill(Absolute Convergence)
  \item $L>1$\hfill(Divergence)
  \item $L=1$\hfill(Inconclusive)
\end{enumerate}
\vspace{0.5em}
\textbf{Alternating Series Test} for terms $a_n = (-1)^{n}b_n$ or $a_n = (-1)^{n-1}b_n$, where $b_n$ is decreasing:\vspace{-0.5em}
\begin{align*}
  \displaystyle\lim_{n\rightarrow\infty}b_n=0 \implies \sum^{\infty}_{n=1}a_n\text{ converges}
\end{align*}

\textbf{Radius of Convergence} $R=\frac{1}{L}$ about $x=a$ for power series $b_n = c_n(x-a)^n$ is interval for absolute convergence:\vspace{-0.5em}
\begin{align*}
  \displaystyle \lim_{n\rightarrow\infty}\left|\frac{c_{n+1}}{c_n}\right| = L\text{ or }\lim_{n\rightarrow\infty}\sqrt[n]{|c_n|}=L
\end{align*}
\colbreak

Power series represented functions for $R > 0$ have:
\begin{enumerate}[label=\roman*., parsep=-1em]
  \item $\displaystyle f'(x) = \sum^{\infty}_{n=1}nc_n(x-a)^{n-1}$
  \item $\displaystyle \int f(x)dx = \sum^{\infty}_{n=0}c_n\frac{(x-a)^{n+1}}{n+1}$
\end{enumerate}

Taylor Series for a function with power series representation is:\\
{\centering
  $\displaystyle f(x) = \sum^{\infty}_{n=0}\frac{f^n(a)}{n!}(x-a)^n$
\par}
\vspace{-0.5em}
with MacLaurin Series at $x=0$:\\
{\centering
  $\displaystyle f(x) = \sum^{\infty}_{n=0}\frac{f^n(0)}{n!}x^n$
\par}

\subsection{Common Series}
{\centering
\setlength{\extrarowheight}{4pt}
\begin{tabular}{|c|c|}
\hline
$\mathbf{a_n}$ & $\mathbf{\lim_{n\rightarrow\infty}\sum^n_{r=1}a_r}$ \\ \hline
$ar^{n-1}, |r|<1$ & $\displaystyle\frac{a}{1-r}$ \\ \hline
$\displaystyle\frac{1}{n}$ & diverges \\ \hline
$(-1)^{n-1}\displaystyle\frac{1}{n}$ & $\ln 2$ \\ \hline
$\displaystyle\frac{1}{n^2}$ & $2$ \\ \hline
\end{tabular}
\par}

\subsection{Common Expansions}
{\centering
\setlength{\extrarowheight}{7pt}
\begin{tabular}{|c|c|}
\hline
$e^x$ & $\sum^{\infty}_{n=0} \displaystyle\frac{x^n}{n!}$ \\ \hline
$\sin x$ & $\sum^{\infty}_{n=0} (-1)^n \displaystyle\frac{x^{2n+1}}{(2n+1)!}$ \\ \hline
$\cos x$ & $\sum^{\infty}_{n=0} (-1)^n \displaystyle\frac{x^{2n}}{(2n)!}$ \\ \hline
$\ln (1+x)$ & $\sum^{\infty}_{n=1} (-1)^{n+1} \displaystyle\frac{x^n}{n}$ \\ \hline
$\displaystyle\frac{1}{1-x}$ & $\sum^{\infty}_{n=0} x^n$ \\ \hline
$\displaystyle\frac{1}{1+x}$ & $\sum^{\infty}_{n=0} (-1)^n x^n$ \\ \hline
$\displaystyle\frac{1}{1+x^2}$ & $\sum^{\infty}_{n=0} (-1)^n x^{2n}$ \\ \hline
$(1+x)^n, |x|<1$ & $\sum^{n}_{k=0} \displaystyle\binom nk x^k$ \\ \hline
$(a+b)^n, n>1$ & $\sum^{n}_{k=0} \displaystyle \binom nk a^{n-k}b^k$\\ \hline
\end{tabular}
\par}
\section{Vectors}
Projection of $\vec{b}$ onto $\vec{a}$ is given by:
\begin{align*}
  \displaystyle \proj_{\vec{a}}\vec{b} = \comp_{\vec{a}}\vec{b} \times \vhat{a} = (\vec{b} \cdot \vhat{a}) \vhat{a}
\end{align*}

Perpendicular distance from position vector $\vec{b}$ to $\vec{a}$ is given by:\\
{\centering
  $\displaystyle \norm{\vec{b} \times \vhat{a}}$
\par}

Projection of $\vec{b}$ onto plane $\Pi: \vec{r} \cdot \vec{n} = D$:
\begin{align*}
  \displaystyle \proj_{\Pi}\vec{b} &= \vec{b} - \proj_{\vhat{n}}\vec{b} \\&= \vec{b} - (\vec{b} \cdot \vhat{n})\vhat{n}\\
  \displaystyle \norm{\proj_{\Pi}\vec{b}} &= \norm{\vec{b} \times \vhat{n}}
\end{align*}

Perpendicular distance from position vector $\vec{b}$ to plane $\vec{r} \cdot \vec{n} = D$:\\
{\centering
  $\displaystyle \frac{|D - \vec{b}\cdot \vec{n}|}{\norm{\vec{n}}},\quad \frac{|D - D_1|}{\norm{\vec{n}}}\text{ (to plane)}$
\par}

Dot and Cross product are given by:
\begin{align*}
  \vec{a} \cdot \vec{b} = \norm{\vec{a}}\norm{\vec{b}}\cos\theta\\
  \norm{\vec{a}\times \vec{b}} = \norm{\vec{a}}\norm{\vec{b}}\sin\theta\\
  \vec{a} \times \vec{b} = \left(\begin{array}{c} a_{2}\,b_{3}-a_{3}\,b_{2}\\ a_{3}\,b_{1}-a_{1}\,b_{3}\\ a_{1}\,b_{2}-a_{2}\,b_{1} \end{array}\right)
\end{align*}

\subsection{Vector-valued Functions}
Derivative of vector-valued function $\vec{r}(t)$ is given by:\\\vspace{0.5em}
{\centering
  $\displaystyle \vec{r'}(t) = \lim_{\Delta t\rightarrow 0} \frac{\vec{r}(t+\Delta t) - \vec{r}(t)}{\Delta t}$
\par}
Derivative of $\vec{r}(t) = \langle f(t), g(t), h(t)\rangle$ is given by:\\\vspace{0.5em}
{\centering
  $\displaystyle \vec{r'}(t) = \langle f'(t), g'(t), h'(t)\rangle$
\par}

Arc length of a path measured along $t$:
\begin{align*}
  l = \int^b_a \sqrt{[f'(t)]^2 + [g'(t)]^2 + [h'(t)]^2}dt
\end{align*}
\colbreak

\section{Multivariate Calculus}
Derivative of parametric function $z=f(x,y)$ where $x=g(t)$ and $y=h(t)$ is given by:\\\vspace{0.5em}
{\centering
  $\displaystyle \frac{dz}{dt} = \frac{\partial f}{\partial x}\frac{dx}{dt} + \frac{\partial f}{\partial y} \frac{dy}{dt}$
\par}

Derivative of parametric function $z=f(x,y)$ where $x=g(s,t)$, $y=h(s,t)$ is given by:\\\vspace{0.5em}
{\centering
  $\displaystyle \frac{\partial z}{\partial s} = \frac{\partial f}{\partial x}\frac{\partial x}{\partial s} + \frac{\partial f}{\partial y} \frac{\partial y}{\partial s}$,\quad$\displaystyle \frac{\partial z}{\partial t} = \frac{\partial f}{\partial x}\frac{\partial x}{\partial t} + \frac{\partial f}{\partial y} \frac{\partial y}{\partial t}$
\par}

Derivative of $z$ in implicit function $F(x,y,z) = 0$ is given by:\\
{\centering
  $\displaystyle \frac{\partial z}{\partial x} = -\frac{F_x}{F_z},\quad\frac{\partial z}{\partial y} = -\frac{F_y}{F_z}$
\par}

\vspace{1em}
Directional derivative of $f$ at $P = (x_0,y_0)$ in direction of unit vector $\vhat{u}$ making angle $\theta$ with $\nabla f$ is given by:
\begin{align*}
  D_{\vhat{u}}f(P) &= \nabla f(x_0,y_0)\cdot \vhat{u}\\
                   &= \norm{\nabla f(x_0, y_0)}\cos\theta
\end{align*}
where gradient vector $\nabla f$ is given by:
\begin{align*}
  \nabla f = \langle f_x, f_y \rangle
\end{align*}
and rate of change is optimized at:
\begin{enumerate}[\roman*.]
  \item $\norm{\nabla f(P)}\text{ in direction }\nabla f(P)$\hfill{(Max.)}
  \item $-\norm{\nabla f(P)}\text{ in direction }-\nabla f(P)$\hfill{(Min.)}
\end{enumerate}

\vspace{1em}
Critical points at $(a,b)$ of function $f$ are non-endpoints where $f_x(a,b)=f_y(a,b)=0$ or a partial derivative does not exist.

\textbf{Second Derivative Test:}
\begin{align*}
  D = f_{xx}(a,b)f_{yy}(a,b)-[f_{xy}(a,b)]^2
\end{align*}
\begin{enumerate}[\roman*.]
  \item $D>0$ and $f_{xx}(a,b)<0$\hfill(Local max.)
  \item $D>0$ and $f_{xx}(a,b)>0$\hfill(Local min.)
  \item $D<0$\hfill(Saddle point)
  \item $D=0$\hfill(Inconclusive)
\end{enumerate}
\colbreak
\subsection{Increments and Differentials}
Increment of $z = f(x, y)$, where $\Delta x$ and $\Delta y$ are increments in $x$ and $y$ is given by:
\begin{align*}
\Delta z = f(x + \Delta x, y + \Delta y) - f(x, y)
\end{align*}
Differentials $dx$ and $dy$ are defined as:
\begin{align*}
dx = \Delta x, \quad dy = \Delta y
\end{align*}
Total differential $dz$ is the linear approximation of the increment $\Delta z$ and is given by:
\begin{align*}
\Delta z \approx dz = f_x(x, y) dx + f_y(x, y) dy
\end{align*}

\subsection{Level Curves/Surfaces vs. $\nabla f$}
$\nabla f(x_0, y_0)$ is normal to the level curve of $f(x,y) = k$ at $(x_0,y_0)$.

$\nabla F(x_0, y_0, z_0)$ is normal to the level surface of $F(x,y,z) = k$ at $(x_0,y_0,z_0)$.\\
If $\vec{r}(t_0) = \langle x_0, y_0, z_0 \rangle$, then $\nabla F(x_0, y_0, z_0)\cdot \vec{r}(t_0) = 0$

\subsection{Tangent Planes}
Tangent plane to surface $z = f(x, y)$ at $(x_0, y_0)$ has normal vector $\langle f_x(x_0, y_0), f_y(x_0, y_0), -1 \rangle$ with equation:
\begin{align*}
f_x(x_0, y_0)(x - x_0) + f_y(x_0, y_0)(y - y_0) - (z - f(x_0, y_0)) = 0
\end{align*}

Tangent plane to level surface $F(x, y, z) = k$ at $ (x_0, y_0, z_0)$ has normal vector $\nabla F(x_0, y_0, z_0)$ with equation:
\begin{align*}
\nabla F(x_0, y_0, z_0) \cdot \langle x - x_0, y - y_0, z - z_0 \rangle = 0.
\end{align*}
\colbreak

\section{Double Integrals}
Double Integral $\displaystyle \iint_R f(x,y)dA$ over rectangular region $R = [a,b]\times[c,d]$:
\begin{align*}
  \int^b_a\int^d_c f(x,y)dydx = \int^d_c\int^b_af(x,y)dxdy
\end{align*}
with special case when $f(x,y) = g(x)h(y)$:
\begin{align*}
  \int^b_a g(x)dx \cdot \int^d_ch(y)dy
\end{align*}

Area of general plane region $D$: $\displaystyle\iint_D dA$

Surface area: $\displaystyle \iint_D\sqrt{f_x^2+f_y^2+1}dA$

Polar coordinates: $\displaystyle \iint_Df(r\cos\theta,r\sin\theta)r d\theta dr$

\section{ODEs}
Separable ODEs, reducing if necessary by $\displaystyle v=\frac{y}{x}$ or $u=ax+by$:
\begin{align*}
  \frac{dy}{dx} = f(x)g(y)\Rightarrow \int \frac{1}{g(y)}dy = \int f(x) dx + C
\end{align*}

Linear ODEs using $I(x) = e^{\int P(x) dx}$:
\begin{align*}
  \frac{dy}{dx} + P(x)y = Q(x)\Rightarrow yI(x) = \int Q(x)I(x)dx
\end{align*}

Bernoulli equation using $u=y^{1-n}$:
\begin{align*}
  \frac{dy}{dx} + P(x)y &= Q(x)y^n\\
  \implies \frac{du}{dx} + (1-n)P(x)u&=(1-n)Q(x)
\end{align*}
\colbreak
\section*{Appendix}
\subsection{Trignometric Identities}
\vspace{-1em}
\begin{gather*}
  \sin^2x + \cos^2x = 1\\
  \tan^2x + 1 = \sec^2x\\
  \cot^2x + 1=\cosec^2x\\
  \sin(A\pm B) = \sin A\cos B \pm \cos A\sin B\\
  \cos(A\pm B) = \cos A\cos B \mp \sin A\sin B\\
  \displaystyle \tan(A\pm B) = \frac{\tan A \pm \tan B}{1 \mp \tan A\tan B}\\
  \sin 2A = 2\sin A\cos A\\
  \cos 2A = \cos^2A -\sin^2A = 2\cos^2A-1 = 1-2\sin^2A\\
  \tan 2A = \frac{2\tan A}{1 -\tan^2 A}\\
  \sin P + \sin Q = 2\sin \frac{1}{2}(P+Q)\cos \frac{1}{2}(P-Q)\\
  \sin P - \sin Q = 2\cos \frac{1}{2}(P+Q)\sin \frac{1}{2}(P-Q)\\
  \cos P + \cos Q = 2\cos \frac{1}{2}(P+Q)\cos \frac{1}{2}(P-Q)\\
  \cos P - \cos Q = -2\sin \frac{1}{2}(P+Q)\sin \frac{1}{2}(P-Q)\\
  \sin A\cos B = \frac{1}{2}\sin(A+B) + \frac{1}{2}\sin(A-B)\\
  \cos A\sin B = \frac{1}{2}\sin(A+B) - \frac{1}{2}\sin(A-B)\\
  \cos A\cos B = \frac{1}{2}\cos(A+B) + \frac{1}{2}\cos(A-B)\\
  \sin A\sin B = -\frac{1}{2}\cos(A+B) + \frac{1}{2}\cos(A-B)
\end{gather*}
\subsection{Partial Fractions}
\vspace{-1em}
\begin{gather*}
  \frac{px+q}{(ax+b)(cx+d)} = \frac{A}{ax+b} + \frac{B}{cx+d}\\
  \frac{px^2+qx+r}{(ax+b)(cx+d)^2} = \frac{A}{ax+b} + \frac{B}{cx+d} + \frac{C}{(cx+d)^2}\\
  \frac{px^2+qx+r}{(ax+b)(x^2+c^2)} = \frac{A}{ax+b} + \frac{Bx+C}{x^2+c^2}\\
\end{gather*}

\subsection{Cylinders and Quadric Surfaces}
Cylinders are planes such that all other parallel planes intersect the surface in the same curve. Any equation in $x,y,z$ with a missing variable is a cylinder such as:
\begin{align*}
  y^2+z^2=1\\
  z=x^2
\end{align*}

Elliptic parabolids are open quadraic surfaces symmetric about the $z$-axis. If $c > 0$, it opens up in the positive $z$-axis. If $c < 0$, it opens down to the negative $z$-axis. General equation is given by:
\begin{align*}
  \frac{(x-x_0)^2}{a^2}+\frac{(y-y_0)^2}{b^2}=\frac{z}{c}
\end{align*}

Ellipsoids are closed quadric surfaces. If $a=b=c$, it is a sphere. General equation is given by:
\begin{align*}
  \frac{(x-x_0)^2}{a^2}+\frac{(y-y_0)^2}{b^2} + \frac{(z-z_0)^2}{c^2} = 1
\end{align*}

\begin{center}
  \incimg{quad}
\end{center}

\colbreak
\subsection{Forming ODEs}
General rate of change is given by:
\begin{align*}
  \text{Rate of change = rate of increase }-\text{ rate of decrease}
\end{align*}

\textbf{Example: }At time $t = 0$, a tank contains $20$kg of salt dissolved in $100$ litres of water.
Assume that water containing $\frac{1}{4}$kg of salt per litre is entering the tank at the rate of $3$ litre per
min, and the well-stirred solution is leaving the tank at the rate of $4$ litre per min. Find the amount of salt at
any time t.
\begin{gather*}
  \text{Let $S$ denote the amount of salt in kg at time $t$}\\
  \frac{dS}{dt} = \text{ salt input } - \text{ salt output}\\= (3 \times \frac{1}{4}) - (\frac{3 \times S}{100 - (4 + 3)t})\\
  \text{which resolves to an ODE}
\end{gather*}

\textbf{Example: } Formulae for half-life of a radioactive $y$ substance with half-life $T$ and initial amount $y_0$ with respect to time $t$ is given by:
\begin{align*}
  y = y_0(\frac{1}{2})^{\frac{t}{T}},\quad \text{or }y = y_0e^{-\frac{\ln 2 t}{T}}
\end{align*}


%%%%%%%%%%%%%%%%%%%%%%%%%%%%%%%%%%%%%%%%%%%%%%%%%%%%%%
%                       End                          %
%%%%%%%%%%%%%%%%%%%%%%%%%%%%%%%%%%%%%%%%%%%%%%%%%%%%%%
\end{multicols*}
\end{document}

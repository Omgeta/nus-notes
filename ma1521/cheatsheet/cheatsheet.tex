\documentclass[12pt, a4paper]{article}

\usepackage[utf8]{inputenc}
\usepackage[mathscr]{euscript}
\let\euscr\mathscr \let\mathscr\relax
\usepackage[scr]{rsfso}
\usepackage{amssymb,amsmath,amsthm,amsfonts}
\usepackage[shortlabels]{enumitem}
\usepackage{multicol,multirow}
\usepackage{lipsum}
\usepackage{balance}
\usepackage{calc}
\usepackage[colorlinks=true,citecolor=blue,linkcolor=blue]{hyperref}
\usepackage{import}
\usepackage{xifthen}
\usepackage{pdfpages}
\usepackage{transparent}
\usepackage{listings}

\newcommand{\incfig}[2][1.0]{
    \def\svgwidth{#1\columnwidth}
    \import{./figures/}{#2.pdf_tex}
}

\newlist{enumproof}{enumerate}{4}
\setlist[enumproof,1]{label=\arabic*., parsep=1em}
\setlist[enumproof,2]{label=\arabic{enumproofi}.\arabic*., parsep=1em}
\setlist[enumproof,3]{label=\arabic{enumproofi}.\arabic{enumproofii}.\arabic*., parsep=1em}
\setlist[enumproof,4]{label=\arabic{enumproofi}.\arabic{enumproofii}.\arabic{enumproofiii}.\arabic*., parsep=1em}

\renewcommand{\qedsymbol}{\ensuremath{\blacksquare}}

\lstdefinestyle{mystyle}{
  language=C, % Set the language to C
  commentstyle=\color{codegray}, % Color for comments
  keywordstyle=\color{orange}, % Color for basic keywords
  stringstyle=\color{mauve}, % Color for strings
  basicstyle={\ttfamily\footnotesize}, % Basic font style
  breakatwhitespace=false,         
  breaklines=true,                 
  captionpos=b,                    
  keepspaces=true,                 
  numbers=none,                    
  tabsize=2,
  morekeywords=[2]{\#include, \#define, \#ifdef, \#ifndef, \#endif, \#pragma, \#else, \#elif}, % Preprocessor directives
  keywordstyle=[2]\color{codegreen}, % Style for preprocessor directives
  morekeywords=[3]{int, char, float, double, void, struct, union, enum, const, volatile, static, extern, register, inline, restrict, _Bool, _Complex, _Imaginary, size_t, ssize_t, FILE}, % C standard types and common identifiers
  keywordstyle=[3]\color{identblue}, % Style for types and common identifiers
  morekeywords=[4]{printf, scanf, fopen, fclose, malloc, free, calloc, realloc, perror, strtok, strncpy, strcpy, strcmp, strlen}, % Standard library functions
  keywordstyle=[4]\color{cyan}, % Style for library functions
}

\usepackage{ifthen}
\usepackage[landscape]{geometry}
\usepackage[shortlabels]{enumitem}

\ifthenelse{\lengthtest { \paperwidth = 11in}}
    { \geometry{top=.5in,left=.5in,right=.5in,bottom=.5in} }
	{\ifthenelse{ \lengthtest{ \paperwidth = 297mm}}
		{\geometry{top=1cm,left=1cm,right=1cm,bottom=1cm} }
		{\geometry{top=1cm,left=1cm,right=1cm,bottom=1cm} }
	}

\pagestyle{empty}
\makeatletter
\renewcommand\thesection{\arabic{section}.}
\renewcommand{\section}{\@startsection{section}{1}{0mm}%
                                {-1ex plus -.5ex minus -.2ex}%
                                {0.05ex}%x
                                {\normalfont\normalsize\bfseries}}
\renewcommand{\subsection}{\@startsection{subsection}{2}{0mm}%
                                {-1ex plus -.5ex minus -.2ex}%
                                {0.05ex}%
                                {\normalfont\small\bfseries}}
\renewcommand{\subsubsection}{\@startsection{subsubsection}{3}{0mm}%
                                {-1ex plus -.5ex minus -.2ex}%
                                {0.05ex}%
                                {\normalfont\footnotesize\bfseries}}
\newcommand{\colbreak}{\vfill\null\columnbreak}
\makeatother
\setcounter{secnumdepth}{1}
\setlength{\parindent}{0pt}
\setlength{\parskip}{0.7em}

\setlist[itemize]{itemsep=0.6ex, topsep=-2pt, partopsep=0pt, parsep=0pt}
\setlist[enumerate]{itemsep=0.6ex, topsep=-2pt, partopsep=0pt, parsep=0pt}

% Things Lie
\newcommand{\kb}{\mathfrak b}
\newcommand{\kg}{\mathfrak g}
\newcommand{\kh}{\mathfrak h}
\newcommand{\kn}{\mathfrak n}
\newcommand{\ku}{\mathfrak u}
\newcommand{\kz}{\mathfrak z}
\DeclareMathOperator{\Ext}{Ext} % Ext functor
\DeclareMathOperator{\Tor}{Tor} % Tor functor
\newcommand{\gl}{\opname{\mathfrak{gl}}} % frak gl group
\renewcommand{\sl}{\opname{\mathfrak{sl}}} % frak sl group chktex 6

% More script letters etc.
\newcommand{\SA}{\mathcal A}
\newcommand{\SB}{\mathcal B}
\newcommand{\SC}{\mathcal C}
\newcommand{\SF}{\mathcal F}
\newcommand{\SG}{\mathcal G}
\newcommand{\SH}{\mathcal H}
\newcommand{\OO}{\mathcal O}

\newcommand{\SCA}{\mathscr A}
\newcommand{\SCB}{\mathscr B}
\newcommand{\SCC}{\mathscr C}
\newcommand{\SCD}{\mathscr D}
\newcommand{\SCE}{\mathscr E}
\newcommand{\SCF}{\mathscr F}
\newcommand{\SCG}{\mathscr G}
\newcommand{\SCH}{\mathscr H}

% Mathfrak primes
\newcommand{\km}{\mathfrak m}
\newcommand{\kp}{\mathfrak p}
\newcommand{\kq}{\mathfrak q}

% number sets
\newcommand{\RR}[1][]{\ensuremath{\ifstrempty{#1}{\mathbb{R}}{\mathbb{R}^{#1}}}}
\newcommand{\NN}[1][]{\ensuremath{\ifstrempty{#1}{\mathbb{N}}{\mathbb{N}^{#1}}}}
\newcommand{\ZZ}[1][]{\ensuremath{\ifstrempty{#1}{\mathbb{Z}}{\mathbb{Z}^{#1}}}}
\newcommand{\QQ}[1][]{\ensuremath{\ifstrempty{#1}{\mathbb{Q}}{\mathbb{Q}^{#1}}}}
\newcommand{\CC}[1][]{\ensuremath{\ifstrempty{#1}{\mathbb{C}}{\mathbb{C}^{#1}}}}
\newcommand{\PP}[1][]{\ensuremath{\ifstrempty{#1}{\mathbb{P}}{\mathbb{P}^{#1}}}}
\newcommand{\HH}[1][]{\ensuremath{\ifstrempty{#1}{\mathbb{H}}{\mathbb{H}^{#1}}}}
\newcommand{\FF}[1][]{\ensuremath{\ifstrempty{#1}{\mathbb{F}}{\mathbb{F}^{#1}}}}
% expected value
\newcommand{\EE}{\ensuremath{\mathbb{E}}}
\newcommand{\charin}{\text{ char }}
\DeclareMathOperator{\sign}{sign}
\DeclareMathOperator{\Aut}{Aut}
\DeclareMathOperator{\Inn}{Inn}
\DeclareMathOperator{\Syl}{Syl}
\DeclareMathOperator{\Gal}{Gal}
\DeclareMathOperator{\GL}{GL} % General linear group
\DeclareMathOperator{\SL}{SL} % Special linear group

%---------------------------------------
% BlackBoard Math Fonts :-
%---------------------------------------

%Captital Letters
\newcommand{\bbA}{\mathbb{A}}	\newcommand{\bbB}{\mathbb{B}}
\newcommand{\bbC}{\mathbb{C}}	\newcommand{\bbD}{\mathbb{D}}
\newcommand{\bbE}{\mathbb{E}}	\newcommand{\bbF}{\mathbb{F}}
\newcommand{\bbG}{\mathbb{G}}	\newcommand{\bbH}{\mathbb{H}}
\newcommand{\bbI}{\mathbb{I}}	\newcommand{\bbJ}{\mathbb{J}}
\newcommand{\bbK}{\mathbb{K}}	\newcommand{\bbL}{\mathbb{L}}
\newcommand{\bbM}{\mathbb{M}}	\newcommand{\bbN}{\mathbb{N}}
\newcommand{\bbO}{\mathbb{O}}	\newcommand{\bbP}{\mathbb{P}}
\newcommand{\bbQ}{\mathbb{Q}}	\newcommand{\bbR}{\mathbb{R}}
\newcommand{\bbS}{\mathbb{S}}	\newcommand{\bbT}{\mathbb{T}}
\newcommand{\bbU}{\mathbb{U}}	\newcommand{\bbV}{\mathbb{V}}
\newcommand{\bbW}{\mathbb{W}}	\newcommand{\bbX}{\mathbb{X}}
\newcommand{\bbY}{\mathbb{Y}}	\newcommand{\bbZ}{\mathbb{Z}}

%---------------------------------------
% MathCal Fonts :-
%---------------------------------------

%Captital Letters
\newcommand{\mcA}{\mathcal{A}}	\newcommand{\mcB}{\mathcal{B}}
\newcommand{\mcC}{\mathcal{C}}	\newcommand{\mcD}{\mathcal{D}}
\newcommand{\mcE}{\mathcal{E}}	\newcommand{\mcF}{\mathcal{F}}
\newcommand{\mcG}{\mathcal{G}}	\newcommand{\mcH}{\mathcal{H}}
\newcommand{\mcI}{\mathcal{I}}	\newcommand{\mcJ}{\mathcal{J}}
\newcommand{\mcK}{\mathcal{K}}	\newcommand{\mcL}{\mathcal{L}}
\newcommand{\mcM}{\mathcal{M}}	\newcommand{\mcN}{\mathcal{N}}
\newcommand{\mcO}{\mathcal{O}}	\newcommand{\mcP}{\mathcal{P}}
\newcommand{\mcQ}{\mathcal{Q}}	\newcommand{\mcR}{\mathcal{R}}
\newcommand{\mcS}{\mathcal{S}}	\newcommand{\mcT}{\mathcal{T}}
\newcommand{\mcU}{\mathcal{U}}	\newcommand{\mcV}{\mathcal{V}}
\newcommand{\mcW}{\mathcal{W}}	\newcommand{\mcX}{\mathcal{X}}
\newcommand{\mcY}{\mathcal{Y}}	\newcommand{\mcZ}{\mathcal{Z}}

%---------------------------------------
% Bold Math Fonts :-
%---------------------------------------

%Captital Letters
\newcommand{\bmA}{\boldsymbol{A}}	\newcommand{\bmB}{\boldsymbol{B}}
\newcommand{\bmC}{\boldsymbol{C}}	\newcommand{\bmD}{\boldsymbol{D}}
\newcommand{\bmE}{\boldsymbol{E}}	\newcommand{\bmF}{\boldsymbol{F}}
\newcommand{\bmG}{\boldsymbol{G}}	\newcommand{\bmH}{\boldsymbol{H}}
\newcommand{\bmI}{\boldsymbol{I}}	\newcommand{\bmJ}{\boldsymbol{J}}
\newcommand{\bmK}{\boldsymbol{K}}	\newcommand{\bmL}{\boldsymbol{L}}
\newcommand{\bmM}{\boldsymbol{M}}	\newcommand{\bmN}{\boldsymbol{N}}
\newcommand{\bmO}{\boldsymbol{O}}	\newcommand{\bmP}{\boldsymbol{P}}
\newcommand{\bmQ}{\boldsymbol{Q}}	\newcommand{\bmR}{\boldsymbol{R}}
\newcommand{\bmS}{\boldsymbol{S}}	\newcommand{\bmT}{\boldsymbol{T}}
\newcommand{\bmU}{\boldsymbol{U}}	\newcommand{\bmV}{\boldsymbol{V}}
\newcommand{\bmW}{\boldsymbol{W}}	\newcommand{\bmX}{\boldsymbol{X}}
\newcommand{\bmY}{\boldsymbol{Y}}	\newcommand{\bmZ}{\boldsymbol{Z}}
%Small Letters
\newcommand{\bma}{\boldsymbol{a}}	\newcommand{\bmb}{\boldsymbol{b}}
\newcommand{\bmc}{\boldsymbol{c}}	\newcommand{\bmd}{\boldsymbol{d}}
\newcommand{\bme}{\boldsymbol{e}}	\newcommand{\bmf}{\boldsymbol{f}}
\newcommand{\bmg}{\boldsymbol{g}}	\newcommand{\bmh}{\boldsymbol{h}}
\newcommand{\bmi}{\boldsymbol{i}}	\newcommand{\bmj}{\boldsymbol{j}}
\newcommand{\bmk}{\boldsymbol{k}}	\newcommand{\bml}{\boldsymbol{l}}
\newcommand{\bmm}{\boldsymbol{m}}	\newcommand{\bmn}{\boldsymbol{n}}
\newcommand{\bmo}{\boldsymbol{o}}	\newcommand{\bmp}{\boldsymbol{p}}
\newcommand{\bmq}{\boldsymbol{q}}	\newcommand{\bmr}{\boldsymbol{r}}
\newcommand{\bms}{\boldsymbol{s}}	\newcommand{\bmt}{\boldsymbol{t}}
\newcommand{\bmu}{\boldsymbol{u}}	\newcommand{\bmv}{\boldsymbol{v}}
\newcommand{\bmw}{\boldsymbol{w}}	\newcommand{\bmx}{\boldsymbol{x}}
\newcommand{\bmy}{\boldsymbol{y}}	\newcommand{\bmz}{\boldsymbol{z}}

%---------------------------------------
% Scr Math Fonts :-
%---------------------------------------

\newcommand{\sA}{{\mathscr{A}}}   \newcommand{\sB}{{\mathscr{B}}}
\newcommand{\sC}{{\mathscr{C}}}   \newcommand{\sD}{{\mathscr{D}}}
\newcommand{\sE}{{\mathscr{E}}}   \newcommand{\sF}{{\mathscr{F}}}
\newcommand{\sG}{{\mathscr{G}}}   \newcommand{\sH}{{\mathscr{H}}}
\newcommand{\sI}{{\mathscr{I}}}   \newcommand{\sJ}{{\mathscr{J}}}
\newcommand{\sK}{{\mathscr{K}}}   \newcommand{\sL}{{\mathscr{L}}}
\newcommand{\sM}{{\mathscr{M}}}   \newcommand{\sN}{{\mathscr{N}}}
\newcommand{\sO}{{\mathscr{O}}}   \newcommand{\sP}{{\mathscr{P}}}
\newcommand{\sQ}{{\mathscr{Q}}}   \newcommand{\sR}{{\mathscr{R}}}
\newcommand{\sS}{{\mathscr{S}}}   \newcommand{\sT}{{\mathscr{T}}}
\newcommand{\sU}{{\mathscr{U}}}   \newcommand{\sV}{{\mathscr{V}}}
\newcommand{\sW}{{\mathscr{W}}}   \newcommand{\sX}{{\mathscr{X}}}
\newcommand{\sY}{{\mathscr{Y}}}   \newcommand{\sZ}{{\mathscr{Z}}}


%---------------------------------------
% Math Fraktur Font
%---------------------------------------

%Captital Letters
\newcommand{\mfA}{\mathfrak{A}}	\newcommand{\mfB}{\mathfrak{B}}
\newcommand{\mfC}{\mathfrak{C}}	\newcommand{\mfD}{\mathfrak{D}}
\newcommand{\mfE}{\mathfrak{E}}	\newcommand{\mfF}{\mathfrak{F}}
\newcommand{\mfG}{\mathfrak{G}}	\newcommand{\mfH}{\mathfrak{H}}
\newcommand{\mfI}{\mathfrak{I}}	\newcommand{\mfJ}{\mathfrak{J}}
\newcommand{\mfK}{\mathfrak{K}}	\newcommand{\mfL}{\mathfrak{L}}
\newcommand{\mfM}{\mathfrak{M}}	\newcommand{\mfN}{\mathfrak{N}}
\newcommand{\mfO}{\mathfrak{O}}	\newcommand{\mfP}{\mathfrak{P}}
\newcommand{\mfQ}{\mathfrak{Q}}	\newcommand{\mfR}{\mathfrak{R}}
\newcommand{\mfS}{\mathfrak{S}}	\newcommand{\mfT}{\mathfrak{T}}
\newcommand{\mfU}{\mathfrak{U}}	\newcommand{\mfV}{\mathfrak{V}}
\newcommand{\mfW}{\mathfrak{W}}	\newcommand{\mfX}{\mathfrak{X}}
\newcommand{\mfY}{\mathfrak{Y}}	\newcommand{\mfZ}{\mathfrak{Z}}
%Small Letters
\newcommand{\mfa}{\mathfrak{a}}	\newcommand{\mfb}{\mathfrak{b}}
\newcommand{\mfc}{\mathfrak{c}}	\newcommand{\mfd}{\mathfrak{d}}
\newcommand{\mfe}{\mathfrak{e}}	\newcommand{\mff}{\mathfrak{f}}
\newcommand{\mfg}{\mathfrak{g}}	\newcommand{\mfh}{\mathfrak{h}}
\newcommand{\mfi}{\mathfrak{i}}	\newcommand{\mfj}{\mathfrak{j}}
\newcommand{\mfk}{\mathfrak{k}}	\newcommand{\mfl}{\mathfrak{l}}
\newcommand{\mfm}{\mathfrak{m}}	\newcommand{\mfn}{\mathfrak{n}}
\newcommand{\mfo}{\mathfrak{o}}	\newcommand{\mfp}{\mathfrak{p}}
\newcommand{\mfq}{\mathfrak{q}}	\newcommand{\mfr}{\mathfrak{r}}
\newcommand{\mfs}{\mathfrak{s}}	\newcommand{\mft}{\mathfrak{t}}
\newcommand{\mfu}{\mathfrak{u}}	\newcommand{\mfv}{\mathfrak{v}}
\newcommand{\mfw}{\mathfrak{w}}	\newcommand{\mfx}{\mathfrak{x}}
\newcommand{\mfy}{\mathfrak{y}}	\newcommand{\mfz}{\mathfrak{z}}


\newcommand{\mytitle}{MA1521 Calculus for Computing}
\newcommand{\myauthor}{github/omgeta}
\newcommand{\mydate}{AY 24/25 Sem 1}

\begin{document}
\raggedright
\footnotesize
\begin{multicols*}{3}
\setlength{\premulticols}{1pt}
\setlength{\postmulticols}{1pt}
\setlength{\multicolsep}{1pt}
\setlength{\columnsep}{2pt}

{\normalsize{\textbf{\mytitle}}} \\
{\footnotesize{\mydate\hspace{2pt}\textemdash\hspace{2pt}\myauthor}}
%%%%%%%%%%%%%%%%%%%%%%%%%%%%%%%%%%%%%%%%%%%%%%%%%%%%%%
%                      Begin                         %
%%%%%%%%%%%%%%%%%%%%%%%%%%%%%%%%%%%%%%%%%%%%%%%%%%%%%%
\section{Limits}
Limit of a function $f(x)$ is given by:
\begin{enumerate}[\roman*.]
  \item $\displaystyle \lim_{x\rightarrow c^-} f(x) = L$\hfill($f(x) \rightarrow L$ from left)
  \item $\displaystyle \lim_{x\rightarrow c^+} f(x) = L$\hfill($f(x) \rightarrow L$ from right)
  \item $\displaystyle \lim_{x\rightarrow c} f(x) = L$\hfill($f(x) \rightarrow L$ from both)
\end{enumerate}

Function $f(x)$ is continuous at $x=c$ if and only if it is differentiable at $c$ or $\displaystyle \lim_{x\rightarrow c}f(x)$ exists and $\displaystyle \lim_{x\rightarrow c}f(x) = f(c)$.

Laws of Limits:
\begin{enumerate}[\roman*.]
  \item $\displaystyle \lim_{x\rightarrow c}(f(x) \pm g(x)) = \lim_{x\rightarrow c}f(x) \pm \lim_{x\rightarrow c}g(x)$
  \item $\displaystyle \lim_{x\rightarrow c}kf(x) = k\lim_{x\rightarrow c}f(x)$
  \item $\displaystyle \lim_{x\rightarrow c}(f(x)g(x)) = (\lim_{x\rightarrow c}f(x))(\lim_{x\rightarrow c}g(x))$
  \item $\displaystyle \lim_{x\rightarrow c} \frac{f(x)}{g(x)} = \frac{\lim_{x\rightarrow c}f(x)}{\lim_{x\rightarrow c}g(x)}$
  \item $g$ is continuous at $x=b \land \displaystyle\lim_{x\rightarrow c}f(x)=b $\\$\implies \lim_{x\rightarrow c}g(f(x)) = g(b) = g(\lim_{x\rightarrow c}f(x))$
\end{enumerate}
\textbf{Squeeze Theorem:}\\
{\centering
  $g(x) \leq f(x) \leq h(x) \land \displaystyle \lim_{x\rightarrow c}g(x) = \lim_{x\rightarrow c}h(x) = L $\\$\implies \displaystyle \lim_{x\rightarrow c}f(x) = L$
\par}

\textbf{Intermediate Value Theorem:}\\
{\centering
  $f$ is continuous on $[a,b] \land k$ is between $f(a)$ and $f(b)$\\
  $\implies f(c) = k$ for some $c \in [a,b]$
\par}

\textbf{Trignometric Identities:}\\
{\centering
  $\displaystyle\lim_{x\rightarrow c}g(x) = 0$\\
  $\implies \displaystyle \lim_{x\rightarrow c} \frac{g(x)}{\sin(g(x))} = \lim_{x\rightarrow c} \frac{\sin(g(x))}{g(x)} = 1$
  $\implies \displaystyle \lim_{x\rightarrow c} \frac{g(x)}{\tan(g(x))} = \lim_{x\rightarrow c} \frac{\tan(g(x))}{g(x)} = 1$
\par}

\textbf{L'Hôpital's Rule}\\
{\centering
  $\displaystyle \lim_{x\rightarrow c}\frac{f(x)}{g(x)} = \frac{0}{0} \text{ or } \frac{\infty}{\infty}\implies \lim_{x\rightarrow c}\frac{f(x)}{g(x)} = \lim_{x\rightarrow c}\frac{f'(x)}{g'(x)}$
\par}
\colbreak

\section{Differentiation}
Derivative of a function $f$ at $x = x_0$ is given by:\\
{\centering
  $\displaystyle f'(x_0) = \lim_{h\rightarrow0} \frac{f(x_0+h)-f(x_0)}{h}$
\par}
\vspace{-0.5em}
Derivative of a parametric function in $t$ is given by:\\
{\centering
  $\displaystyle \frac{dy}{dx} = \frac{dy}{dt} \div \frac{dx}{dt},\quad \frac{dy^2}{dx^2} = \frac{d}{dt}(\frac{dy}{dx}) \div \frac{dx}{dt}$
\par}

Critical points at $x=c$ of function $f$ are non-endpoints where $f'(c)$ is $0$ or does not exist.

\textbf{First Derivative Test:}
\begin{enumerate}[\roman*.]
  \item $f'(c^-) > 0 \land f'(c^+)<0$\hfill(Local maximum)
  \item $f'(c^-) < 0 \land f'(c^+)>0$\hfill(Local minimum)
  \item Otherwise\hfill(Point of inflection)
\end{enumerate}

\textbf{Second Derivative Test:}
\begin{enumerate}[\roman*.]
  \item $f''(c)<0$\hfill(Local maximum)
  \item $f''(c)>0$\hfill(Local minimum)
\end{enumerate}

\textbf{Rolle's Theorem:}\\
{\centering
  $f$ continuous on $[a,b]$, differentiable on $(a,b) \land f(a) = f(b)$\\
  $\implies f'(c) = 0$ for some $c \in [a,b]$
\par}

\textbf{Mean Value Theorem:}\\
{\centering
  $f$ is continuous on $[a,b] \land f$ is differentiable on $(a,b)$\\
  $\implies f'(c) = 0$ for some $c \in [a,b]$
\par}

\subsection{Standard Derivatives}
{\centering
\begin{tabular}{|c|c|}
\hline
$\mathbf{f(x)}$ & $\mathbf{f'(x)}$ \\ \hline
$\tan(g(x))$ & $g'(x)\sec^2(g(x))$ \\ \hline
$\sec(g(x))$ & $g'(x)\sec(g(x))\tan(g(x))$ \\ \hline
$\cosec(g(x))$ & $-g'(x)\cosec(g(x))\cot(g(x))$ \\ \hline
$\cot(g(x))$ & $-g'(x)\cosec^2(g(x))$ \\ \hline
$\sin^{-1}(g(x))$ & $\frac{g'(x)}{\sqrt{1-g(x)^2}}$ \\ \hline
$\cos^{-1}(g(x))$ & $-\frac{g'(x)}{\sqrt{1-g(x)^2}}$ \\ \hline
$\tan^{-1}(g(x))$ & $\frac{g'(x)}{1+g(x)^2}$ \\ \hline
$\cot^{-1}(g(x))$ & $-\frac{g'(x)}{1+g(x)^2}$ \\ \hline
$\sec^{-1}(g(x))$ & $\frac{g'(x)}{|g(x)|\sqrt{g(x)^2 - 1}}, |g(x)|>1$ \\ \hline
$\cosec^{-1}(g(x))$ & $-\frac{g'(x)}{|g(x)|\sqrt{g(x)^2 - 1}}, |g(x)|>1$ \\ \hline
$a^x$ & $a^x\ln(a)$ \\ \hline
\end{tabular}
\par}
\section{Integration}
Definite integrals of function $f$ have Riemann Sum:\\
{\centering
  $\displaystyle \int^b_a f(x)dx = \lim_{n\rightarrow\infty}\Sigma^n_{i=1} \frac{b-a}{n}f\left(a+(b-a)\frac{i}{n}\right)$
\par}

Integration by substitution involves choosing $u=g(x)$ and replacing all original variables, limits and $dx$.

Integration by parts for $\int f(x)g(x)dx$ involves choosing $u$ and $\frac{dv}{dx}$ ($u$ by LIATE) so $\int u \frac{dv}{dx}dx = uv - \int v \frac{du}{dx} dx$

Volume of revolution about same axis, in a disk:\\
{\centering
  $V = \pi\int^b_a [f(x)]^2dx,\quad V=\pi\int^d_c[g(y)]^2dy$
\par}
\vspace{-0.5em}
Volume of revolution about diff. axis, in a cylindrical shell:\\
{\centering
  $V = 2\pi\int^b_a x|f(x)|dx,\quad V = 2\pi\int^d_c y|g(y)|dy$
\par}

Arc length of a curve measured along $x$ or $y$:\\
{\centering
  $l = \int^b_a \sqrt{1+[f'(x)]^2}dx,\quad l=\int^d_c \sqrt{1+[g'(y)]^2}dy$
\par}

\subsection{Standard Integrals}
{\centering
\begin{tabular}{|c|c|}
\hline
$\mathbf{f(x)}$ & $\mathbf{F(x) - C}$ \\ \hline
$[f(x)]^n,$ $n\neq -1$ & $\frac{[f(x)]^{n+1}}{(n+1)f'(x)} $\\ \hline
$\tan(f(x))$ & $\frac{1}{f'(x)}\ln|\sec(f(x))| $ \\ \hline
$\sec(f(x))$ & $\frac{1}{f'(x)}\ln|\sec(f(x)) + \tan(f(x))| $ \\ \hline
$\cosec(f(x))$ & $-\frac{1}{f'(x)}\ln|\cosec(f(x)) + \cot(f(x))| $ \\ \hline
$\cot(f(x))$ & $-\frac{1}{f'(x)}\ln|\cosec(f(x))| $ \\ \hline
$\sec^2(f(x))$ & $\frac{1}{f'(x)}\tan(f(x)) $ \\ \hline
$\cosec^2(f(x))$ & $-\frac{1}{f'(x)}\cot(f(x)) $ \\ \hline
$\sec(f(x))\tan(f(x))$ & $\frac{1}{f'(x)}\sec(f(x)) $ \\ \hline
$\cosec(f(x))\cot(f(x))$ & $-\frac{1}{f'(x)}\cosec(f(x)) $ \\ \hline
$\frac{1}{a^2+[f(x)]^2}$ & $\frac{1}{af'(x)}\tan^{-1}(\frac{f(x)}{a})$ \\ \hline
$\frac{1}{\sqrt{a^2-[f(x)]^2}}$ & $\frac{1}{f'(x)}\sin^{-1}(\frac{f(x)}{a})$ \\ \hline
$-\frac{1}{\sqrt{a^2-[f(x)]^2}}$ & $\frac{1}{f'(x)}\cos^{-1}(\frac{f(x)}{a})$ \\ \hline
$\frac{1}{a^2-[f(x)]^2}$ & $\frac{1}{2af'(x)}\ln|\frac{f(x)+a}{f(x)-a}|$ \\ \hline
$\frac{1}{[f(x)]^2-a^2}$ & $\frac{1}{2af'(x)}\ln|\frac{f(x)-a}{f(x)+a}|$ \\ \hline
$\frac{1}{\sqrt{[f(x)]^2+a^2}}$ & $\frac{1}{f'(x)}\ln|f(x)+\sqrt{[f(x)]^2+a^2}|$ \\ \hline
$\frac{1}{\sqrt{[f(x)]^2-a^2}}$ & $\frac{1}{f'(x)}\ln|f(x)+\sqrt{[f(x)]^2-a^2}|$ \\ \hline
$\sqrt{a^2-x^2}$ & $\frac{x}{2}\sqrt{a^2-x^2}+\frac{a^2}{2}\sin^{-1}(\frac{x}{a})$ \\ \hline
$\sqrt{x^2-a^2}$ & $\frac{x}{2}\sqrt{x^2-a^2}+\frac{a^2}{2}\ln|x+\sqrt{x^2-a^2}|$ \\ \hline
\end{tabular}
\par}
\end{multicols*}
\begin{multicols*}{4}
\section{Sequences and Series}

\mbox{\textbf{$n^{\text{th}}$ Term:} $\displaystyle\lim_{n\rightarrow\infty} a_n\neq 0 \Rightarrow \sum^{\infty}_{n=1}a_n$ diverges}

\textbf{Integral Test} for $a_n = f(n)$, where $f$ is continuous, positive, decreasing for $x \geq 1$:\\
{\centering
  $\displaystyle \int^{\infty}_1 f(x)\text{ converges} \iff \sum^{\infty}_{n=1}a_n\text{ converges}$
\par}

\mbox{\textbf{$p$-series:} $\displaystyle \sum^{\infty}_{n=1} \frac{1}{n^p}\text{ converges} \iff p > 1$}

\textbf{Comparison Test} for $0 \leq a_n \leq b_n$:\\
{\centering
  $\displaystyle \sum^{\infty}_{n=1}b_n\text{ converges}\implies\sum^{\infty}_{n=1}a_n\text{ converges}$
   $\displaystyle \sum^{\infty}_{n=1}a_n\text{ diverges}\implies\sum^{\infty}_{n=1}b_n\text{ diverges}$
\par}

\textbf{Ratio/Root Test:}\\
{\centering
  $\displaystyle \lim_{n\rightarrow\infty}\left|\frac{a_{n+1}}{a_n}\right| = L\text{ or }\lim_{n\rightarrow\infty}\sqrt[n]{|a_n|}=L$
\par}
\begin{enumerate}[\roman*.]
  \item $0\leq L<1$\hfill(Absolute Convergence)
  \item $L>1$\hfill(Divergence)
  \item $L=1$\hfill(Inconclusive)
\end{enumerate}

\textbf{Alternating Series Test} for terms $a_n = (-1)^{n}b_n$ or $a_n = (-1)^{n-1}b_n$, where $b_n$ is decreasing:\\
{\centering
  $\displaystyle\lim_{n\rightarrow\infty}b_n=0 \implies \sum^{\infty}_{n=1}a_n\text{ converges}$
\par}

\textbf{Radius of Convergence} $R=\frac{1}{L}$ about $x=a$ for power series $b_n = c_n(x-a)^n$ is interval for absolute convergence:\\
{\centering
  $\displaystyle \lim_{n\rightarrow\infty}\left|\frac{c_{n+1}}{c_n}\right| = L\text{ or }\lim_{n\rightarrow\infty}\sqrt[n]{|c_n|}=L$
\par}

Functions with power series representation for $R > 0$ have:
\begin{enumerate}[label=\roman*., parsep=-1em]
  \item $\displaystyle f'(x) = \sum^{\infty}_{n=1}nc_n(x-a)^{n-1}$
  \item $\displaystyle \int f(x)dx = \sum^{\infty}_{n=0}c_n\frac{(x-a)^{n+1}}{n+1}$
\end{enumerate}

Taylor Series for a function with power series representation is:\\
{\centering
  $\displaystyle f(x) = \sum^{\infty}_{n=0}\frac{f^n(a)}{n!}(x-a)^n$
\par}
\vspace{-0.5em}
with MacLaurin Series at $x=0$:\\
{\centering
  $\displaystyle f(x) = \sum^{\infty}_{n=0}\frac{f^n(0)}{n!}x^n$
\par}

\section{Vectors}
Projection of $\vec{b}$ onto $\vec{a}$ is given by:\\
{\centering
  $\displaystyle \proj_{\vec{a}}\vec{b} = \comp_{\vec{a}}\vec{b} \times \vhat{a} = (\vec{b} \cdot \vhat{a}) \vhat{a}$
\par}

Perpendicular distance from position vector $\vec{b}$ to $\vec{a}$ is given by:\\
{\centering
  $\displaystyle \norm{\vec{b} \times \vhat{a}}$
\par}

Projection of $\vec{b}$ onto plane $\Pi: \vec{r} \cdot \vec{n} = D$:\\
{\centering
  $\displaystyle \proj_{\Pi}\vec{b} = \vec{b} - \proj_{\vhat{n}}\vec{b} = \vec{b} - (\vec{b} \cdot \vhat{n})\vhat{n}$\\
  $\displaystyle \norm{\proj_{\Pi}\vec{b}} = \norm{\vec{b} \times \vhat{n}}$
\par}

Perpendicular distance from position vector $\vec{b}$ to plane $\vec{r} \cdot \vec{n} = D$:\\
{\centering
  $\displaystyle \frac{|D - \vec{b}\cdot \vec{n}|}{\norm{\vec{n}}}$
\par}

Dot and Cross product are given by:\\
{\centering
  $\vec{a} \cdot \vec{b} = \norm{\vec{a}}\norm{\vec{b}}\cos\theta$\\
  $\norm{\vec{a}\times \vec{b}} = \norm{\vec{a}}\norm{\vec{b}}\sin\theta$\\
  $\vec{a} \times \vec{b} = \left(\begin{array}{c} a_{2}\,b_{3}-a_{3}\,b_{2}\\ a_{3}\,b_{1}-a_{1}\,b_{3}\\ a_{1}\,b_{2}-a_{2}\,b_{1} \end{array}\right)$
\par}

\subsection{Vector-valued Functions}
Derivative of $r(t) = \langle f(t), g(t), h(t)\rangle$ at $t=a$ is given by:\\
{\centering
  $r'(a) = \langle f'(a), g'(a), h'(a)\rangle$
\par}

Arc length of a path measured along $t$:\\
{\centering
  $l = \int^b_a \sqrt{[f'(t)]^2 + [g'(t)]^2 + [h'(t)]^2}dt$
\par}
\colbreak

\section{Multivariate Calculus}
Derivative of $z=f(x,y)$ where $x=g(t)$ and $y=h(t)$ is given by:\\
{\centering
  $\displaystyle \frac{dz}{dt} = \frac{\partial f}{\partial x}\frac{dx}{dt} + \frac{\partial f}{\partial y} \frac{dy}{dt}$
\par}

Derivative of $z=f(x,y)$ where $x=g(s,t)$ and $y=h(s,t)$ is given by:\\
{\centering
  $\displaystyle \frac{\partial z}{\partial s} = \frac{\partial f}{\partial x}\frac{\partial x}{\partial s} + \frac{\partial f}{\partial y} \frac{\partial y}{\partial s}$\\
  $\displaystyle \frac{\partial z}{\partial t} = \frac{\partial f}{\partial x}\frac{\partial x}{\partial t} + \frac{\partial f}{\partial y} \frac{\partial y}{\partial t}$
\par}

Normal vector to the tangent plane for $z=f(x,y)$ at $(x_0,y_0)$ is given by:\\
{\centering
  $\langle f_x(a,b), f_y(a,b), -1 \rangle$
\par}

Derivative of $z$ in $F(x,y,z) = 0$ is given by:\\
{\centering
  $\displaystyle \frac{\partial z}{\partial x} = -\frac{F_x}{F_z},\quad\frac{\partial z}{\partial y} = -\frac{F_y}{F_z}$
\par}

Normal vector to the tangent plane for level surface of $F(x,y,z)$ at $(x_0,y_0,z_0)$ is $\nabla F(x_0,y_0,z_0)$

Directional derivative of $f$ at $P = (x_0,y_0)$ in direction of unit vector $\vhat{u}$ is given by:\\
{\centering
  $D_{\vhat{u}}f(P) = \nabla f(x_0,y_0)\cdot \vhat{u}$
\par}
\vspace{-0.5em}
where $\nabla f = \langle f_x, f_y \rangle$ is the gradient vector
\vspace{-0.5em}
and rate of change is optimized at:
\vspace{-0.5em}
\begin{align*}
  \norm{\nabla f(P)}&\text{ in direction }\nabla f(P)\tag*{(Max.)}\\
  -\norm{\nabla f(P)}&\text{ in direction }-\nabla f(P)\tag*{(Min.)}
\end{align*}
Gradient vector $\nabla f(x_0, y_0) \neq 0$ is normal to level curve $f(x,y)=k$ at $(x_0, y_0)$

Critical points at $(a,b)$ of function $f$ are non-endpoints where $f_x(a,b)=f_y(a,b)=0$ or a partial derivative does not exist.

\textbf{Second Derivative Test:}\\
{\centering
  $D = f_{xx}(a,b)f_{yy}(a,b)-[f_{xy}(a,b)]^2$
\par}
\begin{enumerate}[\roman*.]
  \item $D>0 \land f_{xx}(a,b)<0$\hfill(Local max.)
  \item $D>0 \land f_{xx}(a,b)>0$\hfill(Local min.)
  \item $D<0$\hfill(Saddle point)
  \item $D=0$\hfill(Inconclusive)
\end{enumerate}

\section{Double Integrals}
Double Integral $\iint_R f(x,y)dA$ over rectangular region $R = [a,b]\times[c,d]$:\\
{\centering
  $\displaystyle\int^b_a\int^d_c f(x,y)dydx = \int^d_c\int^b_af(x,y)dxdy$
\par}
\vspace{-0.5em}
with special case $f(x,y) = g(x)h(y)$:\\
{\centering
  $\displaystyle\int^b_a g(x)dx + \int^d_ch(y)dy$
\par}

Area of general plane region $D$:$\displaystyle\iint_D dA$
\mbox{Surface area: $\displaystyle \iint_D\sqrt{f_x^2+f_y^2+1}dA$}
\mbox{Polar coordinates: $\displaystyle \iint_Df(r\cos\theta,r\sin\theta)r d\theta dr$}

\section{ODEs}
Separable ODEs, reducing if necessary by $\displaystyle v=\frac{y}{x}$ or $u=ax+by$:\\
\mbox{$ \frac{dy}{dx} = f(x)g(y)\Rightarrow \int \frac{1}{g(y)}dy = \int f(x) dx + C$}

Linear ODEs using $I(x) = e^{\int P(x) dx}$:\\
\mbox{$ \frac{dy}{dx} + P(x)y = Q(x)\Rightarrow yI(x) = \int Q(x)I(x)dx$}

Bernoulli equation using $u=y^{1-n}$:\\
{\centering
  $ \frac{dy}{dx} + P(x)y = Q(x)y^n$\\$\implies \frac{du}{dx} + (1-n)P(x)u=(1-n)Q(x)$
\par}
\section*{Appendix}
{\centering
  $\sec^2x-1=\tan^2x$\\
  $\cosec^2x-1=\cot^2x$\\
  $\sin(A\pm B) = \sin A\cos B \pm \cos A\sin B$\\
  $\cos(A\pm B) = \cos A\cos B \mp \sin A\sin B$\\
  $\displaystyle \tan(A\pm B) = \frac{\tan A \pm \tan B}{1 \mp \tan A\tan B}$\\
  $\cos 2A = \cos^2A -\sin^2A = 2\cos^2A-1 = 1-2\sin^2A$\\
  $\sin P + \sin Q = 2\sin \frac{1}{2}(P+Q)\cos \frac{1}{2}(P-Q)$\\
  $\sin P - \sin Q = 2\cos \frac{1}{2}(P+Q)\sin \frac{1}{2}(P-Q)$\\
  $\cos P + \cos Q = 2\cos \frac{1}{2}(P+Q)\cos \frac{1}{2}(P-Q)$\\
  $\cos P - \cos Q = -2\sin \frac{1}{2}(P+Q)\sin \frac{1}{2}(P-Q)$\\
\par}
%%%%%%%%%%%%%%%%%%%%%%%%%%%%%%%%%%%%%%%%%%%%%%%%%%%%%%
%                       End                          %
%%%%%%%%%%%%%%%%%%%%%%%%%%%%%%%%%%%%%%%%%%%%%%%%%%%%%%
\end{multicols*}
\end{document}

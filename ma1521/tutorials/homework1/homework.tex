\documentclass[12pt, a4paper]{article}

\usepackage[a4paper, margin=1in]{geometry}

\usepackage[utf8]{inputenc}
\usepackage[mathscr]{euscript}
\let\euscr\mathscr \let\mathscr\relax
\usepackage[scr]{rsfso}
\usepackage{amssymb,amsmath,amsthm,amsfonts}
\usepackage[shortlabels]{enumitem}
\usepackage{multicol,multirow}
\usepackage{lipsum}
\usepackage{balance}
\usepackage{calc}
\usepackage[colorlinks=true,citecolor=blue,linkcolor=blue]{hyperref}
\usepackage{import}
\usepackage{xifthen}
\usepackage{pdfpages}
\usepackage{transparent}
\usepackage{listings}

\newcommand{\incfig}[2][1.0]{
    \def\svgwidth{#1\columnwidth}
    \import{./figures/}{#2.pdf_tex}
}

\newlist{enumproof}{enumerate}{4}
\setlist[enumproof,1]{label=\arabic*., parsep=1em}
\setlist[enumproof,2]{label=\arabic{enumproofi}.\arabic*., parsep=1em}
\setlist[enumproof,3]{label=\arabic{enumproofi}.\arabic{enumproofii}.\arabic*., parsep=1em}
\setlist[enumproof,4]{label=\arabic{enumproofi}.\arabic{enumproofii}.\arabic{enumproofiii}.\arabic*., parsep=1em}

\renewcommand{\qedsymbol}{\ensuremath{\blacksquare}}

\lstdefinestyle{mystyle}{
  language=C, % Set the language to C
  commentstyle=\color{codegray}, % Color for comments
  keywordstyle=\color{orange}, % Color for basic keywords
  stringstyle=\color{mauve}, % Color for strings
  basicstyle={\ttfamily\footnotesize}, % Basic font style
  breakatwhitespace=false,         
  breaklines=true,                 
  captionpos=b,                    
  keepspaces=true,                 
  numbers=none,                    
  tabsize=2,
  morekeywords=[2]{\#include, \#define, \#ifdef, \#ifndef, \#endif, \#pragma, \#else, \#elif}, % Preprocessor directives
  keywordstyle=[2]\color{codegreen}, % Style for preprocessor directives
  morekeywords=[3]{int, char, float, double, void, struct, union, enum, const, volatile, static, extern, register, inline, restrict, _Bool, _Complex, _Imaginary, size_t, ssize_t, FILE}, % C standard types and common identifiers
  keywordstyle=[3]\color{identblue}, % Style for types and common identifiers
  morekeywords=[4]{printf, scanf, fopen, fclose, malloc, free, calloc, realloc, perror, strtok, strncpy, strcpy, strcmp, strlen}, % Standard library functions
  keywordstyle=[4]\color{cyan}, % Style for library functions
}

% Things Lie
\newcommand{\kb}{\mathfrak b}
\newcommand{\kg}{\mathfrak g}
\newcommand{\kh}{\mathfrak h}
\newcommand{\kn}{\mathfrak n}
\newcommand{\ku}{\mathfrak u}
\newcommand{\kz}{\mathfrak z}
\DeclareMathOperator{\Ext}{Ext} % Ext functor
\DeclareMathOperator{\Tor}{Tor} % Tor functor
\newcommand{\gl}{\opname{\mathfrak{gl}}} % frak gl group
\renewcommand{\sl}{\opname{\mathfrak{sl}}} % frak sl group chktex 6

% More script letters etc.
\newcommand{\SA}{\mathcal A}
\newcommand{\SB}{\mathcal B}
\newcommand{\SC}{\mathcal C}
\newcommand{\SF}{\mathcal F}
\newcommand{\SG}{\mathcal G}
\newcommand{\SH}{\mathcal H}
\newcommand{\OO}{\mathcal O}

\newcommand{\SCA}{\mathscr A}
\newcommand{\SCB}{\mathscr B}
\newcommand{\SCC}{\mathscr C}
\newcommand{\SCD}{\mathscr D}
\newcommand{\SCE}{\mathscr E}
\newcommand{\SCF}{\mathscr F}
\newcommand{\SCG}{\mathscr G}
\newcommand{\SCH}{\mathscr H}

% Mathfrak primes
\newcommand{\km}{\mathfrak m}
\newcommand{\kp}{\mathfrak p}
\newcommand{\kq}{\mathfrak q}

% number sets
\newcommand{\RR}[1][]{\ensuremath{\ifstrempty{#1}{\mathbb{R}}{\mathbb{R}^{#1}}}}
\newcommand{\NN}[1][]{\ensuremath{\ifstrempty{#1}{\mathbb{N}}{\mathbb{N}^{#1}}}}
\newcommand{\ZZ}[1][]{\ensuremath{\ifstrempty{#1}{\mathbb{Z}}{\mathbb{Z}^{#1}}}}
\newcommand{\QQ}[1][]{\ensuremath{\ifstrempty{#1}{\mathbb{Q}}{\mathbb{Q}^{#1}}}}
\newcommand{\CC}[1][]{\ensuremath{\ifstrempty{#1}{\mathbb{C}}{\mathbb{C}^{#1}}}}
\newcommand{\PP}[1][]{\ensuremath{\ifstrempty{#1}{\mathbb{P}}{\mathbb{P}^{#1}}}}
\newcommand{\HH}[1][]{\ensuremath{\ifstrempty{#1}{\mathbb{H}}{\mathbb{H}^{#1}}}}
\newcommand{\FF}[1][]{\ensuremath{\ifstrempty{#1}{\mathbb{F}}{\mathbb{F}^{#1}}}}
% expected value
\newcommand{\EE}{\ensuremath{\mathbb{E}}}
\newcommand{\charin}{\text{ char }}
\DeclareMathOperator{\sign}{sign}
\DeclareMathOperator{\Aut}{Aut}
\DeclareMathOperator{\Inn}{Inn}
\DeclareMathOperator{\Syl}{Syl}
\DeclareMathOperator{\Gal}{Gal}
\DeclareMathOperator{\GL}{GL} % General linear group
\DeclareMathOperator{\SL}{SL} % Special linear group

%---------------------------------------
% BlackBoard Math Fonts :-
%---------------------------------------

%Captital Letters
\newcommand{\bbA}{\mathbb{A}}	\newcommand{\bbB}{\mathbb{B}}
\newcommand{\bbC}{\mathbb{C}}	\newcommand{\bbD}{\mathbb{D}}
\newcommand{\bbE}{\mathbb{E}}	\newcommand{\bbF}{\mathbb{F}}
\newcommand{\bbG}{\mathbb{G}}	\newcommand{\bbH}{\mathbb{H}}
\newcommand{\bbI}{\mathbb{I}}	\newcommand{\bbJ}{\mathbb{J}}
\newcommand{\bbK}{\mathbb{K}}	\newcommand{\bbL}{\mathbb{L}}
\newcommand{\bbM}{\mathbb{M}}	\newcommand{\bbN}{\mathbb{N}}
\newcommand{\bbO}{\mathbb{O}}	\newcommand{\bbP}{\mathbb{P}}
\newcommand{\bbQ}{\mathbb{Q}}	\newcommand{\bbR}{\mathbb{R}}
\newcommand{\bbS}{\mathbb{S}}	\newcommand{\bbT}{\mathbb{T}}
\newcommand{\bbU}{\mathbb{U}}	\newcommand{\bbV}{\mathbb{V}}
\newcommand{\bbW}{\mathbb{W}}	\newcommand{\bbX}{\mathbb{X}}
\newcommand{\bbY}{\mathbb{Y}}	\newcommand{\bbZ}{\mathbb{Z}}

%---------------------------------------
% MathCal Fonts :-
%---------------------------------------

%Captital Letters
\newcommand{\mcA}{\mathcal{A}}	\newcommand{\mcB}{\mathcal{B}}
\newcommand{\mcC}{\mathcal{C}}	\newcommand{\mcD}{\mathcal{D}}
\newcommand{\mcE}{\mathcal{E}}	\newcommand{\mcF}{\mathcal{F}}
\newcommand{\mcG}{\mathcal{G}}	\newcommand{\mcH}{\mathcal{H}}
\newcommand{\mcI}{\mathcal{I}}	\newcommand{\mcJ}{\mathcal{J}}
\newcommand{\mcK}{\mathcal{K}}	\newcommand{\mcL}{\mathcal{L}}
\newcommand{\mcM}{\mathcal{M}}	\newcommand{\mcN}{\mathcal{N}}
\newcommand{\mcO}{\mathcal{O}}	\newcommand{\mcP}{\mathcal{P}}
\newcommand{\mcQ}{\mathcal{Q}}	\newcommand{\mcR}{\mathcal{R}}
\newcommand{\mcS}{\mathcal{S}}	\newcommand{\mcT}{\mathcal{T}}
\newcommand{\mcU}{\mathcal{U}}	\newcommand{\mcV}{\mathcal{V}}
\newcommand{\mcW}{\mathcal{W}}	\newcommand{\mcX}{\mathcal{X}}
\newcommand{\mcY}{\mathcal{Y}}	\newcommand{\mcZ}{\mathcal{Z}}

%---------------------------------------
% Bold Math Fonts :-
%---------------------------------------

%Captital Letters
\newcommand{\bmA}{\boldsymbol{A}}	\newcommand{\bmB}{\boldsymbol{B}}
\newcommand{\bmC}{\boldsymbol{C}}	\newcommand{\bmD}{\boldsymbol{D}}
\newcommand{\bmE}{\boldsymbol{E}}	\newcommand{\bmF}{\boldsymbol{F}}
\newcommand{\bmG}{\boldsymbol{G}}	\newcommand{\bmH}{\boldsymbol{H}}
\newcommand{\bmI}{\boldsymbol{I}}	\newcommand{\bmJ}{\boldsymbol{J}}
\newcommand{\bmK}{\boldsymbol{K}}	\newcommand{\bmL}{\boldsymbol{L}}
\newcommand{\bmM}{\boldsymbol{M}}	\newcommand{\bmN}{\boldsymbol{N}}
\newcommand{\bmO}{\boldsymbol{O}}	\newcommand{\bmP}{\boldsymbol{P}}
\newcommand{\bmQ}{\boldsymbol{Q}}	\newcommand{\bmR}{\boldsymbol{R}}
\newcommand{\bmS}{\boldsymbol{S}}	\newcommand{\bmT}{\boldsymbol{T}}
\newcommand{\bmU}{\boldsymbol{U}}	\newcommand{\bmV}{\boldsymbol{V}}
\newcommand{\bmW}{\boldsymbol{W}}	\newcommand{\bmX}{\boldsymbol{X}}
\newcommand{\bmY}{\boldsymbol{Y}}	\newcommand{\bmZ}{\boldsymbol{Z}}
%Small Letters
\newcommand{\bma}{\boldsymbol{a}}	\newcommand{\bmb}{\boldsymbol{b}}
\newcommand{\bmc}{\boldsymbol{c}}	\newcommand{\bmd}{\boldsymbol{d}}
\newcommand{\bme}{\boldsymbol{e}}	\newcommand{\bmf}{\boldsymbol{f}}
\newcommand{\bmg}{\boldsymbol{g}}	\newcommand{\bmh}{\boldsymbol{h}}
\newcommand{\bmi}{\boldsymbol{i}}	\newcommand{\bmj}{\boldsymbol{j}}
\newcommand{\bmk}{\boldsymbol{k}}	\newcommand{\bml}{\boldsymbol{l}}
\newcommand{\bmm}{\boldsymbol{m}}	\newcommand{\bmn}{\boldsymbol{n}}
\newcommand{\bmo}{\boldsymbol{o}}	\newcommand{\bmp}{\boldsymbol{p}}
\newcommand{\bmq}{\boldsymbol{q}}	\newcommand{\bmr}{\boldsymbol{r}}
\newcommand{\bms}{\boldsymbol{s}}	\newcommand{\bmt}{\boldsymbol{t}}
\newcommand{\bmu}{\boldsymbol{u}}	\newcommand{\bmv}{\boldsymbol{v}}
\newcommand{\bmw}{\boldsymbol{w}}	\newcommand{\bmx}{\boldsymbol{x}}
\newcommand{\bmy}{\boldsymbol{y}}	\newcommand{\bmz}{\boldsymbol{z}}

%---------------------------------------
% Scr Math Fonts :-
%---------------------------------------

\newcommand{\sA}{{\mathscr{A}}}   \newcommand{\sB}{{\mathscr{B}}}
\newcommand{\sC}{{\mathscr{C}}}   \newcommand{\sD}{{\mathscr{D}}}
\newcommand{\sE}{{\mathscr{E}}}   \newcommand{\sF}{{\mathscr{F}}}
\newcommand{\sG}{{\mathscr{G}}}   \newcommand{\sH}{{\mathscr{H}}}
\newcommand{\sI}{{\mathscr{I}}}   \newcommand{\sJ}{{\mathscr{J}}}
\newcommand{\sK}{{\mathscr{K}}}   \newcommand{\sL}{{\mathscr{L}}}
\newcommand{\sM}{{\mathscr{M}}}   \newcommand{\sN}{{\mathscr{N}}}
\newcommand{\sO}{{\mathscr{O}}}   \newcommand{\sP}{{\mathscr{P}}}
\newcommand{\sQ}{{\mathscr{Q}}}   \newcommand{\sR}{{\mathscr{R}}}
\newcommand{\sS}{{\mathscr{S}}}   \newcommand{\sT}{{\mathscr{T}}}
\newcommand{\sU}{{\mathscr{U}}}   \newcommand{\sV}{{\mathscr{V}}}
\newcommand{\sW}{{\mathscr{W}}}   \newcommand{\sX}{{\mathscr{X}}}
\newcommand{\sY}{{\mathscr{Y}}}   \newcommand{\sZ}{{\mathscr{Z}}}


%---------------------------------------
% Math Fraktur Font
%---------------------------------------

%Captital Letters
\newcommand{\mfA}{\mathfrak{A}}	\newcommand{\mfB}{\mathfrak{B}}
\newcommand{\mfC}{\mathfrak{C}}	\newcommand{\mfD}{\mathfrak{D}}
\newcommand{\mfE}{\mathfrak{E}}	\newcommand{\mfF}{\mathfrak{F}}
\newcommand{\mfG}{\mathfrak{G}}	\newcommand{\mfH}{\mathfrak{H}}
\newcommand{\mfI}{\mathfrak{I}}	\newcommand{\mfJ}{\mathfrak{J}}
\newcommand{\mfK}{\mathfrak{K}}	\newcommand{\mfL}{\mathfrak{L}}
\newcommand{\mfM}{\mathfrak{M}}	\newcommand{\mfN}{\mathfrak{N}}
\newcommand{\mfO}{\mathfrak{O}}	\newcommand{\mfP}{\mathfrak{P}}
\newcommand{\mfQ}{\mathfrak{Q}}	\newcommand{\mfR}{\mathfrak{R}}
\newcommand{\mfS}{\mathfrak{S}}	\newcommand{\mfT}{\mathfrak{T}}
\newcommand{\mfU}{\mathfrak{U}}	\newcommand{\mfV}{\mathfrak{V}}
\newcommand{\mfW}{\mathfrak{W}}	\newcommand{\mfX}{\mathfrak{X}}
\newcommand{\mfY}{\mathfrak{Y}}	\newcommand{\mfZ}{\mathfrak{Z}}
%Small Letters
\newcommand{\mfa}{\mathfrak{a}}	\newcommand{\mfb}{\mathfrak{b}}
\newcommand{\mfc}{\mathfrak{c}}	\newcommand{\mfd}{\mathfrak{d}}
\newcommand{\mfe}{\mathfrak{e}}	\newcommand{\mff}{\mathfrak{f}}
\newcommand{\mfg}{\mathfrak{g}}	\newcommand{\mfh}{\mathfrak{h}}
\newcommand{\mfi}{\mathfrak{i}}	\newcommand{\mfj}{\mathfrak{j}}
\newcommand{\mfk}{\mathfrak{k}}	\newcommand{\mfl}{\mathfrak{l}}
\newcommand{\mfm}{\mathfrak{m}}	\newcommand{\mfn}{\mathfrak{n}}
\newcommand{\mfo}{\mathfrak{o}}	\newcommand{\mfp}{\mathfrak{p}}
\newcommand{\mfq}{\mathfrak{q}}	\newcommand{\mfr}{\mathfrak{r}}
\newcommand{\mfs}{\mathfrak{s}}	\newcommand{\mft}{\mathfrak{t}}
\newcommand{\mfu}{\mathfrak{u}}	\newcommand{\mfv}{\mathfrak{v}}
\newcommand{\mfw}{\mathfrak{w}}	\newcommand{\mfx}{\mathfrak{x}}
\newcommand{\mfy}{\mathfrak{y}}	\newcommand{\mfz}{\mathfrak{z}}


\newcommand{\mytitle}{MA1521 Homework 1}
\newcommand{\myauthor}{github/omgeta}
\newcommand{\mydate}{AY 24/25 Sem 1}

\begin{document}
\raggedright
\footnotesize
\begin{center}
{\normalsize{\textbf{\mytitle}}} \\
{\footnotesize{\mydate\hspace{2pt}\textemdash\hspace{2pt}\myauthor}}
\end{center}
\setlist{topsep=-1em, itemsep=-1em, parsep=2em}

%%%%%%%%%%%%%%%%%%%%%%%%%%%%%%%%%%%%%%%%%%%%%%%%%%%%%%
%                      Begin                         %
%%%%%%%%%%%%%%%%%%%%%%%%%%%%%%%%%%%%%%%%%%%%%%%%%%%%%%
\begin{enumerate}[Q\arabic*.]
  \item For each of the following functions, find all real values of $x$ for which it is defined, i.e.
the maximal domain of each function:
  \begin{enumerate}[(\alph*)]
    \item $\displaystyle f(x) = \frac{81 - x^2}{(4+x^2)(27-x^3)(16-x^4)}$

      For $f(x)$ to be defined, denominator $(4+x^2)(27-x^3)(16-x^4) \neq 0$.
    \begin{align*}
      4 + x^2 = 0 &\implies \text{(no real solutions)} \\
      27 - x^3 = 0 &\implies x = 3 \\
      16 - x^4 = 0 &\implies x = \pm 2
    \end{align*}
    Therefore, domain of \( f(x) \) is:
    \[
    \RR \setminus \{-2, 2, 3\} \qed
    \]
  \item $g(x) = \sqrt{2 - \ln(x+1)}$

    For $g(x)$ to be defined, argument $(2 - \ln(x+1))$ must be non-negative
    \begin{align*}
      2 - \ln(x+1) &\geq 0 \\
      2 &\geq \ln(x+1) \\
      e^2 &\geq x + 1 \\
      e^2 - 1 &\geq x
    \end{align*}
    For $\ln(x+1)$ to be defined, $x + 1 > 0 \implies x > -1 $. 
    
    Therefore, domain of $g(x)$ is:
    \[
      \{x \in \RR : -1 < x \leq e^2 - 1\} \qed
    \]

  \item $\displaystyle h(x) = \frac{\ln(\sqrt{16-4x}+1)}{\sqrt{\ln x}-1}$

      For $\ln(\sqrt{16-4x}+1)$ to be defined
        \begin{align*}
          \sqrt{16-4x} + 1 &> 0 \\
          16 - 4x &\geq 0 \\
          x &\leq 4
        \end{align*}
      For $\sqrt{\ln x}$ to be defined, $x > 0$

      For $h(x)$ to be defined, denominator must be non-zero 
        \begin{align*}
          \sqrt{\ln x} - 1 &\neq 0 \\
          \ln x &\neq 1 \\
          x &\neq e
        \end{align*}
    Therefore, domain of \( h(x) \) is:
    \[
    0 < x \leq 4 \text{ and } x \neq e \qed
    \] 
  \end{enumerate}

\item Let $f(x)$ be defined on $(-\infty, \infty)$ such that $f(x) = \begin{cases} 
      4 & x \leq -2 \\
      x^2-1 & -2 < x\leq-1 \\
      0 & -1< x \leq 1 \\
      \displaystyle\frac{1}{x-1} & x > 1
   \end{cases}
$ \\ Find all $x$ such that $f$ is not continuous at $x$

\[
x = -2, 1 \qed
\]

\item Let $f(x)$ be defined on $[0, 8]$ such that $f(x) = \begin{cases} 
    p^\frac{1}{3}\sqrt{x} & 0 \leq x < 4 \\
      7 & x=4 \\
      q(x-2)^2 + 5 & 4 < x \leq 6 \\
      \displaystyle\frac{2r}{x-5} & 6 < x \leq 8
   \end{cases}$ \\
   It is given that $f$ is continuous at $x = 4$ and $\lim_{x\to6} f(x)$  exists. Find the values of $p, q, r$.
   
Since $f$ is continuous at $x = 4$,
\begin{align*}
  p^\frac{1}{3}\sqrt{4} &= 7 \\
  p^\frac{1}{3} &= \frac{7}{2} \\
  p &= \frac{343}{8} \qed \\
    &\\
  q(4 - 2)^2 + 5 &= 7 \\
  4q &= 2 \\ 
  q &= \frac{1}{2} \qed
\end{align*}
Since $\lim_{x\to6} f(x)$ exists, when $x = 6$,
\begin{align*}
  \frac{1}{2}(6-2)^2 + 5 &= \frac{2r}{6-5} \\
  8 + 5 &= 2r \\
  r &= \frac{13}{2} \qed
\end{align*}

\item Evaluate each of the following limits if it exists:
  \begin{enumerate}[(\alph*)]
    \item $\displaystyle \lim_{x\to2}\frac{4-x^2}{x^2-3x+2}$
      \begin{align*}
        \lim_{x\to2}\frac{4-x^2}{x^2-3x+2} &=  \lim_{x\to2}\frac{-(x-2)(x+2)}{(x-2)(x-1)} \\
                                           &=  \lim_{x\to2}\frac{-(x+2)}{x-1} \\
                                           &= \frac{-(2+2)}{2-1} \\
                                           &= -4 \qed
      \end{align*}
    \item $\displaystyle \lim_{x\to-2}\frac{4-x^2}{\sqrt{x^2-x-2}-\sqrt{2-x}}$
      \begin{align*}
        \lim_{x\to-2}\frac{4-x^2}{\sqrt{x^2-x-2}-\sqrt{2-x}} &= \lim_{x\to-2}\frac{(4-x^2)(\sqrt{x^2-x-2}+\sqrt{2-x})}{(\sqrt{x^2-x-2})^2-(\sqrt{2-x})^2} \\
                                                             &= \lim_{x\to-2}\frac{(4-x^2)(\sqrt{x^2-x-2}+\sqrt{2-x})}{(x^2-x-2)-(2-x)} \\
                                                             &= \lim_{x\to-2}\frac{-(x^2-4)(\sqrt{x^2-x-2}+\sqrt{2-x})}{x^2-4} \\
                                                             &= \lim_{x\to-2}-(\sqrt{x^2-x-2}+\sqrt{2-x}) \\
                                                             &= -4 \qed
      \end{align*}
    \item $\displaystyle \lim_{x\to2}\frac{x^3-8}{(x-2)^2}$
      \begin{align*}
        \lim_{x\to2}\frac{x^3-8}{(x-2)^2} &= \lim_{x\to2}\frac{(x-2)(x^2+2x+4)}{(x-2)^2}
                                          &= \lim_{x\to2}\frac{x^2+2x+4}{x-2}
                                          &= \frac{12}{0}
      \end{align*}
       Limit is $\pm\infty$ depending on LHS or RHS limit, therefore limit does not exist \qed
  \end{enumerate}

  \item Evaluate the following limits:
    \begin{enumerate}[(\alph*)]
      \item $\displaystyle \lim_{x\to\infty}\sqrt{\frac{9x^{10}+3x-1}{(x^2+3x-5)^3(2x+5)^4}}$
        \begin{align*}
        \lim_{x\to\infty}\sqrt{\frac{9x^{10}+3x-1}{(x^2+3x-5)^3(2x+5)^4}} &= \lim_{x\to\infty}\sqrt{\frac{9x^{10}+\ldots}{16x^{10}+\ldots}}\\
                                                                          &= \sqrt{\frac{9}{16}} \\
                                                                          &= \frac{3}{4} \qed
        \end{align*}
      \item $\displaystyle \lim_{x\to-\infty}\frac{1}{x}\sqrt{\frac{9x^{10}+3x-1}{(x^2+3x-5)^3(2x+5)^2}}$
        \begin{align*}
          \lim_{x\to-\infty}\frac{1}{x}\sqrt{\frac{9x^{10}+3x-1}{(x^2+3x-5)^3(2x+5)^2}} &= \lim_{x\to-\infty}\frac{1}{x}\sqrt{\frac{9x^{10}+\ldots}{4x^8+\ldots}} \\
                                                                                        &= \lim_{x\to-\infty}\sqrt{\frac{9x^{10}+\ldots}{4x^{10}+\ldots}} \\
                                                                 &= -\sqrt{\frac{9}{4}} \\
                                                                 &= -\frac{3}{2} \qed
        \end{align*}
      \item $\displaystyle \lim_{x\to-\infty}\frac{\sqrt{9x^{10}+3x-1}}{(1+2x)^2(x^2+x-1)}$
        \begin{align*}
          \lim_{x\to-\infty}\frac{\sqrt{9x^{10}+3x-1}}{(1+2x)^2(x^2+x-1)} &= \lim_{x\to-\infty}\frac{3x^5+\ldots}{4x^4+\ldots} \\
                                                                          &= -\infty \qed
        \end{align*}
    \end{enumerate}
\end{enumerate}
%%%%%%%%%%%%%%%%%%%%%%%%%%%%%%%%%%%%%%%%%%%%%%%%%%%%%%
%                       End                          %
%%%%%%%%%%%%%%%%%%%%%%%%%%%%%%%%%%%%%%%%%%%%%%%%%%%%%%

\end{document}

\documentclass[12pt, a4paper]{article}

\usepackage[a4paper, margin=1in]{geometry}

\usepackage[utf8]{inputenc}
\usepackage[mathscr]{euscript}
\let\euscr\mathscr \let\mathscr\relax
\usepackage[scr]{rsfso}
\usepackage{amssymb,amsmath,amsthm,amsfonts}
\usepackage[shortlabels]{enumitem}
\usepackage{multicol,multirow}
\usepackage{lipsum}
\usepackage{balance}
\usepackage{calc}
\usepackage[colorlinks=true,citecolor=blue,linkcolor=blue]{hyperref}
\usepackage{import}
\usepackage{xifthen}
\usepackage{pdfpages}
\usepackage{transparent}
\usepackage{listings}

\newcommand{\incfig}[2][1.0]{
    \def\svgwidth{#1\columnwidth}
    \import{./figures/}{#2.pdf_tex}
}

\newlist{enumproof}{enumerate}{4}
\setlist[enumproof,1]{label=\arabic*., parsep=1em}
\setlist[enumproof,2]{label=\arabic{enumproofi}.\arabic*., parsep=1em}
\setlist[enumproof,3]{label=\arabic{enumproofi}.\arabic{enumproofii}.\arabic*., parsep=1em}
\setlist[enumproof,4]{label=\arabic{enumproofi}.\arabic{enumproofii}.\arabic{enumproofiii}.\arabic*., parsep=1em}

\renewcommand{\qedsymbol}{\ensuremath{\blacksquare}}

\lstdefinestyle{mystyle}{
  language=C, % Set the language to C
  commentstyle=\color{codegray}, % Color for comments
  keywordstyle=\color{orange}, % Color for basic keywords
  stringstyle=\color{mauve}, % Color for strings
  basicstyle={\ttfamily\footnotesize}, % Basic font style
  breakatwhitespace=false,         
  breaklines=true,                 
  captionpos=b,                    
  keepspaces=true,                 
  numbers=none,                    
  tabsize=2,
  morekeywords=[2]{\#include, \#define, \#ifdef, \#ifndef, \#endif, \#pragma, \#else, \#elif}, % Preprocessor directives
  keywordstyle=[2]\color{codegreen}, % Style for preprocessor directives
  morekeywords=[3]{int, char, float, double, void, struct, union, enum, const, volatile, static, extern, register, inline, restrict, _Bool, _Complex, _Imaginary, size_t, ssize_t, FILE}, % C standard types and common identifiers
  keywordstyle=[3]\color{identblue}, % Style for types and common identifiers
  morekeywords=[4]{printf, scanf, fopen, fclose, malloc, free, calloc, realloc, perror, strtok, strncpy, strcpy, strcmp, strlen}, % Standard library functions
  keywordstyle=[4]\color{cyan}, % Style for library functions
}

% Things Lie
\newcommand{\kb}{\mathfrak b}
\newcommand{\kg}{\mathfrak g}
\newcommand{\kh}{\mathfrak h}
\newcommand{\kn}{\mathfrak n}
\newcommand{\ku}{\mathfrak u}
\newcommand{\kz}{\mathfrak z}
\DeclareMathOperator{\Ext}{Ext} % Ext functor
\DeclareMathOperator{\Tor}{Tor} % Tor functor
\newcommand{\gl}{\opname{\mathfrak{gl}}} % frak gl group
\renewcommand{\sl}{\opname{\mathfrak{sl}}} % frak sl group chktex 6

% More script letters etc.
\newcommand{\SA}{\mathcal A}
\newcommand{\SB}{\mathcal B}
\newcommand{\SC}{\mathcal C}
\newcommand{\SF}{\mathcal F}
\newcommand{\SG}{\mathcal G}
\newcommand{\SH}{\mathcal H}
\newcommand{\OO}{\mathcal O}

\newcommand{\SCA}{\mathscr A}
\newcommand{\SCB}{\mathscr B}
\newcommand{\SCC}{\mathscr C}
\newcommand{\SCD}{\mathscr D}
\newcommand{\SCE}{\mathscr E}
\newcommand{\SCF}{\mathscr F}
\newcommand{\SCG}{\mathscr G}
\newcommand{\SCH}{\mathscr H}

% Mathfrak primes
\newcommand{\km}{\mathfrak m}
\newcommand{\kp}{\mathfrak p}
\newcommand{\kq}{\mathfrak q}

% number sets
\newcommand{\RR}[1][]{\ensuremath{\ifstrempty{#1}{\mathbb{R}}{\mathbb{R}^{#1}}}}
\newcommand{\NN}[1][]{\ensuremath{\ifstrempty{#1}{\mathbb{N}}{\mathbb{N}^{#1}}}}
\newcommand{\ZZ}[1][]{\ensuremath{\ifstrempty{#1}{\mathbb{Z}}{\mathbb{Z}^{#1}}}}
\newcommand{\QQ}[1][]{\ensuremath{\ifstrempty{#1}{\mathbb{Q}}{\mathbb{Q}^{#1}}}}
\newcommand{\CC}[1][]{\ensuremath{\ifstrempty{#1}{\mathbb{C}}{\mathbb{C}^{#1}}}}
\newcommand{\PP}[1][]{\ensuremath{\ifstrempty{#1}{\mathbb{P}}{\mathbb{P}^{#1}}}}
\newcommand{\HH}[1][]{\ensuremath{\ifstrempty{#1}{\mathbb{H}}{\mathbb{H}^{#1}}}}
\newcommand{\FF}[1][]{\ensuremath{\ifstrempty{#1}{\mathbb{F}}{\mathbb{F}^{#1}}}}
% expected value
\newcommand{\EE}{\ensuremath{\mathbb{E}}}
\newcommand{\charin}{\text{ char }}
\DeclareMathOperator{\sign}{sign}
\DeclareMathOperator{\Aut}{Aut}
\DeclareMathOperator{\Inn}{Inn}
\DeclareMathOperator{\Syl}{Syl}
\DeclareMathOperator{\Gal}{Gal}
\DeclareMathOperator{\GL}{GL} % General linear group
\DeclareMathOperator{\SL}{SL} % Special linear group

%---------------------------------------
% BlackBoard Math Fonts :-
%---------------------------------------

%Captital Letters
\newcommand{\bbA}{\mathbb{A}}	\newcommand{\bbB}{\mathbb{B}}
\newcommand{\bbC}{\mathbb{C}}	\newcommand{\bbD}{\mathbb{D}}
\newcommand{\bbE}{\mathbb{E}}	\newcommand{\bbF}{\mathbb{F}}
\newcommand{\bbG}{\mathbb{G}}	\newcommand{\bbH}{\mathbb{H}}
\newcommand{\bbI}{\mathbb{I}}	\newcommand{\bbJ}{\mathbb{J}}
\newcommand{\bbK}{\mathbb{K}}	\newcommand{\bbL}{\mathbb{L}}
\newcommand{\bbM}{\mathbb{M}}	\newcommand{\bbN}{\mathbb{N}}
\newcommand{\bbO}{\mathbb{O}}	\newcommand{\bbP}{\mathbb{P}}
\newcommand{\bbQ}{\mathbb{Q}}	\newcommand{\bbR}{\mathbb{R}}
\newcommand{\bbS}{\mathbb{S}}	\newcommand{\bbT}{\mathbb{T}}
\newcommand{\bbU}{\mathbb{U}}	\newcommand{\bbV}{\mathbb{V}}
\newcommand{\bbW}{\mathbb{W}}	\newcommand{\bbX}{\mathbb{X}}
\newcommand{\bbY}{\mathbb{Y}}	\newcommand{\bbZ}{\mathbb{Z}}

%---------------------------------------
% MathCal Fonts :-
%---------------------------------------

%Captital Letters
\newcommand{\mcA}{\mathcal{A}}	\newcommand{\mcB}{\mathcal{B}}
\newcommand{\mcC}{\mathcal{C}}	\newcommand{\mcD}{\mathcal{D}}
\newcommand{\mcE}{\mathcal{E}}	\newcommand{\mcF}{\mathcal{F}}
\newcommand{\mcG}{\mathcal{G}}	\newcommand{\mcH}{\mathcal{H}}
\newcommand{\mcI}{\mathcal{I}}	\newcommand{\mcJ}{\mathcal{J}}
\newcommand{\mcK}{\mathcal{K}}	\newcommand{\mcL}{\mathcal{L}}
\newcommand{\mcM}{\mathcal{M}}	\newcommand{\mcN}{\mathcal{N}}
\newcommand{\mcO}{\mathcal{O}}	\newcommand{\mcP}{\mathcal{P}}
\newcommand{\mcQ}{\mathcal{Q}}	\newcommand{\mcR}{\mathcal{R}}
\newcommand{\mcS}{\mathcal{S}}	\newcommand{\mcT}{\mathcal{T}}
\newcommand{\mcU}{\mathcal{U}}	\newcommand{\mcV}{\mathcal{V}}
\newcommand{\mcW}{\mathcal{W}}	\newcommand{\mcX}{\mathcal{X}}
\newcommand{\mcY}{\mathcal{Y}}	\newcommand{\mcZ}{\mathcal{Z}}

%---------------------------------------
% Bold Math Fonts :-
%---------------------------------------

%Captital Letters
\newcommand{\bmA}{\boldsymbol{A}}	\newcommand{\bmB}{\boldsymbol{B}}
\newcommand{\bmC}{\boldsymbol{C}}	\newcommand{\bmD}{\boldsymbol{D}}
\newcommand{\bmE}{\boldsymbol{E}}	\newcommand{\bmF}{\boldsymbol{F}}
\newcommand{\bmG}{\boldsymbol{G}}	\newcommand{\bmH}{\boldsymbol{H}}
\newcommand{\bmI}{\boldsymbol{I}}	\newcommand{\bmJ}{\boldsymbol{J}}
\newcommand{\bmK}{\boldsymbol{K}}	\newcommand{\bmL}{\boldsymbol{L}}
\newcommand{\bmM}{\boldsymbol{M}}	\newcommand{\bmN}{\boldsymbol{N}}
\newcommand{\bmO}{\boldsymbol{O}}	\newcommand{\bmP}{\boldsymbol{P}}
\newcommand{\bmQ}{\boldsymbol{Q}}	\newcommand{\bmR}{\boldsymbol{R}}
\newcommand{\bmS}{\boldsymbol{S}}	\newcommand{\bmT}{\boldsymbol{T}}
\newcommand{\bmU}{\boldsymbol{U}}	\newcommand{\bmV}{\boldsymbol{V}}
\newcommand{\bmW}{\boldsymbol{W}}	\newcommand{\bmX}{\boldsymbol{X}}
\newcommand{\bmY}{\boldsymbol{Y}}	\newcommand{\bmZ}{\boldsymbol{Z}}
%Small Letters
\newcommand{\bma}{\boldsymbol{a}}	\newcommand{\bmb}{\boldsymbol{b}}
\newcommand{\bmc}{\boldsymbol{c}}	\newcommand{\bmd}{\boldsymbol{d}}
\newcommand{\bme}{\boldsymbol{e}}	\newcommand{\bmf}{\boldsymbol{f}}
\newcommand{\bmg}{\boldsymbol{g}}	\newcommand{\bmh}{\boldsymbol{h}}
\newcommand{\bmi}{\boldsymbol{i}}	\newcommand{\bmj}{\boldsymbol{j}}
\newcommand{\bmk}{\boldsymbol{k}}	\newcommand{\bml}{\boldsymbol{l}}
\newcommand{\bmm}{\boldsymbol{m}}	\newcommand{\bmn}{\boldsymbol{n}}
\newcommand{\bmo}{\boldsymbol{o}}	\newcommand{\bmp}{\boldsymbol{p}}
\newcommand{\bmq}{\boldsymbol{q}}	\newcommand{\bmr}{\boldsymbol{r}}
\newcommand{\bms}{\boldsymbol{s}}	\newcommand{\bmt}{\boldsymbol{t}}
\newcommand{\bmu}{\boldsymbol{u}}	\newcommand{\bmv}{\boldsymbol{v}}
\newcommand{\bmw}{\boldsymbol{w}}	\newcommand{\bmx}{\boldsymbol{x}}
\newcommand{\bmy}{\boldsymbol{y}}	\newcommand{\bmz}{\boldsymbol{z}}

%---------------------------------------
% Scr Math Fonts :-
%---------------------------------------

\newcommand{\sA}{{\mathscr{A}}}   \newcommand{\sB}{{\mathscr{B}}}
\newcommand{\sC}{{\mathscr{C}}}   \newcommand{\sD}{{\mathscr{D}}}
\newcommand{\sE}{{\mathscr{E}}}   \newcommand{\sF}{{\mathscr{F}}}
\newcommand{\sG}{{\mathscr{G}}}   \newcommand{\sH}{{\mathscr{H}}}
\newcommand{\sI}{{\mathscr{I}}}   \newcommand{\sJ}{{\mathscr{J}}}
\newcommand{\sK}{{\mathscr{K}}}   \newcommand{\sL}{{\mathscr{L}}}
\newcommand{\sM}{{\mathscr{M}}}   \newcommand{\sN}{{\mathscr{N}}}
\newcommand{\sO}{{\mathscr{O}}}   \newcommand{\sP}{{\mathscr{P}}}
\newcommand{\sQ}{{\mathscr{Q}}}   \newcommand{\sR}{{\mathscr{R}}}
\newcommand{\sS}{{\mathscr{S}}}   \newcommand{\sT}{{\mathscr{T}}}
\newcommand{\sU}{{\mathscr{U}}}   \newcommand{\sV}{{\mathscr{V}}}
\newcommand{\sW}{{\mathscr{W}}}   \newcommand{\sX}{{\mathscr{X}}}
\newcommand{\sY}{{\mathscr{Y}}}   \newcommand{\sZ}{{\mathscr{Z}}}


%---------------------------------------
% Math Fraktur Font
%---------------------------------------

%Captital Letters
\newcommand{\mfA}{\mathfrak{A}}	\newcommand{\mfB}{\mathfrak{B}}
\newcommand{\mfC}{\mathfrak{C}}	\newcommand{\mfD}{\mathfrak{D}}
\newcommand{\mfE}{\mathfrak{E}}	\newcommand{\mfF}{\mathfrak{F}}
\newcommand{\mfG}{\mathfrak{G}}	\newcommand{\mfH}{\mathfrak{H}}
\newcommand{\mfI}{\mathfrak{I}}	\newcommand{\mfJ}{\mathfrak{J}}
\newcommand{\mfK}{\mathfrak{K}}	\newcommand{\mfL}{\mathfrak{L}}
\newcommand{\mfM}{\mathfrak{M}}	\newcommand{\mfN}{\mathfrak{N}}
\newcommand{\mfO}{\mathfrak{O}}	\newcommand{\mfP}{\mathfrak{P}}
\newcommand{\mfQ}{\mathfrak{Q}}	\newcommand{\mfR}{\mathfrak{R}}
\newcommand{\mfS}{\mathfrak{S}}	\newcommand{\mfT}{\mathfrak{T}}
\newcommand{\mfU}{\mathfrak{U}}	\newcommand{\mfV}{\mathfrak{V}}
\newcommand{\mfW}{\mathfrak{W}}	\newcommand{\mfX}{\mathfrak{X}}
\newcommand{\mfY}{\mathfrak{Y}}	\newcommand{\mfZ}{\mathfrak{Z}}
%Small Letters
\newcommand{\mfa}{\mathfrak{a}}	\newcommand{\mfb}{\mathfrak{b}}
\newcommand{\mfc}{\mathfrak{c}}	\newcommand{\mfd}{\mathfrak{d}}
\newcommand{\mfe}{\mathfrak{e}}	\newcommand{\mff}{\mathfrak{f}}
\newcommand{\mfg}{\mathfrak{g}}	\newcommand{\mfh}{\mathfrak{h}}
\newcommand{\mfi}{\mathfrak{i}}	\newcommand{\mfj}{\mathfrak{j}}
\newcommand{\mfk}{\mathfrak{k}}	\newcommand{\mfl}{\mathfrak{l}}
\newcommand{\mfm}{\mathfrak{m}}	\newcommand{\mfn}{\mathfrak{n}}
\newcommand{\mfo}{\mathfrak{o}}	\newcommand{\mfp}{\mathfrak{p}}
\newcommand{\mfq}{\mathfrak{q}}	\newcommand{\mfr}{\mathfrak{r}}
\newcommand{\mfs}{\mathfrak{s}}	\newcommand{\mft}{\mathfrak{t}}
\newcommand{\mfu}{\mathfrak{u}}	\newcommand{\mfv}{\mathfrak{v}}
\newcommand{\mfw}{\mathfrak{w}}	\newcommand{\mfx}{\mathfrak{x}}
\newcommand{\mfy}{\mathfrak{y}}	\newcommand{\mfz}{\mathfrak{z}}

\newcommand{\mytitle}{MA1521 Homework 3}
\newcommand{\myauthor}{github/omgeta}
\newcommand{\mydate}{AY 24/25 Sem 1}

\begin{document}
\raggedright
\footnotesize
\begin{center}
{\normalsize{\textbf{\mytitle}}} \\
{\footnotesize{\mydate\hspace{2pt}\textemdash\hspace{2pt}\myauthor}}
\end{center}
\setlist{topsep=-1em, itemsep=-1em, parsep=2em}

%%%%%%%%%%%%%%%%%%%%%%%%%%%%%%%%%%%%%%%%%%%%%%%%%%%%%%
%                      Begin                         %
%%%%%%%%%%%%%%%%%%%%%%%%%%%%%%%%%%%%%%%%%%%%%%%%%%%%%%
\begin{enumerate}[Q\arabic*.]
  \item \quad\\
    \begin{figure}[ht]
      \centering
      \incfig[0.4]{graph}
    \end{figure}

  \item $f'(x) = \sec^2x + \sec x + \tan x > 0$ when $f'(x)$ and $f(x)$ are defined.\\
    $\tan x$ is not defined for $x = \frac{\pi}{2}, \frac{3\pi}{2}$ $\implies f(x)$ is not defined for $x = \frac{\pi}{2}, \frac{3\pi}{2}$.\\
    $\therefore f$ is defined and $f'(x) > 0$ in the interval $(0, 2\pi)$ for $x \in (0, 2\pi) \setminus \{\frac{\pi}{2}, \frac{3\pi}{2}\} \qed$
  \item By observing the graph,\\ 
    \textbf{Local maximum: }$x = 1, 6 \qed$\\
    \textbf{Local minimum: }$x = -2, 2 \qed$\\
    \textbf{Absolute maximum: }$x = 6 \qed$\\
    \textbf{Absolute minimum: }$x = 2 \qed$\\
  \item 
    \begin{enumerate}[(\roman*)]
      \item Suppose $\displaystyle y = \frac{x+1}{x^2+1}$, for $x \in [-3, 3]$.\\
        First, find the first derivative $y'$,
        \begin{align*}
          y' &= \frac{(x^2+1)(1) - (x+1)(2x)}{(x^2+1)^2} \\
             &= \frac{-x^2-2x+1}{(x^2+1)^2}\\
        \end{align*}
        Critical points are found at the points where $f'(x) = 0$
        \begin{align*}
          \frac{-x^2-2x+1}{(x^2+1)^2}&= 0\\
          -x^2-2x+1 &= 0\\
          x^2+2x-1 &= 0\\
          x &= \frac{-2\pm\sqrt{2^2-4(1)(-1)}}{2(1)}\\
            &= \frac{-2\pm\sqrt{8}}{2}\\
            &= -1\pm\sqrt{2}
        \end{align*}
        When $x = -1+\sqrt{2}, y = \frac{\sqrt{2}}{4-2\sqrt{2}} = \frac{\sqrt{2}+1}{2}$\\
        When $x = -1-\sqrt{2}, y = \frac{-\sqrt{2}}{4+2\sqrt{2}} = \frac{-\sqrt{2}+1}{2}$\\
        $\therefore$ the critical points are $\displaystyle (-1+\sqrt{2}, \frac{\sqrt{2}+1}{2}), (-1-\sqrt{2}, \frac{-\sqrt{2}+1}{2}) \qed$

    \item When $x = (-1+\sqrt{2})_-$, $y' > 0$ and when $x = (-1+\sqrt{2})_+$, $y' < 0$. \\
    When $x = (-1-\sqrt{2})_-$, $y' < 0$ and when $x = (-1-\sqrt{2})_+$, $y' > 0$. \\
    $\therefore f$ is decreasing in $[-3, -1-\sqrt{2}) \cup (-1+\sqrt{2}, 3]$ and increasing in $(-1-\sqrt{2}, -1+\sqrt{2}) \qed$ 

  \item By the first derivative test, $\displaystyle y = \frac{\sqrt{2}+1}{2}$ is a local and absolute minimum, $\displaystyle y = \frac{-\sqrt{2}+1}{2}$ is a local and the absolute maximum $\qed$.
    \end{enumerate}

  \item Let $C_g, C_s$ be the cost of installing the fiber-optic cable underground and undersea respectively in \$a. Suppose the cost per km of underground cable is $\$a$. The total cost $C$ is given by:
    \begin{align*}
      C_g &= (13-x)\cdot 1 = (13-x)\\
      C_s &= \sqrt{5^2+x^2}\cdot1.4 = 1.4\sqrt{25+x^2}\\
      C &= (13-x) + 1.4\sqrt{25+x^2}
    \end{align*}
    First find the rate of change of $C$ w.r.t $x$:
    \begin{align*}
      \frac{dC}{dx} &= -1 + 1.4(\frac{1}{2})(25+x^2)^{-\frac{1}{2}}(2x)\\
                    &= \frac{1.4x}{\sqrt{25+x^2}}-1
    \end{align*}
    At critical points, $\frac{dC}{dx} = 0$:
    \begin{align*}
      \frac{1.4x}{\sqrt{25+x^2}} - 1 &= 0\\
      \frac{x}{\sqrt{25+x^2}} &= \frac{1}{1.4}\\
      \frac{x}{\sqrt{25+x^2}} &= \frac{5}{7}\\
      \frac{x^2}{25+x^2} &= \frac{25}{49}\\
      49x^2 &= 625 + 25x^2\\
      24x^2 &= 625\\
      x^2 &= \frac{625}{24}\\
      x &= \pm \frac{25}{\sqrt{24}}\\
      x &= \frac{25}{\sqrt{24}}\tag*{\text{(Distance $x \geq 0$)}}
    \end{align*}
    By first derivative test, $\frac{dC}{dx}|_{x=\frac{25}{\sqrt{24}}-} < 0$ and $\frac{dC}{dx}|_{x=\frac{25}{\sqrt{24}}+} > 0$, implies that $C$ is a local minimum when $x = \frac{25}{\sqrt{24}}$.\\
    Since $C$ at $x = \frac{25}{\sqrt{24}}$ is the only minima for $x \in [0, 13]$, it is also the absolute minimum cost.\\
    $\therefore$ the distance between B and C if the total cost of installing the cable is to be minimized is $\frac{25}{\sqrt{24}} \approx 5.1$km.$\qed$
  \item \begin{enumerate}[(\alph*)]
      \item $\displaystyle \lim_{x\to\pi/2}\frac{1-\sin x}{1+\cos 2x}$
        \begin{align*}
          \lim_{x\to\pi/2}\frac{1-\sin x}{1+\cos 2x} &= \lim_{x\to\pi/2}\frac{-\cos x}{-2\sin 2x}\tag*{\text{(By L'Hopital's Rule)}}\\
                                                     &= \lim_{x\to\pi/2}\frac{-\sin x}{4\cos 2x}\tag*{\text{(By L'Hopital's Rule)}}\\
                                                     &= \frac{-1}{-4}\\
                                                     &= \frac{1}{4} \qed
        \end{align*}

      \item $\displaystyle \lim_{x\to 0}\frac{\ln(\cos ax)}{\ln(\cos bx)}$
        \begin{align*}
          \lim_{x\to 0}\frac{\ln(\cos ax)}{\ln(\cos bx)} &= \lim_{x\to 0}\frac{\frac{-a\sin ax}{\cos ax}}{\frac{-b\sin bx}{\cos bx}}\tag*{\text{(By L'Hopital's Rule)}}\\
                                                         &= \lim_{x\to 0}\frac{a\tan ax}{b\tan bx}\\
                                                         &= \lim_{x\to 0}\frac{a^2x}{b^2x}\tag*{\text{(Small angle approximation)}}\\
                                                         &= \lim_{x\to0}\frac{a^2}{b^2} \\
                                                         &= \frac{a^2}{b^2} \qed
       \end{align*}

     \item $\displaystyle \lim_{x\to 1}x^{\frac{1}{1-x}}$
       \begin{align*}
         \ln x^{\frac{1}{1-x}} &= \frac{\ln x}{1-x}\\
         \lim_{x\to 1}\ln x^{\frac{1}{1-x}} &= \lim_{x\to 1}\frac{\ln x}{1-x}\\
                                            &= \lim_{x\to 1}\frac{\frac{1}{x}}{-1}\tag*{\text{(By L'Hopital's Rule)}}\\
                                            &= -1\\
         \therefore \lim_{x\to 1}x^{\frac{1}{1-x}} &= e^{-1} \qed
       \end{align*}

     \item $\displaystyle \lim_{x\to0^+}x^{\sin x}$
       \begin{align*}
         \ln x^{\sin x} &= \sin x \ln x \\
         \lim_{x\to0^+}\ln x^{\sin x} &= \lim_{x\to0^+}\sin x\ln x\\
                                      &= \lim_{x\to0^+}x\ln x\tag*{\text{(By L'Hopital's Rule)}}\\
                                      &= \lim_{x\to0^+}\frac{\ln x}{\frac{1}{x}}\\
                                      &= \lim_{x\to0^+}\frac{\frac{1}{x}}{-\frac{1}{x^2}}\tag*{\text{(By L'Hopital's Rule)}}\\
                                      &= \lim_{x\to0^+}-x\\
                                      &= 0\\
         \therefore \lim_{x\to0^+}x^{\sin x} &= e^0\\
                                             &= 1\qed
       \end{align*}
    \end{enumerate}
\end{enumerate}
%%%%%%%%%%%%%%%%%%%%%%%%%%%%%%%%%%%%%%%%%%%%%%%%%%%%%%
%                       End                          %
%%%%%%%%%%%%%%%%%%%%%%%%%%%%%%%%%%%%%%%%%%%%%%%%%%%%%%

\end{document}

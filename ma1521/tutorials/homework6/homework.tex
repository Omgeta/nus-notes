\documentclass[12pt, a4paper]{article}

\usepackage[a4paper, margin=1in]{geometry}

\usepackage[utf8]{inputenc}
\usepackage[mathscr]{euscript}
\let\euscr\mathscr \let\mathscr\relax
\usepackage[scr]{rsfso}
\usepackage{amssymb,amsmath,amsthm,amsfonts}
\usepackage[shortlabels]{enumitem}
\usepackage{multicol,multirow}
\usepackage{lipsum}
\usepackage{balance}
\usepackage{calc}
\usepackage[colorlinks=true,citecolor=blue,linkcolor=blue]{hyperref}
\usepackage{import}
\usepackage{xifthen}
\usepackage{pdfpages}
\usepackage{transparent}
\usepackage{tabularx}

\newcommand{\incfig}[2][1.0]{
    \def\svgwidth{#1\columnwidth}
    \import{./figures/}{#2.pdf_tex}
}
\newcommand{\incimg}[2][1.0]{
  \includegraphics[width=#1\columnwidth]{./figures/#2}
}


\input{letterfonts}
\newcommand{\mytitle}{MA1521 Homework 6}
\newcommand{\myauthor}{github/omgeta}
\newcommand{\mydate}{AY 24/25 Sem 1}

\begin{document}
\raggedright
\footnotesize
\begin{center}
{\normalsize{\textbf{\mytitle}}} \\
{\footnotesize{\mydate\hspace{2pt}\textemdash\hspace{2pt}\myauthor}}
\end{center}
\setlist{topsep=-1em, itemsep=-1em, parsep=2em}

%%%%%%%%%%%%%%%%%%%%%%%%%%%%%%%%%%%%%%%%%%%%%%%%%%%%%%
%                      Begin                         %
%%%%%%%%%%%%%%%%%%%%%%%%%%%%%%%%%%%%%%%%%%%%%%%%%%%%%%
\begin{enumerate}[Q\arabic*.]
  \item 
    \begin{enumerate}[(\alph*)]
      \item $\displaystyle \sum^{\infty}_{n=1}\cos^2\frac{1}{n}$
      \begin{align*}
        \lim_{n\rightarrow\infty}\cos^2\frac{1}{n} = 1
      \end{align*}
      $\therefore$ By nth term test, $\sum^{\infty}_{n=1}\cos^2\frac{1}{n}$ diverges. $\qed$ 

      \item $\displaystyle \sum^{\infty}_{n=2}\frac{1}{n(\ln n)^{r+1}}$
        \begin{align*}
        \int^{\infty}_2 \frac{1}{x(\ln x)^{r+1}} dx &= \int^{\infty}_{\ln 2}\frac{1}{u^{r+1}} du\tag*{(Sub $u=\ln x \implies dx = x du$)}\\
                                                    &= -\frac{1}{r}[u^{-r}]^{\infty}_{\ln 2}\\
                                                    &= -\frac{1}{r}[0 - (\ln  2)^{-r}]\\
                                                    &= \frac{1}{r(\ln 2)^r}
      \end{align*}
      $\therefore$ By integral test, $\sum^{\infty}_{n=2}\frac{1}{n(\ln n)^{r+1}}$ is convergent. $\qed$

    \item $\displaystyle \sum^{\infty}_{n=1} \sin^{2n}(\frac{1}{\sqrt{n}})$
      \begin{align*}
        \sin^{2n}(\frac{1}{\sqrt{n}}) &\approx (\frac{1}{\sqrt{n}})^{2n}\tag*{(For large $n$)}\\
                                      &= \frac{1}{n^2}\\
      \end{align*}
      Since $\displaystyle 0 \leq \sin^{2n}\frac{1}{\sqrt{n}} \leq \frac{1}{n^2}$, and $\displaystyle \sum^{\infty}_{n=1}\frac{1}{n^2}$ is a convergent p-series, then by comparison test, $\displaystyle \sum^{\infty}_{n=1} \sin^{2n}\frac{1}{\sqrt{n}}$ is convergent. $\qed$

    \item $\displaystyle \sum^{\infty}_{n=1}(-1)^n\frac{c}{\sqrt{d+n^2}}$
      \begin{align*}
        \frac{d}{dn}(\frac{c}{\sqrt{d+n^2}}) &= -\frac{c \cdot n}{(d+n^2)^{3/2}}\\
                                             &< 0\text{, for all $n \geq 1$}\\
        \lim_{n\rightarrow\infty}\frac{c}{\sqrt{d+n^2}} &= \lim_{n\rightarrow\infty}\frac{c}{n}\\
                                                        &= 0
      \end{align*}
      $\therefore$ by alternating series test, $\displaystyle \sum^{\infty}_{n=1}(-1)^n\frac{c}{\sqrt{d+n^2}}$ is convergent. $\qed$

    \item $\displaystyle \sum^{\infty}_{n=1} \frac{3+\sin n}{n^3}$
      \begin{align*}
      -1 \leq \sin n \leq 1\\
      2 \leq 3 + \sin n \leq 4\\
      \frac{2}{n^3} \leq \frac{3+\sin n}{n^3} \leq \frac{4}{n^3}\\
      \end{align*}
      Since both $\displaystyle \sum^{\infty}_{n=1} \frac{2}{n^3}$ and $\displaystyle \sum^{\infty}_{n=1} \frac{4}{n^3}$ are convergent p-series, then by comparison test, $\displaystyle \sum^{\infty}_{n=1} \frac{3+\sin n}{n^3}$ is convergent. $\qed$ 
    \item $\displaystyle \sum^{\infty}_{n=1}\frac{2^{1+3n}(n+1)}{n^25^{1+n}}$
      \begin{align*}
        \lim_{n\rightarrow\infty}|\frac{\frac{2^{1+3n+3}(n+2)}{(n+1)^25^{2+n}}}{\frac{2^{1+3n}(n+1)}{n^25^{1+n}}}| &= \lim_{n\rightarrow\infty}|\frac{2^3\cdot n^2(n+2)}{5(n+1)^3}|\\
                                                                                                                   &= \lim_{n\rightarrow\infty}|\frac{2^3(n^3+\ldots)}{5(n^3+\ldots)}|\\
                                                                                                                   &= \frac{8}{5}\\
                                                                                                                   &> 1
      \end{align*}
      $\therefore$ by ratio test, $\displaystyle \sum^{\infty}_{n=1}\frac{2^{1+3n}(n+1)}{n^25^{1+n}}$ is divergent. $\qed$ 
    \end{enumerate}

  \pagebreak
  \item 
    \begin{enumerate}[(\alph*)]
      \item $\displaystyle \sum^{\infty}_{n=1}(-1)^n \frac{(2x+3)^n}{n}$\\
        For the power series to be convergent:
        \begin{align*}
          \lim_{n\rightarrow\infty}|\frac{(\frac{(-1)^{n+1}(2x+3)^{n+1}}{n+1})}{\frac{(-1)^n(2x+3)^n}{n}}| &< 1\\
          \lim_{n\rightarrow\infty}|-(2x+3)\cdot \frac{n}{n+1}| &< 1\\
          |-(2x+3)| &< 1\\
          |2x+3| &< 1\\
          -1 < 2x+3 &< 1\\
          -2 < x &< -1
        \end{align*}
        At $x=-1$:
        \begin{align*}
          \sum^{\infty}_{n=1}(-1)^n \frac{(2x+3)^n}{n} &= \sum^{\infty}_{n=1}\frac{1}{n}\\&\text{ which is the divergent harmonic series}
        \end{align*}
        At $x=-2$:
        \begin{align*}
          \sum^{\infty}_{n=1}(-1)^n \frac{(2x+3)^n}{n} &= \sum^{\infty}_{n=1}\frac{(-1)^n}{n}\\&\text{ which is the convergent alternating harmonic series}
        \end{align*}

        $\therefore$ radius of convergence is $\frac{1}{2}$ and interval of convergence is $(-2, -1] \qed$ 

      \item $\displaystyle \sum^{\infty}_{n=1} (nx)^{n/5}$\\
        By ratio test:
        \begin{align*}
          \lim_{n\rightarrow\infty}|\frac{((n+1)x)^{n+1/5}}{(nx)^{n/5}}| &= \lim_{n\rightarrow\infty}|\frac{(n+1)^{n+1/5}}{n^{n/5}} \cdot x^{1 /5}|\\
                                                                         &= \lim_{n\rightarrow\infty}|(\frac{n+1}{n})^{n/5}\cdot (n+1)^{1/5} \cdot x^{1/5}|\\
                                                                         &= \lim_{n\rightarrow\infty}|e^{1/5} \cdot (n+1)^{1/5} \cdot x^{1/5}|\\
                                                                         &= \infty
        \end{align*}
        $\therefore$ radius of convergence is $0 \qed$
    \end{enumerate}
    \pagebreak

  \item $\displaystyle \sum^{\infty}_{n=1} a_n(-1)^nx^{2n}$\\
    For the power series to be convergent:
    \begin{align*}
      \lim_{n\rightarrow\infty}|\frac{a_{n+1}(-1)^{n+1}x^{2n+2}}{a_n(-1)^nx^{2n}}| &< 1\\
      \frac{1}{5}|x|^2 &< 1\\
      |x|^2 &< 5\\
      |x| &< \sqrt{5}\\
    \end{align*}
    $\therefore$ radius of convergence is $\sqrt{5} \qed$

  \item 
    \begin{enumerate}[(\alph*)]
      \item $\displaystyle \frac{x}{1-x}$ at $x=0$
        \begin{align*}
          \frac{x}{1-x} &= x \cdot \frac{1}{1-x}\\
                        &= x \cdot \sum^{\infty}_{n=0}x^n\\
                        &= \sum^{\infty}_{n=0}x^{n+1} \qed
        \end{align*}

      \item $\displaystyle \frac{1}{x^2}$ at $x=1$
        \begin{align*}
          f(x) &= \frac{1}{x^2}\\
          f'(x) &= -\frac{2}{x^3}\\
          f''(x) &= \frac{6}{x^4}\\
          f'''(x) &= -\frac{24}{x^5}\\
          \frac{1}{x^2} &= 1 + (-\frac{2}{(1)^3})(\frac{1}{1!})(x-1) + (\frac{6}{1^4})(\frac{1}{2!})(x-1)^2 + \ldots\\
                        &= 1 -2(x-1) + 3(x-1)^2 -4(x-1)^3 +\ldots\\
                        &= \sum^{\infty}_{n=0} (-1)^n(n+1)(x-1)^n \qed
        \end{align*}

      \item $\displaystyle \frac{x}{1+x}$ at $x=-2$
        \begin{align*}
          f(x) &= \frac{x}{1+x}\\
          f'(x) &= \frac{1}{(1+x)^2}\\
          f''(x) &= \frac{-2}{(1+x)^3}\\
          \frac{x}{1+x} &= 2(\frac{1}{0!})(x+2)^0 + 1(\frac{1}{1!})(x+2)^1 + (2)(\frac{1}{2!})(x+2)^2 + \ldots\\
                        &= 2 + \sum^{\infty}_{n=1}(x+2)^n \qed
        \end{align*}
    \end{enumerate}
  \pagebreak
  \item 
    Given $\displaystyle e^x = \sum^{\infty}_{n=0} \frac{x^n}{n!}$:
    \begin{align*}
      xe^x &= x\sum^{\infty}_{n=0} \frac{x^n}{n!}\\
           &= \sum^{\infty}_{n=0} \frac{x^{n+1}}{n!}\\
      \int^1_0 xe^x dx &= [\sum^{\infty}_{n=0} \frac{x^{n+2}}{(n+2)n!}]^1_0\\
      [xe^x - e^x]^1_0 &= \sum^{\infty}_{n=0} \frac{1}{(n+2)n!}\\
      1 &= S\\
      \therefore S &= 1 \qed
    \end{align*}
\end{enumerate}
%%%%%%%%%%%%%%%%%%%%%%%%%%%%%%%%%%%%%%%%%%%%%%%%%%%%%%
%                       End                          %
%%%%%%%%%%%%%%%%%%%%%%%%%%%%%%%%%%%%%%%%%%%%%%%%%%%%%%

\end{document}

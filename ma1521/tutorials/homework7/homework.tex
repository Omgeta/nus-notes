\documentclass[12pt, a4paper]{article}

\usepackage[a4paper, margin=1in]{geometry}

\usepackage[utf8]{inputenc}
\usepackage[mathscr]{euscript}
\let\euscr\mathscr \let\mathscr\relax
\usepackage[scr]{rsfso}
\usepackage{amssymb,amsmath,amsthm,amsfonts}
\usepackage[shortlabels]{enumitem}
\usepackage{multicol,multirow}
\usepackage{lipsum}
\usepackage{balance}
\usepackage{calc}
\usepackage[colorlinks=true,citecolor=blue,linkcolor=blue]{hyperref}
\usepackage{import}
\usepackage{xifthen}
\usepackage{pdfpages}
\usepackage{transparent}
\usepackage{tabularx}

\newcommand{\incfig}[2][1.0]{
    \def\svgwidth{#1\columnwidth}
    \import{./figures/}{#2.pdf_tex}
}
\newcommand{\incimg}[2][1.0]{
  \includegraphics[width=#1\columnwidth]{./figures/#2}
}


\input{letterfonts}
\newcommand{\mytitle}{MA1521 Homework 7}
\newcommand{\myauthor}{github/omgeta}
\newcommand{\mydate}{AY 24/25 Sem 1}

\begin{document}
\raggedright
\footnotesize
\begin{center}
{\normalsize{\textbf{\mytitle}}} \\
{\footnotesize{\mydate\hspace{2pt}\textemdash\hspace{2pt}\myauthor}}
\end{center}
\setlist{topsep=-1em, itemsep=-1em, parsep=2em}

%%%%%%%%%%%%%%%%%%%%%%%%%%%%%%%%%%%%%%%%%%%%%%%%%%%%%%
%                      Begin                         %
%%%%%%%%%%%%%%%%%%%%%%%%%%%%%%%%%%%%%%%%%%%%%%%%%%%%%%
\begin{enumerate}[Q\arabic*.]
  \item 
    \begin{enumerate}[(\alph*)]
      \item $\vec{w} = 3 \hat{i} - \hat{j} + 4\hat{k}\qed$
      \item Suppose $Q$ is a point on the plane:
        \begin{align*}
          3(0) -1(a) +4(0) &= 0\\
          a &= -1 \qed
        \end{align*}
      \item Find $\vec{QP}$:
        \begin{align*}
          \vec{QP} &= \vec{OP} - \vec{OQ}\\
                   &= \langle 2,1,-3\rangle - \langle 0,-1,0\rangle\\
                   &= \langle 2,2,-3\rangle
        \end{align*}
        Find the projection of $\vec{QP}$ onto the plane:
        \begin{align*}
          proj_{\mathcal{P}}\vec{QP} &= (\frac{\vec{QP}\cdot \vec{w}}{||\vec{w}||^2})\vec{w}\\
                                     &= (\frac{2(3) + 2(-1) -3(4)}{3^2 + 1^2 + 4^2})\vec{w}\\
                                     &= -\frac{4}{13}(3 \hat{i} - \hat{j} + 4 \hat{k})\qed
        \end{align*}
      \item Since $Q$ lies on the plane, the distance is given by:
        \begin{align*}
          ||proj_{\mathcal{P}}\vec{QP}|| &= ||-\frac{4}{13}\langle 3, -1, 4\rangle\\
                                         &= \frac{4}{13}\sqrt{3^2 + 1^ + 4^2}\\
                                         &= \frac{4\sqrt{26}}{13}\qed
        \end{align*}
    \end{enumerate}
  \pagebreak

  \item 
    \begin{enumerate}[(\alph*)]
      \item Equation of the plane is given by:
        \begin{align*}
          -2(x-6) +5(y-3) + (z-2) &= 0\\
          -2x + 12 + 5y - 15 + z - 2 &= 0\\
          -2x + 5y + z &= 5 \qed
        \end{align*}
      \item Equation of the plane is given by:
        \begin{align*}
          2(x-3) + 4(y-0) + 8(z-8) &= 0\\
          2x - 6 + 4y + 8z - 64 &= 0\\
          2x + 4y + 8z &= 70\qed
        \end{align*}
      \item Line of intersection between planes is:
        \begin{align*}
          \vec{d_1} &= \langle 1,1,-1\rangle \times \langle 2,-1,3\rangle\\
                    &= \begin{vmatrix}
                      \hat{i} & \hat{j} & \hat{k}\\
                      1 & 1 & -1\\
                      2 & -1 & 3
                    \end{vmatrix}\\
                    &= 2\hat{i} - 5\hat{j} -3\hat{k}
        \end{align*}
        Find a point on the line of intersection, let $z=0$:
        \begin{align*}
          x+y&=2\\
          2x-y&=1\\
          \implies x=y&=1 
        \end{align*}
        Then we have a second vector on the plane:
        \begin{align*}
          \vec{d_2} &= \langle -1,2,1\rangle - \langle 1,1,0\rangle\\
                    &= \langle -2,1,1\rangle
        \end{align*}
        Find the normal of the plane:
        \begin{align*}
          \vec{n} &= \vec{d_1} \times \vec{d_2}\\
                  &= \begin{vmatrix}
                    \hat{i} & \hat{j} & \hat{k}\\
                    2 & -5 & -3\\
                    -2 & 1 & 1
                  \end{vmatrix}\\
                  &= \langle 2,-4,8 \rangle
        \end{align*}
        Equation of the plane is given by:
        \begin{align*}
          2(x+1)-4(y-2)+8(z-1) &= 0\\
          2x-4y+8z&=-2\\
          x-2y+4z&=-1 \qed
        \end{align*}
    \end{enumerate}
  
  \item Notice the equivalent form of $\Pi_2$:
    \begin{align*}
      \Pi_2: 2x-3y+6z=-2
    \end{align*}
    Then, the distance between the two planes is given by:
    \begin{align*}
      \frac{|16-(-2)|}{\sqrt{2^2 + 3^2 + 6^2}}\\
      = \frac{18}{7} \qed
    \end{align*}
    \pagebreak

  \item
      Find two vectors on the plane:
      \begin{align*}
        \vec{AB} &= \langle 3,0,1\rangle - \langle 3,3,0\rangle\\
                 &= \langle 0,-3,1\rangle\\
        \vec{AC} &= \langle 0,2,1\rangle - \langle 3,3,0\rangle\\
                 &= \langle -3,-1,1\rangle
      \end{align*}
      Find the normal to the plane:
      \begin{align*}
        \vec{n} = \vec{AB} \times \vec{AC}
                &= \begin{vmatrix}
                  \hat{i} & \hat{j} & \hat{k}\\
                  0 & -3 & 1\\
                  -3 & -1&1\\
                \end{vmatrix}\\
                &= \langle -2, -3, -9\rangle
      \end{align*}
      Plane $\Pi$ is given by:
      \begin{align*}
        -2(x-3) -3(y-3) -9(z-0) &= 0\\
        2x +3y + 9z &= 15 \qed
      \end{align*}

      \item Shortest distance from origin is given by:
        \begin{align*}
          \frac{15}{\sqrt{2^2 + 3^2 + 9^2}}\\
          = \frac{15}{\sqrt{94}} \qed
        \end{align*}

      \item Line segment of $\vec{OD}$ is given by:
        \begin{align*}
          \vec{OD}: t\langle 4,2,1\rangle,\quad t\in \RR 
        \end{align*}
        At the point of intersection:
        \begin{align*}
          2(4t) + 3(2t) + 9(t) &= 15\\
          8t + 6t + 9t &= 15\\
          t &= \frac{15}{23}
        \end{align*}
        Therefore, the point of intersection is:
        \begin{gather*}
          \frac{15}{23}\langle 4,2,1\rangle \\
          = (\frac{60}{23},\frac{30}{23},\frac{15}{23}) \qed
        \end{gather*}

  \item When the curves intersect:
    \begin{align*}
      r_1(t_1) &= r_2(t_2)\\
      t_1 \hat{i} + t_1^2 \hat{j} + t_1^3 \hat{k} &= (1+2t_2)\hat{i} + (1+6t_2)\hat{j} + (1+14t_2)\hat{k}\\
    \end{align*}
    which gives the system of equations:
    \begin{align*}
      t_1 = 1 + 2t_2,\quad t_1^2 = 1+6t_2,\quad t_1^3 = 1+14t_2
    \end{align*}
    Solving these equations simultaneously gives the times when the curves are at the same point:
    \begin{align*}
      t_1 = 1 &\implies t_2 = 0\\
      t_1 = 2 &\implies t_2 = \frac{1}{2}
    \end{align*}
    \begin{enumerate}[(\alph*)]
      \item No; the intersections there are no intersections where $t_1 = t_2 \qed$
      \item Yes; there are two points where the paths intersect at $t_1=1,t_2=0$ and $t_1=2,t_2=\frac{1}{2} \qed$
    \end{enumerate}
\end{enumerate}
%%%%%%%%%%%%%%%%%%%%%%%%%%%%%%%%%%%%%%%%%%%%%%%%%%%%%%
%                       End                          %
%%%%%%%%%%%%%%%%%%%%%%%%%%%%%%%%%%%%%%%%%%%%%%%%%%%%%%

\end{document}

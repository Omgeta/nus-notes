\documentclass[12pt, a4paper]{article}

\usepackage[a4paper, margin=1in]{geometry}

\usepackage{fancyhdr}
\pagestyle{fancy}
\fancyhf{}
\fancyhead[R]{\thepage}
\renewcommand{\headrulewidth}{0pt}

\usepackage[utf8]{inputenc}
\usepackage[mathscr]{euscript}
\let\euscr\mathscr \let\mathscr\relax
\usepackage[scr]{rsfso}
\usepackage{amssymb,amsmath,amsthm,amsfonts}
\usepackage[shortlabels]{enumitem}
\usepackage{multicol,multirow}
\usepackage{lipsum}
\usepackage{balance}
\usepackage{calc}
\usepackage[colorlinks=true,citecolor=blue,linkcolor=blue]{hyperref}
\usepackage{import}
\usepackage{xifthen}
\usepackage{pdfpages}
\usepackage{transparent}
\usepackage{listings}

\newcommand{\incfig}[2][1.0]{
    \def\svgwidth{#1\columnwidth}
    \import{./figures/}{#2.pdf_tex}
}

\newlist{enumproof}{enumerate}{4}
\setlist[enumproof,1]{label=\arabic*., parsep=1em}
\setlist[enumproof,2]{label=\arabic{enumproofi}.\arabic*., parsep=1em}
\setlist[enumproof,3]{label=\arabic{enumproofi}.\arabic{enumproofii}.\arabic*., parsep=1em}
\setlist[enumproof,4]{label=\arabic{enumproofi}.\arabic{enumproofii}.\arabic{enumproofiii}.\arabic*., parsep=1em}

\renewcommand{\qedsymbol}{\ensuremath{\blacksquare}}

\lstdefinestyle{mystyle}{
  language=C, % Set the language to C
  commentstyle=\color{codegray}, % Color for comments
  keywordstyle=\color{orange}, % Color for basic keywords
  stringstyle=\color{mauve}, % Color for strings
  basicstyle={\ttfamily\footnotesize}, % Basic font style
  breakatwhitespace=false,         
  breaklines=true,                 
  captionpos=b,                    
  keepspaces=true,                 
  numbers=none,                    
  tabsize=2,
  morekeywords=[2]{\#include, \#define, \#ifdef, \#ifndef, \#endif, \#pragma, \#else, \#elif}, % Preprocessor directives
  keywordstyle=[2]\color{codegreen}, % Style for preprocessor directives
  morekeywords=[3]{int, char, float, double, void, struct, union, enum, const, volatile, static, extern, register, inline, restrict, _Bool, _Complex, _Imaginary, size_t, ssize_t, FILE}, % C standard types and common identifiers
  keywordstyle=[3]\color{identblue}, % Style for types and common identifiers
  morekeywords=[4]{printf, scanf, fopen, fclose, malloc, free, calloc, realloc, perror, strtok, strncpy, strcpy, strcmp, strlen}, % Standard library functions
  keywordstyle=[4]\color{cyan}, % Style for library functions
}

% Things Lie
\newcommand{\kb}{\mathfrak b}
\newcommand{\kg}{\mathfrak g}
\newcommand{\kh}{\mathfrak h}
\newcommand{\kn}{\mathfrak n}
\newcommand{\ku}{\mathfrak u}
\newcommand{\kz}{\mathfrak z}
\DeclareMathOperator{\Ext}{Ext} % Ext functor
\DeclareMathOperator{\Tor}{Tor} % Tor functor
\newcommand{\gl}{\opname{\mathfrak{gl}}} % frak gl group
\renewcommand{\sl}{\opname{\mathfrak{sl}}} % frak sl group chktex 6

% More script letters etc.
\newcommand{\SA}{\mathcal A}
\newcommand{\SB}{\mathcal B}
\newcommand{\SC}{\mathcal C}
\newcommand{\SF}{\mathcal F}
\newcommand{\SG}{\mathcal G}
\newcommand{\SH}{\mathcal H}
\newcommand{\OO}{\mathcal O}

\newcommand{\SCA}{\mathscr A}
\newcommand{\SCB}{\mathscr B}
\newcommand{\SCC}{\mathscr C}
\newcommand{\SCD}{\mathscr D}
\newcommand{\SCE}{\mathscr E}
\newcommand{\SCF}{\mathscr F}
\newcommand{\SCG}{\mathscr G}
\newcommand{\SCH}{\mathscr H}

% Mathfrak primes
\newcommand{\km}{\mathfrak m}
\newcommand{\kp}{\mathfrak p}
\newcommand{\kq}{\mathfrak q}

% number sets
\newcommand{\RR}[1][]{\ensuremath{\ifstrempty{#1}{\mathbb{R}}{\mathbb{R}^{#1}}}}
\newcommand{\NN}[1][]{\ensuremath{\ifstrempty{#1}{\mathbb{N}}{\mathbb{N}^{#1}}}}
\newcommand{\ZZ}[1][]{\ensuremath{\ifstrempty{#1}{\mathbb{Z}}{\mathbb{Z}^{#1}}}}
\newcommand{\QQ}[1][]{\ensuremath{\ifstrempty{#1}{\mathbb{Q}}{\mathbb{Q}^{#1}}}}
\newcommand{\CC}[1][]{\ensuremath{\ifstrempty{#1}{\mathbb{C}}{\mathbb{C}^{#1}}}}
\newcommand{\PP}[1][]{\ensuremath{\ifstrempty{#1}{\mathbb{P}}{\mathbb{P}^{#1}}}}
\newcommand{\HH}[1][]{\ensuremath{\ifstrempty{#1}{\mathbb{H}}{\mathbb{H}^{#1}}}}
\newcommand{\FF}[1][]{\ensuremath{\ifstrempty{#1}{\mathbb{F}}{\mathbb{F}^{#1}}}}
% expected value
\newcommand{\EE}{\ensuremath{\mathbb{E}}}
\newcommand{\charin}{\text{ char }}
\DeclareMathOperator{\sign}{sign}
\DeclareMathOperator{\Aut}{Aut}
\DeclareMathOperator{\Inn}{Inn}
\DeclareMathOperator{\Syl}{Syl}
\DeclareMathOperator{\Gal}{Gal}
\DeclareMathOperator{\GL}{GL} % General linear group
\DeclareMathOperator{\SL}{SL} % Special linear group

%---------------------------------------
% BlackBoard Math Fonts :-
%---------------------------------------

%Captital Letters
\newcommand{\bbA}{\mathbb{A}}	\newcommand{\bbB}{\mathbb{B}}
\newcommand{\bbC}{\mathbb{C}}	\newcommand{\bbD}{\mathbb{D}}
\newcommand{\bbE}{\mathbb{E}}	\newcommand{\bbF}{\mathbb{F}}
\newcommand{\bbG}{\mathbb{G}}	\newcommand{\bbH}{\mathbb{H}}
\newcommand{\bbI}{\mathbb{I}}	\newcommand{\bbJ}{\mathbb{J}}
\newcommand{\bbK}{\mathbb{K}}	\newcommand{\bbL}{\mathbb{L}}
\newcommand{\bbM}{\mathbb{M}}	\newcommand{\bbN}{\mathbb{N}}
\newcommand{\bbO}{\mathbb{O}}	\newcommand{\bbP}{\mathbb{P}}
\newcommand{\bbQ}{\mathbb{Q}}	\newcommand{\bbR}{\mathbb{R}}
\newcommand{\bbS}{\mathbb{S}}	\newcommand{\bbT}{\mathbb{T}}
\newcommand{\bbU}{\mathbb{U}}	\newcommand{\bbV}{\mathbb{V}}
\newcommand{\bbW}{\mathbb{W}}	\newcommand{\bbX}{\mathbb{X}}
\newcommand{\bbY}{\mathbb{Y}}	\newcommand{\bbZ}{\mathbb{Z}}

%---------------------------------------
% MathCal Fonts :-
%---------------------------------------

%Captital Letters
\newcommand{\mcA}{\mathcal{A}}	\newcommand{\mcB}{\mathcal{B}}
\newcommand{\mcC}{\mathcal{C}}	\newcommand{\mcD}{\mathcal{D}}
\newcommand{\mcE}{\mathcal{E}}	\newcommand{\mcF}{\mathcal{F}}
\newcommand{\mcG}{\mathcal{G}}	\newcommand{\mcH}{\mathcal{H}}
\newcommand{\mcI}{\mathcal{I}}	\newcommand{\mcJ}{\mathcal{J}}
\newcommand{\mcK}{\mathcal{K}}	\newcommand{\mcL}{\mathcal{L}}
\newcommand{\mcM}{\mathcal{M}}	\newcommand{\mcN}{\mathcal{N}}
\newcommand{\mcO}{\mathcal{O}}	\newcommand{\mcP}{\mathcal{P}}
\newcommand{\mcQ}{\mathcal{Q}}	\newcommand{\mcR}{\mathcal{R}}
\newcommand{\mcS}{\mathcal{S}}	\newcommand{\mcT}{\mathcal{T}}
\newcommand{\mcU}{\mathcal{U}}	\newcommand{\mcV}{\mathcal{V}}
\newcommand{\mcW}{\mathcal{W}}	\newcommand{\mcX}{\mathcal{X}}
\newcommand{\mcY}{\mathcal{Y}}	\newcommand{\mcZ}{\mathcal{Z}}

%---------------------------------------
% Bold Math Fonts :-
%---------------------------------------

%Captital Letters
\newcommand{\bmA}{\boldsymbol{A}}	\newcommand{\bmB}{\boldsymbol{B}}
\newcommand{\bmC}{\boldsymbol{C}}	\newcommand{\bmD}{\boldsymbol{D}}
\newcommand{\bmE}{\boldsymbol{E}}	\newcommand{\bmF}{\boldsymbol{F}}
\newcommand{\bmG}{\boldsymbol{G}}	\newcommand{\bmH}{\boldsymbol{H}}
\newcommand{\bmI}{\boldsymbol{I}}	\newcommand{\bmJ}{\boldsymbol{J}}
\newcommand{\bmK}{\boldsymbol{K}}	\newcommand{\bmL}{\boldsymbol{L}}
\newcommand{\bmM}{\boldsymbol{M}}	\newcommand{\bmN}{\boldsymbol{N}}
\newcommand{\bmO}{\boldsymbol{O}}	\newcommand{\bmP}{\boldsymbol{P}}
\newcommand{\bmQ}{\boldsymbol{Q}}	\newcommand{\bmR}{\boldsymbol{R}}
\newcommand{\bmS}{\boldsymbol{S}}	\newcommand{\bmT}{\boldsymbol{T}}
\newcommand{\bmU}{\boldsymbol{U}}	\newcommand{\bmV}{\boldsymbol{V}}
\newcommand{\bmW}{\boldsymbol{W}}	\newcommand{\bmX}{\boldsymbol{X}}
\newcommand{\bmY}{\boldsymbol{Y}}	\newcommand{\bmZ}{\boldsymbol{Z}}
%Small Letters
\newcommand{\bma}{\boldsymbol{a}}	\newcommand{\bmb}{\boldsymbol{b}}
\newcommand{\bmc}{\boldsymbol{c}}	\newcommand{\bmd}{\boldsymbol{d}}
\newcommand{\bme}{\boldsymbol{e}}	\newcommand{\bmf}{\boldsymbol{f}}
\newcommand{\bmg}{\boldsymbol{g}}	\newcommand{\bmh}{\boldsymbol{h}}
\newcommand{\bmi}{\boldsymbol{i}}	\newcommand{\bmj}{\boldsymbol{j}}
\newcommand{\bmk}{\boldsymbol{k}}	\newcommand{\bml}{\boldsymbol{l}}
\newcommand{\bmm}{\boldsymbol{m}}	\newcommand{\bmn}{\boldsymbol{n}}
\newcommand{\bmo}{\boldsymbol{o}}	\newcommand{\bmp}{\boldsymbol{p}}
\newcommand{\bmq}{\boldsymbol{q}}	\newcommand{\bmr}{\boldsymbol{r}}
\newcommand{\bms}{\boldsymbol{s}}	\newcommand{\bmt}{\boldsymbol{t}}
\newcommand{\bmu}{\boldsymbol{u}}	\newcommand{\bmv}{\boldsymbol{v}}
\newcommand{\bmw}{\boldsymbol{w}}	\newcommand{\bmx}{\boldsymbol{x}}
\newcommand{\bmy}{\boldsymbol{y}}	\newcommand{\bmz}{\boldsymbol{z}}

%---------------------------------------
% Scr Math Fonts :-
%---------------------------------------

\newcommand{\sA}{{\mathscr{A}}}   \newcommand{\sB}{{\mathscr{B}}}
\newcommand{\sC}{{\mathscr{C}}}   \newcommand{\sD}{{\mathscr{D}}}
\newcommand{\sE}{{\mathscr{E}}}   \newcommand{\sF}{{\mathscr{F}}}
\newcommand{\sG}{{\mathscr{G}}}   \newcommand{\sH}{{\mathscr{H}}}
\newcommand{\sI}{{\mathscr{I}}}   \newcommand{\sJ}{{\mathscr{J}}}
\newcommand{\sK}{{\mathscr{K}}}   \newcommand{\sL}{{\mathscr{L}}}
\newcommand{\sM}{{\mathscr{M}}}   \newcommand{\sN}{{\mathscr{N}}}
\newcommand{\sO}{{\mathscr{O}}}   \newcommand{\sP}{{\mathscr{P}}}
\newcommand{\sQ}{{\mathscr{Q}}}   \newcommand{\sR}{{\mathscr{R}}}
\newcommand{\sS}{{\mathscr{S}}}   \newcommand{\sT}{{\mathscr{T}}}
\newcommand{\sU}{{\mathscr{U}}}   \newcommand{\sV}{{\mathscr{V}}}
\newcommand{\sW}{{\mathscr{W}}}   \newcommand{\sX}{{\mathscr{X}}}
\newcommand{\sY}{{\mathscr{Y}}}   \newcommand{\sZ}{{\mathscr{Z}}}


%---------------------------------------
% Math Fraktur Font
%---------------------------------------

%Captital Letters
\newcommand{\mfA}{\mathfrak{A}}	\newcommand{\mfB}{\mathfrak{B}}
\newcommand{\mfC}{\mathfrak{C}}	\newcommand{\mfD}{\mathfrak{D}}
\newcommand{\mfE}{\mathfrak{E}}	\newcommand{\mfF}{\mathfrak{F}}
\newcommand{\mfG}{\mathfrak{G}}	\newcommand{\mfH}{\mathfrak{H}}
\newcommand{\mfI}{\mathfrak{I}}	\newcommand{\mfJ}{\mathfrak{J}}
\newcommand{\mfK}{\mathfrak{K}}	\newcommand{\mfL}{\mathfrak{L}}
\newcommand{\mfM}{\mathfrak{M}}	\newcommand{\mfN}{\mathfrak{N}}
\newcommand{\mfO}{\mathfrak{O}}	\newcommand{\mfP}{\mathfrak{P}}
\newcommand{\mfQ}{\mathfrak{Q}}	\newcommand{\mfR}{\mathfrak{R}}
\newcommand{\mfS}{\mathfrak{S}}	\newcommand{\mfT}{\mathfrak{T}}
\newcommand{\mfU}{\mathfrak{U}}	\newcommand{\mfV}{\mathfrak{V}}
\newcommand{\mfW}{\mathfrak{W}}	\newcommand{\mfX}{\mathfrak{X}}
\newcommand{\mfY}{\mathfrak{Y}}	\newcommand{\mfZ}{\mathfrak{Z}}
%Small Letters
\newcommand{\mfa}{\mathfrak{a}}	\newcommand{\mfb}{\mathfrak{b}}
\newcommand{\mfc}{\mathfrak{c}}	\newcommand{\mfd}{\mathfrak{d}}
\newcommand{\mfe}{\mathfrak{e}}	\newcommand{\mff}{\mathfrak{f}}
\newcommand{\mfg}{\mathfrak{g}}	\newcommand{\mfh}{\mathfrak{h}}
\newcommand{\mfi}{\mathfrak{i}}	\newcommand{\mfj}{\mathfrak{j}}
\newcommand{\mfk}{\mathfrak{k}}	\newcommand{\mfl}{\mathfrak{l}}
\newcommand{\mfm}{\mathfrak{m}}	\newcommand{\mfn}{\mathfrak{n}}
\newcommand{\mfo}{\mathfrak{o}}	\newcommand{\mfp}{\mathfrak{p}}
\newcommand{\mfq}{\mathfrak{q}}	\newcommand{\mfr}{\mathfrak{r}}
\newcommand{\mfs}{\mathfrak{s}}	\newcommand{\mft}{\mathfrak{t}}
\newcommand{\mfu}{\mathfrak{u}}	\newcommand{\mfv}{\mathfrak{v}}
\newcommand{\mfw}{\mathfrak{w}}	\newcommand{\mfx}{\mathfrak{x}}
\newcommand{\mfy}{\mathfrak{y}}	\newcommand{\mfz}{\mathfrak{z}}


\newcommand{\mytitle}{MA1522 Homework 1}
\newcommand{\myauthor}{github/omgeta}
\newcommand{\mydate}{AY 24/25 Sem 1}

\begin{document}
\raggedright
\footnotesize
\begin{center}
{\normalsize{\textbf{\mytitle}}} \\
{\footnotesize{\mydate\hspace{2pt}\textemdash\hspace{2pt}\myauthor}}
\end{center}
\setlist{topsep=-1em, itemsep=-1em, parsep=2em}

%%%%%%%%%%%%%%%%%%%%%%%%%%%%%%%%%%%%%%%%%%%%%%%%%%%%%%
%                      Begin                         %
%%%%%%%%%%%%%%%%%%%%%%%%%%%%%%%%%%%%%%%%%%%%%%%%%%%%%%
\begin{enumerate}[Q\arabic*.]
  \item 
    \begin{enumerate}[(\alph*)]
      \item \begin{align*}
          2x + 3y + 4z &= 400\qed\tag*{(i)}\\
          1x + 2y + 1z &= 200\qed\tag*{(ii)}\\
          2y + 4z &= 160\qed\tag*{(iii)}\\
        \end{align*}

      \item Form and reduce the corresponding augmented matrix for the system of equations:
        \begin{align*}
          \begin{bmatrix}
            2&3&4&400\\
            1&2&1&200\\
            0&2&4&160
          \end{bmatrix}\xrightarrow{R_1\leftrightarrow R_2}
          &\begin{bmatrix}
            1&2&1&200\\
            2&3&4&400\\
            0&2&4&160
          \end{bmatrix}\\\xrightarrow{R_2-2R_1}
          &\begin{bmatrix}
            1&2&1&200\\
            0&-1&2&0\\
            0&2&4&160
          \end{bmatrix}\\\xrightarrow{R_1+2R_2}
          &\begin{bmatrix}
            1&0&5&200\\
            0&-1&2&0\\
            0&2&4&160
          \end{bmatrix}\\\xrightarrow{R_3+2R_2}
          &\begin{bmatrix}
            1&0&5&200\\
            0&-1&2&0\\
            0&0&8&160
          \end{bmatrix}\\\xrightarrow{\frac{1}{8}R_3}
          &\begin{bmatrix}
            1&0&5&200\\
            0&-1&2&0\\
            0&0&1&20
          \end{bmatrix}\\\xrightarrow{R_2-2R_3}
          &\begin{bmatrix}
            1&0&5&200\\
            0&-1&0&-40\\
            0&0&1&20
          \end{bmatrix}\\\xrightarrow{-R_2}
          &\begin{bmatrix}
            1&0&5&200\\
            0&1&0&40\\
            0&0&1&20
          \end{bmatrix}\\\xrightarrow{R_1-5R_3}
          &\begin{bmatrix}
            1&0&0&100\\
            0&1&0&40\\
            0&0&1&20
          \end{bmatrix}
        \end{align*}
      Hence, we find $x = 100, y = 40, z = 20$. Therefore, pumps of type X pump $100$ litres/sec, pumps of type Y pump $40$ litres/sec, and pumps of type Z pump $20$ litres/sec. $\qed$
    \end{enumerate}
    \pagebreak

  \item Reduce the corresponding augmented matrix:
    \begin{align*}
      \begin{bmatrix}
        a & 2 & a & (a+b) & (a-b)\\
        a & 2 & a & a & (a-b)\\
        3 & 3 & -b & 3 & -b\\
        (a+1) & 3 & (a+1) & (a+1) & (a-b+1)\\
      \end{bmatrix}
      &\xrightarrow{R_1-R_2}
      \begin{bmatrix}
        0 & 0 & 0 & b & 0\\
        a & 2 & a & a & (a-b)\\
        3 & 3 & -b & 3 & -b\\
        (a+1) & 3 & (a+1) & (a+1) & (a-b+1)\\
      \end{bmatrix}\\
      \xrightarrow{R_4-R_2}
      \begin{bmatrix}
        0 & 0 & 0 & b & 0\\
        a & 2 & a & a & (a-b)\\
        3 & 3 & -b & 3 & -b\\
        1 & 1 & 1 & 1 & 1\\
      \end{bmatrix}
      &\xrightarrow{R_3-3R_4}
      \begin{bmatrix}
        0 & 0 & 0 & b & 0\\
        a & 2 & a & a & (a-b)\\
        0 & 0 & (-b-3) & 0 & (-b-3)\\
        1 & 1 & 1 & 1 & 1\\
      \end{bmatrix}\\
      \xrightarrow{R_2-aR_4}
      \begin{bmatrix}
        0 & 0 & 0 & b & 0\\
        0 & (2-a) & 0 & 0 & -b\\
        0 & 0 & (-b-3) & 0 & (-b-3)\\
        1 & 1 & 1 & 1 & 1\\
      \end{bmatrix}
      &\xrightarrow{R_1\leftrightarrow R_4}
      \begin{bmatrix}
        1 & 1 & 1 & 1 & 1\\
        0 & (2-a) & 0 & 0 & -b\\
        0 & 0 & (-b-3) & 0 & (-b-3)\\
        0 & 0 & 0 & b & 0\\
      \end{bmatrix}\\
    \end{align*}

    \begin{enumerate}[(\alph*)]
      \item No solution: $a = 2 \land b \neq 0$. (Row 2 will have inconsistent equation $0 \neq 0$)$\qed$
      \item Unique solution: $a \neq 2 \land b \neq -3 \land b \neq 0$. (RREF has pivot in every row)$\qed$\\
        \begin{align*}
          \xrightarrow{\frac{1}{b}R_4}
          \begin{bmatrix}
            1 & 1 & 1 & 1 & 1\\
            0 & (2-a) & 0 & 0 & -b\\
            0 & 0 & (-b-3) & 0 & (-b-3)\\
            0 & 0 & 0 & 1 & 0\\
          \end{bmatrix}
          &\xrightarrow{R_1-R_4}
          \begin{bmatrix}
            1 & 1 & 1 & 0 & 1\\
            0 & (2-a) & 0 & 0 & -b\\
            0 & 0 & (-b-3) & 0 & (-b-3)\\
            0 & 0 & 0 & 1 & 0\\
          \end{bmatrix}\\
          \xrightarrow{\frac{1}{-b-3}R_3}
          \begin{bmatrix}
            1 & 1 & 1 & 0 & 1\\
            0 & (2-a) & 0 & 0 & -b\\
            0 & 0 & 1 & 0 & 1\\
            0 & 0 & 0 & 1 & 0\\
          \end{bmatrix}
          &\xrightarrow{R_1-3R_3}
          \begin{bmatrix}
            1 & 1 & 0 & 0 & 0\\
            0 & (2-a) & 0 & 0 & -b\\
            0 & 0 & 1 & 0 & 1\\
            0 & 0 & 0 & 1 & 0\\
          \end{bmatrix}\\
          \xrightarrow{\frac{1}{2-a}R_2}
          \begin{bmatrix}
            1 & 1 & 0 & 0 & 0\\
            0 & 1 & 0 & 0 & -\frac{b}{2-a}\\
            0 & 0 & 1 & 0 & 1\\
            0 & 0 & 0 & 1 & 0\\
          \end{bmatrix}
          &\xrightarrow{R_1-R_2}
          \begin{bmatrix}
            1 & 0 & 0 & 0 & \frac{b}{2-a}\\
            0 & 1 & 0 & 0 & -\frac{b}{2-a}\\
            0 & 0 & 1 & 0 & 1\\
            0 & 0 & 0 & 1 & 0\\
          \end{bmatrix}
        \end{align*}
        Hence, the unique solution will be $x_1 = -\frac{b}{a-2}, x_2 = \frac{b}{a-2}, x_3 = 1, x_4 = 0 \qed$
      \item Infinite solutions w/ 1 parameter: $a \neq 2 \land (b=-3 \lor b = 0)$. (There will be exactly one zero row $\implies$ there will be exactly 1 free variable)$\qed$  

        Suppose, there are infinitely many solutions and $x_3 = 1, x_4 = 0$:\\
        \textbf{Case 1:} $a\neq_2, b=-3$:
        \begin{align*}
          \xrightarrow{\frac{1}{2-a}R_2}
          \begin{bmatrix}
            1&1&1&1&1\\
            0&1&0&0&\frac{3}{2-a}\\
            0&0&0&0&0\\
            0&0&0&-3&0
          \end{bmatrix}
          &\xrightarrow{R_1-R_2}
          \begin{bmatrix}
            1&0&1&1&1-\frac{3}{2-a}\\
            0&1&0&0&\frac{3}{2-a}\\
            0&0&0&0&0\\
            0&0&0&-3&0
          \end{bmatrix}\\
          \xrightarrow{-\frac{1}{b}R_4}
          \begin{bmatrix}
            1&0&1&1&1-\frac{3}{2-a}\\
            0&1&0&0&\frac{3}{2-a}\\
            0&0&0&0&0\\
            0&0&0&1&0
          \end{bmatrix}
          &\xrightarrow{R_1-R_4}
          \begin{bmatrix}
            1&0&1&0&1-\frac{3}{2-a}\\
            0&1&0&0&\frac{3}{2-a}\\
            0&0&0&0&0\\
            0&0&0&1&0
          \end{bmatrix}\\
        \end{align*}
        Then, $x_1 = 1 - \frac{3}{2-a} - x_3 = \frac{3}{a-2} \qed$  

        \textbf{Case 2:} $a\neq_2, b=0$:
        \begin{align*}
          \xrightarrow{\frac{1}{2-a}R_2}
          \begin{bmatrix}
            1&1&1&1&1\\
            0&1&0&0&0\\
            0&0&-3&0&-3\\
            0&0&0&0&0
          \end{bmatrix}
          &\xrightarrow{R_1-R_2}
          \begin{bmatrix}
            1&0&1&1&1\\
            0&1&0&0&0\\
            0&0&-3&0&-3\\
            0&0&0&0&0
          \end{bmatrix}\\
          \xrightarrow{-\frac{1}{3}R_3}
          \begin{bmatrix}
            1&0&1&1&1\\
            0&1&0&0&0\\
            0&0&1&0&1\\
            0&0&0&0&0
          \end{bmatrix}
          &\xrightarrow{R_1-R_4}
          \begin{bmatrix}
            1&0&0&1&0\\
            0&1&0&0&0\\
            0&0&1&0&1\\
            0&0&0&0&0
          \end{bmatrix}\\
        \end{align*}
        Then, $x_1 = 0 - x_4 = 0 \qed$  
      \item It is not possible to have infinite solutions w/ 3 parameters. \\Firstly, we need to ensure the system is not inconsistent (when $a=2, b\neq 0$). \\Note that $x_1$ can never be free since it is not dependent on any variables $a, b$. \\Then, we can set $x_2$ free (when $a = 2 \land b=0$), $x_3$ free (when $b=-3$), or $x_4$ free (when $b=0$). \\
        However, since $a, b$ can only take one value at a time, we can at most satisfy 2 of the conditions simultaneously (when $a=2 \land b=0$), allowing at most 2 parameters $x_2, x_4$. $\qed$ 
    \end{enumerate}
    \pagebreak

  \item \begin{enumerate}[(\alph*)]
      \item Elementary row operations $A\xrightarrow{R_3+R_1}\xrightarrow{R_4-R_2}\xrightarrow{R_2+5R_1}U$ can also be expressed in terms with matrix multiplication of $A$ by elementary row matrices such as:
      \begin{align*}
        E_3E_2E_1A &= U \\
        E_3 = \begin{bmatrix}1&0&0&0\\5&1&0&0\\0&0&1&0\\0&0&0&1\end{bmatrix}
        ,\quad E_2 = &\begin{bmatrix}1&0&0&0\\0&1&0&0\\0&0&1&0\\0&-1&0&1\end{bmatrix}
        ,\quad E_1 = \begin{bmatrix}1&0&0&0\\0&1&0&0\\1&0&1&0\\0&0&0&1\end{bmatrix}
      \end{align*}
      Suppose there exists the LU factorisation, $A = LU$:
      \begin{align*}
        E_3E_2E_1LU &= U\tag*{(Substitute $A=LU$)}\\
        LU &= E_1^{-1}E_2^{-1}E_3^{-1}U\tag*{(Definition of inverse $EE^{-1} = E^{-1}E = I$)}\\ 
        L &= E_1^{-1}E_2^{-1}E_3^{-1} \\
          &= \begin{bmatrix}1&0&0&0\\0&1&0&0\\-1&0&1&0\\0&0&0&1\end{bmatrix} \begin{bmatrix}1&0&0&0\\0&1&0&0\\0&0&1&0\\0&1&0&1\end{bmatrix}  \begin{bmatrix}1&0&0&0\\-5&1&0&0\\0&0&1&0\\0&0&0&1\end{bmatrix}\\
          &= \begin{bmatrix}1&0&0&0\\-5&1&0&0\\-1&0&1&0\\-5&1&0&1\end{bmatrix} \qed
      \end{align*}

    \item To solve $A \vec{x} = LU \vec{x} = \begin{bmatrix}2\\8\\-2\\5\end{bmatrix}$, let $U \vec{x} = \vec{y}$ and solve $L \vec{y} = \begin{bmatrix}2\\8\\-2\\5\end{bmatrix}$:
      \begin{align*}
        \begin{bmatrix}
          1&0&0&0&2\\
          -5&1&0&0&8\\
          -1&0&1&0&-2\\
          -5&1&0&1&5
        \end{bmatrix} &\xrightarrow{RREF}
        \begin{bmatrix}
          1&0&0&0&2\\
          0&1&0&0&18\\
          0&0&1&0&0\\
          0&0&0&1&-2
        \end{bmatrix}\\
        \therefore \vec{y} &= \begin{bmatrix}2\\18\\0\\-2\end{bmatrix}
      \end{align*}
      Then solve $U \vec{x} =  \begin{bmatrix}2\\18\\0\\-2\end{bmatrix}$:
      \begin{align*}
        \begin{bmatrix}
          -1&2&-2&4&2\\
          0&15&-7&24&18\\
          0&0&1&-3&0\\
          0&0&0&-3&-2
        \end{bmatrix} &\xrightarrow{RREF}
        \begin{bmatrix}
          1&0&0&0&-2\\
          0&1&0&0&1\\
          0&0&1&0&3\\
          0&0&0&1&1
        \end{bmatrix}\\
        \therefore \vec{x} &= \begin{bmatrix}-2\\1\\3\\1\end{bmatrix}\qed
      \end{align*}

      \item By using properties of the determinant:
        \begin{align*}
          \det(A) &= \det(L) \cdot \det(U)\tag*{(Lay T3.6 Multiplicative property)}\\
                  &= (1 \cdot 1 \cdot 1 \cdot 1) \cdot (-1 \cdot 15 \cdot 1 \cdot -3)\tag*{(Lay T3.2 Determinant of triangular matrix)}\\
                  &= 1 \cdot 45\\
                  &= 45 \qed
        \end{align*}
    \end{enumerate}
    \pagebreak

  \item \begin{enumerate}[(\alph*)]
      \item \begin{enumerate}[(\roman*)]
          \item By the Invertible Matrix Theorem, $A$ is invertible $\iff \det(A) \neq 0$, so we find the values of $a$ for which $\det(A) \neq 0$:
            \begin{align*}
              \det(A) &\neq 0\\
              \begin{vmatrix}
                a&a&a\\
                1&1&0\\
                0&1&1
              \end{vmatrix} &\neq 0\\
              a \begin{vmatrix}1&0\\1&1\end{vmatrix} -  a \begin{vmatrix}1&0\\0&1\end{vmatrix} +  a \begin{vmatrix}1&1\\0&1\end{vmatrix} &\neq 0\tag*{\text{(Lay T3.1 Cofactor expansion)}}\\
              a(1-0) - a(1-0) + a(1-0) &\neq 0\tag*{\text{(Determinant of 2$\times$2 matrices)}}\\
              a &\neq 0
            \end{align*}
            Therefore, for $A$ to be invertible, $a \neq 0. \qed$

          \item Suppose $C_{ij}$ is the $(i, j)$ cofactor of $A$, and $M_{ij}$ is the $(i, j)$ matrix minor of $A$ obtained by deletion of the $i$th row and $j$th column. $C_{ij}$ is given by:
            \begin{align*}
              C_{ij} &= (-1)^{i+j}\det(M_{ij})\\
            \end{align*}
          First, find the cofactors of $A$, finding the determinant of each $M_{ij}$ by definition of determinant for $2\times2$ matrices:
            \begin{align*}
              C_{11} = +\begin{vmatrix}1&0\\1&1\end{vmatrix} = 1               ,\quad C_{21} &= -\begin{vmatrix}a&a\\1&1\end{vmatrix} = 0
              ,\quad C_{31} = +\begin{vmatrix}a&a\\1&0\end{vmatrix} = -a\\
              C_{12} = -\begin{vmatrix}1&0\\0&1\end{vmatrix} = -1
              ,\quad C_{22} &= +\begin{vmatrix}a&a\\0&1\end{vmatrix} = a
              ,\quad C_{32} = -\begin{vmatrix}a&a\\1&0\end{vmatrix} = a\\
              C_{13} = +\begin{vmatrix}1&1\\0&1\end{vmatrix} = 1
              ,\quad C_{23} &= -\begin{vmatrix}a&a\\0&1\end{vmatrix} = -a
              ,\quad C_{33} = +\begin{vmatrix}a&a\\1&1\end{vmatrix} = 0\\
            \end{align*}
            Then, $\adj(A)$ is then given by:
            \begin{align*}
              \adj(A) &= (C_{ij})^T\\
                      &= \begin{bmatrix}1&0&-a\\-1&a&a\\1&-a&0\end{bmatrix}\qed
            \end{align*}
      
      \item Suppose $A$ is invertible:
        \begin{align*}
          A^{-1} &= \frac{1}{\det(A)}\adj(A)\tag*{(Lay T3.8 Adjoint Formula for Inverse)}\\
                 &= \frac{1}{a}\begin{bmatrix}1&0&-a\\-1&a&a\\1&-a&0\end{bmatrix}\tag*{(From (i) and (ii))}\qed
        \end{align*}
        \end{enumerate}

      \item Suppose there is some matrix $cA$, where $c \in \RR$:
        \begin{align*}
          (cA)^{-1} &= \frac{1}{\det(cA)}\adj(cA)\tag*{(Lay T3.8 Adjoint Formula for Inverse)}\\
          \frac{1}{c}A^{-1} &= \frac{1}{\det(cA)}\adj(cA) \tag*{\text{(Chapter 2 Slide 87, $(aA)^{-1} = \frac{1}{a}A^{-1}$)}}\\
          \frac{1}{c} \cdot \frac{1}{\det(A)}\adj(A) &= \frac{1}{\det(cA)}\adj(cA) \tag*{\text{(Substituting in $A^{-1}$)}}\\
                                                     &= \frac{1}{c^n\det(A)}\adj(cA)\tag*{(Chapter 2 Slide 158, $\det(cA) = c^n\det(A)$)}\\
                                                     &= \frac{1}{c^n} \cdot \frac{1}{\det(A)}\adj(cA)\\
          \adj(A) &= \frac{1}{c^{n-1}} \adj(cA) \tag*{\text{(Cancelling common terms)}}\\
          \therefore \adj(cA) &= c^{n-1}\adj(A)
        \end{align*}
        Hence, for $\adj(A) = \begin{bmatrix}1&1&1&0\\1&1&0&1\\1&0&1&1\\0&1&1&1\end{bmatrix}$, $\adj(3A)$ is given by:
        \begin{align*}
          \adj(3A) &= 3^3 \cdot \adj(A)\\
                   &= 27\cdot \adj(A)\\
                   &= \begin{bmatrix}27&27&27&0\\27&27&0&27\\27&0&27&27\\0&27&27&27\end{bmatrix} \qed
        \end{align*}
    \end{enumerate}
\end{enumerate}
%%%%%%%%%%%%%%%%%%%%%%%%%%%%%%%%%%%%%%%%%%%%%%%%%%%%%%
%                       End                          %
%%%%%%%%%%%%%%%%%%%%%%%%%%%%%%%%%%%%%%%%%%%%%%%%%%%%%%

\end{document}

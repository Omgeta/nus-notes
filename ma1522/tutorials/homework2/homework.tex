\documentclass[12pt, a4paper]{article}

\usepackage[a4paper, margin=1in]{geometry}

\usepackage{fancyhdr}
\pagestyle{fancy}
\fancyhf{}
\fancyhead[R]{\thepage}
\renewcommand{\headrulewidth}{0pt}

\usepackage[utf8]{inputenc}
\usepackage[mathscr]{euscript}
\let\euscr\mathscr \let\mathscr\relax
\usepackage[scr]{rsfso}
\usepackage{amssymb,amsmath,amsthm,amsfonts}
\usepackage[shortlabels]{enumitem}
\usepackage{multicol,multirow}
\usepackage{lipsum}
\usepackage{balance}
\usepackage{calc}
\usepackage[colorlinks=true,citecolor=blue,linkcolor=blue]{hyperref}
\usepackage{import}
\usepackage{xifthen}
\usepackage{pdfpages}
\usepackage{transparent}
\usepackage{listings}

\newcommand{\incfig}[2][1.0]{
    \def\svgwidth{#1\columnwidth}
    \import{./figures/}{#2.pdf_tex}
}

\newlist{enumproof}{enumerate}{4}
\setlist[enumproof,1]{label=\arabic*., parsep=1em}
\setlist[enumproof,2]{label=\arabic{enumproofi}.\arabic*., parsep=1em}
\setlist[enumproof,3]{label=\arabic{enumproofi}.\arabic{enumproofii}.\arabic*., parsep=1em}
\setlist[enumproof,4]{label=\arabic{enumproofi}.\arabic{enumproofii}.\arabic{enumproofiii}.\arabic*., parsep=1em}

\renewcommand{\qedsymbol}{\ensuremath{\blacksquare}}

\lstdefinestyle{mystyle}{
  language=C, % Set the language to C
  commentstyle=\color{codegray}, % Color for comments
  keywordstyle=\color{orange}, % Color for basic keywords
  stringstyle=\color{mauve}, % Color for strings
  basicstyle={\ttfamily\footnotesize}, % Basic font style
  breakatwhitespace=false,         
  breaklines=true,                 
  captionpos=b,                    
  keepspaces=true,                 
  numbers=none,                    
  tabsize=2,
  morekeywords=[2]{\#include, \#define, \#ifdef, \#ifndef, \#endif, \#pragma, \#else, \#elif}, % Preprocessor directives
  keywordstyle=[2]\color{codegreen}, % Style for preprocessor directives
  morekeywords=[3]{int, char, float, double, void, struct, union, enum, const, volatile, static, extern, register, inline, restrict, _Bool, _Complex, _Imaginary, size_t, ssize_t, FILE}, % C standard types and common identifiers
  keywordstyle=[3]\color{identblue}, % Style for types and common identifiers
  morekeywords=[4]{printf, scanf, fopen, fclose, malloc, free, calloc, realloc, perror, strtok, strncpy, strcpy, strcmp, strlen}, % Standard library functions
  keywordstyle=[4]\color{cyan}, % Style for library functions
}

% Things Lie
\newcommand{\kb}{\mathfrak b}
\newcommand{\kg}{\mathfrak g}
\newcommand{\kh}{\mathfrak h}
\newcommand{\kn}{\mathfrak n}
\newcommand{\ku}{\mathfrak u}
\newcommand{\kz}{\mathfrak z}
\DeclareMathOperator{\Ext}{Ext} % Ext functor
\DeclareMathOperator{\Tor}{Tor} % Tor functor
\newcommand{\gl}{\opname{\mathfrak{gl}}} % frak gl group
\renewcommand{\sl}{\opname{\mathfrak{sl}}} % frak sl group chktex 6

% More script letters etc.
\newcommand{\SA}{\mathcal A}
\newcommand{\SB}{\mathcal B}
\newcommand{\SC}{\mathcal C}
\newcommand{\SF}{\mathcal F}
\newcommand{\SG}{\mathcal G}
\newcommand{\SH}{\mathcal H}
\newcommand{\OO}{\mathcal O}

\newcommand{\SCA}{\mathscr A}
\newcommand{\SCB}{\mathscr B}
\newcommand{\SCC}{\mathscr C}
\newcommand{\SCD}{\mathscr D}
\newcommand{\SCE}{\mathscr E}
\newcommand{\SCF}{\mathscr F}
\newcommand{\SCG}{\mathscr G}
\newcommand{\SCH}{\mathscr H}

% Mathfrak primes
\newcommand{\km}{\mathfrak m}
\newcommand{\kp}{\mathfrak p}
\newcommand{\kq}{\mathfrak q}

% number sets
\newcommand{\RR}[1][]{\ensuremath{\ifstrempty{#1}{\mathbb{R}}{\mathbb{R}^{#1}}}}
\newcommand{\NN}[1][]{\ensuremath{\ifstrempty{#1}{\mathbb{N}}{\mathbb{N}^{#1}}}}
\newcommand{\ZZ}[1][]{\ensuremath{\ifstrempty{#1}{\mathbb{Z}}{\mathbb{Z}^{#1}}}}
\newcommand{\QQ}[1][]{\ensuremath{\ifstrempty{#1}{\mathbb{Q}}{\mathbb{Q}^{#1}}}}
\newcommand{\CC}[1][]{\ensuremath{\ifstrempty{#1}{\mathbb{C}}{\mathbb{C}^{#1}}}}
\newcommand{\PP}[1][]{\ensuremath{\ifstrempty{#1}{\mathbb{P}}{\mathbb{P}^{#1}}}}
\newcommand{\HH}[1][]{\ensuremath{\ifstrempty{#1}{\mathbb{H}}{\mathbb{H}^{#1}}}}
\newcommand{\FF}[1][]{\ensuremath{\ifstrempty{#1}{\mathbb{F}}{\mathbb{F}^{#1}}}}
% expected value
\newcommand{\EE}{\ensuremath{\mathbb{E}}}
\newcommand{\charin}{\text{ char }}
\DeclareMathOperator{\sign}{sign}
\DeclareMathOperator{\Aut}{Aut}
\DeclareMathOperator{\Inn}{Inn}
\DeclareMathOperator{\Syl}{Syl}
\DeclareMathOperator{\Gal}{Gal}
\DeclareMathOperator{\GL}{GL} % General linear group
\DeclareMathOperator{\SL}{SL} % Special linear group

%---------------------------------------
% BlackBoard Math Fonts :-
%---------------------------------------

%Captital Letters
\newcommand{\bbA}{\mathbb{A}}	\newcommand{\bbB}{\mathbb{B}}
\newcommand{\bbC}{\mathbb{C}}	\newcommand{\bbD}{\mathbb{D}}
\newcommand{\bbE}{\mathbb{E}}	\newcommand{\bbF}{\mathbb{F}}
\newcommand{\bbG}{\mathbb{G}}	\newcommand{\bbH}{\mathbb{H}}
\newcommand{\bbI}{\mathbb{I}}	\newcommand{\bbJ}{\mathbb{J}}
\newcommand{\bbK}{\mathbb{K}}	\newcommand{\bbL}{\mathbb{L}}
\newcommand{\bbM}{\mathbb{M}}	\newcommand{\bbN}{\mathbb{N}}
\newcommand{\bbO}{\mathbb{O}}	\newcommand{\bbP}{\mathbb{P}}
\newcommand{\bbQ}{\mathbb{Q}}	\newcommand{\bbR}{\mathbb{R}}
\newcommand{\bbS}{\mathbb{S}}	\newcommand{\bbT}{\mathbb{T}}
\newcommand{\bbU}{\mathbb{U}}	\newcommand{\bbV}{\mathbb{V}}
\newcommand{\bbW}{\mathbb{W}}	\newcommand{\bbX}{\mathbb{X}}
\newcommand{\bbY}{\mathbb{Y}}	\newcommand{\bbZ}{\mathbb{Z}}

%---------------------------------------
% MathCal Fonts :-
%---------------------------------------

%Captital Letters
\newcommand{\mcA}{\mathcal{A}}	\newcommand{\mcB}{\mathcal{B}}
\newcommand{\mcC}{\mathcal{C}}	\newcommand{\mcD}{\mathcal{D}}
\newcommand{\mcE}{\mathcal{E}}	\newcommand{\mcF}{\mathcal{F}}
\newcommand{\mcG}{\mathcal{G}}	\newcommand{\mcH}{\mathcal{H}}
\newcommand{\mcI}{\mathcal{I}}	\newcommand{\mcJ}{\mathcal{J}}
\newcommand{\mcK}{\mathcal{K}}	\newcommand{\mcL}{\mathcal{L}}
\newcommand{\mcM}{\mathcal{M}}	\newcommand{\mcN}{\mathcal{N}}
\newcommand{\mcO}{\mathcal{O}}	\newcommand{\mcP}{\mathcal{P}}
\newcommand{\mcQ}{\mathcal{Q}}	\newcommand{\mcR}{\mathcal{R}}
\newcommand{\mcS}{\mathcal{S}}	\newcommand{\mcT}{\mathcal{T}}
\newcommand{\mcU}{\mathcal{U}}	\newcommand{\mcV}{\mathcal{V}}
\newcommand{\mcW}{\mathcal{W}}	\newcommand{\mcX}{\mathcal{X}}
\newcommand{\mcY}{\mathcal{Y}}	\newcommand{\mcZ}{\mathcal{Z}}

%---------------------------------------
% Bold Math Fonts :-
%---------------------------------------

%Captital Letters
\newcommand{\bmA}{\boldsymbol{A}}	\newcommand{\bmB}{\boldsymbol{B}}
\newcommand{\bmC}{\boldsymbol{C}}	\newcommand{\bmD}{\boldsymbol{D}}
\newcommand{\bmE}{\boldsymbol{E}}	\newcommand{\bmF}{\boldsymbol{F}}
\newcommand{\bmG}{\boldsymbol{G}}	\newcommand{\bmH}{\boldsymbol{H}}
\newcommand{\bmI}{\boldsymbol{I}}	\newcommand{\bmJ}{\boldsymbol{J}}
\newcommand{\bmK}{\boldsymbol{K}}	\newcommand{\bmL}{\boldsymbol{L}}
\newcommand{\bmM}{\boldsymbol{M}}	\newcommand{\bmN}{\boldsymbol{N}}
\newcommand{\bmO}{\boldsymbol{O}}	\newcommand{\bmP}{\boldsymbol{P}}
\newcommand{\bmQ}{\boldsymbol{Q}}	\newcommand{\bmR}{\boldsymbol{R}}
\newcommand{\bmS}{\boldsymbol{S}}	\newcommand{\bmT}{\boldsymbol{T}}
\newcommand{\bmU}{\boldsymbol{U}}	\newcommand{\bmV}{\boldsymbol{V}}
\newcommand{\bmW}{\boldsymbol{W}}	\newcommand{\bmX}{\boldsymbol{X}}
\newcommand{\bmY}{\boldsymbol{Y}}	\newcommand{\bmZ}{\boldsymbol{Z}}
%Small Letters
\newcommand{\bma}{\boldsymbol{a}}	\newcommand{\bmb}{\boldsymbol{b}}
\newcommand{\bmc}{\boldsymbol{c}}	\newcommand{\bmd}{\boldsymbol{d}}
\newcommand{\bme}{\boldsymbol{e}}	\newcommand{\bmf}{\boldsymbol{f}}
\newcommand{\bmg}{\boldsymbol{g}}	\newcommand{\bmh}{\boldsymbol{h}}
\newcommand{\bmi}{\boldsymbol{i}}	\newcommand{\bmj}{\boldsymbol{j}}
\newcommand{\bmk}{\boldsymbol{k}}	\newcommand{\bml}{\boldsymbol{l}}
\newcommand{\bmm}{\boldsymbol{m}}	\newcommand{\bmn}{\boldsymbol{n}}
\newcommand{\bmo}{\boldsymbol{o}}	\newcommand{\bmp}{\boldsymbol{p}}
\newcommand{\bmq}{\boldsymbol{q}}	\newcommand{\bmr}{\boldsymbol{r}}
\newcommand{\bms}{\boldsymbol{s}}	\newcommand{\bmt}{\boldsymbol{t}}
\newcommand{\bmu}{\boldsymbol{u}}	\newcommand{\bmv}{\boldsymbol{v}}
\newcommand{\bmw}{\boldsymbol{w}}	\newcommand{\bmx}{\boldsymbol{x}}
\newcommand{\bmy}{\boldsymbol{y}}	\newcommand{\bmz}{\boldsymbol{z}}

%---------------------------------------
% Scr Math Fonts :-
%---------------------------------------

\newcommand{\sA}{{\mathscr{A}}}   \newcommand{\sB}{{\mathscr{B}}}
\newcommand{\sC}{{\mathscr{C}}}   \newcommand{\sD}{{\mathscr{D}}}
\newcommand{\sE}{{\mathscr{E}}}   \newcommand{\sF}{{\mathscr{F}}}
\newcommand{\sG}{{\mathscr{G}}}   \newcommand{\sH}{{\mathscr{H}}}
\newcommand{\sI}{{\mathscr{I}}}   \newcommand{\sJ}{{\mathscr{J}}}
\newcommand{\sK}{{\mathscr{K}}}   \newcommand{\sL}{{\mathscr{L}}}
\newcommand{\sM}{{\mathscr{M}}}   \newcommand{\sN}{{\mathscr{N}}}
\newcommand{\sO}{{\mathscr{O}}}   \newcommand{\sP}{{\mathscr{P}}}
\newcommand{\sQ}{{\mathscr{Q}}}   \newcommand{\sR}{{\mathscr{R}}}
\newcommand{\sS}{{\mathscr{S}}}   \newcommand{\sT}{{\mathscr{T}}}
\newcommand{\sU}{{\mathscr{U}}}   \newcommand{\sV}{{\mathscr{V}}}
\newcommand{\sW}{{\mathscr{W}}}   \newcommand{\sX}{{\mathscr{X}}}
\newcommand{\sY}{{\mathscr{Y}}}   \newcommand{\sZ}{{\mathscr{Z}}}


%---------------------------------------
% Math Fraktur Font
%---------------------------------------

%Captital Letters
\newcommand{\mfA}{\mathfrak{A}}	\newcommand{\mfB}{\mathfrak{B}}
\newcommand{\mfC}{\mathfrak{C}}	\newcommand{\mfD}{\mathfrak{D}}
\newcommand{\mfE}{\mathfrak{E}}	\newcommand{\mfF}{\mathfrak{F}}
\newcommand{\mfG}{\mathfrak{G}}	\newcommand{\mfH}{\mathfrak{H}}
\newcommand{\mfI}{\mathfrak{I}}	\newcommand{\mfJ}{\mathfrak{J}}
\newcommand{\mfK}{\mathfrak{K}}	\newcommand{\mfL}{\mathfrak{L}}
\newcommand{\mfM}{\mathfrak{M}}	\newcommand{\mfN}{\mathfrak{N}}
\newcommand{\mfO}{\mathfrak{O}}	\newcommand{\mfP}{\mathfrak{P}}
\newcommand{\mfQ}{\mathfrak{Q}}	\newcommand{\mfR}{\mathfrak{R}}
\newcommand{\mfS}{\mathfrak{S}}	\newcommand{\mfT}{\mathfrak{T}}
\newcommand{\mfU}{\mathfrak{U}}	\newcommand{\mfV}{\mathfrak{V}}
\newcommand{\mfW}{\mathfrak{W}}	\newcommand{\mfX}{\mathfrak{X}}
\newcommand{\mfY}{\mathfrak{Y}}	\newcommand{\mfZ}{\mathfrak{Z}}
%Small Letters
\newcommand{\mfa}{\mathfrak{a}}	\newcommand{\mfb}{\mathfrak{b}}
\newcommand{\mfc}{\mathfrak{c}}	\newcommand{\mfd}{\mathfrak{d}}
\newcommand{\mfe}{\mathfrak{e}}	\newcommand{\mff}{\mathfrak{f}}
\newcommand{\mfg}{\mathfrak{g}}	\newcommand{\mfh}{\mathfrak{h}}
\newcommand{\mfi}{\mathfrak{i}}	\newcommand{\mfj}{\mathfrak{j}}
\newcommand{\mfk}{\mathfrak{k}}	\newcommand{\mfl}{\mathfrak{l}}
\newcommand{\mfm}{\mathfrak{m}}	\newcommand{\mfn}{\mathfrak{n}}
\newcommand{\mfo}{\mathfrak{o}}	\newcommand{\mfp}{\mathfrak{p}}
\newcommand{\mfq}{\mathfrak{q}}	\newcommand{\mfr}{\mathfrak{r}}
\newcommand{\mfs}{\mathfrak{s}}	\newcommand{\mft}{\mathfrak{t}}
\newcommand{\mfu}{\mathfrak{u}}	\newcommand{\mfv}{\mathfrak{v}}
\newcommand{\mfw}{\mathfrak{w}}	\newcommand{\mfx}{\mathfrak{x}}
\newcommand{\mfy}{\mathfrak{y}}	\newcommand{\mfz}{\mathfrak{z}}


\newcommand{\mytitle}{MA1522 Homework 2}
\newcommand{\myauthor}{github/omgeta}
\newcommand{\mydate}{AY 24/25 Sem 1}

\begin{document}
\raggedright
\footnotesize
\begin{center}
{\normalsize{\textbf{\mytitle}}} \\
{\footnotesize{\mydate\hspace{2pt}\textemdash\hspace{2pt}\myauthor}}
\end{center}
\setlist{topsep=-1em, itemsep=-1em, parsep=2em}

%%%%%%%%%%%%%%%%%%%%%%%%%%%%%%%%%%%%%%%%%%%%%%%%%%%%%%
%                      Begin                         %
%%%%%%%%%%%%%%%%%%%%%%%%%%%%%%%%%%%%%%%%%%%%%%%%%%%%%%
\begin{enumerate}[Q\arabic*.]
  \item 
    \begin{enumerate}[(\alph*)]
      \item Check V by substituting $x_1 = 1, x_2 = 2, x_3 = 0, x_4 = 1$ into the equation of V:
        \begin{gather*}
          2(1) + 1(2) -3(1) = 1 \neq 0\\
                            \implies \text{ the vector does not satisfy the equation for }V \\
                            \implies  \begin{bmatrix}1\\2\\0\\1\end{bmatrix} \not\in V
        \end{gather*}
        Check $U$ by RREF:
        \begin{gather*}
          \begin{bmatrix}
            3 & 1 & 7 & 1 & | &1\\
            3 & 3 & 1 & 4 & | &2\\
            -3 & -1 & -3 & -3 & | &0\\
            3 & 2 & 5 & 2 & | &1
          \end{bmatrix} \xrightarrow{RREF}
          \begin{bmatrix}
            1 & 0 & 0 & 3 /2 & | & 0\\
            0 & 1 & 0 & 0 & | & 0\\
            0 & 0 & 1 & -1 /2 & | & 0\\
            0 & 0 & 0 & 0 & | & 1
          \end{bmatrix}\\
          \implies \text{inconsistent equation in the last row}\\
          \implies \text{the vector is not in the column space}\\
          \implies  \begin{bmatrix}1\\2\\0\\1\end{bmatrix} \not\in U
        \end{gather*}
        Therefore, $\begin{bmatrix}1\\2\\0\\1\end{bmatrix} \not\in V \land \begin{bmatrix}1\\2\\0\\1\end{bmatrix} \not\in U \qed$

      \item From the RREF of the matrix formed by the spanning set of $U$ in (a), we can see the 4th column vector is a linear combination of the 1st and 3rd. By removing either the 1st, 3rd, or 4th vector, we can get a linearly independent set which also spans $U$.\\Therefore, a possible basis is $\left\{\begin{bmatrix}3\\3\\-3\\3\end{bmatrix}, \begin{bmatrix}1\\3\\-1\\2\end{bmatrix}, \begin{bmatrix}7\\1\\-3\\5\end{bmatrix}\right\} \qed$ 

      \item Suppose there is a linear equation for $U: a_1x_1 + a_2x_2 + a_3x_3 + a_4x_4 = \vec{b}$, since $\vec{0} \in U$, the following equations must hold:
        \begin{align*}
          3a_1 + 3a_2 -3a_3 + 3a_4 &= 0\\
          1a_1 + 3a_2 -1a_3 + 2a_4 &= 0\\
          7a_1 + 1a_2 -3a_3 + 5a_4 &= 0
        \end{align*}
        Reduce the corresponding matrix:
        \begin{align*}
          \begin{bmatrix}
            3 & 3 & -3 & 3 &|& 0\\
            1 & 3 & -1 & 2 &|& 0\\
            7 & 1 & -3 & 5 &|& 0\\
          \end{bmatrix}&\xrightarrow{RREF}
          \begin{bmatrix}
            1 & 0 & 0 & 3 /4 &|& 0\\
            0 & 1 & 0 & 1 /2 &|& 0\\
            0 & 0 & 1 & 1 /4 &|& 0
          \end{bmatrix}\\
          \implies \begin{bmatrix}a_1\\a_2\\a_3\\a_4\end{bmatrix} &= s\begin{bmatrix}3\\2\\1\\-4\end{bmatrix}, s \in \RR
        \end{align*}
        Therefore, a linear equation is $3x_1 + 2x_2 + x_3 -4x_4 = 0 \qed$

      \item Firstly, check if $T$ is linearly independent:
        \begin{gather*}
          \begin{bmatrix}
            1 & 1 & 2\\
            1 & 1 & -1\\
            -2 & 1 & 0\\
            1 & 1 & 1
          \end{bmatrix}\xrightarrow{RREF}
          \begin{bmatrix}
            1 & 0 & 0\\
            0 & 1 & 0\\
            0 & 0 & 1\\
            0 & 0 & 0
          \end{bmatrix}\\
          \implies T\text{ is linearly independent}
        \end{gather*}
        Secondly, check if $\Span(T) \subseteq V$:
        \begin{gather*}
          2(1) + 1(1) - 3(1) = 0\\
          2(2) + 1(-1) - 3(1) = 0\\
          \implies \begin{bmatrix}1\\1\\-2\\1\end{bmatrix}, \begin{bmatrix}1\\1\\1\\1\end{bmatrix}, \begin{bmatrix}2\\-1\\0\\1\end{bmatrix} \in V\\
          \implies \Span(T) \subseteq V\text{, by closure over addition and multiplication}
        \end{gather*}
        Thirdly, check if $V \subseteq \Span(T)$:
        \begin{align*}
          \vec{v} \in V &= \begin{bmatrix}x_1\\x_2\\x_3\\x_4\end{bmatrix} = \begin{bmatrix}x_1\\3x_4-2x_1\\x_3\\x_4\end{bmatrix}\\
                        &= s\begin{bmatrix}1\\-2\\0\\0\end{bmatrix} + t\begin{bmatrix}0\\0\\1\\0\end{bmatrix} + u\begin{bmatrix}0\\3\\0\\1\end{bmatrix},\quad s,t,u\in \RR\\
          \therefore V &= \Span\left\{\begin{bmatrix}1\\-2\\0\\0\end{bmatrix},\begin{bmatrix}0\\0\\1\\0\end{bmatrix},\begin{bmatrix}0\\3\\0\\1\end{bmatrix}\right\}\\
        \end{align*}
        \vspace{-2em}
        \begin{gather*}
          \begin{bmatrix}
            1 & 1 & 2 & | &1 & 0 & 0\\
            1 & 1 & -1 & | &-2 & 0 & 3\\
            -2 & 1 & 0 & | &0 & 1 & 0\\
            1 & 1 & 1 & | &0 & 0 & 1\\
          \end{bmatrix}\xrightarrow{RREF}
          \begin{bmatrix}
            1 & 0 & 0 &|& -1 /3 & -1 /3 & 2 /3\\
            0 & 1 & 0 &|& -2 /3 & 1 /3 & 4 /3\\
            0 & 0 & 1 &|& 1 & 0 & -1\\
            0 & 0 & 0 &|& 0 & 0 & 0
          \end{bmatrix}\\
                       \implies V \subseteq \Span(T)
        \end{gather*}
      Since $Span(T) \subseteq V$ and $V \subseteq \Span(T)$, then $\Span(T) = V$, by the definition of set equality. Therefore, $\Span(T) = V$ and $T$ is linearly independent $\implies T$ is a basis for $V \qed$

    \item For a vector $\vec{v} \in U \cap V$:
      \begin{align*}
        2x_1 + x_2 + 0x_3 - 3x_4 &= 0\\
        3x_1 + 2x_2 + 1x_3 -4x_3 &= 0\\
      \end{align*}
      Reduce the corresponding matrix:
      \begin{align*}
        \begin{bmatrix}
          2 & 1 & 0 & -3 & |&0\\
          3 & 2 & 1 & -4 & |&0
        \end{bmatrix}&\xrightarrow{RREF}
        \begin{bmatrix}
          1 & 0 & -1 & -2 & |&0\\
          0 & 1 & 2 & 1 & |&0\\
        \end{bmatrix}\\
        \therefore \vec{v} = s\begin{bmatrix}1\\-2\\1\\0\end{bmatrix} + t\begin{bmatrix}2\\-1\\0\\1\end{bmatrix}&,\quad s, t \in \RR
      \end{align*}
      Therefore, a basis for $U \cap V$ is $\left\{\begin{bmatrix}1\\-2\\1\\0\end{bmatrix},\begin{bmatrix}2\\-1\\0\\1\end{bmatrix}\right\} \qed$
  \end{enumerate}
  \pagebreak
  \item Yes.\\
    Includes zero vector: when $s_1 = -1, s_2 = 1, s_3 = 2, \vec{v} = \vec{0}$\\
    Closure over addition:
    \begin{align*}
      \vec{u} + \vec{v} &= \begin{bmatrix}4\\-3\\9\\1\end{bmatrix} + s_1\begin{bmatrix}6\\6\\8\\2\end{bmatrix} + s_2\begin{bmatrix}6\\3\\7\\1\end{bmatrix} + s_3\begin{bmatrix}-2\\3\\-4\\0\end{bmatrix}\\
      &\quad+ \begin{bmatrix}4\\-3\\9\\1\end{bmatrix} + t_1\begin{bmatrix}6\\6\\8\\2\end{bmatrix} + t_2\begin{bmatrix}6\\3\\7\\1\end{bmatrix} + t_3\begin{bmatrix}-2\\3\\-4\\0\end{bmatrix},\text{ where }\vec{u}, \vec{v} \in V\\\\
      &= \begin{bmatrix}4\\-3\\9\\1\end{bmatrix} + (s_1+t_1+1)\begin{bmatrix}6\\6\\8\\2\end{bmatrix} + (s_2+t_2-1)\begin{bmatrix}6\\3\\7\\1\end{bmatrix} + (s_3+t_3-2)\begin{bmatrix}-2\\3\\-4\\0\end{bmatrix}\\
      &= \begin{bmatrix}4\\-3\\9\\1\end{bmatrix} + s_1'\begin{bmatrix}6\\6\\8\\2\end{bmatrix} + s_2'\begin{bmatrix}6\\3\\7\\1\end{bmatrix} + s_3'\begin{bmatrix}-2\\3\\-4\\0\end{bmatrix}\tag*{($\RR$ closed over addition)}\\
      &\implies u+v \in V
    \end{align*}
    Closure over multiplication:
    \begin{align*}
      c \cdot \vec{v} &= c\begin{bmatrix}4\\-3\\9\\1\end{bmatrix} + cs_1\begin{bmatrix}6\\6\\8\\2\end{bmatrix} + cs_2\begin{bmatrix}6\\3\\7\\1\end{bmatrix} + cs_3\begin{bmatrix}-2\\3\\-4\\0\end{bmatrix}, \text{ where }c \in \RR, \vec{v} \in V\\
      &= \begin{bmatrix}4\\-3\\9\\1\end{bmatrix} + (cs_1+c)\begin{bmatrix}6\\6\\8\\2\end{bmatrix} + (cs_2-c)\begin{bmatrix}6\\3\\7\\1\end{bmatrix} + (cs_3-2c)\begin{bmatrix}-2\\3\\-4\\0\end{bmatrix}\\
      &= \begin{bmatrix}4\\-3\\9\\1\end{bmatrix} + s_1'\begin{bmatrix}6\\6\\8\\2\end{bmatrix} + s_2'\begin{bmatrix}6\\3\\7\\1\end{bmatrix} + s_3'\begin{bmatrix}-2\\3\\-4\\0\end{bmatrix}\tag*{($\RR$ closed over addition/multiplication)}\\
      &\implies c \vec{v} \in V
    \end{align*}

  To check if the spanning set is a basis, check for linearly independence:
  \begin{gather*}
    \begin{bmatrix}
      6 & 6 & -2\\
      6 & 3 & 3\\
      8 & 7 & -4\\
      2 & 1 & 0
    \end{bmatrix}\xrightarrow{RREF}
    \begin{bmatrix}
      1 & 0 & 0\\
      0 & 1 & 0\\
      0 & 0 & 1\\
      0 & 0 & 0
    \end{bmatrix}\\
    \implies \text{ all the vectors are linearly independent}
  \end{gather*}
  Therefore, a basis for $V$ is $\left\{\begin{bmatrix}6\\6\\8\\2\end{bmatrix}, \begin{bmatrix}6\\3\\7\\1\end{bmatrix}, \begin{bmatrix}-2\\3\\-4\\0\end{bmatrix}\right\} \qed$

  \pagebreak
  \item
    \begin{enumerate}[(\alph*)]
      \item Reduce matrix A:
        \begin{align*}
          \begin{bmatrix}
            2 & 4 & 3\\
            9 & 6 & 3\\
            -1 & 3 & 4\\
            1 & 1 & 1
          \end{bmatrix}\xrightarrow{RREF}
          \begin{bmatrix}
            1 & 0 & 0\\
            0 & 1 & 0\\
            0 & 0 & 1\\
            0 & 0 & 0
          \end{bmatrix}
        \end{align*}
        $\rank(A) = \dim(\col(A)) =$ number of pivot columns of A $ = 3\qed$

    \item Yes; Suppose $B = \begin{bmatrix}a & b & c & d\\e & f & g & h\\i & j & k & l\end{bmatrix}$, then since $BA = I_3 \iff A^TB^T = I_3$, to solve, we reduce the system $[A^T \mid I_3]$:
      \begin{align*}
        \begin{bmatrix}
          2 & 9 & -1 & 1 & \mid & 1 & 0 & 0\\
          4 & 6 & 3 & 1 & \mid & 0 & 1 & 0\\
          3 & 3 & 4 & 1 & \mid & 0 & 0 & 1
        \end{bmatrix}\xrightarrow{RREF}
        \begin{bmatrix}
          1 & 0 & 0 & -1 /3 & \mid & -5 /9 & 13 /9 & -11 /9\\
          0 & 1 & 0 & 2 /9 & \mid & 7 /27 & -11 /27 & 10 /27\\
          0 & 0 & 1 & 1 /3 & \mid & 2 /9 & -7 /9 & 8 /9
        \end{bmatrix}
      \end{align*}
      Therefore, $B = \begin{bmatrix}
        -5 /9 + s /3 & 7 /27 - 2s /9 & 2 /9 - s /3 & s\\
        13 /9 + t /3 & -11 /27 -2t /9 & -7 /9 -t /3 & t\\
        -11 /9 + u /3 & 10 /27 -2u /9 & 8 /9 - u /3 & u
      \end{bmatrix}, s, t, u \in \RR \qed$

    \item No; $\rank(A) \neq \text{number of rows}= 4 \implies A$ has no right inverse.$\qed$\hfill(Math Cafe 7, Slide 30)

    \item No; $\nul(A) \perp \row(A) \implies \nul(A^T) \perp \col(A) \implies \forall v \in \nul(A^T),$  $\vec{v} \not\in \col(A)\qed$\\
      \quad\\
      Alternatively, suppose nonzero $\vec{v} \in \nul(A^T)$, then $A^T \vec{v} = \vec{0}$:
      \begin{align*}
        \begin{bmatrix}
          2 & 9 & -1 & 1 & | & 0 \\
          4 & 6 & 3 & 1 & |& 0\\
          3 & 3 & 4 & 1 & | & 0\\
        \end{bmatrix}&\xrightarrow{RREF}
        \begin{bmatrix}
          1 & 0 & 0 & -1 /3 & | & 0 \\
          0 & 1 & 0 & 2 /9 & |& 0\\
          0 & 0 & 1 & -1 /3 & | & 0\\
        \end{bmatrix}\\
        \therefore \vec{v} &= s\begin{bmatrix}3\\-2\\3\\-9\end{bmatrix},\quad s\in \RR \setminus \{0\}
      \end{align*}
      However, $A\vec{x} = \vec{v}$ will not be consistent as shown below:
      \begin{align*}
        \begin{bmatrix}
          2 & 4 & 3 & | & 3\\
          9 & 6 & 3 & | & -2\\
          -1 & 3 & 4 & | & 3\\
          1 & 1 & 1 & | & -9
        \end{bmatrix}\xrightarrow{RREF}
        \begin{bmatrix}
          1 & 0 & 0 & | & 0\\
          0 & 1 & 0 & | & 0\\
          0 & 0 & 1 & | & 0\\
          0 & 0 & 0 & | & 1\\
        \end{bmatrix} 
      \end{align*}
      Therefore, there exists no nonzero vector $\vec{v} \in \nul(A^T) \land A \vec{x} = \vec{v} \qed$
    \end{enumerate}
  \pagebreak

\item 
  \begin{enumerate}[(\alph*)]
    \item Find the transition matrix from $S$ to $T$: 
      \begin{align*}
        \begin{bmatrix}
          1 & 1 & 0 & | & 1 & 0 & 0\\
          3 & -3 & 6 & | & 0 & 3 & 0\\
          -1 & 3 & -1 & | & 1 & -2 & 3\\
          0 & 0 & 1 & | & 0 & 0 & 1
        \end{bmatrix}\xrightarrow{RREF}
        \begin{bmatrix}
          1 & 0 & 0 & | & 1 /2 & 1 /2 & -1\\
          0 & 1 & 0 & | & 1 /2 & -1 /2 & 1\\
          0 & 0 & 1 & | & 0 & 0 & 1\\
          0 & 0 & 0 & | & 0 & 0 & 0
        \end{bmatrix}\\
      \end{align*}
      Therefore $[w]_T = \begin{bmatrix}1 /2 & 1 /2 & -1\\ 1 /2 & -1 /2 & 1\\ 0 & 0 & 1\end{bmatrix}\begin{bmatrix}x_1\\x_2\\x_3\end{bmatrix} = \begin{bmatrix}\displaystyle \frac{x_1+x_2}{2}-x_3\\\displaystyle \frac{x_1-x_2}{2}+x_3\\x_3\end{bmatrix} \qed$

    \item Find the transition matrix from $B$ to the standard basis, using $P_S$, the transition matrix from $S$ to the standard basis: 
      \begin{align*}
        P_SP &= \begin{bmatrix}1&0&0\\0&3&0\\1&-2&3\\0&0&1\end{bmatrix}\begin{bmatrix}1&-1&2\\-1&1&1\\1&0&2\end{bmatrix}\\
             &= \begin{bmatrix}1&-1&2\\-3&3&3\\6&-3&6\\1&0&2\end{bmatrix}
      \end{align*}
      Therefore, a basis for $B$ is $\left\{\begin{bmatrix}1\\-3\\6\\1\end{bmatrix},\begin{bmatrix}-1\\3\\-3\\0\end{bmatrix},\begin{bmatrix}2\\3\\6\\2\end{bmatrix}\right\} \qed$
  \end{enumerate}


\end{enumerate}
%%%%%%%%%%%%%%%%%%%%%%%%%%%%%%%%%%%%%%%%%%%%%%%%%%%%%%
%                       End                          %
%%%%%%%%%%%%%%%%%%%%%%%%%%%%%%%%%%%%%%%%%%%%%%%%%%%%%%

\end{document}

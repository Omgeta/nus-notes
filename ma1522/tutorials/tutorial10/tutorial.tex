\documentclass[12pt, a4paper]{article}

\usepackage[a4paper, margin=1in]{geometry}

\usepackage[utf8]{inputenc}
\usepackage[mathscr]{euscript}
\let\euscr\mathscr \let\mathscr\relax
\usepackage[scr]{rsfso}
\usepackage{amssymb,amsmath,amsthm,amsfonts}
\usepackage[shortlabels]{enumitem}
\usepackage{multicol,multirow}
\usepackage{lipsum}
\usepackage{balance}
\usepackage{calc}
\usepackage[colorlinks=true,citecolor=blue,linkcolor=blue]{hyperref}
\usepackage{import}
\usepackage{xifthen}
\usepackage{pdfpages}
\usepackage{transparent}
\usepackage{tabularx}

\newcommand{\incfig}[2][1.0]{
    \def\svgwidth{#1\columnwidth}
    \import{./figures/}{#2.pdf_tex}
}
\newcommand{\incimg}[2][1.0]{
  \includegraphics[width=#1\columnwidth]{./figures/#2}
}


\input{letterfonts}

\newcommand{\mytitle}{MA1522 Tutorial 10}
\newcommand{\myauthor}{github/omgeta}
\newcommand{\mydate}{AY 24/25 Sem 1}

\begin{document}
\raggedright
\footnotesize
\begin{center}
{\normalsize{\textbf{\mytitle}}} \\
{\footnotesize{\mydate\hspace{2pt}\textemdash\hspace{2pt}\myauthor}}
\end{center}
\setlist{topsep=-1em, itemsep=-1em, parsep=2em}

%%%%%%%%%%%%%%%%%%%%%%%%%%%%%%%%%%%%%%%%%%%%%%%%%%%%%%
%                      Begin                         %
%%%%%%%%%%%%%%%%%%%%%%%%%%%%%%%%%%%%%%%%%%%%%%%%%%%%%%
\begin{enumerate}[Q\arabic*.]
  \item 
    \begin{enumerate}[(\alph*)]
      \item $A = \left(\begin{array}{ccc} 0.4 & 0.2 & 0.4\\ 0.1 & 0.6 & 0.3\\ 0.5 & 0.2 & 0.3 \end{array}\right)$ which has each column adding up to $1$ and all values between $0$ and $1 \qed$

      \item $A = \left(\begin{array}{ccc} 1 & -1 & 4\\ 1 & 0 & -5\\ 1 & 1 & 1 \end{array}\right)\left(\begin{array}{ccc} 1 & 0 & 0\\ 0 & -\frac{1}{10} & 0\\ 0 & 0 & \frac{2}{5} \end{array}\right) \left(\begin{array}{ccc} 1 & -1 & 4\\ 1 & 0 & -5\\ 1 & 1 & 1 \end{array}\right)^{-1}$ so:
        \begin{align*}
          x_3 &= A^3x_0 = \left(\begin{array}{ccc} 1 & -1 & 4\\ 1 & 0 & -5\\ 1 & 1 & 1 \end{array}\right)\left(\begin{array}{ccc} 1^3 & 0 & 0\\ 0 & -\frac{1}{10}^3 & 0\\ 0 & 0 & \frac{2}{5}^3 \end{array}\right) \left(\begin{array}{ccc} 1 & -1 & 4\\ 1 & 0 & -5\\ 1 & 1 & 1 \end{array}\right)^{-1}\left(\begin{array}{c} 100\\ 0\\ 0 \end{array}\right)\\
              &= \left(\begin{array}{c} 35\\ 31.2\\ 33.8 \end{array}\right) \qed
        \end{align*}


      \item $A = \left(\begin{array}{ccc} 1 & -1 & 4\\ 1 & 0 & -5\\ 1 & 1 & 1 \end{array}\right)\left(\begin{array}{ccc} 1 & 0 & 0\\ 0 & -\frac{1}{10} & 0\\ 0 & 0 & \frac{2}{5} \end{array}\right) \left(\begin{array}{ccc} 1 & -1 & 4\\ 1 & 0 & -5\\ 1 & 1 & 1 \end{array}\right)^{-1}$ which is the same result as in (b)

      \item As $n\rightarrow\infty$, $D^n = \left(\begin{array}{ccc} 1^n & 0 & 0\\ 0 & -\frac{1}{10}^n & 0\\ 0 & 0 & \frac{2}{5}^n \end{array}\right) \rightarrow \left(\begin{array}{ccc} 1 & 0 & 0\\ 0 & 0 & 0\\ 0 & 0 & 0 \end{array}\right)$:
        \begin{align*}
          x_{\infty} &= \left(\begin{array}{ccc} 1 & -1 & 4\\ 1 & 0 & -5\\ 1 & 1 & 1 \end{array}\right)\left(\begin{array}{ccc} 1 & 0 & 0\\ 0 & 0 & 0\\ 0 & 0 & 0 \end{array}\right) \left(\begin{array}{ccc} 1 & -1 & 4\\ 1 & 0 & -5\\ 1 & 1 & 1 \end{array}\right)^{-1}\left(\begin{array}{c} 100\\ 0\\ 0 \end{array}\right)\\
              &= \left(\begin{array}{c} 33.3\\ 33.3\\ 33.3 \end{array}\right) \qed
        \end{align*}
        so in the long run all ants will be equally distributed $\qed$

      \item As $n\rightarrow\infty$, $D^n = \left(\begin{array}{ccc} 1^n & 0 & 0\\ 0 & -\frac{1}{10}^n & 0\\ 0 & 0 & \frac{2}{5}^n \end{array}\right) \rightarrow \left(\begin{array}{ccc} 1 & 0 & 0\\ 0 & 0 & 0\\ 0 & 0 & 0 \end{array}\right)$:
        \begin{align*}
          x_{\infty} &= \left(\begin{array}{ccc} 1 & -1 & 4\\ 1 & 0 & -5\\ 1 & 1 & 1 \end{array}\right)\left(\begin{array}{ccc} 1 & 0 & 0\\ 0 & 0 & 0\\ 0 & 0 & 0 \end{array}\right) \left(\begin{array}{ccc} 1 & -1 & 4\\ 1 & 0 & -5\\ 1 & 1 & 1 \end{array}\right)^{-1}\left(\begin{array}{c} \alpha\\ \beta\\ \gamma \end{array}\right)\\
              &= \frac{1}{3}\left(\begin{array}{c} \alpha + \beta + \gamma\\ \alpha + \beta + \gamma\\ \alpha + \beta + \gamma \end{array}\right) \qed
        \end{align*}
        which is always an equilibrium vector $\qed$
    \end{enumerate}

  \item $A =  \left(\begin{array}{ccc} 1 & 0 & 1\\ 0 & 1 & 0\\ 0 & 0 & 1 \end{array}\right) \left(\begin{array}{ccc} 1 & 0 & 0\\ 0 & 4 & 0\\ 0 & 0 & 4 \end{array}\right) \left(\begin{array}{ccc} 1 & 0 & 1\\ 0 & 1 & 0\\ 0 & 0 & 1 \end{array}\right)^{-1}$, so $B = \left(\begin{array}{ccc} 1 & 0 & 1\\ 0 & 1 & 0\\ 0 & 0 & 1 \end{array}\right) \left(\begin{array}{ccc} 1 & 0 & 0\\ 0 & 2 & 0\\ 0 & 0 & 2 \end{array}\right) \left(\begin{array}{ccc} 1 & 0 & 1\\ 0 & 1 & 0\\ 0 & 0 & 1 \end{array}\right)^{-1} \qed$
  \pagebreak

  \item 
    \begin{enumerate}[(\alph*)]
      \item $P = \left(\begin{array}{cc} -\frac{\sqrt{2}}{2} & \frac{\sqrt{2}}{2}\\ \frac{\sqrt{2}}{2} & \frac{\sqrt{2}}{2} \end{array}\right) \qed$

      \item $P = \left(\begin{array}{ccc} \frac{1}{3} & \frac{2\,\sqrt{5}}{5} & -\frac{2\,\sqrt{5}}{15}\\ -\frac{2}{3} & \frac{\sqrt{5}}{5} & \frac{4\,\sqrt{5}}{15}\\ \frac{2}{3} & 0 & \frac{\sqrt{5}}{3} \end{array}\right) \qed$
    \end{enumerate}

  \item 
    \begin{enumerate}[(\alph*)]
      \item $P = \left(\begin{array}{cccc} \frac{\sqrt{2}}{2} & 0 & -\frac{\sqrt{2}}{2} & 0\\ \frac{\sqrt{2}}{2} & 0 & \frac{\sqrt{2}}{2} & 0\\ 0 & \frac{\sqrt{2}}{2} & 0 & -\frac{\sqrt{2}}{2}\\ 0 & \frac{\sqrt{2}}{2} & 0 & \frac{\sqrt{2}}{2} \end{array}\right)$ and $P^TAP = D = \left(\begin{array}{cccc} -1 & 0 & 0 & 0\\ 0 & -1 & 0 & 0\\ 0 & 0 & 3 & 0\\ 0 & 0 & 0 & 3 \end{array}\right) \qed$

      \item Result is the same $\qed$
    \end{enumerate}

  \item 
    \begin{enumerate}[(\alph*)]
      \item $A = \left(\begin{array}{ccc} \frac{\sqrt{2}}{2} & \frac{\sqrt{2}}{6} & -\frac{2}{3}\\ \frac{\sqrt{2}}{2} & -\frac{\sqrt{2}}{6} & \frac{2}{3}\\ 0 & \frac{2\,\sqrt{2}}{3} & \frac{1}{3} \end{array}\right) \left(\begin{array}{cc} 5 & 0\\ 0 & 3\\ 0 & 0 \end{array}\right) \left(\begin{array}{cc} \frac{\sqrt{2}}{2} & \frac{\sqrt{2}}{2}\\ \frac{\sqrt{2}}{2} & -\frac{\sqrt{2}}{2} \end{array}\right) \qed$

      \item Since $A$ is the transpose of the matrix in (a), $A = \left(\begin{array}{cc} \frac{\sqrt{2}}{2} & \frac{\sqrt{2}}{2}\\ \frac{\sqrt{2}}{2} & -\frac{\sqrt{2}}{2} \end{array}\right) \left(\begin{array}{ccc} 5 & 0 & 0\\ 0 & 3 & 0 \end{array}\right) \left(\begin{array}{ccc} \frac{\sqrt{2}}{2} & \frac{\sqrt{2}}{2} & 0\\ \frac{\sqrt{2}}{6} & -\frac{\sqrt{2}}{6} & \frac{2\,\sqrt{2}}{3}\\ -\frac{2}{3} & \frac{2}{3} & \frac{1}{3} \end{array}\right) \qed$

      \item $A = \left(\begin{array}{ccc} -\frac{\sqrt{6}}{6} & \frac{\sqrt{2}}{2} & -\frac{\sqrt{3}}{3}\\ -\frac{\sqrt{6}}{6} & -\frac{\sqrt{2}}{2} & -\frac{\sqrt{3}}{3}\\ -\frac{\sqrt{2}\,\sqrt{3}}{3} & 0 & \frac{\sqrt{3}}{3} \end{array}\right) \left(\begin{array}{ccc} 3 & 0 & 0\\ 0 & 1 & 0\\ 0 & 0 & 0 \end{array}\right) \left(\begin{array}{ccc} -\frac{\sqrt{6}}{6} & -\frac{\sqrt{6}}{6} & -\frac{\sqrt{2}\,\sqrt{3}}{3}\\ \frac{\sqrt{2}}{2} & -\frac{\sqrt{2}}{2} & 0\\ -\frac{\sqrt{3}}{3} & -\frac{\sqrt{3}}{3} & \frac{\sqrt{3}}{3} \end{array}\right) \qed$ 
    \end{enumerate}

  \item 
    \begin{enumerate}[(\alph*)]
      \item $A = \left(\begin{array}{cccc} -\frac{1}{2} & \frac{1}{2} & \frac{1}{2} & -\frac{1}{2}\\ -\frac{1}{2} & -\frac{1}{2} & -\frac{1}{2} & -\frac{1}{2}\\ -\frac{1}{2} & \frac{1}{2} & -\frac{1}{2} & \frac{1}{2}\\ -\frac{1}{2} & -\frac{1}{2} & \frac{1}{2} & \frac{1}{2} \end{array}\right) \left(\begin{array}{cccc} 40 & 0 & 0 & 0\\ 0 & 20 & 0 & 0\\ 0 & 0 & 10 & 0\\ 0 & 0 & 0 & 0 \end{array}\right) \left(\begin{array}{cccc} \frac{2}{5} & -\frac{4}{5} & \frac{1}{5} & -\frac{2}{5}\\ -\frac{4}{5} & -\frac{2}{5} & -\frac{2}{5} & -\frac{1}{5}\\ -\frac{2}{5} & \frac{1}{5} & \frac{4}{5} & -\frac{2}{5}\\ \frac{1}{5} & \frac{2}{5} & -\frac{2}{5} & -\frac{4}{5} \end{array}\right) \qed$

      \item Result is the same $\qed$
    \end{enumerate}
\end{enumerate}
%%%%%%%%%%%%%%%%%%%%%%%%%%%%%%%%%%%%%%%%%%%%%%%%%%%%%%
%                       End                          %
%%%%%%%%%%%%%%%%%%%%%%%%%%%%%%%%%%%%%%%%%%%%%%%%%%%%%%

\end{document}

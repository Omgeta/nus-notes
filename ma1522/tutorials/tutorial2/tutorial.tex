\documentclass[12pt, a4paper]{article}

\usepackage[a4paper, margin=1in]{geometry}

\usepackage[utf8]{inputenc}
\usepackage[mathscr]{euscript}
\let\euscr\mathscr \let\mathscr\relax
\usepackage[scr]{rsfso}
\usepackage{amssymb,amsmath,amsthm,amsfonts}
\usepackage[shortlabels]{enumitem}
\usepackage{multicol,multirow}
\usepackage{lipsum}
\usepackage{balance}
\usepackage{calc}
\usepackage[colorlinks=true,citecolor=blue,linkcolor=blue]{hyperref}
\usepackage{import}
\usepackage{xifthen}
\usepackage{pdfpages}
\usepackage{transparent}
\usepackage{tabularx}

\newcommand{\incfig}[2][1.0]{
    \def\svgwidth{#1\columnwidth}
    \import{./figures/}{#2.pdf_tex}
}
\newcommand{\incimg}[2][1.0]{
  \includegraphics[width=#1\columnwidth]{./figures/#2}
}


\input{letterfonts}

\newcommand{\mytitle}{MA1522 Tutorial 2}
\newcommand{\myauthor}{github/omgeta}
\newcommand{\mydate}{AY 24/25 Sem 1}

\begin{document}
\raggedright
\footnotesize
\begin{center}
{\normalsize{\textbf{\mytitle}}} \\
{\footnotesize{\mydate\hspace{2pt}\textemdash\hspace{2pt}\myauthor}}
\end{center}
\setlist{topsep=-1em, itemsep=-1em, parsep=2em}

%%%%%%%%%%%%%%%%%%%%%%%%%%%%%%%%%%%%%%%%%%%%%%%%%%%%%%
%                      Begin                         %
%%%%%%%%%%%%%%%%%%%%%%%%%%%%%%%%%%%%%%%%%%%%%%%%%%%%%%
\begin{enumerate}[Q\arabic*.]
  \item
    \begin{enumerate}[(\alph*)]
      \item Suppose $B \vec{x} = \vec{0}$ has infinitely many solutions $\vec{u}$, then there are also infinitely many vectors $B\vec{u} = \vec{0}$ which when multiplied by matrix $A$ satisfy the homogenous equation $AB \vec{u} = \vec{0}$. 

        Therefore if $B\vec{x} = \vec{0}$ has infinitely many solutions, then $AB \vec{x} = \vec{0}$ also has infinitely many solutions. $\qed$
      \item Suppose $B \vec{x} = \vec{0}$ has only the trivial solution $\vec{x} = \vec{0}$. For example, let $B = \begin{bmatrix}1&0\\0&1\end{bmatrix}$. Consider two cases:
        \begin{enumerate}[(\roman*)]
          \item $A = \begin{bmatrix}1&0\\0&1\end{bmatrix}$. Then $AB \vec{x} = \vec{0}$ has only the trivial solution.
          \item $A = \begin{bmatrix}0&0\\0&0\end{bmatrix}$. Then $AB \vec{x} = \vec{0}$ has infinitely many solutions.
        \end{enumerate}

        Therefore, if $B \vec{x} = \vec{0}$ has only the trivial solution, it is not possible to know the number of solutions for $AB \vec{x} = \vec{0}. \qed$
    \end{enumerate}
  \item \begin{enumerate}[(\alph*)]
      \item To solve for $X$, reduce the augmented matrix $AX = I_3$:
      \begin{align*}
        \begin{bmatrix}1&1&0&1&1&0&0\\0&1&1&0&0&1&0\\0&0&1&1&0&0&1\end{bmatrix} &\sim \begin{bmatrix}1&0&0&2&1&-1&1\\0&1&0&-1&0&1&-1\\0&0&1&1&0&0&1\end{bmatrix} \\
        \therefore x_1 &= \begin{bmatrix}1\\0\\0\\0\end{bmatrix} + x_{14}\begin{bmatrix}-2\\1\\-1\\1\end{bmatrix} \\
        \therefore x_2 &= \begin{bmatrix}-1\\1\\0\\0\end{bmatrix} + x_{24}\begin{bmatrix}-2\\1\\-1\\1\end{bmatrix} \\
        \therefore x_3 &= \begin{bmatrix}1\\-1\\1\\0\end{bmatrix} + x_{34}\begin{bmatrix}-2\\1\\-1\\1\end{bmatrix} \\
      \end{align*}
      Therefore, a possible solution is $X = \begin{bmatrix}1&-1&1\\0&1&-1\\ 0&0&1\\ 0&0&0\end{bmatrix} \qed$

    \item Solve $B^TY^T = (YB)^T = I_3$ instead. Then $Y = (y_1, y_2, y_3)$ and $Y^T = \begin{bmatrix}y_1&y_2&y_3\end{bmatrix}$.
        \begin{align*}
          \begin{bmatrix}1&1&0&0&1&0&0\\0&1&1&0&0&1&0\\1&0&1&1&0&0&1\end{bmatrix} &\sim
          \begin{bmatrix}1&0&0&1/2&1/2&-1/2&1/2\\0&1&0&-1/2&1/2&1/2&-1/2\\0&0&1&1/2&-1/2&1/2&1/2\end{bmatrix}\\
          \therefore y_1 &= \begin{bmatrix}1/2\\1/2\\-1/2\\0\end{bmatrix} + s_1\begin{bmatrix}-1/2\\1/2\\-1/2\\1\end{bmatrix}\\
          \therefore y_2 &= \begin{bmatrix}-1/2\\1/2\\1/2\\0\end{bmatrix} + s_2\begin{bmatrix}-1/2\\1/2\\-1/2\\1\end{bmatrix}\\
          \therefore y_1 &= \begin{bmatrix}1/2\\-1/2\\1/2\\0\end{bmatrix} + s_1\begin{bmatrix}-1/2\\1/2\\-1/2\\1\end{bmatrix}\\
        \end{align*}
      Therefore, a possible solution is $Y = \frac{1}{2}\begin{bmatrix}1&1&-1&0\\-1&1&1&0\\1&-1&1&0\end{bmatrix} \qed$
    \end{enumerate}

  \item
    \begin{enumerate}[label=(a\roman*)]
      \item \begin{align*}
          A = \begin{bmatrix}5&-2&6&0\\-2&1&3&1\end{bmatrix} &\xrightarrow{R_1+2R_2}
          \begin{bmatrix}1&0&12&2\\-2&1&3&1\end{bmatrix} \\&\xrightarrow{R_2+2R_1} \begin{bmatrix}1&0&12&2\\0&1&27&5\end{bmatrix} = R \qed
        \end{align*}
      \item \begin{align*}
          E_1 &= \begin{bmatrix}1&2\\0&1\end{bmatrix}\tag*{($R_1+2R_2$)} \qed\\
          E_2 &= \begin{bmatrix}1&0\\2&1\end{bmatrix}\tag*{($R_2+2R_1$)} \qed\\
        \end{align*}
      \item \begin{align*}
          A &= E_1^{-1}E_2^{-1}R \\
            &= \begin{bmatrix}1&-2\\0&1\end{bmatrix} \begin{bmatrix}1&0\\-2&1\end{bmatrix}\begin{bmatrix}1&0&12&2\\0&1&27&5\end{bmatrix} \qed
        \end{align*}
    \end{enumerate}

    \begin{enumerate}[label=(b\roman*)]
      \item \begin{align*}
          A = \begin{bmatrix}-1&3&-4\\2&4&1\\-4&2&-9\end{bmatrix} &\xrightarrow{-R_1}
          \begin{bmatrix}1&-3&4\\2&4&1\\-4&2&-9\end{bmatrix} \\&\xrightarrow[R_3+4R_1]{R_2-2R_1} 
          \begin{bmatrix}1&-3&4\\0&10&-7\\0&-10&7\end{bmatrix}\\&\xrightarrow{\frac{1}{10}R_2}
          \begin{bmatrix}1&-3&4\\0&1&-7/10\\0&-10&7\end{bmatrix}\\&\xrightarrow[R_3+10R_2]{R_1+3R_2}
          \begin{bmatrix}1&0&19/10\\0&1&-7/10\\0&0&0\end{bmatrix} = R \qed
        \end{align*}
      \item \begin{align*}
          E_1 &= \begin{bmatrix}-1&0&0\\0&1&0\\0&0&1\end{bmatrix}\tag*{($-R_1$)} \qed\\
          E_2 &= \begin{bmatrix}1&0&0\\-2&1&0\\0&0&1\end{bmatrix}\tag*{($R_2-2R_1$)} \qed\\
          E_3 &= \begin{bmatrix}1&0&0\\0&1&0\\4&0&1\end{bmatrix}\tag*{($R_3+4R_1$)} \qed\\
          E_4 &= \begin{bmatrix}1&0&0\\0&1/10&0\\0&0&1\end{bmatrix}\tag*{($\frac{1}{10}R_2$)} \qed\\
          E_5 &= \begin{bmatrix}1&3&0\\0&1&0\\0&0&1\end{bmatrix}\tag*{($R_1+3R_2$)} \qed\\
          E_6 &= \begin{bmatrix}1&0&0\\0&1&0\\0&10&1\end{bmatrix}\tag*{($R_3+10R_2$)} \qed\\
        \end{align*}
      \item \begin{align*}
          A &= E_1^{-1}E_2^{-1}E_3^{-1}E_4^{-1}E_5^{-1}E_6^{-1}R\\
            &= \begin{bmatrix}-1&0&0\\0&1&0\\0&0&1\end{bmatrix} \begin{bmatrix}1&0&0\\2&1&0\\0&0&1\end{bmatrix}  \begin{bmatrix}1&0&0\\0&1&0\\-4&0&1\end{bmatrix}  \begin{bmatrix}1&0&0\\0&10&0\\0&0&1\end{bmatrix}  \begin{bmatrix}1&-3&0\\0&1&0\\0&0&1\end{bmatrix}  \begin{bmatrix}1&0&0\\0&1&0\\0&-10&1\end{bmatrix}  \begin{bmatrix}1&0&19/10\\0&1&-7/10\\0&0&0\end{bmatrix} \qed
        \end{align*}
    \end{enumerate}

    \begin{enumerate}[label=(c\roman*)]
      \item \begin{align*}
          A = \begin{bmatrix}1&-1&0\\2&-2&1\\1&2&3\end{bmatrix} &\xrightarrow[R_3-R_1]{R_2-2R_1}
          \begin{bmatrix}1&-1&0\\0&0&1\\0&3&3\end{bmatrix} \\&\xrightarrow{R_2\leftrightarrow R_3} \begin{bmatrix}1&-1&0\\0&3&3\\0&0&1\end{bmatrix}\\&\xrightarrow{\frac{1}{3}R_2}
          \begin{bmatrix}1&-1&0\\0&1&1\\0&0&1\end{bmatrix}\\&\xrightarrow{R_2-R_1}
          \begin{bmatrix}1&-1&0\\0&1&0\\0&0&1\end{bmatrix}\\&\xrightarrow{R_1+R_2}
          \begin{bmatrix}1&0&0\\0&1&0\\0&0&1\end{bmatrix} = R \qed
        \end{align*}
      \item \begin{align*}
          E_1 &= \begin{bmatrix}1&0&0\\-2&1&0\\0&0&1\end{bmatrix}\tag*{($R_2-2R_1$)} \qed\\
          E_2 &= \begin{bmatrix}1&0&0\\0&1&0\\-1&0&1\end{bmatrix}\tag*{($R_3-R_1$)} \qed\\
          E_3 &= \begin{bmatrix}1&0&0\\0&0&1\\0&1&0\end{bmatrix}\tag*{($R_2\leftrightarrow R_3$)} \qed\\
          E_4 &= \begin{bmatrix}1&0&0\\0&1/3&0\\0&0&1\end{bmatrix}\tag*{($\frac{1}{3}R_2$)} \qed\\
          E_5 &= \begin{bmatrix}1&0&0\\-1&1&0\\0&0&1\end{bmatrix}\tag*{($R_2-R_1$)} \qed\\
          E_6 &= \begin{bmatrix}1&1&0\\0&1&0\\0&0&1\end{bmatrix}\tag*{($R_1+R_2$)} \qed\\
        \end{align*}
      \item \begin{align*}
          A &= E_1E_2E_3E_4E_5E_6R \\
            &=  \begin{bmatrix}1&0&0\\2&1&0\\0&0&1\end{bmatrix}  \begin{bmatrix}1&0&0\\0&1&0\\1&0&1\end{bmatrix}  \begin{bmatrix}1&0&0\\0&0&1\\0&1&0\end{bmatrix}  \begin{bmatrix}1&0&0\\0&3&0\\0&0&1\end{bmatrix}  \begin{bmatrix}1&0&0\\1&1&0\\0&0&1\end{bmatrix}  \begin{bmatrix}1&-1&0\\0&1&0\\0&0&1\end{bmatrix} \qed
        \end{align*}
    \end{enumerate}
  \item
    \begin{enumerate}[(\alph*)]
      \item By reducing the matrix into RREF
        \begin{align*}
          \begin{bmatrix}
            -1 & 3 \\ 3 & -2
          \end{bmatrix} &\sim 
          \begin{bmatrix}
            1 & 0 \\ 0 & 1
          \end{bmatrix}
        \end{align*}
        We see that all of its columns are pivot columns. Therefore, it has an inverse.
        To find the inverse we augment the matrix with the identity matrix $I_2$, and reduce to RREF
        \begin{align*}
          \begin{bmatrix}
            -1 & 3 & 1 & 0 \\ 3 & -2 & 0 & 1
          \end{bmatrix} &\sim
          \begin{bmatrix}
            2/7 & 3/7 \\ 3/7 & 1/7
          \end{bmatrix}
        \end{align*}
        Therefore, the inverse is the matrix $\begin{bmatrix}
            2/7 & 3/7 \\ 3/7 & 1/7
            \end{bmatrix} = \displaystyle \frac{1}{7}\begin{bmatrix}2&3\\3&1\end{bmatrix}\qed$
          
      \item By reducing the matrix into RREF
        \begin{align*}
          \begin{bmatrix}
            -1 & 3 & -4 \\ 2 & 4 & 1 \\ -4 & 2 & -9
          \end{bmatrix} &\sim 
          \begin{bmatrix}
            1 & 0 & 19/10 \\ 0 & 1 & -7/10 \\ 0 & 0 & 0
          \end{bmatrix}
        \end{align*}
        We see that not all columns are pivot columns. Therefore, the inverse doesn't exist. $\qed$
    \end{enumerate}

  \item Reduce the matrix to REF 
    \begin{align*}
      \begin{bmatrix}
        1 & 1 & 1 \\ a & b & c \\ a^2 & b^2 & c^2
      \end{bmatrix} &\xrightarrow[R_3-aR_1]{R_2-aR_1}
      \begin{bmatrix}
        1 & 1 & 1 \\ 0 & b - a & c - a \\ 0 & b^2 - a^2 & c^2 - a^2
      \end{bmatrix} \\ &\xrightarrow{R_3-(b+a)R_2}
      \begin{bmatrix}
        1 & 1 & 1 \\ 0 & b-a & c-a \\ 0 & 0 & c^2 -a^2 - (c-a)(b+a)
      \end{bmatrix} \\ &\sim
      \begin{bmatrix}
        1 & 1 & 1 \\ 0 & b-a & c-a \\ 0 & 0 & (c-a)(c-b)
      \end{bmatrix}
    \end{align*}
    It can be seen that for the matrix to have 3 pivot columns, $a, b, c$ must be distinct. $\qed$

    \item
      \begin{enumerate}[(\alph*)]
        \item If the inverse of $I - A$ is $I + A$, then $(I - A)(I + A) = I$, law of inverse matrix multiplication.
          \begin{align*}
            (I-A)(I+A) &= I^2 + IA - AI - A^2 \\
                       &= I + A - A - A^2 \\
                       &= I \qed
          \end{align*}

        \item Suppose we have the matrix $B = (I - A)^-1$ where $B = 1 + A + A^2$. To prove that B is an inverse for $I - A$ we must show that $(I-A)B = I$.
          \begin{align*}
            (I-A)(1 + A + A^2) &= I + IA + IA^2 - A - A^2 - A^3 \\
                               &= I + A + A^2 - A - A^2 - A^3 \\
                               &= I \qed
          \end{align*}

        \item Suppose $A^n = 0$ and $B = (1-A)^-1$ where $B = 1 + A + \ldots + A^{n-1}$. To show that for any nilpotent $A$, $I - A$ has an inverse, we must show that $(I-A)B  = I$.
          \begin{align*}
            (I-A)(I + A + \ldots + A^{n-1}) &= (I + A + \ldots + A^{n-1}) - (A + A^2 + A^n) \\
                                            &= I - A^n \\
                                            &= I \qed
          \end{align*}
      \end{enumerate}
\end{enumerate}
%%%%%%%%%%%%%%%%%%%%%%%%%%%%%%%%%%%%%%%%%%%%%%%%%%%%%%
%                       End                          %
%%%%%%%%%%%%%%%%%%%%%%%%%%%%%%%%%%%%%%%%%%%%%%%%%%%%%%

\end{document}

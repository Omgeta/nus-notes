\documentclass[12pt, a4paper]{article}

\usepackage[a4paper, margin=1in]{geometry}

\usepackage[utf8]{inputenc}
\usepackage[mathscr]{euscript}
\let\euscr\mathscr \let\mathscr\relax
\usepackage[scr]{rsfso}
\usepackage{amssymb,amsmath,amsthm,amsfonts}
\usepackage[shortlabels]{enumitem}
\usepackage{multicol,multirow}
\usepackage{lipsum}
\usepackage{balance}
\usepackage{calc}
\usepackage[colorlinks=true,citecolor=blue,linkcolor=blue]{hyperref}
\usepackage{import}
\usepackage{xifthen}
\usepackage{pdfpages}
\usepackage{transparent}
\usepackage{listings}

\newcommand{\incfig}[2][1.0]{
    \def\svgwidth{#1\columnwidth}
    \import{./figures/}{#2.pdf_tex}
}

\newlist{enumproof}{enumerate}{4}
\setlist[enumproof,1]{label=\arabic*., parsep=1em}
\setlist[enumproof,2]{label=\arabic{enumproofi}.\arabic*., parsep=1em}
\setlist[enumproof,3]{label=\arabic{enumproofi}.\arabic{enumproofii}.\arabic*., parsep=1em}
\setlist[enumproof,4]{label=\arabic{enumproofi}.\arabic{enumproofii}.\arabic{enumproofiii}.\arabic*., parsep=1em}

\renewcommand{\qedsymbol}{\ensuremath{\blacksquare}}

\lstdefinestyle{mystyle}{
  language=C, % Set the language to C
  commentstyle=\color{codegray}, % Color for comments
  keywordstyle=\color{orange}, % Color for basic keywords
  stringstyle=\color{mauve}, % Color for strings
  basicstyle={\ttfamily\footnotesize}, % Basic font style
  breakatwhitespace=false,         
  breaklines=true,                 
  captionpos=b,                    
  keepspaces=true,                 
  numbers=none,                    
  tabsize=2,
  morekeywords=[2]{\#include, \#define, \#ifdef, \#ifndef, \#endif, \#pragma, \#else, \#elif}, % Preprocessor directives
  keywordstyle=[2]\color{codegreen}, % Style for preprocessor directives
  morekeywords=[3]{int, char, float, double, void, struct, union, enum, const, volatile, static, extern, register, inline, restrict, _Bool, _Complex, _Imaginary, size_t, ssize_t, FILE}, % C standard types and common identifiers
  keywordstyle=[3]\color{identblue}, % Style for types and common identifiers
  morekeywords=[4]{printf, scanf, fopen, fclose, malloc, free, calloc, realloc, perror, strtok, strncpy, strcpy, strcmp, strlen}, % Standard library functions
  keywordstyle=[4]\color{cyan}, % Style for library functions
}

% Things Lie
\newcommand{\kb}{\mathfrak b}
\newcommand{\kg}{\mathfrak g}
\newcommand{\kh}{\mathfrak h}
\newcommand{\kn}{\mathfrak n}
\newcommand{\ku}{\mathfrak u}
\newcommand{\kz}{\mathfrak z}
\DeclareMathOperator{\Ext}{Ext} % Ext functor
\DeclareMathOperator{\Tor}{Tor} % Tor functor
\newcommand{\gl}{\opname{\mathfrak{gl}}} % frak gl group
\renewcommand{\sl}{\opname{\mathfrak{sl}}} % frak sl group chktex 6

% More script letters etc.
\newcommand{\SA}{\mathcal A}
\newcommand{\SB}{\mathcal B}
\newcommand{\SC}{\mathcal C}
\newcommand{\SF}{\mathcal F}
\newcommand{\SG}{\mathcal G}
\newcommand{\SH}{\mathcal H}
\newcommand{\OO}{\mathcal O}

\newcommand{\SCA}{\mathscr A}
\newcommand{\SCB}{\mathscr B}
\newcommand{\SCC}{\mathscr C}
\newcommand{\SCD}{\mathscr D}
\newcommand{\SCE}{\mathscr E}
\newcommand{\SCF}{\mathscr F}
\newcommand{\SCG}{\mathscr G}
\newcommand{\SCH}{\mathscr H}

% Mathfrak primes
\newcommand{\km}{\mathfrak m}
\newcommand{\kp}{\mathfrak p}
\newcommand{\kq}{\mathfrak q}

% number sets
\newcommand{\RR}[1][]{\ensuremath{\ifstrempty{#1}{\mathbb{R}}{\mathbb{R}^{#1}}}}
\newcommand{\NN}[1][]{\ensuremath{\ifstrempty{#1}{\mathbb{N}}{\mathbb{N}^{#1}}}}
\newcommand{\ZZ}[1][]{\ensuremath{\ifstrempty{#1}{\mathbb{Z}}{\mathbb{Z}^{#1}}}}
\newcommand{\QQ}[1][]{\ensuremath{\ifstrempty{#1}{\mathbb{Q}}{\mathbb{Q}^{#1}}}}
\newcommand{\CC}[1][]{\ensuremath{\ifstrempty{#1}{\mathbb{C}}{\mathbb{C}^{#1}}}}
\newcommand{\PP}[1][]{\ensuremath{\ifstrempty{#1}{\mathbb{P}}{\mathbb{P}^{#1}}}}
\newcommand{\HH}[1][]{\ensuremath{\ifstrempty{#1}{\mathbb{H}}{\mathbb{H}^{#1}}}}
\newcommand{\FF}[1][]{\ensuremath{\ifstrempty{#1}{\mathbb{F}}{\mathbb{F}^{#1}}}}
% expected value
\newcommand{\EE}{\ensuremath{\mathbb{E}}}
\newcommand{\charin}{\text{ char }}
\DeclareMathOperator{\sign}{sign}
\DeclareMathOperator{\Aut}{Aut}
\DeclareMathOperator{\Inn}{Inn}
\DeclareMathOperator{\Syl}{Syl}
\DeclareMathOperator{\Gal}{Gal}
\DeclareMathOperator{\GL}{GL} % General linear group
\DeclareMathOperator{\SL}{SL} % Special linear group

%---------------------------------------
% BlackBoard Math Fonts :-
%---------------------------------------

%Captital Letters
\newcommand{\bbA}{\mathbb{A}}	\newcommand{\bbB}{\mathbb{B}}
\newcommand{\bbC}{\mathbb{C}}	\newcommand{\bbD}{\mathbb{D}}
\newcommand{\bbE}{\mathbb{E}}	\newcommand{\bbF}{\mathbb{F}}
\newcommand{\bbG}{\mathbb{G}}	\newcommand{\bbH}{\mathbb{H}}
\newcommand{\bbI}{\mathbb{I}}	\newcommand{\bbJ}{\mathbb{J}}
\newcommand{\bbK}{\mathbb{K}}	\newcommand{\bbL}{\mathbb{L}}
\newcommand{\bbM}{\mathbb{M}}	\newcommand{\bbN}{\mathbb{N}}
\newcommand{\bbO}{\mathbb{O}}	\newcommand{\bbP}{\mathbb{P}}
\newcommand{\bbQ}{\mathbb{Q}}	\newcommand{\bbR}{\mathbb{R}}
\newcommand{\bbS}{\mathbb{S}}	\newcommand{\bbT}{\mathbb{T}}
\newcommand{\bbU}{\mathbb{U}}	\newcommand{\bbV}{\mathbb{V}}
\newcommand{\bbW}{\mathbb{W}}	\newcommand{\bbX}{\mathbb{X}}
\newcommand{\bbY}{\mathbb{Y}}	\newcommand{\bbZ}{\mathbb{Z}}

%---------------------------------------
% MathCal Fonts :-
%---------------------------------------

%Captital Letters
\newcommand{\mcA}{\mathcal{A}}	\newcommand{\mcB}{\mathcal{B}}
\newcommand{\mcC}{\mathcal{C}}	\newcommand{\mcD}{\mathcal{D}}
\newcommand{\mcE}{\mathcal{E}}	\newcommand{\mcF}{\mathcal{F}}
\newcommand{\mcG}{\mathcal{G}}	\newcommand{\mcH}{\mathcal{H}}
\newcommand{\mcI}{\mathcal{I}}	\newcommand{\mcJ}{\mathcal{J}}
\newcommand{\mcK}{\mathcal{K}}	\newcommand{\mcL}{\mathcal{L}}
\newcommand{\mcM}{\mathcal{M}}	\newcommand{\mcN}{\mathcal{N}}
\newcommand{\mcO}{\mathcal{O}}	\newcommand{\mcP}{\mathcal{P}}
\newcommand{\mcQ}{\mathcal{Q}}	\newcommand{\mcR}{\mathcal{R}}
\newcommand{\mcS}{\mathcal{S}}	\newcommand{\mcT}{\mathcal{T}}
\newcommand{\mcU}{\mathcal{U}}	\newcommand{\mcV}{\mathcal{V}}
\newcommand{\mcW}{\mathcal{W}}	\newcommand{\mcX}{\mathcal{X}}
\newcommand{\mcY}{\mathcal{Y}}	\newcommand{\mcZ}{\mathcal{Z}}

%---------------------------------------
% Bold Math Fonts :-
%---------------------------------------

%Captital Letters
\newcommand{\bmA}{\boldsymbol{A}}	\newcommand{\bmB}{\boldsymbol{B}}
\newcommand{\bmC}{\boldsymbol{C}}	\newcommand{\bmD}{\boldsymbol{D}}
\newcommand{\bmE}{\boldsymbol{E}}	\newcommand{\bmF}{\boldsymbol{F}}
\newcommand{\bmG}{\boldsymbol{G}}	\newcommand{\bmH}{\boldsymbol{H}}
\newcommand{\bmI}{\boldsymbol{I}}	\newcommand{\bmJ}{\boldsymbol{J}}
\newcommand{\bmK}{\boldsymbol{K}}	\newcommand{\bmL}{\boldsymbol{L}}
\newcommand{\bmM}{\boldsymbol{M}}	\newcommand{\bmN}{\boldsymbol{N}}
\newcommand{\bmO}{\boldsymbol{O}}	\newcommand{\bmP}{\boldsymbol{P}}
\newcommand{\bmQ}{\boldsymbol{Q}}	\newcommand{\bmR}{\boldsymbol{R}}
\newcommand{\bmS}{\boldsymbol{S}}	\newcommand{\bmT}{\boldsymbol{T}}
\newcommand{\bmU}{\boldsymbol{U}}	\newcommand{\bmV}{\boldsymbol{V}}
\newcommand{\bmW}{\boldsymbol{W}}	\newcommand{\bmX}{\boldsymbol{X}}
\newcommand{\bmY}{\boldsymbol{Y}}	\newcommand{\bmZ}{\boldsymbol{Z}}
%Small Letters
\newcommand{\bma}{\boldsymbol{a}}	\newcommand{\bmb}{\boldsymbol{b}}
\newcommand{\bmc}{\boldsymbol{c}}	\newcommand{\bmd}{\boldsymbol{d}}
\newcommand{\bme}{\boldsymbol{e}}	\newcommand{\bmf}{\boldsymbol{f}}
\newcommand{\bmg}{\boldsymbol{g}}	\newcommand{\bmh}{\boldsymbol{h}}
\newcommand{\bmi}{\boldsymbol{i}}	\newcommand{\bmj}{\boldsymbol{j}}
\newcommand{\bmk}{\boldsymbol{k}}	\newcommand{\bml}{\boldsymbol{l}}
\newcommand{\bmm}{\boldsymbol{m}}	\newcommand{\bmn}{\boldsymbol{n}}
\newcommand{\bmo}{\boldsymbol{o}}	\newcommand{\bmp}{\boldsymbol{p}}
\newcommand{\bmq}{\boldsymbol{q}}	\newcommand{\bmr}{\boldsymbol{r}}
\newcommand{\bms}{\boldsymbol{s}}	\newcommand{\bmt}{\boldsymbol{t}}
\newcommand{\bmu}{\boldsymbol{u}}	\newcommand{\bmv}{\boldsymbol{v}}
\newcommand{\bmw}{\boldsymbol{w}}	\newcommand{\bmx}{\boldsymbol{x}}
\newcommand{\bmy}{\boldsymbol{y}}	\newcommand{\bmz}{\boldsymbol{z}}

%---------------------------------------
% Scr Math Fonts :-
%---------------------------------------

\newcommand{\sA}{{\mathscr{A}}}   \newcommand{\sB}{{\mathscr{B}}}
\newcommand{\sC}{{\mathscr{C}}}   \newcommand{\sD}{{\mathscr{D}}}
\newcommand{\sE}{{\mathscr{E}}}   \newcommand{\sF}{{\mathscr{F}}}
\newcommand{\sG}{{\mathscr{G}}}   \newcommand{\sH}{{\mathscr{H}}}
\newcommand{\sI}{{\mathscr{I}}}   \newcommand{\sJ}{{\mathscr{J}}}
\newcommand{\sK}{{\mathscr{K}}}   \newcommand{\sL}{{\mathscr{L}}}
\newcommand{\sM}{{\mathscr{M}}}   \newcommand{\sN}{{\mathscr{N}}}
\newcommand{\sO}{{\mathscr{O}}}   \newcommand{\sP}{{\mathscr{P}}}
\newcommand{\sQ}{{\mathscr{Q}}}   \newcommand{\sR}{{\mathscr{R}}}
\newcommand{\sS}{{\mathscr{S}}}   \newcommand{\sT}{{\mathscr{T}}}
\newcommand{\sU}{{\mathscr{U}}}   \newcommand{\sV}{{\mathscr{V}}}
\newcommand{\sW}{{\mathscr{W}}}   \newcommand{\sX}{{\mathscr{X}}}
\newcommand{\sY}{{\mathscr{Y}}}   \newcommand{\sZ}{{\mathscr{Z}}}


%---------------------------------------
% Math Fraktur Font
%---------------------------------------

%Captital Letters
\newcommand{\mfA}{\mathfrak{A}}	\newcommand{\mfB}{\mathfrak{B}}
\newcommand{\mfC}{\mathfrak{C}}	\newcommand{\mfD}{\mathfrak{D}}
\newcommand{\mfE}{\mathfrak{E}}	\newcommand{\mfF}{\mathfrak{F}}
\newcommand{\mfG}{\mathfrak{G}}	\newcommand{\mfH}{\mathfrak{H}}
\newcommand{\mfI}{\mathfrak{I}}	\newcommand{\mfJ}{\mathfrak{J}}
\newcommand{\mfK}{\mathfrak{K}}	\newcommand{\mfL}{\mathfrak{L}}
\newcommand{\mfM}{\mathfrak{M}}	\newcommand{\mfN}{\mathfrak{N}}
\newcommand{\mfO}{\mathfrak{O}}	\newcommand{\mfP}{\mathfrak{P}}
\newcommand{\mfQ}{\mathfrak{Q}}	\newcommand{\mfR}{\mathfrak{R}}
\newcommand{\mfS}{\mathfrak{S}}	\newcommand{\mfT}{\mathfrak{T}}
\newcommand{\mfU}{\mathfrak{U}}	\newcommand{\mfV}{\mathfrak{V}}
\newcommand{\mfW}{\mathfrak{W}}	\newcommand{\mfX}{\mathfrak{X}}
\newcommand{\mfY}{\mathfrak{Y}}	\newcommand{\mfZ}{\mathfrak{Z}}
%Small Letters
\newcommand{\mfa}{\mathfrak{a}}	\newcommand{\mfb}{\mathfrak{b}}
\newcommand{\mfc}{\mathfrak{c}}	\newcommand{\mfd}{\mathfrak{d}}
\newcommand{\mfe}{\mathfrak{e}}	\newcommand{\mff}{\mathfrak{f}}
\newcommand{\mfg}{\mathfrak{g}}	\newcommand{\mfh}{\mathfrak{h}}
\newcommand{\mfi}{\mathfrak{i}}	\newcommand{\mfj}{\mathfrak{j}}
\newcommand{\mfk}{\mathfrak{k}}	\newcommand{\mfl}{\mathfrak{l}}
\newcommand{\mfm}{\mathfrak{m}}	\newcommand{\mfn}{\mathfrak{n}}
\newcommand{\mfo}{\mathfrak{o}}	\newcommand{\mfp}{\mathfrak{p}}
\newcommand{\mfq}{\mathfrak{q}}	\newcommand{\mfr}{\mathfrak{r}}
\newcommand{\mfs}{\mathfrak{s}}	\newcommand{\mft}{\mathfrak{t}}
\newcommand{\mfu}{\mathfrak{u}}	\newcommand{\mfv}{\mathfrak{v}}
\newcommand{\mfw}{\mathfrak{w}}	\newcommand{\mfx}{\mathfrak{x}}
\newcommand{\mfy}{\mathfrak{y}}	\newcommand{\mfz}{\mathfrak{z}}


\newcommand{\mytitle}{MA1522 Tutorial 2}
\newcommand{\myauthor}{github/omgeta}
\newcommand{\mydate}{AY 24/25 Sem 1}

\begin{document}
\raggedright
\footnotesize
\begin{center}
{\normalsize{\textbf{\mytitle}}} \\
{\footnotesize{\mydate\hspace{2pt}\textemdash\hspace{2pt}\myauthor}}
\end{center}
\setlist{topsep=-1em, itemsep=-1em, parsep=2em}

%%%%%%%%%%%%%%%%%%%%%%%%%%%%%%%%%%%%%%%%%%%%%%%%%%%%%%
%                      Begin                         %
%%%%%%%%%%%%%%%%%%%%%%%%%%%%%%%%%%%%%%%%%%%%%%%%%%%%%%
\begin{enumerate}[Q\arabic*.]
  \item
    \begin{enumerate}[(\alph*)]
      \item Suppose $B \vec{x} = \vec{0}$ has infinitely many solutions $\vec{u}$, then there are also infinitely many vectors $B\vec{u} = \vec{0}$ which when multiplied by matrix $A$ satisfy the homogenous equation $AB \vec{u} = \vec{0}$. 

        Therefore if $B\vec{x} = \vec{0}$ has infinitely many solutions, then $AB \vec{x} = \vec{0}$ also has infinitely many solutions. $\qed$
      \item Suppose $B \vec{x} = \vec{0}$ has only the trivial solution $\vec{x} = \vec{0}$. For example, let $B = \begin{bmatrix}1&0\\0&1\end{bmatrix}$. Consider two cases:
        \begin{enumerate}[(\roman*)]
          \item $A = \begin{bmatrix}1&0\\0&1\end{bmatrix}$. Then $AB \vec{x} = \vec{0}$ has only the trivial solution.
          \item $A = \begin{bmatrix}0&0\\0&0\end{bmatrix}$. Then $AB \vec{x} = \vec{0}$ has infinitely many solutions.
        \end{enumerate}

        Therefore, if $B \vec{x} = \vec{0}$ has only the trivial solution, it is not possible to know the number of solutions for $AB \vec{x} = \vec{0}. \qed$
    \end{enumerate}
  \item \begin{enumerate}[(\alph*)]
      \item To solve for $X$, reduce the augmented matrix $AX = I_3$:
      \begin{align*}
        \begin{bmatrix}1&1&0&1&1&0&0\\0&1&1&0&0&1&0\\0&0&1&1&0&0&1\end{bmatrix} &\sim \begin{bmatrix}1&0&0&2&1&-1&1\\0&1&0&-1&0&1&-1\\0&0&1&1&0&0&1\end{bmatrix} \\
        \therefore x_1 &= \begin{bmatrix}1\\0\\0\\0\end{bmatrix} + x_{14}\begin{bmatrix}-2\\1\\-1\\1\end{bmatrix} \\
        \therefore x_2 &= \begin{bmatrix}-1\\1\\0\\0\end{bmatrix} + x_{24}\begin{bmatrix}-2\\1\\-1\\1\end{bmatrix} \\
        \therefore x_3 &= \begin{bmatrix}1\\-1\\1\\0\end{bmatrix} + x_{34}\begin{bmatrix}-2\\1\\-1\\1\end{bmatrix} \\
      \end{align*}
      Therefore, a possible solution is $X = \begin{bmatrix}1&-1&1\\0&1&-1\\ 0&0&1\\ 0&0&0\end{bmatrix} \qed$

    \item Solve $B^TY^T = (YB)^T = I_3$ instead. Then $Y = (y_1, y_2, y_3)$ and $Y^T = \begin{bmatrix}y_1&y_2&y_3\end{bmatrix}$.
        \begin{align*}
          \begin{bmatrix}1&1&0&0&1&0&0\\0&1&1&0&0&1&0\\1&0&1&1&0&0&1\end{bmatrix} &\sim
          \begin{bmatrix}1&0&0&1/2&1/2&-1/2&1/2\\0&1&0&-1/2&1/2&1/2&-1/2\\0&0&1&1/2&-1/2&1/2&1/2\end{bmatrix}\\
          \therefore y_1 &= \begin{bmatrix}1/2\\1/2\\-1/2\\0\end{bmatrix} + s_1\begin{bmatrix}-1/2\\1/2\\-1/2\\1\end{bmatrix}\\
          \therefore y_2 &= \begin{bmatrix}-1/2\\1/2\\1/2\\0\end{bmatrix} + s_2\begin{bmatrix}-1/2\\1/2\\-1/2\\1\end{bmatrix}\\
          \therefore y_1 &= \begin{bmatrix}1/2\\-1/2\\1/2\\0\end{bmatrix} + s_1\begin{bmatrix}-1/2\\1/2\\-1/2\\1\end{bmatrix}\\
        \end{align*}
      Therefore, a possible solution is $Y = \frac{1}{2}\begin{bmatrix}1&1&-1&0\\-1&1&1&0\\1&-1&1&0\end{bmatrix} \qed$
    \end{enumerate}

  \item
    \begin{enumerate}[label=(a\roman*)]
      \item \begin{align*}
          A = \begin{bmatrix}5&-2&6&0\\-2&1&3&1\end{bmatrix} &\xrightarrow{R_1+2R_2}
          \begin{bmatrix}1&0&12&2\\-2&1&3&1\end{bmatrix} \\&\xrightarrow{R_2+2R_1} \begin{bmatrix}1&0&12&2\\0&1&27&5\end{bmatrix} = R \qed
        \end{align*}
      \item \begin{align*}
          E_1 &= \begin{bmatrix}1&2\\0&1\end{bmatrix}\tag*{($R_1+2R_2$)} \qed\\
          E_2 &= \begin{bmatrix}1&0\\2&1\end{bmatrix}\tag*{($R_2+2R_1$)} \qed\\
        \end{align*}
      \item \begin{align*}
          A &= E_1^{-1}E_2^{-1}R \\
            &= \begin{bmatrix}1&-2\\0&1\end{bmatrix} \begin{bmatrix}1&0\\-2&1\end{bmatrix}\begin{bmatrix}1&0&12&2\\0&1&27&5\end{bmatrix} \qed
        \end{align*}
    \end{enumerate}

    \begin{enumerate}[label=(b\roman*)]
      \item \begin{align*}
          A = \begin{bmatrix}-1&3&-4\\2&4&1\\-4&2&-9\end{bmatrix} &\xrightarrow{-R_1}
          \begin{bmatrix}1&-3&4\\2&4&1\\-4&2&-9\end{bmatrix} \\&\xrightarrow[R_3+4R_1]{R_2-2R_1} 
          \begin{bmatrix}1&-3&4\\0&10&-7\\0&-10&7\end{bmatrix}\\&\xrightarrow{\frac{1}{10}R_2}
          \begin{bmatrix}1&-3&4\\0&1&-7/10\\0&-10&7\end{bmatrix}\\&\xrightarrow[R_3+10R_2]{R_1+3R_2}
          \begin{bmatrix}1&0&19/10\\0&1&-7/10\\0&0&0\end{bmatrix} = R \qed
        \end{align*}
      \item \begin{align*}
          E_1 &= \begin{bmatrix}-1&0&0\\0&1&0\\0&0&1\end{bmatrix}\tag*{($-R_1$)} \qed\\
          E_2 &= \begin{bmatrix}1&0&0\\-2&1&0\\0&0&1\end{bmatrix}\tag*{($R_2-2R_1$)} \qed\\
          E_3 &= \begin{bmatrix}1&0&0\\0&1&0\\4&0&1\end{bmatrix}\tag*{($R_3+4R_1$)} \qed\\
          E_4 &= \begin{bmatrix}1&0&0\\0&1/10&0\\0&0&1\end{bmatrix}\tag*{($\frac{1}{10}R_2$)} \qed\\
          E_5 &= \begin{bmatrix}1&3&0\\0&1&0\\0&0&1\end{bmatrix}\tag*{($R_1+3R_2$)} \qed\\
          E_6 &= \begin{bmatrix}1&0&0\\0&1&0\\0&10&1\end{bmatrix}\tag*{($R_3+10R_2$)} \qed\\
        \end{align*}
      \item \begin{align*}
          A &= E_1^{-1}E_2^{-1}E_3^{-1}E_4^{-1}E_5^{-1}E_6^{-1}R\\
            &= \begin{bmatrix}-1&0&0\\0&1&0\\0&0&1\end{bmatrix} \begin{bmatrix}1&0&0\\2&1&0\\0&0&1\end{bmatrix}  \begin{bmatrix}1&0&0\\0&1&0\\-4&0&1\end{bmatrix}  \begin{bmatrix}1&0&0\\0&10&0\\0&0&1\end{bmatrix}  \begin{bmatrix}1&-3&0\\0&1&0\\0&0&1\end{bmatrix}  \begin{bmatrix}1&0&0\\0&1&0\\0&-10&1\end{bmatrix}  \begin{bmatrix}1&0&19/10\\0&1&-7/10\\0&0&0\end{bmatrix} \qed
        \end{align*}
    \end{enumerate}

    \begin{enumerate}[label=(c\roman*)]
      \item \begin{align*}
          A = \begin{bmatrix}1&-1&0\\2&-2&1\\1&2&3\end{bmatrix} &\xrightarrow[R_3-R_1]{R_2-2R_1}
          \begin{bmatrix}1&-1&0\\0&0&1\\0&3&3\end{bmatrix} \\&\xrightarrow{R_2\leftrightarrow R_3} \begin{bmatrix}1&-1&0\\0&3&3\\0&0&1\end{bmatrix}\\&\xrightarrow{\frac{1}{3}R_2}
          \begin{bmatrix}1&-1&0\\0&1&1\\0&0&1\end{bmatrix}\\&\xrightarrow{R_2-R_1}
          \begin{bmatrix}1&-1&0\\0&1&0\\0&0&1\end{bmatrix}\\&\xrightarrow{R_1+R_2}
          \begin{bmatrix}1&0&0\\0&1&0\\0&0&1\end{bmatrix} = R \qed
        \end{align*}
      \item \begin{align*}
          E_1 &= \begin{bmatrix}1&0&0\\-2&1&0\\0&0&1\end{bmatrix}\tag*{($R_2-2R_1$)} \qed\\
          E_2 &= \begin{bmatrix}1&0&0\\0&1&0\\-1&0&1\end{bmatrix}\tag*{($R_3-R_1$)} \qed\\
          E_3 &= \begin{bmatrix}1&0&0\\0&0&1\\0&1&0\end{bmatrix}\tag*{($R_2\leftrightarrow R_3$)} \qed\\
          E_4 &= \begin{bmatrix}1&0&0\\0&1/3&0\\0&0&1\end{bmatrix}\tag*{($\frac{1}{3}R_2$)} \qed\\
          E_5 &= \begin{bmatrix}1&0&0\\-1&1&0\\0&0&1\end{bmatrix}\tag*{($R_2-R_1$)} \qed\\
          E_6 &= \begin{bmatrix}1&1&0\\0&1&0\\0&0&1\end{bmatrix}\tag*{($R_1+R_2$)} \qed\\
        \end{align*}
      \item \begin{align*}
          A &= E_1E_2E_3E_4E_5E_6R \\
            &=  \begin{bmatrix}1&0&0\\2&1&0\\0&0&1\end{bmatrix}  \begin{bmatrix}1&0&0\\0&1&0\\1&0&1\end{bmatrix}  \begin{bmatrix}1&0&0\\0&0&1\\0&1&0\end{bmatrix}  \begin{bmatrix}1&0&0\\0&3&0\\0&0&1\end{bmatrix}  \begin{bmatrix}1&0&0\\1&1&0\\0&0&1\end{bmatrix}  \begin{bmatrix}1&-1&0\\0&1&0\\0&0&1\end{bmatrix} \qed
        \end{align*}
    \end{enumerate}
  \item
    \begin{enumerate}[(\alph*)]
      \item By reducing the matrix into RREF
        \begin{align*}
          \begin{bmatrix}
            -1 & 3 \\ 3 & -2
          \end{bmatrix} &\sim 
          \begin{bmatrix}
            1 & 0 \\ 0 & 1
          \end{bmatrix}
        \end{align*}
        We see that all of its columns are pivot columns. Therefore, it has an inverse.
        To find the inverse we augment the matrix with the identity matrix $I_2$, and reduce to RREF
        \begin{align*}
          \begin{bmatrix}
            -1 & 3 & 1 & 0 \\ 3 & -2 & 0 & 1
          \end{bmatrix} &\sim
          \begin{bmatrix}
            2/7 & 3/7 \\ 3/7 & 1/7
          \end{bmatrix}
        \end{align*}
        Therefore, the inverse is the matrix $\begin{bmatrix}
            2/7 & 3/7 \\ 3/7 & 1/7
            \end{bmatrix} = \displaystyle \frac{1}{7}\begin{bmatrix}2&3\\3&1\end{bmatrix}\qed$
          
      \item By reducing the matrix into RREF
        \begin{align*}
          \begin{bmatrix}
            -1 & 3 & -4 \\ 2 & 4 & 1 \\ -4 & 2 & -9
          \end{bmatrix} &\sim 
          \begin{bmatrix}
            1 & 0 & 19/10 \\ 0 & 1 & -7/10 \\ 0 & 0 & 0
          \end{bmatrix}
        \end{align*}
        We see that not all columns are pivot columns. Therefore, the inverse doesn't exist. $\qed$
    \end{enumerate}

  \item Reduce the matrix to REF 
    \begin{align*}
      \begin{bmatrix}
        1 & 1 & 1 \\ a & b & c \\ a^2 & b^2 & c^2
      \end{bmatrix} &\xrightarrow[R_3-aR_1]{R_2-aR_1}
      \begin{bmatrix}
        1 & 1 & 1 \\ 0 & b - a & c - a \\ 0 & b^2 - a^2 & c^2 - a^2
      \end{bmatrix} \\ &\xrightarrow{R_3-(b+a)R_2}
      \begin{bmatrix}
        1 & 1 & 1 \\ 0 & b-a & c-a \\ 0 & 0 & c^2 -a^2 - (c-a)(b+a)
      \end{bmatrix} \\ &\sim
      \begin{bmatrix}
        1 & 1 & 1 \\ 0 & b-a & c-a \\ 0 & 0 & (c-a)(c-b)
      \end{bmatrix}
    \end{align*}
    It can be seen that for the matrix to have 3 pivot columns, $a, b, c$ must be distinct. $\qed$

    \item
      \begin{enumerate}[(\alph*)]
        \item If the inverse of $I - A$ is $I + A$, then $(I - A)(I + A) = I$, law of inverse matrix multiplication.
          \begin{align*}
            (I-A)(I+A) &= I^2 + IA - AI - A^2 \\
                       &= I + A - A - A^2 \\
                       &= I \qed
          \end{align*}

        \item Suppose we have the matrix $B = (I - A)^-1$ where $B = 1 + A + A^2$. To prove that B is an inverse for $I - A$ we must show that $(I-A)B = I$.
          \begin{align*}
            (I-A)(1 + A + A^2) &= I + IA + IA^2 - A - A^2 - A^3 \\
                               &= I + A + A^2 - A - A^2 - A^3 \\
                               &= I \qed
          \end{align*}

        \item Suppose $A^n = 0$ and $B = (1-A)^-1$ where $B = 1 + A + \ldots + A^{n-1}$. To show that for any nilpotent $A$, $I - A$ has an inverse, we must show that $(I-A)B  = I$.
          \begin{align*}
            (I-A)(I + A + \ldots + A^{n-1}) &= (I + A + \ldots + A^{n-1}) - (A + A^2 + A^n) \\
                                            &= I - A^n \\
                                            &= I \qed
          \end{align*}
      \end{enumerate}
\end{enumerate}
%%%%%%%%%%%%%%%%%%%%%%%%%%%%%%%%%%%%%%%%%%%%%%%%%%%%%%
%                       End                          %
%%%%%%%%%%%%%%%%%%%%%%%%%%%%%%%%%%%%%%%%%%%%%%%%%%%%%%

\end{document}

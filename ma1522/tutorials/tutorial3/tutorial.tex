\documentclass[12pt, a4paper]{article}

\usepackage[a4paper, margin=1in]{geometry}

\usepackage[utf8]{inputenc}
\usepackage[mathscr]{euscript}
\let\euscr\mathscr \let\mathscr\relax
\usepackage[scr]{rsfso}
\usepackage{amssymb,amsmath,amsthm,amsfonts}
\usepackage[shortlabels]{enumitem}
\usepackage{multicol,multirow}
\usepackage{lipsum}
\usepackage{balance}
\usepackage{calc}
\usepackage[colorlinks=true,citecolor=blue,linkcolor=blue]{hyperref}
\usepackage{import}
\usepackage{xifthen}
\usepackage{pdfpages}
\usepackage{transparent}
\usepackage{listings}

\newcommand{\incfig}[2][1.0]{
    \def\svgwidth{#1\columnwidth}
    \import{./figures/}{#2.pdf_tex}
}

\newlist{enumproof}{enumerate}{4}
\setlist[enumproof,1]{label=\arabic*., parsep=1em}
\setlist[enumproof,2]{label=\arabic{enumproofi}.\arabic*., parsep=1em}
\setlist[enumproof,3]{label=\arabic{enumproofi}.\arabic{enumproofii}.\arabic*., parsep=1em}
\setlist[enumproof,4]{label=\arabic{enumproofi}.\arabic{enumproofii}.\arabic{enumproofiii}.\arabic*., parsep=1em}

\renewcommand{\qedsymbol}{\ensuremath{\blacksquare}}

\lstdefinestyle{mystyle}{
  language=C, % Set the language to C
  commentstyle=\color{codegray}, % Color for comments
  keywordstyle=\color{orange}, % Color for basic keywords
  stringstyle=\color{mauve}, % Color for strings
  basicstyle={\ttfamily\footnotesize}, % Basic font style
  breakatwhitespace=false,         
  breaklines=true,                 
  captionpos=b,                    
  keepspaces=true,                 
  numbers=none,                    
  tabsize=2,
  morekeywords=[2]{\#include, \#define, \#ifdef, \#ifndef, \#endif, \#pragma, \#else, \#elif}, % Preprocessor directives
  keywordstyle=[2]\color{codegreen}, % Style for preprocessor directives
  morekeywords=[3]{int, char, float, double, void, struct, union, enum, const, volatile, static, extern, register, inline, restrict, _Bool, _Complex, _Imaginary, size_t, ssize_t, FILE}, % C standard types and common identifiers
  keywordstyle=[3]\color{identblue}, % Style for types and common identifiers
  morekeywords=[4]{printf, scanf, fopen, fclose, malloc, free, calloc, realloc, perror, strtok, strncpy, strcpy, strcmp, strlen}, % Standard library functions
  keywordstyle=[4]\color{cyan}, % Style for library functions
}

% Things Lie
\newcommand{\kb}{\mathfrak b}
\newcommand{\kg}{\mathfrak g}
\newcommand{\kh}{\mathfrak h}
\newcommand{\kn}{\mathfrak n}
\newcommand{\ku}{\mathfrak u}
\newcommand{\kz}{\mathfrak z}
\DeclareMathOperator{\Ext}{Ext} % Ext functor
\DeclareMathOperator{\Tor}{Tor} % Tor functor
\newcommand{\gl}{\opname{\mathfrak{gl}}} % frak gl group
\renewcommand{\sl}{\opname{\mathfrak{sl}}} % frak sl group chktex 6

% More script letters etc.
\newcommand{\SA}{\mathcal A}
\newcommand{\SB}{\mathcal B}
\newcommand{\SC}{\mathcal C}
\newcommand{\SF}{\mathcal F}
\newcommand{\SG}{\mathcal G}
\newcommand{\SH}{\mathcal H}
\newcommand{\OO}{\mathcal O}

\newcommand{\SCA}{\mathscr A}
\newcommand{\SCB}{\mathscr B}
\newcommand{\SCC}{\mathscr C}
\newcommand{\SCD}{\mathscr D}
\newcommand{\SCE}{\mathscr E}
\newcommand{\SCF}{\mathscr F}
\newcommand{\SCG}{\mathscr G}
\newcommand{\SCH}{\mathscr H}

% Mathfrak primes
\newcommand{\km}{\mathfrak m}
\newcommand{\kp}{\mathfrak p}
\newcommand{\kq}{\mathfrak q}

% number sets
\newcommand{\RR}[1][]{\ensuremath{\ifstrempty{#1}{\mathbb{R}}{\mathbb{R}^{#1}}}}
\newcommand{\NN}[1][]{\ensuremath{\ifstrempty{#1}{\mathbb{N}}{\mathbb{N}^{#1}}}}
\newcommand{\ZZ}[1][]{\ensuremath{\ifstrempty{#1}{\mathbb{Z}}{\mathbb{Z}^{#1}}}}
\newcommand{\QQ}[1][]{\ensuremath{\ifstrempty{#1}{\mathbb{Q}}{\mathbb{Q}^{#1}}}}
\newcommand{\CC}[1][]{\ensuremath{\ifstrempty{#1}{\mathbb{C}}{\mathbb{C}^{#1}}}}
\newcommand{\PP}[1][]{\ensuremath{\ifstrempty{#1}{\mathbb{P}}{\mathbb{P}^{#1}}}}
\newcommand{\HH}[1][]{\ensuremath{\ifstrempty{#1}{\mathbb{H}}{\mathbb{H}^{#1}}}}
\newcommand{\FF}[1][]{\ensuremath{\ifstrempty{#1}{\mathbb{F}}{\mathbb{F}^{#1}}}}
% expected value
\newcommand{\EE}{\ensuremath{\mathbb{E}}}
\newcommand{\charin}{\text{ char }}
\DeclareMathOperator{\sign}{sign}
\DeclareMathOperator{\Aut}{Aut}
\DeclareMathOperator{\Inn}{Inn}
\DeclareMathOperator{\Syl}{Syl}
\DeclareMathOperator{\Gal}{Gal}
\DeclareMathOperator{\GL}{GL} % General linear group
\DeclareMathOperator{\SL}{SL} % Special linear group

%---------------------------------------
% BlackBoard Math Fonts :-
%---------------------------------------

%Captital Letters
\newcommand{\bbA}{\mathbb{A}}	\newcommand{\bbB}{\mathbb{B}}
\newcommand{\bbC}{\mathbb{C}}	\newcommand{\bbD}{\mathbb{D}}
\newcommand{\bbE}{\mathbb{E}}	\newcommand{\bbF}{\mathbb{F}}
\newcommand{\bbG}{\mathbb{G}}	\newcommand{\bbH}{\mathbb{H}}
\newcommand{\bbI}{\mathbb{I}}	\newcommand{\bbJ}{\mathbb{J}}
\newcommand{\bbK}{\mathbb{K}}	\newcommand{\bbL}{\mathbb{L}}
\newcommand{\bbM}{\mathbb{M}}	\newcommand{\bbN}{\mathbb{N}}
\newcommand{\bbO}{\mathbb{O}}	\newcommand{\bbP}{\mathbb{P}}
\newcommand{\bbQ}{\mathbb{Q}}	\newcommand{\bbR}{\mathbb{R}}
\newcommand{\bbS}{\mathbb{S}}	\newcommand{\bbT}{\mathbb{T}}
\newcommand{\bbU}{\mathbb{U}}	\newcommand{\bbV}{\mathbb{V}}
\newcommand{\bbW}{\mathbb{W}}	\newcommand{\bbX}{\mathbb{X}}
\newcommand{\bbY}{\mathbb{Y}}	\newcommand{\bbZ}{\mathbb{Z}}

%---------------------------------------
% MathCal Fonts :-
%---------------------------------------

%Captital Letters
\newcommand{\mcA}{\mathcal{A}}	\newcommand{\mcB}{\mathcal{B}}
\newcommand{\mcC}{\mathcal{C}}	\newcommand{\mcD}{\mathcal{D}}
\newcommand{\mcE}{\mathcal{E}}	\newcommand{\mcF}{\mathcal{F}}
\newcommand{\mcG}{\mathcal{G}}	\newcommand{\mcH}{\mathcal{H}}
\newcommand{\mcI}{\mathcal{I}}	\newcommand{\mcJ}{\mathcal{J}}
\newcommand{\mcK}{\mathcal{K}}	\newcommand{\mcL}{\mathcal{L}}
\newcommand{\mcM}{\mathcal{M}}	\newcommand{\mcN}{\mathcal{N}}
\newcommand{\mcO}{\mathcal{O}}	\newcommand{\mcP}{\mathcal{P}}
\newcommand{\mcQ}{\mathcal{Q}}	\newcommand{\mcR}{\mathcal{R}}
\newcommand{\mcS}{\mathcal{S}}	\newcommand{\mcT}{\mathcal{T}}
\newcommand{\mcU}{\mathcal{U}}	\newcommand{\mcV}{\mathcal{V}}
\newcommand{\mcW}{\mathcal{W}}	\newcommand{\mcX}{\mathcal{X}}
\newcommand{\mcY}{\mathcal{Y}}	\newcommand{\mcZ}{\mathcal{Z}}

%---------------------------------------
% Bold Math Fonts :-
%---------------------------------------

%Captital Letters
\newcommand{\bmA}{\boldsymbol{A}}	\newcommand{\bmB}{\boldsymbol{B}}
\newcommand{\bmC}{\boldsymbol{C}}	\newcommand{\bmD}{\boldsymbol{D}}
\newcommand{\bmE}{\boldsymbol{E}}	\newcommand{\bmF}{\boldsymbol{F}}
\newcommand{\bmG}{\boldsymbol{G}}	\newcommand{\bmH}{\boldsymbol{H}}
\newcommand{\bmI}{\boldsymbol{I}}	\newcommand{\bmJ}{\boldsymbol{J}}
\newcommand{\bmK}{\boldsymbol{K}}	\newcommand{\bmL}{\boldsymbol{L}}
\newcommand{\bmM}{\boldsymbol{M}}	\newcommand{\bmN}{\boldsymbol{N}}
\newcommand{\bmO}{\boldsymbol{O}}	\newcommand{\bmP}{\boldsymbol{P}}
\newcommand{\bmQ}{\boldsymbol{Q}}	\newcommand{\bmR}{\boldsymbol{R}}
\newcommand{\bmS}{\boldsymbol{S}}	\newcommand{\bmT}{\boldsymbol{T}}
\newcommand{\bmU}{\boldsymbol{U}}	\newcommand{\bmV}{\boldsymbol{V}}
\newcommand{\bmW}{\boldsymbol{W}}	\newcommand{\bmX}{\boldsymbol{X}}
\newcommand{\bmY}{\boldsymbol{Y}}	\newcommand{\bmZ}{\boldsymbol{Z}}
%Small Letters
\newcommand{\bma}{\boldsymbol{a}}	\newcommand{\bmb}{\boldsymbol{b}}
\newcommand{\bmc}{\boldsymbol{c}}	\newcommand{\bmd}{\boldsymbol{d}}
\newcommand{\bme}{\boldsymbol{e}}	\newcommand{\bmf}{\boldsymbol{f}}
\newcommand{\bmg}{\boldsymbol{g}}	\newcommand{\bmh}{\boldsymbol{h}}
\newcommand{\bmi}{\boldsymbol{i}}	\newcommand{\bmj}{\boldsymbol{j}}
\newcommand{\bmk}{\boldsymbol{k}}	\newcommand{\bml}{\boldsymbol{l}}
\newcommand{\bmm}{\boldsymbol{m}}	\newcommand{\bmn}{\boldsymbol{n}}
\newcommand{\bmo}{\boldsymbol{o}}	\newcommand{\bmp}{\boldsymbol{p}}
\newcommand{\bmq}{\boldsymbol{q}}	\newcommand{\bmr}{\boldsymbol{r}}
\newcommand{\bms}{\boldsymbol{s}}	\newcommand{\bmt}{\boldsymbol{t}}
\newcommand{\bmu}{\boldsymbol{u}}	\newcommand{\bmv}{\boldsymbol{v}}
\newcommand{\bmw}{\boldsymbol{w}}	\newcommand{\bmx}{\boldsymbol{x}}
\newcommand{\bmy}{\boldsymbol{y}}	\newcommand{\bmz}{\boldsymbol{z}}

%---------------------------------------
% Scr Math Fonts :-
%---------------------------------------

\newcommand{\sA}{{\mathscr{A}}}   \newcommand{\sB}{{\mathscr{B}}}
\newcommand{\sC}{{\mathscr{C}}}   \newcommand{\sD}{{\mathscr{D}}}
\newcommand{\sE}{{\mathscr{E}}}   \newcommand{\sF}{{\mathscr{F}}}
\newcommand{\sG}{{\mathscr{G}}}   \newcommand{\sH}{{\mathscr{H}}}
\newcommand{\sI}{{\mathscr{I}}}   \newcommand{\sJ}{{\mathscr{J}}}
\newcommand{\sK}{{\mathscr{K}}}   \newcommand{\sL}{{\mathscr{L}}}
\newcommand{\sM}{{\mathscr{M}}}   \newcommand{\sN}{{\mathscr{N}}}
\newcommand{\sO}{{\mathscr{O}}}   \newcommand{\sP}{{\mathscr{P}}}
\newcommand{\sQ}{{\mathscr{Q}}}   \newcommand{\sR}{{\mathscr{R}}}
\newcommand{\sS}{{\mathscr{S}}}   \newcommand{\sT}{{\mathscr{T}}}
\newcommand{\sU}{{\mathscr{U}}}   \newcommand{\sV}{{\mathscr{V}}}
\newcommand{\sW}{{\mathscr{W}}}   \newcommand{\sX}{{\mathscr{X}}}
\newcommand{\sY}{{\mathscr{Y}}}   \newcommand{\sZ}{{\mathscr{Z}}}


%---------------------------------------
% Math Fraktur Font
%---------------------------------------

%Captital Letters
\newcommand{\mfA}{\mathfrak{A}}	\newcommand{\mfB}{\mathfrak{B}}
\newcommand{\mfC}{\mathfrak{C}}	\newcommand{\mfD}{\mathfrak{D}}
\newcommand{\mfE}{\mathfrak{E}}	\newcommand{\mfF}{\mathfrak{F}}
\newcommand{\mfG}{\mathfrak{G}}	\newcommand{\mfH}{\mathfrak{H}}
\newcommand{\mfI}{\mathfrak{I}}	\newcommand{\mfJ}{\mathfrak{J}}
\newcommand{\mfK}{\mathfrak{K}}	\newcommand{\mfL}{\mathfrak{L}}
\newcommand{\mfM}{\mathfrak{M}}	\newcommand{\mfN}{\mathfrak{N}}
\newcommand{\mfO}{\mathfrak{O}}	\newcommand{\mfP}{\mathfrak{P}}
\newcommand{\mfQ}{\mathfrak{Q}}	\newcommand{\mfR}{\mathfrak{R}}
\newcommand{\mfS}{\mathfrak{S}}	\newcommand{\mfT}{\mathfrak{T}}
\newcommand{\mfU}{\mathfrak{U}}	\newcommand{\mfV}{\mathfrak{V}}
\newcommand{\mfW}{\mathfrak{W}}	\newcommand{\mfX}{\mathfrak{X}}
\newcommand{\mfY}{\mathfrak{Y}}	\newcommand{\mfZ}{\mathfrak{Z}}
%Small Letters
\newcommand{\mfa}{\mathfrak{a}}	\newcommand{\mfb}{\mathfrak{b}}
\newcommand{\mfc}{\mathfrak{c}}	\newcommand{\mfd}{\mathfrak{d}}
\newcommand{\mfe}{\mathfrak{e}}	\newcommand{\mff}{\mathfrak{f}}
\newcommand{\mfg}{\mathfrak{g}}	\newcommand{\mfh}{\mathfrak{h}}
\newcommand{\mfi}{\mathfrak{i}}	\newcommand{\mfj}{\mathfrak{j}}
\newcommand{\mfk}{\mathfrak{k}}	\newcommand{\mfl}{\mathfrak{l}}
\newcommand{\mfm}{\mathfrak{m}}	\newcommand{\mfn}{\mathfrak{n}}
\newcommand{\mfo}{\mathfrak{o}}	\newcommand{\mfp}{\mathfrak{p}}
\newcommand{\mfq}{\mathfrak{q}}	\newcommand{\mfr}{\mathfrak{r}}
\newcommand{\mfs}{\mathfrak{s}}	\newcommand{\mft}{\mathfrak{t}}
\newcommand{\mfu}{\mathfrak{u}}	\newcommand{\mfv}{\mathfrak{v}}
\newcommand{\mfw}{\mathfrak{w}}	\newcommand{\mfx}{\mathfrak{x}}
\newcommand{\mfy}{\mathfrak{y}}	\newcommand{\mfz}{\mathfrak{z}}


\newcommand{\mytitle}{MA1522 Tutorial 3}
\newcommand{\myauthor}{github/omgeta}
\newcommand{\mydate}{AY 24/25 Sem 1}

\begin{document}
\raggedright
\footnotesize
\begin{center}
{\normalsize{\textbf{\mytitle}}} \\
{\footnotesize{\mydate\hspace{2pt}\textemdash\hspace{2pt}\myauthor}}
\end{center}
\setlist{topsep=-1em, itemsep=-1em, parsep=2em}

%%%%%%%%%%%%%%%%%%%%%%%%%%%%%%%%%%%%%%%%%%%%%%%%%%%%%%
%                      Begin                         %
%%%%%%%%%%%%%%%%%%%%%%%%%%%%%%%%%%%%%%%%%%%%%%%%%%%%%%
\begin{enumerate}[Q\arabic*.]
  \item If $A \in \RR^{4\times4}$ is obtained from $I$ by the following sequence of elementary row operations:
    \begin{align*}
      I \xrightarrow{\frac{1}{2}R_2}\xrightarrow{R_1-R_2}\xrightarrow{R_2\leftrightarrow R_4}\xrightarrow{R_3+3R_1}A
    \end{align*}
    Then $A$ is also obtained from $I$ by the following matrix multiplications:
    \begin{align*}
      A = E_4E_3E_2E_1I
    \end{align*}
    Where the elementary matrices $E_i$ are given by:
    \begin{align*}
      E_1 &= \begin{bmatrix}1&0&0&0\\0&1/2&0&0\\0&0&1&0\\0&0&0&1\end{bmatrix} \\
      E_2 &= \begin{bmatrix}1&-1&0&0\\0&1&0&0\\0&0&1&0\\0&0&0&1\end{bmatrix} \\
      E_3 &= \begin{bmatrix}1&0&0&0\\0&0&0&1\\0&0&1&0\\0&1&0&0\end{bmatrix} \\
      E_4 &= \begin{bmatrix}1&0&0&0\\0&1&0&0\\3&0&1&0\\0&0&0&1\end{bmatrix} \\
    \end{align*}
    And their inverse are given by:
    \begin{align*}
      E_1^{-1} &= \begin{bmatrix}1&0&0&0\\0&2&0&0\\0&0&1&0\\0&0&0&1\end{bmatrix} \\
      E_2^{-1} &= \begin{bmatrix}1&1&0&0\\0&1&0&0\\0&0&1&0\\0&0&0&1\end{bmatrix} \\
      E_3^{-1} &= \begin{bmatrix}1&0&0&0\\0&0&0&1\\0&0&1&0\\0&1&0&0\end{bmatrix} \\
      E_4^{-1} &= \begin{bmatrix}1&0&0&0\\0&1&0&0\\-3&0&1&0\\0&0&0&1\end{bmatrix} \\
    \end{align*}
    Then the inverse $A^{-1}$ is obtained from $I$ by the following matrix mulitiplications:
    \begin{align*}
      A^{-1} &= (E_4E_3E_2E_1I)^{-1} \\
             &= E_1^{-1}E_2^{-1}E_3^{-1}E_4^{-1} \qed
    \end{align*}

  \item 
    \begin{enumerate}[(\alph*)]
      \item Find the LU factorisation for $A$:
        \begin{align*}
          A = \begin{bmatrix}2&-1&2\\-6&0&-2\\8&-1&5\end{bmatrix} &\xrightarrow{R_2+3R_1}
          \begin{bmatrix}2&-1&2\\0&-3&4\\8&-1&5\end{bmatrix} \\&\xrightarrow{R_3-4R_1}
          \begin{bmatrix}2&-1&2\\0&-3&4\\8&3&-3\end{bmatrix} \\&\xrightarrow{R_2+3R_1}
          \begin{bmatrix}2&-1&2\\0&-3&4\\0&0&1\end{bmatrix} = U \\
          L = \begin{bmatrix}1&0&0\\-3&1&0\\4&-1&1\end{bmatrix} \\
          \therefore A =  \begin{bmatrix}1&0&0\\-3&1&0\\4&-1&1\end{bmatrix}\begin{bmatrix}2&-1&2\\0&-3&4\\0&0&1\end{bmatrix} \qed 
        \end{align*}
        Let $\vec{y} = U \vec{x}$ and solve $L \vec{y} = \vec{b}$
        \begin{align*}
          \begin{bmatrix}1&0&0&1\\-3&1&0&0\\4&-1&1&4\end{bmatrix} \xrightarrow{\text{RREF}}
          \begin{bmatrix}1&0&0&1\\0&1&0&1\\0&0&1&3\end{bmatrix}
        \end{align*}
        Then solve $U \vec{x} = \begin{bmatrix}1\\1\\3\end{bmatrix}$
        \begin{align*}
          \begin{bmatrix}2&-1&2&1\\0&-3&4&1\\0&0&1&3\end{bmatrix} \xrightarrow{\text{RREF}}
          \begin{bmatrix}1&0&0&-2/3\\0&1&0&11/3\\0&0&1&3\end{bmatrix}
        \end{align*}
        Therefore, $\vec{x} = \begin{bmatrix}-2/3\\11/3\\3\end{bmatrix} \qed$


      \item Find the LU factorisation for $A$:
        \begin{align*}
          A = \begin{bmatrix}2&-4&4&-2\\6&-9&7&-3\\-1&-4&8&0\end{bmatrix} &\xrightarrow[R_2+\frac{1}{2}R_1]{R_2-3R_1}
          \begin{bmatrix}2&-4&4&-2\\0&3&-5&3\\0&-6&10&-1\end{bmatrix} \\&\xrightarrow{R_3+2R_2}
          \begin{bmatrix}2&-4&4&-2\\0&3&-5&3\\0&0&0&5\end{bmatrix} = U \\
          L = \begin{bmatrix}1&0&0\\3&1&0\\-1/2&-2&1\end{bmatrix}\\
          \therefore A =  \begin{bmatrix}1&0&0\\3&1&0\\-1/2&-2&1\end{bmatrix} \begin{bmatrix}2&-4&4&-2\\0&3&-5&3\\0&0&0&5\end{bmatrix} \qed 
        \end{align*}
        Let $\vec{y} = U \vec{x}$ and solve $L \vec{y} = \vec{b}$
        \begin{align*}
          \begin{bmatrix}1&0&0&0\\3&1&0&0\\-1/2&-2&1&17\end{bmatrix} \xrightarrow{\text{RREF}}
          \begin{bmatrix}1&0&0&0\\0&1&0&0\\0&0&1&17\end{bmatrix}
        \end{align*}
        Then solve $U \vec{x} = \begin{bmatrix}0\\0\\17\end{bmatrix}$
        \begin{align*}
          \begin{bmatrix}2&-4&4&-2&0\\0&3&-5&3&0\\0&0&0&5&17\end{bmatrix} \xrightarrow{\text{RREF}}
          \begin{bmatrix}1&0&-4/3&0&-17/5\\0&1&-5/3&0&-17/5\\0&0&0&1&17/5\end{bmatrix}
        \end{align*}
        Therefore, $\vec{x} = \displaystyle \frac{17}{5}\begin{bmatrix}-1\\-1\\0\\1\end{bmatrix} + \frac{x_3}{3}\begin{bmatrix}-4\\-5\\3\\0\end{bmatrix} \qed$
    \end{enumerate}

  \item \begin{enumerate}[(\alph*)]
      \item Find an LU factorisation for A 
        \begin{align*}
          A = \begin{bmatrix}
            2 & -6 & 6\\
            -4 & 5 & -7\\
            3 & 5 & -1\\
            -6 & 4 & -8\\
            8 & -3 & 9
          \end{bmatrix} &\sim
          \begin{bmatrix}
            2 & -6 & 6\\
            0 & -7 & 5\\
            0 & 14 & -19\\
            0 & -14 & 10\\
            0 & 21 & 15
          \end{bmatrix}\\&\sim
          \begin{bmatrix}
            2 & -6 & 6\\
            0 & -7 & 5\\
            0 & 0 & -9\\
            0 & 0 & 0 \\
            0 & 0 & 30
          \end{bmatrix}\\&\sim
          \begin{bmatrix}
            2 & -6 & 6\\
            0 & -7 & 5\\
            0 & 0 & 0\\
            0 & 0 & 0 \\
            0 & 0 & 0
          \end{bmatrix} = U \qed \\
          L = \begin{bmatrix}
            1 & 0 & 0 & 0 & 0\\
            -2 & 1 & 0 & 0 & 0\\
            3/2 & -2 & 1 & 0 & 0\\
            -3 & 2 & 0 & 1 & 0\\
            4 & -3 & 0 & 0 & 1
          \end{bmatrix} \qed
        \end{align*}
      \item The MATLAB LU is the same. $\qed$
    \end{enumerate}

  \item First calculate the determinant by cofactor expansion:
    \begin{align*}
      \det(A) &= -x\begin{vmatrix}-x & 1 \\ -5 & 4-x\end{vmatrix} - \begin{vmatrix}0 & 1 \\ 2 & 4-x\end{vmatrix} \\
              &= -x[-x(4-x) - 1(-5)] - [0(4-x) - 1(2)]\\
              &= -x(-4x + x^2 + 5) - (-2) \\
              &= -x^3 + 4x^2 -5x + 2 \qed
    \end{align*}
    For $A$ to be singular, $\det(A) = 0$:
    \begin{align*}
      -x^3 + 4x^2 - 5x + 2 &= 0 \\
      (x-1)^2(x-2) &= 0\\
      x &= 1, 2 \qed 
    \end{align*}

  \item First, reduce the relevant matrices:
    \begin{align*}
      \begin{bmatrix}
        a+px & b+qx & c+rx\\
        p+ux & q+vx & r+wx\\
        u+ax & v+bx & w+cx
      \end{bmatrix}&\xrightarrow{R_2-xR_3}
      \begin{bmatrix}
        a+px & b+qx & c+rx\\
        p-ax^2 & q-bx^2 & r-cx^2\\
        u+ax & v+bx & w+cx
      \end{bmatrix}\\&\xrightarrow{R_1-xR_2}
      \begin{bmatrix}
        a(1+x^3) & b(1+x^3) & c(1+x^3)\\
        p-ax^2 & q-bx^2 & r-cx^2\\
        u+ax & v+bx & w+cx
      \end{bmatrix}\\
      \begin{bmatrix}
        a & b & c\\
        p-ax^2 & q-bx^2 & r-cx^2\\
        u+ax & v+bx & w+cx
      \end{bmatrix}&\xrightarrow{R_2+x^2R_1}
      \begin{bmatrix}
        a & b & c\\
        p & q & r\\
        u+ax & v+bx & w+cx
      \end{bmatrix}\\&\xrightarrow{R_3-xR_1}
      \begin{bmatrix}
        a & b & c\\
        p & q & r\\
        u & v & w
      \end{bmatrix}
    \end{align*}
    Then, we can conclude:
    \begin{align*}
      \begin{vmatrix}
        a+px & b+qx & c+rx\\
        p+ux & q+vx & r+wx\\
        u+ax & v+bx & w+cx
      \end{vmatrix}&=
      (1+x^3)
      \begin{vmatrix}
        a & b & c\\
        p-ax^2 & q-bx^2 & r-cx^2\\
        u+ax & v+bx & w+cx
      \end{vmatrix}\\ &=
      (1+x^3)
      \begin{vmatrix}
        a & b & c\\
        p & q & r\\
        u & v & w
      \end{vmatrix}
    \end{align*}

  \item First, find $\det(A)$ and $\det(B)$
    \begin{align*}
      \det(A) &= 1\begin{vmatrix}2&6&3\\0&1&2\\0&1&1\end{vmatrix}\\
              &= 2\begin{vmatrix}1&2\\1&1\end{vmatrix}\\
              &= 2[1(1)-2(1)]\\
              &= -2 \\
      \det(B) &= 1\begin{vmatrix}1&1&-1\\0&1&2\\0&0&3\end{vmatrix}\\
              &= 1\begin{vmatrix}1&2\\0&3\end{vmatrix}\\
              &= 3
    \end{align*}
    \begin{enumerate}[(\alph*)]
      \item $\det(3A^T) = 3^4\det(A) = -162 \qed$
      \item $\det(3AB^{-1}) = 3^4\frac{\det(A)}{\det(B)} = -54 \qed$
      \item $\det(3A^T) = \frac{1}{\det(3B)} = \frac{1}{3^4\det(B)} =  \frac{1}{243}\qed$
    \end{enumerate}

  \item Let $A = \begin{bmatrix}1&5&3\\0&2&-2\\0&1&3\end{bmatrix}, \vec{b} = \begin{bmatrix}1\\2\\0\end{bmatrix}$. Then, $\det(A) = 1\begin{vmatrix}2&-2\\1&3\end{vmatrix} = 8$. By Cramer's rule:
    \begin{align*}
      x_1 &= \frac{\begin{vmatrix}1&5&3\\2&2&-2\\0&1&3\end{vmatrix}}{8} \\
                &= \frac{(6+2)-2(15-3)}{8} \\
                &= -2 \\
      x_2 &= \frac{\begin{vmatrix}1&1&3\\0&2&-2\\0&0&3\end{vmatrix}}{8} \\
                &= \frac{6}{8} \\
                &= \frac{3}{4} \\
      x_3 &= \frac{\begin{vmatrix}1&5&1\\0&2&2\\0&1&0\end{vmatrix}}{8} \\
                &= \frac{-2}{8} \\
                &= -\frac{1}{4} \\
    \end{align*}
    Therefore, $\vec{x} = \begin{bmatrix}-2\\3/4\\-1/4\end{bmatrix} \qed$

  \item First find the adjoint of $A$:
    \begin{align*}
      \adj(A) &= \begin{bmatrix}A_{11}&A_{21}&A_{31}\\A_{12}&A_{22}&A_{32}\\A_{13}&A_{23}&A_{33}\end{bmatrix}\\
              &= \begin{bmatrix}12 & 6 & -5\\ 3 & 0 & -1\\ -6 & -3 & 2\end{bmatrix} \qed
    \end{align*}
    Then find the determinant:
    \begin{align*}
      \det(A) &= 1\begin{vmatrix}2&1\\0&6\end{vmatrix} + 3\begin{vmatrix}-1&2\\2&1\end{vmatrix}\\
              &= 12 + 3(-1 - 4)\\
              &= -3          
    \end{align*}
    Then the inverse $A^{-1}$ is given by:
    \begin{align*}
      A^{-1} &= \frac{1}{\det(A)}\adj(A)\\
             &= -\frac{1}{3}\begin{bmatrix}12&6&-5\\3&0&-1\\-6&-3&2\end{bmatrix} \qed
    \end{align*}
\end{enumerate}
%%%%%%%%%%%%%%%%%%%%%%%%%%%%%%%%%%%%%%%%%%%%%%%%%%%%%%
%                       End                          %
%%%%%%%%%%%%%%%%%%%%%%%%%%%%%%%%%%%%%%%%%%%%%%%%%%%%%%

\end{document}

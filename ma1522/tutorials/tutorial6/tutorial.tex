\documentclass[12pt, a4paper]{article}

\usepackage[a4paper, margin=1in]{geometry}

\usepackage[utf8]{inputenc}
\usepackage[mathscr]{euscript}
\let\euscr\mathscr \let\mathscr\relax
\usepackage[scr]{rsfso}
\usepackage{amssymb,amsmath,amsthm,amsfonts}
\usepackage[shortlabels]{enumitem}
\usepackage{multicol,multirow}
\usepackage{lipsum}
\usepackage{balance}
\usepackage{calc}
\usepackage[colorlinks=true,citecolor=blue,linkcolor=blue]{hyperref}
\usepackage{import}
\usepackage{xifthen}
\usepackage{pdfpages}
\usepackage{transparent}
\usepackage{tabularx}

\newcommand{\incfig}[2][1.0]{
    \def\svgwidth{#1\columnwidth}
    \import{./figures/}{#2.pdf_tex}
}
\newcommand{\incimg}[2][1.0]{
  \includegraphics[width=#1\columnwidth]{./figures/#2}
}


\input{letterfonts}

\newcommand{\mytitle}{MA1522 Tutorial 6}
\newcommand{\myauthor}{github/omgeta}
\newcommand{\mydate}{AY 24/25 Sem 1}

\begin{document}
\raggedright
\footnotesize
\begin{center}
{\normalsize{\textbf{\mytitle}}} \\
{\footnotesize{\mydate\hspace{2pt}\textemdash\hspace{2pt}\myauthor}}
\end{center}
\setlist{topsep=-1em, itemsep=-1em, parsep=2em}

%%%%%%%%%%%%%%%%%%%%%%%%%%%%%%%%%%%%%%%%%%%%%%%%%%%%%%
%                      Begin                         %
%%%%%%%%%%%%%%%%%%%%%%%%%%%%%%%%%%%%%%%%%%%%%%%%%%%%%%
\begin{enumerate}[Q\arabic*.]
  \item 
    \begin{enumerate}[(\alph*)]
      \item Reduce the corresponding matrix $P_S$:
        \begin{align*}
          \left(\begin{array}{ccc} 1 & 0 & 0\\ 2 & 2 & -1\\ -1 & 1 & 3 \end{array}\right)
          \xrightarrow{RREF}
          \left(\begin{array}{ccc} 1 & 0 & 0\\ 0 & 1 & 0\\ 0 & 0 & 1 \end{array}\right)
        \end{align*}
      Therefore, $\rank(P_S)=3\implies S \text{ is a basis for } \RR^3 \qed$

      \item Use the transition matrix $P_S^{-1}$ to change $\vec{w}$ from the standard basis to $S$:
        \begin{align*}
          [\vec{w}]_S &= P_S^{-1}\vec{w}\\
                      &= \left(\begin{array}{ccc} 1 & 0 & 0\\ 2 & 2 & -1\\ -1 & 1 & 3 \end{array}\right)^{-1}\left(\begin{array}{c} 1\\ 1\\ 1 \end{array}\right)\\
                      &= \left(\begin{array}{c} 1\\ -1 /7\\ 5 /7 \end{array}\right) \qed
        \end{align*}

      \item Reduce the matrix $[P_S | P_T]$:
        \begin{align*}
          \left(\begin{array}{ccccccc} 1 & 0 & 0 &|& 1 & -1 & 2\\ 2 & 2 & -1 &|& 5 & 3 & 2\\ -1 & 1 & 3 &|& 4 & 7 & 4 \end{array}\right)
          \xrightarrow{RREF}
          \left(\begin{array}{ccccccc} 1 & 0 & 0 &|& 1 & -1 & 2\\ 0 & 1 & 0 &|& 2 & 3 & 0\\ 0 & 0 & 1 &|& 1 & 1 & 2 \end{array}\right)
        \end{align*}
        Therefore, $P_{S\leftarrow T}=\left(\begin{array}{ccc} 1 & -1 & 2\\ 2 & 3 & 0\\ 1 & 1 & 2 \end{array}\right) \qed$

      \item Find the inverse of $P_{S\leftarrow T}$:
        \begin{align*}
          P_{T\leftarrow S} = P_{S\leftarrow T}^{-1} = \left(\begin{array}{ccc} \frac{3}{4} & \frac{1}{2} & -\frac{3}{4}\\ -\frac{1}{2} & 0 & \frac{1}{2}\\ -\frac{1}{8} & -\frac{1}{4} & \frac{5}{8} \end{array}\right) \qed
        \end{align*}

      \item Use the required transition matrix to find $[\vec{w}]_T$:
        \begin{align*}
          [\vec{w}]_T &= P_{T\leftarrow S}[\vec{w}]_S\\
                      &= \left(\begin{array}{ccc} \frac{3}{4} & \frac{1}{2} & -\frac{3}{4}\\ -\frac{1}{2} & 0 & \frac{1}{2}\\ -\frac{1}{8} & -\frac{1}{4} & \frac{5}{8} \end{array}\right)\left(\begin{array}{c} 1\\ -\frac{1}{7}\\ \frac{5}{7} \end{array}\right)\\
                      &= \left(\begin{array}{c} \frac{1}{7}\\ -\frac{1}{7}\\ \frac{5}{14} \end{array}\right) \qed
        \end{align*}
    \end{enumerate}

    \pagebreak

  \item
    \begin{enumerate}[(\alph*)]
      \item $v_1, v_2, v_3 \in V \implies \Span(T) \subseteq V$ and $|T|=3=\dim(V)$. Consider also:
        \begin{align*}
          c_1 \vec{v_1} + c_2 \vec{v_2} + c_3 \vec{v_3} &= 0\\
          c_1(\vec{u_1} + \vec{u_2} + \vec{u_3}) + c_2(\vec{u_2} + \vec{u_3}) + c_3(\vec{u_2} - \vec{u_3}) &= 0\\
          c_1 \vec{u_1} + (c_1 + c_2 + c_3)\vec{u_2} + (c_1+c_2-c_3)\vec{u} &= 0
        \end{align*}
        which has only the trivial solution $c_1 = c_2 = c_3 = 0 \implies T$ is linearly independent. Therefore, $T$ is a basis for $V \qed$ 

      \item Transition matrix from $S$ to $T$ is $P_{S\leftarrow T}^{-1} = \left(\begin{array}{ccc} 1 & 0 & 0\\ 1 & 1 & 1\\ 1 & 1 & -1 \end{array}\right)^{-1} = \left(\begin{array}{ccc} 1 & 0 & 0\\ -1 & \frac{1}{2} & \frac{1}{2}\\ 0 & \frac{1}{2} & -\frac{1}{2} \end{array}\right) \qed$
    \end{enumerate}

  \item
    \begin{enumerate}[(\alph*)]
      \item Check if $\vec{b} \in \col(A)$ by reducing the matrix: 
        \begin{align*}
          \left(\begin{array}{ccccc} 1 & -1 & 1 &|& 2\\ 1 & 1 & -1 &|& 1\\ -1 & -1 & 1 &|& 0 \end{array}\right)
          \xrightarrow{RREF}
          \left(\begin{array}{ccccc} 1 & 0 & 0 &|& 0\\ 0 & 1 & -1 &|& 0\\ 0 & 0 & 0 &|& 1 \end{array}\right)
        \end{align*}
        This is an inconsistent equation, therefore $\vec{b} \not\in \col(A) \qed$

      \item Check if $\vec{b}^T \in \col(A^T)$ by reducing the matrix:
        \begin{align*}
          \left(\begin{array}{ccccc} 1 & -1 & 1 &|& 5\\ 9 & 3 & 1 &|& 1\\ 1 & 1 & 1 &|& -1 \end{array}\right)
          \xrightarrow{RREF}
          \left(\begin{array}{cccc} 1 & 0 & 0 & 1\\ 0 & 1 & 0 & -3\\ 0 & 0 & 1 & 1 \end{array}\right)
        \end{align*}
        Therefore, $\vec{b} \in \row(A)$ and $\vec{b} = 1(1,9,1) -3(-1,3,1) +1(1,1,1) \qed$

      \item Reduce matrix $A$:
        \begin{align*}
          \left(\begin{array}{cccc} 1 & 2 & 0 & 1\\ 0 & 1 & 2 & 1\\ 1 & 2 & 1 & 3\\ 0 & 1 & 2 & 2 \end{array}\right)
          \xrightarrow{RREF}
          \left(\begin{array}{cccc} 1 & 0 & 0 & 0\\ 0 & 1 & 0 & 0\\ 0 & 0 & 1 & 0\\ 0 & 0 & 0 & 1 \end{array}\right)
        \end{align*}
        Therefore, $\col(A)=\RR^4$. By invertible matrix theorem, $\row(A)=\RR^4 \qed$
    \end{enumerate}

    \pagebreak
    \item
      \begin{enumerate}[(\alph*)]
        \item Reduce $A$:
          \begin{align*}
            \left(\begin{array}{cccc} 1 & 2 & 5 & 3\\ 1 & -4 & -1 & -9\\ -1 & 0 & -3 & 1\\ 2 & 1 & 7 & 0\\ 0 & 1 & 1 & 2 \end{array}\right)
            \xrightarrow{RREF}
            \left(\begin{array}{cccc} 1 & 0 & 3 & -1\\ 0 & 1 & 1 & 2\\ 0 & 0 & 0 & 0\\ 0 & 0 & 0 & 0\\ 0 & 0 & 0 & 0 \end{array}\right)
          \end{align*}
          \begin{enumerate}[(\roman*)]
            \item Basis for $\row(A) = \{(1,0,3,-1),(0,1,1,2)\} \qed$ 
            \item Basis for $\col(A) = \left\{\left(\begin{array}{c} 1\\ 1\\ -1\\ 2\\ 0 \end{array}\right),\left(\begin{array}{c} 2\\ -4\\ 0\\ 1\\ 1 \end{array}\right)\right\} \qed$ 
            \item Basis for $\nul(A) = \left\{\left(\begin{array}{c} -3\\-1\\1\\0 \end{array}\right),\left(\begin{array}{c} 1\\-2\\0\\1 \end{array}\right)\right\} \qed$ 
            \item $\rank(A) + \nullity(A) = 2 + 2 = 4 = \text{columns of }A$, so rank-nullity is verified $\qed$
            \item $\rank(A)=2<\text{min}\{4,5\}$, therefore $A$ is not full rank$\qed$
          \end{enumerate}

        \item Reduce $A$:
          \begin{align*}
            \left(\begin{array}{ccc} 1 & 3 & 7\\ 2 & 1 & 8\\ 3 & -5 & -1\\ 2 & -2 & 2\\ 1 & 1 & 5 \end{array}\right)
            \xrightarrow{RREF}
            \left(\begin{array}{ccc} 1 & 0 & 0\\ 0 & 1 & 0\\ 0 & 0 & 1\\ 0 & 0 & 0\\ 0 & 0 & 0 \end{array}\right)
          \end{align*}
          \begin{enumerate}[(\roman*)]
          \item Basis for $\row(A) = \left\{\left(\begin{array}{ccc} 1 & 0 & 0 \end{array}\right), \left(\begin{array}{ccc} 0 & 1 & 0 \end{array}\right), \left(\begin{array}{ccc} 0 & 0 & 1 \end{array}\right)\right\} \qed$ 
            \item Basis for $\col(A) = \left\{\left(\begin{array}{c} 1\\ 2\\ 3\\ 2\\ 1 \end{array}\right), \left(\begin{array}{c} 3\\ 1\\ -5\\ -2\\ 1 \end{array}\right), \left(\begin{array}{c} 7\\ 8\\ -1\\ 2\\ 5 \end{array}\right)\right\} \qed$ 
            \item Basis for $\nul(A) = \phi \qed$ 
            \item $\rank(A) + \nullity(A) = 3 + 0 = 3 = \text{columns of }A$, so rank-nullity is verified $\qed$
            \item $\rank(A)=3=\text{min}\{3,5\}$, therefore $A$ is full rank$\qed$
          \end{enumerate}
      \end{enumerate}

    \pagebreak
  \item Reduce the matrix formed by the columns of $W^T$:
    \begin{align*}
      \left(\begin{array}{ccccc} 1 & -2 & 0 & 0 & 3\\ 2 & -5 & -3 & -2 & 6\\ 0 & 5 & 15 & 10 & 0\\ 2 & 1 & 15 & 8 & 6 \end{array}\right)
      \xrightarrow{RREF}
      \left(\begin{array}{ccccc} 1 & 0 & 6 & 0 & 3\\ 0 & 1 & 3 & 0 & 0\\ 0 & 0 & 0 & 1 & 0\\ 0 & 0 & 0 & 0 & 0 \end{array}\right)
    \end{align*}
    \begin{enumerate}[(\alph*)]
      \item Basis for $W$ = $\left\{ \left(\begin{array}{c} 1\\ 0\\ 6\\ 0\\ 3 \end{array}\right), \left(\begin{array}{c} 0\\ 1\\ 3\\ 0\\ 0 \end{array}\right), \left(\begin{array}{c} 0\\ 0\\ 0\\ 1\\ 0 \end{array}\right)\right\} \qed$ 
      \item $\dim(W) = 3\qed$
      \item Basis for $\RR^5$ = $\left\{ \left(\begin{array}{c} 1\\ 0\\ 6\\ 0\\ 3 \end{array}\right), \left(\begin{array}{c} 0\\ 1\\ 3\\ 0\\ 0 \end{array}\right), \left(\begin{array}{c} 0\\ 0\\ 0\\ 1\\ 0 \end{array}\right), \left(\begin{array}{c} 0\\ 0\\ 1\\ 0\\ 0 \end{array}\right), \left(\begin{array}{c} 0\\ 0\\ 0\\ 0\\ 1 \end{array}\right)\right\} \qed$ 
    \end{enumerate}

  \item Reduce the matrix formed by the vectors in $S$ to find linear independence:
    \begin{align*}
      \left(\begin{array}{ccccc} 1 & 2 & -1 & 0 & 3\\ 0 & -1 & 3 & 1 & -1\\ 1 & 0 & 5 & 2 & 1\\ 3 & 1 & 12 & 5 & 4 \end{array}\right)
      \xrightarrow{RREF}
      \left(\begin{array}{ccccc} 1 & 0 & 5 & 2 & 1\\ 0 & 1 & -3 & -1 & 1\\ 0 & 0 & 0 & 0 & 0\\ 0 & 0 & 0 & 0 & 0 \end{array}\right)
    \end{align*}
    Therefore, $S'=\left\{\left(\begin{array}{c} 1\\ 0\\ 1\\ 3 \end{array}\right), \left(\begin{array}{c} 2\\ -1\\ 0\\ 1 \end{array}\right)\right\} \qed$
\end{enumerate}
%%%%%%%%%%%%%%%%%%%%%%%%%%%%%%%%%%%%%%%%%%%%%%%%%%%%%%
%                       End                          %
%%%%%%%%%%%%%%%%%%%%%%%%%%%%%%%%%%%%%%%%%%%%%%%%%%%%%%

\end{document}

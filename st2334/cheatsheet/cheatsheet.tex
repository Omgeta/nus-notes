\documentclass[12pt, a4paper]{article}

\usepackage[utf8]{inputenc}
\usepackage[mathscr]{euscript}
\let\euscr\mathscr \let\mathscr\relax
\usepackage[scr]{rsfso}
\usepackage{amssymb,amsmath,amsthm,amsfonts}
\usepackage[shortlabels]{enumitem}
\usepackage{multicol,multirow}
\usepackage{lipsum}
\usepackage{balance}
\usepackage{calc}
\usepackage[colorlinks=true,citecolor=blue,linkcolor=blue]{hyperref}
\usepackage{import}
\usepackage{xifthen}
\usepackage{pdfpages}
\usepackage{transparent}
\usepackage{listings}

\newcommand{\incfig}[2][1.0]{
    \def\svgwidth{#1\columnwidth}
    \import{./figures/}{#2.pdf_tex}
}

\newlist{enumproof}{enumerate}{4}
\setlist[enumproof,1]{label=\arabic*., parsep=1em}
\setlist[enumproof,2]{label=\arabic{enumproofi}.\arabic*., parsep=1em}
\setlist[enumproof,3]{label=\arabic{enumproofi}.\arabic{enumproofii}.\arabic*., parsep=1em}
\setlist[enumproof,4]{label=\arabic{enumproofi}.\arabic{enumproofii}.\arabic{enumproofiii}.\arabic*., parsep=1em}

\renewcommand{\qedsymbol}{\ensuremath{\blacksquare}}

\lstdefinestyle{mystyle}{
  language=C, % Set the language to C
  commentstyle=\color{codegray}, % Color for comments
  keywordstyle=\color{orange}, % Color for basic keywords
  stringstyle=\color{mauve}, % Color for strings
  basicstyle={\ttfamily\footnotesize}, % Basic font style
  breakatwhitespace=false,         
  breaklines=true,                 
  captionpos=b,                    
  keepspaces=true,                 
  numbers=none,                    
  tabsize=2,
  morekeywords=[2]{\#include, \#define, \#ifdef, \#ifndef, \#endif, \#pragma, \#else, \#elif}, % Preprocessor directives
  keywordstyle=[2]\color{codegreen}, % Style for preprocessor directives
  morekeywords=[3]{int, char, float, double, void, struct, union, enum, const, volatile, static, extern, register, inline, restrict, _Bool, _Complex, _Imaginary, size_t, ssize_t, FILE}, % C standard types and common identifiers
  keywordstyle=[3]\color{identblue}, % Style for types and common identifiers
  morekeywords=[4]{printf, scanf, fopen, fclose, malloc, free, calloc, realloc, perror, strtok, strncpy, strcpy, strcmp, strlen}, % Standard library functions
  keywordstyle=[4]\color{cyan}, % Style for library functions
}

\usepackage{ifthen}
\usepackage[landscape]{geometry}
\usepackage[shortlabels]{enumitem}

\ifthenelse{\lengthtest { \paperwidth = 11in}}
    { \geometry{top=.5in,left=.5in,right=.5in,bottom=.5in} }
	{\ifthenelse{ \lengthtest{ \paperwidth = 297mm}}
		{\geometry{top=1cm,left=1cm,right=1cm,bottom=1cm} }
		{\geometry{top=1cm,left=1cm,right=1cm,bottom=1cm} }
	}

\pagestyle{empty}
\makeatletter
\renewcommand\thesection{\arabic{section}.}
\renewcommand{\section}{\@startsection{section}{1}{0mm}%
                                {-1ex plus -.5ex minus -.2ex}%
                                {0.05ex}%x
                                {\normalfont\normalsize\bfseries}}
\renewcommand{\subsection}{\@startsection{subsection}{2}{0mm}%
                                {-1ex plus -.5ex minus -.2ex}%
                                {0.05ex}%
                                {\normalfont\small\bfseries}}
\renewcommand{\subsubsection}{\@startsection{subsubsection}{3}{0mm}%
                                {-1ex plus -.5ex minus -.2ex}%
                                {0.05ex}%
                                {\normalfont\footnotesize\bfseries}}
\newcommand{\colbreak}{\vfill\null\columnbreak}
\makeatother
\setcounter{secnumdepth}{1}
\setlength{\parindent}{0pt}
\setlength{\parskip}{0.7em}

\setlist[itemize]{itemsep=0.6ex, topsep=-2pt, partopsep=0pt, parsep=0pt}
\setlist[enumerate]{itemsep=0.6ex, topsep=-2pt, partopsep=0pt, parsep=0pt}

% Things Lie
\newcommand{\kb}{\mathfrak b}
\newcommand{\kg}{\mathfrak g}
\newcommand{\kh}{\mathfrak h}
\newcommand{\kn}{\mathfrak n}
\newcommand{\ku}{\mathfrak u}
\newcommand{\kz}{\mathfrak z}
\DeclareMathOperator{\Ext}{Ext} % Ext functor
\DeclareMathOperator{\Tor}{Tor} % Tor functor
\newcommand{\gl}{\opname{\mathfrak{gl}}} % frak gl group
\renewcommand{\sl}{\opname{\mathfrak{sl}}} % frak sl group chktex 6

% More script letters etc.
\newcommand{\SA}{\mathcal A}
\newcommand{\SB}{\mathcal B}
\newcommand{\SC}{\mathcal C}
\newcommand{\SF}{\mathcal F}
\newcommand{\SG}{\mathcal G}
\newcommand{\SH}{\mathcal H}
\newcommand{\OO}{\mathcal O}

\newcommand{\SCA}{\mathscr A}
\newcommand{\SCB}{\mathscr B}
\newcommand{\SCC}{\mathscr C}
\newcommand{\SCD}{\mathscr D}
\newcommand{\SCE}{\mathscr E}
\newcommand{\SCF}{\mathscr F}
\newcommand{\SCG}{\mathscr G}
\newcommand{\SCH}{\mathscr H}

% Mathfrak primes
\newcommand{\km}{\mathfrak m}
\newcommand{\kp}{\mathfrak p}
\newcommand{\kq}{\mathfrak q}

% number sets
\newcommand{\RR}[1][]{\ensuremath{\ifstrempty{#1}{\mathbb{R}}{\mathbb{R}^{#1}}}}
\newcommand{\NN}[1][]{\ensuremath{\ifstrempty{#1}{\mathbb{N}}{\mathbb{N}^{#1}}}}
\newcommand{\ZZ}[1][]{\ensuremath{\ifstrempty{#1}{\mathbb{Z}}{\mathbb{Z}^{#1}}}}
\newcommand{\QQ}[1][]{\ensuremath{\ifstrempty{#1}{\mathbb{Q}}{\mathbb{Q}^{#1}}}}
\newcommand{\CC}[1][]{\ensuremath{\ifstrempty{#1}{\mathbb{C}}{\mathbb{C}^{#1}}}}
\newcommand{\PP}[1][]{\ensuremath{\ifstrempty{#1}{\mathbb{P}}{\mathbb{P}^{#1}}}}
\newcommand{\HH}[1][]{\ensuremath{\ifstrempty{#1}{\mathbb{H}}{\mathbb{H}^{#1}}}}
\newcommand{\FF}[1][]{\ensuremath{\ifstrempty{#1}{\mathbb{F}}{\mathbb{F}^{#1}}}}
% expected value
\newcommand{\EE}{\ensuremath{\mathbb{E}}}
\newcommand{\charin}{\text{ char }}
\DeclareMathOperator{\sign}{sign}
\DeclareMathOperator{\Aut}{Aut}
\DeclareMathOperator{\Inn}{Inn}
\DeclareMathOperator{\Syl}{Syl}
\DeclareMathOperator{\Gal}{Gal}
\DeclareMathOperator{\GL}{GL} % General linear group
\DeclareMathOperator{\SL}{SL} % Special linear group

%---------------------------------------
% BlackBoard Math Fonts :-
%---------------------------------------

%Captital Letters
\newcommand{\bbA}{\mathbb{A}}	\newcommand{\bbB}{\mathbb{B}}
\newcommand{\bbC}{\mathbb{C}}	\newcommand{\bbD}{\mathbb{D}}
\newcommand{\bbE}{\mathbb{E}}	\newcommand{\bbF}{\mathbb{F}}
\newcommand{\bbG}{\mathbb{G}}	\newcommand{\bbH}{\mathbb{H}}
\newcommand{\bbI}{\mathbb{I}}	\newcommand{\bbJ}{\mathbb{J}}
\newcommand{\bbK}{\mathbb{K}}	\newcommand{\bbL}{\mathbb{L}}
\newcommand{\bbM}{\mathbb{M}}	\newcommand{\bbN}{\mathbb{N}}
\newcommand{\bbO}{\mathbb{O}}	\newcommand{\bbP}{\mathbb{P}}
\newcommand{\bbQ}{\mathbb{Q}}	\newcommand{\bbR}{\mathbb{R}}
\newcommand{\bbS}{\mathbb{S}}	\newcommand{\bbT}{\mathbb{T}}
\newcommand{\bbU}{\mathbb{U}}	\newcommand{\bbV}{\mathbb{V}}
\newcommand{\bbW}{\mathbb{W}}	\newcommand{\bbX}{\mathbb{X}}
\newcommand{\bbY}{\mathbb{Y}}	\newcommand{\bbZ}{\mathbb{Z}}

%---------------------------------------
% MathCal Fonts :-
%---------------------------------------

%Captital Letters
\newcommand{\mcA}{\mathcal{A}}	\newcommand{\mcB}{\mathcal{B}}
\newcommand{\mcC}{\mathcal{C}}	\newcommand{\mcD}{\mathcal{D}}
\newcommand{\mcE}{\mathcal{E}}	\newcommand{\mcF}{\mathcal{F}}
\newcommand{\mcG}{\mathcal{G}}	\newcommand{\mcH}{\mathcal{H}}
\newcommand{\mcI}{\mathcal{I}}	\newcommand{\mcJ}{\mathcal{J}}
\newcommand{\mcK}{\mathcal{K}}	\newcommand{\mcL}{\mathcal{L}}
\newcommand{\mcM}{\mathcal{M}}	\newcommand{\mcN}{\mathcal{N}}
\newcommand{\mcO}{\mathcal{O}}	\newcommand{\mcP}{\mathcal{P}}
\newcommand{\mcQ}{\mathcal{Q}}	\newcommand{\mcR}{\mathcal{R}}
\newcommand{\mcS}{\mathcal{S}}	\newcommand{\mcT}{\mathcal{T}}
\newcommand{\mcU}{\mathcal{U}}	\newcommand{\mcV}{\mathcal{V}}
\newcommand{\mcW}{\mathcal{W}}	\newcommand{\mcX}{\mathcal{X}}
\newcommand{\mcY}{\mathcal{Y}}	\newcommand{\mcZ}{\mathcal{Z}}

%---------------------------------------
% Bold Math Fonts :-
%---------------------------------------

%Captital Letters
\newcommand{\bmA}{\boldsymbol{A}}	\newcommand{\bmB}{\boldsymbol{B}}
\newcommand{\bmC}{\boldsymbol{C}}	\newcommand{\bmD}{\boldsymbol{D}}
\newcommand{\bmE}{\boldsymbol{E}}	\newcommand{\bmF}{\boldsymbol{F}}
\newcommand{\bmG}{\boldsymbol{G}}	\newcommand{\bmH}{\boldsymbol{H}}
\newcommand{\bmI}{\boldsymbol{I}}	\newcommand{\bmJ}{\boldsymbol{J}}
\newcommand{\bmK}{\boldsymbol{K}}	\newcommand{\bmL}{\boldsymbol{L}}
\newcommand{\bmM}{\boldsymbol{M}}	\newcommand{\bmN}{\boldsymbol{N}}
\newcommand{\bmO}{\boldsymbol{O}}	\newcommand{\bmP}{\boldsymbol{P}}
\newcommand{\bmQ}{\boldsymbol{Q}}	\newcommand{\bmR}{\boldsymbol{R}}
\newcommand{\bmS}{\boldsymbol{S}}	\newcommand{\bmT}{\boldsymbol{T}}
\newcommand{\bmU}{\boldsymbol{U}}	\newcommand{\bmV}{\boldsymbol{V}}
\newcommand{\bmW}{\boldsymbol{W}}	\newcommand{\bmX}{\boldsymbol{X}}
\newcommand{\bmY}{\boldsymbol{Y}}	\newcommand{\bmZ}{\boldsymbol{Z}}
%Small Letters
\newcommand{\bma}{\boldsymbol{a}}	\newcommand{\bmb}{\boldsymbol{b}}
\newcommand{\bmc}{\boldsymbol{c}}	\newcommand{\bmd}{\boldsymbol{d}}
\newcommand{\bme}{\boldsymbol{e}}	\newcommand{\bmf}{\boldsymbol{f}}
\newcommand{\bmg}{\boldsymbol{g}}	\newcommand{\bmh}{\boldsymbol{h}}
\newcommand{\bmi}{\boldsymbol{i}}	\newcommand{\bmj}{\boldsymbol{j}}
\newcommand{\bmk}{\boldsymbol{k}}	\newcommand{\bml}{\boldsymbol{l}}
\newcommand{\bmm}{\boldsymbol{m}}	\newcommand{\bmn}{\boldsymbol{n}}
\newcommand{\bmo}{\boldsymbol{o}}	\newcommand{\bmp}{\boldsymbol{p}}
\newcommand{\bmq}{\boldsymbol{q}}	\newcommand{\bmr}{\boldsymbol{r}}
\newcommand{\bms}{\boldsymbol{s}}	\newcommand{\bmt}{\boldsymbol{t}}
\newcommand{\bmu}{\boldsymbol{u}}	\newcommand{\bmv}{\boldsymbol{v}}
\newcommand{\bmw}{\boldsymbol{w}}	\newcommand{\bmx}{\boldsymbol{x}}
\newcommand{\bmy}{\boldsymbol{y}}	\newcommand{\bmz}{\boldsymbol{z}}

%---------------------------------------
% Scr Math Fonts :-
%---------------------------------------

\newcommand{\sA}{{\mathscr{A}}}   \newcommand{\sB}{{\mathscr{B}}}
\newcommand{\sC}{{\mathscr{C}}}   \newcommand{\sD}{{\mathscr{D}}}
\newcommand{\sE}{{\mathscr{E}}}   \newcommand{\sF}{{\mathscr{F}}}
\newcommand{\sG}{{\mathscr{G}}}   \newcommand{\sH}{{\mathscr{H}}}
\newcommand{\sI}{{\mathscr{I}}}   \newcommand{\sJ}{{\mathscr{J}}}
\newcommand{\sK}{{\mathscr{K}}}   \newcommand{\sL}{{\mathscr{L}}}
\newcommand{\sM}{{\mathscr{M}}}   \newcommand{\sN}{{\mathscr{N}}}
\newcommand{\sO}{{\mathscr{O}}}   \newcommand{\sP}{{\mathscr{P}}}
\newcommand{\sQ}{{\mathscr{Q}}}   \newcommand{\sR}{{\mathscr{R}}}
\newcommand{\sS}{{\mathscr{S}}}   \newcommand{\sT}{{\mathscr{T}}}
\newcommand{\sU}{{\mathscr{U}}}   \newcommand{\sV}{{\mathscr{V}}}
\newcommand{\sW}{{\mathscr{W}}}   \newcommand{\sX}{{\mathscr{X}}}
\newcommand{\sY}{{\mathscr{Y}}}   \newcommand{\sZ}{{\mathscr{Z}}}


%---------------------------------------
% Math Fraktur Font
%---------------------------------------

%Captital Letters
\newcommand{\mfA}{\mathfrak{A}}	\newcommand{\mfB}{\mathfrak{B}}
\newcommand{\mfC}{\mathfrak{C}}	\newcommand{\mfD}{\mathfrak{D}}
\newcommand{\mfE}{\mathfrak{E}}	\newcommand{\mfF}{\mathfrak{F}}
\newcommand{\mfG}{\mathfrak{G}}	\newcommand{\mfH}{\mathfrak{H}}
\newcommand{\mfI}{\mathfrak{I}}	\newcommand{\mfJ}{\mathfrak{J}}
\newcommand{\mfK}{\mathfrak{K}}	\newcommand{\mfL}{\mathfrak{L}}
\newcommand{\mfM}{\mathfrak{M}}	\newcommand{\mfN}{\mathfrak{N}}
\newcommand{\mfO}{\mathfrak{O}}	\newcommand{\mfP}{\mathfrak{P}}
\newcommand{\mfQ}{\mathfrak{Q}}	\newcommand{\mfR}{\mathfrak{R}}
\newcommand{\mfS}{\mathfrak{S}}	\newcommand{\mfT}{\mathfrak{T}}
\newcommand{\mfU}{\mathfrak{U}}	\newcommand{\mfV}{\mathfrak{V}}
\newcommand{\mfW}{\mathfrak{W}}	\newcommand{\mfX}{\mathfrak{X}}
\newcommand{\mfY}{\mathfrak{Y}}	\newcommand{\mfZ}{\mathfrak{Z}}
%Small Letters
\newcommand{\mfa}{\mathfrak{a}}	\newcommand{\mfb}{\mathfrak{b}}
\newcommand{\mfc}{\mathfrak{c}}	\newcommand{\mfd}{\mathfrak{d}}
\newcommand{\mfe}{\mathfrak{e}}	\newcommand{\mff}{\mathfrak{f}}
\newcommand{\mfg}{\mathfrak{g}}	\newcommand{\mfh}{\mathfrak{h}}
\newcommand{\mfi}{\mathfrak{i}}	\newcommand{\mfj}{\mathfrak{j}}
\newcommand{\mfk}{\mathfrak{k}}	\newcommand{\mfl}{\mathfrak{l}}
\newcommand{\mfm}{\mathfrak{m}}	\newcommand{\mfn}{\mathfrak{n}}
\newcommand{\mfo}{\mathfrak{o}}	\newcommand{\mfp}{\mathfrak{p}}
\newcommand{\mfq}{\mathfrak{q}}	\newcommand{\mfr}{\mathfrak{r}}
\newcommand{\mfs}{\mathfrak{s}}	\newcommand{\mft}{\mathfrak{t}}
\newcommand{\mfu}{\mathfrak{u}}	\newcommand{\mfv}{\mathfrak{v}}
\newcommand{\mfw}{\mathfrak{w}}	\newcommand{\mfx}{\mathfrak{x}}
\newcommand{\mfy}{\mathfrak{y}}	\newcommand{\mfz}{\mathfrak{z}}


\newcommand{\mytitle}{ST2334 Probability and Statistics}
\newcommand{\myauthor}{github/omgeta}
\newcommand{\mydate}{AY 25/26 Sem 1}

\begin{document}
\raggedright
\footnotesize
\begin{multicols*}{3}
\setlength{\premulticols}{1pt}
\setlength{\postmulticols}{1pt}
\setlength{\multicolsep}{1pt}
\setlength{\columnsep}{2pt}

{\normalsize{\textbf{\mytitle}}} \\
{\footnotesize{\mydate\hspace{2pt}\textemdash\hspace{2pt}\myauthor}}
%%%%%%%%%%%%%%%%%%%%%%%%%%%%%%%%%%%%%%%%%%%%%%%%%%%%%%
%                      Begin                         %
%%%%%%%%%%%%%%%%%%%%%%%%%%%%%%%%%%%%%%%%%%%%%%%%%%%%%%
\section{Counting}
Counting Formula: $\displaystyle \binom nr = \frac{n!}{r!(n-r)!}, P(n, r) = \frac{n!}{(n-r)!}$

DeMorgan's Laws:
\begin{enumerate}[\roman*.]
  \item $(A\cup B)^c = A^c \cap B^c$
  \item $(A\cap B)^c = A^c \cup B^c$
\end{enumerate}

Inclusion/Exclusion Principle for finite sets $A,B,C$:
\begin{enumerate}[\roman*.]
  \item $|A \cup B|= |A| + |B| - |A\cap B|$
  \item $|A \cup B \cup C|= |A| + |B| + |C| + |A\cap B\cap C|$\\\hspace{6.7em}$- |A\cap B| - |A\cap C| - |B\cap C|$
\end{enumerate}

Number of ways to:
\begin{enumerate}[\roman*.]
  \item Permute $n$ distinct $= n!$
  \item Permute $n$ with $n_1, n_2$ identical $= \frac{n!}{n_1!n_2!}$
  \item Choose $r$ of $n$ distinct $= \binom nr$
  \item Choose $r$ groups of $n$ identical $= \binom{n+r-1}n$\\
    $(x_1+\cdots+x_r=n)$
  \item Permute $r$ of $n$ distinct $= P(n,r)$
  \item Permute $r$ of $n$ distinct (repeat) $= n^r$
\end{enumerate}
Useful results:
\begin{enumerate}[\roman*.]
  \item Choose 2 groups of $r,m$ from $n$ distinct $= \binom nr \binom {n-r}m$
  \item Choose $k$ groups of $r$ from $n$ distinct $= \frac{\binom nr \binom{n-r}r \cdots \binom rr}{k!}$
  \item Permute $n$ distinct with $r$ together $= (n-r+1)!r!$ 
  \item Permute $n$, $m$ distinct but separated $= m! \binom{m+1}n n!$ 
  \item Permute $n$ distinct in a circle $= (n-1)!$
  \item Permute $n$ distinct with $r$ together in a circle\\$= (n-r)!r!$
  \item Permute $n, m$ distinct but separated in a circle \\$= m! \binom mn n!$
  \item Permute $n$ distinct in a circle with 2 opposite $= (n-2)!$
  \item Permute $n$ distinct in a circle with $r$ identical\\$=\frac{(n-1)!}{r!}$
\end{enumerate}

\colbreak
\section{Probability}
Probability of event $E$ in sample space $S$, $P(E)$, is:
\begin{enumerate}[\roman*.]
  \item $P(E) = \displaystyle \frac{|E|}{|S|}$, where $0 \leq P(E) \leq 1$
  \item $P(E^c) = 1 - P(E)$\hfill(Complement)
  \item $A \cap B = \phi \rightarrow P(A\cup B) = P(A) + P(B)$\hfill(Disjoint)
  \item $P(A\cup B) = P(A) + P(B)- P(A\cap B)$\hfill(Union)
\end{enumerate}

Conditional probability of $B$ given $A$, $P(B\mid A)$, is:
\begin{enumerate}[\roman*.]
  \item $P(B\mid A) = \displaystyle\frac{P(A\cap B)}{P(A)} = \frac{P(A\mid B)\cdot P(B)}{P(A)}$\hfill
\end{enumerate}

Total Probability for partition $B_1, \cdots B_n$ of $S$:
\begin{enumerate}[\roman*.]
  \item $P(A) = \displaystyle \sum^n_{i=1} P(A\mid B_i)\cdot P(B_i)$\\$\quad\quad\quad=\displaystyle \sum^n_{i=1}P(A\cap B_i)$
  \item $P(A\mid C) = \displaystyle \sum^n_{i=1} P(A\mid B_i \cap C)\cdot P(B_i\mid C)$\\$\quad\quad\quad\quad=\displaystyle \sum^n_{i=1} P(A\cap B_i\mid C)$
\end{enumerate}

Baye's Theorem for partition $B_1, \cdots, B_n$ of $S$:
\begin{enumerate}[\roman*.]
  \item $P(B_i\mid A) = \displaystyle \frac{P(A\mid B_i)\cdot P(B_i)}{P(A)}$
  \item $P(B_i\mid A \cap C) = \displaystyle \frac{P(A\mid B_i \cap C)\cdot P(B_i \cap C)}{P(A\cap C)}$
  \item $\displaystyle \frac{P(B\mid A)}{P(B^c\mid A)} = \frac{P(A\mid B)}{P(A\mid B^c)} \cdot \frac{P(B)}{P(B^c)}$\hfill(Odds)
\end{enumerate}

Mutually exclusive events $A,B$ have special results:
\begin{enumerate}[\roman*.]
  \item $P(A\cap B) = 0$\hfill(Intersection)
  \item $P(A\cup B) = P(A) + P(B)$\hfill(Union)
\end{enumerate}

Independent events $A,B$ have special results:
\begin{enumerate}[\roman*.]
  \item $P(A\cap B) = P(A)\cdot P(B)$\hfill(Intersection)
  \item $P(A\mid B) = P(A)$\hfill(Conditional)
\end{enumerate}

\colbreak
\section{Random Variables}
Probability mass function (PMF) of a discrete random variable $X$ is:
\begin{enumerate}[\roman*.]
  \item $f(x) = P(X=x)$
  \item $f(x) \geq 0, \forall x\in R_x$ and $f(x) = 0,\quad\forall x \not\in R_x$
  \item $\sum_{R_x} f(x) = 1$
\end{enumerate}

Probability density function (PDF) of a continuous random variable $X$ is:
\begin{enumerate}[\roman*.]
  \item $P(a\leq X\leq b) = \int^b_a f(x) dx$
  \item $f(x) \geq 0, \forall x\in R_x$ and $f(x) = 0,\quad\forall x \not\in R_x$
  \item $\int_x f(x) dx \geq 0$ but not necessarily $\leq 1$
  \item $\int_{R_x} f(x) dx = 1$
\end{enumerate}

Cumulative density function (CDF) of any random variable $X$ is:
\begin{enumerate}[\roman*.]
  \item $F(x) = P(X \geq x)$
  \item $F(x) = \int^x_{-\infty}f(t)dt$ and $f(x) = F'(x)$
  \item Increasing and right continuous such that \\$F_x(x) \rightarrow 0$ as $x\rightarrow -\infty$ and $F_x(x) \rightarrow 1$ as $x\rightarrow \infty$
\end{enumerate}

\subsection{Expectation and Variance}
Expectation of random variable $X$, $E(X)$ or $\mu_X$, is:
\begin{enumerate}[\roman*.]
  \item $E(X) = \sum_{R_x} x\cdot f(x)$ or $\int^{\infty}_{-\infty}x\cdot f(x) dx$
  \item $E[g(X)] = \sum_{R_x} g(x)\cdot f(x)$ or $\int^{\infty}_{-\infty}g(x)\cdot f(x) dx$
  \item $E(aX+b) = aE(X) + b$
  \item $E(X+Y) = E(X) + E(Y)$
\end{enumerate}

Variance of random variable $X$, $V(X)$ or $\sigma^2_X$, is:
\begin{enumerate}[\roman*.]
  \item $V(X) = \sum_{R_x}(x-\mu_X)^2f(x)$\\\quad\quad\quad or $\int^{\infty}_{-\infty}(x-\mu_{X})^2f(x)dx$\\$\quad\quad\quad=E(X^2)-[E(X)]^2$
  \item $\forall X, V(X) \geq 0$
  \item $V(aX+b) = a^2V(X)$
  \item Standard deviation, $SD(X) = \sqrt{V(X)}$
  \item $V(aX+bY) = a^2V(X)+b^2V(Y)+2ab\cdot Cov(X,Y)$
  \item $V(X+Y) = V(X) + V(Y) + 2 Cov(X,Y)$ and $V(\sum^n_{i=1}X_i) = \sum^n_{i=1}V(X_i) + 2\sum_{i<j}Cov(X_i, X_j)$
\vspace{-1em}
\end{enumerate}
\colbreak
\section{Joint Distributions}
Joint PMF of discrete random variable $X$ is:
\begin{enumerate}[\roman*.]
  \item $f(x,y) = P(X=x, Y=y)$
  \item $f(x, y) \geq 0,\quad\forall (x, y)\in R_{X,Y}$ and $f(x,y) = 0,\quad\forall (x,y) \not\in R_{X,Y}$
  \item $\sum_{R_X}\sum_{R_Y} f(x,y) = 1$
\end{enumerate}

Joint PDF of continuous random variable $X$ is:
\begin{enumerate}[\roman*.]
  \item $P((X, Y) \in D) = \iint_D f(x, y) dxdy$
  \item $f(x, y) \geq 0,\quad\forall (x, y)\in R_{X,Y}$ and $f(x,y) = 0,\quad\forall (x,y) \not\in R_{X,Y}$
  \item $\iint_{R_{X, Y}} f(x,y) = 1$
\end{enumerate}

Marginal distribution is:
\begin{enumerate}[\roman*.]
  \item $f_X(x) = \sum_y f_{X, Y}(x,y)$ or $\int^{\infty}_{-\infty}f_{X,Y}$
\end{enumerate}

Conditional probability function of $Y$ given $X$ is:
\begin{enumerate}[\roman*.]
  \item $f_{Y\mid X}(y\mid x) = P(Y=y, X=x) = \displaystyle\frac{f_{X, Y}(x, y)}{f_X(x)}$
\end{enumerate}

Independent random variables $X, Y$ have special results:
\begin{enumerate}[\roman*.]
  \item $f_{X,Y}(x,y) = f_X(x)\cdot f_Y(y),\quad\forall (x,y) \in R_{X, Y}$\\$\iff f_{X,Y}(x,y)=C\times g_1(x) \cdot g_2(y)$
  \item $R_{X, Y}$ is a product space, $R_{X, Y} = R_X \times R_Y $
\end{enumerate}

\subsection{Expectation and Variance}
Expectation of random variables $X, Y$, $E(X, Y)$, is:
\begin{enumerate}[\roman*.]
  \item $E[g(X, Y)] = \sum_{R_X}\sum_{R_Y} g(x, y)\cdot f_{X, Y}(x, y)$ or\\$\quad\quad\quad\quad\quad\quad\int^{\infty}_{-\infty}\int^{\infty}_{-\infty}g(x, y)\cdot f_{X, Y}(x, y) dxdy$
\end{enumerate}

Covariance of random variables $X, Y$, $Cov(X, Y)$, is:
\begin{enumerate}[\roman*.]
  \item $Cov(X, Y) = \sum_{R_X}\sum_{R_Y}(x-\mu_X)(y-\mu_Y)f_{X, Y}(x, y)$\\\quad\quad\quad or $\int^{\infty}_{-\infty}\int^{\infty}_{-\infty} (x-\mu_{X})(y-\mu_{Y})f_{X, Y}(x, y)dxdy$\\\quad\quad\quad\quad\quad$=E[(X-\mu_X)(Y-\mu_Y)]$\\$\quad\quad\quad\quad\quad=E(XY)-\mu_X\mu_Y$
  \item $Cov(X, Y) = Cov(Y, X)$ and $Cov(X, X) = V(X)$
  \item $X, Y$ are independent $\implies$ $Cov(X, Y) = 0$
  \item $Cov(aX + b, cY+d) = ac\cdot Cov(X,Y)$
  \item $Cov(W+X, Y+Z) = Cov(W, Y) + Cov(W, Z) + Cov(X, Y) + Cov(X, Z)$
\end{enumerate}
\vspace{-1em}
\colbreak
\section{Discrete Probability Distributions}
\textbf{Uniform Distribution}: $X \sim \Unif({x_1,\cdots,x_k})$
\begin{enumerate}[\roman*.]
  \item $f_X(x) = \frac{1}{k},\quad x \in {x_1,\dots,x_k}$
  \item $\mu_X = \frac{x_1+\cdots+x_k}{k},\quad\sigma^2_X = \frac{1}{k}\sum^k_{i=1}(x_i-\mu_X)^2$
\end{enumerate}

\textbf{Bernoulli Trial}: $X \sim \Bern(p)$ is the outcome of a single trial with success probability $p$
\begin{enumerate}[\roman*.]
  \item $f_X(x) = p^x(1-p)^{1-x}$, $\quad x=0$ (fail), $1$ (success)
  \item $\mu_X = p,\quad \sigma^2_X = p(1-p)$
\end{enumerate}

\textbf{Binomial Distribution}: $X \sim \Bin(n, p) = \sum X_i$ is the successes in $n$ independent Bernoulli trials $X_i \sim \Bern(p)$
\begin{enumerate}[\roman*.]
  \item $f_X(x) = \binom nx p^x(1-p)^{n-x},\quad x=0,1,\dots n$
  \item $\mu_X = np,\quad \sigma^2_X = np(1-p)$
\end{enumerate}

\textbf{Negative Binomial Distribution}: $X \sim \NB(k, p)$ is the number of independent Bernoulli trials until $k^{th}$ success
\begin{enumerate}[\roman*.]
  \item $f_X(x) = \binom nx p^x(1-p)^{n-x},\quad x=k, k+1,\dots$
  \item $\mu_X = np,\quad\sigma^2_X = np(1-p)$
\end{enumerate}

\textbf{Geometric Distribution}: $X \sim \Geom(p)$ is the number of independent Bernoulli trials until the first success
\begin{enumerate}[\roman*.]
  \item $f_X(x) = p(1 - p)^{x - 1}$
  \item $\mu_X = \frac{1}{p},\quad\sigma^2_X = \frac{1 - p}{p^2}$
\end{enumerate}

\textbf{Poisson Distribution}: $X \sim \Poisson(\lambda)$ is the number of events occurring in a fixed interval or region where $\lambda > 0$ is expected number of occurences in the interval
\begin{enumerate}[\roman*.]
  \item $f_X(x) = \frac{e^{-\lambda} \lambda^x}{x!}, \quad x = 0, 1, \dots$
  \item $\mu_X = \sigma^2_X = \lambda$
  \item As $n\rightarrow \infty$ and $p \rightarrow 0$, $X \sim \Bin(n, p)$ converges to $X \sim Poisson(\lambda = np)$. Good approximation if:
    \begin{itemize}\vspace{1pt }
      \item $n\geq 20$ and $p \leq 0.05$, or if 
      \item $n\geq 100$ and $np \leq 10$
    \end{itemize}
  \item Poisson process counts the number of events within a scaled interval of time, such that:
    \begin{itemize}\vspace{1pt}
      \item expected occurences in interval $T$ is $\alpha T$
      \item no simultaneous occurences
      \item number of occurences in disjoint time intervals are independent 
    \end{itemize}
\end{enumerate}

\section{Continuous Probability Distributions}
\textbf{Uniform Distribution}: $X \sim \Unif(a, b)$
\begin{enumerate}[\roman*.]
  \item $ f_X(x) = \frac{1}{b - a},\quad a \leq x \leq b $
  \item $ \mu_X = \frac{a + b}{2},\quad \sigma^2_X = \frac{(b - a)^2}{12} $
  \item CDF, $F_X(x) = \frac{x-a}{x-b},\quad a\leq x\leq b$
\end{enumerate}

\textbf{Exponential Distribution}: $ X \sim \Exp(\lambda)$ is the waiting time for first success in continuous time
\begin{enumerate}[\roman*.]
  \item $ f_X(x) = \lambda e^{-\lambda x},\quad x \geq 0 $
  \item $ \mu_X = \frac{1}{\lambda},\quad \sigma^2_X = \frac{1}{\lambda^2} $
  \item CDF, $F_X(x) = 1-e^{-\lambda x},\quad x\geq 0$
  \item $P(X>s+t\mid X>s) = P(X>t)$\hfill(Memoryless)
\end{enumerate}

\textbf{Normal Distribution}: $ X \sim \Normal(\mu, \sigma^2)$ is symmetric about $\mu$ and flattens out as $\sigma$ increases
\begin{enumerate}[\roman*.]
  \item $ f_X(x) = \frac{1}{\sqrt{2\pi \sigma^2}} e^{-\frac{(x - \mu)^2}{2\sigma^2}},\quad x \in \mathbb{R} $
  \item $ \mu_X = \mu,\quad \sigma^2_X = \sigma^2 $
  \item Standard normal: $ Z \sim N(0, 1) = \frac{X-\mu}{\sigma}$
    \begin{itemize}\vspace{2pt }
      \item Upper $\alpha$ quartile $z_{\alpha}$ is the value s.t. $P(Z > z_{\alpha}) = \alpha$ 
    \end{itemize}
  \item As $n\rightarrow \infty$ and $p \rightarrow 0$, $X \sim \Bin(n, p)$ converges to $X \sim \Normal(np, np(1-p))$. Good approximation if:
    \begin{itemize}\vspace{2pt }
      \item $np>5$ and $n(1-p)>5$
    \end{itemize}
  \item Apply the continuity corrections for approximating:\\\vspace{1em}
    \begin{tabular}{|c|c|}
      \hline
      \textbf{Discrete Probability} & \textbf{Normal Approx.} \\
      \hline
      \( P(X = k) \) & \( P\left(k - \frac{1}{2} < X < k + \frac{1}{2}\right) \) \\
      \hline
      \( P(a \leq X \leq b) \) & \( P\left(a - \frac{1}{2} < X < b + \frac{1}{2}\right) \) \\
      \hline
      \( P(a < X < b) \) & \( P\left(a + \frac{1}{2} < X < b - \frac{1}{2}\right) \) \\
      \hline
      \( P(X \leq c) \) & \( P\left(0 \leq X < c + \frac{1}{2}\right) \) \\
      \hline
      \( P(X > c) \) & \( P\left(c + \frac{1}{2} < X < n\right) \) \\
      \hline
      \end{tabular}
\end{enumerate}

\colbreak
\section{Sampling}

Population is the entire group of interest.\\ Population parameter is a population's numerical fact.\\Sample of a population is used to make inferences. 

Probability sampling:
\begin{enumerate}[\roman*.]
  \item Simple Random Sampling: sample is chosen s.t. every subset of $n$ observations of the population has the same probability of being selected.
\end{enumerate}

Statistic is a function of sample data:
\begin{enumerate}[\roman*.]
  \item Sampling Mean, $\overline{X} = \frac{1}{n}\sum^n_{i=1}X_i$
  \item Sampling Variance, $S^2 = \frac{1}{n-1} \sum^n_{i=1}(X_i-\overline{X})^2$
\end{enumerate}

Standard Deviation, $\lambda_{\overline{X}}$ describes how much $\overline{x}$ tends to vary from sample to sample of size $n$

Law of Large Numbers: 
\begin{enumerate}[\roman*.]
  \item As sample size $n\rightarrow\infty$, $\frac{\sigma^2}{n}\rightarrow 0$ and $\overline{X} \rightarrow\mu_X$, $P(|\overline{X}-\mu_X| > \epsilon) \rightarrow 0$
\end{enumerate}

Central Limit Theorem:
\begin{enumerate}[\roman*.]
  \item Sampling distribution of sample mean $\overline{X}$ is approximately normal if $n$ is sufficiently large 
  \item $Z = \frac{\overline{X}-\mu}{\sigma / \sqrt{n}}$ follows approximately $N(0, 1)$
\end{enumerate}

\colbreak
\section{Sampling Distribution}
\textbf{Diff. of Sample Means}: $\overline{X_1} - \overline{X_2} = \frac{\overline{X_1}-\overline{X_2} - \mu_{\overline{X_1}-\overline{X_2}}}{\sigma_{\overline{X_1}-\overline{X_2}}}$ approx. $N(0, 1)$ for independent random variables $\overline{X_1}\sim \Normal(\mu_1, \sigma_1^2 /n_1), \overline{X_2}\sim\Normal(\mu_2, \sigma_2^2 /n_2)$
\begin{enumerate}[\roman*.]
  \item $\mu_{\overline{X_1}-\overline{X_2}} = \mu_1 - \mu_2,\quad \sigma^2_{\overline{X_1}-\overline{X_2}} = \frac{\sigma_1^2}{n_1}+\frac{\sigma_2^2}{n_2}$
\end{enumerate}

\textbf{Chi-Squared Distribution}: $Y \sim \chi^2(n) = \sum^n Z_i^2$ is the sum of $n$ independent and identically distributed standard normal random variables, with long right tail and $n$ degrees of freedom
\begin{enumerate}[\roman*.]
  \item $\mu_Y = n,\quad \sigma^2_Y = 2n$
  \item $\chi^2(n;\alpha) = k \implies P(Y>k) = \alpha$
  \item $Y_1 \sim \chi^2(n_1), Y_2 \sim \chi^2(n_2)\implies Y_1+Y_2\sim \chi^2(n_1+n_2)$
  \item As $n$ increases, $\chi^2(n)$ is approximately $N(n, 2n)$
  \item If $S^2$ is sample variance of size $n$ from normal population of variance $\sigma^2$, $\frac{(n-1)S^2}{\sigma^2}\sim\chi^2(n-1)$
\end{enumerate}

\textbf{t-Distribution}: $T \sim t_n = \frac{Z}{\sqrt{U /n}}$ for independent random variables $Z\sim N(0, 1)$ and $U\sim\chi^2(n)$ resembles standard normal with $n$ degrees of freedom 
\begin{enumerate}[\roman*.]
  \item $\mu_T = 0,\quad \sigma^2_T = \frac{n}{n - 2}$ for $n > 2$
  \item $t(n;\alpha) = k \implies P(T>k) = \alpha$
  \item When $n \geq 30$, can be replaced by $N(0, 1)$
  \item If random sample selected from normal population, $T = \frac{\overline{X}-\mu}{S /\sqrt{n}} \sim t_{n-1}$ 
\end{enumerate}

\textbf{F-Distribution}: $F \sim F{n, m} = \frac{U}{n} / \frac{V}{m}$ for independent random variables $U \sim \chi^2(n), V\sim\chi^2(m)$
\begin{enumerate}[\roman*.]
  \item $\mu_F = \frac{m}{m-2}$ for $m > 2$
  \item $\sigma^2_T = \frac{2m^2(n+m-2)}{n(m-2)^2(m-4)}$ for $n>4$
  \item $F(n, m;\alpha) = k \implies P(F> k) = \alpha$
  \item $\frac{1}{F}\sim F(m, n)$
  \item $F(n, m; \alpha) = \frac{1}{F(m, n; 1-\alpha)}$
\end{enumerate}

\colbreak
\section{Estimation}

Estimators are rules used to compute an estimate from the sample.
\begin{enumerate}[\roman*.]
  \item Point Estimator: A single number is calculated
    \begin{itemize}\vspace{2pt}
    \item Unbiased Estimator: An estimator $\hat{\theta}$ of a parameter $\theta$ is unbiased if $E(\hat{\theta}) = \theta$.
    \end{itemize}
  \item Interval Estimation: An interval is calculated for some confidence level
\end{enumerate}

Maximum error $E$ for estimating $\mu$ using $\bar{X}$ when $\sigma$ is known for confidence level $(1-\alpha)$ is: $E = z_{\alpha/2} \cdot \frac{\sigma}{\sqrt{n}}$

Sample size to achieve maximum error $E_0$ with confidence level $(1 - \alpha)$ is: $n \geq \left( \frac{z_{\alpha/2} \cdot \sigma}{E_0} \right)^2$

\section{Hypothesis Testing}

Hypothesis test can be used given a null hypothesis $H_0$, a alternative hypothesis $H_1$, and a significance value $\alpha$.

\begin{center}
\begin{tabular}{|c|c|c|}
\hline
 & \textbf{Do not reject \( H_0 \)} & \textbf{Reject \( H_0 \)} \\
\hline
\textbf{\( H_0 \) true} & Correct & \textbf{Type I Error} \\
\hline
\textbf{\( H_0 \) false} & \textbf{Type II Error} & Correct \\
\hline
\end{tabular}
\end{center}

Level of significance $\alpha$ is the probability of Type I error:
\[
\alpha = P(\text{Type I Error}) = P(\text{Reject } H_0 \mid H_0 \text{ is true})
\]
Power is the probability of correctly rejecting a false $H_0$. Let \( \beta \) denote the probability of a Type II error:
\[
\beta = P(\text{Type II Error}) = P(\text{Do not reject } H_0 \mid H_0 \text{ is false})
\]
\[
\text{Power} = 1 - \beta = P(\text{Reject } H_0 \mid H_0 \text{ is false})
\]

$p$-value can be defined as:
\begin{enumerate}[\roman*.]
  \item Probability of obtaining a sample statistic as extreme or more extreme than the observed statistic, assuming $H_0$ is true.
  \item Smallest level of significance at which $H_0$ is rejected, assuming $H_0$ is true
\end{enumerate}
where we reject $H_0$ in favour of $H_1$ when $p$-value $< \alpha$\\
or not reject $H_0$ (doesn't imply $H_0$ true) when $p$-value $\geq \alpha$

\colbreak
\subsection*{Test Statistics for Population Mean}

\begin{center}
\renewcommand{\arraystretch}{1.5}
\begin{tabular}{|c|c|c|c|c|c|}
\hline
\textbf{Case} & \textbf{Population} & \( \sigma \) & \( n \) & \textbf{CI} & \textbf{Statistic} \\
\hline
I & Normal & known & any & 
\( \displaystyle\bar{x} \pm z_{\alpha /2} \cdot \frac{\sigma}{\sqrt{n}} \) & 
\(\displaystyle Z = \frac{\bar{X} - \mu}{\sigma / \sqrt{n}} \sim N(0, 1) \) \\
\hline
II & any & known & \( \geq 30 \) & 
\(\displaystyle \bar{x} \pm z_{\alpha /2} \cdot \frac{\sigma}{\sqrt{n}} \) & 
\(\displaystyle Z = \frac{\bar{X} - \mu}{\sigma / \sqrt{n}} \sim N(0, 1) \) \\
\hline
III & Normal & unknown & \( < 30 \) & 
\(\displaystyle \bar{x} \pm t_{n-1; \alpha /2} \cdot \frac{s}{\sqrt{n}} \) & 
\(\displaystyle T = \frac{\bar{X} - \mu}{s / \sqrt{n}} \sim t_{n-1} \) \\
\hline
IV & any & unknown & \( \geq 30 \) & 
\(\displaystyle \bar{x} \pm z_{\alpha /2} \cdot \frac{s}{\sqrt{n}} \) & 
\(\displaystyle Z = \frac{\bar{X} - \mu}{s / \sqrt{n}} \sim N(0, 1) \) \\
\hline
\end{tabular}
\end{center}

\vspace{1em}

\subsection*{Test Statistics for Independent Samples}

\begin{center}
\renewcommand{\arraystretch}{1.5}
\begin{tabular}{|c|c|c|c|c|c|}
\hline
\textbf{Population} & \textbf{Variance} & \( \sigma_1, \sigma_2 \) & \( n \) & \textbf{CI} & \textbf{Statistic} \\
\hline
any & known & unequal & \( \geq 30 \) & 
\(\displaystyle (\bar{x} - \bar{y}) \pm z_{\alpha /2} \cdot \sqrt{\frac{\sigma_1^2}{n_1} + \frac{\sigma_2^2}{n_2}} \) & 
\(\displaystyle Z = \frac{\bar{X} - \bar{Y}}{\sqrt{\frac{\sigma_1^2}{n_1} + \frac{\sigma_2^2}{n_2}}} \sim N(0, 1) \) \\
\hline
Normal & known & unequal & any & 
\(\displaystyle (\bar{x} - \bar{y}) \pm z_{\alpha /2} \cdot \sqrt{\frac{\sigma_1^2}{n_1} + \frac{\sigma_2^2}{n_2}} \) & 
\(\displaystyle Z = \frac{\bar{X} - \bar{Y}}{\sqrt{\frac{\sigma_1^2}{n_1} + \frac{\sigma_2^2}{n_2}}} \sim N(0, 1) \) \\
\hline
any & unknown & unequal & \( \geq 30 \) & 
\(\displaystyle (\bar{x} - \bar{y}) \pm z_{\alpha /2} \cdot \sqrt{\frac{s_1^2}{n_1} + \frac{s_2^2}{n_2}} \) & 
\(\displaystyle Z = \frac{\bar{X} - \bar{Y}}{\sqrt{\frac{s_1^2}{n_1} + \frac{s_2^2}{n_2}}} \sim N(0, 1) \) \\
\hline
Normal & unknown & equal & \( < 30 \) & 
\(\displaystyle (\bar{x} - \bar{y}) \pm t_{n_1 + n_2 - 2; \alpha/2} \cdot s_p \cdot \sqrt{\frac{1}{n_1} + \frac{1}{n_2}} \) & 
\(\displaystyle T = \frac{\bar{X} - \bar{Y}}{s_p \cdot \sqrt{\frac{1}{n_1} + \frac{1}{n_2}}} \sim t_{n_1 + n_2 - 2} \) \\
\hline
any & unknown & equal & \( \geq 30 \) & 
\(\displaystyle (\bar{x} - \bar{y}) \pm z_{\alpha /2} \cdot s_p \cdot \sqrt{\frac{1}{n_1} + \frac{1}{n_2}} \) & 
\(\displaystyle Z =\frac{\bar{X} - \bar{Y}}{s_p \cdot \sqrt{\frac{1}{n_1} + \frac{1}{n_2}}} \sim N(0, 1) \) \\
\hline
\end{tabular}
\end{center}

*Variance assumed equal if $\frac{1}{2} < \frac{s_1}{s_2} < 2$

\subsection*{Pooled Estimator}

$\displaystyle S^2_p = \frac{(n_1-1)S^2_1+(n_2-1)S^2_2}{n_1+n_2-2}$

\colbreak
%%%%%%%%%%%%%%%%%%%%%%%%%%%%%%%%%%%%%%%%%%%%%%%%%%%%%%
%                       End                          %
%%%%%%%%%%%%%%%%%%%%%%%%%%%%%%%%%%%%%%%%%%%%%%%%%%%%%%
\end{multicols*}
\end{document}

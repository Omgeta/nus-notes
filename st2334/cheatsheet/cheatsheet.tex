\documentclass[12pt, a4paper]{article}

\usepackage[utf8]{inputenc}
\usepackage[mathscr]{euscript}
\let\euscr\mathscr \let\mathscr\relax
\usepackage[scr]{rsfso}
\usepackage{amssymb,amsmath,amsthm,amsfonts}
\usepackage[shortlabels]{enumitem}
\usepackage{multicol,multirow}
\usepackage{lipsum}
\usepackage{balance}
\usepackage{calc}
\usepackage[colorlinks=true,citecolor=blue,linkcolor=blue]{hyperref}
\usepackage{import}
\usepackage{xifthen}
\usepackage{pdfpages}
\usepackage{transparent}
\usepackage{tabularx}

\newcommand{\incfig}[2][1.0]{
    \def\svgwidth{#1\columnwidth}
    \import{./figures/}{#2.pdf_tex}
}
\newcommand{\incimg}[2][1.0]{
  \includegraphics[width=#1\columnwidth]{./figures/#2}
}


\usepackage{ifthen}
\usepackage[landscape]{geometry}
\usepackage[shortlabels]{enumitem}

\ifthenelse{\lengthtest { \paperwidth = 11in}}
    { \geometry{top=.5in,left=.5in,right=.5in,bottom=.5in} }
	{\ifthenelse{ \lengthtest{ \paperwidth = 297mm}}
		{\geometry{top=1cm,left=1cm,right=1cm,bottom=1cm} }
		{\geometry{top=1cm,left=1cm,right=1cm,bottom=1cm} }
	}

\pagestyle{empty}
\makeatletter
\renewcommand\thesection{\arabic{section}.}
\renewcommand{\section}{\@startsection{section}{1}{0mm}%
                                {-1ex plus -.5ex minus -.2ex}%
                                {0.05ex}%x
                                {\normalfont\normalsize\bfseries}}
\renewcommand{\subsection}{\@startsection{subsection}{2}{0mm}%
                                {-1ex plus -.5ex minus -.2ex}%
                                {0.05ex}%
                                {\normalfont\small\bfseries}}
\renewcommand{\subsubsection}{\@startsection{subsubsection}{3}{0mm}%
                                {-1ex plus -.5ex minus -.2ex}%
                                {0.05ex}%
                                {\normalfont\footnotesize\bfseries}}
\newcommand{\colbreak}{\vfill\null\columnbreak}
\makeatother
\setcounter{secnumdepth}{1}
\setlength{\parindent}{0pt}
\setlength{\parskip}{0.7em}

\setlist[itemize]{itemsep=0.6ex, topsep=-2pt, partopsep=0pt, parsep=0pt}
\setlist[enumerate]{itemsep=0.6ex, topsep=-2pt, partopsep=0pt, parsep=0pt}

\input{letterfonts}

\newcommand{\mytitle}{ST2334 Probability and Statistics}
\newcommand{\myauthor}{github/omgeta}
\newcommand{\mydate}{AY 25/26 Sem 1}

\begin{document}
\raggedright
\footnotesize
\begin{multicols*}{3}
\setlength{\premulticols}{1pt}
\setlength{\postmulticols}{1pt}
\setlength{\multicolsep}{1pt}
\setlength{\columnsep}{2pt}

{\normalsize{\textbf{\mytitle}}} \\
{\footnotesize{\mydate\hspace{2pt}\textemdash\hspace{2pt}\myauthor}}
%%%%%%%%%%%%%%%%%%%%%%%%%%%%%%%%%%%%%%%%%%%%%%%%%%%%%%
%                      Begin                         %
%%%%%%%%%%%%%%%%%%%%%%%%%%%%%%%%%%%%%%%%%%%%%%%%%%%%%%
\section{Counting}
Counting Formula: $\displaystyle \binom nr = \frac{n!}{r!(n-r)!}, P(n, r) = \frac{n!}{(n-r)!}$

DeMorgan's Laws:
\begin{enumerate}[\roman*.]
  \item $(A\cup B)' = A' \cap B'$
  \item $(A\cap B)' = A' \cup B'$
\end{enumerate}

Inclusion/Exclusion Principle for finite sets $A,B,C$:
\begin{enumerate}[\roman*.]
  \item $|A \cup B|= |A| + |B| - |A\cap B|$
  \item $|A \cup B \cup C|= |A| + |B| + |C| + |A\cap B\cap C|$\\\hspace{6.7em}$- |A\cap B| - |A\cap C| - |B\cap C|$
\end{enumerate}

Number of ways to:
\begin{enumerate}[\roman*.]
  \item Permute $n$ distinct $= n!$
  \item Permute $n$ with $n_1, n_2$ identical $= \frac{n!}{n_1!n_2!}$
  \item Choose $r$ of $n$ distinct $= \binom nr$
  \item Choose $r$ groups of $n$ identical $= \binom{n+r-1}n = \binom{n+r-1}{r-1}$\\
    $(x_1+\cdots+x_r=n)$
  \item Permute $r$ of $n$ distinct $= P(n,r)$
  \item Permute $r$ of $n$ distinct (repeat) $= n^r$
\end{enumerate}
Useful results:
\begin{enumerate}[\roman*.]
  \item Choose 2 groups of $r,m$ from $n$ distinct $= \binom nr \binom {n-r}m$
  \item Choose $k$ groups of $r$ from $n$ distinct $= \frac{\binom nr \binom{n-r}r \cdots \binom rr}{k!}$
  \item Permute $n$ distinct with $r$ together $= (n-r+1)!r!$ 
  \item Permute $n$, $m$ distinct but separated $= m! \binom{m+1}n n!$ 
  \item Permute $n$ distinct in a circle $= (n-1)!$
  \item Permute $n$ distinct with $r$ together in a circle\\$= (n-r)!r!$
  \item Permute $n, m$ distinct but separated in a circle \\$= m! \binom mn n!$
  \item Permute $n$ distinct in a circle with 2 opposite $= (n-2)!$
  \item Permute $n$ distinct in a circle with $r$ identical\\$=\frac{(n-1)!}{r!}$
\end{enumerate}

\colbreak
\section{Probability}
Probability of event $E$ in sample space $S$, $P(E)$, is:
\begin{enumerate}[\roman*.]
  \item $P(E) = \displaystyle \frac{|E|}{|S|}$, where $0 \leq P(E) \leq 1$
  \item $P(E') = 1 - P(E)$\hfill(Complement)
  \item $P(A\cup B) = P(A) + P(B)- P(A\cap B)$\hfill(Union)
\end{enumerate}

Conditional probability of $B$ given $A$, $P(B\mid A)$, is:
\begin{enumerate}[\roman*.]
  \item $P(B\mid A) = \displaystyle\frac{P(A\cap B)}{P(A)} = \frac{P(A\mid B) P(B)}{P(A)}$\hfill
\end{enumerate}

Mutually exclusive events $A,B$ have special results:
\begin{enumerate}[\roman*.]
  \item $P(A\cap B) = 0$\hfill(Intersection)
  \item $P(A\cup B) = P(A) + P(B)$\hfill(Union)
\end{enumerate}

Independent events $A \perp B$ have special results:
\begin{enumerate}[\roman*.]
  \item $P(A\cap B) = P(A) P(B)$\hfill(Intersection)
  \item $P(A\mid B) = P(A)$\hfill(Conditional)
\end{enumerate}

Total Probability for event $B$, partition $B_1, s B_n$ of $S$:
\begin{enumerate}[\roman*.]
  \item $P(A) = P(A \mid B) P(B) + P(A \mid B') P(B')$
  \item $P(A) = \displaystyle \sum^n_{i=1}P(A\cap B_i)$\\$\quad\quad\quad=\displaystyle \sum^n_{i=1} P(A\mid B_i) P(B_i)$
  \item $P(A\mid C) = \displaystyle \sum^n_{i=1} P(A\cap B_i\mid C)$\\$\quad\quad\quad\quad=\displaystyle  \sum^n_{i=1} P(A\mid B_i \cap C) P(B_i\mid C)$
\end{enumerate}

Baye's Theorem for event $B$, partition $B_1, s, B_n$ of $S$:
\begin{enumerate}[\roman*.]
  \item $P(B\mid A) = \displaystyle \frac{P(A\mid B) P(B)}{P(A\mid B) P(B) + P(A\mid B') P(B')}$
  \item $P(B_k\mid A) = \displaystyle \frac{P(A\mid B_k) P(B_k)}{\sum^n_{i=1} P(A\mid B_i)}$
  \item $P(B_k\mid A \cap C) = \displaystyle \frac{P(A\mid B_k \cap C) P(B_k \cap C)}{P(A\cap C)}$
  \item $\displaystyle \frac{P(B\mid A)}{P(B'\mid A)} = \frac{P(A\mid B)}{P(A\mid B')}  \frac{P(B)}{P(B')}$\hfill(Odds)
\end{enumerate}


\colbreak
\section{Random Variables}
Probability mass function (PMF) of a discrete random variable $X$ is:
\begin{enumerate}[\roman*.]
  \item $f(x) = P(X=x)$
  \item $0 \leq f(x_i) \leq 1, \forall x_i\in R_x$ and $f(x_i) = 0,\forall x_i \not\in R_x$
  \item $\sum_{x_i\in R_x} f(x_i) = 1$
\end{enumerate}

Probability density function (PDF) of a continuous random variable $X$ is:
\begin{enumerate}[\roman*.]
  \item $\int^b_a f(x)\;dx = P(a\leq X\leq b)$
  \item $f(x) \geq 0, \forall x\in R_x$ and $f(x) = 0, \forall x \not\in R_x$
  \item $\int_a^b f(x)\;dx \geq 0$ but not necessarily $\leq 1$
  \item $\int_{R_x} f(x)\;dx = 1$
\end{enumerate}

Cumulative density function (CDF) of any random variable $X$ is:
\begin{enumerate}[\roman*.]
  \item $F(x) = P(X \leq x)$
  \item $F(x) = \int^x_{-\infty}f(t)dt$ and $f(x) = F'(x)$
  \item Non-decreasing and right continuous 
  \item $0 \leq F(x) \leq 1$
\end{enumerate}

\subsection{Expectation and Variance}
Expectation of random variable $X$, $E(X)$ or $\mu_X$, is:
\begin{enumerate}[\roman*.]
  \item $E(X) = \sum_{x_i\in R_x} x_if(x_i)$ or $\int^{\infty}_{-\infty}xf(x)\;dx$
  \item $E[g(X)] = \sum_{x_i\in R_x} g(x_i)f(x_i)$ or $\int^{\infty}_{-\infty}g(x)f(x)\;dx$
  \item $E(aX+b) = aE(X) + b$
  \item $E(X+Y) = E(X) + E(Y)$
\end{enumerate}

Variance of random variable $X$, $V(X)$ or $\sigma^2_X$, is:
\begin{enumerate}[\roman*.]
  \item $V(X) = \sum_{x_i\in R_x}(x_i-\mu_X)^2f(x_i)$\\\quad\quad\quad or $\int^{\infty}_{-\infty}(x-\mu_{X})^2f(x)\;dx$\\$\quad\quad\quad= E(X - \mu_X)^2=E(X^2)-[E(X)]^2$
  \item $\forall X, V(X) \geq 0$
  \item $V(aX+b) = a^2V(X)$
  \item Standard deviation, $SD(X) = \sqrt{V(X)}$
  \item $V(X+Y) = V(X) + V(Y) + 2 Cov(X,Y)$ and $V(\sum^n_{i=1}X_i) = \sum^n_{i=1}V(X_i) + 2\sum_{i<j}Cov(X_i, X_j)$
  \item $V(aX+bY) = a^2V(X)+b^2V(Y)+2ab Cov(X,Y)$
\vspace{-1em}
\end{enumerate}
\colbreak
\section{Joint Distributions}
Joint PMF of discrete random variables $X, Y$ is:
\begin{enumerate}[\roman*.]
  \item $f_{X,Y}(x,y) = P(X=x, Y=y)$
  \item $0 \leq f_{X,Y}(x, y) \leq 1,\quad\forall (x, y)\in R_{X,Y}$ and $f_{X,Y}(x,y) = 0,\quad\quad\quad\forall (x,y) \not\in R_{X,Y}$
  \item $\sum\sum_{(x,y)\in R_{X,Y}} f_{X,Y}(x,y) = 1$
\end{enumerate}

Joint PDF of continuous random variables $X, Y $ is:
\begin{enumerate}[\roman*.]
  \item $P((X, Y) \in D) = \iint_{D} f(x, y)\;dx\;dy$
  \item $f(x, y) \geq 0,\quad\forall (x, y)\in R_{X,Y}$ and $f(x,y) = 0,\quad\forall (x,y) \not\in R_{X,Y}$
  \item $\iint_{R_{X, Y}} f(x,y)\;dx\;dy = 1$
\end{enumerate}

Marginal distribution and conditional distribution are:
\begin{enumerate}[\roman*.]
  \item $f_X(x) = \sum_y f_{X, Y}(x,y)$ or $\int^{\infty}_{-\infty}f_{X,Y}(x,y)\;dy$
  \item $f_{Y\mid X}(y\mid x) = P(Y=y \mid X=x) = \displaystyle\frac{f_{X, Y}(x, y)}{f_X(x)}$
\end{enumerate}

Independent random variables $X, Y$ have special results:
\begin{enumerate}[\roman*.]
  \item $f_{X,Y}(x,y) = f_X(x)f_Y(y),\quad\forall (x,y) > 0 \in R_{X, Y}$
  \item $R_{X, Y}$ is a product space, $R_{X, Y} = R_X \times R_Y $
\end{enumerate}

\subsection{Expectation and Variance}
Expectation of random variables $X, Y$, $E(X, Y)$, is:
\begin{enumerate}[\roman*.]
  \item $E[g(X, Y)] = \sum_{R_X}\sum_{R_Y} g(x, y) f_{X, Y}(x, y)$ or\\$\quad\quad\quad\quad\quad\quad\int^{\infty}_{-\infty}\int^{\infty}_{-\infty}g(x, y) f_{X, Y}(x, y)\;dx\,dy$
  \item If independent, $E(XY) = E(X)E(Y)$
\end{enumerate}

Covariance of random variables $X, Y$, $Cov(X, Y)$, is:
\begin{enumerate}[\roman*.]
  \item $Cov(X, Y) = \sum_{R_X}\sum_{R_Y}(x-\mu_X)(y-\mu_Y)f_{X, Y}(x, y)$\\\quad\quad\quad or $\int^{\infty}_{-\infty}\int^{\infty}_{-\infty} (x-\mu_{X})(y-\mu_{Y})f_{X, Y}(x, y)\;dxdy$\\\quad\quad\quad\quad\quad$=E[(X-\mu_X)(Y-\mu_Y)]$\\$\quad\quad\quad\quad\quad=E(XY)-\mu_X\mu_Y$
  \item $X, Y$ are independent $\implies$ $Cov(X, Y) = 0$
  \item $Cov(X, Y) = Cov(Y, X)$ and $Cov(X, X) = V(X)$
  \item $Cov(aX + b, cY+d) = ac\cdot Cov(X,Y)$
  \item $Cov(W+X, Y+Z) = Cov(W, Y) + Cov(W, Z) + Cov(X, Y) + Cov(X, Z)$
\end{enumerate}
\vspace{-1em}
\colbreak
\section{Discrete Probability Distributions}
\textbf{Uniform Distribution}: $X \sim \Unif({x_1,\cdots,x_k})$
\begin{enumerate}[\roman*.]
  \item $f_X(x) = \frac{1}{k},\quad x \in {x_1,\dots,x_k}$
  \item $\mu_X = \frac{x_1+\cdots+x_k}{k},\quad\sigma^2_X = \frac{1}{k}\sum^k_{i=1}x_i^2-\mu_X^2$
\end{enumerate}

\textbf{Bernoulli Trial}: $X \sim \Bern(p)$ is the outcome of a single trial with success probability $p$, fail probability $q=1-p$
\begin{enumerate}[\roman*.]
  \item $f_X(x) = p^xq^{1-x}$, $\quad x=0$ (fail), $1$ (success)
  \item $\mu_X = p,\quad \sigma^2_X = pq$
\end{enumerate}

\textbf{Binomial Distribution}: $X \sim \Bin(n, p) = \sum X_i$ is the successes in $n$ independent Bernoulli trials $X_i \sim \Bern(p)$
\begin{enumerate}[\roman*.]
  \item $f_X(x) = \binom nx p^xq^{n-x},\quad x=0,1,\dots n$
  \item $\mu_X = np,\quad \sigma^2_X = npq$
\end{enumerate}

\textbf{Negative Binomial Distribution}: $X \sim \NB(k, p)$ is the number of independent Bernoulli trials until $k^{th}$ success
\begin{enumerate}[\roman*.]
  \item $f_X(x) = \binom {x-1}{k-1} p^kq^{x-k},\quad x=k, k+1,\dots$
  \item $\mu_X = \frac{k}{p},\quad\sigma^2_X = \frac{qk}{p^2}$
\end{enumerate}

\textbf{Geometric Distribution}: $X \sim \Geom(p)$ is the number of independent Bernoulli trials until the first success
\begin{enumerate}[\roman*.]
  \item $f_X(x) = pq^{x - 1}, F_X(x) = 1 - q^x\quad x = 1, 2, \dots$
  \item $\mu_X = \frac{1}{p},\quad\sigma^2_X = \frac{q}{p^2}$
\end{enumerate}

\textbf{Poisson Distribution}: $X \sim \Poisson(\lambda)$ is the number of events occurring in a fixed interval or region where $\lambda > 0$ is expected number of occurences in the interval
\begin{enumerate}[\roman*.]
  \item $f_X(x) = \frac{e^{-\lambda} \lambda^x}{x!}, \quad x = 0, 1, \dots$
  \item $\mu_X = \sigma^2_X = \lambda$
  \item As $n\rightarrow \infty$ and $p \rightarrow 0$, $X \sim \Bin(n, p)$ converges to $X \sim Poisson(\lambda = np)$. Good approximation if:
    \begin{itemize}\vspace{1pt }
      \item $n\geq 20$ and $p \leq 0.05$, or if 
      \item $n\geq 100$ and $np \leq 10$
    \end{itemize}
  \item Poisson process counts the number of events within interval of time scaled by rate $\alpha$, such that:
    \begin{itemize}\vspace{1pt}
      \item expected occurences in interval $T$ is $\alpha T$
      \item no simultaneous occurences
      \item number of occurences in disjoint time intervals are independent 
    \end{itemize}
\end{enumerate}

\section{Continuous Probability Distributions}
\textbf{Uniform Distribution}: $X \sim \Unif(a, b)$
\begin{enumerate}[\roman*.]
  \item $ f_X(x) = \frac{1}{b - a},\quad a \leq x \leq b $
  \item $ \mu_X = \frac{a + b}{2},\quad \sigma^2_X = \frac{(b - a)^2}{12} $
  \item CDF, $F_X(x) = \frac{x-a}{b-a},\quad a\leq x\leq b$
\end{enumerate}

\textbf{Exponential Distribution}: $ X \sim \Exp(\lambda)$ is the waiting time for first success in continuous time
\begin{enumerate}[\roman*.]
  \item $ f_X(x) = \lambda e^{-\lambda x},\quad x \geq 0 $
  \item $ \mu_X = \frac{1}{\lambda},\quad \sigma^2_X = \frac{1}{\lambda^2} $
  \item CDF, $F_X(x) = 1-e^{-\lambda x},\quad x\geq 0$
  \item $P(X>s+t\mid X>s) = P(X>t)$\hfill(Memoryless)
\end{enumerate}

\textbf{Normal Distribution}: $ X \sim \Normal(\mu, \sigma^2)$ is symmetric about $\mu$ and flattens out as $\sigma$ increases
\begin{enumerate}[\roman*.]
  \item $ f_X(x) = \frac{1}{\sqrt{2\pi \sigma^2}} e^{-(x - \mu)^2 /(2\sigma^2)},\quad x \in \mathbb{R} $
  \item $ \mu_X = \mu,\quad \sigma^2_X = \sigma^2 $
  \item Standard normal: $ Z \sim N(0, 1) = \frac{X-\mu}{\sigma}$
    \begin{itemize}[leftmargin=*]\vspace{3pt }
      \item $\phi(z) = f_Z(z) = \frac{1}{\sqrt{2\pi}}$exp$(-\frac{z^2}{2})$
      \item $\Phi(z) = \int^z_{-\infty}\phi(t)dt = \frac{1}{\sqrt{2\pi}}\int^z_{-\infty}\text{exp}(-\frac{t^2}{2})dt$
      \item $P(x_1 < X < x_2) = \Phi(\frac{x_2-\mu}{\sigma})- \Phi(\frac{x_{1}-\mu}{\sigma})$ 
      \item $P(Z \geq 0) = P(Z \leq 0) = \Phi(0) = 0.5$
      \item $\Phi(z) = P(Z\leq z) = P(Z \geq -z)= 1 - \Phi(-z)$
      \item $\sigma Z + \mu \sim N(\mu, \sigma^2)$
    \end{itemize}
  \item Upper $\alpha$ quantile $x_{\alpha}$ satisfies:
    \begin{itemize}[leftmargin=*]\vspace{3pt}
      \item $P(X \geq x_{\alpha}) = \alpha$
      \item $P(Z \geq z_{\alpha}) = P(Z \leq -z_{\alpha})= \alpha$
    \end{itemize}
  \item As $n\rightarrow \infty$ but $p$ remains constant, $X \sim \Bin(n, p)$ approximates $Z = \frac{X-np}{\sqrt{np(1-p)}} \sim N(0, 1)$.\\Good approximation if: $np>5$ and $n(1-p)>5$
  \item Apply the continuity corrections for approximating:\\\vspace{1em}
    \begin{tabular}{|c|c|}
      \hline
      \textbf{Discrete Probability} & \textbf{Normal Approx.} \\
      \hline
      \( P(X = k) \) & \( P\left(k - \frac{1}{2} < X < k + \frac{1}{2}\right) \) \\
      \hline
      \( P(a \leq X \leq b) \) & \( P\left(a - \frac{1}{2} < X < b + \frac{1}{2}\right) \) \\
      \hline
      \( P(a < X < b) \) & \( P\left(a + \frac{1}{2} < X < b - \frac{1}{2}\right) \) \\
      \hline
      \( P(X \leq c) \) & \( P\left( 0 \leq X \leq c \right) \) \\
      \hline
      \( P(X > c) \) & \( P\left(c < X \leq n\right) \) \\
      \hline
      \end{tabular}
\end{enumerate}
\vspace{-1em}
\colbreak
\section{Sampling}

Population is the totality of all possible outcomes in an experiment. They can be finite or infinite.\\ Population parameter is a population's numerical fact.\\Sample is any subset of a population. 

Probability sampling:
\begin{enumerate}[\roman*.]
  \item Simple Random Sampling (finite pop.):\\every subset of $n$ observations of the population has the same probability of being selected.
  \item Simple Random Sampling (infinite pop.):\\random sample $X_1, \dots, X_n$ are $n$ independent random variables with same distribution $f_X(x)$ as $X$ s.t. $f_{X_1,\dots,X_n}(x_1, \dots, x_n) = f_{X_1}(x_1)\dots f_{X_n}(x_n)$. 
\end{enumerate}

Statistic is random variable functions of sample data:
\begin{enumerate}[\roman*.]
  \item Sampling Mean, $\displaystyle\overline{X} = \frac{1}{n}\sum^n_{i=1}X_i$
  \begin{itemize}[leftmargin=*]
    \item $\mu_{\overline{X}} = \mu_{X}, \quad\sigma^2_{\overline{X}} = \frac{\sigma^2_X}{n}$ 
    \item Standard error, $\sigma_{\overline{X}}$ describes how much $\overline{x}$ tends to vary from sample to sample of size $n$
  \end{itemize}
  \item Sampling Variance, $\displaystyle S^2 = \frac{1}{n-1} \sum^n_{i=1}(X_i-\overline{X})^2$
  \begin{itemize}[leftmargin=*]
    \item $\mu_{S^2} = \sigma^2, \quad\sigma^2_{S^2} = \frac{1}{n}(\mu_4 - \frac{n-3}{n-1}\sigma^4)$ 
  \end{itemize}
\end{enumerate}

Law of Large Numbers: 
\begin{enumerate}[\roman*.]
  \item As sample size $n\rightarrow\infty$, $\frac{\sigma^2}{n}\rightarrow 0$ and $\overline{X} \rightarrow\mu_X$, $P(|\overline{X}-\mu_X| > \epsilon) \rightarrow 0$
\end{enumerate}

Central Limit Theorem (for means):
\begin{enumerate}[\roman*.]
  \item Sampling distribution of sample mean $\overline{X}$ is approximately normal if $n$ is sufficiently large 
  \item $\frac{\overline{X}-\mu}{\sigma / \sqrt{n}} \rightarrow Z \sim N(0, 1)$ as $n \rightarrow \infty$
  \item Conditions for population of sample:
    \begin{itemize}[leftmargin=*]\vspace{3pt}
      \item Symmetric with no outliers, needs 15-20 samples 
      \item Moderately skewed (exponential or $\chi^2$), can take 30-50 samples 
      \item Extreme skewness may not be appropriate for CLT even with 1000 samples
    \end{itemize}
\end{enumerate}

\colbreak
\section{Sampling Distribution}
\textbf{Diff. of Sample Means}: $\overline{X_1} - \overline{X_2} = \frac{\overline{X_1}-\overline{X_2} - \mu_{\overline{X_1}-\overline{X_2}}}{\sigma_{\overline{X_1}-\overline{X_2}}}$ approx. $N(0, 1)$ for independent random variables $\overline{X_1}\sim \Normal(\mu_1, \sigma_1^2 /n_1), \overline{X_2}\sim\Normal(\mu_2, \sigma_2^2 /n_2)$
\begin{enumerate}[\roman*.]
  \item $\mu_{\overline{X_1}-\overline{X_2}} = \mu_1 - \mu_2,\quad \sigma^2_{\overline{X_1}-\overline{X_2}} = \frac{\sigma_1^2}{n_1}+\frac{\sigma_2^2}{n_2}$
\end{enumerate}

\textbf{Chi-Squared Distribution}: $Y \sim \chi^2(n) = \sum^n_{i=1} Z_i^2$ is the sum of $n$ independent and identically distributed standard normal random variables, with long right tail and $n$ degrees of freedom
\begin{enumerate}[\roman*.]
  \item $\mu_Y = n,\quad \sigma^2_Y = 2n$
  \item For large $n$, $\chi^2(n)$ is approximately $N(n, 2n)$
  \item $Y_1 \sim \chi^2(m), Y_2 \sim \chi^2(n)$ are independent $\implies Y_1+Y_2\sim \chi^2(m+n)$
  \item Define $\chi^2(n;\alpha)$: $ P(Y>\chi^2(n;\alpha)) = \alpha$
  \item If $S^2$ is sample variance of size $n$ from normal population of variance $\sigma^2$, then $\frac{(n-1)S^2}{\sigma^2} = \frac{\sum^n_{i=1}(X_i-\overline{X})^2}{\sigma^2}\sim\chi^2(n-1)$
\end{enumerate}

\textbf{t-Distribution}: $T \sim t(n) = \frac{Z}{\sqrt{U /n}}$ for independent $Z\sim N(0, 1)$ and $U\sim\chi^2(n)$, with $n$ degrees of freedom is symmetric vertical and resembles standard normal graph 
\begin{enumerate}[\roman*.]
  \item $\mu_T = 0,\quad \sigma^2_T = \frac{n}{n - 2}$ for $n > 2$
  \item For $n \geq 30$, can be replaced by $N(0, 1)$
  \item Define $t_{n;\alpha}$: $P(T>t_{n;\alpha}) = \alpha$
  \item If $X_1,\dots,X_n$ are independent and identically distributed normal random variables with mean $\mu$ and variance $\sigma^2$, $T = \frac{\overline{X}-\mu}{S /\sqrt{n}} \sim t_{n-1}$ 
\end{enumerate}

\textbf{F-Distribution}: $F \sim F(m, n) = \frac{U /m}{V /n}$ for independent $U \sim \chi^2(m), V\sim\chi^2(n)$ has $(m, n)$ degrees of freedom
\begin{enumerate}[\roman*.]
  \item $\mu_F = \frac{n}{n-2}$ for $n > 2$\\
    $\sigma^2_F = \frac{2n^2(m+n-2)}{m(n-2)^2(n-4)}$ for $n>4$
  \item $\frac{1}{F}\sim F(n, m)$
  \item Define: $F(m, n;\alpha)$: $P(F> F(m, n; \alpha)) = \alpha$
  \item $F(m, n; 1-\alpha) = \frac{1}{F(n, m; \alpha)}$
\end{enumerate}

\colbreak
\section{Estimation}

Estimators are rules, usually formulas, used to compute an estimate from the sample.
\begin{enumerate}[\roman*.]
  \item Point Estimator: A single number is calculated
    \begin{itemize}[leftmargin=*]\vspace{2pt}
    \item Unbiased Estimator: An estimator $\hat{\theta}$ of a parameter $\theta$ is unbiased if $E(\hat{\theta}) = \theta$.
    \end{itemize}
  \item Interval Estimation: An interval is calculated for some confidence level
\end{enumerate}

Maximum error $E$ for estimating $\mu$ using $\bar{X}$ when $\sigma$ is known for confidence level $(1-\alpha)$ is: $E = z_{\alpha/2} \cdot \frac{\sigma}{\sqrt{n}}$

Sample size to achieve maximum error $E_0$ with confidence level $(1 - \alpha)$ is: $n \geq \left( \frac{z_{\alpha/2} \cdot \sigma}{E_0} \right)^2$

\section{Hypothesis Testing}

Hypothesis test can be used given a null hypothesis $H_0$, a alternative hypothesis $H_1$, and level of significance $\alpha$.
\begin{center}
\begin{tabular}{|c|c|c|}
\hline
 & \textbf{Do not reject \( H_0 \)} & \textbf{Reject \( H_0 \)} \\
\hline
\textbf{\( H_0 \) true} & Correct & \textbf{Type I Error} \\
\hline
\textbf{\( H_0 \) false} & \textbf{Type II Error} & Correct \\
\hline
\end{tabular}
\end{center}
\begin{enumerate}[\roman*.]
  \item LOS $\alpha = P(\text{Type I}) = P(\text{Reject } H_0 \mid H_0 \text{ is true})$
  \item $\beta = P(\text{Type II}) = P(\text{Do not reject } H_0 \mid H_0 \text{ is false})$
  \item Power $1 - \beta$ is given by $P(\text{Reject } H_0 \mid H_0 \text{ is false})$
\end{enumerate}

Test statistic (e.g. $z, t$) is a function of sample data.

$p$-value can be defined as:
\begin{enumerate}[\roman*.]
  \item Probability of obtaining a sample statistic as extreme or more extreme than the observed statistic, assuming $H_0$ is true.
  \item Smallest level of significance at which $H_0$ is rejected, assuming $H_0$ is true
\end{enumerate}
where we reject $H_0$ in favour of $H_1$ when $p$-value $< \alpha$\\
or not reject $H_0$ (doesn't imply $H_0$ true) when $p$-value $\geq \alpha$


\end{multicols*}
\noindent
\begin{minipage}[t]{0.6\textwidth}
\subsection*{Test Statistics for Population Mean}
\renewcommand{\arraystretch}{1.5}
\begin{tabular}{|c|c|c|c|c|c|c|}
\hline
\textbf{Case} & \textbf{Population} & \( \sigma \) & \( n \) & \textbf{CI} & \textbf{Statistic} & \textbf{$n$ for desired $E_0, \alpha$}\\
\hline
I & Normal & known & any & 
\( \overline{x}\pm\displaystyle z_{\alpha /2} \cdot \frac{\sigma}{\sqrt{n}} \) & 
\(\displaystyle Z = \frac{\bar{X} - \mu}{\sigma / \sqrt{n}} \) & $\displaystyle\left(\frac{z_{\alpha / 2}\cdot \sigma}{E_0}\right)^2$\\
\hline
II & any & known & \( \geq 30 \) & 
\(\overline{x}\pm\displaystyle z_{\alpha /2} \cdot \frac{\sigma}{\sqrt{n}} \) & 
\(\displaystyle Z = \frac{\bar{X} - \mu}{\sigma / \sqrt{n}}  \) & $\displaystyle\left(\frac{z_{\alpha / 2} \cdot \sigma}{E_0}\right)^2$ \\
\hline
III & Normal & unknown & \( < 30 \) & 
\(\overline{x}\pm\displaystyle t_{n-1; \alpha /2} \cdot \frac{s}{\sqrt{n}} \) & 
\(\displaystyle T = \frac{\bar{X} - \mu}{s / \sqrt{n}}  \) & $\displaystyle\left( \frac{t_{n-1;\alpha /2} \cdot s}{E_0}\right)^2$ \\
\hline
IV & any & unknown & \( \geq 30 \) & 
\(\overline{x}\pm\displaystyle z_{\alpha /2} \cdot \frac{s}{\sqrt{n}} \) & 
\(\displaystyle Z = \frac{\bar{X} - \mu}{s / \sqrt{n}} \) & $\displaystyle\left(\frac{z_{\alpha / 2} \cdot s}{E_0}\right)^2$\\
\hline
\end{tabular}
\end{minipage}
\begin{minipage}[t]{0.38\textwidth}
For example, if $T \sim t_{n-1}$ under $H_0$:
\begin{itemize}
  \item Right-tailed test $H_1: \mu > \mu_0$: reject $H_0$ if $T > t_{n-1;\alpha}$.
  \item Left-tailed test $H_1: \mu < \mu_0$: reject $H_0$ if $T < -t_{n-1;\alpha}$.
  \item Two-tailed test $H_1: \mu \neq \mu_0$: reject $H_0$ if $|T| > t_{n-1;\alpha/2}$.
\end{itemize}

For a given observed test statistic $t_{\text{obs}}$:
\begin{itemize}
  \item Right-tailed: $p$-value $= P(T \ge t_{\text{obs}} \mid H_0)$.
  \item Left-tailed: $p$-value $= P(T \le t_{\text{obs}} \mid H_0)$.
  \item Two-tailed: $p$-value $= P(|T| \ge |t_{\text{obs}}| \mid H_0)$.
\end{itemize}

Given CI for $H_0: \mu = \mu_0$:
\begin{itemize}
  \item do not reject $H_0$ if $\mu_0$ in CI 
  \item reject $H_0$ if $\mu_0$ not in CI
\end{itemize}
\end{minipage}

\subsection*{Test Statistics for Independent Samples}

\renewcommand{\arraystretch}{1.5}
\begin{tabular}{|c|c|c|c|c|c|}
\hline
\textbf{Population} & \textbf{Variance} & \( \sigma_1, \sigma_2 \) & \( n \) & \textbf{CI} & \textbf{Statistic} \\
\hline
any & known & unequal & \( \geq 30 \) & 
\(\displaystyle (\bar{x} - \bar{y}) \pm z_{\alpha /2} \cdot \sqrt{\frac{\sigma_1^2}{n_1} + \frac{\sigma_2^2}{n_2}} \) & 
\(\displaystyle Z = \frac{\bar{X} - \bar{Y}}{\sqrt{\frac{\sigma_1^2}{n_1} + \frac{\sigma_2^2}{n_2}}} \sim N(0, 1) \) \\
\hline
Normal & known & unequal & any & 
\(\displaystyle (\bar{x} - \bar{y}) \pm z_{\alpha /2} \cdot \sqrt{\frac{\sigma_1^2}{n_1} + \frac{\sigma_2^2}{n_2}} \) & 
\(\displaystyle Z = \frac{\bar{X} - \bar{Y}}{\sqrt{\frac{\sigma_1^2}{n_1} + \frac{\sigma_2^2}{n_2}}} \sim N(0, 1) \) \\
\hline
any & unknown & unequal & \( \geq 30 \) & 
\(\displaystyle (\bar{x} - \bar{y}) \pm z_{\alpha /2} \cdot \sqrt{\frac{s_1^2}{n_1} + \frac{s_2^2}{n_2}} \) & 
\(\displaystyle Z = \frac{\bar{X} - \bar{Y}}{\sqrt{\frac{s_1^2}{n_1} + \frac{s_2^2}{n_2}}} \sim N(0, 1) \) \\
\hline
Normal & unknown & equal & \( < 30 \) & 
\(\displaystyle (\bar{x} - \bar{y}) \pm t_{n_1 + n_2 - 2; \alpha/2} \cdot s_p \cdot \sqrt{\frac{1}{n_1} + \frac{1}{n_2}} \) & 
\(\displaystyle T = \frac{\bar{X} - \bar{Y}}{s_p \cdot \sqrt{\frac{1}{n_1} + \frac{1}{n_2}}} \sim t_{n_1 + n_2 - 2} \) \\
\hline
any & unknown & equal & \( \geq 30 \) & 
\(\displaystyle (\bar{x} - \bar{y}) \pm z_{\alpha /2} \cdot s_p \cdot \sqrt{\frac{1}{n_1} + \frac{1}{n_2}} \) & 
\(\displaystyle Z =\frac{\bar{X} - \bar{Y}}{s_p \cdot \sqrt{\frac{1}{n_1} + \frac{1}{n_2}}} \sim N(0, 1) \) \\
\hline
\end{tabular}

*Variance assumed equal if $\frac{1}{2} < \frac{s_1}{s_2} < 2$\\
** For dependent samples, consider sample $D_i - X_i - Y_i$ and use results for Population Mean.


Pooled Estimator: $\displaystyle S^2_p = \frac{(n_1-1)S^2_1+(n_2-1)S^2_2}{n_1+n_2-2}$

Maclaurin series for $e^x = \displaystyle \sum^\infty_{k=0}\frac{x^k}{k!}$

Geometric series: $\displaystyle a + ar + ar^2 + \cdots + ar^{n-1} = \frac{a(1-r^n)}{1-r}$

%%%%%%%%%%%%%%%%%%%%%%%%%%%%%%%%%%%%%%%%%%%%%%%%%%%%%%
%                       End                          %
%%%%%%%%%%%%%%%%%%%%%%%%%%%%%%%%%%%%%%%%%%%%%%%%%%%%%%
\end{document}
